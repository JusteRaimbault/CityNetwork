%%%%%%%%%%%%%%%%%%%%%%%%%
%% Header for standard beamer presentation
%%
%%  PresentationHeader.tex
%%
%%%%%%%%%%%%%%%%%%%%%%%%%

\documentclass[english,10pt]{beamer}

%%%%%%%%%%%%%%%%%%%%
%% Include general header where common packages are defined
%%%%%%%%%%%%%%%%%%%%

% general packages without options
\usepackage{amsmath,amssymb,bbm}




%%%%%%%%%%%%%%%%%%%%
%% Idem general commands
%%%%%%%%%%%%%%%%%%%%

%% Commands

\newcommand{\noun}[1]{\textsc{#1}}


%% Math

\DeclareMathOperator{\Cov}{Cov}
\newcommand{\Covb}[2]{\ensuremath{\Cov\!\left[#1,#2\right]}}





\usetheme{Warsaw}

\setbeamertemplate{footline}[text line]{}
\setbeamercolor{structure}{fg=purple!50!blue, bg=purple!50!blue}

\setbeamercovered{transparent}


% shortened command for a justified frame
\newcommand{\jframe}[2]{\frame{\frametitle{#1}\justify{#2}}}



%%%%%%%%%%%%%%%%%%%%%
%% Begin doc
%%%%%%%%%%%%%%%%%%%%%

\begin{document}



\title{Thesis Progress Meeting}


\author{J.~Raimbault$^{1,2}$}

\institute{$^{1}$G{\'e}ographie-cit{\'e}s (UMR 8504 CNRS)\\
$^{2}$LVMT (UMR-T 9403 IFSTTAR)}


\date{December 08th 2015}


%%%%%%%%%%%%%%%%%%%%%%%%%%%%%%%%
\begin{frame}
\titlepage
\end{frame}

%\begin{frame}
%\tableofcontents
%\end{frame}
%%%%%%%%%%%%%%%%%%%%%%%%%%%%%%%%


%\section{Projects Organization}

%\jframe{Projects Organization}{
%   \includegraphics[width=\textwidth,height=0.8\textheight]{figures/orgaProjects}
%}



\section{Achieved Work}


\jframe{Achieved Work (by projects)}{
\begin{itemize}
\item Biblio [0.2w]
\item Meetings/Seminars/Organisation [0.5w]
\item Cybergeo Project [1w]
\item Synthetic Data/Density Model [0.5w]
\item Technical : OpenMole and Genetic Algorithms [0.4w]
\item Theory construction [1w]
\item Monitorat [2,4w]
\end{itemize}

}



\section{Theoretical Framework}


\jframe{Theoretical Framework}{
Construction of a theoretical framework at a high abstraction level, to try to capture driving questions and intuitions in various projects done until now.

\medskip

$\rightarrow$ \textbf{Not yet} a thematic geographical theory or theoretical positioning, but it is supposed to be the direct next step : application of the meta-theory yields theories.

\medskip

$\rightarrow$ Aims to capture structure of \emph{models of socio-technical systems} from an epistemological point of view.

\medskip

$\rightarrow$ Not a theory of systems (like cybernetics, synergetics, systems of systems, etc.) nor a meta-modeling framework for complex systems nor a general epistemology of systems.


}


\jframe{Objectives}{
The theory must explicitly include :
\begin{itemize}
\item a precise definition and emphasis on the notion of coupling between subsystems, in particular allowing to qualify or quantify a certain degree of coupling : dependence, interdependence, etc. between components.
\item a precise definition of scale
\item a precise definition of what is a system.
\item the notion of emergence in order to capture multi-scale aspects of systems.
\item a central place of ontology in the definition of systems, i.e. of the sense in the real world given to its objects
\item heterogeneous aspects of the same system, that could be heterogeneous components but also complementary intersecting views.
\end{itemize}

}


\jframe{Theory Summary}{
$\rightarrow$ Starting from a perspectivist approach to science~\cite{giere2010scientific}, a system is the superposition of perspectives on it, that are dataflow machines~\cite{golden2012modeling} with ontologies~\cite{livet2010}.

\medskip

$\rightarrow$ Compatible notions of \emph{emergence}, nominal and weak emergence~\cite{bedau2002downward}, yield pre-order relations on ontologies.

\medskip

$\rightarrow$ An ontological graph is constructed by induction.

\medskip

$\rightarrow$ The graph can be mapped to a minimal tree (directed forest), that captures a hierarchical structure of the system regarding emergence. ``Strongly coupled'' subsystems are encoded within nodes of the tree.
}


\jframe{Theory Construction}{
%\vspace{2cm}
\textit{See Working Paper for detailed construction of the theory}
}

\jframe{Theory Applications}{
Direct applications that will be looked at in next steps :

\bigskip

\begin{itemize}
\item Positioning regarding classical definitions of geographical systems (e.g. \cite{dollfus1975some}).\medskip
\item Simple examples ; clarification and guidelines for application. In particular for our thesis, proposes definition \medskip
\item Definition of co-evolving subsystems.\medskip
\item Link with Multi-Modeling framework~\cite{cottineau2015incremental}
\end{itemize}

}


\begin{frame}[allowframebreaks]

\frametitle{Positioning within our thesis}

\begin{justify}

\begin{enumerate}
\item The perspectivist approach implies a broad understanding of existing perspectives on a system, and of possibility of coupling between them ; thus an emphasis on applied epistemology, i.e. \textbf{Algorithmic Systematic Review}, \textbf{Disciplines Mapping} and \textbf{Datamining for Content Analysis}.\medskip
\item At a finer level of particularization, the knowledge of perspectives means \textbf{Knowledge of stylized facts}, i.e. empirical analysis of cases studies.\medskip
\item The emphasis on coupled subsystems at different scales implies a deep understanding of coupling mechanisms, thus the need of methodological and technical developments : \textbf{Methods for Statistical Control}, \textbf{Methods for Model Exploration}, \textbf{Theoretical Study of Coupling}, \textbf{Multi-Modeling}, etc.\medskip
\item Furthermore, the possibility of hidden elements within the ontology implies the test for causal relations and intermediate processes at the origin of emergence (thus e.g. the exploration of new paradigms such as role of governance).\medskip
\item Finally, the idea behind system structure contained within the ontological forest is a large set of coupled models for a given system : it means that a proper system definition (i.e. thematic problematization and exploration) and construction should yield to a structured family of models : parallel branches can be different implementations of the same process or various processes trying to explain the emerging ontology ; therefore the final objective of a family of models tackling the thematic question.
\end{enumerate}

\end{justify}

\end{frame}


\section{Next Steps}

%%%%%%%%%%%%%%%%%%%%%%%%%%%%%%%%
\jframe{Next steps (until January 2016)}{
\begin{itemize}
\item Develop theory applications and examples [1w]\medskip
\item Construct the thematic problematization within the theoretical framework [1w]\medskip
\item Correlated Synthetic Data : exploration of network generation [0.5w]\medskip
\item Cybergeo [0.5w]\medskip
\item Monitorat 1w
\end{itemize}
}






%%%%%%%%%%%%%%%%%%%%%%%%%%%%%%%%
\begin{frame}[allowframebreaks]
\frametitle{References}
\bibliographystyle{apalike}
\bibliography{/Users/Juste/Documents/ComplexSystems/CityNetwork/Biblio/Bibtex/CityNetwork}
\end{frame}
%%%%%%%%%%%%%%%%%%%%%%%%%%%%%%%%


\end{document}
