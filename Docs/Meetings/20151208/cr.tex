%%%%%%%%%%%%%%%%%%%%%%%%%%%%%
% Standard header for working papers
%
% WPHeader.tex
%
%%%%%%%%%%%%%%%%%%%%%%%%%%%%%

\documentclass[11pt]{article}

% packages without options
\usepackage{amsmath,bbm}

% geometry
\usepackage[margin=2cm]{geometry}






\title{Compte-rendu R{\'e}unions Th{\`e}se\bigskip\\
%\textit{Working Paper}
}
%\author{\noun{Juste Raimbault}}
\date{}


\maketitle

\justify


\section*{R{\'e}union Arnaud 09/12}


\subsection*{Retour sur le doc}

\begin{itemize}
\item Repositionner mieux par rapport à la th{\'e}orie des syst{\`e}mes en g{\'e}ographie $\rightarrow$ Allen, Forrester etc. : relire th{\`e}se de Seb.
\item N{\'e}cessit{\'e} de developper sur un exemple, ou introduire la th. par un exemple ; ne pas rester dans la g{\'e}n{\'e}ralit{\'e} absolue qui peut rapidement ne plus avoir de sens ; ¡ th{\'e}orie $\neq$ outil, qui est plut{\^o}t ce qu'on d{\'e}veloppe ici. Importance de la d{\'e}marche de construction, donc de l'exemple amenant la th.
\item Mieux justifier choix perspectivisme (pour pas que ce soit uniquement vu comme d{\'e}coulant histoire perso)
\item D{\'e}tail : taclage de R{\'e}mi/Barth/Batty gratuit, citer un passage pr{\'e}cis avec mots exacts pour moins de gratuit{\'e}
\item Consulter Varenne ? Mauvaise id{\'e}e, risque d'y passer l'ensemble de la th{\`e}se apr{\`e}s
\end{itemize}

\subsection*{Directions th{\'e}matiques}

\begin{itemize}
\item Faire le lien avec ce qui ce fait dans diff{\'e}rents domaines : vraie {\'e}pist{\'e}mologie ; difficile, ne peut {\^e}tre qu'une {\'e}bauche
\item Economie vs geo : eco a le carcan des maths, vu comme une securit{\'e} par eco mais comme une restriction par geographes, qui s'adpatent aux objets. Voir point de vue architectes, urbanistes, physiciens. Jacob premi{\`e}re à penser en s.complexes. regarder l'objet ville, ce que signifie pour chacun, en quoi rel{\`e}ve de th. $\neq$.
\item Voir Offner, Orfeuil, Ascher, Roncayolo, Raffestin, Choay. (pas Berque trop d{\'e}lirant). 
\end{itemize}

\subsection*{R{\'e}seaux et territoires}

\begin{itemize}
\item clarifier la notion de territoire. quels objets simples ? pas ville car trop restrictif.
\item le r{\'e}seau n'existe-t-il pas d{\'e}j{\`a} quand les objets sont en interaction ? alors r{\'e}seau physique comme la concr{\'e}tisation d'un potentiel interaction, une seule partie du ph{\'e}nom{\`e}ne.
\item analogie electrique : que se passe t'il si capacit{\'e} infinie ? (// supraconducteurs ?)
\item s'abstraire de l'objet technique (mais pas totalement car il intervient alors par retroaction par congestion et effets offre/demande) 
\item Territoire initialement comme portion d'espace, population et ressources (distrib continue) $\rightarrow$ interactions $\rightarrow$ distance/h{\'e}t{\'e}rog{\'e}n{\'e}it{\'e} de l'espace comme frein à l'interaction $\rightarrow$ r{\'e}seau canalise, support des interactions, et retroaction sur les interactions. Quels types d'interactions, quelles echelles ? les couplages et boucles de retroaction s'op{\`e}rent à de multiples echelles.
\end{itemize}


\section*{R{\'e}union Florent 11/12}

\subsection*{Retour sur le doc}

\begin{itemize}
\item peu utile tel quel, besoin d'une application, exemple concret. tout de m{\^e}me importance d'une grille de lecture th{\'e}orique sous-jacente pour structurer d{\'e}marche.
\item Element manquant (en lien avec compatibilit{\'e} des ontologies ) : outil de mesure pour mesurer ``r{\'e}alit{\'e}'' d'une perspective : ontologie de ref qui mesure ? // 3 mondes de Popper (th., perception, r{\'e}alit{\'e} physique).
\item Mod{\'e}les pour comprendres $\neq$ mod{\'e}les d'aide d{\'e}cision (qui dit qque chose auquel les gens croient) : presque plus important de savoir ce qu'on en fait que la ``r{\'e}alit{\'e}''.
\item Notion d'h{\'e}t{\'e}rog{\'e}n{\'e}it{\'e} confuse : besoin d'empirique/th{\'e}orique (au sens th{\'e}matique) pour clarifier cela
\end{itemize}

\subsection*{Etude empirique}

\begin{itemize}
\item Obtenir des faits stylis{\'e}s dans un cadre th. simple pour commencer : prendre territoire avec population/ressources (emploi) avec accessibilit{\'e} (laquelle ? ex acc. Graham) donne d{\'e}j{\`a} un syst{\`e}me tr{\`e}s complexe.  Deux cas limite : le transport suit une demande (par ruptures li{\'e}s processus d{\'e}cision politique etc. ; et le transport cr{\'e}e demande, ville s'adapte en quelque sorte (ex. Deauville) (attention echelle mobilit{\'e}$\neq$ devlpmt urbain).
\item importance du multi echelle pour comprendre quel axe transport apparu dans quel cadre (ex. autoroute bassin parisien : A4 a progressivement chang{\'e} d'echelle d'utilisation, apparition d'{\'e}changeurs locaux, diminution de l'effet tunnel.
\item Donn{\'e}es IdF, bassin parisien {\'e}largi : autoroutes chronologique, P+E communes, flux (recensement). cas d'{\'e}tude interessant : N104 Roissy Cergy. Base BIEN ? peu concordance temporelle (surtout avec projets r{\'e}cents), difficilement utilisable ?
\item fait stylis{\'e}s : croissance du zonage en aires urbaines ; longueur des d{\'e}placements pendulaires ; emergence du peri-urbain.
\item Attention empirique : choisir quel niveau granularit{\'e}. Pas y passer trop de temps, peu vite devenir chronophage.
\item Etudes possibles ? Poser bien le th. pour avoir des hyp à tester, ex. $\Delta P \rightarrow \Delta T$, $\Delta P,\Delta E \rightarrow \Delta T$, $\Delta T \rightarrow \Delta X \rightarrow \Delta P, \Delta E$.
\item Test possible au del{\`a} simples correlations : Granger causation ($\sim$ lagged correlation) ; space and time ?
\end{itemize}






%%%%%%%%%%%%%%%%%%%%
%% Biblio
%%%%%%%%%%%%%%%%%%%%

\bibliographystyle{apalike}
\bibliography{/Users/Juste/Documents/ComplexSystems/CityNetwork/Biblio/Bibtex/CityNetwork}


\end{document}
