%%%%%%%%%%%%%%%%%%%%%%%%%
%% Header for standard beamer presentation
%%
%%  PresentationHeader.tex
%%
%%%%%%%%%%%%%%%%%%%%%%%%%

\documentclass[english,10pt]{beamer}

%%%%%%%%%%%%%%%%%%%%
%% Include general header where common packages are defined
%%%%%%%%%%%%%%%%%%%%

% general packages without options
\usepackage{amsmath,amssymb,bbm}




%%%%%%%%%%%%%%%%%%%%
%% Idem general commands
%%%%%%%%%%%%%%%%%%%%

%% Commands

\newcommand{\noun}[1]{\textsc{#1}}


%% Math

\DeclareMathOperator{\Cov}{Cov}
\newcommand{\Covb}[2]{\ensuremath{\Cov\!\left[#1,#2\right]}}





\usetheme{Warsaw}

\setbeamertemplate{footline}[text line]{}
\setbeamercolor{structure}{fg=purple!50!blue, bg=purple!50!blue}

\setbeamercovered{transparent}


% shortened command for a justified frame
\newcommand{\jframe}[2]{\frame{\frametitle{#1}\justify{#2}}}



%%%%%%%%%%%%%%%%%%%%%
%% Begin doc
%%%%%%%%%%%%%%%%%%%%%

\begin{document}



\title{Thesis Progress Meeting}


\author{J.~Raimbault$^{1,2}$}

\institute{$^{1}$G{\'e}ographie-cit{\'e}s (UMR 8504 CNRS)\\
$^{2}$LVMT (UMR-T 9403 IFSTTAR)}


\date{March 10th 2016}


%%%%%%%%%%%%%%%%%%%%%%%%%%%%%%%%
\begin{frame}
\titlepage
\end{frame}

%\begin{frame}
%\tableofcontents
%\end{frame}
%%%%%%%%%%%%%%%%%%%%%%%%%%%%%%%%



\section{Achieved Work}


\jframe{Achieved Work (by projects)}{
\begin{itemize}
\item Biblio/Meetings/Organisation [0.5w]
\item Seminars : Cartha-g{\'e}o-prisme ; mandatory English course [0.8w]
\item Memoire [2.2w] (ETA 2w)
\item Cybergeo Project [1.8w] (ETA 0.5w)
\item Network-Density Statistics [1.2w] (ETA 0.5w)
\end{itemize}
}



\section{Theoretical Framework}
% chapter one and two of Memoire

\sframe{Subject Construction}{
\textit{Definition of Territorial Systems ?}
\bigskip

$\rightarrow$ Raffestin Human Territoriality~\cite{raffestin1988reperes} to introduce the subject

$\rightarrow$ Privileged role of Networks, following Dupuy \textit{Territorial Theory of Networks}~\cite{dupuy1987vers}

$\rightarrow$ Debate on Structural Effects of Transportation Networks still active today~\cite{espacegeo2014effets}


}


\section{Methodology}
% detail osm nw simplification


\sframe{From Static Correlations to Dynamical Correlations}{
Assumptions on the spatio-temporal stochastic processes $Y_i\left[\vec{x},t\right]$ :
\begin{enumerate}
\item Local spatial autocorrelation is present on a maximal span of $l_{\rho}$ : for any $\vec{x}$ and $t$, $\left|\rho_{\norm{\Delta \vec{x}} < l_{\rho}}\left[Y_i (\vec{x}+\Delta \vec{x},t), Y_i (\vec{x},t) \right]\right| > 0$.
\item Processes are locally parametrized : $Y_i = Y_i\left[\alpha_i\right]$, where $\alpha_i (\vec{x})$ varies with $l_{\alpha}$, with $l_{\alpha} \gg l_{\rho}$.
\item Spatial correlations between processes have a sense at an intermediate scale $l$ such that $l_{\alpha}\gg l \gg l{\rho}$.
\item Processes covariance stationarity times scale as $\sqrt{l}$.
\item Local ergodicity is present at scale $l$ and dynamics are locally chaotic.
\end{enumerate}
}


\sframe{Road Network Simplification}{
OpenStreetMap Network
}







%%%%%%%%%%%%%%%%%%%%%%%%%%%%%%%%
\jframe{Next steps (until April 15th 2016)}{
\begin{itemize}
\item Theory exemplification, paper finalization  [1w]\medskip
\item Spatial \sout{Econometrics} Statistics / Case study [0.5w]\medskip
\item Cybergeo [0.5w]\medskip
\item Wrap everything within a 1-year Memoire [1w]\medskip
\end{itemize}
}






%%%%%%%%%%%%%%%%%%%%%%%%%%%%%%%%
\begin{frame}[allowframebreaks]
\frametitle{References}
\bibliographystyle{apalike}
\bibliography{/Users/Juste/Documents/ComplexSystems/CityNetwork/Biblio/Bibtex/CityNetwork}
\end{frame}
%%%%%%%%%%%%%%%%%%%%%%%%%%%%%%%%


\end{document}




