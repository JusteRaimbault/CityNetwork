\chapter{Technical Developments}{Technical Developments} % Chapter title

\label{app:technical} % For referencing the chapter elsewhere, use \autoref{ch:name} 

%----------------------------------------------------------------------------------------


\headercit{C'est hardcore tes calculs.}{Anonyme}{}



This chapter gathers various technical developments, that have the common points to be not essential to the core of the thesis and difficult to digest.


\section{Derivations for Urban Growth Models}{Dérivations pour les modèles de croissance urbaine}





\begin{lemma}
The limit of a Preferential Attachment model when $\lambda \ll 1$ is a linear-growth Gibrat model, with limit parameters $\mu_i(t)=1+\frac{\lambda}{m\cdot (t-1)}$.
\end{lemma}

\begin{proof}

Starting with first moment, we denote $\bar{P}_i(t)=\Eb{P_i(t)}$. Independence of Gibrat growth rate yields directly $\bar{P}_i(t)=\Eb{R_i(t)}\cdot \bar{P}_i(t-1)$. Starting for the preferential attachment model, we have $\bar{P}_i(t) = \Eb{P_i(t)} = \sum_{k=0}^{+\infty}{k\Pb{P_i(t)=k}}$. But
\[
\{P_i(t)=k\}=\bigcup_{\delta=0}^{\infty}{\left(\{P_i(t-1)=k-\delta\}\cap \{P_i\leftarrow P_i + 1\}^{\delta}\right)}
\]

where the second event corresponds to city $i$ being increased $\delta$ times between $t-1$ and $t$ (note that events are empty for $\delta \geq k$). Thus, being careful on the conditional nature of preferential attachment formulation, stating that $\Pb{\{P_i\leftarrow P_i + 1\} | P_i(t-1)=p} = \lambda\cdot\frac{p}{P(t-1)}$ (total population $P(t)$ assumed deterministic), we obtain

\begin{equation*}
\begin{split}
\Pb{\{P_i\leftarrow P_i + 1\}} & = \sum_{p}{\Pb{\{P_i\leftarrow P_i + 1\} | P_i(t-1)=p}\cdot \Pb{P_i(t-1)=p}}\\
&=\sum_{p}{\lambda\cdot\frac{p}{P(t-1)}\Pb{P_i(t-1)=p}}=\lambda\cdot\frac{\bar{P}_i(t-1)}{P(t-1)}\\
\end{split}
\end{equation*}

It gives therefore, knowing that $P(t-1)=P_0 + m\cdot (t-1)$ and denoting $q=\lambda\cdot\frac{\bar{P}_i(t-1)}{P_0 + m\cdot (t-1)}$

\[
\begin{split}
\bar{P}_i(t) & =\sum_{k=0}^{\infty}{\sum_{\delta=0}^{\infty}{k\cdot \left(\lambda\cdot\frac{\bar{P}_i(t-1)}{P_0 + m\cdot (t-1)}\right)^{\delta}\cdot \Pb{P_i(t-1)=k-\delta}}}\\
& = \sum_{\delta^{\prime}=0}^{\infty}{\sum_{k^{\prime}=0}^{\infty}{\left(k^\prime + \delta^{\prime}\right)\cdot q^{\delta^{\prime}} \cdot \Pb{P_i(t-1)=k^\prime}}}\\
& = \sum_{\delta^{\prime}=0}^{\infty}{q^{\delta^{\prime}}\cdot \left(\delta^{\prime} + \bar{P}_i(t-1)\right)} = \frac{q}{(1-q)^2} + \frac{\bar{P}_i(t-1)}{(1-q)}\\
& = \frac{\bar{P}_i(t-1)}{1-q}\left[1 + \frac{1}{\bar{P}_i(t-1)}\frac{q}{(1-q)}\right]
\end{split}
\]

%& = \bar{P}_i(t-1)\cdot \frac{1}{1-\lambda\cdot\frac{\bar{P}_i(t-1)}{P_0 + m\cdot (t-1)}} \left[1 + \frac{\lambda}{P_0 + m\cdot (t-1)}\cdot \frac{1}{1-\lambda\cdot\frac{\bar{P}_i(t-1)}{P_0 + m\cdot (t-1)}} \right]


As it is not expected to have $\bar{P}_i(t)\ll P(t)$ (fat tail distributions), a limit can be taken only through $\lambda$. Taking $\lambda \ll 1$ yields, as $0 < \bar{P}_i(t)/P(t) < 1$, that $q=\lambda\cdot\frac{\bar{P}_i(t-1)}{P_0 + m\cdot (t-1)} \ll 1$ and thus we can expand in first order of $q$, what gives $\bar{P}_i(t)=\bar{P}_i(t-1)\cdot \left[1 + \left(1+\frac{1}{\bar{P}_i(t-1)}\right)q + o(q))\right]$

\[
\bar{P}_i(t) \simeq \left[1 + \frac{\lambda}{P_0 + m\cdot (t-1)}\right]\cdot \bar{P}_i(t-1)
\]

It means that this limit is equivalent in expectancy to a Gibrat model with $\mu_i(t) = \mu(t)=1 + \frac{\lambda}{P_0 + m\cdot (t-1)}$.

For the second moment, we can do an analog computation. We have still \[\Eb{P_i(t)^2} = \Eb{R_i(t)^2}\cdot \Eb{P_i(t-1)^2}\]
and
\[\Eb{P_i(t)^2}=\sum_{k=0}^{+\infty}{k^2 \Pb{P_i(t)=k}}\] 

We obtain the same way 

\[
\begin{split}
\Eb{P_i(t)^2} & = \sum_{\delta^{\prime}=0}^{\infty}{\sum_{k^{\prime}=0}^{\infty}{\left(k^\prime + \delta^{\prime}\right)^2\cdot q^{\delta^{\prime}} \cdot \Pb{P_i(t-1)=k^\prime}}}\\ 
& = \sum_{\delta^{\prime}=0}^{\infty}{q^{\delta^{\prime}}\cdot \left(\Eb{P_i(t-1)^2}+2\delta^{\prime}\bar{P}_i(t-1) + {\delta^{\prime}}^2\right)}\\
& = \frac{\Eb{P_i(t-1)^2}}{1-q} + \frac{2 q \bar{P}_i(t-1)}{(1-q)^2} + \frac{q(q+1)}{(1-q)^3}\\
& = \frac{\Eb{P_i(t-1)^2}}{1-q}\left[1 + \frac{q}{\Eb{P_i(t-1)^2}}\left(\frac{2\bar{P}_i(t-1)}{1-q} + \frac{(1+q)}{(1-q)^2}\right)\right]
\end{split}
\]



We have therefore an equivalence between the Gibrat model as a continuous formulation of a Preferential Attachment (or Simon model) in a certain limit. \qed

\end{proof}







%-------------------------------------------------------




%%%%%%%%%%%%%%%%%%%%
\section{Sensitivity of Urban Scaling}{Sensibilité des Lois d'Echelle Urbaines}

\label{sec:formalization}

We formalize the simple theoretical context in which we will derive the sensitivity of scaling to city definition. Let consider a polycentric city system, which spatial density distributions can be reasonably constructed as the superposition of monocentric fast-decreasing spatial kernels, such as an exponential mixture model~\cite{anas1998urban}. Taking a geographical space as $\mathbb{R}^2$, we take for any $\vec{x}\in\mathbb{R}^2$ \comment{(Florent) attention à la sensibilité de certains géographes}
the density of population as
\begin{equation}
d(\vec{x}) = \sum_{i=1}^{N}{d_i(\vec{x})} = \sum_{i=1}^{N}{d_i^0\cdot \exp{\left(\frac{-\norm{\vec{x}-\vec{x}_i}}{r_i}\right)}}
\end{equation}

where $r_i$ are spread parameters of kernels, $d_i^0$ densities at origins, $\vec{x}_i$ positions of centers. We furthermore assume the following constraints :

\begin{enumerate}
\item To simplify, cities are monocentric, in the sense that for all $i\neq j$, we have $\norm{\vec{x}_i - \vec{x}_j}\gg r_i$.
\item It allows to impose structural scaling in the urban system by the simple constraint on city populations $P_i$. One can compute by integration that $P_i=2\pi d_i^0 r_i^2$, what gives by injection into the scaling hypothesis $\ln{P_i}=\ln{P_{max}}-\alpha \ln{i}$, the following relation between parameters : $\ln{\left[d_i^0 r_i^2\right]}=K' - \alpha \ln{i}$.
\end{enumerate}

To study scaling relations, we consider a random scalar spatial variable $a(\vec{x})$ representing one aspect of the city, that can be everything but has the dimension of a spatial density, such that the indicator $A(D)=\Eb{\iint_D{a(\vec{x})d\vec{x}}}$ represents the expected quantity of $a$ in area $D$. We make the assumption that $a\in \{0;1\}$ (``counting'' indicator) and that its law is given by $\Pb{a(\vec{x})=1}=f(d(\vec{x}))$. Following the empirical work done in~\cite{cottineau2015scaling}, the integrated indicator on city $i$ as a function of $\theta$ is given by
\[
A_i(\theta) = A(D(\vec{x}_i, \theta))
\]

where $D(\vec{x}_i, \theta)$ is the area centered in $\vec{x}_i$ where $d(\vec{x})>\theta$. Assumption 1 ensures that the area are roughly disjoint circles. We take furthermore a simple amenity such that it follows a local scaling law in the sense that $f(d)=\lambda\cdot d^\beta$. It seems a reasonable assumption since it was shown that many urban variable follow a fractal behavior at the intra-urban scale~\cite{keersmaecker2003using} and that it implies necessarily a power-law distribution~\cite{chen2010characterizing}. We make the additional assumption that $r_i=r_0$ does not depend on $i$, what is reasonable if the urban system is considered from a large scale. This assumption should be relaxed in numerical simulations. The estimated scaling exponent $\alpha(\theta)$ is then the result of the log-regression of $(A_i(\theta))_i$ against $(P_i(\theta))_i$ where $P_i(\theta)=\iint_{D(\vec{x}_i,\theta)}{d}$.


%%%%%%%%%%%%%%%%%%%%
\subsection{Analytical Derivation of Sensitivity}{Dérivation Analytique de la Sensibilité}

With above notations, let derive the expression of estimated exponent for quantity $a$ as a function of density threshold parameter $\theta$. The quantity computed for a given city $i$ is, thanks to the monocentric assumption and in a spatial range and a range for $\theta$ such that $\theta \gg \sum_{j\neq i}{d_j(\vec{x})}$, allowing to approximate $d(\vec{x})\simeq d_i(\vec{x})$ on $D(\vec{x}_i,\theta)$, is computed by
\[
\begin{split}
A_i(\theta) & = \lambda\cdot \iint_{D(\vec{x}_i,\theta)}{d^\beta} = 2\pi\lambda {d_i^0}^{\beta} \int_{r=0}^{r_0 \ln{\frac{d_i^0}{\theta}}}{r\exp{\left(-\frac{r\beta}{r_0}\right)}dr}\\
& = \frac{2\pi {d_i^0}^\beta r_0^2}{\beta^2} \left[1 + \beta \ln{\frac{\theta}{d_i^0}\left(\frac{\theta}{d_i^0}\right)^\beta} - \left(\frac{\theta}{d_i^0}\right)^\beta\right]
\end{split}
\]

We obtain in a similar way the expression of $P_i(\theta)$
\[
P_i(\theta) = 2\pi d_i^0 r_0^2 \left[1 + \ln{\left[\frac{\theta}{d_i^0}\right]}\frac{\theta}{d_i^0} - \frac{\theta}{d_i^0}\right]
\]

The Ordinary-Least-Square estimation, solving the problem $\inf_{\alpha,C}\norm{(\ln{A_i(\theta)} - C - \alpha \ln{P_i(\theta)})_i}^2$, gives the value $\alpha(\theta) = \frac{\Covb{(\ln{A_i(\theta)})_i}{(\ln{P_i(\theta)})_i}}{\Varb{(\ln{P_i(\theta)})_i}}$. As we work on city boundaries, threshold is expected to be significantly smaller than center density, i.e. $\theta / d_i^0 \ll 1$. We can develop the expression in the first order of $\theta / d_i^0$ and use the global scaling law for city sizes, what gives $\ln{A_i(\theta)} \simeq K_A - \alpha \ln{i} + (\beta - 1)\ln{d_i^0} + \beta \ln{\frac{\theta}{d_i^0}\left(\frac{\theta}{d_i^0}\right)^\beta} $ and $\ln{P_i(\theta)} = K_P - \alpha \ln{i} + \ln{\left[\frac{\theta}{d_i^0}\right]}\frac{\theta}{d_i^0}$. Developing the covariance and variance gives finally an expression of the scaling exponent as a function of $\theta$, where $k_j,{k_j}'$ are constants obtained in the development :

\begin{equation}
\label{eq:th}
\alpha(\theta) = \frac{k_0 + k_1 \theta + k_2 \theta^\beta + k_3 \theta^{\beta + 1} +  k_4 \theta \ln{\theta} + k_5 \theta^\beta \ln{\theta} + k_6 \theta^\beta (\ln{\theta})^2 + k_7 \theta^{\beta + 1}(\ln{\theta})^2 + k_8 \theta^{\beta + 1}\ln{\theta}}{k_0'+k_1' \ln{\theta} + k_2' \theta \ln{\theta} + k_3' \theta^2 + k_4' \theta^2\ln{\theta} + k_5' \theta^2 (\ln{\theta})^2}
\end{equation}

This rational fraction predicts the evolution of the scaling exponent when the threshold varies. We study numerically its behavior in the next section, among other numerical experiments.


%%%%%%%%%%%%%%%%%%%%
\subsection{Numerical Simulations}{Simulations Numériques}

\paragraph{Implementation}{Implémentation}


\comment{(Florent) définir ton champ d'investigation (des grilles carrées de taille prédéfinies, ce n'est pas du tout standard)}

We implement empirically the density model given in section~\ref{sec:formalization}. Centers are successively chosen such that in a given region of space only one kernel dominates in the sense that the sum of other contributions are above a given threshold $\theta_e$.\comment{(Florent) est-ce toujours possible, y'a t-il unicité du centre ? Par quelle méthode précise détermine tu le centre ?}
 In practice, adapting $N$ to world size allows to respect the monocentric condition. Population are distributed in order to follow the scaling law with fixed $\alpha$ and $r_i$ (arbitrary choice) by computing corresponding $d_i^0$. Technical details of the implementation done in R~\cite{R-Core-Team:2015fk} and using the package \texttt{kernlab} for efficient kernel mixture methods~\cite{Karatzoglou:2004uq} are given as comments in source code\footnote{available at \texttt{https://github.com/JusteRaimbault/CityNetwork/tree/master/Models/Scaling}}. \comment{(Florent) cela ne suffit pas, il faut en dire plus sur la méthode}[sure, surtout qu'on formule cette requete dans la partie méthodologique précédente, tout cela est un peu contradictoire..]
  We show in figure~\ref{fig:ex-distrib} example of synthetic density distributions on which the numerical study is conducted. The validation of theoretical results on these experimental mixtures must still be conducted, along with sensitivity tests to random perturbations, influence of kernel type, and two-parameters phase diagram when adding in the computational model functional density distribution and associated cut-off threshold.
%Theoretical result obtained in Eq.~\ref{eq:th} are studied and confronted to emprically computed values for various parameter as shown in Fig.~\ref{fig:th_results}.

\comment{(Florent) TB mais la encore, on ne sait pas précisément pourquoi tu te lances là dedans}


%%%%%%%%%%%%%%%%%%
\begin{figure}
\centering
\includegraphics[width=0.4\textwidth]{Figures/Scaling/example_exp_mixture}
\caption[Synthetic density distribution]{Example of a synthetic density distribution obtained with the exponential mixture, with a grid of size $400\times 400$ and parameters $N=20$, $r_0=10$, $P_{max}=200$, $\alpha=0.5$, $\theta_C = 0.01$.}{}
\label{fig:ex-distrib}
\end{figure}
%%%%%%%%%%%%%%%%%%



%\begin{figure}
%\centering
%
%\caption{Validation of theoretical result through numerical simulation.}
%\label{fig:th_results}
%\end{figure}



\paragraph{Random Perturbations}{Perturbations aléatoires}

The simple model used is quite reducing for maximal densities and radius distribution. We aim to proceed to an empirical study of the influence of noise in the system by fixing $d_i^0$ and $r_i$ the following way :
\begin{itemize}
\item $d_i^0$ follows a reversed log-normal distribution with maximal value being a realistic maximal density
\item Radiuses are computed to respect rank-size law and then perturbed by a white noise. \comment{(Florent)  pourquoi ?}
\end{itemize}

%Results shown in Fig.~\ref{fig:random-density} are quantitatively different from previous one, as expected, but the same qualitative behavior is reproduced.


%\begin{figure}
%\centering

%\caption{Variation of exponents with variable origin density and radius.}
%\label{fig:random-density}
%\end{figure}



\paragraph{Kernel Type}{Type de Noyau}

We shall test the influence of the type of spatial kernel used on results. We can test gaussian kernels and quadratic kernels with parameters within reasonable ranges analog to the exponential kernel. %As shown in Fig.~\ref{fig:other-kernels}, we obtain the same qualitative results that is the significant variation of $\alpha(\theta)$ as a function of $\theta$.


%
%\begin{figure}
%\centering

%\caption{Scaling exponents for other kernels.}
%\label{fig:other-kernels}
%\end{figure}

%\paragraph{Two-parameters phase diagram}

%We introduce now a second spatial variable that has also an influence on the definition of urban entities, that is the proportion of actives working in city center, as done on empirical data in~\cite{cottineau2015scaling}. To simplify, it is used only to define urban parameter but assumed as having no influence on the local probability distribution of the amenity which stays the same function of the density. We write 

%\begin{figure}
%\centering

%\caption{Two parameters phase diagram.}
%\label{fig:two-params}
%\end{figure}

%%%%%%%%%%%%%%%%%%%%
%\subsection{Discussion}

%%%%%%%%%%%%%%%%%%%%
%\subsection{Conclusion}













