% Publications - a page listing research articles written using content in the thesis

\pdfbookmark[1]{Publications}{Publications} % Bookmark name visible in a PDF viewer

\chapter*{Publications}{Publications} % Publications page text

%Some ideas and figures have appeared previously in the following publications:\\

%\noindent Put your publications from the thesis here. The packages \texttt{multibib} or \texttt{bibtopic} etc. can be used to handle multiple different bibliographies in your document.

%\begin{refsection}[ownpubs]
%    \small
%    \nocite{*} % is local to to the enclosing refsection
%    \printbibliography[heading=none]
%\end{refsection}

%\emph{Attention}: This requires a separate run of \texttt{bibtex} for your \texttt{refsection}, \eg, \texttt{ClassicThesis1-blx} for this file. You might also use \texttt{biber} as the backend for \texttt{biblatex}. See also \url{http://tex.stackexchange.com/questions/128196/problem-with-refsection}.

\bpar{
The following works have an highly overlapping content with this thesis:
}{
Les travaux suivants contiennent une grande partie du contenu de cette thèse:
}

% NOTE : on self-plagiarism, be careful to precise when extract from a published paper : 
%  - http://academia.stackexchange.com/questions/12342/self-plagiarism-in-phd-thesis
%  - http://academia.stackexchange.com/questions/2029/can-i-use-the-work-in-my-journal-conference-publications-as-chapters-in-my-disse
%  - http://academia.stackexchange.com/questions/149/what-is-a-sandwich-thesis


\section*{Publications}{Publications}


\noindent Antelope, C., Hubatsch, L., Raimbault, J., and Serna, J. M. (2016). An interdisciplinary approach to morphogenesis. Forthcoming in Proceedings of Santa Fe Institute CSSS 2016.


\bigskip

\noindent Raimbault, J. (2017). A Discrepancy-Based Framework to Compare Robustness Between Multi-attribute Evaluations. In Complex Systems Design \& Management (pp. 141-154). Springer International Publishing. \cite{raimbault2016discrepancy}

\bigskip

\noindent Raimbault, J. (2016). Investigating the Empirical Existence of Static User Equilibrium, \textit{forthcoming in EWGT 2016 proceedings, Transportation Research Procedia.} arxiv:1608.05266 \cite{raimbault2016investigating}


\bigskip


\noindent Raimbault, J. (2016). Generation of Correlated Synthetic Data, forthcoming in \textit{Actes des Journ{\'e}es de Rochebrune 2016.}


\bigskip

\noindent Raimbault, J. (2015). Models Coupling Urban Growth and Transportation Network Growth: An Algorithmic Systematic Review Approach, forthcoming in \textit{ECTQG 2015 proceedings.} arxiv:1605.08888


\section*{Communications}{Communications}

\noindent Towards a Theory of Co-evolutive Networked Territorial Systems: Insights from Transportation Governance Modeling in Pearl River Delta, China, \textit{MEDIUM Seminar : Sustainable Development in Zhuhai, Guangzhou, Dec 2016.}


\bigskip


\noindent Models of growth for system of cities : Back to the simple, \textit{Conference on Complex Systems 2016, Amsterdam, Sep 2016.}



%Raimbault J., Bergeaud A. and Potiron Y. (2016). Investigating Patterns of Technological Innovation. \textit{Conference on Complex Systems 2016, Amsterdam, Sep 2016.}


\bigskip

\noindent For a Cautious Use of Big Data and Computation. \textit{Royal Geographical Society - Annual Conference 2016 - Session : Geocomputation, the Next 20 Years (1), London, Aug 2016.}


\bigskip

\noindent Indirect Bibliometrics by Complex Network Analysis. \textit{20e Anniversaire de Cybergeo, Paris, May 2016.}


\bigskip

\noindent Raimbault, J. \& Serra, H. (2016). Game-based Tools as Media to Transmit Freshwater Ecology Concepts, \textit{poster corner at SETAC 2016 (Nantes, May 2016).}


\bigskip

\noindent Le Néchet, F. \& Raimbault, J. (2015). Modeling the emergence of metropolitan transport authority in a polycentric urban region, \textit{ECTQG 2015, Bari, Sep 2015).}


\bigskip

\noindent Hybrid Modeling of a Bike-Sharing Transportation System, \textit{poster presented at ICCSS 2015, Helsinki, June 2015.}

\bigskip

\noindent Raimbault, J. \& Gonzales, J. (2015). Application de la Morphog{\'e}n{\`e}se de R{\'e}seaux Biologiques {\`a} la Conception Optimale d'Infrastructures de Transport, \textit{poster presented at Rencontres du Labex Dynamite, Paris, May 2015.}


