



\chapter*{Introduction}{Introduction}

% to have header for non-numbered introduction
\markboth{Introduction}{Introduction}


\headercit{}{}{}





\bigskip




\begin{table}
\begin{tabular}[6pt]{c|c|c}
Processus & Analyse Empirique & Echelles & Type \comment{find a typology of processes} & Modèle \\\hline
Attachement préférentiel & & Croissance Urbaine & \\\hline
Diffusion/Etalement & & Forme Urbaine & \\\hline
Accessibilité  & & Réseau / Ville & \\\hline
Gouvernance des Transports & & & \\\hline
Flux direct  & & & \\\hline
Flux indirect/Effet tunnel \comment{c'est le même processus, vu sous un angle différent : l'effet tunnel est l'absence de nw feedback} & & & \\\hline
Centralité de proximité (accessibilité : generalisation) & & & \\\hline
Centralité de Chemin (correspond aux flux indirect : différents niveaux de généralité / sous-processus-sous-classif ?) & & & \\\hline
Proximité au réseau & & & \\\hline
Distance au centre (= agrégation ?) & & & RBD \\\hline
\end{tabular}
\caption{Description des différents processus pris en compte dans les modèles de co-évolution}
\end{table}


\comment{faire le même tableau pour les modèles existants : vue plus large de l'ensemble des processus. pour chacun de ces modèles et de nos modèles, lister tous les processus potentiels ; faire une typologie ensuite. Q : typologie différente d'une pure empirique ? a creuser, et peut être intéressant dans le cadre du knowledge framework, comme illustration coevol connaissances.}


