

% Chapter 

\chapter{Empirical Analysis : Insights from Stylized Facts}{Analyse Empirique} % Chapter title

\label{ch:empirical} % For referencing the chapter elsewhere, use \autoref{ch:name} 

%----------------------------------------------------------------------------------------


\headercit{Mais ce n'est pas une question d'{\^a}ge, de chiffres et de stats\\ Moi je te parle surtout de rage, de kif et d'espoir}{Youssoupha}{\textit{, Esperance de Vie}}


\bigskip


%  plan : 

%  1) static morphological analysis : requires a formal link between temporal and spatial correlations ?  -- typology etc can already be interesting --


%  2) presentation of BP case study

%  3) base Bien

%  4) Work with Solène



%----------------------------------------------------------------------------------------


As this quote suggests, a purely quantitative view of the world makes no sense without qualitative counterbalancing. More precisely, we argue that the \textit{clich{\'e}} of an opposition between quantitative and qualitative analysis is an illusion. No distinct boundary exists between both. We propose to call quantitative any process involving computation by a Turing machine, whereas the qualitative will be for us the modeling design process and its interpretations. \comment{(Florent) je ne sais pas si je rangerais l'interprétation dans le qualitatif ; ok pour dire (même si connait rien en machine de Turing) que certaines observations via ``Turing'' peuvent s'appeler quantitatives. mais dans un cas comme dans l'autre, ensuite, il faut interpreter}
 Therefore both are necessarily closely interlaced in any of our approaches. In particular concerning the construction and the validation or refutation of our theory, empirical analysis on real case studies, implying the extraction and qualification of stylized facts, follows that schema.

% articulation with theoretical questions
% articulation with modeling

\comment{(Florent) Entre cette phrase (next) et le détail projet par projet, il serait bien que tu te positionnes sur le cadre analytique général, justement comme tu l'introduis en parlant des objets et des échelles}

We propose in this chapter various empirical analysis on different objects at different scales. A first section begins the examination of static spatial correlations between morphological measures of population density and road network measures on Europe at a 500m resolution. Applying last section of the methodological chapter should provide information on typical spatial scales of interaction between these indicators \comment{(Florent) à quel niveau se situe l'indicateur ?}
 of territory and network and on dynamical correlations between these. These computation furthermore provide empirical measures on which one model will be calibrated. We then describe a roadmap for statistical analysis on dynamical data of interactions for Bassin Parisien in the last fifty years. An other project using Real Estate transaction data for Parisian Metropolitan Region aim at seeking early warning of network breakdowns. We finally describe potential analyses on South African historical data. \comment{(Florent) ah bon et maintenant la Chine ? :) }



%----------------------------------------------------------------------------------------


%%%%%%%%
%  Section : static analysis
%%%%%%%%

\newpage

\section{Static correlations of urban form and network shape}{Corrélations Statiques entre Forme Urbaine et Forme de Réseau}


\comment{(Florent) c'est trop technique comme entrée en matière ; pourquoi faudrait il qu'il y ait de la diffusion ? et de quels processus parles tu ? la forme urbain / réseaux / les deux ?}

Spatio-temporal processes implying diffusion or propagation phenomena generally have a specific structure of correlation. In particular, as derived in section~\ref{sec:spatiotempcorrs}, a static computation of correlation between different instances of a system may under certain conditions provide information on dynamical correlations implied.


% justify scale studied here


%%%%%%%%%%%%%%%%%%
\subsection{Morphological Measures of European Population Density}{Mesures morphologiques de la densité de population européenne}

\subsubsection{Context}{Contexte}


\bpar{
At the macroscopic scale of system of cities, spatialization \comment{(Florent) sens ?}
 of the urban system is reasonably captured by cities position, associated with aggregated city variable to represent entirely the system (see e.g. ontologies of Simpop models~\cite{pumain2012multi} or its successor Marius~\cite{cottineau2014evolution}). At the mesoscopic scale at which we aim to capture morphological manifestations of interactions between transportation networks and territories, structure of the territorial system can be specified by more refined indicators for the morphological aspect.
% TODO biblio spatial structure etc Florent
}{
A l'échelle macroscopique du système de ville, le caractère spatial du système urbain est capturé de manière raisonnable par les positions des villes, associées aux variables agrégées au niveau de la ville qui représentent entièrement le système (voir e.g. l'ontologie des modèles Simpop~\cite{pumain2012multi} ou de leur successeur Marius~\cite{cottineau2014evolution}). A l'échelle mesoscopique, à laquelle nous nous attendons à capturer des manifestations morphologiques des interactions entre ville et transport, la structure du système territorial peut être spécifiée par des indicateurs plus raffinés pour l'aspect morphologique. \comment{(Florent) sur quels critère juger de la pertinence d'un indicateur ?}
}


\subsubsection{Empirical Analysis}{Analyse Empirique}

We study systematically morphological indicators for constant size areas covering European Community. The choice of fixed size areas can be questioned regarding definition of a territorial system, that can be otherwise understood as a consistent spatial entity at a given scale and along certain criteria : \emph{Human territories} as defined by Raffestin (op. cit.) or more generally functionally autonomous spaces\footnote{for example, a tentative of definition of a \textit{Parisian} territory would present many facets. From the subjective territory point of view, intra-muros Parisians consider a strict boundary at \textit{Boulevard Periph{\'e}rique}, \comment{(Florent) attention à bien intégrer les travaux des géographes sur cette épineuse Q cf Guerois Paulus 2002 \cite{guerois2002commune}}
 whereas close and even further suburbs will be seen as Parisians from the Province. The functional territory of \textit{Metropolitain} extends slightly further than the administrative boundary. \comment{(Florent)laquelle la région, pas Paris}
  Governance perimeters are currently mutating with the Metropolitan governance project. Complementary perceptions of the territory can thus be multiplied.}. Here we choose the mesoscopic scale of a metropolitan center ($\simeq$ 50km) for comparability purposes and because greater scale are no more relevant regarding urban form, whereas smaller scales must contain too much noise. 
  
  \comment{(Florent) tu ne peux pas aller aussi vite sans choquer tout géographe quanti ; le fait de prendre des carrés 50x50 est arbitraire et si cela peut se justifier, il ne faut pas prétendre que tu as là des territoires comparables}

Data is the European population density grid~\cite{eurostat} \comment{(Florent) qui a été critiquée (cf Bretagnolle)}
 and indicators computation is implemented in parallel using \texttt{R} with Fast convolution raster functions. We show in next figures computed values of morphological indicators (see \cite{le2015forme} for a precise formulation of indicators that are Moran index, average distance, entropy and hierarchy). \comment{(Florent) pourquoi ceux là ?}



%%%%%%%%%%%%%%%%%%
\begin{figure}
\includegraphics[width=1.2\textwidth]{Figures/Static/Density/hists_GOOD}
\caption[Empirical Distribution of Morphological Indicators]{Empirical Distribution of Morphological Indicators}{Distribution empirique des indicateurs morphologiques\comment{(Florent) titres/labels illisibles ; dans quelle perspective as tu calculé ces indicateurs/affiché l'histogramme ?}}
\end{figure}
%%%%%%%%%%%%%%%%%%

%%%%%%%%%%%%%%%%%%
\begin{figure}
\hspace{-5cm}
\includegraphics[angle=90,width=1.7\textwidth,height=\textheight]{Figures/Static/Density/all_50km}
\caption[Geographical Distribution of Morphologies]{Geographical Distribution of Morphologies : value of indicators across Europe.}{}
\end{figure}
%%%%%%%%%%%%%%%%%%


%%%%%%%%%%%%%%%%%%
\begin{figure}
\hspace{-3cm}
\includegraphics[angle=90,width=1.7\textwidth,height=\textheight]{Figures/Static/Density/clust_k3-11}
\caption[Clustering Analysis of Morphologies]{Clustering Analysis of Morphologies. We present the results of an average k-means for different values of }{\comment{(Florent) attention au positionnement non controlé des figures en latex parfois malheureux comme ici}[pb with page title ? add mdframed ?]}
\end{figure}
%%%%%%%%%%%%%%%%%%






\subsubsection{Further developments}{Développements}

In \cite{10.1371/journal.pone.0107042} density grids for other countries across the world (ex. China) are provided\footnote{available at \url{http://www.worldpop.org.uk/}} soo we may repeat our analysis to other regions for comparison purposes. \comment{(Florent) comparer quoi ?}



%%%%%%%%%%%%%%%%%%
\subsection{Network Measures}{Mesures de Réseau}

\comment{(Florent) et entre les deux il y a donc les indicateurs d'accessibilité que tu as évacué trop vite je pense (d'autant qu'étant pile à l'interface forme urbaine/réseau - ils me semblent particulièrement indiqués vu ta problématique}

\todo{redo these analyses with accessibility indicators}

We consider network aggregated indicators as a way to characterize transportation network properties on a given territory, the same way morphological indicators yielded information on urban structure. We propose to compute some simple indicators on same extents as for morphology, to be able to explore relations between these static measures. Static network analysis has been extensively documented in the literature, see \cite{louf2014typology} for a cross-sectional study of cities or \cite{2015arXiv151201268L} for exploration of new measures for the road network.


\subsubsection{Data preprocessing}{Pré-traitement des données}

We work in a first time on road network, which structure is finely conditioned to territorial configuration of population densities. Furthermore, data for present day road network is available through the OpenStreetMap project~\cite{openstreetmap}. Its quality was investigated for different countries such as England~\cite{haklay2010good} and France~\cite{girres2010quality}. It was found to be of a quality equivalent to official surveys for the primary road network.

% data collected from http://download.geofabrik.de/europe.html

% TODO : remarque : why not include artwork as appendix. science<->art (// poésies ?)



\paragraph{Simplification algorithm}{Algorithme de simplification}

For a given dataset corresponding to a subset of the overall road network, it is necessary to simplify network structure by spatial aggregation as initial data presents very detailed features and thus a very large numbers of nodes ($\simeq 10^10$ for Europe dataset). \comment{(Florent) c'est un peu confus, tu devrais d'abord dire : ce que tu as, les pb que ça pose, comment les résoudre}
 Such a level of precision is not needed in our study since density data is already aggregated at 500m resolution. It is possible to drastically reduce network size by spatial aggregation of nodes and link replacements. More precisely we use the following procedure :
\begin{itemize}
\item a background raster (which resolution $r$ gives the snapping parameter for aggregation) is constructed from a reference raster and the extent of network. This grid gives spatial aggregation units for network nodes.
\item for each feature of the road dataset, corresponding connected raster cells are stored with corresponding impedance and distance in a sparse adjacency matrix.
\item Network is simplified by iterative suppression of nodes with degree two, with keeping link speed and real length to their effective value.
\end{itemize}

% \cite{2016arXiv161101890B} : interactive appli for network simplification



\paragraph{Implementation}{Implémentation}

A \texttt{PostGIS} database is used to store raw and simplified network, in order to perform efficient spatial requests, compared for example to initial \texttt{osm} data formats (\texttt{osm} or \texttt{pbf}). However the size of storage of data into this base is much higher (factor 10) so processing was parallelized between european countries. Consistence is ensured by the use of the same common density raster as simplification canvas. Final network is stored into the Postgis database for efficient indicator computation given a spatial extent. \comment{(Florent) y'a t'il un effet de bord dans les carrés 50x50 qui se trouvent à la frontière de 2 pays}[pas avec nouvelle parallelisation pas par pays mais par split and merge (TODO rewrite nouvel algo)]


\paragraph{Sensitivity to simplification parameters}{Sensibilité aux paramètres de simplification}

Sensitivity of indicators to raster resolution and to degree simplification algorithm must still be tested to ensure the relevance of data preprocessing.


\subsubsection{Indicators}{Indicateurs}

Network macroscopic structure is summarized by the following set of indicators, after the simplifications and reductions done in the previous step. Assuming network given by $N=(V,E)$, nodes spatial positions $\vec{x}(V)$ and edges \emph{effective distances} $d(E)$ taking into account impedances and real distances (to include basically network hierarchy), we have indicators :

\comment{(Florent) tb à présenter de la même manière, plus en même temps + justification pour la forme urbaine}

\begin{itemize}
\item connectivity
\item degree distribution
\item centrality, taken as normalized mean \emph{betweenness-centrality}
\item average path length
\item network diameter
\item mean network speed
\end{itemize}

These indicators are used to capture a rough picture of the structure. Refined work at smaller scales (intra-urban road network) and with more elaborated measures that allow to differentiate more precisely local form, was recently done by Lagesse in~\cite{2015arXiv151201268L}.



\subsubsection{Results}{Résultats}


\bpar{}{
Les indicateurs de réseau ont été calculés sur des zones similaires aux indicateurs de forme urbaine, 

}






%%%%%%%%%%%%%%%%%%
\subsection{Effective static correlations}{Correlations Statiques Effectives}







%%%%%%%%%%%%%%%%%%
\subsection{Spatial non-stationarity and non-ergodicity}{Non-stationnarité spatiale et non-ergodicité}



Case study : Context and Rationale
% interactions between networks and territories

\textbf{Study of interactions between network and territories :}

\medskip

$\rightarrow$ \textit{searching for stylized facts, what can be learnt from static correlations between urban form and road network ?}

\bigskip

\textbf{Theoretical Background : } \textit{A Theory of co-evolutive networked human territories} proposed in~\cite{raimbault2016memoire}, that in particular postulates an important role of networks in the morphogenesis of complex adaptive urban systems that are human territories

\bigskip

$\rightarrow$ \textit{investigation of stationarity and ergodicity properties of relation between road network and population distribution ; implies spatiality of correlations and link static-dynamic}
% because means link between dynamic and static ; and also spatiality of correlations




Dataset construction
% brief description of nw simplification ; 

Computation of topological road network for all Europe, at 100m granularity scale (to be used consistently with population grid~\cite{eurostat})

\medskip

$\rightarrow$ Import of OSM into \texttt{pgsql}, simplification at 100m granularity, topological simplification with split/merge algorithm

\bigskip
%
%\begin{columns}
%\begin{column}{width=0.7\textwidth}
%
%\end{column}
%\begin{column}{width=0.3\textwidth}
%\textit{.} % TODO summary stats here.
%\end{column}
%\end{columns}

%\begin{columns}
%\begin{column}{0.7\textwidth}
    %\includegraphics[width=\textwidth,height=0.5\textheight]{figures/ex_nw}
%\end{column}
%\begin{column}{0.3\textwidth}
   \textit{$\simeq 44\cdot 10^6$ links in initial OSM db, $\simeq 61\cdot 10^6$ in first simplified layer, $\simeq 21\cdot 10^6$ in final database}
%\end{column}
%\end{columns}







%%%%%%%%%%%%%%%%%
Results : Computation of Indicators


\textit{Computation of urban form indicators~\cite{le2015forme} and network indicators on $l_0=10km$ side square}

%\includegraphics[width=\textwidth]{figures/pop-alphaBetweenness}




%%%%%%%%%%%%%%%%%
Results : Spatial Correlations

\textit{Computation of spatial correlation on square areas of width $\delta\cdot l_0$ (with typically $\delta = 4, \ldots , 16$)}


%\includegraphics[width=\textwidth,height=0.7\textheight]{figures/corr_PCA_delta4}

$\rightarrow$ \textit{local spatial stationarity of processes}



%%%%%%%%%%%%%%%%%
%\sframe{Results : Stationarity scales}{
% maps
% 
%}



%%%%%%%%%%%%%%%%%
Results : Multi-scale Processes
% plots
%\includegraphics[width=0.33\textwidth]{figures/corrs-distrib_varyingdelta_bytype} % -> in supp material
%\includegraphics[width=0.5\textwidth]{figures/corrs-summary_varyingdelta_bytype_extended1}
%\includegraphics[width=0.5\textwidth]{figures/normalized_CI_delta}

\medskip

$\rightarrow$ Significant variation of mean correlation with $\delta$ (Left) and of normalized confidence interval (Right) given by $\left|\rho_+ - \rho_-\right|\cdot \delta$, as bounds theoretically vary as $\sqrt{N} \sim \sqrt{\delta^2}$ : implies multi-scalarity






Empirical Findings (Formalization)

$Y_i\left[\vec{x},t\right]$ spatio-temporal stochastic process, verifies empirically :

\bigskip

\begin{enumerate}
\item Local spatial autocorrelation is present and bounded by $l_{\rho}$ (in other words the processes are continuous in space) : at any $\vec{x}$ and $t$, $\left|\rho_{\norm{\Delta \vec{x}} < l_{\rho}}\left[Y_i (\vec{x}+\Delta \vec{x},t), Y_i (\vec{x},t) \right]\right| > 0$.
\medskip
\item Processes are locally parametrized : $Y_i = Y_i\left[\alpha_i\right]$, where $\alpha_i (\vec{x})$ varies with $l_{\alpha}$, with $l_{\alpha} \gg l_{\rho}$ and weakly locally stationary in space.
\medskip
%\item Spatial correlations between processes have a sense at an intermediate scale $l$ such that $l_{\alpha}\gg l \gg l_{\rho}$.
%\item Processes covariance stationarity times scale as $\sqrt{l}$.
%\item Local ergodicity is present at scale $l$ and dynamics are locally chaotic.
\item Processes are multi-scalar : since $\rho(\delta = \infty) > \rho (\delta = 0 )$, a necessary non-linear correction on processes spatial averages in correlation computation is present.
% add computation in supplementary materials / papers. -> later
\end{enumerate}



Analytical Deductions

1. \textbf{Regimes of temporal correlations.} Let assume local ergodicity in $\vec{x}_0$ at scale $\delta \cdot l_0$ (reasonable with urban growth and network extension in recent times). The Ergodic theorem implies that $\exists \mathcal{T}$ such that

\[<Y_i (t) >_{\norm{\vec{x}-\vec{x}_0} < \delta\cdot l_0} = <Y_i (\vec{x}_0)>_{t\in \mathcal{T}}\] 

With spatial stationarity, $<Y_i>_{\vec{x}_0}=<Y_i>_{\vec{x}_1}$, thus $\mathcal{T}$ must be constant to be invariant by translation. By contraposition and (2), processes have different dynamical characteristics.
% if translate in a given direction, looses a small part, must be compensated by the area translated by delta (overlap), thus must be constant.

\bigskip

2. \textbf{Global non-ergodicity.} Let $X_k$ a partition of space into local areas. We have $<\cdot>_x = \sum_k w_k <\cdot>_{x_k} =_{(1)} \sum_k w_k <\cdot>_{\mathcal{T}_k} $. On the other hand, global ergodicity would give $<\cdot>_t = <\cdot>_{\mathcal{T}} = \sum_k w_k <\cdot>_{\mathcal{T}}$ and $\sum_k w_k \left(<\cdot>_{\mathcal{T}} - <\cdot>_{\mathcal{T}_k}\right) = 0$. Being true on each subset implies $\mathcal{T}=\mathcal{T}_k$, what contradicts (1).





%
%
%%%%%%%%%%%%%%%%%%
%\sframe{Stationarity and Ergodicity}{
%
%\begin{itemize}
%\item Assuming local ergodicity, spatial local stationarity implies and temporal local stationarity
%
%\item Spatial non-stationarity \textbf{at the second order}$\implies$ temporal scale variations $\implies$ non-ergodicity
%
%\end{itemize}
%
%}
%




Case study : implications
% thematic conclusion of the case study on ergodicity

$\rightarrow$ Still points to explore :
\begin{itemize}
\item variable correlations areas (size and shape in space)
\item same work on cities population/train network data, which are also dynamical databases : extrapolation of ergodicity parameters ?
\item correlations of returns : link between $\rho\left[\Delta_t Y\right]$ and $\rho\left[\Delta_x Y\right]$ (more difficult : if pure local ergodicity, $\exists$ a permutation making the correspondance) % may be difficult to identify 
\item Link between $\Delta_{\delta}\rho (\delta)$ and process derivatives ?
\end{itemize}

\bigskip

$\rightarrow$ We show the regional nature of network-territories interactions, in particular the non-ergodicity of urban systems on \textbf{the interaction these components}

\bigskip

$\rightarrow$ No direct results on time dynamics, but indirect : spatio-temporal processes do not have same speed and react/diffuse differently









%%%%%%%%%%%%%%%%%%
\subsection{Application to China}{Application à la Chine}


\bpar{
Although \cite{zheng2014assessing} underlined a quick acceleration of OpenStreetMap road data completeness and accuracy, its use for computation of network indicators may be questioned. \cite{zhang2015density} highlights four regimes of data quality, partitioning China into regions among which qualitative behavior of OSM data varies. We choose in the following to work only on the regions where the quality is the highest, i.e. with high density and high diversity. % TODO precise which areas
For density data, we use the gridded population data from~\cite{fu1km}. % TODO check also that
}{

}







%%%%%%%%%%%%%%%%%%
%\subsection{Insights for interaction processes}













%----------------------------------------------------------------------------------------


\newpage

\section[Disentangling co-evolutions from causal relations]{Disentangling co-evolutions from causal relations : a case study on \emph{Bassin Parisien}}{Isoler la Co-évolution des Relations causales}

Spatial statistics studies on dynamical relations between network and territories are relatively rare. \cite{levinson2008density} does so on London metropolitan area and identifies causalities using lagged variables, but does not disentangle relations in the sense of coupled statistical models that would isolate endogenous effects from coupling effects.

 % study on london with temporal and spatial lag (weird use of spatial statistics) -> expected conclusions but does not really disentangle ?



\subsection{Context Formalization}{Formalisation}

%\subsubsection{Variables}

%\paragraph{Description}

We assume a dynamic transportation network $n(\vec{x},t)$ within a dynamic territorial landscape $\vec{T}(\vec{x},t)$, which components are to simplify population $p(\vec{x},t)$ and employments $e(\vec{x},t)$. Data is structured the following way :
\begin{itemize}
\item Observation of territorial variables are discretized in space and in time, i.e. the spatial field $\vec{T}$ is summarized by $\mathbf{T} = \left(\vec{T}(\vec{x}_i,t_j^{(T)})\right)_{i,j}$ with $1\leq i \leq N$ and $1\leq j \leq T$. They concretely correspond to census on administrative units (\emph{communes} in our case) at different dates.
\item Network has a continuous spatial position but is represented by the vector of network distances $\mathbf{N}$ \comment{(Florent) vol d'oiseau/distance temps ? second faisable et à privilégier je pense}
\end{itemize}



%\paragraph{Definitions}



\subsection{On Accessibility}{Sur l'accessibilité}

% accessibility : need to introduce it ?
%  -> read Weibull

The notion of accessibility has been central to regional science since its introduction and systematization in planning around 1970. 

%\paragraph{Existence of accessibility}

%An elegant axiomatic definition is derived in~\cite{weibull1976axiomatic}. Starting from expected properties of an accessibility function $A$ that associate a value to \emph{attraction} $a$ and distance $d$, defined on the set of discrete spatial configurations $\mathcal{C} = \cup_{n\in \mathbb{N}}{(d_i,a_i)_{1\leq i \leq n}}$. These properties include (among technical others with no thematic meaning) :
%\begin{enumerate}
%\item $A$ is invariant regarding the order of the configuration
%\item $A$ decrease with distance at fixed attraction and increase with attraction at fixed distance
%\item $A$ is invariant when adding null attractions and constant configurations
%\end{enumerate}

%A canonical decomposition of any accessibility function 


As already introduced in the first chapter, we question the notion of accessibility : \textit{Is the notion of accessibility crucial for statistical analysis ?}

\medskip


Weibull has proposed an axiomatic approach to accessibility~\cite{weibull1976axiomatic}, deriving a canonical decomposition for any \emph{attraction-accessibility} function $A(a,d)$, assuming expected thematic axioms among others technical ones that are :
\begin{enumerate}
\item $A$ is invariant regarding the order of the configuration
\item $A$ decrease with distance at fixed attraction and increase with attraction at fixed distance
\item $A$ is invariant when adding null attractions and constant configurations
\end{enumerate}
Then $A$ verifies these if and only if it is of the form

\[
A\left[(a_i,d_i)\right] = T\left(\bigoplus_i z(d_i,a_i)\right)
\]

where $T$ is increasing with null origin, $z$ is a \emph{distance substitution function} (i.e. verifying axiom 2) and $\oplus$ a \emph{standard composition} associating two attractions at zero distance to th corresponding unique one. 

It means that well suited matrices of autocorrelation should capture accessibility in regressions ; \comment{(Florent)pas sur de comprendre, à discuter}
 or it must be captured by non-linear regression on $\mathbf{N}$. It may reveal some kind of intrinsic accessibility that is related to real phenomena (that we expect to fit with calibrated functions of accessibility based on Hedonic models e.g.) Seeing accessibility as a potential field is an equivalent vision : given any stationary dynamic for $n,\vec{T}$, Helmoltz theorem states that it derives from a potential (can be adapted to non-stationary dynamics with a time-varying potential).

%\paragraph{Continuous approach and accessibility potential}

% Paul : Helmoltz-Hodge theorem to infer potential field from speed spatial field ?
%  Q : what are trajectories ? dirac field has no rotational -> continuous approach does not work ?


\subsection{Data}{Données}

We will work on a novel dataset provided by \noun{Le Nechet}, that consists in main road infrastructures with their opening dates and train network for network dynamics, and in population and employments of communes at census dates, for Bassin Parisien on the last fifty year. The temporal granularity due to census temporal step may be an obstacle to obtain good dynamical statistics. \comment{(Florent) enfin c'est surtout INSEE, IGN, et Wiki[?] qu'il faut citer (c'est vrai qu'il y a du formatage, mais en tout cas il faut citer les sources de première main)}


\subsection{Statistical Tests}{Tests Statistiques}


The following large set of analysis are to be tested (non exhaustive) :

\comment{(Florent) interprétation ? si O/N}

\begin{itemize}
\item On raw data :
\begin{itemize}
\item Multivariate models
\[\mathcal{L}\left[\mathbf{T},\mathbf{N}\right]\sim \varepsilon\]
\item Autocorrelated univariate models
\[(\mathbf{I} - \Sigma R W) \mathbf{X} \sim \varepsilon\]
\item Autocorrelated multivariate models \[(\mathcal{L}' - \Sigma R W)\left[\mathbf{T}+\mathbf{N}\right] \sim \varepsilon\]
\item Geographically Weighted Regression~\cite{brunsdon1998geographically}
\[
\mathcal{L}\left[\mathcal{G}\left(\mathbf{T},\mathbf{N}\right)\right] \sim \varepsilon
\]
\item Granger causality tests : \cite{xie2009streetcars} use for example Granger causality to link transit with land-use changes.
\end{itemize}
\item On data returns :
\begin{itemize}
\item Autoregressive multivariate models
\[\mathcal{L}\left[(\Delta \mathbf{T}(t_{j'}))_{j'\leq j},(\Delta \mathbf{N}(t_{j'}))_{j'\leq j}\right] \sim \varepsilon\]
\item Autoregressive autocorrelated multivariate models : idem with spatial autocorrelation term.
\item Synthetic Instrumental Variables : static territory and/or network ?
\end{itemize}
\end{itemize}



%\subsubsection{Bivariate linear models}

%\subsubsection{Autocorrelated univariate models}

%\subsubsection{Autocorrelated multivariate models}

%\subsubsection{Granger causality tests}

%\cite{xie2009streetcars} use Granger causality to link transit with land-use changes.

%\subsubsection{Autoregressive multivariate models}

%\subsubsection{Autoregressive autocorrelated multivariate models}


\bigskip

%\subsection{Expected results}

%We expect from these analyses to test at these spatial and temporal scales, and on a particular metropolitan case study, the assumption on network necessity for the territorial system of functional job commutings.




%%%%%%%%%%%%%%%%%%
\subsection{Generic Method}{Méthode Générique}

% TODO introduce here systematic granger causality testing - first tests on RBD ?

% TODO : Spatial Statistics / causalities ?


\subsubsection{Description}{Description}


\bpar{
We describe here a generic method, based on a test similar to Granger causality test~\cite{}, aiming to identify causal relations in spatial systems. Let $X_j(\vec{x},t)$ unidimensional spatial random processes. A realization of a territorial subsystem is given by set of trajectories for each process $x_{i,j,t}$. We assume the existence of correspondance functions $\Phi_{j1,j2}$ that associate realizations of each components to a unique index (in the simplest case, variables on the same patch will be associated). If $\textrm{argmax}_{\tau} \hat{\rho}\left[x_{j1},x_{j2}\right]$ is clearly defined, its sign will then give the sense of the causality between components $j1$ and $j2$.
}{
Nous décrivons ici une méthode générique, basée sur un test similaire à la causalité de Granger~\cite{}, pour tenter d'identifier des relations causales dans des systèmes spatiaux. Soit $X_j(\vec{x},t)$ des processus aléatoires spatiaux unidimensionnels. Une réalisation d'un sous-système territorial est donnée par des ensembles de trajectoires pour chaque processus $x_{i,j,t}$. On suppose l'existence de fonctions de correspondance $\Phi_{j1,j2}$ permettant de faire correspondre les réalisations de chaque composantes à un index unique (dans le cas le plus simple, on associera les variables sur les mêmes patches). Si $\textrm{argmax}_{\tau} \hat{\rho}\left[x_{j1},x_{j2}\right]$ est clairement défini % TODO notion trop floue ? le problème est que des modèles stat conditionnent trop ?
, son signe donnera alors le sens de la causalité entre les composantes $j1$ et $j2$.
}

% TODO some kind of smoothing has to be introduced, either as preprocessing or as part of the optimization process (at least for first observed behavior on synthetic data. maybe it is typical of the model ?)
% - formalize mean estimator on repetitions, compare it to a direct estimator (// computation or aggregated data ?)


\cite{luo2013spatio} : Spatio-temporal Granger causality for fMRI data ; not really spatial in the sense that no real distance effect.

\cite{liu2011discovering} : with a specific kind of data (traffic data in this case), a suited definition of causality yield specialized algorithms (difficultly reusable).



\subsubsection{Synthetic Data}{Données Synthétiques}


\paragraph{Case study}{Etude de cas}


\bpar{
This method must in a first time be tested and partially validated, what we propose to do on synthetic data, an approach which use is documented and illustrated in chapter~\ref{}. \cite{raimbault2014hybrid} is a simple model of urban morphogenesis (RBD model) being an interesting candidate for our test. Indeed, variables explaining urban growth, the network extension process and the coupling between urban density and network are rather elementary. However, outside extreme cases (distance to the center determining solely real estate value, the network will depend in a causal way on density, or distance to network alone, the causality should be reversed), the mixed regimes do not exhibit any obvious causality: it is then a perfect case to test if the method is able to detect some.
}{
Cette méthode doit dans un premier temps être testée et partiellement validée, ce que nous proposons de faire sur des données synthétiques, approche dont l'utilisation est documentée et illustrée au chapitre~\ref{}. \cite{raimbault2014hybrid} est un modèle simple de morphogenèse urbaine (modèle RBD) faisant un candidat intéressant pour notre test. En effet, les variables explicatives de la croissance urbaine, les processus d'extension du réseau et le couplage entre densité urbaine et réseau sont assez élémentaires. Cependant, hormis dans des cas extrêmes (distance au centre détermine valeur foncière uniquement, le réseau dépendra de manière causale de la densité, ou distance au réseau seule, la causalité devrait être inversée), les régimes mixtes n'exhibent pas de causalités évidentes : c'est donc un parfait cas pour tester si la méthode est capable d'en détecter.
}

\paragraph{Application}{Application}

Nous explorons une grille de l'espace des paramètres du modèle RBD. Pour chaque valeur des paramètres, nous procédons à $N=100$ répétitions. % Le modèle a par ailleurs été exploré de nouveau pour reproduction et extension des résultats. % TODO appendice with full RBD exploration.
Nous calculons sur l'ensemble des patches les corrélations retardées entre les variables suivantes : densité locale, distance au centre et distance au réseau. Une grille de pas 0.5 (26 combinaisons) est entièrement explorée en Appendice~\ref{}. Des regroupements par régimes peuvent se faire, lorsqu'un paramètre domine et impose le régime (par exemple )


%----------------------------------------------------------------------------------------

\newpage

\section[Real Estate Trajectories]{Early warnings of Network Breakdowns : socio-economic and real estate trajectories}{Trajectoires de Marchés Immobiliers}


\subsection{Context}{Contexte}



\bpar{
Various aspects of territories are concerned by interactions with networks. In previous empirical studies, no socio-economic attributes of populations inhabiting the territory nor economic values for land and real estate was considered. Both are however crucial elements of territorial dynamics and are extensively studied in fields such as territorial analysis or urban economics : for example, \cite{homocianu:tel-00359302} studies households residential choices to understand land-use transportation interactions. We propose here to use a database of Real Estate transactions for Parisian region on the last 20 years, with 2 years temporal granularity and exact spatial coordinates. \cite{guerois2009dynamique} used it to make typologies of spatial dynamics of Parisian real estate. This project is conjointly done with \noun{Le Corre} whose strong thematic knowledge on Real Estate financial properties will bring insight for spatial typologies of temporal trajectories.% results using base bien.
}{
Des aspects très variés des territoires sont concernés par l'interaction avec les réseaux. Dans nos études précédentes, aucun aspect socio-économique des populations habitant le territoire ni des valeurs économiques pour le foncier et l'immobilier n'ont été considérés. \comment{(Florent) les emplois tout de même}
 Il s'agit cependant d'éléments cruciaux des dynamiques territoriales et sont étudiés de manière intensive dans des champs comme l'analyse territoriale ou l'économie urbaine : par exemple, \cite{homocianu:tel-00359302} étudie les choix résidentiels des ménages pour comprendre les interactions entre usage du sol et transport. Nous proposons \comment{(Florent) propose : faux ami ?}
 ici d'utiliser une base de données de transactions immobilières pour la région parisienne sur les 20 dernières années, avec une granularité temporelle de 2 ans et coordonnées spatiales exactes. \cite{guerois2009dynamique} l'utilise pour établir une typologie des dynamiques spatiales du marché immobilier parisien. \comment{(Florent) strong thematic knowledge : quest ce que cela veut dire, tu ouvres beaucoup de portes, on a envie de savoir ce qu'il y a derrière}
}


\subsection{Preliminary Results}{Résultats Préliminaire}

\comment{(Florent) economic ou financial ?}

We show in Fig.~\ref{fig:realestate} typologies of temporal transactional profiles for total stocks. Temporal dynamics show different reactions of local territories to the 2008 crisis, in particular a strong differentiation between urban and rural areas. More precise classification into urban territories are still to be investigated when the analysis will be pushed further.



%%%%%%%%%%%%%%%
\begin{figure}
\hspace{-3cm}\includegraphics[width=1.4\textwidth]{Figures/RealEstate/normalized_k2-10}
\smallskip
\hspace{-3cm}\includegraphics[width=1.4\textwidth]{Figures/RealEstate/trajectories_normalized_k=2-10}
\caption[Typology of Real Estate trajectories]{Typology of Real Estate trajectories. Locations were categorized using averaged k-means on time-series. We show maps and time series for value of k from 2 to 10.}{\comment{(Florent) ces graphes ne sont pas appropriables par le lecteur}}
\label{fig:realestate}
\end{figure}
%%%%%%%%%%%%%%%



\subsection{A strategy to investigate early warnings of network breakdowns}{Une Stratégie pour étudier les signes précurseurs de rupture de potentiels}


The span of the end of this database coincides with planification phases of the Grand Paris Express that we already mentioned. \comment{(Florent) que signifie cette phrase ?}
 We aim to seek for early warnings of potential station implantation, in correspondance with different stages of the project, in order to verify if intrinsic territorial dynamics were already present or if the announcement of a new station induced a local phase transition.





%----------------------------------------------------------------------------------------

\newpage

\section[South-African historical events as instruments]{South-African historical events as instruments to understand network-territory relations}{Relations Réseaux-territoires en Afrique du Sud}

\subsection{Context}{Contexte}


\noun{Baffi} studied in her thesis project~\cite{baffi2016thesis} qualitatively the role of South African railways in segregations and integration processes, aims to use an extensive database of railway growth and population dynamics in cities on the last 100 years produced during the thesis. In particular, she showed qualitatively that dynamics between territories and networks profoundly changed at the end of the apartheid, transforming a tool of sordid \comment{(Florent) pas un terme scientifique} planned segregation (network shaped was optimized to minimize unwanted accessibility) into an integration tool thanks to recent changes in network topology patterns.

\subsection{Objectives}{Objectifs}

We can use first the particular shape of that network to control on local and global topology effects (but this is quite equivalent as controlling on accessibility), and in a second time the historical events as statistic instruments, assuming that territorial dynamics and network dynamics responded differently to these. We expect to learn from these project informations on interactions at long time scale and large spatial scale, in a very particular context of constrained growth. \comment{(Florent) à discuter}






\subsection{Possible developments}{Développements possibles}


The method of instruments in statistics~\cite{angrist1996identification} is used to identify causal relationships between variables, in a different way than Granger causality test for example. Trying to identify causalities between network dynamics and territorial dynamics is of crucial importance to test our theoretical assumption on the existence of co-evolution.








