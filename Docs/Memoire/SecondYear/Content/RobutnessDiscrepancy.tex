
% Chapter 

\chapter{A Discrepancy-based Framework to Compare Robustness between Multi-Attribute Evaluations} % Chapter title

\label{ch:robustness}

% NOTE : wont stay a chapter 

%----------------------------------------------------------------------------------------



\bpar{
Multi-objective evaluation is a necessary aspect when managing complex systems, as the intrinsic complexity of a system is generally closely linked to the potential number of optimization objectives. However, an evaluation makes no sense without its robustness being given (in the sense of its reliability). Statistical robustness computation methods are highly dependent of underlying statistical models. We propose a formulation of a model-independent framework in the case of integrated aggregated indicators (multi-attribute evaluation), that allows to define a relative measure of robustness taking into account data structure and indicator values. We implement and apply it to a synthetic case of urban systems based on Paris districts geography, and to real data for evaluation of income segregation for Greater Paris metropolitan area. First numerical results show the potentialities of this new method. Furthermore, its relative independence to system type and model may position it as an alternative to classical statistical robustness methods.
}{
Les évaluations multi-objectifs sont un aspect essentiel de la gestion de systèmes complexes, puisque la complexité intrinsèque d'un système est généralement étroitement liée au nombre d'objectifs d'optimisation potentiels. Cependant, une évaluation ne fait pas sens si sa robustesse, au sens de sa fiabilité, n'est pas donnée. Les méthodes statistiques usuelles fournissant une mesure de robustesse sont très dépendantes des modèles sous-jacents. Nous proposons une formulation d'un cadre indépendant du modèle, dans le cas d'indicateurs intégrés et agrégés (évaluation multi-attributs), qui permet de définir une mesure de robustesse relative prenant en compte la structure des données et les valeurs des indicateurs. La méthode est testée sur données urbaines synthétiques associées aux arrondissements de Paris, et à des données réelles de revenus pour l'évaluation de la ségrégation urbaine dans la région métropolitaine du Grand Paris. Les premiers résultats numériques montrent les potentialités de cette nouvelle méthode. De plus, sa relative indépendance au type de système et au modèle pourrait la positionner comme une alternative aux méthodes statistiques classiques d'évaluation de la robustesse.
}


%%%%%%%%%%%%%%%%
%% Intro
%%%%%%%%%%%%%%%%
\section{Introduction}{Introduction}

%%%%%%%%%%%%%%%%
\subsection{General Context}


\bpar{
Multi-objective problems are organically linked to the complexity of underlying systems. Indeed, either in the field of \emph{Complex Industrial Systems}, in the sense of engineered systems, where construction of Systems of Systems (SoS) by coupling and integration often leads to contradictory objectives~\cite{marler2004survey}, or in the field of \emph{Natural Complex Systems}, in the sense of non engineered physical, biological or social systems that exhibit emergence and self-organization properties, where objectives can e.g. be the result of heterogeneous interacting agents (see~\cite{newman2011complex} for a large survey of systems concerned by this approach), multi-objective optimization can be explicitly introduced to study or design the system but is often already implicitly ruling the internal mechanisms of the system. The case of socio-technical Complex Systems is particularly interesting as, following~\cite{haken2003face}, they can be seen as hybrid systems embedding social agents into ``technical artifacts'' (sometimes to an unexpected degree creating what \noun{Picon} describes as \emph{cyborgs}~\cite{picon2013smart}), and thus cumulate propensity to be at the origin of multi-objective issues\footnote{We design by \emph{Multi-Objective Evaluation} all practices including the computation of multiple indicators of a system (it can be multi-objective optimization for system design, multi-objective evaluation of an existing system, multi-attribute evaluation ; our particular framework corresponds to the last case).}. The new notion of \emph{eco-districts}~\cite{souami2012ecoquartiers} is a typical example where sustainability implies contradictory objectives. The example of transportation systems, which conception shifted during the second half of the 20th century from cost-benefit analysis to multi-criteria decision-making, is also typical of such systems~\cite{bavoux2005geographie}. Geographical system are now well studied from such a point of view in particular thanks to the integration of multi-objective frameworks within Geographical Information Systems~\cite{carver1991integrating}. As for the micro-case of eco-districts, meso and macro urban planning and design may be made sustainable through indicators evaluation~\cite{jegou2012evaluation}.
}{
Les problèmes multi-objectifs sont organiquement liés à la complexité des systèmes sous-jacents. En effet, que ce soit dans le champ des \emph{Systèmes Complexes Industriels}, dans le sens de systèmes conçus par ingénierie, où la construction de Systèmes de Systèmes (SoS) par couplage et intégration induit souvent des objectifs contradictoires~\cite{marler2004survey}, ou dans le champ des \emph{Systèmes Complexes Naturels}, au sens de systèmes non désignés, physiques, biologiques ou sociaux, qui présentent des propriétés d'émergence et d'auto-organisation, pour lesquels les objectifs peuvent e.g. être le résultat de l'interaction d'agents hétérogènes (voir~\cite{newman2011complex} pour une revue étendue des types de systèmes concernés par cette approche), l'optimisation multi-objectifs peut être explicitement introduite pour étudier ou désigner le système, mais régit généralement déjà implicitement les mécanismes internes du système. Le cas des Systèmes Complexes Sociaux-techniques est particulièrement intéressant puisque selon Haken~\cite{haken2003face}, ils peuvent être vus comme des systèmes hybrides embarquant des agents sociaux dans des ``artefacts techniques'' (parfois jusqu'à un niveau inattendu, créant ce que \noun{Picon} décrit comme \emph{cyborgs}~\cite{picon2013smart}), et cumulent ainsi la potentialité d'être à l'origine de problèmes multi-objectifs\footnote{Nous désignons ici par \emph{Evaluation Multi-objectifs} toutes les pratiques incluant le calcul de multiples indicateurs d'un système (il peut s'agir d'optimisation multi-objectif pour un design de système, une évaluation multi-objectif d'un système existant, une évaluation multi-attributs ; notre cadre particulier correspondra au dernier cas).}. La notion récente d'\emph{éco-quartier}~\cite{souami2012ecoquartiers} est un exemple typique pour lequel la durabilité implique des objectifs contradictoires. L'exemple des systèmes de transport, dont la conception a glissé durant la seconde moitié du 20ème siècle d'analyses coût-bénéfices à la price de décision multi-critères, est également typique de tels systèmes~\cite{bavoux2005geographie}. Les systèmes géographiques sont à présent bien étudiés d'un tel point de vue, en particulier grâce à l'intégration des cadres multi-objectifs au sein des Systèmes d'Information Géographiques~\cite{carver1991integrating}. Comme dans le cas microscopique des éco-quartiers, la planification et le design urbains mésoscopiques et macroscopiques peuvent être rendus durables grâce aux évaluations par indicateurs~\cite{jegou2012evaluation}.
}



\bpar{
A crucial aspect of an evaluation is a certain notion of its reliability, that we call here \emph{robustness}. % Various definitions of robustness are possible in different frames, and it will have a precise definition in our framework.
Statistics naturally include this notion since the construction and estimation of statistical models give diverse indicators of the consistence of results~\cite{launer2014robustness}. The first example that comes to mind is the application of the law of large numbers to obtain the \emph{p-value} of a model fit, that can be interpreted as a confidence measure of estimates. Besides, confidence intervals and \emph{beta-power} are other important indicators of statistical robustness. Bayesian inference provide also measures of robustness when distribution of parameters are sequentially estimated. Concerning multi-objective optimization, in particular through heuristic algorithms (for example genetic algorithms, or operational research solvers), the notion of robustness of a solution concerns more the stability of the solution on the phase space of the corresponding dynamical system. Recent progresses have been done towards unified formulation of robustness for a multi-objective optimization problem, such as~\cite{deb2006introducing} where robust Pareto-front as defined as solutions that are insensitive to small perturbations. In~\cite{1688537}, the notion of degree of robustness is introduced, formalized as a sort of continuity of other solutions in successive neighborhood of a solution.
}{
Un aspect crucial de l'évaluation est une certaine notion de sa fiabilité, que nous nommerons ici \emph{robustesse}. Les méthodes statistiques incluent naturellement cette notion puisque la construction et l'estimation de modèles statistiques donne divers indicateurs de la consistence des résultats~\cite{launer2014robustness}. Le premier exemple venant à l'esprit est l'application de la loi des grands nombres pour obtenir la \emph{p-valeur} d'une estimation de modèle, qui peut être interprété comme une mesure de confiance en les valeurs estimées. D'autre part, les intervalles de confiance et le  \emph{beta-power} sont d'autres indicateurs importants de robustesse statistique. L'inférence bayésienne fournit également des mesures de robustesse quand la distribution des paramètres est estimée de manière séquentielle. Concernant les optimisations multi-objectifs, en particulier par des algorithmes heuristiques (comme par exemple les algorithmes génétiques, ou les solveurs de recherche opérationelle), la notion de robustesse d'une solution consiste plus en la stabilité de la solution dans l'espace des phases du système dynamique correspondant. Des progrès récents ont été faits vers une formulation unifiée de la robustesse pour les problèmes d'optimisation multi-objectifs, comme dans~\cite{deb2006introducing} où les fronts de Pareto robustes sont définis comme des solutions insensibles aux petites perturbations. Dans~\cite{1688537}, la notion de degré de robustesse est introduite, formalisée comme une sorte de continuité des autres solutions dans des voisinages successifs d'une solution.
}


\bpar{
However, there still lack generic methods to estimate robustness of an evaluation that would be model-independent, i.e. that would be extracted from data structure and indicators but that would not depend on the method used. Some advantages could be for example an \emph{a priori} estimation of potential robustness of an evaluation and thus to decide if the evaluation is worth doing. We propose here a framework answering this issue in the particular case of Multi-attribute evaluations, i.e. when the problem is made unidimensional by objectives aggregation. It is data-driven and not model-driven in the sense that robustness estimation does not depend on how indicators are computed, as soon as they respect some assumptions that will be detailed in the following.
}{
Cependant, il n'existe pas de méthode générique qui permettrait une évaluation de la robustesse de façon indépendante au modèle, i.e. qui serait extraite de la structure des données et des indicateurs mais ne dépendrait pas de la méthode utilisée. Un avantage serait par exemple une estimation \emph{a priori} de la robustesse potentielle d'une évaluation et de décider ainsi si elle vaut la peine d'être faite. Nous proposons un cadre répondant à cette contrainte dans le cas particulier des évaluations multi-attributs, i.e. quand le problème est rendu unidimensionnel par agrégation des objectifs. Il est basé sur les données et non sur les modèles, au sens ou l'estimation de la robustesse ne dépendra pas de la manière dont les indicateurs sont calculés, tant qu'ils respectent certaines hypothèses détaillées par la suite.
}



%%%%%%%%%%%%%%%%
\subsection{Proposed Approach}



\paragraph{Objectives as Spatial Integrals}


\bpar{
We assume that objectives can be expressed as spatial integrals, so it should apply to any territorial system and our application cases are urban systems. It is not that restrictive in terms of possible indicators if one uses suitable variables and integrated kernels : in a way analog to the method of geographically weighted regression~\cite{brunsdon1998geographically}, any spatial variable can be integrated against regular kernels of variable size and the result will be a spatial aggregation which sense depends on kernel size. The example we use in the following such as conditional means or sums suit well the assumption. Even an already spatially aggregated indicator can be interpreted as a spatial indicator by using a Dirac distribution on the centroid of the corresponding area.
}{
Nous supposons que les objectifs peuvent être exprimés comme intégrales spatiales, ce qui devrait s'appliquer à tout système territorial, et nos cas d'application sont des systèmes urbains. Ce n'est pas si restrictif en terme d'indicateurs possibles si l'on utilise les bonnes variables et noyaux intégrés : de façon analogue à la méthode de Regression Géographique Pondérée~\cite{brunsdon1998geographically}, toute variable spatiale peut être intégrée contre des noyaux réguliers de taille variable et le résultats sera une agrégation spatiale dont la signification dépendra de l'étendue du noyau. Les exemples utilisés par la suite comme des moyennes conditionnelles ou des sommes vérifient parfaitement cette hypothèse. Même un indicateur déjà agrégé dans l'espace peut être interprété comme une intégrale spatiale en utilisant une distribution de Dirac au centroïde de la zone correspondante. 
}



\paragraph{Linearly Aggregated Objectives}

\bpar{
A second assumption we make is that the multi-objective evaluation is done through linear aggregation of objectives, i.e. that we are tackling a multi-attribute optimization problem. If $(q_i(\vec{x}))_i$ are values of objectives functions, then weights $(w_i)_i$ are defined in order to build the aggregated decision-making function $q(\vec{x})=\sum_i{w_i q_i(\vec{x})}$, which value determines then the performance of the solution. It is analog to aggregated utility techniques in economics and is used in many fields. The subtlety lies in the choice of weights, i.e. the shape of the projection function, and various approaches have been developed to find weights depending on the nature of the problem. Recent work~\cite{dobbie2013robustness} proposed to compare robustness of different aggregation techniques through sensitivity analysis, performed by Monte-Carlo simulations on synthetic data. Distribution of biases where obtained for various techniques and some showed to perform significantly better than others. Robustness assessment still depended on models used in that work.
}{
Une seconde hypothèse que nous faisons est que l'évaluation multi-objectifs est effectuée par agrégation linéaire des objectif, c'est à dire qu'on se place dans le cadre d'un problème d'optimisation multi-attributs. Si $(q_i(\vec{x}))_i$ sont les valeurs des fonctions objectifs, on définit alors des poids $(w_i)_i$ afin de construire la fonction de prise de décision $q(\vec{x})=\sum_i{w_i q_i(\vec{x})}$, dont la valeur détermine ensuite la performance d'une solution. Cette approche est analogue aux utilités agrégées en économie et est utilisée dans de nombreux domaines. La subtilité réside dans le choix des poids, i.e. de la forme de la fonction de projection, et différentes solutions ont été dévelopées pour obtenir des poids selon la nature du problème. Récemment, \cite{dobbie2013robustness} a proposé de comparer la robustesse des différentes techniques d'agrégation par une analyse de sensibilité, effectuée par simulations de Monte-Carlo pour produire des données synthétiques, ce qui permet d'obtenir la distribution des biais pour les différentes techniques, certaines étant significativement plus performantes que d'autres. Toutefois, la quantification de la robustesse dépend toujours des modèles utilisés dans ce travail.
}


\bigskip


\bpar{
The rest of the paper is organized as follows : section 2 describes intuitively and mathematically the proposed framework ; section 3 then details implementation, data collection for case studies and numerical results for an artificial intra-urban case and a metropolitan real case ; section 4 finally discuss limitations and potentialities of the method.
}{
Le reste de cette monographie est organisé de la façon suivante : la section 2 décrit intuitivement puis mathématiquement le cadre proposé ; la section 3 détaille ensuite l'implémentation, la collecte des données pour les cas d'étude et les résultats numériques pour une évaluation intra-urbaine synthétique et un cas réel métropolitain ; la section 4 discute finalement les limitations et les potentialités de la méthode.
}




%%%%%%%%%%%%%%%%
%% Framework Description
%%%%%%%%%%%%%%%%
\section{Framework Description}{Description du Cadre}


%%%%%%%%%%%%%%%%
\subsection{Intuitive Description}


\bpar{
We describe now the abstract framework allowing theoretically to compare robustnesses of evaluations of two different urban systems. Our framework is a generalization of an empirical method proposed in~\cite{ecodistrictReport} besides a more general benchmarking study on indicator sense and relevance in a sustainability context. Intuitively, it relies on empirical base resulting from the following axioms :
\begin{itemize}
\item Urban systems can be seen from the information available, i.e. raw data describing the system. As a data-driven approach, this raw data is the basis of our framework and robustness will be determined by its structure.
\item From data are computed indicators (objective functions). We assume that a choice of indicators is an intention to translate particular aspects of the system, i.e. to capture a realization of an ``urban fact'' (\emph{fait urbain}) in the sense of \noun{Mangin}~\cite{mangin1999projet} - a sort of stylized fact in terms of processes and mechanisms, having various realizations on spatially distinct systems, depending on each precise context.
\item Given many systems and associated indicators, a common space can be built to compare them. In that space, data represents more or less well real systems, depending e.g. on initial scale, precision of data, missing data. We precisely propose to capture that through the notion of point cloud discrepancy, which is a mathematical tool coming from sampling theory expressing how a dataset is distributed in the space it is embedded in~\cite{dick2010digital}. %Coupling discrepancies with appropriate weights depending on indicator importance allows to introduce a relative ratio of robustness between two evaluations.
\end{itemize}


Synthesizing these requirements, we propose a notion of \emph{Robustness} of an evaluation that captures both, by combining data reliability with relative importance,
\begin{enumerate}
\item \emph{Missing Data} : an evaluation based on more refined datasets will naturally be more robust.
\item \emph{Indicator importance} : indicators with more relative influence will weight more on the total robustness.
\end{enumerate}
}{
Nous décrivons à présent le cadre proposé pour permettre théoriquement de comparer la robustesse d'évaluation de deux systèmes urbains différents. Ce cadre est une généralisation d'une méthode empirique proposée dans~\cite{ecodistrictReport} pour accompagner une étude dans un autre contexte effectuant une comparaison du sens et de la pertinence des indicateurs dans un contexte de durabilité. Intuitivement, la base empirique se base sur les principes suivants :

\begin{itemize}
\item Les systèmes urbains peuvent être vus selon l'information disponible, i.e. les données brutes décrivant le système. Dans une approche basée sur les données, celles-ci sont la base de notre cadre et la robustesse sera déterminée par leur structure.
\item A partir des données sont capturés des indicateurs (fonctions objectifs). Nous supposons qu'un choix d'indicateurs est une intention particulière de traduire des aspects particuliers du système, i.e. de capturer une réalisation d'un ``fait urbain'' au sens de \noun{Mangin}~\cite{mangin1999projet} - une sorte de fait stylisé en terme de processus et de mécanismes, ayant différentes réalisations sur des systèmes distincts dans l'espace, dépendant de chaque contexte géographique précis.
\item Etant donné plusieurs systèmes et indicateurs associés, un espace commun peut être construit pour les comparer. Dans cet espace, les données représentent plus ou moins bien le système réel, c'est à dire qu'elles sont imprécises en fonction de l'échelle initiale, de la précision effective des données. Nous proposons de capturer exactement ces différents aspects au travers de la notion de discrépance d'un nuage de points, qui est un outil mathématique provenant des théories d'échantillonnage, permettant d'exprimer la façon dont un jeu de données rempli l'espace dans lequel il s'insère~\cite{dick2010digital}.
\end{itemize}

Synthétisant ces contraintes, nous proposons une notion de \emph{Robustesse} d'une évaluation qui capture à la fois, en combinant la fiabilité des données à l'importance relative des indicateurs,


\begin{enumerate}
\item \emph{Données manquantes} : une évaluation se basant sur des jeux de données plus raffinés sera naturellement plus robuste.
\item \emph{Importance des indicateurs} : les indicateurs avec plus d'importance relative pèseront plus dans la robustesse totale.
\end{enumerate}
}




%%%%%%%%%%%%%%%%
\subsection{Formal Description}


%%%%%%%%%%%%%%%%
\paragraph{Indicators}


\bpar{
Let $(S_{i})_{1\leq i\leq N}$ be a finite number of geographically disjoints territorial systems, that we assume described through raw data and intermediate indicators, yielding $S_{i}=(\mathbf{X}_{i},\mathbf{Y}_{i})\in\mathcal{X}_{i}\times\mathcal{Y}_{i}$ with $\mathcal{X}_{i}=\prod_{k}\mathcal{X}_{i,k}$ such that each subspace contain real matrices : $\mathcal{X}_{i,k}=\mathbb{R}^{n_{i,k}^{X}p_{i,k}^{X}}$ (the same holding for $\mathcal{Y}_{i}$). We also define an ontological index function $I_{X}(i,k)$ (resp. $I_{Y}(i,k)$) taking integer values which coincide if and only if the two variables have the same ontology in the sense of~\cite{livet2010}, i.e. they are supposed to represent the same real object. We distinguish ``raw data'' $\mathbf{X}_{i}$ from which indicators are computed via explicit deterministic functions, from ``intermediate indicators'' $\mathbf{Y}_{i}$ that are already integrated and can be e.g. outputs of elaborated models simulating some aspects of the urban system. We define the partial characteristic space of the ``urban fact'' by 

\begin{equation}
%\begin{split}
%(\mathcal{X},\mathcal{Y}) & \underset{def}{=} \left(\prod\tilde{\mathcal{X}}_{c}\right)\times\left(\prod\tilde{\mathcal{Y}}_{c}\right)\\
%& =  \left(\prod_{\mathcal{X}_{i,k}\in\mathcal{D}_{\mathcal{X}}}\mathbb{R}^{p_{i,k}^{X}}\right)\times\left(\prod_{\mathcal{Y}_{i,k}\in\mathcal{D}_{\mathcal{Y}}}\mathbb{R}^{p_{i,k}^{Y}}\right)
%\end{split}
(\mathcal{X},\mathcal{Y}) \underset{def}{=} \left(\prod\tilde{\mathcal{X}}_{c}\right)\times\left(\prod\tilde{\mathcal{Y}}_{c}\right) = \left(\prod_{\mathcal{X}_{i,k}\in\mathcal{D}_{\mathcal{X}}}\mathbb{R}^{p_{i,k}^{X}}\right)\times\left(\prod_{\mathcal{Y}_{i,k}\in\mathcal{D}_{\mathcal{Y}}}\mathbb{R}^{p_{i,k}^{Y}}\right)
\end{equation}


with $\mathcal{D}_{\mathcal{X}}=\{\mathcal{X}_{i,k}|I(i,k)\textrm{ distincts},n_{i,k}^{X}\mbox{ maximal}\}$
(the same holding for $\mathcal{Y}_{i}$). It is indeed the abstract space on which indicators are integrated. The indices $c$ introduced as a definition here correspond to different indicators across all systems. This space is the minimal space common to all systems allowing a common definition for indicators on each.
}{
Soit $(S_{i})_{1\leq i\leq N}$ un nombre fini de systèmes territoriaux géographiquement disjoints, % TODO : Q pourquoi nécessaire des les avoir spatially disjoints, could be different indicators on the same area ? maybe makes less sense ? missing point for comparability ?
que nous supposons décrits par les données brutes et des indicateurs intermédiaires, donnés par $S_{i}=(\mathbf{X}_{i},\mathbf{Y}_{i})\in\mathcal{X}_{i}\times\mathcal{Y}_{i}$ avec $\mathcal{X}_{i}=\prod_{k}\mathcal{X}_{i,k}$ tel que chaque sous-espace contient des matrices réelles : $\mathcal{X}_{i,k}=\mathbb{R}^{n_{i,k}^{X}p_{i,k}^{X}}$ (de la même façon pour $\mathcal{Y}_{i}$). Nous définissons également une fonction d'indice ontologique $I_{X}(i,k)$ (resp. $I_{Y}(i,k)$) prenant des valeurs entières qui coincident si et seulement si les deux variables ont même ontologie au sens de~\cite{livet2010}, c'est à dire qu'elles sont supposées représenter le même objet réel. On distingue les ``données brutes'' $\mathbf{X}_{i}$ à partir desquelles les indicateurs sont calculés généralement par des fonctions déterministes explicites, % TODO not that free on the computation here !
 des ``indicateurs intermédiaires'' $\mathbf{Y}_{i}$ qui sont déjà intégrés et peuvent être par exemple les sorties de modèles élaborés simulant certains aspects du système urbain. Nous définissons l'espace caractéristique du ``fait urbain'' par


\begin{equation}
(\mathcal{X},\mathcal{Y}) \underset{def}{=} \left(\prod\tilde{\mathcal{X}}_{c}\right)\times\left(\prod\tilde{\mathcal{Y}}_{c}\right) = \left(\prod_{\mathcal{X}_{i,k}\in\mathcal{D}_{\mathcal{X}}}\mathbb{R}^{p_{i,k}^{X}}\right)\times\left(\prod_{\mathcal{Y}_{i,k}\in\mathcal{D}_{\mathcal{Y}}}\mathbb{R}^{p_{i,k}^{Y}}\right)
\end{equation}

avec $\mathcal{D}_{\mathcal{X}}=\{\mathcal{X}_{i,k}|I(i,k)\textrm{ distincts},n_{i,k}^{X}\mbox{ maximal}\}$
(de même pour $\mathcal{Y}_{i}$). Il s'agit en fait de l'espace abstrait sur lequel les indicateurs sont intégrés. Les indices $c$ introduit par définition correspondent aux différents indicateurs au sein des différents systèmes. Cette espace est l'espace minimal commun à tous les systèmes permettant une définition commune des indicateurs pour tous.
}



\bpar{
Let $\mathbf{X}_{i,c}$ be the data canonically projected in the corresponding subspace, well defined for all $i$ and all $c$. We make the key assumption that all indicators are computed by integration against a certain kernel, i.e. that for all $c$, there exists $H_{c}$ space of real-valued functions on $(\tilde{\mathcal{X}}_{c},\tilde{\mathcal{Y}}_{c})$, such that for all $h\in H_{c}$ :
\begin{enumerate}
\item $h$ is ``enough'' regular (tempered distributions e.g.)
\item $q_c=\int_{(\tilde{\mathcal{X}}_{c},\tilde{\mathcal{Y}}_{c})}h$ is a function describing the ``urban fact'' (the indicator in itself)
\end{enumerate}

Typical concrete example of kernels can be :

\begin{itemize}
\item A mean of rows of $\mathbf{X}_{i,c}$ is computed with $h(x)=x\cdot f_{i,c}(x)$ where $f_{i,c}$ is the density of the distribution of the assumed underlying variable.
\item A rate of elements respecting a given condition $C$, $h(x)=f_{i,c}(x)\chi_{C(x)}$ 
\item For already aggregated variables $\mathbf{Y}$, a Dirac distribution allows to express them also as a kernel integral. 
\end{itemize}
}{
Soit $\mathbf{X}_{i,c}$ les données projetées canoniquement sur le sous-espace correspondant, bien définies pour tout $i$ et tout $c$. Nous faisons donc l'hypothèse clé que tous les indicateurs sont calculés par intégration contre un noyau donné, i.e. pour tout $c$ il existe $H_{c}$ espace de fonctions à valeurs réelles sur $(\tilde{\mathcal{X}}_{c},\tilde{\mathcal{Y}}_{c})$, tel que pour tout $h\in H_{c}$ : 

\begin{enumerate}
\item $h$ est ``suffisamment'' régulière (distribution tempérée par exemple)
\item $q_c=\int_{(\tilde{\mathcal{X}}_{c},\tilde{\mathcal{Y}}_{c})}h$ est une fonction décrivant le ``fait urbain'' (l'indicateur en lui-même)
\end{enumerate}

Des exemples typiques de noyaux peuvent être :


\begin{itemize}
\item Une moyenne des lignes de $\mathbf{X}_{i,c}$ est calculée par $h(x)=x\cdot f_{i,c}(x)$ où $f_{i,c}$ est la densité de la distribution de la variable sous-jacente.
\item Un taux d'éléments du jeu de données respectant une condition donnée $C$, $h(x)=f_{i,c}(x)\chi_{C(x)}$.
\item Pour des variables déjà agrégées $\mathbf{Y}$, une distribution de Dirac permet des les exprimer également comme des intégrales de noyaux.
\end{itemize}

}


%%%%%%%%%%%%%%%%
\paragraph{Aggregation}


\bpar{
Weighting objectives in multi-attribute decision-making is indeed the crucial point of the processes, and numerous methods are available (see~\cite{wang2009review} for a review for the particular case of sustainable energy management). Let define weights for the linear aggregation. We assume the indicators normalized, i.e. $q_c \in [0,1]$, for a more simple construction of relative weights. For $i,c$ and $h_{c}\in H_{c}$ given, the weight $w_{i,c}$ is simply constituted by the relative importance of the indicator $w_{i,c}^{L}=\frac{\hat{q}_{i,c}}{\sum_{c}\hat{q}_{i,c}}$ where $\hat{q}_{i,c}$ is an estimator of $q_{c}$ for data $\mathbf{X}_{i,c}$ (i.e. the effectively calculated value). Note that this step can be extended to any sets of weight attributions, by taking for example $\tilde{w}_{i,c} = w_{i,c} \cdot w'_{i,c}$ if $\mathbf{w}'$ are the weights attributed by the decision-maker. We focus here on the relative influence of attributes and thus choose this simple form for weights.
}{
La détermination des poids est en fait le point crucial des processus de prise de décision multi-attributs, et de nombreuses méthodes sont disponibles (voir~\cite{wang2009review} pour une revue dans le cas particulier de la gestion de l'énergie durable). Définissons les poids pour l'agrégation linéaire. Nous supposons les indicateurs normalisés, i.e. $q_c \in [0,1]$, pour une construction plus simple des poids relatifs. % TODO : indeed $h_c \in [0,1]$ is the right assumption
Pour $i,c$ et $h_{c}\in H_{c}$ donnés, le poids $w_{i,c}$ est simplement constitué par l'importance relative de l'indicateur $w_{i,c}^{L}=\frac{\hat{q}_{i,c}}{\sum_{c}\hat{q}_{i,c}}$ où $\hat{q}_{i,c}$ est un estimateur de $q_{c}$ pour les données $\mathbf{X}_{i,c}$ (i.e. la valeur calculée effectivement). On peut noter que cette étape n'est pas contraignante et que cela peut être étendu à tout ensemble d'attribution de poids, en prenant par exemple $\tilde{w}_{i,c} = w_{i,c} \cdot w'_{i,c}$ si $\mathbf{w}'$ sont les poids fixés par le preneur de décisions. Nous nous concentrons sur l'influence relative des attributs et pour cela choisissons cette forme simple pour les poids. 
}










%%%%%%%%%%%%%%%%
\paragraph{Robustness Estimation}


\bpar{
The scene is now set up to be able to estimate the robustness of the evaluation done through the aggregated function. Therefore, we apply an integral approximation method similar to methods introduced in~\cite{varet2010developpement}, since the integrated form of indicators indeed brings the benefits of such powerful theoretical results. Let $\mathbf{X}_{i,c}=(\vec{X}_{i,c,l})_{1\leq l\leq n_{i,c}}$ and $D_{i,c}=Disc_{\tilde{\mathcal{X}}_{c},L^2}(\mathbf{X}_{i,c})$ the discrepancy of data points cloud\footnote{The discrepancy is defined as the $L2$-norm of local discrepancy which is for normalized data points $\mathbf{X}=(x_{ij})\in \left[0,1\right]^d$, a function of $\mathbf{t}\in \left[0,1\right]^d$ comparing the number of points falling in the corresponding hypercube with its volume, by $disc(\mathbf{t}) = \frac{1}{n}\sum_i \mathbbm{1}_{\prod_j x_{ij}<t_j} - \prod_j t_j$. It is a measure of how the point cloud covers the space.}~\cite{niederreiter1972discrepancy}. With $h\in H_{c}$, we have the upper bound on the integral approximation error

\[
\left\Vert \int h_{c}-\frac{1}{n_{i,c}}\sum_{l}h_{c}(\vec{X}_{i,c,l})\right\Vert \leq K\cdot\left|\left|\left|h_{c}\right|\right|\right|\cdot D_{i,c}
\]

where $K$ is a constant independent of data points and objective function. It directly yields

\[
\left\Vert \int\sum w_{i,c}h_{c}-\frac{1}{n_{i,c}}\sum_{l}w_{i,c}h_{c}(\vec{X}_{i,c,l})\right\Vert \leq K\sum_{c}\left|w_{i,c}\right|\left|\left|\left|h_{c}\right|\right|\right|\cdot D_{i,c}
\]

Assuming the error reasonably realized (``worst case'' scenario for knowledge of the theoretical value of aggregated function), we take this upper bound as an approximation of its magnitude. Furthermore, taking normalized indicators implies $\left|\left|\left|h_c\right|\right|\right| = 1$. We propose then to compare error bounds between two evaluations. They depend only on data distribution (equivalent to \emph{statistical robustness}) and on indicators chosen (sort of \emph{ontological robustness}, i.e. do the indicators have a real sense in the chosen context and do their values make sense), and are a way to combine these two type of robustnesses into a single value.
}{
La scène est à présent prête pour permettre d'estimer la robustesse d'une évaluation faite par la fonction d'agrégation. Pour cela, nous appliquons un théorème d'approximation d'intégrale similaire au méthodes introduites dans~\cite{varet2010developpement}, puisque la forme intégrée des indicateurs permet justement de bénéficier de tels résultats théoriquement puissant. Soit $\mathbf{X}_{i,c}=(\vec{X}_{i,c,l})_{1\leq l\leq n_{i,c}}$ et $D_{i,c}=Disc_{\tilde{\mathcal{X}}_{c},L^2}(\mathbf{X}_{i,c})$ le discrépance du jeu de données\footnote{La discrépance est définie comme la norme-$L2$ de la discrépance locale qui est pour des points de données normalisés $\mathbf{X}=(x_{ij})\in \left[0,1\right]^d$, une fonction de $\mathbf{t}\in \left[0,1\right]^d$ comparant le nombre de points compris dans le volume de l'hypercube correspondant, donné par $disc(\mathbf{t}) = \frac{1}{n}\sum_i \mathbbm{1}_{\prod_j x_{ij}<t_j} - \prod_j t_j$. C'est une mesure de la manière dont le nuage de points couvre l'espace.}~\cite{niederreiter1972discrepancy}. Avec $h\in H_{c}$, on a la borne supérieure sur l'erreur d'approximation de l'intégrale

\[
\left\Vert \int h_{c}-\frac{1}{n_{i,c}}\sum_{l}h_{c}(\vec{X}_{i,c,l})\right\Vert \leq K\cdot\left|\left|\left|h_{c}\right|\right|\right|\cdot D_{i,c}
\]

où $K$ est une constante indépendante des points de données et des fonctions objectifs. Cela donne directement

\[
\left\Vert \int\sum w_{i,c}h_{c}-\frac{1}{n_{i,c}}\sum_{l}w_{i,c}h_{c}(\vec{X}_{i,c,l})\right\Vert \leq K\sum_{c}\left|w_{i,c}\right|\left|\left|\left|h_{c}\right|\right|\right|\cdot D_{i,c}
\]


En supposant l'erreur réalisée de manière raisonnable (scénario du ``pire de cas'' pour la connaissance de la valeur théorique de la fonction agrégée), nous prenons cette borne supérieure comme une approximation de sa magnitude. De plus, la normalisation des indicateurs implique que $\left|\left|\left|h_c\right|\right|\right| = 1$. Nous proposons alors de comparer les bornes d'erreurs entre deux évaluations. Elle dépendent seulement de la distribution des données (équivalence à la \emph{robustesse statistique}) et des indicateurs choisis (sorte de \emph{robustesse ontologique}, i.e. est-ce que les indicateurs ont un sens réel dans le contexte choisi et est-ce que leur valeur fait sens), et sont un moyen de combiner ces deux types de robustesse dans une seule valeur.

}


\bpar{
We thus define a \emph{robustness ratio} to compare the robustness of two evaluations by
}{
Nous définissons ainsi un \emph{ratio de robustesse} pour comparer la robustesse de deux évaluations par 
}

\begin{equation}
R_{i,i'}=\frac{\sum_{c}w_{i,c}\cdot D_{i,c}}{\sum_{c}w_{i',c}\cdot D_{i',c}}
\end{equation}

\bpar{
The intuitive sense of this definition is that one compares robustness of evaluations by comparing the highest error done in each based on data structure and relative importance.
}{
L'interprétation intuitive de cette définition est que l'on compare la robustesse des évaluations en comparant la plus grande erreur faite dans chaque cas selon la structure des données et l'importance relative.
}



\bpar{
By taking then an order relation on evaluations by comparing the position of the ratio to one, it is obvious that we obtain a complete order on all possible evaluations. This ratio should theoretically allow to compare any evaluation of an urban system. To keep an ontological sense to it, it should be used to compare disjoints sub-systems with a reasonable proportion of indicators in common, or the same sub-system with varying indicators. Note that it provides a way to test the influence of indicators on an evaluation by analyzing the sensitivity if the ratio to their removal. On the contrary, finding a ``minimal'' number of indicators each making the ratio strongly vary should be a way to isolate essential parameters ruling the sub-system.
}{
En construisant une relation d'ordre sur les évaluations en comparant la position du ratio par rapport à un, il est clair qu'on obtient un ordre complet sur l'ensemble des évaluations possibles. Ce ratio devrait en théorie permettre de comparer n'importe quelle évaluation d'un système urbain. Afin de garder un sens ontologique à cela, il devrait être utilisé pour comparer des sous-systèmes disjoints  avec une proportion raisonnable d'indicateurs en commun, ou le même sous-système avec des indicateurs différents. On peut noter que cela fournit un moyen de tester l'influence des indicateurs sur une évaluation, en analysant la sensibilité du ratio à leur suppression. Au contraire, la détermination d'un nombre ``minimal'' d'indicateurs faisant chacun varier le ratio fortement pourrait être un moyen d'isoler des paramètres essentiels régissant le sous-système.
}





%%%%%%%%%%%%%%%%
%% Results
%%%%%%%%%%%%%%%%
\section{Results}{Résultats}

%We apply our framework to a ``toy-model'', as elaborated indicator definition nor raw data collection was not the primary aim of this work, which was more to theoretically introduce the framework. It is however essential for such a data-driven paradigm to demonstrate through simulation the potentialities of the method, what we do here on semi-synthetic data.


%%%%%%%%%%%%%%%%
\paragraph{Implementation}


\bpar{
Preprocessing of geographical data is made through QGIS~\cite{qgis2011quantum} for performance reasons. Core implementation of the framework is done in R~\cite{team2000r} for the flexibility of data management and statistical computations. Furthermore, the package \texttt{DiceDesign}~\cite{franco20092} written for numerical experiments and sampling purposes, allows an efficient and direct computation of discrepancies. Last but not least, all source code is openly available on the \texttt{git} repository of the project\footnote{at \url{https://github.com/JusteRaimbault/RobustnessDiscrepancy}} for reproducibility purposes~\cite{ram2013git}.
}{
Le pré-traitement des données géographiques est fait via QGIS~\cite{qgis2011quantum} pour des raisons de performances. % TODO plutôt ergonomie ?
L'implémentation du coeur est faite en R~\cite{team2000r} pour la flexibilité de la gestion des données et du traitement statistique. De plus, le package \texttt{DiceDesign}~\cite{franco20092} conçu pour les expériences numériques et l'échantillonnage, permet un calcul efficient et direct des discrépances. Enfin, tout aussi important, l'ensemble du code source est disponible de manière ouverte sur le dépôt \texttt{git}du projet\footnote{à \url{https://github.com/JusteRaimbault/RobustnessDiscrepancy}} pour permettre la reproductibilité et la réutilisation~\cite{ram2013git}.
}



%%%%%%%%%%%%%%%%
\subsection{Implementation on Synthetic Data}


\bpar{
We propose in a first time to illustrate the implementation with an application to synthetic data and indicators, for intra-urban quality indicators in the city of Paris.
}{
Nous proposons dans un premier temps d'illustrer l'implémentation par une application à des données et indicateurs synthétiques, pour des indicateurs de qualité de vie intra-urbaine pour la ville de Paris.
}


%%%%%%%%%%%%%%%%
\paragraph{Data Collection}

\bpar{
We base our virtual case on real geographical data, in particular for \emph{arrondissements} of Paris. We use open data available through the OpenStreetMap project~\cite{bennett2010openstreetmap} that provides accurate high definition data for many urban features. We use the street network and position of buildings within the city of Paris. %\footnote{for which the third-party website \url{https://mapzen.com/metro-extracts/} provides already packaged extracts as for many metropolitan areas of the world}
Limits of \emph{arrondissements}, used to overlay and extract features when working on single districts, are also extracted from the same source. We use centroids of buildings polygons, and segments of street network. Dataset overall consists of around $200k$ building features and $100k$ road segments.
%An example of visualization of used data is shown in Fig.~\ref{fig:data_ex}.
}{
Le cas virtuel se base sur des données géographiques réelle, en particulier pour les arrondissements parisiens. Nous utilisons les données disponibles par le projet OpenStreetMap~\cite{bennett2010openstreetmap} qui fournit déjà des données précises en haute définition pour de nombreux aspects urbains. Nous utilisons le réseau de rues et la position des bâtiments dans la ville de Paris. Les limites des arrondissements, utilisées pour agréger et extraire les features lorsqu'on travaille sur un seul district, sont aussi pris de la même source. Nous utilisons les centroïdes des polygones des bâtiments et les segments du réseau de rues. Le jeu de données brutes consiste d'environ $200k$ bâtiments et $100k$ segments de rues.
}


%%%%%%%%%%%%%%%%
\paragraph{Virtual Cases}

\bpar{
We work on each district of Paris (from the 1st to the 20th) as an evaluated urban system. We construct random synthetic data associated to spatial features, so each district has to be evaluated many time to obtain mean statistical behavior of toy indicators and robustness ratios. The indicators chosen need to be computed on residential and street network spatial data. We implement two mean kernels and a conditional mean to show different examples, linked to environmental sustainability and quality of life, that are required to be maximized. Note that these indicators have a real meaning but no particular reason to be aggregated, they are chosen here for the convenience of the toy model and the generation of synthetic data. With $a\in \{1\ldots 20\}$ the number of the district, $A(a)$ corresponding spatial extent, $b\in B$ building coordinates and $s\in S$ street segments, we take
\begin{itemize}
\item Complementary of the average daily distance to work with car per individual, approximated by, with $n_{cars}(b)$ number of cars in the building (randomly generated by associated of cars to a number of building proportional to motorization rate $\alpha_m ~ 0.4$ in Paris), $d_w$ distance to work of individuals (generated from the building to a uniformly generated random point in spatial extent of the dataset), and $d_{max}$ the diameter of Paris area, $\bar{d}_w = 1 - \frac{1}{|b\in A(a)|} \cdot \sum_{b\in A(a)}{n_{cars}(b)\cdot \frac{d_w}{d_{max}}}$
\item Complementary of average car flows within the streets in the district, approximated by, with $\varphi(s)$ relative flow in street segment $s$, generated through the minimum of 1 and a log-normal distribution adjusted to have $95\%$ of mass smaller than 1 what mimics the hierarchical distribution of street use (corresponding to betweenness centrality), and $l(s)$ segment length, $\bar{\varphi} = 1 - \frac{1}{|s\in A(a)|} \cdot \sum_{s \in A(a)}{\varphi(s)\cdot \frac{l(s)}{\max{(l(s))}}}$

\item Relative length of pedestrian streets $\bar{p}$, computed through a randomly uniformly generated dummy variable adjusted to have a fixed global proportion of segments that are pedestrian.
\end{itemize}
}{
Nous travaillons sur chaque arrondissement de Paris (du 1er au 20ème) comme un système urbain évalué. Des données synthétiques aléatoires sont associées aux features spatiales, chaque arrondissement pouvant alors être évalué de manière stochastique, et des répétitions permettent d'obtenir le comportement statistique moyen des indicateurs jouets et des ratios de robustesse. Les indicateurs choisis doivent être calculés comme des indicateurs résidentiels et du réseau de rues. Pour montrer différents exemples, nous implémentons deux kernels moyens et une moyenne conditionnelle, tous liés à la durabilité environnementale et la qualité de vie, chacun devant être maximisés. On peut noter que ces indicateurs ont un sens réel mais pas de raison particulière d'être agrégés, ils sont ici choisis pour l'aspect pratique du modèle jouet et de la génération de données synthétiques. Avec $a\in \{1\ldots 20\}$ le nombre d'arrondissements, $A(a)$ l'aire spatiale correspondante à chacun, $b\in B$ les coordonnées des bâtiments et $s\in S$ les segments de rues, nous prenons

\begin{itemize}
\item Le complémentaire de la distance journalière moyenne au travail en voiture par individu, approché par, avec $n_{cars}(b)$ nombre de voiture dans le bâtiment (généré aléatoirement en associant des voiture à bâtiments proportionnel au taux de motorisation attendu $\alpha_m ~ 0.4$ à Paris), $d_w$ distance des individus à leur travail (généré à partir du bâtiment vers un point aléatoire distribué uniformément dans l'étendue spatiale du jeu de données), et $d_{max}$ le diamètre de l'aire de Paris, $\bar{d}_w = 1 - \frac{1}{|b\in A(a)|} \cdot \sum_{b\in A(a)}{n_{cars}(b)\cdot \frac{d_w}{d_{max}}}$
\item Le complémentaire des flots moyens de voitures des rues dans la zone, approché par, avec $\varphi(s)$ flot relatif dans le segment de rue $s$, généré par le minimum entre 1 et une distribution log-normale ajustée pour avoir $95\%$ de masse plus petite que 1, ce qui mimique la distribution hiérarchique de l'utilisation des rues (qui correspond à la centralité de chemin), et $l(s)$ longueur du segment, $\bar{\varphi} = 1 - \frac{1}{|s\in A(a)|} \cdot \sum_{s \in A(a)}{\varphi(s)\cdot \frac{l(s)}{\max{(l(s))}}}$
\item Longueur relative de rues piétonnes $\bar{p}$, calculé vie une dummy variable aléatoire uniforme ajustée pour obtenir une proportion fixée de segments pédestre.
\end{itemize}
}




%%%%%%%%%%%%%%%%%
%% Results table
%%%%%%%%%%%%%%%%

\begin{table}[h!]
\hspace{-1cm}
\begin{tabular}[6pt]{c|c|c|c|c}
Arrdt & $<\bar{d}_w> \pm \sigma (\bar{d}_w)$ & $<\bar{\varphi}> \pm \sigma (\bar{\varphi})$ & $<\bar{p}> \pm \sigma (\bar{p})$ & $R_{i,1}$ \\[3pt]
\hline
1 th & 0.731655 $\pm$ 0.041099 & 0.917462 $\pm$ 0.026637 & 0.191615 $\pm$ 0.052142 & 1.000000 $\pm$ 0.000000\\[3pt]
\hline
2 th & 0.723225 $\pm$ 0.032539 & 0.844350 $\pm$ 0.036085 & 0.209467 $\pm$ 0.058675 & 1.002098 $\pm$ 0.039972\\[3pt]
\hline
3 th & 0.713716 $\pm$ 0.044789 & 0.797313 $\pm$ 0.057480 & 0.185541 $\pm$ 0.065089 & 0.999341 $\pm$ 0.048825\\[3pt]
\hline
4 th & 0.712394 $\pm$ 0.042897 & 0.861635 $\pm$ 0.030859 & 0.201236 $\pm$ 0.044395 & 0.973045 $\pm$ 0.036993\\[3pt]
\hline
5 th & 0.715557 $\pm$ 0.026328 & 0.894675 $\pm$ 0.020730 & 0.209965 $\pm$ 0.050093 & 0.963466 $\pm$ 0.040722\\[3pt]
\hline
6 th & 0.733249 $\pm$ 0.026890 & 0.875613 $\pm$ 0.029169 & 0.206690 $\pm$ 0.054850 & 0.990676 $\pm$ 0.031666\\[3pt]
\hline
7 th & 0.719775 $\pm$ 0.029072 & 0.891861 $\pm$ 0.026695 & 0.209265 $\pm$ 0.041337 & 0.966103 $\pm$ 0.037132\\[3pt]
\hline
8 th & 0.713602 $\pm$ 0.034423 & 0.931776 $\pm$ 0.015356 & 0.208923 $\pm$ 0.036814 & 0.973975 $\pm$ 0.033809\\[3pt]
\hline
9 th & 0.712441 $\pm$ 0.027587 & 0.910817 $\pm$ 0.015915 & 0.202283 $\pm$ 0.049044 & 0.971889 $\pm$ 0.035381\\[3pt]
\hline
10 th & 0.713072 $\pm$ 0.028918 & 0.881710 $\pm$ 0.021668 & 0.210118 $\pm$ 0.040435 & 0.991036 $\pm$ 0.038942\\[3pt]
\hline
11 th & 0.682905 $\pm$ 0.034225 & 0.875217 $\pm$ 0.019678 & 0.203195 $\pm$ 0.047049 & 0.949828 $\pm$ 0.035122\\[3pt]
\hline
12 th & 0.646328 $\pm$ 0.039668 & 0.920086 $\pm$ 0.019238 & 0.198986 $\pm$ 0.023012 & 0.960192 $\pm$ 0.034854\\[3pt]
\hline
13 th & 0.697512 $\pm$ 0.025461 & 0.890253 $\pm$ 0.022778 & 0.201406 $\pm$ 0.030348 & 0.960534 $\pm$ 0.033730\\[3pt]
\hline
14 th & 0.703224 $\pm$ 0.019900 & 0.902898 $\pm$ 0.019830 & 0.205575 $\pm$ 0.038635 & 0.932755 $\pm$ 0.033616\\[3pt]
\hline
15 th & 0.692050 $\pm$ 0.027536 & 0.891654 $\pm$ 0.018239 & 0.200860 $\pm$ 0.024085 & 0.929006 $\pm$ 0.031675\\[3pt]
\hline
16 th & 0.654609 $\pm$ 0.028141 & 0.928181 $\pm$ 0.013477 & 0.202355 $\pm$ 0.017180 & 0.963143 $\pm$ 0.033232\\[3pt]
\hline
17 th & 0.683020 $\pm$ 0.025644 & 0.890392 $\pm$ 0.023586 & 0.198464 $\pm$ 0.033714 & 0.941025 $\pm$ 0.034951\\[3pt]
\hline
18 th & 0.699170 $\pm$ 0.025487 & 0.911382 $\pm$ 0.027290 & 0.188802 $\pm$ 0.036537 & 0.950874 $\pm$ 0.028669\\[3pt]
\hline
19 th & 0.655108 $\pm$ 0.031857 & 0.884214 $\pm$ 0.027816 & 0.209234 $\pm$ 0.032466 & 0.962966 $\pm$ 0.034187\\[3pt]
\hline
20 th & 0.637446 $\pm$ 0.032562 & 0.873755 $\pm$ 0.036792 & 0.196807 $\pm$ 0.026001 & 0.952410 $\pm$ 0.038702\\[3pt]
\hline
\end{tabular}

\bigskip

\caption{}
%\caption{\bpar{Numerical results of simulation for each district with $N=50$ repetitions. Each toy indicator value is given by mean on repetitions and associated standard deviation. Robustness ratio is computed relative to first district (arbitrary choice). A ratio smaller than 1 means that integral bound is smaller for upper district, i.e. that evaluation is more robust for this district. Because of the small size of first district, we expected a majority of district to give ratio smaller than 1, what is confirmed by results, even when adding standard deviations.}{Résultats numériques des simulations pour chaque arrondissement avec $N=50$ répétitions. Chaque valeur des indicateurs factice est donnée par sa moyenne sur les répétitions et la déviation standard associée. Le ratio de robustesse est calculé par rapport au premier arrondissement (choix arbitraire). Un ratio inférieur à 1 signifie que la borne de l'intégrale est plus petite pour le premier système, i.e. que l'évaluation est plus robuste pour celui-ci. % TODO : wrong à l'oral 15th block size ? A cause de la petite taille du 1er arrondissement, on s'attend que la majorité des arrondissements aient un ratio plus petit que 1, ce qui se confirme dans les résultats même lorsque l'on ajoute la déviation standard.}}


\end{table}



\bpar{
As synthetic data are stochastic, we run the computation for each district $N=50$ times, what was a reasonable compromise between statistical convergence and time required for computation. Table 1 shows results (mean and standard deviations) of indicator values and robustness ratio computation. Obtained standard deviation confirm that this number of repetitions give consistent results. Indicators obtained through a fixed ratio show small variability what may a limit of this toy approach. However, we obtain the interesting result that a majority of districts give more robust evaluations than 1st district, what was expected because of the size and content of this district : it is indeed a small one with large administrative buildings, what means less spatial elements and thus a less robust evaluation following our definition of the robustness.
}{
Comme les données synthétiques sont stochastiques, les simulations sont lancées pour chaque quartier $N=50$ fois, ce qui était un compromis raisonnable entre convergence statistique et temps nécessaire au calcul. La table 1 montre les résultats (moyennes et déviations standard) des valeurs des indicateurs et le calcul du ratio de robustesse. Les déviations standard obtenues confirment que ce nombre de simulations donnent des résultats consistants. Les indicateurs obtenus en fixant un ratio fixe montre peu de variabilité, ce qui peut être une limite de cette approche jouet. On obtient toutefois le résultat intéressant que la majorité des arrondissements donne des évaluations plus robustes que le 1er arrondissement, ce qui était attendu par la taille et la fonction de ce quartier: il s'agit en effet d'un petit quartier avec de grand bâtiment administratifs, ce qui implique moins d'éléments spatiaux et pour cela une évaluation moins robuste selon la définition qu'on en a donnée.
}






%%%%%%%%%%%%%%%%
\subsection{Application to a Real Case : Metropolitan Segregation}


\bpar{
The first example was aimed to show potentialities of the method but was purely synthetic, hence yielding no concrete conclusion nor implications for policy. We propose now to apply it to real data for the example of metropolitan segregation.
}{
Le premier exemple avait pour but de montrer les potentialités de la méthode mais était purement synthétique, ne pouvant pour cela fournir pas de conclusion concrete ni d'implications pour la gouvernance. Nous proposons maintenant de l'appliquer à des données réelles dans le cas de la ségrégation métropolitaine.
}


\paragraph{Data}

\bpar{
We work on income data available for France at an intra-urban level (basic statistical units IRIS) for the year 2011 under the form of summary statistics (deciles if the area is populated enough to ensure anonymity), provided by INSEE\footnote{\texttt{http://www.insee.fr}}. Data are associated with geographical extent of statistical units, allowing computation of spatial analysis indicators. 
}{
Nous travaillons sur les données de revenus, disponible pour la France à un niveau intra-urbain (unités statistiques élémentaires IRIS) pour l'année 2011 sous la forme de résumé statistiques (déciles uniquement si la zone est peuplée suffisamment pour assurer l'anonymat), fournies par l'INSEE\footnote{\texttt{http://www.insee.fr}}. Les données sont associées à l'étendue géographique des unités statistiques, permettant le calcul d'indicateurs d'analyse spatiale.
}


\paragraph{Indicators}


\bpar{
We use here three indicators of segregation integrated on a geographical area. Let assume the area divided into covering units $\mathcal{S}_i$ for $1\leq i \leq N$ with centroids $(x_i,y_i)$. Each unit has characteristics of population $P_i$ and median income $X_i$. We define spatial weights used to quantify strength of geographical interactions between units $i,j$, with $d_{ij}$ euclidian distance between centroids : $w_{ij} = \frac{P_i P_j}{\left(\sum_k P_k\right)^2}\cdot \frac{1}{d_{ij}}$ if $i\neq i$ and $w_{ii} = 0$. The normalized indicators are the following

\begin{itemize}
\item Spatial autocorrelation Moran index, defined as weighted normalized covariance of median income by $\rho = \frac{N}{\sum_{ij}w_{ij}}\cdot \frac{\sum_{ij}w_{ij}\left(X_i - \bar{X}\right)\left(X_j - \bar{X}\right)}{\sum_i \left(X_i - \bar{X}\right)^2}$
\item Dissimilarity index (close to Moran but integrating local dissimilarities rather than correlations), given by $d =  \frac{1}{\sum_{ij}w_{ij}} \sum_{ij} w_{ij} \left|\tilde{X}_i - \tilde{X}_j\right|$\\ with $\tilde{X}_i = \frac{X_i - \min(X_k)}{\max(X_k) - \min(X_k)}$
\item Complementary of the entropy of income distribution that is a way to capture global inequalities $\varepsilon = 1 + \frac{1}{\log(N)} \sum_i \frac{X_i}{\sum_k X_k} \cdot \log\left(\frac{X_i}{\sum_k X_k}\right)$
\end{itemize}
}{
Nous utilisons ici trois indicateurs de ségrégation intégrés sur une zone géographique. Supposons la zone divisée en unités couvrantes $\mathcal{S}_i$ pour $1\leq i \leq N$ avec pour centroïdes $(x_i,y_i)$. Chaque unité a des caractéristiques de population $P_i$ et de revenu médian $X_i$. On définit des poids spatiaux utilisés pour quantifier l'intensité des interactions géographiques entre unités $i,j$, avec $d_{ij}$ distance euclidienne entre centroïdes: $w_{ij} = \frac{P_i P_j}{\left(\sum_k P_k\right)^2}\cdot \frac{1}{d_{ij}}$ si $i\neq j$ % TODO typo eng paper
 et $w_{ii} = 0$. Les indicateurs normalisés sont les suivants


\begin{itemize}
\item Indice d'autocorrelation spatiale de Moran, défini comme la covariance pondérée normalisée du revenu médian par $\rho = \frac{N}{\sum_{ij}w_{ij}}\cdot \frac{\sum_{ij}w_{ij}\left(X_i - \bar{X}\right)\left(X_j - \bar{X}\right)}{\sum_i \left(X_i - \bar{X}\right)^2}$
\item Indice de dissimilarité (proche du Moran mais intégrant les dissimilarités locales plutôt que les corrélations), donné par $d =  \frac{1}{\sum_{ij}w_{ij}} \sum_{ij} w_{ij} \left|\tilde{X}_i - \tilde{X}_j\right|$\\ avec $\tilde{X}_i = \frac{X_i - \min(X_k)}{\max(X_k) - \min(X_k)}$
\item Le complémentaire de l'entropie de la distribution des revenus, qui est une façon de capturer des inégalités globales $\varepsilon = 1 + \frac{1}{\log(N)} \sum_i \frac{X_i}{\sum_k X_k} \cdot \log\left(\frac{X_i}{\sum_k X_k}\right)$
\end{itemize}
}


\bpar{
Numerous measures of segregation with various meanings and at different scales are available, as for example at the level of the unit by comparison of empirical wage distribution with a theoretical null model~\cite{louf2015patterns}. The choice here is arbitrary in order to illustrate our method with a reasonable number of dimensions.
}{
De nombreuses mesures de ségrégation avec différentes signification à différentes échelles existent, comme par exemple à l'échelle d'une unité spatiale élémentaire par comparaison de la distribution de revenus empirique avec un modèle nul~\cite{louf2015patterns}. Le choix est ici arbitraire, afin d'illustrer la méthode avec un nombre raisonnable de dimensions.
}


%%%%%%%%%%%%%%%%
\begin{figure}[h!]
\centering
%\vspace{-1.5cm}
\hspace{-3cm}\includegraphics[width=1.2\textwidth]{Figures/RobustnessDiscrepancy/grandParis_income_moran.pdf}\hspace{-2cm}
\vspace{-2.5cm}
\caption{}
%\caption{\bpar{\textbf{Maps of Metropolitan Segregation.} Maps show yearly median income on basic statistical units (IRIS) for the three departments constituting mainly the Great Paris metropolitan area, and the corresponding local Moran spatial autocorrelation index, defined for unit $i$ as $\rho_i = N/\sum_{j}w_{ij} \cdot \frac{\sum_{j} w_{ij} (X_j - \bar{X})(X_i - \bar{X})}{\sum_i (X_i - \bar{X})^2}$. The most segregated areas coincide with the richest and the poorest, suggesting an increase of segregation in extreme situations.}{\textbf{Carte de ségregation métropolitaine.} Les cartes montrent le revenu annuel médian pour les unités statistiques élémentaires (IRIS) pour les trois départements correspondant globalement à la métropole du Grand Paris, et l'index local d'autocorrelation spatiale de Moran correspondant, défini pour l'unité $i$ par $\rho_i = N/\sum_{j}w_{ij} \cdot \frac{\sum_{j} w_{ij} (X_j - \bar{X})(X_i - \bar{X})}{\sum_i (X_i - \bar{X})^2}$. Les zones les plus ségréguées coincident avec les plus riches et les plus pauvres, suggérant une augmentation de la ségrégation dans les cas extremes.}}
\end{figure}
%%%%%%%%%%%%%%%%

\paragraph{Results}


\bpar{
We apply our method with these indicators on the Greater Paris area, constituted of four \emph{d{\'e}partements} that are intermediate administrative units. The recent creation of a new metropolitan governance system~\cite{gilli2009paris} underlines interrogations on its consistence, and in particular on its relation to intermediate spatial inequalities. We show in Fig. 1 maps of spatial distribution of median income and corresponding local index of autocorrelation. We observe the well-known West-East opposition and district disparities inside Paris as they were formulated in various studies, such as~\cite{guerois2009dynamique} through the analysis of real estate transactions dynamics. We then apply our framework to answer a concrete question that has implications for urban policy : \textit{how are the evaluation of segregation within different territories sensitive to missing data ?} To do so, we proceed to Monte Carlo simulations (75 repetitions) during which a fixed proportion of data is randomly removed, and the corresponding robustness index is evaluated with renormalized indicators. Simulations are done on each \emph{department} separately, each time relatively to the robustness of the evaluation of full Greater Paris. Results are shown in Fig. 2. All areas present a slightly better robustness than the reference, what could be explained by local homogeneity and thus more fiable segregation values. Implications for policy that can be drawn are for example direct comparisons between areas : a loss of 30\% of information on 93 area corresponds to a loss of only 25\% in 92 area. The first being a deprived area, the inequality is increased by this relative lower quality of statistical information. The study of standard deviations suggest further investigations as different response regimes to data removal seem to exist.
}{
La méthode est appliquée avec ces indicateurs à la zone du Grand Paris, constitué de 4 département qui sont des niveaux administratifs intermédiaires. La création récente d'un nouveau système de gouvernance métropolitaine~\cite{gilli2009paris} met en évidence des interrogations sur sa pertinence, notamment sur ses capacités d'atténuer les inégalités spatiales. On peut voir en Fig.~\ref{fig:segregmap} % TODO no fig label 
 les cartes de la distribution spatiale du revenu médian et de l'index local d'autocorrelation spatiale correspondant. La dichotomie bien connue entre est et ouest est retrouvée ainsi que la disparité des quartiers intra-muros, comme cela été présenté par diverses études, comme~\cite{guerois2009dynamique} à travers l'analyse des dynamiques des transactions immobilières. Notre cadre d'étude est ensuite appliqué à une question concrète ayant des implications pour la prise de décision : \textit{dans quelle mesure une évaluation de la ségrégation au sein de différents territoires est sensible aux données manquantes ?} Pour cela, on procède à des simulations de Monte-Carlo (75 répétitions) pour lesquelles une proportion fixe de données est supprimée aléatoirement, et l'indice de robustesse correspondant est évalué avec les indicateurs normalisés. Les simulations sont faites sur chaque département de façon indépendante, à chaque fois pour une robustesse relative à l'évaluation du Grand Paris complet. Les résultats sont présentés en Fig.~\ref{fig:missingdata}. Toutes les zones ont une robustesse légèrement meilleure que la référence, ce qui pourrait être expliqué par une homogénéité locale et donc des indices de ségrégation plus fiables. Les implications pour la prise de décision qui peuvent être par exemple tirées sont des comparaisons directes entre les zones : une perte de 30\% de l'information sur le 93 correspond à une perte de seulement 25\% pour le 92. La première zone étant déjà défavorisée socio-économiquement, l'inégalité est augmentée par cette qualité moindre de l'information statistique. L'étude des déviations standard suggère des études plus approfondies comme différents régimes de réponse à la suppression de données semblent exister.
}



%%%%%%%%%%%%%%%%
\begin{figure}
\centering
%\vspace{-1cm}
\hspace{-2cm}\includegraphics[width=0.55\textwidth]{Figures/RobustnessDiscrepancy/alldeps_rob_renormindics.pdf}
\includegraphics[width=0.55\textwidth]{Figures/RobustnessDiscrepancy/alldeps_robsd_renormindics.pdf}\hspace{-2cm}
\caption{}
%\caption{\bpar{\textbf{Sensitivity of robustness to missing data.} \textit{Left.} For each department, Monte Carlo simulations (N=75 repetitions) are used to determine the impact of missing data on robustness of segregation evaluation. Robustness ratios are all computed relatively to full metropolitan area with all available data. Quasi-linear behavior translates an approximative linear decrease of discrepancy as a function of data size. The similar trajectory of poorest departments (93,94) suggest the correction to linear behavior being driven be segregation patterns. \textit{Right.} Corresponding standard deviations of robustness ratios. Different regimes (in particular 93 against others) unveil phase transitions at different levels of missing data, meaning that the evaluation in 94 is from this point of view more sensitive to missing data.}{\textbf{Sensibilité de la robustesse aux données manquantes.} \textit{Gauche.} Pour chaque département, des simulations de Monte-Carlo (N=75 répétitions) sont utilisées pour déterminer l'impact des données manquantes sur la robustesse de l'évaluation de la ségrégation. Les ratios de robustesse sont tous calculés relativement à la région métropolitaine complète avec toutes les données disponibles. Le comportement quasi-linéaire traduit une décroissance approximativement linéaire de la discrépance en fonction de la taille des données. Les trajectoires similaires des départements les plus pauvres (93,94) suggère que la correction au comportement linéaire est fonction des motifs de ségrégation. \textit{Droite.} Déviations standard des ratios de robustesse. Les différents régimes (en particulier le 93 contre les autres) révèlent des transitions de phase à différents niveaux de données manquantes, signifiant que l'évaluation dans le 94 % TODO typo département ? est de ce point de vue plus sensible aux données manquantes.}}
\end{figure}
%%%%%%%%%%%%%%%%





%%%%%%%%%%%%%%%%
%% Discussion
%%%%%%%%%%%%%%%%
\section{Discussion}{Discussion}


%%%%%%%%%%%%%%%%
\subsection{Applicability to Real situations}


\paragraph{Implications for Decision-making}


\bpar{
The application of our method to concrete decision-making can be thought in different ways. First in the case of a comparative multi-attribute decision process, such as the determination of a transportation corridor, the identification of territories on which the evaluation may be flawed (i.e. has a poor relative robustness) could allow a more refined focus on these and a corresponding revision of datasets or an adapted revision of weights. In any case the overall decision-making process should be made more reliable. A second direction lays in the spirit of the real application we have proposed, i.e. the sensitivity of evaluation to various parameters such as missing data. If a decision appears as reliable because data have few missing points, but the evaluation is very sensitive to it, one will be more careful in the interpretation of results and taking the final decision. Further work and testing will however be needed to understand framework behavior in different contexts and be able to pilot its application in various real situations.
}{
L'application de notre méthode à des situations concrètes de prise de décision peu être pensée de différentes manières. Tout d'abord dans le cas d'un processus multi-attributs à but comparatif, comme la détermination d'un corridor pour une nouvelle infrastructure de transport, l'identification des territoires sur lesquels l'évaluation pourrait être biaisée (i.e. avec une mauvaise robustesse relative) devrait permettre une attention particulière pour ceux-ci, et l'adaptation des jeux de données ou la révision des points en conséquence. Dans tous les cas le processus total devrait être plus fiable. Une autre possibilité ressemble à l'application réelle que nous avons développé, i.e. la sensibilité de l'évaluation à divers paramètres comme les données manquantes. Si une décision parait fiable car la taille de données est grande, mais que l'évaluation est très sensible à la suppression de données, il faudra être prudent pour l'interprétation des résultats et pour la prise de décision finale. Un travail approfondi et de test sera cependant nécessaire pour comprendre le comportement du cadre dans différents contextes et pouvoir piloter son application dans des situations réelles diverses.
}


\paragraph{Integration Within Existing Frameworks}


\bpar{
The applicability of the method on real cases will directly depend on its potential integration within existing framework. Beyond technical difficulties that will surely appear when trying to couple or integrate implementations, more theoretical obstacles could occur, such as fuzzy formulations of functions or data types, consistency issues in databases, etc. Such multi-criteria framework are numerous. Further interesting work would be to attempt integration into an open one, such as e.g. the one described in~\cite{tivadar2014oasis} which calculates various indices of urban segregation, as we have already illustrated the application on metropolitan segregation indexes.
}{
L'applicabilité de la méthode à des cas réels dépendra directement de son intégration potentielle dans des environnements existants. Au delà des difficultés techniques qui apparaissent nécessairement en essayant de coupler ou d'intégrer des implémentations existantes, des obstacles plus théoriques pourraient émerger, comme des formulations floues des fonctions ou des types de données, la cohérence des bases de données, etc. De tels cadres multi-critères sont nombreux. Un développement possible serait l'intégration dans un cadre open-source, comme par exemple celui décrit dans~\cite{tivadar2014oasis} qui calcule divers indices de ségrégation urbaine, comme on l'a déjà illustré pour l'application à la ségrégation métropolitaine.
}

\paragraph{Availability of Raw Data}


\bpar{
In general, sensitive data such as transportation questionnaires, or very fine granularity census data are not openly available but provided already aggregated at a certain level (for instance French Insee Data are publicly available at basic statistical unit level or larger areas depending on variables and minimal population constraints, more precise data is under restricted access). It means that applying the framework may imply complicated data research procedure, its advantage to be flexible being thus reduced through additional constraints.
}{
De manière générale, des données sensibles comme des questionnaires de transport, ou des données de sondage à granularité très fine, ne sont pas disponibles de manière ouverte, mais fournis de manière déjà agrégée à un certain niveau (comme par exemple les données françaises de l'Insee sont disponibles publiquement au niveau des unités statistiques élémentaires ou pour des zones plus grandes selon les variables et des contraintes de population minimale, les données plus précises étant à accès restreint). Cela signifie que l'application de notre cadre peut impliquer une procédure de recherche de données laborieuse, l'avantage d'être flexible étant alors compensé par ces contraintes additionnelles.
}


%%%%%%%%%%%%%%%%
\subsection{Validity of Theoretical Assumptions}


\bpar{
A possible limitation of our approach is the validity of the assumption formulating indicators as spatial integrals. Indeed, many socio-economic indicators are not necessarily depending explicitly on space, and trying to associate them with spatial coordinates may become a slippery slope (e.g. associate individual economic variables with individual residential coordinates will have a sense only if the use of the variable has a relation with space, otherwise it is a non-legitimate artifact). Even indicators which have a spatial value may derive from non-spatial variables, as~\cite{kwan1998space} points out concerning accessibility, when opposing integrated accessibility measures with individual-based non necessarily spatial-based (e.g. individual decisions) measures. Constraining a theoretical representation of a system to fit a framework by changing some of its ontological properties (always in the sense of real meaning of objects) can be understood as a violation of a fundamental rule of modeling and simulation in social science given in~\cite{banos2013HDR}, that is that there can be an universal ``language'' for modeling and some can not express some systems, having for consequence misleading conclusion due to ontology breaking in the case of an over-constrained formulation.
}{
Une limitation possible de notre approche est la validité de l'hypothèse qui formule les indicateurs comme des intégrales spatiales. En fait, de nombreux indicateurs socio-économiques ne dépendent pas nécessairement directement de l'espace, et essayer de les associer à des coordonnées peut entraîner sur une pente glissante (par exemple, associer des variables économiques individuelles à des coordonnées résidentielles aura un sens seulement si la variable à une relation à l'espace, autrement un devient un artefact superflu). Même des indicateurs qui ont une valeur spatiale peuvent dériver de variables non-spatiales, comme~ \cite{kwan1998space} le souligne au sujet de l'accessibilité, en opposant les mesures d'accessibilité intégrée aux mesures individu-centrées mais pas forcément basée sur l'espace (comme par exemple des décisions individuelles). Contraindre une représentation théorique d'un système pour le faire rentrer dans un cadre en changeant certaines de ses propriétés ontologiques (toujours dans le sens de la signification réelle des objets) peut être compris comme une violation d'une des règles pour la modélisation et la simulation en sciences sociales données par~\cite{banos2013HDR}, car cela impliquerait qu'il pourrait exister un langage universel pour la modélisation, malgré qu'il ne puisse retranscrire certains systèmes, ayant pour conséquences des conclusions errantes à cause d'une rupture d'ontologie dans le cas d'une formulation sur-contrainte.
}

%%%%%%%%%%%%%%%%
\subsection{Framework Generality}


\bpar{
We argue that the fundamental advantage of the proposed framework is its generality and flexibility, since robustness of the evaluations are obtained only through data structure if ones relaxes constraints on the value of weight. Further work should go towards a more general formulation, suppressing for example the linear aggregation assumption. Non-linear aggregation functions would require however to present particular properties regarding integral inequalities. For example, similar results could search in the direction of integral inequalities for Lipschitzian functions such as the one-dimensional results of~\cite{dragomir1999ostrowski}.
}{
Nous soutenons qu'un des avantages fondamentaux de notre cadre est sa généralité et sa flexibilité, puisque la robustesse des évaluations est obtenue seulement par la structure des données si l'on relaxe les hypothèses sur les valeurs des poids. Des approfondissement pourraient inclure une formulation plus générale, en supprimant par exemple l'hypothèse d'agrégation linéaire. Des fonctions d'agrégation non-linéaires demanderaient toutefois de vérifier certaines propriétés regardant les inégalités intégrales. Par exemple, des résultats similaires pourraient être obtenus en s'orientant vers des inégalités intégrales pour fonctions Lipschitziennes, comme les résultats en une dimension de~\cite{dragomir1999ostrowski}.
}


%%%%%%%%%%%%%%%%
%% Conclusion
%%%%%%%%%%%%%%%%
\section*{Conclusion}{Conclusion}


\bpar{
We have proposed a model-independent framework to compare the robustness of multi-attribute evaluations between different urban systems. Based on data discrepancy, it provide a general definition of relative robustness without any assumption on model for the system, but with limiting assumptions that are the need of linear aggregation and of indicators being expressed through spatial kernel integrals. We propose a toy implementation based on real data for the city of Paris, numerical results confirming general expected behavior, and an implementation on real data for income segregation on Greater Paris metropolitan areas, giving possible insights into concrete policy questions. Further work should be oriented towards sensitivity analysis of the method, application to other real cases and theoretical assumptions relaxation, i.e. the relaxation of linear aggregation and spatial integration.
}{
Nous avons proposé un cadre indépendant du modèle pour comparer la robustesse d'évaluations multi-attributs entre différents systèmes urbains. A partir de la discrépance des données, on fournit une définition générale de la robustesse relative sans aucune hypothèse de modèle pour le système, mais en supposant une agrégation linéaire des objectifs et des indicateurs exprimés comme des intégrales à noyaux. Nous proposons une première implémentation preuve de concept pour la ville de Paris pour laquelle les résultats numériques confirment la tendance générale attendue, et une implémentation sur des données réelles pour la ségrégation de revenus pour la région métropolitaine du Grand Paris, fournissant des réponses possibles à des questions de prise de décision plus concrètes. Des développements possibles peuvent inclure une analyse de sensibilité de la méthode, des applications à d'autres cas réels et une relaxation des hypothèses théoriques, c'est à dire de l'agrégation linéaire et de l'intégration spatiale. % TODO confusion spatial/ kernel ? -> idem à l'oral ?
}


%\section*{Acknowledgments}

%The author would like to thank Julien Keutchayan (Ecole Polytechnique de Montr{\'e}al) for suggesting the original idea of using discrepancy, and anonymous reviewers for the useful comments and insights.



