


% Chapter 

\chapter{Methodological Developments} % Chapter title

\label{ch:methodology} % For referencing the chapter elsewhere, use \autoref{ch:name} 

%----------------------------------------------------------------------------------------

\headercit{We are now building a rigorous Science of Cities, contrarily to what was done before.}{Marc Barth{\'e}l{\'e}my}{}

\bigskip

Such a shocking phrase was pronounced during the introduction of a \emph{Network} course for students of Complex System Science. Besides the fact that the spirit of CSS is precisely the opposite, {\ie} the construction of integrative disciplines (vertical integration that is necessarily founded on the existing body of knowledge of concerned fields) that answer transversal questions (horizontal integration that imply interdisciplinarity) - see {\eg} the roadmap for CS~\cite{2009arXiv0907.2221B}, it reveals how methodological considerations shape the perceptions of disciplines. From a background in Physics, ``rigorous'' implies the use of tools and methods judged more rigorous (analytical derivations, large datasets statistics, etc.). But what is rigorous for someone will not be for an other discipline\footnote{a funny but sad anecdote told by a friend comes to mind : defending his PhD in statistics, he was told at the end by economists how they were impressed by the mathematical rigor of his work, whereas a mathematician judged that ``he could have done everything on the back of an enveloppe''.}, depending on the purpose of each piece of research (perspectivism~\cite{giere2010scientific} poses the \emph{model}, that includes methods, as the articulating core of research entreprises). Thus the full role of methodology aside and not beside theory and experiments. We go in this chapter into various methodological developments which may be precisely used later or contribute to the global background.

We first propose a kind of essay insisting on the importance of reproducibility in science. More than a guideline, it is a way to practice science that a necessary condition for its rigor. Any non-reproducible work is not scientific. We then derive technical results on models of urban growth and on the sensitivity of scaling laws, that are both recurrent themes in the modeling of complex urban systems. We then introduce a method in the context of systematic model exploration and model behavior. We finally work on a link between static and dynamic correlations in a geographical system. This chapter is rather heteroclite as sections may correspond to a particular technical need at a point in the thesis, to global methodological directions, or global research directions.



%----------------------------------------------------------------------------------------

\newpage

% Section : Reproducibility


\section{Reproducibility}




The strength of science comes from the cumulative and collective nature of research, as progresses are made as Newton said ``standing on the shoulder of giants'', meaning that the scientific enterprise at a given time relies on all the work done before and that advances would not be possible without constructing on it. It includes development of new theories, but also extension, testing or falsifiability of previous ones. In that context 





As scientific reproducibility is an essential requirement for any study, its practice seems to be increasing~\cite{stodden2010scientific} and technical means to achieve it are always more developed (as e.g. ways to make data openly available, or to be transparent on the research process such as \texttt{git}~\cite{ram2013git}, or to integrate document creation and data analysis such as \texttt{knitr}~\cite{xie2013knitr}), at least in the field of numerical modeling and simulation. However, the devil is indeed in the details and obstacles judged at first sight as minor become rapidly a burden for reproducing and using results obtained in some previous researches. We describe two cases studies where models of simulation are apparently highly reproducible but unveil as puzzles on which research-time balance is significantly under zero, in the sense that trying to exploit their results may cost more time than developing from scratch similar models.






%%%%%%%%%%%%%%%%%%%%%%%%%%%%%%%%%%
\subsection{On the Need to Explicit the Model}

A current myth is that providing entire source code and data will be a sufficient condition for reproducibility. It will work if the objective is to produce exactly same plots or statistical analysis, assuming that code provided is the one which was indeed used to produce the given results. It is however not the nature of reproducibility. First, results must be as much implementation-independent as possible for clear robustness purposes. Then, in relation with the precedent point, one of the purposes of reproducibility is the reuse of methods or results as basis or modules for further research (what includes implementation in another language or adaptation of the method), in the sense that reproducibility is not replicability as it must be adaptable~\cite{drummond2009replicability}.

Our first case study fits exactly that scheme, as it was undoubtedly aimed to be shared with and used by the community since it is a model of simulation provided with the Agent-Based simulation platform NetLogo~\cite{wilensky1999netlogo}. The model is also available online~\cite{de2007netlogo} and is presented as a tool to simulate socio-economic dynamics of low-income residents in a city based on a synthetic urban environment, generated to be close in stylized facts from the real town of Tijuana, Mexico. Beside providing the source code, the model appears to be poorly documented in the literature or in comments and description of the implementation. Comments made thereafter are based on the study of the urban morphogenesis part of the model (setup for the ``residential dynamics'' component) as it is our global context of study~\cite{raimbault2014vers}. In the frame of that study, source code was modified and commented, which last version is available on the repository of the project\footnote{at \texttt{https://github.com/JusteRaimbault/CityNetwork/tree/master/Models/Reproduction/UrbanSuite}}.




\paragraph{Rigorous Formalization}

An obvious part of model construction is its rigorous formalization in a formal framework distinct from source code. There is of course no universal language to formulate it~\cite{banos2013pour}, and many possibilities are offered by various fields (e.g. UML, DEVS, pure mathematical formulation). No paper nor documentation is provided with the model, apart from the embedded NetLogo documentation since it only thematically describes in natural language the ideas behind each step without developing more and provides information about role of different elements of the interface.

This formulation is a key for it to be understood, reproduced and adapted ; but it also avoids implementation biases such as
\begin{itemize}
\item Architecturally dangerous elements : in the model, world context is a torus and agents may ``jump'' in the euclidian representation, what is not acceptable for a 2D projection of real world. To avoid that, many tricky tests and functions were used, including unadvised practices (e.g. dead of agents based on position to avoid them jumping).
\item Lack of internal consistence : the example of the patch variable \texttt{land-value} used to represent different geographical quantities at different steps of the model (morphogenesis and residential dynamics), what becomes an internal inconsistence when both steps are coupled when option \texttt{city-growth?} is activated.
\item Coding errors : in an untyped language such as NetLogo, mixing types may conduct to unexpected runtime errors, what is the case of the patch variable \texttt{transport} in the model (although no error occurs in most of run configurations from the interface, what is more dangerous as the developer thinks implementation is secure). Such problems should be avoided if implementation is done from an exact formal description of the model.
\end{itemize}


\paragraph{Transparent Implementation}

A totally transparent implementation is expected, including ergonomics in architecture and coding, but 

\paragraph{Expected Model Behavior}

Whatever the definition, a model can not be reduced to its formulation and/or implementation, as expected model behavior or model usage can be viewed as being part of the model itself. In the frame of \noun{Giere}'s perspectivism~\cite{giere2010scientific}, the definition of model includes the purpose of use but also the agent who aims to use it. Therefore a minimal explication of model behavior and exploration of parameter roles is highly advised to decrease chances of misuses or misinterpretations of it. It includes simple runtime charts that are immediate on the NetLogo platform, but also indicators computations to evaluate outputs of the model. It can also be improved visualizations during runtime and model exploration, such as showed in Fig.~\ref{fig:example_tij_viz}.

\begin{figure}
\centering

\hspace{-2cm}
\includegraphics[width=0.33\textwidth]{Figures/PartI/Methodology/Reproducibility/stdView}
\hfill
\includegraphics[width=0.33\textwidth]{Figures/PartI/Methodology/Reproducibility/ViewRoads}
\hfill
\includegraphics[width=0.33\textwidth]{Figures/PartI/Methodology/Reproducibility/landValues_cityFinished}
\caption[Reproducibility and visualization]{Example of simple improvement in visualization that can help understanding mechanisms implied in the model. \textit{Left : } example of original output ; \textit{Middle : } visualization of main roads (in red) and underlying patches attribution, suggesting possible implementation bias in the use of discretized trace of roads to track their positions ; \textit{Right : }Visualization of land values using a more readable color gradient. This step confirms the hypothesis, through the form of value distribution, that the morphogenesis step is an unnecessary detour to generate a random field for which simple diffusion method should provide similar results, as detailed in the paragraph on implementation.}
\label{fig:example_tij_viz}
\end{figure}




\subsection{On the Need of Exactitude in Model Implementation}

% Barthelemy paper, pb in model description/implementation
%  - test different analyses with possible biaises -

Possible divergences between model description in a paper and the effectively implemented processes may have grave consequences on the reproducibility of science. The road network growth model given in~\cite{barthelemy2008modeling} is one example that we are currently investigating. A strict implementation of model mechanisms provide slightly different results than the one presented in the paper, and as source code is not provided we need to test different hypotheses on possible mechanisms added by the programmer (that seems to be a connexion rule to intersections under a certain distance threshold). Lessons that could be possibly drawn from this examples are 
\begin{itemize}
\item the necessity of providing source code
\item the necessity of providing architecture description along with code (if model description is in a langage too far from architectural specifications) in order to identify possible implementation biaises
\item the necessity of performing and detailing explicitly model explorations, that would in that case have helped to identify the implementation bias.
\end{itemize}

The last point, if first not provided, may ensure a limited risk of scientific falsification as it may be more complicated to fake false exploration results than to effectively explore the model. A joint project currently done is the writing of a false modeling paper in the spirit of~\cite{zilsel2015canular}, in which opposite results to the effective results of a given model are provided, without providing model implementation. A first bunch of test is the acceptance of a clearly non-reproducible paper in diverse journals, possibly with a control on textual elements (using or not ``buzz-words'' associated to the journal, etc.). Depending on results, a second experiment may be tested with providing open source code for model implementation but still with false results, to verify if reviewers effectively try to reproduce results when they pretend to want the code (in reasonable computational power limits of course, HPC being not currently broadly available in Humanities).



\subsection{Perspectives}


Again, reproducibility and transparency is a non-negotiable feature of contemporaneous science, along with Open practices and Open Access. Too much examples (see a very recent one in experimental economics~\cite{camerer2016evaluating}) show in various disciplines the lack of reproducibility of experiments, that is a falsification of previous results or a result in itself. Falsification is a costly practice, and even if necessary~\cite{chavalarias2005nobel}, could be made more efficient through more transparency and direct reproducibility, increase therein the global workflow of science. We develop in parallel of this thesis various tools aimed to ease reproducibility, for which an overview is given in appendix~\ref{app:workflow}.



%----------------------------------------------------------------------------------------

% Section : stochastic framework for urban growth


\newpage

\section{An unified framework for stochastic models of urban growth}

Urban growth modeling fall in the case of tentatives to find self-consistent rules reproducing dynamics of an urban system, and thus in our logic of system morphogenesis. We examine here methodological issues linked to different frameworks of urban growth.

%%%%%%%%%%%%%%%%%%%%
\subsection{Introduction}
%%%%%%%%%%%%%%%%%%%%

Various stochastic models aiming to reproduce population patterns on large temporal and spatial scales (city systems) have been discussed across various fields of the literature, from economics to geography, including models proposed by physicists. We propose here a general framework that allows to include different famous models (in particular Gibrat, Simon and Preferential Attachment model) within an unified vision. It brings first an insight into epistemological debates on the relevance of models. Furthermore, bridges between models lead to the possible transfer of analytical results to some models that are not directly tractable.


%%%%%%%%%%%%%%%%%%%%
%\subsubsection{Context}

Seminal models of urban growth are Simon~\cite{simon1955class} (later generalized as e.g. \cite{haran1973modified}) and Gibrat models. Many examples can be given across disciplines. \cite{benguigui2007dynamic} give an equation-based dynamical model, whereas \cite{gabaix1999zipf} solves a stationary model. \cite{Gabaix20042341} reviews urban growth approaches in economics. A model adapted from evolutive urban theory is solved in~\cite{favaro2011gibrat} and improves Gibrat models. The question of empirical scales at which it is consistent to study urban growth was also tackled in the particular case of France~\cite{bretagnolle2002time}. We stay to a certain level of tractability to include models as essence of our approach is links between models but do not make ontologic assumptions.


%%%%%%%%%%%%%%%%%%%%
%\subsubsection{Notations}



%%%%%%%%%%%%%%%%%%%%
\subsection{Framework}
%%%%%%%%%%%%%%%%%%%%


%%%%%%%%%%%%%%%%%%%%
%\subsubsection{Formulation}

\paragraph{Presentation}
What we propose as a framework can be understood as a meta-model in the sense of~\cite{cottineau2015incremental}, i.e. an modular general modeling process within each model can be understood as a limit case or as a specific case of another model. More simply it should be a diagram of formal relations between models. The ontological aspect is also tackled by embedding the diagram into an ontological state space (which discretization corresponds to the ``bricks'' of the incremental construction of~\cite{cottineau2015incremental}). It constructs a sort of model classification or modelography.

We are still at the stage of different derivations of links between models that are presented hereafter.

%\subsubsection{Models Included}

%The following models are included in our framework. The list is arbitrary but aims to offer a broad view of disciplines concerned

%\subsubsection{Thematic Classification}


%\subsubsection{Framework Formulation}
%Diagram linking various models ; first embedded into time/population plane, cases Discrete/Continous. Other aspects more sparse (ex. spatialization) ; how represent it ?

%%%%%%%%%%%%%%%%%%%%
%\subsection{Models formulation}



%%%%%%%%%%%%%%%%%%%%
\subsection{Derivations}

\subsubsection{Generalization of Preferential Attachment}

\cite{yamasaki2006preferential} give a generalization of the classical Preferential Attachment Network Growth model, as a birth and death model with evolving entities. More precisely, network units gain and lose population (equivalent to links connexions) at fixed probabilities, and new unit can be created at a fixed rate.

\subsubsection{Link between Gibrat and Preferential Attachment Models}


\bpar{
Let consider a strictly positive growth Gibrat model given by $P_i(t)=R_i(t)\cdot P_{i}(t-1)$ with $R_i(t)>1$, $\mu_i(t)=\Eb{R_i(t)}$ and $\sigma_i(t)=\Eb{R_i(t)^2}$. On the other hand, we take a simple preferential attachment, with fixed attachment probability $\lambda \in [0,1]$ and new arrivants number $m>0$. We derive that Gibrat model can be statistically equivalent to a limit of the preferential attachment model, assuming that the moment-generating function of $R_i(t)$ exists. Classical distributions that could be used in that case, e.g. log-normal distribution, are entirely defined by two first moments, making this assumption reasonable.
}{
Considérons un modèle de croissance strictement positive de Gibrat donnée par $P_i(t)=R_i(t)\cdot P_{i}(t-1)$ avec $R_i(t)>1$, $\mu_i(t)=\Eb{R_i(t)}$ et $\sigma_i(t)=\Eb{R_i(t)^2}$. D'autre part, soit un modèle simple d'attachement préférentiel, avec une probabilité d'attachement $\lambda \in [0,1]$ et un nombre de nouveau arrivants $m>0$. Il est possible de dériver que le Gibrat est statistiquement équivalent à une limite de l'attachement préférentiel, sous l'hypothèse que toutes les fonctions génératrices des moments de $R_i(t)$ existent. Les distributions classiques qui peuvent être utilisées dans ce cas, e.g. une distribution normale ou log-normale, sont entièrement déterminées par leur deux premiers moments, ce qui rend cette hypothèse raisonnable.
}



\begin{lemma}
The limit of a Preferential Attachment model when $\lambda \ll 1$ is a linear-growth Gibrat model, with limit parameters $\mu_i(t)=1+\frac{\lambda}{m\cdot (t-1)}$.
\end{lemma}


\bpar{
The proof is given in Appendix~\ref{app:technical}.
}{
La preuve est donnée en Annexe~\ref{app:technical}.
}


\subsubsection{Link between Simon and Preferential Attachment}
%\label{subsubsec:gibrat-simon}

A rewriting of Simon model yields a particular case of the generalized preferential attachment, in particular by vanishing death probability.

\subsubsection{Link between Favaro-Pumain and Gibrat}

\cite{favaro2011gibrat} generalizes Gibrat models with innovation propagation dynamics, being therefore a generalization of that model. Theoretically, a process-based model equivalent to the Favaro-pumain should then fill the missing case in model classification at the corresponding discretization. Simpop models do not fill that case as they stay at the scale of city systems, as for Marius models~\cite{cottineau2014evolution}. These must also have their counterparts in discrete microscopic formulation.

\subsubsection{Link between Bettencourt-West and Pumain}

We are considering to study Bettencourt-West model for urban scaling laws \cite{bettencourt2008large} as entering the stochastic urban growth framework as stationary component of a random growth model, but investigation are still ongoing.


\subsubsection{Other Models}

\cite{gabaix1999zipf} develops an economic model giving a Simon equivalent formulation. They in particular find out that in upper tail, proportional growth process occurs. We find the same result as a consequence of the derivation of the link between Gibrat and Preferential attachment models.




%%%%%%%%%%%%%%%%%%%%
%\subsection{Application and Perspectives}
%%%%%%%%%%%%%%%%%%%%







%%%%%%%%%%%%%%%%%%%%
%\subsection{Discussion}
%%%%%%%%%%%%%%%%%%%%

%%%%%%%%%%%%%%%%%%%%
%\subsection*{Conclusion}
%%%%%%%%%%%%%%%%%%%%





%----------------------------------------------------------------------------------------

\newpage


%  section : scaling sensitivity : useful ?


\section{Analytical Sensitivity of Urban Scaling Laws to Spatial Extent}

At the center of evolutive urban theory are hierarchy and associated scaling laws. We begin here an methodological investigation on the sensitivity of scaling laws to city definition.


%%%%%%%%%%%%%%%%%%%%
\subsection{Introduction}



Scaling laws have been shown to be universal of urban systems at many scales and for many indicators. Recent studies question however the consistence of scaling exponents determination, as their value can vary significantly depending on thresholds used to define urban entities on which quantities are integrated, even crossing the qualitative border of linear scaling, from infra-linear to supra-linear scaling. We use a simple theoretical model of spatial distribution of densities and urban functions to show analytically that such behavior can be derived as a consequence of the type of spatial distribution and the method used. Numerical simulation confirm the theoretical results and reveals that results are reasonably independent of spatial kernel used to distribute density.



Scaling laws for urban systems, starting from the well-known rank-size Zipf's law for city size distribution~\cite{gabaix1999zipf}, have been shown to be a recurrent feature of urban systems, at many scales and for many types of indicators. They reside in the empirical constatation that indicators computed on elements of an urban system, that can be cities for system of cities, but also smaller entities at a smaller scale, do fit relatively well a power-law distribution as a function of entity size, i.e. that for entity $i$ with population $P_i$, we have for an integrated quantity $A_i$, the relation $A_i \simeq A_0\cdot \left(\frac{P_i}{P_0}\right)^{\alpha}$. Scaling exponent $\alpha$ can be smaller or greater than 1, leading to infra- or supra-linear effects. Various thematic interpretation of this phenomena have been proposed, typically under the form of processes analysis. The economic literature has produced abundant work on the subject (see~\cite{Gabaix20042341} for a review), but that are generally weakly spatial, thus of poor interest to our approach that deals precisely with spatial organization. Simple economic rules such as energetic equilibria can lead to simple power-laws~\cite{bettencourt2008large} but are difficult to fit empirically. A interesting proposition by Pumain is that they are intrinsically due to the evolutionary character of city systems, where complex emergent interaction between cities generate such global distributions~\cite{pumain2006evolutionary}. Although a tempting parallel can be done with self-organizing biological systems, Pumain insists on the fact that the ergodicity assumption for such systems is not reasonable in the case of geographical systems and that the analogy cannot be exploited~\cite{pumain2012urban}. Other explanations have been proposed at other scales, such as the urban growth model at the mesoscopic scale (city scale) given in~\cite{2014arXiv1401.8200L} that shows that the congestion within transportation networks may be one reason for city shapes and corresponding scaling laws. Note that ``classic'' urban growth models such as Gibrat's model do provide first order approximation of scaling systems, but that interactions between agents have to be incorporated into the model to obtain better fit on real data, such as the Favaro-Pumain model for innovation cycles propagation proposed in~\cite{favaro2011gibrat}, that generalize a Gibrat model and provide better fits on data for French cities.

However, the blind application of scaling exponents computations was recently pointed as misleading in most cases~\cite{louf2014scaling}, confirmed by empirical works such as~\cite{2013arXiv1301.1674A} that showed the variability of computed exponents to the parameters defining urban areas, such as density thresholds. An ongoing work by Cottineau \& \textit{al.} presented at~\cite{cottineau2015scaling}, studies empirically for French Cities the influence of 3 parameters playing a role in city definition, that are a density threshold $\theta$ to delimitate boundaries of an urban area, a number of commuters threshold $\theta_c$ that is the proportion of commuters going to core area over which the unity is considered belonging to the area, and a cut-off parameter $P_c$ under which entities are not taken into account for the linear regression providing the scaling exponent. Remarquable results are that exponents can significantly vary and move from infra-linear to supra-linear when threshold varies. A systematic exploration of parameter space produces phase diagrams of exponents for various quantities. One question raising immediately is how these variation can be explained by the features of spatial distribution of variables. Do they result from intrinsic mechanisms present in the system or can they be explained more simply by the fact that the system is particularly spatialized ? We prove on a toy analytical model that even simple distributions can lead to such significant variations in the exponents, along one dimension of parameters (density threshold), directing the response towards the second explanation.
%The rest of the section is organized as follows : we formalize the simple framework used and derive an analytical relation between estimated exponent and density threshold parameter. We then present a numerical implementation of the model that confirms numerically theoretical results, explore other form of kernels that would be less tractable, and study the sensitivity along two parameters. We finally discuss the implications of our results and further work needed.

%%%%%%%%%%%%%%%%%%%%
\subsection{Formalization}
\label{sec:formalization}

We formalize the simple theoretical context in which we will derive the sensitivity of scaling to city definition. Let consider a polycentric city system, which spatial density distributions can be reasonably constructed as the superposition of monocentric fast-decreasing spatial kernels, such as an exponential mixture model~\cite{anas1998urban}. Taking a geographical space as $\mathbb{R}^2$, we take for any $\vec{x}\in\mathbb{R}^2$ the density of population as
\begin{equation}
d(\vec{x}) = \sum_{i=1}^{N}{d_i(\vec{x})} = \sum_{i=1}^{N}{d_i^0\cdot \exp{\left(\frac{-\norm{\vec{x}-\vec{x}_i}}{r_i}\right)}}
\end{equation}

where $r_i$ are spread parameters of kernels, $d_i^0$ densities at origins, $\vec{x}_i$ positions of centers. We furthermore assume the following constraints :

\begin{enumerate}
\item To simplify, cities are monocentric, in the sense that for all $i\neq j$, we have $\norm{\vec{x}_i - \vec{x}_j}\gg r_i$.
\item It allows to impose structural scaling in the urban system by the simple constraint on city populations $P_i$. One can compute by integration that $P_i=2\pi d_i^0 r_i^2$, what gives by injection into the scaling hypothesis $\ln{P_i}=\ln{P_{max}}-\alpha \ln{i}$, the following relation between parameters : $\ln{\left[d_i^0 r_i^2\right]}=K' - \alpha \ln{i}$.
\end{enumerate}

To study scaling relations, we consider a random scalar spatial variable $a(\vec{x})$ representing one aspect of the city, that can be everything but has the dimension of a spatial density, such that the indicator $A(D)=\Eb{\iint_D{a(\vec{x})d\vec{x}}}$ represents the expected quantity of $a$ in area $D$. We make the assumption that $a\in \{0;1\}$ (``counting'' indicator) and that its law is given by $\Pb{a(\vec{x})=1}=f(d(\vec{x}))$. Following the empirical work done in~\cite{cottineau2015scaling}, the integrated indicator on city $i$ as a function of $\theta$ is given by
\[
A_i(\theta) = A(D(\vec{x}_i, \theta))
\]

where $D(\vec{x}_i, \theta)$ is the area centered in $\vec{x}_i$ where $d(\vec{x})>\theta$. Assumption 1 ensures that the area are roughly disjoint circles. We take furthermore a simple amenity such that it follows a local scaling law in the sense that $f(d)=\lambda\cdot d^\beta$. It seems a reasonable assumption since it was shown that many urban variable follow a fractal behavior at the intra-urban scale~\cite{keersmaecker2003using} and that it implies necessarily a power-law distribution~\cite{chen2010characterizing}. We make the additional assumption that $r_i=r_0$ does not depend on $i$, what is reasonable if the urban system is considered from a large scale. This assumption should be relaxed in numerical simulations. The estimated scaling exponent $\alpha(\theta)$ is then the result of the log-regression of $(A_i(\theta))_i$ against $(P_i(\theta))_i$ where $P_i(\theta)=\iint_{D(\vec{x}_i,\theta)}{d}$.


%%%%%%%%%%%%%%%%%%%%
\subsection{Analytical Derivation of Sensitivity}

With above notations, let derive the expression of estimated exponent for quantity $a$ as a function of density threshold parameter $\theta$. The quantity computed for a given city $i$ is, thanks to the monocentric assumption and in a spatial range and a range for $\theta$ such that $\theta \gg \sum_{j\neq i}{d_j(\vec{x})}$, allowing to approximate $d(\vec{x})\simeq d_i(\vec{x})$ on $D(\vec{x}_i,\theta)$, is computed by
\[
\begin{split}
A_i(\theta) & = \lambda\cdot \iint_{D(\vec{x}_i,\theta)}{d^\beta} = 2\pi\lambda {d_i^0}^{\beta} \int_{r=0}^{r_0 \ln{\frac{d_i^0}{\theta}}}{r\exp{\left(-\frac{r\beta}{r_0}\right)}dr}\\
& = \frac{2\pi {d_i^0}^\beta r_0^2}{\beta^2} \left[1 + \beta \ln{\frac{\theta}{d_i^0}\left(\frac{\theta}{d_i^0}\right)^\beta} - \left(\frac{\theta}{d_i^0}\right)^\beta\right]
\end{split}
\]

We obtain in a similar way the expression of $P_i(\theta)$
\[
P_i(\theta) = 2\pi d_i^0 r_0^2 \left[1 + \ln{\left[\frac{\theta}{d_i^0}\right]}\frac{\theta}{d_i^0} - \frac{\theta}{d_i^0}\right]
\]

The Ordinary-Least-Square estimation, solving the problem $\inf_{\alpha,C}\norm{(\ln{A_i(\theta)} - C - \alpha \ln{P_i(\theta)})_i}^2$, gives the value $\alpha(\theta) = \frac{\Covb{(\ln{A_i(\theta)})_i}{(\ln{P_i(\theta)})_i}}{\Varb{(\ln{P_i(\theta)})_i}}$. As we work on city boundaries, threshold is expected to be significantly smaller than center density, i.e. $\theta / d_i^0 \ll 1$. We can develop the expression in the first order of $\theta / d_i^0$ and use the global scaling law for city sizes, what gives $\ln{A_i(\theta)} \simeq K_A - \alpha \ln{i} + (\beta - 1)\ln{d_i^0} + \beta \ln{\frac{\theta}{d_i^0}\left(\frac{\theta}{d_i^0}\right)^\beta} $ and $\ln{P_i(\theta)} = K_P - \alpha \ln{i} + \ln{\left[\frac{\theta}{d_i^0}\right]}\frac{\theta}{d_i^0}$. Developing the covariance and variance gives finally an expression of the scaling exponent as a function of $\theta$, where $k_j,{k_j}'$ are constants obtained in the development :

\begin{equation}
\label{eq:th}
\alpha(\theta) = \frac{k_0 + k_1 \theta + k_2 \theta^\beta + k_3 \theta^{\beta + 1} +  k_4 \theta \ln{\theta} + k_5 \theta^\beta \ln{\theta} + k_6 \theta^\beta (\ln{\theta})^2 + k_7 \theta^{\beta + 1}(\ln{\theta})^2 + k_8 \theta^{\beta + 1}\ln{\theta}}{k_0'+k_1' \ln{\theta} + k_2' \theta \ln{\theta} + k_3' \theta^2 + k_4' \theta^2\ln{\theta} + k_5' \theta^2 (\ln{\theta})^2}
\end{equation}

This rational fraction predicts the evolution of the scaling exponent when the threshold varies. We study numerically its behavior in the next section, among other numerical experiments.


%%%%%%%%%%%%%%%%%%%%
\subsection{Numerical Simulations}

\paragraph{Implementation}

We implement empirically the density model given in section~\ref{sec:formalization}. Centers are successively chosen such that in a given region of space only one kernel dominates in the sense that the sum of other contributions are above a given threshold $\theta_e$. In practice, adapting $N$ to world size allows to respect the monocentric condition. Population are distributed in order to follow the scaling law with fixed $\alpha$ and $r_i$ (arbitrary choice) by computing corresponding $d_i^0$. Technical details of the implementation done in R~\cite{R-Core-Team:2015fk} and using the package \texttt{kernlab} for efficient kernel mixture methods~\cite{Karatzoglou:2004uq} are given as comments in source code\footnote{available at \texttt{https://github.com/JusteRaimbault/CityNetwork/tree/master/Models/Scaling}}. We show in figure~\ref{fig:ex-distrib} example of synthetic density distributions on which the numerical study is conducted. The validation of theoretical results on these experimental mixtures must still be conducted, along with sensitivity tests to random perturbations, influence of kernel type, and two-parameters phase diagram when adding in the computational model functional density distribution and associated cut-off threshold.
%Theoretical result obtained in Eq.~\ref{eq:th} are studied and confronted to emprically computed values for various parameter as shown in Fig.~\ref{fig:th_results}.

\begin{figure}
\centering
\includegraphics[width=0.4\textwidth]{Figures/PartI/Methodology/Scaling/example_exp_mixture}
\caption[Synthetic density distribution]{Example of a synthetic density distribution obtained with the exponential mixture, with a grid of size $400\times 400$ and parameters $N=20$, $r_0=10$, $P_{max}=200$, $\alpha=0.5$, $\theta_C = 0.01$.}
\label{fig:ex-distrib}
\end{figure}


%\begin{figure}
%\centering
%
%\caption{Validation of theoretical result through numerical simulation.}
%\label{fig:th_results}
%\end{figure}



\paragraph{Random Perturbations}

The simple model used is quite reducing for maximal densities and radius distribution. We aim to proceed to an empirical study of the influence of noise in the system by fixing $d_i^0$ and $r_i$ the following way :
\begin{itemize}
\item $d_i^0$ follows a reversed log-normal distribution with maximal value being a realistic maximal density
\item Radiuses are computed to respect rank-size law and then perturbed by a white noise.
\end{itemize}

%Results shown in Fig.~\ref{fig:random-density} are quantitatively different from previous one, as expected, but the same qualitative behavior is reproduced.


%\begin{figure}
%\centering

%\caption{Variation of exponents with variable origin density and radius.}
%\label{fig:random-density}
%\end{figure}



\paragraph{Kernel Type}

We shall test the influence of the type of spatial kernel used on results. We can test gaussian kernels and quadratic kernels with parameters within reasonable ranges analog to the exponential kernel. %As shown in Fig.~\ref{fig:other-kernels}, we obtain the same qualitative results that is the significant variation of $\alpha(\theta)$ as a function of $\theta$.


%
%\begin{figure}
%\centering

%\caption{Scaling exponents for other kernels.}
%\label{fig:other-kernels}
%\end{figure}

%\paragraph{Two-parameters phase diagram}

%We introduce now a second spatial variable that has also an influence on the definition of urban entities, that is the proportion of actives working in city center, as done on empirical data in~\cite{cottineau2015scaling}. To simplify, it is used only to define urban parameter but assumed as having no influence on the local probability distribution of the amenity which stays the same function of the density. We write 

%\begin{figure}
%\centering

%\caption{Two parameters phase diagram.}
%\label{fig:two-params}
%\end{figure}

%%%%%%%%%%%%%%%%%%%%
%\subsection{Discussion}

%%%%%%%%%%%%%%%%%%%%
%\subsection{Conclusion}








%----------------------------------------------------------------------------------------


\newpage


%  section : synthetic data control - introduces rochebrune paper, feasible correlation space etc, and forthcoming applications ?


\section{Statistical Control on Initial Conditions by Synthetic Data Generation}

\subsection{Context}

When evaluating data-driven models, or even more simple partially data-driven models involving simplified parametrization, an unavoidable issue is the lack of control on ``underlying system parameters'' (what is a ill-defined notion but should be seen in our sense as parameters governing system dynamics). Indeed, a statistics extracted from running the model on enough different datasets can become strongly biased by the presence of confounding in the underlying real data, as it is impossible to know if result is due to processes the model tries to translate or to a hidden structure common to all data.

%Let illustrate the issue with a simple example.

We formalize briefly a proposition of method that would allow to add controls on meta-parameters, in the sense of parameters driving the represented system at a higher temporal and spatial scale, for a model of simulation. We make the hypothesis that such method is valid under constraints of disjonction for scales and/or ontologies between the model of simulation and the domain of meta-parameters.


\subsection{Description}

An advanced knowledge of the behavior of computational models on their parameter space is a necessary condition for deductions of thematic conclusions or their practical application~\cite{banos2013pour}. But the choice of varying parameters is always subjective, as some may be fixed by a real-world parametrization, or other may be interpreted as arbitrarily fixed initial conditions. It raises methodological and epistemological issues for the sensitivity analysis, as the scope of the model may become ill-defined.

Let consider the concrete example of the Schelling Segregation model~\cite{schelling1971dynamic}. One of its crucial features on which the literature has been rather controversial is the influence of the spatial structure of the container on which agents evolve.%~(\textit{Biblio Marion}). 
 The thematic aim of the project developed in~\cite{cottineau2015revisiting} is to clarify this point through a systematic model exploration. A methodological contribution is the construction of a framework allowing the analysis of the sensitivity of models to \emph{meta-parameters}, i.e. to parameters considered as fixed initial conditions (e.g. the spatial structure for the Schelling model), or to parameters of another model generating an initial configuration%, as detailed in Fig.~\ref{} \textit{[insert scheme describing the approach]}, 
 yielding thus a \emph{simple coupling} between models (serial coupling). The benefits of such an approach are various but include for example the knowledge of model behavior in an extended frame, the possibility of statistical control when regressing model outputs, a finer exploration of model derivatives than with a naive approach. Some remarks can be made on the approach :
\begin{itemize}
\item What knowledge are brought by adding the upstream model, rather than for example in the Schelling case exploring a large set of initial geometries ? 

$\rightarrow$ \textit{to obtain a sufficiently large set of initial configuration, one quickly needs a model to generate them ; in that case a quasi-random generation followed by a filtering on morphological constraint will be a morphogenesis model, which parameters are the ones of the generation and the filtering methods. Furthermore, as detailed further, the determination of the derivative of the downstream model is made possible by the coupling and knowledge of the upstream model.}
\item Statistical noise is added by coupling models

$\rightarrow$ \textit{Repetitions needed for convergence are indeed larger as the final expectance has to be determined by repeating on the first times the second model ; but it is exactly the same as exploring directly many configuration, to obtain statistical robustness in that case one must repeat on similar configurations.}

\item Complexity is added by coupling models

% check Varenne citation
$\rightarrow$ \textit{In the sense of Varenne~\cite{varenne2010framework} , coupling is simple and no complexity is thus added.}
\end{itemize}
 
%\paragraph{Context}

%Let $M_{m}$ a stochastic model of simulation, which inputs are to simplify initial conditions $D_0$ and parameters $\vec{\alpha}$, and output $M_{m}\left[\vec{\alpha},D_0\right](t)$ at a given time $t$. We assume that it is partially data-driven in the sense that $D_0$ is supposed to represent a real situation at a given time, and model performance is measured by the distance of its output at final time to the real situation at the corresponding time, i.e. error function is of the form $\norm{\Eb{\vec{g}(M_{m}\left[\vec{\alpha},D_0\right](t_f))}-\vec{g}(D_f)}$ where $\vec{g}$ is a deterministic field corresponding to given indicators.

%\paragraph{Position of the Problem}

%Evaluating the model on real data is rapidly limited in control possibilities, being restricted to the search of datasets allowing natural control groups. Furthermore, statistical behaviors are generally poorly characterized because of the small number of realizations. Working with synthetic data first allows to solve this issue of robustness of statistics, and then gives possibilities of control on some ``meta-parameters'' in the sense described before.




%%%%%%%%%%%%%%%%%%%%
\subsection{Formal Description}

%\subsubsection{Deterministic Formulation}

One has the composition of the derivative along the meta-parameter

\[
\partial_{\alpha}\left[M_u \circ M_d\right] = \left(\partial_{\alpha} M_u \circ M_d \right)\cdot \partial_{\alpha} M_d
\]

$\rightarrow$ \textit{the sensitivity of the downstream model (Schelling) can be determined by studying the serial coupling and the upstream model ; thematic knowledge : sensitivity to an implicit meta-parameter ; and computational gain : generation of controlled differentiates in the ``initial space'' is quasi impossible.}

%\subsubsection{Stochasticity}

The question of stochasticity in simply coupled models causes no additional issue as $\Eb{X}=\Eb{\Eb{X|Y}}$. It naturally multiplies the number of repetition needed for convergence what is the expected behavior.






%----------------------------------------------------------------------------------------

\newpage




\section[Spatio-temporal Correlations]{Linking dynamic and static spatio-temporal correlations under simplified assumptions}

\label{sec:spatiotempcorrs}

Space and Time are both crucial for the study of geographical systems when aiming to understand \emph{processes} (by definition dynamical~\cite{hypergeo}) evolving in a \emph{spatial structure} in the sense of~\cite{dollfus1975some}. Space is more than coordinates for elements of the system, but a dimension in itself that drives interactions and thus system properties. Reading geographical systems from the point of view of \emph{spatio-temporal processes} emphasizes the fact that \emph{space actually matters}. Space and time are closely linked in such processes, and depending on the underlying mechanisms, one can expect to extract useful information from one on the other : in certain cases that we will investigate in this part, it is for example possible to learn about dynamics from static information. Spatio-temporal correlations approaches, linked to spatio-temporal dynamics, are present in very broad fields such as dynamical image processing (including video compression)~\cite{chalidabhongse1997fast,hansen2004accelerated,ke2007spatio}, target tracking~\cite{belouchrani1997direction,vuran2004spatio}, climate science~\cite{cressie1999classes}, Earth sciences~\cite{ma2002spatio}, city systems dynamics~\cite{hernando2015memory,pigozzi1980interurban}, among others.


The capture of neighborhood effects in statistical models is a wisely used practice in spatial statistics, as the technique of Geographically Weighted Regression illustrates~\cite{brunsdon1998geographically}. A possible interpretation among many definitions of spatial autocorrelation~\cite{griffith1992spatial} yields that by estimating a plausible characteristic distance for spatial correlations or auto-correlations, one can isolate independent effects between variables from effects due to neighborhood interactions\footnote{note that the formal link between models of spatial autocorrelation (see e.g. \cite{griffith2012advanced}) is not clear and should be further investigated}. The study of the spatial covariance structure is a cornerstone of advanced spatial statistics that was early formulated~\cite{griffith1980towards}. We propose now to study possible links between spatial and temporal correlations, using spatio-temporal covariance structure to infer information on dynamical processes.


\subsection{Notations}

We consider a multivariate spatio-temporal stochastic process denoted by $\vec{Y}\left[\vec{x},t\right]$. At a given point $\vec{x}_0$ in space, we can define temporal covariance structure by
\[
\mathbf{C}_t (\vec{x}_0) = \Varb{\vec{Y}\left[\vec{x}_0, \cdot\right]}
\]

and spatial covariance structure at fixed time by
\[
\mathbf{C}_x (t) = \Varb{\vec{Y}\left[\cdot, t\right]}
\]

It is clear that these quantities will be in practice first ill-defined because of the difficulty in interpreting such a process by a spatio-temporal random variable, secondly highly non-stationary in space and time. We stay however at a theoretical level to gain structural knowledge, reviewing simple cases in which a formal link can be established.


\subsection{Wave Equation}

In the case of propagating waves, there is an immediate link. Let assume that a wave equation if verified by ``deterministic'' parts of components

\begin{equation}
c^2 \cdot \partial^2_{t} \bar{Y}_i = \Delta \bar{Y}_i
\end{equation}

with $Y_i = \bar{Y}_i + \varepsilon_i$. If errors are uncorrelated and processes are stationary, we have then directly

\begin{equation}
\mathbf{C}_t \left[ \partial^2_t Y_i , \partial^2_t Y_j \right] = \frac{1}{c^2} \cdot\mathbf{C}_x \left[ \Delta Y_i , \Delta Y_j \right]
\end{equation}

This gives us however few insight on real systems as local diffusion, stationary assumptions and uncorrelated noises are far from being verified in empirical situations.

\subsection{Fokker-Planck Equation}

An other interesting approach may when the process verifies a Fokker-Planck equation on probabilities of the state of the system when it is given by its position (diffusion of particles in that case)

\begin{equation}
\partial_t P(x_i,t) = - d \cdot \partial_x P(x_i,t) + \frac{\sigma^2}{2} \partial^2_x P(x_i,t)
\end{equation}

With no cross-correlation terms in the Fokker-Planck equation, covariance between processes vanish. We have finally in that case only a relation between averaged spatial and temporal variances that brings no information to our question.

\subsection{Master Equation}

In the case of a master equation on probabilities of discrete states of the system

\begin{equation}
\partial_t \vec{P} = \mathbb{W} \vec{P}
\end{equation}

we have then for state $i$, $\partial_t P_i = \sum_j W_{ij}P_j$. As this relation is at a fixed time we can average in time to obtain an equation on temporal covariance. It is not clear how to make the link with spatial covariance as these will depend on spatial specification of discrete states. This question is still under investigation.


\subsection{Consistent spatio-temporal sampling}

In a more empirical way, we propose to not assume any contraint of process dynamics but to however investigate how the computation of spatial correlations can inform on temporal correlations. We try to formulate easily verifiable assumptions under which this is possible.

We make the following assumptions on the spatio-temporal stochastic processes $Y_i\left[\vec{x},t\right]$ :
\begin{enumerate}
\item Local spatial autocorrelation is present and bounded by $l_{\rho}$ (in other words the processes are continuous in space) : at any $\vec{x}$ and $t$, $\left|\rho_{\norm{\Delta \vec{x}} < l_{\rho}}\left[Y_i (\vec{x}+\Delta \vec{x},t), Y_i (\vec{x},t) \right]\right| > 0$.
\item Processes are locally parametrized : $Y_i = Y_i\left[\alpha_i\right]$, where $\alpha_i (\vec{x})$ varies with $l_{\alpha}$, with $l_{\alpha} \gg l_{\rho}$.
\item Spatial correlations between processes have a sense at an intermediate scale $l$ such that $l_{\alpha}\gg l \gg l{\rho}$.
\item Processes covariance stationarity times scale as $\sqrt{l}$.
\item Local ergodicity is present at scale $l$ and dynamics are locally chaotic.
\end{enumerate}


Assumptions one to three can be tested empirically and allow to compare spatial correlation estimated on spatial samplings at scale $l$. Assumption four is more delicate as we are precisely constructing this methodology because we have no temporal information on processes. It is however typical of spatial diffusion processes, and population or innovation diffusion should verify this assumption. The last assumption can be tested if feasible space is known, by checking cribbing on image space on the spatial sample. Under these conditions, local spatial sampling is equivalent to temporal sampling and spatial correlation estimators provide estimator of temporal correlations.








