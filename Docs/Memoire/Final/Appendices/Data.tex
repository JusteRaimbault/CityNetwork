%\chapter{Datasets}{Données} % Chapter title
\chapter{Données}

\markboth{\thechapter\space Données}{\thechapter\space Données}


\label{app:data} % For referencing the chapter elsewhere, use \autoref{ch:name} 

%----------------------------------------------------------------------------------------


%\headercit{}{}{}

%when possible, specify data citation (ex. traffic data : TransportationEquilibrium paper) ; try to put all on dataverse ; laius sur dataverse, partage des données etc.



\bpar{
This appendix lists and describes the different open datasets created and used in the thesis.
}{
Cette annexe liste et décrit les différents jeux de données ouvertes que nous avons été amenés à créer et à utiliser dans la thèse. Les données sont en effet bien un domaine de connaissance propre, et les opérations de collecte et de consolidation sont une étape scientifique à part entière.
}





%%%%%%%%%%%%%%
\section{Grand Paris Traffic Data}{Données de Traffic du Grand Paris}

% données syntadin : checker la licence

\subsection{Specification}{Spécification}

\paragraph{Citation}{Citation}

\paragraph{Type and Format}{Type et Format}

\paragraph{License}{Licence}

\paragraph{Availability}{Disponibilité}


\subsection{Description}{Description}




%%%%%%%
%% -- ON HOLD --
% (clarifier degré d'ouverture possible 


%%%%%%%%%%%%%%
%\section{US Gaz Prices}{Prix de l'Essence aux Etats-Unis}


%\subsection{Specification}{Spécification}

%\paragraph{Citation}{Citation}

%\paragraph{Type and Format}{Type et Format}

%\paragraph{License}{Licence}

%\paragraph{Availability}{Disponibilité}

%\subsection{Description}{Description}







%%%%%%%%%%%%%%
\section{Topological Road Network}{Graphes topologiques des Réseaux Routiers}

La simplification des réseaux routiers, opérée à grande échelle pour l'Europe et la Chine sur les données d'OpenStreetMap, produit les graphes topologiques correspondants. 


\subsection{Description}{Description}




\subsection{Specification}{Spécification}

\paragraph{Citation}{Citation} 

Raimbault, Juste, 2018, "Simplified road networks, Europe and China", doi:10.7910/DVN/RKDZMV, Harvard Dataverse, V1

\paragraph{Type and Format}{Type et Format}

Edge lists of graphs. Format : postgis dumps

\paragraph{License}{Licence}

CC0

\paragraph{Availability}{Disponibilité}

La base est disponible sur le Harvard Dataverse à \url{http://dx.doi.org/10.7910/DVN/RKDZMV}.




%%%%%%%%%%%%%%
%% ON HOLD

%\section{French Freeway Dynamical Network}{Réseau Dynamique des Autoroutes Françaises}

%\comment{Merger avec la base bassin parisien de Florent, faire un data paper.}
%  -> dans une autre vie !





%%%%%%%%%%%%%%
\section{Interviews}{Interviews}

\label{app:sec:interviews}

% laius sur pourquoi données "quali" devraient pas être plus dispo (quand accord intervié) ; outils idem ex. git, dissocié quanti : cf exemple galère excel notes ridicule, refus systématique et catégorique d'une alternative stable et fiable...

Un matériau de recherche qui serait plus ``qualitatif'' au sens classique, n'a pas de raison d'être moins ouvert que des bases de données ``quantitatives''. Dans le cas d'entretiens, l'ouverture des retranscriptions est essentielle pour la reproductibilité puisqu'il s'agit du dernier (et du premier) stade avant la traduction non reproductible en interprétations. Nous pensons également qu'elle est cruciale pour exploiter l'ensemble de leur potentiel, l'ouverture permettant leur réutilisation et donc possiblement réactions ou débats.


\subsection{Description}{Description}

\subsubsection{Interview with Denise Pumain, 2017/03/31}{Entretien avec Denise Pumain, 2017/03/31}

Cet entretien est intervenu dans le contexte d'une collecte de matériau empirique pour la rédaction de~\cite{raimbault2017applied}, qui a permis entre autre la construction du cadre de connaissances développé en~\ref{sec:knowledgeframework}. L'entretien est principalement centré sur la genèse de la Théorie Evolutive des Villes.

\subsubsection{Interview with Romain Reuillon, 2017/04/11}{Entretien avec Romain Reuillon, 2017/04/11}

Cet entretien intervient dans le même contexte, en cherchant à apporter un éclairage du point de vue des méthodes et outils. Il retrace en particulier la genèse d'OpenMole.

\subsubsection{Interview with Clémentine Cottineau, 2017/05/05}{Entretien avec Clémentine Cottineau, 2017/05/05}

Géographe à l'interface interdisciplinaire




\subsubsection{Interview with Denise Pumain, 2017/12/15}{Entretien avec Denise Pumain, 2017/12/15}

Effets structurants des infrastructures de transport et co-évolution, du point de vue de la géographie.




\subsubsection{Interview with Alain Bonnafous, 2018/01/09}{Entretien avec Alain Bonnafous, 2018/01/09}



Effets structurants des infrastructures de transport, du point de vue de l'économie des transports.



\subsection{Specification}{Spécification}

\paragraph{Citation}{Citation}

\paragraph{Type and Format}{Type et Format}

Données textuelles : transcription des entretiens au format texte.

\paragraph{License}{Licence}

% 

\paragraph{Availability}{Disponibilité}

Dépôt git : \url{https://github.com/JusteRaimbault/Entretiens}








%%%%%%%%%%%%%%
\section{Synthetic Data and simulation results}{Données Synthétiques et résultats de simulations}

Les résultats de calculs ou de simulations utilisés pour l'ensemble des résultats présentés sont disponibles de manière ouverte, soit sur le dépôt git soit sur un dépôt dataverse dédié dans le cas d'articles autonomes ou de fichiers massifs. Les liens sont les suivants pour les dépôts particuliers :

\begin{itemize}
	\item Résultats de l'exploration du corpus Cybergeo \url{http://dx.doi.org/10.7910/DVN/VU2XKT}
	\item Indicateurs morphologiques et topologiques pour l'Europe et la Chine \url{http://dx.doi.org/10.7910/DVN/RHLM5Q}
	\item Simulation de données synthétiques 
	\item Simulation et calibration du modèle macroscopique d'interactions
	\item Simulation et calibration du modèle de morphogenèse pour la densité \url{http://dx.doi.org/10.7910/DVN/WSUSBA}
	\item Simulations du modèle de co-évolution macroscopique \url{}
	\item Simulations du modèle de co-évolution mesoscopique \url{}
	\item Simulations du modèle Lutecia \url{}
\end{itemize}













