


%\chapter{Supplementary Information}{Informations supplémentaires}
\chapter{Informations supplémentaires}

\markboth{\thechapter\space Informations supplémentaires}{\thechapter\space Informations supplémentaires}

\label{app:supplementary} % For referencing the chapter elsewhere, use \autoref{ch:name} 

%----------------------------------------------------------------------------------------



\bpar{
This appendix gathers various supplementary materials, necessary for the robustness but not necessary to the main argument. It includes for example more precise model explorations, generally needed to support conclusions in main text but too long or repetitive to be included.
}{
Cette annexe regroupe divers matériaux supplémentaires, nécessaire à la robustesse des études mais pas à l'argumentaire général. Elle inclut par exemple dans le cas des modèles de simulation des explorations plus précises et des analyses de sensibilité.
}

% Q : model behavior to be put in the thesis or in metadata link to git repo ?
%  -> as code, unreadable directly : put listing of statistical analysis
%   find a way to automatically generate stat anal files from R ?


Elle inclut notamment les points suivants :
\begin{itemize}
	\item Relevés de terrain en Chine en~\ref{app:sec:qualitative}, pour les résultats qualitatifs présentés en Chapitre~\ref{ch:thematic}.
	\item Précisions pour l'épistémologie quantitative de~\ref{sec:quantepistemo} en~\ref{app:sec:quantepistemo}.
	\item Résultats complets pour la modélographie de~\ref{sec:modelography} en~\ref{app:sec:modelography}.
	\item Pour les corrélations statiques de~\ref{sec:staticcorrelations} : résultats pour la Chine, analyses de sensibilité, algorithme de simplification de réseau, dérivation analytiques pour le caractère multi-échelle en~\ref{app:sec:staticcorrelations}.
	\item Dérivations pour l'expression des corrélations retardées sur données synthétiques de~\ref{sec:causalityregimes} en~\ref{app:sec:causalityregimes}.
	\item Comportement du modèle et étude semi-analytique du modèle d'agrégation-diffusion de~\ref{sec:densitygeneration} en~\ref{sec:modelography}.
	\item Corrélations faisable pour le couplage faible de~\ref{sec:correlatedsyntheticdata} en~\ref{app:sec:correlatedsyntheticdata}.
	\item Figures étendues pour l'exploration du modèle SimpopNet de~\ref{sec:macrocoevolexplo} en~\ref{app:sec:macrocoevolexplo}.
	\item Figures étendues pour l'exploration du modèle macroscopique de co-évolution de~\ref{sec:macrocoevol} en~\ref{app:sec:macrocoevolexplo}.
	\item Détails du modèle \emph{slime mould} utilisé en~\ref{sec:networkgrowth}, et figures étendues en~\ref{app:sec:networkgrowth}
	\item Processus de calibration au second ordre du modèle mesoscopique de co-évolution de~\ref{sec:mesocoevolmodel} en~\ref{app:sec:mesocoevolmodel}.
	\item Pour le modèle Lutecia de~\ref{sec:lutecia}, étude du modèle d'usage du sol, dérivations de probabilités de coopération, détails d'implémentation et d'initialisation en~\ref{app:sec:lutecia}.
\end{itemize}


\stars







