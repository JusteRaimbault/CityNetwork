


%----------------------------------------------------------------------------------------


%%%%%%%%%%%%%%%%%%%%%%%
\section{Fieldwork Elements}{Elements de Terrain}


\subsection{Fieldwork Notebook}{Carnet de Terrain}

Nous rendons compte ici avec un certain niveau de détail des différentes sorties de terrain alimentant la section~\ref{sec:qualitative}.

%check standard practices (Brenda, Solène, Cinzia). a priori not standard to make it open ?



\paragraph{29/10/2016}{29/10/2016}

Bus - parc - rando improvisée - JiaLeFu.
Pratique de le nature beaucoup moins rependue qu'a HK
// we du 7/11 : rando a HK. Congestion totale. Du ferry ; metro ok ; de la rando a la descente.

\paragraph{16/01/2017}{16/01/2017}

Objectif : Guangzhou ? Pas credible car pont en construction - parti trop tard. Nouvel objectif Xiaolan par Zhongshan. Bus Zhuhai : villages isoles dans la campagne oppressante (vegetation). Pas sur si la ville est loin. Terminus bus Zhongshan : assez bien optimisé, impression que le bus attend. Deux changements seulement. Bus express : arret ajoute gare. Problem unfinished segments (autoroute qui devrait arriver a droite ?). Impossible de rejoindre la gare pourtant a 300m (traversee 4 voie) : des taxis moto improvises attendent specifiquement les gens qui descendent du bus : gare tres utilisee mais a priori insertion pas optimale ? Alentpurs de la gare : villages urbains type milieu de nulle part. Reseau taxis et moto taxi confirme l'impact "improvise" de la gare. Prendre le billet : 2h d'attente min, comme a Guangzhou le 11.prennent le billet en avance ? Ce train pourrait servir de rer (transposition a l'echelle ?) mais semble pas le meme esprit : effet tunnel, tandis que le rer irrigue les territoires avec une grsnularite assez fine. Peut etre la question de la MCR : bien des villes distinctes ! - reseaux tres independants, peu de determination intermediaire pour les infrastructures.

\paragraph{11/12/2016}{11/12/2016}


De Pekin a Guangzhou a Shenzhen par Dongguan. Taxi a Pekin : congestion - ultra galere - aeroport pas loin mais completement isolé. Bus express marchent bien entre les villes ; aeroport (jamais teste wntre villes direct autre que bus de la fac

Sur la congestion ? Assez aleatoire. Cf voyage en bus au port de ferry le 10/01 : 1h45 au lieu de 50min, meme heure a peu pres.

Le 11/01 : xiaozhou urban village : industrie tpuristique bien rodee. Difficilement accessible sans taxi ; semboe etre monnaie courante. Bateaubus : utilise pour vrais deplacements (//l'echec du batobus a Paris : configuration tres differente) : plus grand, riviere plus large, endroits moins desservis ?

Back sur Dongguan : paysage continu dusines pour migrants dans zone la moins bien desservie. Quelle mobilité de ces gens ? Citer MigrationDynaMics : tres mobiles sur le temps long, pas du tout quotidiennement ? Macao et HongKong sont vraiment des exceptions sur le daily commuting : carte speciale residents de Zhuhai (pas sur Shenzhen) pour HK et Macao - la main d'oeuvre moins chere ?

Voyage a Macao debut novembre : peu question de transports sur une superficie si petite. Gongbei est un peu la gare de Macao. Enorme complexe souterrain : shopping etc - vont a HK aCeter plein de merdes, surement Macao aussi. Bus efficace. Un peu de verdure le parc ancien fort. Pas vu les casinos. Temple. Cantonais. Culture locale particuliere.

Exemples d'interactions ? Developper le xiaolan, tres interessant. Coupler a des cartes d'evolution ? Peu credible..
Trouver photos typiques quelques unes en annexe - certaines dans le main text ? Si vraiment utile..

La situation de Tangjia : a peu pres egale distance de Gongbei et de Zhongshan sud. Polderisation. Role de l'université ? Collectif artistes - conservation du patrimoine - restaus bobos cafés etc. Surement differents degrés - mecs sinceres et dautres qui en profitent ? Hightech zone - conflits de gouvernance ? Tod a le beizhan ? Comparer environs a gare tangjia. Entite relativement indep : propre gare de bus, majorite de services sur place. L'attrait duneuf : le nouveau ktv de rencaigongyu semble faire venir gens en voiture - viennent de loin ?
Le port qui fonctionne pour acheminer du petrole : local ou plus large pour Zhuhai ?

Journee a Doumen : entite urbaine independante - nombreuses usines dans la zone : port au sud etc. Congestion quand traversé. Tours en construction partout.

\paragraph{19/06/2017}{19/06/2017}

Ancienne colonie fr : pas grand chose a dire ; zone renovee : interessant pour les regimes de propriete : pour se debarasser de l'urban village, le go a du racheter les terres, l'urbain est public, seul le rural peut etre possede - par une assemblee d'habitants.

Le 08/10/2016
Premiere impression des transports. Congestion raisonnable, marce bie
n
Strategie ppur eviter le changement pas.bonne idee : tres grand espace entree stations. Carte fausse : differentes perceptions et appropriations de l'espace urbain. L'urban village - sorte de choc ?
Deux mondes cohabitent - coevoluent ? Les etudiants a fond ultra serieux. Monde isole - liaison tunnel via le bus zhen fangbian.
Retour au 19/06/2017 : cite universitaire...


\paragraph{8/06/2017}{8/06/2017}


Seulement 4mois et beaucoup de changement : le tram semble marcher (faire une visite du tram et y refererp) - les mobikes sont partout : incroyable la vitesse a laquelle s'est developpe - semble bien marcher meme dans zone moins dense (ex autpur de la fac)
Avait deja teste en octobre les velos jaunes - developpes au debut pour campus seulement - rempamdus a present. Pas de gps. Probleme de la confiance et du rapport au systeme de transport . Ou securite plus generalement. Dans tous les cas plis de respect pour institutions et ce qu'elles mettent a disposition. Les tours sont finies pour l'exterieur. Mais la liu dong est toujours fantome.
Footing sur la baie (deux en un mois... vraie torture de courir en cette saison) : autres compounds commencent a ouvrir aussi. Mais services fermes, pas wncore les supermarches (pas sur car la nuit) : ville etalee (dense par les tours) - suppose la voiture.

Fin de la visite giangzhou 19/06
Cite universitaire : nombreux campus, type ville nouvelle. En grand et efficace. Tour en bus sous la pluie : peu dinteret ? Importance des investissements pour la normal university.
Connaissait deja indirectement car le bus sy arretait - pas concerne par leffet tunnel - impression donnee par le ppnt de l'autorpute super haut. Science center demesure.

// convention center zhuhai -> visite de terrain du 4/12/2016.
Point de vue desSplanificateurs : super masterplan, plans sur la comete pour ponts. Vue municipalite correspond pas a province ? Survendu. Role prpgres technoques ? (Cf reseaux multitechnique). Smart and shit. Puis village renovation - patrimoine conservation ? (Cf notes). Maison de zhongshan ..: pas fou


\paragraph{11/07/2017}{11/07/2017}

bus direct (faible bouchons, 2h10 quasiment tout de même). Gardien à l'entrée : balance les mobikes et autres, aucune pitié (car gênent la porte a priori ?). alarme des mobikes qui sonne.



\paragraph{24/07/2017}{24/07/2017}

Lifestyle du RenCaiGongYu : bus spécial de l'entreprise vient chercher gens devant. gros 4x4 pour aller au taf. pas mal de gens en bus Nord/Sud (traversent Baipu Lukou) ; pas trop fac. Nord : hightech zone ?




\paragraph{31/07/2017}{31/07/2017}

Voitures et tc : le pont que pour les riches si pas de transit ? Ou bus temps raisonnable ?
Des villes a deux vitesses.
Densite de Mobike : pas forcement fiable dans des configurations comme ca ?
Regarde que transports ? Pas darticulation particukiere a prtite echelle, auniveau archi, ou urba : forme urbaine qui suit plutot topo, cf grande zone polderisee. Un peu different centre ville, mais pas fou non plus (tres different HK, dans moindre mesure Shenzhen puis Guangzhou). Eventuellement un peu different autour gares, pseudo tod ?
Grand axe bagnoles : plusieurs stations service, puis magasins caisse.

Vieux weird, gueulent, mec affale les pieds en l'air.
Mec qui rentre de lhopital sysu, radio medocs.

Brume : pollution ?

Encore station.

Totem a la fin des cores values : shenme ?

Melange batiments tres modernes et vieux trucs petits carreuax.
Tunnel en mode voie express. Compound ferme, nature et batiments bas : pour riches?

Gonganchedao, a certaines heures : ?

Velovert, mobike.
Mibikes a la sation bus : intermoda.lite
(Cf photos)


(11h37)
Le tram. Freq attente ok. Stations pas si espacees. Macines a ticket marchent pas. Presence de l'anglais assez forte.
Peu frequente (deux meufs valises).
Design retrofuturiste un peu kitsch.
Controleur : note gens. Gratuit en test ?
Assez spatieux.
Nature - village urbain.
Controlleur super sympa. En effet periode de test, note sur feuille a la main.
Annonce en anglais (trad amusante - plesae leave to passengers in need)
Pas tres rapide tout de meme, comparable au bus.

Demarche administrative : super fluide cette fois -
(Rq : jamsis vu : centre bayunport ici)
Bureauctratie ultra efficwce. Tampon sur chaque feuille, meme photocopies, code barre etiquette, scanne, code certificat photos scanné (donc enregistre quelque part). Question : doivent avoir leur propre systeme de base de donnees, os aussi ? Prog internet ? Tecno specifiques ?.

En effetbpas plus rapide que le bus.
Pub plan sur petite tele : jusqua daxue, tour de notre ile.
Dessert hopital now (avant dernier arret)
Bureau customs.
Pleun lignes, autres vers le centre.
Dame speciale pour nettoyer
Les vieux lont pris juste pour tester ?
Accessible handicape (couloir etroit sordide)

Panneau delaves, images ideales.
Coordination avec le reseau de train.
7 lignes, 129km.
Vieux bus l1 genre historique : tpuristique ?
La plage artificielle ( naturrllement cailloux), hyper longue et deserte. A priori on se baigne pas
Batiments toujpurs en construction en octobre, espace de vente ouvert.
Laonianrenzuo: laisse la place.
Poste police : assez quadrille. Village pecheurs et nouveau compound.
Les grds parents avec les gosses, classique.
Gated com maisons indiv : ultra riches. Pyis u e autre decrepie.
Rq : une poubelle ds le bus(pomme vieile).
Aitres toyrs a flanc de montagne.
La baie. Rencaigongyu fait 1/6 de la skyline !
Immeubles de la nanmen : quelle date? Tushyguan 2000.
(12h19)
(-> 12h22 check envoi)
Pub macao : danseuse occ. Rock players ombre.





\paragraph{09/08/2017}{09/08/2017}

10h44
Sauté dans le tram, sous la pluie.
Stations pas pratiques // climat.
Toujours en test, pointe.
Sens opposé (a priori terminus)
Encore batiment officiel oppressant. Pompiers ont batiment similaire.
Arrive ligne de train : pas trop mal cpncu sur pa correspnndance. Zone qui semble toute neuve ; pas loin gare : genre to a priori. Terminus. Zone ouverte, equipement culturel. Gare integree, centre teCnique soys dalle ? Mec demande ou vont. Retour arriere une station.
Zone en dvlpmt. Demande ou va, doy vient (ou habutr) traditionnel surcahier.

Autourq station h colline verte, amenagee ?
Chemins de traverse, garages. Animé.vendeurs attendent dehors rue.
Gated community. Face hotel americain.
Centre taxis. Petitgarage encore. Archi bcp verre, lumiere ? Pas ecolo chsleur clim ?
Ligne ttain, va chopper le 70.mec pile, route fin absurde.
Damnque 36, 41. Changsha !
Dustances super longues, vraiment pas walkable. Mais bikeable.quoique pas trop de kobike. Train, pas vu. Autre.
Chantier. A cote genre village urbain pour les oivriers
Enorme pub pour que se voit du train. Mais arbres.
Operationenorme. Infos legales.
Impressionetrange, pas ville, campagnr pas si loin. Le propre de la megacityregion ?
Ok areivw a larret 70.
Deux mecs velo jaune-mobike.
Arrive arret. Dazhaimen : what ? Tric touristiqur ?
Gongancheng.
Suburbsin ? Meme pas vraimen. Difficile a qualifier.
Traverse montagnes a pripri. Le 26 stoppe, personne monte ni descend. Un 70 en face.
Du tod mou ? Car pas evident sans bagnole. A hk : super dense et peu etale. Comparer surface / population. Ici peu protection ?
Zone industrirlle. Rq : chgt avec bus orecedent pas pratique. Regarder si un eligne de tram prevue le lobg train. Type de transport train vs rer : forme et pratiques de la ville tres differentes. Schemas geographoques comparatifs. Light raip jpue le roele rer ? Peut etre decalage hierarchie car scaled with different density n lier au scaling.
Un mec a mobikr. Ouvrier rentre. Pause midi ou type 3-8 ? Bus, mais tout droit.
A hk, shenzhem sagglomere a la peripherie. Ici semble moins. Tailles vraiment pas comparable. Zhuhai pas a macao cr que shenzhrn a hk. Rope de zhongshan, comme dongguan ? Reprendre les slides de zhou.
Taxis racolent, sens inverse.
Pas frequence, prix et horaires min. Max de monde du 26, doivent changer. School de northeastren ? Pas mal de campus vers tangjiazhan. Bus clim et normal, 1 et 2.
Putong. Yes arrive.
Parce/temple, truc tpuristique en toutcas. Loupé.
Grosse pluie tres courte.
Putuo temple.
Meuf mail : buddy at hk university.
(Try schema papier)
Suit le train. 4 voies, campagne. Bout campus. Plus, voies chaque cote train.
Arretnulle part, usines j passage souterrain. Zone industrielle. Montagnes.

(11h42)
Tunnel mais service. Natural mineral water.
Long, plus 1km ? Q : comment font velos ? 1648m.

Residence ? Genre campus.
Yes campus de la beijing normsl university.
Panneau navetye jialefu arret bus.
Juste derriere montgame campus zhongda.
Canal, arbres plantes. Semble beaucoup boulot espaces verts.
Premieres tours. Ou pass le train qd campus ?
Artivee tangjiawanzhan. Lestours exactement la gare.
Bus electtique (10a rouge)
Pub immobilier, photo traon.
Gare, 11h54 - fin. (1h10)

(Trains 12h 2 l, 13h puis 16h) pas ouf frequent ?
Bien 5 yuan par billet si pas retire la gare concenree (depart ou arribee, la tangjiawan)
Tester le domac ! Standard ? Mettnt meuf parlant angalis ou coincid ?
Jeune couple parfait, mec genre manager propre syr lui, vienmet acheter appart dans nouvelle tour.


(12h35)
K3, pas mal d'attente ; a priori pas tres tardifs en plus.
Gare deserte, quartier semble artificiel, pas si dense hormis le compound. A l'ouest idem tres aere.
Modele tod pas comparable, semble pas de reelle volonte des pouvoirs publics (fausrait interviews / au moins check documents officiels). Plus speculation sur l'immobilier autour des gares ? Role de huafa ? Pas loin de beizhan, aucun train (a propri) faut larret au deux mais trop loin pour pied. Tod lointain, joue sur commuting longue distancr, pas forcement journalier ? Bus express, rapide.
Mec arrive sac, check plan : nouveau venu hugh trch zone ? Chauffeur sarrete pas, pas prevu pour le gars ? Si ok arret intermediaire.
Bus electtiqie, semble pas hybride (pas bruit moteur du tput a larret)
Zone humide, protegee avec lile ecologoqie a priori. Pont demesure pour y acceder : ebauche pont vers shenzhen (verifier ca).
Le golf de zhuhai : qui l'utilise ?
Partie assez ancienne dans la hightech zone ; proximite bord mer ou axe routier ? Dvlpmt serait interessant.
Arrivee tangjia. Le k3 super direct, finit a gongbei. Un arret trop tot, pas fait gaffe que le dongbei etaut aussi loin. Taxi sarrete, derriere veget, parve que sur piste cyclable ?
Zone ancienne suburbs de tangjis, ancien noyau urbain. Zone plus vivantr du coup. Pas sur que le bout de la zone soit pareil.
Livreurs velo, transportelec partout. Pleinde gosses avec grds parents a cette heure. Mec gare de bus coure en face, commodites ?
2 arrets tres proches a tangjia. Fille bus pouvait sassoir derriere seul, random ?
Station velo slogan hexie. Jeune regarde bizarrement. Puis encore core values. Bourrage de crane. Gov office local, plein truc different. Chauffatd musique y va. Photo avec rencaigongyu en fo nd
(12h56, gongan)




\paragraph{13/08/2017}{13/08/2017}

Experience du HSR - totalement saturé (vente certain nombre de tickets places debouts) ; ligne ``secondaire''. sur un we beaucoup de touristes. gare aussi secondaire, plus d'un train par heure. sillons sur la ligne doivent être saturés. approche de guangzhounan déjà avant Foshan, perte de temps énorme attente (due travaux gare Foshanxi ? pas que, autre train double) : congestion approche de la gare. dernier train assez tôt ? comparable Europe, selon heure arrivée. pas de trains de banlieue dans la gare, d'ou différente impression. par contre pour le train ``régional'' très tôt, super blindé. système des stations proches desservies par différentes missions pas si idiot ; mais du coup fréquence plus faible. incroyable auto-organisation des transports informels localement, super facile de rentrer, direct une moto-taxi, n'aurait pas pu faire plus rapide (un peu plus de 10km !).

Génie civil de la ligne : nombre impressionnant de ponts/tunnels ; logique vu la topographie chaotique de la région. 238 tunnels, 464km (https://en.wikipedia.org/wiki/Guiyang{\%}E2{\%}80{\%}93Guangzhou{\_}High-Speed{\_}Railway)















