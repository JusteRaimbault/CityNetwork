


%----------------------------------------------------------------------------------------


%%%%%%%%%%%%%%%%%%%%%%%
\section{Fieldwork Elements}{Elements de Terrain}

\label{app:sec:qualitative}

%----------------------------------------------------------------------------------------


%%%%%%%%%%%%%%%%%%%%%
\subsection{Fieldwork Notebook}{Carnet de Terrain}

Nous rendons compte ici de manière synthétique les différentes sorties de terrain alimentant la section~\ref{sec:qualitative}. S'il n'est a priori pas standard de fournir de manière brute et ouverte le contenu des carnets de terrain, \cite{goffman1989fieldwork} souligne que celui-ci peut être un matériau de recherche en lui-même. Les compte-rendus bruts et les photos sont disponibles de manière ouverte à \url{https://github.com/JusteRaimbault/CityNetwork/tree/master/Data/Fieldwork}.


Ci-dessous sont résumés les contextes et observations principaux des sorties.


\paragraph{29/10/2016}{29/10/2016}

Sortie à Zhuhai (Xiangzhou et Gongbei), journée. Nature en ville et utilisation des parcs par les habitants.


\paragraph{07/11/2016}{07/11/2016}

Aller-retour Zhuhai-Hong-Kong. Relation apparente des habitants de Zhuhai à la ZAS.

\paragraph{06/11/2016}{06/11/2016}

Sortie à Macao par Gongbei. Flux journaliers par la frontière de la ZAS.

\paragraph{16/01/2017}{16/01/2017}

Tentative de relier Tangjia à Guangzhou par bus de ville, journée. Itinéraire final Tangjia-Zhongshan-Xiaolan-Zhuahaibei. Transports locaux et franges urbaines.


\paragraph{11/12/2016}{11/12/2016}

De Pekin à Shenzhen par Guangzhou et Dongguan. Transports, difficultés d'accessibilité.


\paragraph{8/06/2017}{8/06/2017}

De Hong-Kong à Tangjia par Zhuhai. Transports.


\paragraph{19/06/2017}{19/06/2017}

Visite de terrain officielle (Conférence Medium, SYSU), Guangzhou. Rénovation Urbaine, projets urbains, patrimoine.


\paragraph{11/07/2017}{11/07/2017}

Aller-retour Tangjia-Guangzhou. Congestion des transports (routier et vélos libre-service).


\paragraph{24/07/2017}{24/07/2017}


Sortie à Tangjia. Discontinuités socio-économiques locales.


\paragraph{31/07/2017}{31/07/2017}

Sortie à Xiangzhou. Test du Tramway, Ligne 2.



\paragraph{09/08/2017}{09/08/2017}

Sortie à Xiangzhou puis Tangjia. Opération de TOD : terminus ouest Tram ; bus pour la gare de Tangjia le long de la ligne à grande vitesse.


\paragraph{13/08/2017}{13/08/2017}

De Yangshuo (Guanxi) à GuangzhouNan par le Train à Grande Vitesse.




\paragraph{17/08/2017}{17/08/2017}


Bureau du Comité de Planification de la zone High Tech de Zhuhai. Administration et bureaucratie.



\paragraph{20/08/2017}{20/08/2017}

Traversée de Leshan (Sichuan) en bus, aller-retour. Transports et Tourisme.



\paragraph{21/08/2017}{21/08/2017}

De Guangzhou Baiyun à Zhongshan Daxue (campus sud) puis Tangjia. Transports, village urbain.




%%%%%%%%%%%%%
%% -- ON HOLD --

%%%%%%%%%%%%%%%%%%%%%%%
%\subsection{Illustrations}{Illustrations}

% important photos : cf tram.








%%%%%%%%%%%%%%%%%%%%%%%
\subsection{Interviews}{Entretiens}


Les ``entretiens'' menés relèvent de l'entretien actif non-structuré~\cite{holstein2004active} lors d'une mise en situation vécue conjointement. Les difficultés linguistiques de part et d'autre ont pu rendre compliqué les dialogues et nous donnons ici une synthèse des informations acquises.

% Jingzi : Emeishan.
% Meuf à Yangshuo.
% Meuf à Zhuhai, de Zhongshan.

















