



%----------------------------------------------------------------------------------------

\newpage

%%%%%%%%%%%%%%%%%%%%%%%
\section{Network growth heuristics}{Heuristiques de génération de réseau}

\label{app:sec:networkgrowth}



%%%%%%%%%%%%%%%%%%%%%%%
\subsection{Slime mould model}{Modèle de slime mould}

Nous rappelons ici la procédure d'évolution du réseau biologique de type \emph{slime mould}, à partir de~\cite{tero2007mathematical}. Le réseau est composé de noeuds caractérisés par leur pression $p_i$ et de liens caractérisés par leur longueur $L_{ij}$, leur diamètre $D_{ij}$, une impédance $Z_ij$ et le flux les traversant $\phi_{ij}$. La relation analogue à la loi d'Ohm pour les liens s'écrit

\[
\phi_{ij} = \frac{D_{ij}}{Z_{ij}\cdot L_{ij}} \left(p_i - p_j\right)
\]

Par ailleurs, la conservation des flux à chaque noeud (loi de Kirchoff) impose

\[
\sum_i \phi_{ij} = 0
\]

pour tout $j$ sauf pour la source et le puit, que nous supposons aux indices $j_+$ et $j_-$, tel que $\sum_i \phi_{ij_+} = I_0$ et $\sum_i \phi_{ij_-} = -I_0$ avec $I_0$ paramètre de flux initial.

La combinaison des contraintes ci-dessus donne pour tout $j$

\[
\sum_i \frac{D_{ij}}{Z_{ij}\cdot L_{ij}} (p_i - p_j) = \mathbbm{1}_{j=j_+} I_0 - \mathbbm{1}_{j=j_-} I_0 
\]

ce qui se simplifie en une équation matricielle, en notant $\mathbf{Z} = \left(\frac{\frac{D_{ij}}{Z_{ij}\cdot L_{ij}}}{\sum_i \frac{D_{ij}}{Z_{ij}\cdot L_{ij}}}\right)_{ij}$, ainsi que $\vec{k} = \frac{\mathbbm{1}_{j=j_+} I_0 - \mathbbm{1}_{j=j_-} I_0 }{\sum_i \frac{D_{ij}}{Z_{ij}\cdot L_{ij}}}$ et $\vec{p} = p_i$, qui se simplifie en

\[
\left(Id - \mathbf{Z}\right) \vec{p} = \vec{k}
\]

Le système admet une solution lorsque $\left(Id - \mathbf{Z}\right)$ est inversible. L'espace des matrices inversible étant dense dans $\mathcal{M}_n(\mathbb{R})$, par multilinéarité du déterminant, une perturbation infinitésimale de la position des noeuds permet d'inverser la matrice si celle-ci est effectivement singulière. On obtient donc les pressions $p_i$ et par conséquent les flux $\phi_{ij}$.

% François le resoud avec https://arxiv.org/pdf/1301.6628.pdf
% https://github.com/fqueyroi/tulip_plugins/blob/master/TransportationNetworks/FlowBetweenness/flowbetweenness.cpp

L'évolution du diamètre $D_{ij}$ entre deux étapes d'équilibre est fonction du flux à l'équilibre, par l'équation 

\[
D_{ij} (t+1) - D_{ij} = \delta t \left[ \frac{\phi_{ij}(t)^\gamma}{1 + \phi_{ij}(t)^\gamma} - D_{ij}(t)\right]
\]

Nous prenons pour simplifier  $\gamma = 1.8$, suivant la configuration utilisée par~\cite{tero2010rules} pour génération d'un réseau dans une configuration réelle. Nous prenons par ailleurs $\delta t = 0.05$ et $I_0 = 10$.

La génération d'un réseau peut s'effectuer à partir d'un réseau initial, jusqu'à atteindre un critère de convergence, par exemple $\sum_{ij} \Delta D_{ij} (t) < \varepsilon$ avec $\varepsilon$ paramètre de seuil fixé. Nous utilisons ce modèle avec un critère de nombre d'itérations, et procédons à une itération pour obtenir des réseaux finaux avec un nombre raisonnable de liens.




%%%%%%%%%%%%%%%%%%%%%%%
\subsection{Results}{Résultats}


Dans l'expérience explorant la distance aux réseaux réels, l'initialisation de la densité est faite selon 50 grilles classées dans 5 classes morphologiques (10 grilles par classe). La Table~\ref{app:tab:networkgrowth:morpho} donne la composition des centres des classes en termes d'indicateurs morphologiques. Les classes peuvent être interprétées de la façon suivante :
\begin{itemize}
	\item Classe 5 : plus bas Moran, distance, hiérarchie et entropie élevées ; nombreux foyers de peuplement localisés et dispersés.
	\item Classe 4 : plus fortes entropie et hiérarchie ; un petit nombre de foyers localisés.
	\item Classe 3 : plus basse distance et entropie ; population diffuse.
	\item Classe 2 : plus haut Moran ; un ou quelques centres de taille conséquente.
	\item Classe 1 : valeurs intermédiaires pour tous les indicateurs ; un certain nombre de centres de taille intermédiaire.
\end{itemize}


%%%%%%%%%%%%%%%
\begin{table}
\apptabcaption{\label{app:tab:networkgrowth:morpho}}{\textbf{Indicateurs morphologiques pour les centres des classes des grilles de densité initiales.}\label{app:tab:networkgrowth:morpho}}
\begin{tabular}{|c|c|c|c|c|}
\hline
Classe & Moran $I$ & distance $\bar{d}$ & entropie $\mathcal{E}$ & slope $\gamma$ \\\hline 
1 & 0.23 & 0.66 & 0.76 & 0.62 \\\hline % medium everything
2 & 0.47 & 0.50 & 0.75 & 0.53 \\\hline % highest moran, medium others
3 & 0.21 & 0.42 & 0.57 & 0.65 \\\hline % lowest distance and entropy, medium slope
4 & 0.24 & 0.75 & 0.90 & 0.87 \\\hline % highest entropy and slope
5 & 0.15 & 0.76 & 0.84 & 0.72 \\\hline % lowest moran, high distance, slope and entropy
\end{tabular}
\end{table}
%%%%%%%%%%%%%%%


Les espaces topologiques des réseaux générés en~\ref{sec:networkgrowth} peuvent être conditionnés aux classes morphologiques pour la distribution de densité initiale. Ce conditionnement est montré en Fig.~\ref{fig:app:networkgrowth:feasiblespace_bymorph}. Nous donnons également les espaces faisables avec les points réels. Les classe 1 et 5 semblent être celle pour laquelle le rapprochement aux points réels est le plus facile, en termes de points extrêmes.


%%%%%%%%%%%%%%%%%
\begin{figure}
%\includegraphics[width=\linewidth]{Figures/NetworkGrowth/feasible_space_pca_bymorph}
%\includegraphics[width=\linewidth]{Figures/NetworkGrowth/feasible_space_withreal_pca_bymorph}
\includegraphics[width=0.9\linewidth]{Figures/Final/A-networkgrowth-feasiblespace_bymorph}
\appcaption{\textbf{Results conditional to morphological classes.}\label{fig:app:networkgrowth:feasiblespace_bymorph}}{\textbf{Conditionnement des résultats aux classes morphologiques pour la densité.} (\textit{Haut}) Espace topologique faisable pour les différentes heuristiques de génération, conditionné à la classe morphologique de densité. (\textit{Bas}) Mêmes graphiques avec les points réels en rouge.\label{fig:app:networkgrowth:feasiblespace_bymorph}}
\end{figure}
%%%%%%%%%%%%%%%%%









%----------------------------------------------------------------------------------------

\newpage

%%%%%%%%%%%%%%%%%%%%%%%
\section{Co-evolution at the meso scale}{Co-évolution à l'échelle mesoscopique}

\label{app:sec:mesocoevolmodel}


%%%%%%%%%%%%%%%%%%%%%%%
\subsection{Calibration}{Calibration}


Afin de justifier l'agrégation des distances pour les indicateurs et pour les corrélations, nous avons contrôlé visuellement la forme des fronts de Pareto pour ces deux objectifs pour une vingtaine de points simulés. Un example pour deux points est donné en Fig.~\ref{fig:app:mesocoevolmodel:paretodists}. Il apparait que ces fronts sont quasi-inexistants, c'est à dire qu'il existe presque un optimum global.

Illustrons dans quelle mesure une agrégation linéaire à coefficient égaux peut être pertinente dans le cas d'un front de Pareto quasiment vertical/horizontal. La fonction
\[
f_{\alpha} : x \mapsto \frac{1}{(x+1)^\alpha}
\]
prend cette forme dans un voisinage de 0 lorsque $\alpha$ devient grand. Considérons alors les deux objectifs $o_1(x) = x$ et $o_2(x) = f_{\alpha}(x)$, qui peuvent soit être considérés pour une minimisation bi-objectifs, soit dans le cadre d'une agrégation linéaire par minimisation de $o(x) = \beta x + (1-\beta) \frac{1}{(x+1)^{\alpha}}$. Cette dernière est minimale en $x = \left(\frac{\beta}{\alpha (1-\beta)}\right)^{\frac{1}{\alpha + 1}} - 1$, terme qui se développe en 
\[
x = \frac{\ln\left(\beta (1-\beta)\right)}{\alpha + 1} + \frac{\ln\alpha}{\alpha + 1} + o(\frac{1}{\alpha})
\]

Par ailleurs, considérons que dans le cadre d'une optimisation bi-objectifs, on prenne le compromis auquel les variations de $o_1$ égalent celles de $o_2$, ce qui revient à prendre $x$ tel que $\frac{\partial f}{\partial x} = \frac{\partial f^{-1}}{\partial x}$. Cette équation conduit à $\frac{x^{\frac{1}{\alpha}}}{x + 1} = \frac{1}{\alpha^{\frac{2}{\alpha + 1}}}$. On peut alors développer au second ordre de chaque côté pour obtenir
\[
\frac{\ln x}{\alpha} = x \left[1 - 2 \frac{\ln \alpha}{\alpha + 1} + o(\frac{1}{\alpha})\right] - 2 \frac{\ln \alpha}{\alpha + 1} + o(\frac{1}{\alpha}) 
\]
Or on a nécessairement $x\rightarrow_{\alpha \rightarrow \infty} 0$, puisque si $x \rightarrow K \neq 0$, on a une contradiction dans l'équation précédente car $1/(1+K) \neq 0$. Cela implique que $\frac{\ln x}{\alpha} = o(\frac{1}{\alpha})$, et donc que
\[
x = 2 \frac{\ln \alpha}{\alpha + 1} + o(\frac{1}{\alpha})
\]
Pour avoir donc les mêmes ordres de grandeur pour les solutions aux deux approches, il faut éliminer le terme en $1/(\alpha + 1)$ dans la première, ce qui revient à prendre $\ln \left(\beta (1- \beta)\right) = 0$ et donc $\beta = 1/2$.

Ainsi, il y a équivalence des ordres de grandeurs en $\alpha$ pour les deux approches si et seulement si $\beta = 1/2$. Vu la forme de nos fronts de Pareto, nous considérons la solution analogue et considérons ainsi la somme des deux distances.


%%%%%%%%%%%%%%
\begin{figure}
	%\includegraphics[width=0.49\linewidth]{Figures/MesoCoEvol/dists_pareto_i1.png}
	%\includegraphics[width=0.49\linewidth]{Figures/MesoCoEvol/dists_pareto_i10.png}
	\includegraphics[width=\linewidth]{Figures/Final/A-mesocoevolmodel-paretodists.jpg}
	\appcaption{\textbf{Example of Pareto fronts for the calibration at the first and second order.}\label{fig:app:mesocoevolmodel:paretodists}}{\textbf{Exemples de fronts de Pareto pour la calibration au premier et au second ordre.} Nous donnons pour deux points particuliers de simulation, les distances aux indicateurs $d_I^2$ et les distances aux corrélations $d_C^2$ pour l'ensemble des points réels.\label{fig:app:mesocoevolmodel:paretodists}}
\end{figure}
%%%%%%%%%%%%%%







%----------------------------------------------------------------------------------------

\newpage

%%%%%%%%%%%%%%%%%%%%%%%
\section{Transportation Network Governance modeling}{Modélisation de la gouvernance du système de transport}

\label{app:sec:lutecia}


%%%%%%%%%%%%%%%%%%%%%%%
\subsection{Land-use model}{Modèle d'usage du sol}


\subsubsection{Convergence}{Convergence}

Nous étudions ici la question de la convergence dans le temps de la distribution des activités, à infrastructure fixe.

Considérons un cas très simple : en prenant $\lambda = 0$ on déspatialise le problème et en prenant $\gamma_A = 1$ on finit de découpler population et emplois. En posant $\beta' = \sum_j E_j \cdot \beta$ et $P_0 = \alpha \cdot \sum_i P_i$, l'existence d'un point fixe pour les populations se ramène à la résolution de
\[
P_i = P_0 \cdot \frac{\exp\left(\beta' \cdot P_i\right)}{\sum \exp\left(\beta' \cdot P_i\right)}
\]

La fonction est bien continue en les $P_i$ et les plages de variations de la population sont $[0,\sum_i P_i]$, elle admet donc un point fixe par le Théorème du Point Fixe de Brouwer. 

En fait, en toute généralité, si on écrit

\[
(\vec{P}(t+1),\vec{E}(t+1)) = f(\vec{P}(t),\vec{E}(t))
\]

pour des valeurs des paramètres arbitraires, la fonction $f$ est également continue en chaque composante, et prend ses valeurs dans un fermé borné (les emplois étant également limités) donc compact. De la même manière que \cite{leurent2014user} l'établit pour un modèle de flux de traffic, on a aussi un point fixe dans notre cas, ce qui correspond à un point d'équilibre. L'unicité n'est cependant pas triviale et il n'y a pas de raison qu'elle soit vérifiée a priori. On vérifie empiriquement la convergence systématique à infrastructure fixe (voir ci-dessous l'exploration de l'espace des paramètres).


\subsubsection{Exploration}{Exploration}


Nous procédons à une exploration du comportement du modèle d'usage du sol seul, i.e. à infrastructure fixe, afin de comprendre l'influence des paramètres sur la forme urbaine. Nous fixons $\alpha = 1$ ici pour étudier le modèle dans un cas extrême.

Nous suivons les indicateurs de forme urbaine définis en~\ref{sec:staticcorrelations}, pour la distribution de la population et pour les emplois, dans le temps et jusqu'après convergence. Nous réduisons l'espace morphologique de la distribution spatiale des actifs dans un plan principal, tel que $PC_1 = -0.98 \cdot I - 0.13 \cdot \mathcal{E} + 0.05 \bar{d} - 0.13 \cdot \gamma $ et $PC_2 = -0.19 \cdot I + 0.57 \cdot \mathcal{E} - 0.16 \bar{d} + 0.77 \cdot \gamma $. La première composante exprime un niveau de dispersion et la seconde une agrégation hiérarchique.


La Fig.~\ref{fig:app:lutecia:morphotrajs} donne des trajectoires temporelles dans le plan $(PC_1,PC_2)$ pour $\gamma_A = 0.9$, $\gamma_E = 0.6$, $v_0 = 6$, pour différentes valeurs de $\lambda$ et de $\beta$ ainsi que pour différents réseaux initiaux.

% morphoActiveTrajs_gammaCDA0.9_gammaCDE0.6
% Morphology
% "PC1" "PC2" "PC3" "PC4"
%"moranActives" -0.980084779269111 -0.195512301926648 -0.0273364571430177 0.0214821635602741
%"entropyActives" -0.137030712105593 0.570339690854727 0.221651600681317 -0.778977399473809
%"meanDistanceActives" 0.0503574253429071 -0.168869841317757 -0.903924010571035 -0.389703078662877
%"slopeActives" -0.134612551783578 0.779724665202014 -0.364752887773974 0.490779215369261



%%%%%%%%%%%%%%
\begin{figure}
	%\includegraphics[width=\linewidth]{Figures/Lutecia/morphoActiveTrajsvaryinglambda_betaDC1_euclpace6.pdf}
	%\includegraphics[width=\linewidth]{Figures/Lutecia/morphoActiveTrajsvaryinglambda_betaDC2_euclpace6.pdf}
	\includegraphics[width=\linewidth]{Figures/Final/A-lutecia-morphotrajs.jpg}
	\appcaption{\label{fig:app:lutecia:morphotrajs}}{\textbf{Trajectoires morphologiques}\label{fig:app:lutecia:morphotrajs}}
\end{figure}
%%%%%%%%%%%%%%


La Fig.~\ref{fig:app:lutecia:morphosens} donne la valeur de $PC_1$ sur l'ensemble de l'espace des paramètres exploré.


%%%%%%%%%%%%%%
\begin{figure}
	%\includegraphics[width=\linewidth]{Figures/Lutecia/PC1_synth_nonw_euclpace6.png}
	\includegraphics[width=\linewidth]{Figures/Final/A-lutecia-morphosens.jpg}
	\appcaption{\textbf{Sensitivity of the urban form.}\label{fig:app:lutecia:morphosens}}{\textbf{Sensibilité de la forme urbaine.}\label{fig:app:lutecia:morphosens}}
\end{figure}
%%%%%%%%%%%%%%



Enfin, afin de comprendre l'influence des paramètres sur la mobilité totale au cours d'une trajectoire complète, nous étudions en Fig.~\ref{fig:app:lutecia:ludiff} la variation cumulée donnée par $\sum_t \sum_k \left|\Delta A_k (t)\right|$.


%%%%%%%%%%%%%%
\begin{figure}
	%\includegraphics[width=\linewidth]{Figures/Lutecia/rdiffact_synth_nonw_euclpace6.png}
	\includegraphics[width=\linewidth]{Figures/Final/A-lutecia-ludiff.jpg}
	\appcaption{\textbf{Cumulated variations of urban configurations.}\label{fig:app:lutecia:ludiff}}{\textbf{Variabilité cumulée des configurations urbaines.}\label{fig:app:lutecia:ludiff}}
\end{figure}
%%%%%%%%%%%%%%





%%%%%%%%%%%%%%%%%%%%%%%
\subsection{Transportation Sub-model}{Module de transport}

Nous n'avons pas pris en compte les flux de transport dans notre implémentation du modèle, supposant que les infrastructures construites sont de capacités suffisantes pour ne pas être significativement sensibles à la congestion.


Pour le calcul des flux entre cellules, l'opération est la suivante : les flux $\phi_{ij}$ sont calculés par résolution sur $p_i,q_j$ par une méthode de point fixe (algorithme de Furness), du système des flux gravitaires :


\[
\begin{cases}
\phi_{ij} = p_i q_j A_i E_j \exp{\left(-\lambda_{tr} d_{ij}\right)}\\
\sum_k \phi_{kj} = E_j\\
\sum_k \phi_{ik} = A_i\\
p_i = \frac{1}{\sum_k{q_k E_k \exp{(-\lambda_{tr}d_{ik})}}}\\
q_j = \frac{1}{\sum_k{p_k A_k \exp{(-\lambda_{tr}d_{kj})}}} 
\end{cases}
\]

où $\lambda_{tr}$ est un paramètre donnant la portée spatiale des flux journaliers.

Pour implémenter l'étape de distribution des flux dans le réseau, une fois les flux entre cellules connus, il faudrait par exemple déterminer les flux de l'Equilibre Utilisateur Statique avec un algorithme approprié. Une affectation par plus courts chemins est implémentée avec le calcul des flux dans le modèle, mais nous désactivons ce processus pour simplifier l'étude du modèle.


\bpar{
Trajectories are then attributed by effective shortest path, and corresponding congestion $c$ is computed on the corresponding flows (we do not complicate with a User Equilibrium or more complicated traffic assignment procedure). The speed of network given by BPR function $v(c) = v_0 \left(1 - \frac{c}{c_max}\right)^{\gamma_c}$. Congestion is not used in current studies (infinite capacity $\kappa$).
}{
La congestion peut être calculée comme un rapport à la capacité, comme $c/c_{max}$ si $c$ est le flux et $c_{max}$ la capacité. La vitesse est obtenue par une fonction BPR sous la forme $v(c) = v_0 \left(1 - \frac{c}{c_max}\right)^{\gamma_c}$. Notre configuration revient à supposer une capacité infinie  $c_{max} = \infty$.
}


\subsection{Cooperation probabilities}{Probabilités de coopération}


L'hypothèse d'équilibre implique que les espérances conditionnelles de chaque joueur sont égales étant donné leur deux choix, i.e. que 
\[
\Eb{U_i|S_i=C} = \Eb{U_i|S_i=NC}
\] 

Cela revient en effet dans ce cas à maximiser $\Eb{U_i}$ par rapport à $p_i$, puisque en conditionnant on a $\Eb{U_i} = p_i \Eb{U_i|S_i = C} + (1 - p_i) \Eb{U_i|S_i = NC}$, et donc $\frac{\partial \Eb{U_i}}{\partial p_i} = \Eb{U_i|S_i = C} - \Eb{U_i| S_i = NC}$.

On a alors

\[
\Eb{U_i|S_i=C} = p_{1-i} U_i(S_i=C,S_{1-i}=C) + (1- p_{1-i}) U_i (S_i=C,S_{1-i}=NC)
\]

et donc 

\begin{equation*}
\hspace{-1cm}
\begin{split}
	p_{1-i} & U_i(S_i=C,S_{1-i}=C) + (1- p_{1-i}) U_i (S_i=C,S_{1-i}=NC) \\
	& = p_{1-i} U_i(S_i=NC,S_{1-i}=C) + (1- p_{1-i}) U_i (S_i=NC,S_{1-i}=NC)
\end{split}
\end{equation*}

ce qui donne

\[
p_{1-i} = - \frac{U_i(C,NC) - U_i(NC,NC)}{\left(U_i(C,C) - U_i(NC,C)\right) - \left(U_i(C,NC) - U_i(NC,NC)\right)}
\]

En substituant les expressions des utilités à partir de la matrice de gain, on obtient l'expression de $p_i$ en fonction du coût de collaboration $J$ et de la différence des différentiels d'accessibilité.


\subsubsection{Discrete choice coordination}{Coordination par choix discrets}


Pour déterminer la probabilité de coopération dans le cas des choix discrets, il s'agit de résoudre $f(p_i) = 0$ avec
\[
f(x) = \frac{1}{1+\exp\left[-\beta_{DC}\frac{\Delta_i}{1 + \exp(-\beta_{DC}(x \Delta_{1-i} - J))} - J\right]} - x
\]
où nous avons noté $\Delta_i = \Delta X_{i}(Z^{\star}_{C}) - \Delta X_{\bar{i}}(Z^{\star}_{i})$.

On a immédiatement $f(0) > 0 $ et $f(1) < 0$ et $f$ est continue, il existe donc toujours une solution $x\in [0,1]$ par le théorème des valeurs intermédiaires.

Concernant l'unicité, il est possible de la montrer sous certaines conditions. Un calcul de $\frac{\partial f}{\partial x}$ donne 
\[
\frac{\partial f}{\partial x} = 2 (\cosh u(x) - 1) + \beta^2 \Delta_i \Delta_{1-i} \frac{\exp(-\beta_{DC}(x \Delta_{1-i} - J))}{(1 + \exp(-\beta_{DC}(x \Delta_{1-i} - J)))^2}
\]

où $u(x) = -\beta_{DC} (\frac{\Delta_i}{1 + \exp(-\beta_{DC}(x \Delta_{1-i} - J))} - J)$.

Comme $\cosh u \geq 1$, on a $\frac{\partial f}{\partial x} > 0$ si $\Delta_i \Delta_{1-i} > 0$. La fonction est dans ce cas strictement croissante et on a une unique solution.

En pratique, la solution est déterminée par algorithme de Brent, avec les bornes $[0,1]$ et une tolérance de $0.01$.


%%%%%%%%%%%%%%%%%%%%%%%
\subsection{Implementation details}{Détails d'implémentation}


\paragraph{Distance Matrix}{Matrice des distances}

\bpar{
Distance via network are updated in a dynamical programming fashion for efficiency purposes (because of the numerous network updates), the following way :
\begin{enumerate}
\item Euclidian distance matrix $d(i,j)$ is computed analytically
%\item Network shortest paths between network intersections (rasterized network) updated in a dynamic way (addition of new paths and update or change of old paths if needed when a link is added), correspondance between network patches and closest intersection also updated dynamically ; $O(N_{inters}^3)$
%\item Weak component clusters and distance between clusters updated ; $O(N_{nw}^2)$
%\item Network distances between network patches updated, through the heuristic of only minimal connexions between clusters ; $O(N_{nw}^2)$
%\item Effective distances (taking paces/congestion into account) updated as minimum between euclidian time and \[\min_{C,C'}{d(i,C)+d_{nw}(p_C(i),p_C'(j))+d(C',j)}\], complexity in $O(N_{clusters}^2\cdot N^2)$ (Approximated with $\min_C$ only in the implementation, consistent within the interaction ranges $\sim$ 5 patches taken in the model). 
\end{enumerate}
}{
La matrice des distances est mise à jour de manière dynamique pour des questions de rapidité d'execution (vu le nombre de mises à jour du réseau), de la façon suivante :
\begin{enumerate}
	\item La matrice de distance euclidienne $d(i,j)$ est calculée analytiquement
	\item Les plus courts chemins entre les intersections des liens (entre les cellules du réseau raster correspondant) sont mis à jour de manière dynamique (étape de complexité $O(N_{inters}^3$) :
	\begin{itemize}
		\item Pour chaque nouvelle intersection, les plus courts chemins vers l'ensemble des autres intersections sont calculés par l'ancienne matrice et le nouveau lien.
		\item Pour l'ensemble des anciens plus courts chemins, ils sont mis à jour si besoin après vérification des éventuels raccourcis par le nouveau lien.
		\item La correspondance entre les cellules quelconques du réseau et les intersections est mise à jour.
	\end{itemize}
	\item Les composantes connexes et les distances entre celles-ci sont mises à jour (complexité en $O(N_{nw}^2)$)
	\item Les distances par le réseau entre les cellules du réseau sont mises à jour, avec l'heuristique des connexions minimales uniquement (un lien unique le plus court entre chaque cluster) (complexité en $O(N_{nw}^2)$)
	\item Les distances effectives entre l'ensemble des cellules (prenant la vitesse et la congestion en compte si celle-ci est implémentée) sont calculées comme le minimum entre la distance euclidienne et 
	\[\min_{C,C'}{d(i,C)+d_{nw}(p_C(i),p_C'(j))+d(C',j)}\]
	dont nous prenons une approximation avec $\min_C$ uniquement dans l'implémentation, ce qui est consistant avec les portées d'interaction relativement faibles considérées. La complexité est en $O(N_{clusters}^2\cdot N^2)$.
\end{enumerate}
}


\paragraph{Network growth}{Croissance du réseau}


Les infrastructures potentielles, au nombre de $N_I$ lors de la recherche heuristique d'une infrastructure optimale, sont tirées aléatoirement parmi l'ensemble des infrastructures possibles ayant une extrémité au centre d'une cellule. Si l'extrémité est à une distance inférieure à un seuil $\theta_I$ d'un lien déjà existant du réseau, celle-ci est remplacée par sa projection sur le lien correspondant. Il s'agit de l'étape d'accrochage permettant d'obtenir un réseau de forme raisonnable localement. En cohérence avec la représentation raster du réseau, nous prenons $\theta_I = 1$, ce qui correspond à la taille d'une cellule. 






%%%%%%%%%%%%%%%%%%
\subsection{Setup}{Initialisation}


\subsubsection{Synthetic setup}{Initialisation synthétique}

\bpar{
This section describes the setup in the case of synthetic configurations of MCR.
}{
Nous décrivons ici les détails de l'initialisation synthétique.
}


\bpar{
Initial distribution of Actives and Employments is done around governance centers at positions $\vec{x}_i$ using exponential kernels by
}{
Les distributions initiales des actifs et des emplois dans la configuration synthétique sont pris autour des centres de gouvernance (maires) aux positions $\vec{x}_i$ avec des noyaux exponentiels par
}

\[
A(\vec{x}) = A_{max} \cdot \exp{\left(\frac{\norm{\vec{x}-\vec{x}_i}}{r_A}\right)} ; 
E(\vec{x}) = E_{max} \cdot \exp{\left(\frac{\norm{\vec{x}-\vec{x}_i}}{r_E}\right)}
\]



\subsubsection{Setup on a real configuration}{Initialisation sur configuration réelle}

Nous montrons en Fig.~\ref{fig:app:lutecia:realsetup} la population et les réseaux sur lesquels les expériences sur données réelles sont menées : à usage du sol fixe,
\begin{itemize}
	\item une expérience sans réseau initial, et avec pour réseau cible de calibration le réseau de 2010 ;
	\item une expérience avec le réseau initial de 2010, et pour réseau cible le réseau planifié.
\end{itemize} 


%%%%%%%%%%%%%
\begin{figure}
	%\includegraphics[width=\linewidth]{Figures/Lutecia/ex_real_filesetup.png}
	%\includegraphics[width=\linewidth]{Figures/Lutecia/realnonw_nolu.png}
	\includegraphics[width=\linewidth]{Figures/Final/A-lutecia-realsetup.jpg}
	\appcaption{\textbf{Setup on real data used for model application.}\label{fig:app:lutecia:realsetup}}{\textbf{Initialisation sur données réelles utilisée lors de l'application.} Pour des raisons de performances computationnelle, le nombre de cellules est ici diminué par rapport à l'illustration en texte principal. (\textit{Gauche}) Réseaux à l'initialisation, en rouge le réseau initial correspondant au réseau en 2010, en violet fin le réseau cible pour la calibration, correspondant au réseau planifié. (\textit{Droite}) Résultat obtenu avec $\alpha = 0$ à $t_f = 11$ après une initialisation sans réseau ; en bleu le réseau cible, qui correspond au réseau de 2010.\label{fig:app:lutecia:realsetup}}
\end{figure}
%%%%%%%%%%%%%

 









