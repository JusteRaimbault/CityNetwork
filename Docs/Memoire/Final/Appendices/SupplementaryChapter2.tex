




%----------------------------------------------------------------------------------------


%%%%%%%%%%%%%%%%%%%%%%%
\section{Quantitative Epistemology}{Epistémologie Quantitative}

\label{app:sec:quantepistemo}


%----------------------------------------------------------------------------------------

\subsection{Algorithmic systematic review}{Revue systématique algorithmique}

\paragraph{Implementation}{Implémentation}


\bpar{
Because of the heterogeneity of operations required by the algorithm (references organisation, catalog requests, text processing), it was found a reasonable choice to implement it in Java. Source code is available on the Github repository of the project\footnote{at \texttt{https://github.com/JusteRaimbault/CityNetwork/tree/master/Models/QuantEpsitemo/AlgoSR}}. Catalog request, consisting in retrieving a set of references from a set of keywords, is done using the Mendeley software API \cite{mendeley} as it allows an open access to a large database. Keyword extraction is done by Natural Language Processing (NLP) techniques, following the workflow given in \cite{chavalarias2013phylomemetic}, calling a Python script that uses \cite{bird2006nltk}.
}{
De par l'hétérogénéité des opérations requises par l'algorithme (organisation des références, requêtes au catalogue, analyse textuelle), le language Java s'est présenté comme une alternative raisonnable. Le code source est disponible sur le dépôt ouvert du projet\footnote{à l'adresse \texttt{https://github.com/JusteRaimbault/CityNetwork/tree/master/Models/QuantEpistemo/AlgoSR}}. Les requêtes au catalogue, qui consistent à récupérer un ensemble de références à partir d'un ensemble de mots-clés, sont faites via l'API du logiciel Mendeley~\cite{mendeley} qui permet un accès ouvert à une base de données conséquente. L'extraction des mots-clés est effectuée par techniques d'Analyse Textuelle (NLP) selon le processus donné dans~\cite{chavalarias2013phylomemetic}, via un script Python qui utilise~\cite{bird2006nltk}.
}


\paragraph{Convergence and Sensitivity Analysis}{Convergence et analyse de sensibilité}


\comment{(Florent) avec quels mots clés as tu validé empiriquement la convergence de l'algo?}

\bpar{
A formal proof of algorithm convergence is not possible as it will depend on the empirical unknown structure of request results and keywords extraction. We need thus to study empirically its behavior. Good convergence properties but various sensitivities to $N_k$ were found as presented in Fig.~\ref{fig:quantepistemo:sensitivity}. We also studied the internal lexical consistence of final corpuses as a function of keywords number. As expected, small number yields more consistent corpuses, but the variability when increasing stays reasonable.
}{
Une preuve formelle de convergence de l'algorithme n'est guère envisageable puisque qu'elle dépendra de la structure empirique inconnue des résultats de requête et d'extraction de mots-clés. Il est donc nécessaire d'étudier le comportement de l'algorithme de manière empirique. Comme présenté en figure~\ref{fig:quantepistemo:sensitivity}, l'algorithme a de bonnes propriétés de convergence mais diverse sensibilités à $N_k$. Nous étudions également la cohérence lexicale interne des corpus finaux et fonction du nombre de mots-clés. Comme attendu, des valeurs faibles produisent des corpus plus cohérents, mais la variabilité lorsque qu'elles augmentent reste raisonnable.
}



%%%%%%%%%%%%%%%%%%%%%%%%%%%%
\begin{figure}
\centering
\includegraphics[width=\textwidth]{Figures/QuantEpistemo/explo} % Convergence and sensitivity analysis of systematic review algorithm
\medskip
\includegraphics[width=0.8\textwidth]{Figures/QuantEpistemo/lexicalConsistence_MeanSd}
\caption[][]{Convergence and sensitivity analysis. Left : Plots of number of references as a function of iteration, for various queries linked to our theme (see further), for various values of $N_k$ (from 2 to 30). We obtain a rapid convergence for most cases, around 10 iterations needed. Final number of references appears to be very sensitive to keyword number depending on queries, what seems logical since encountered landscape should strongly vary depending on terms. Right : Mean lexical consistence and standard error bars for various queries, as a function of keyword number. Lexical consistence is defined though co-occurrences of keywords by, with $N$ final number of keywords, $f$ final step, and $c(i)$ co-occurrences in references, $k = \frac{2}{N(N-1)}\cdot \sum_{i,j \in \mathcal{K}_f}{\left| c(i) - c(j) \right|}$. The stability confirms the consistence of final corpuses.}{\comment{(Florent) illisible}}
\label{fig:quantepistemo:sensitivity}
\end{figure}
%%%%%%%%%%%%%%%%%%%%%%%%%%%%











%----------------------------------------------------------------------------------------

\subsection{Hypernetwork analysis}{Analyse par hyperréseau}


\paragraph{Initial corpus}{Corpus initial}


Le tableau~\ref{tab:app:quantepistemo:corpus} donne la composition du corpus par domaines.

\begin{table}
% a completer
\caption{}
\label{tab:app:quantepistemo:corpus}
\end{table}





\paragraph{Sensitivity analysis}{Analyse de sensibilité}

L'analyse de sensibilité permettant de fixer les paramètres optimaux pour le réseau sémantique est montrée en Fig.~\ref{fig:app:quantepistemo:sensitivity}.




%%%%%%%%%%%%%%%%%%
\begin{figure}
\includegraphics[width=0.49\linewidth]{Figures/Quantepistemo/pareto-com-vertices}
\includegraphics[width=0.49\linewidth]{Figures/Quantepistemo/pareto-modularity-vertices}\\
\includegraphics[width=\linewidth]{Figures/Quantepistemo/sensitivity_freqmin0_normalized}
\caption{Sensitivity analysis of network indicators to filtering parameters}
\label{fig:app:quantepistemo:sensitivity}
\end{figure}
%%%%%%%%%%%%%%%%%%






\stars





%----------------------------------------------------------------------------------------

\newpage

%%%%%%%%%%%%%%%%%%%%%%%
\section{Modelography}{Modélographie}

\label{app:sec:modelography}


%----------------------------------------------------------------------------------------



\subsection{Systematic Review Methodology}{Méthodologie de la revue systématique}


Choix possible : extraire les mots-clés pertinents par sous-communautés du réseau de citations, puis prendre les plus pertinents ensuite.
ou (a priori ce que l'on fait) extraire sur le corpus complet, puis récupérer par sous-communautés. Pour un petit corpus, deuxième plus souhaitable, notion de pertinence moins importante que pour du big-data (mentionner Patents et Cybergeo, ressemblances et différences).












