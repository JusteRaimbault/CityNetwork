%----------------------------------------------------------------------------------------

\newpage

\section{Softwares and Packages}{Softwares and Packages} % Chapter title

\label{app:packages} % For referencing the chapter elsewhere, use \autoref{ch:name} 

%----------------------------------------------------------------------------------------


%\headercit{}{}{}


\bpar{
This appendix lists and describes the different open datasets created and used in the thesis.
}{
Cette annexe recense les contributions logicielles significatives, qui ont fait l'objet d'un \emph{packaging} dans l'esprit d'une science ouverte.
}


% PUBLIER ces logiciels / packages sur des repos après cleaning !

% Q : also density generator ; space matters : generic workflows / grids ?


%%%%%%%%%%%%%%%%%%
\subsection{largeNetwoRk: Network Import and simplification for R}{largeNetwoRk : Import de réseau et simplification pour R}



%%%%%%%%%%%%%%%%%%
\subsection{Scientific Corpus Mining}{Fouille de Corpus scientifique}






%%%%%%%%%%%%%%%%%%
\subsection{Transportation networks and accessibility in R}{Réseaux de transports et accessibilité en R}







%%%%%%%%%%%%%%%%%%
\subsection{The morphology NetLogo extension to measure Urban Form}{morphology : extension NetLogo pour mesurer la forme urbaine}

\label{app:subsec:morphologyextension}


Disponible à \url{https://github.com/JusteRaimbault/nl-spatialmorphology}












