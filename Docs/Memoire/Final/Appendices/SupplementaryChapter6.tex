





%----------------------------------------------------------------------------------------

\newpage

%%%%%%%%%%%%%%%%%%%%%%%
\section{Exploration of the SimpopNet model}{Exploration du modèle SimpopNet}

\label{app:sec:macrocoevolexplo}


Nous donnons ici des figures supplémentaires permettant de se rendre compte de la sensibilité des résultats aux paramètres non présentés en texte principal.


La Fig.~\ref{fig:app:macrocoevolexplo:closeness} permet de visualiser la sensibilité de l'entropie des centralités $\varepsilon \left[\mu_i\right]$ en fonction de $d_G$, $\theta_N$ et $\gamma_G$. La forme des courbes temporelles est principalement sensible à $\gamma_G$.


%%%%%%%%%%%%%%%%%
\begin{figure}
    %\includegraphics[width=0.48\linewidth]{Figures/MacroCoEvolExplo/closenessEntropies_networkGamma2_5_networkSpeed110.pdf}
    \includegraphics[width=\linewidth]{Figures/Final/A-macrocoevolexplo-closeness.jpg}
\appcaption{\textbf{Entropy of closeness centralities.}\label{fig:app:macrocoevolexplo:closeness}}{\textbf{Entropie des centralités de proximité.} Nous donnons $\varepsilon \left[\mu_i\right]$ en fonction du temps $t$, pour $\theta_N$ variable (couleur), $d_G$ variable (colonnes) et $\gamma_G$ variable.\label{fig:app:macrocoevolexplo:closeness}}
\end{figure}
%%%%%%%%%%%%%%%%%


La Fig.~\ref{fig:app:macrocoevolexplo:rankcorrpop} donne les variations de $\rho_r$ en fonction de $d_G$ et $\gamma_G$, pour des valeurs variables de $\theta_N$ et de $\gamma_N$. Nous constatons que la régularité observée en fonction de $d_G$ et de $\gamma_G$ n'est pas visiblement sensible aux variations de $\theta_N$ et de $\gamma_N$.


%%%%%%%%%%%%%%%%%
\begin{figure}
    %\includegraphics[width=0.48\linewidth]{Figures/MacroCoEvolExplo/rankCorrPop_synthRankSize0_5_networkSpeed10}
    \includegraphics[width=\linewidth]{Figures/Final/A-macrocoevolexplo-rankcorrpop.jpg}
\appcaption{\textbf{Population rank correlations.}\label{fig:app:macrocoevolexplo:rankcorrpop}}{\textbf{Corrélations de rang pour la population.} Nous donnons $\rho_r \left[\mu_i\right]$ en fonction de $d_G$, pour $\gamma_G$ variable (couleur), $\theta_N$ variable (colonnes) et $\gamma_N$ variable (lignes).\label{fig:app:macrocoevolexplo:rankcorrpop}}
\end{figure}
%%%%%%%%%%%%%%%%%


La Fig.~\ref{fig:app:macrocoevolexplo:distcorrs} donne les corrélations $\rho_d$ en fonction de la distance pour l'ensemble des couples de variables, pour $d_G$ et $\gamma_G$ variables. Nous retrouvons qualitativement les mêmes comportements que avec $d_G = 0.016$, à l'exception d'une très légère croissance pour les plus grande distances, pour la corrélation entre la population et l'accessibilité, à $d_G=0.001$ et $\gamma_G = 0.5$, qui reste difficile à interpréter.


%%%%%%%%%%%%%%%%%
\begin{figure}
	%\includegraphics[width=0.48\linewidth]{Figures/MacroCoEvolExplo/distcorrs_networkGamma2_5_networkThreshold21_networkSpeed10}
	\includegraphics[width=\linewidth]{Figures/Final/A-macrocoevolexplo-distcorrs.jpg}
	\appcaption{\textbf{Distance correlation.}\label{fig:app:macrocoevolexplo:distcorrs}}{\textbf{Corrélations en fonction de la distance.} Nous donnons les corrélations $\rho_d$ en fonction du décile de la distance, pour l'ensemble des couples de variables (couleur), pour $d_G$ variable (colonnes) et $\gamma_G$ variable (lignes), à $\gamma_N = 2.5$, $\theta_N=21$ et $v_0 = 10$ fixés.\label{fig:app:macrocoevolexplo:distcorrs}}
\end{figure}
%%%%%%%%%%%%%%%%%

Enfin, nous donnons en Fig.~\ref{fig:app:macrocoevolexplo:laggedcorrs} les corrélations retardées $\rho_{\tau}$ entre l'ensemble des couples de variables, pour $d_G$ et $\gamma_G$ variables. De même, les comportements qualitatifs sont globalement stables pour les paramètres autres que $\gamma_G$.


%%%%%%%%%%%%%%%%%
\begin{figure}
	%\includegraphics[width=0.48\linewidth]{Figures/MacroCoEvolExplo/laggedcorrs_networkGamma2_5_networkThreshold21_networkSpeed10}
	\includegraphics[width=\linewidth]{Figures/Final/A-macrocoevolexplo-laggedcorrs.jpg}
	\appcaption{\textbf{Lagged correlations.}\label{fig:app:macrocoevolexplo:laggedcorrs}}{\textbf{Corrélations retardées.} Nous donnons les corrélations retardées $\rho_{\tau}$ en fonction du délai $\tau$, pour l'ensemble des couples de variables (couleur), pour $d_G$ variable (colonnes) et $\gamma_G$ variable (lignes), à $\gamma_N = 2.5$, $\theta_N=21$ et $v_0 = 10$ fixés.\label{fig:app:macrocoevolexplo:laggedcorrs}}
\end{figure}
%%%%%%%%%%%%%%%%%




%----------------------------------------------------------------------------------------

\newpage

%%%%%%%%%%%%%%%%%%%%%%%
\section{Macroscopic co-evolution model}{Modèle de co-évolution macroscopique}

\label{app:sec:macrocoevol}



%%%%%%%%%%%%%%%%%%%
\subsection{Synthetic data}{Données synthétiques}


Nous donnons en Fig.~\ref{fig:app:macrocoevol:behavior-time} la sensibilité des indicateurs temporels pour le modèle de co-évolution sur données synthétiques, en particulier $\bar{c_i}(t)$ et $\varepsilon\left[\mu_i\right](t)$, pour des variations de $d_G$, $\gamma_G$ et $\phi_0$. Le comportement de $\bar{c_i}$ est sensible à $\gamma_G$ et $\phi_0$ mais très peu à $d_G$. Celui de $\varepsilon\left[\mu_i\right]$ ne dépend que de $\gamma_G$ pour son comportement moyen, et de $d_G$ pour sa dispersion dans les faibles valeurs de $d_G$.


%%%%%%%%%%%%%
\begin{figure}
%\includegraphics[width=0.48\linewidth]{Figures/MacroCoEvol/closenessSummaries_meansynthRankSize1_gravityWeight0_001.pdf}
%\includegraphics[width=0.48\linewidth]{Figures/MacroCoEvol/populationEntropiessynthRankSize1_gravityWeight0_001.pdf}
\includegraphics[width=\linewidth,height=0.9\textheight]{Figures/Final/A-macrocoevol-behavior-time.jpg}
\appcaption{\textbf{Behavior of the co-evolution model.}\label{fig:macrocoevol:behavior-time}}{\textbf{Comportement d'indicateurs temporels pour le modèle de co-évolution à l'échelle macroscopique.} \textit{(Haut)} Moyenne des centralités de proximité, en fonction du temps, pour $d_G$ (colonnes), $\gamma_G$ (lignes) et $\phi_0$(couleur) variables, à $w_G = 0.001$ fixé ; \textit{(Bas)} Entropie de populations, en fonction du temps, pour $d_G$ (colonnes), $\gamma_G$ (lignes) et $\phi_0$(couleur) variables, à $w_G = 0.001$ fixé.\label{fig:app:macrocoevol:behavior-time}}
\end{figure}
%%%%%%%%%%%%%




Nous donnons en Fig.~\ref{fig:app:macrocoevol:behavior-aggreg} le comportement d'indicateurs agrégés, à savoir $C\left[Z_i\right]$ et $\rho_r \left[Z_i\right]$. La complexité des trajectoires d'accessibilité varie principalement selon $d_G$, $\gamma_G$ et $\phi_0$ pour les faibles valeurs. La corrélation de rang des accessibilités est quant à elle uniquement sensible à $d_G$ et $\gamma_G$, ce qui veut dire que des différences d'évolution du réseau ne perturbent pas la dynamique de la hiérarchie des accessibilités.



%%%%%%%%%%%%%
\begin{figure}
%\includegraphics[width=0.48\linewidth]{Figures/MacroCoEvol/complexityAccessibility_synthrankSize1_nwGmax0_05}
%\includegraphics[width=0.48\linewidth]{Figures/MacroCoEvol/rankCorrAccessibility_synthrankSize1_nwGmax0_05}
\includegraphics[width=\linewidth,height=0.9\textheight]{Figures/Final/A-macrocoevol-behavior-aggreg.jpg}
\appcaption{\textbf{Aggregated indicators behavior for the model of coevolution at the macroscopic scale.}\label{fig:app:macrocoevol:behavior-aggreg}}{\textbf{Comportement d'indicateurs agrégés pour le modèle de co-évolution à l'échelle macroscopique.} \textit{(Haut)} Complexité des accessibilités, en fonction de $d_G$, pour $\phi_0$ (colonnes), $w_G$ (lignes) et $\gamma_G$ (couleur) variables ; \textit{(Bas)} Corrélations de rang des accessibilités, pour les mêmes paramètres.\label{fig:app:macrocoevol:behavior-aggreg}}
\end{figure}
%%%%%%%%%%%%%


La Fig.~\ref{fig:app:macrocoevol:distcorrs} donne les corrélations $\rho_d$ en fonction des déciles de distance pour l'ensemble des couples de variables. Les fortes valeurs de $d_G$ donnent des corrélations nulles pour l'ensemble des valeurs de la distance, tandis que $d_G = 10$ témoigne de régimes locaux. Une corrélation constante entre centralité et accessibilité émerge pour une valeur intermédiaire $d_G = 60$, qui est éventuellement à mettre en correspondance avec le maximum de complexité pour les accessibilités obtenu précédemment.


%%%%%%%%%%%%%
\begin{figure}
%\includegraphics[width=0.9\linewidth]{Figures/MacroCoEvol/distcorrs_gravityWeight5e-04_nwThreshold4_5}
\includegraphics[width=\linewidth]{Figures/Final/A-macrocoevol-distcorrs.jpg}
\appcaption{\label{fig:app:macrocoevol:distcorrs}}{\textbf{Corrélations en fonction de la distance.} Correlation $\rho_d$ entre couples de variables (donné par la couleur), en fonction de la distance $d$ (discretisée en déciles), pour $d_G$ variable (colonnes) et $\gamma_G$ variable (lignes), à $w_G = 5e-4$ et $\phi_0 = 4.5$.
\label{fig:app:macrocoevol:distcorrs}}
\end{figure}
%%%%%%%%%%%%%


Enfin, la Fig.~\ref{fig:app:macrocoevol:laggedcorrs} donne les corrélations retardées $\rho_{\tau}$ pour l'ensemble des couples de variables. Les variations de $\gamma_G$ influencent peu les régimes obtenus, contrairement à $d_G$, pour lequel on observe une variation continue de la forme qualitative des profils.

%%%%%%%%%%%%%
\begin{figure}
%\includegraphics[width=0.9\linewidth]{Figures/MacroCoEvol/laggedcorrs_gravityWeight5e-04_nwThreshold4_5}
\includegraphics[width=\linewidth]{Figures/Final/A-macrocoevol-laggedcorrs.jpg}
\appcaption{\label{fig:app:macrocoevol:distcorrs}}{\textbf{Corrélations retardées.} Correlations retardées $\rho_{\tau}$ en fonction du retard $\tau$, de manière similaire pour $d_G$ variable (colonnes) et $\gamma_G$ variable (lignes), à $w_G = 5e-4$ et $\phi_0 = 4.5$.
\label{fig:app:macrocoevol:laggedcorrs}}
\end{figure}
%%%%%%%%%%%%%












%%%%%%%%%%%%%%%%%%%
\subsection{Real data}{Données réelles}





%%%%%%%%%%%%%%%%%%%
\begin{figure}
%\includegraphics[width=0.9\linewidth]{Figures/MacroCoEvol/pareto_gravityDecay}
\includegraphics[width=\linewidth]{Figures/Final/A-macrocoevol-pareto.jpg}
\appcaption{\label{fig:macrocoevol:pareto}}{\textbf{Fronts de Pareto pour la calibration bi-objectif population et distance.} Les fronts sont donnés pour chaque période de calibration, et colorés en fonction de $d_G$ (Bas).\label{fig:app:macrocoevol:pareto}}
\end{figure}
%%%%%%%%%%%%%%%%%%%








