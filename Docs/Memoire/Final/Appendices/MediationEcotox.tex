

%--------------------------------

\newpage


\section{Gamed-based tools as media to transmit freshwater ecology concepts}{Communication scientifique par la gamification}

\label{app:sec:mediationecotox}


% illustration of sci communication



\stars

\textit{Cette section est le fruit d'une collaboration interdisciplinaire avec \noun{Dr. Hélène Serra}}


\stars

%--------------------------------


\subsection{Introduction}{Introduction}

\bpar{
There is an increasing expectation on people to be aware and to get involved in the environmental issues that our world is facing. However, expert knowledge is often required to understand most of these issues. One of the challenges in science today lies in explaining complex issues in a simple and understandable way to an unspecialized audience. Games can turn out to be a good medium for scientific vulgarization. Indeed, the first form of learning we all experienced was by playing. Games are very popular, and from an educational point of view, they present many advantages. They are dynamic and interactive. Therefore, the player engagement increases, as well as its knowledge retention. In addition, the player is immerged into a new world and discovers a virtual environment where he needs to develop strategies and to identify crucial processes. Those characteristics can be wisely used to spread scientific topics, and gamification has already been proposed as a tool for an easier propagation of scientific thinking [1] such as in pharmacology [2] or geosciences [3]. In this context, our project aims at developing game-based tools to transmit the basic concepts of freshwater ecology. We choose to focus on a classical board game and on a computer based game because they are complementary in the targeted audience (groups versus online gamers) and the possibilities offered, in particular regarding the interactions between players and the system dynamics. 
}{
L'attente de prise de conscience et d'implication pour le public concernant les questions environnementales est croissante. Toutefois, une connaissance experte est souvent nécessaire pour comprendre le enjeux sous-jacents à la plupart de ces problèmes. L'un des défis de la science aujourd'hui réside dans le fait d'expliquer des questions complexes de façon simple et compréhensible à une audience non-spécialisée. Les jeux apparaissent comme un medium pertinent pour la vulgarisation scientifique. En effet, la première forme d'apprentissage est en général par le jeu. Les jeux sont très populaires et présentent divers avantages d'un point de vue éducatif. Ceux-ci sont dynamiques et interactifs. Ainsi, l'engagement du joueur est augmenté, ainsi que sa rétention de connaissances. De plus, le joueur est immergé dans un monde nouveau et découvre un environnement virtuel où il doit développer des stratégies et identifier les processus fondamentaux. Ces caractéristiques peuvent être aisément utilisées pour transmettre des concepts scientifiques, et la gamification a déjà été proposée comme un outil pour une meilleure propagation de la pensée scientifique~\cite{} comme en pharmacologie~\cite{} ou les géosciences~\cite{}. Dans ce contexte, ce projet vise à développer des outils basés sur les jeux pour transmettre des concepts basiques en écologie d'eau douce. Nous nous intéressons à un jeu de plateau classique et à un jeu informatique car ceux-ci sont complémentaire dans l'audience visée (joueurs en groupe et joueurs en ligne) et dans les possibilités offertes, en particulier concernant les interactions entre joueurs et les dynamiques du système. 
}


\subsection{Material and Methods}{Méthodologie}

\bpar{
The general methodology is divided in five steps: (1) selection of species; (2) definition of the instructions (object, game board, rules); (3) incorporation of environmental stressors (biotic and abiotic), (4) design and construction of interfaces (board and computer model); (5) test with players. All steps are necessarily interdependent and are tackled in parallel during the development of the games.
}{
La méthodologie pour la conception des deux types de jeux est divisée de manière similaire en 5 étapes : (1) sélection des espèces ; (2) définition des instructions (objets, environnement du jeu, règles) ; (3) inclusion des stress environnementaux (biotiques et abiotiques) ; (4) conception et construction des interfaces (plateau et implémentation informatique) ; (5) test avec des joueurs. L'ensemble des étapes sont interdépendantes et sont menées en parallèle pendant le développement des jeux.
}


\bpar{
While the board game is inspired by past experiences of player, the computer game is based on a model of simulation of the ecosystem. In order to introduce notions of equilibrium and its perturbations that occur at a larger time scale than on the board game, we propose to implement an agent-based model (ABM) and to couple its dynamics with gaming actions. ABM have already been widely used in ecology [4]. Therefore, we selected a trophic chain dynamic model (extended prey-predator model) that can capture fish behavioral rules and spatially heterogeneous environment. It is particularly suitable for the game implementation: fish behaviors are influenced by players whereas the ecosystem is disturbed by external events. 
}{
Tandis que le jeu de plateau est inspiré d'experiences de joueurs, le jeu informatique se base sur un modèle de simulation de l'écosystème. De manière à introduire les notions d'équilibre et ses perturbations qui surviennent à une échelle de temps plus longue que celle du jeu de plateau, nous proposons d'implémenter un modèle basé-agent (ABM) et de coupler sa dynamique avec des actions de jeu. Les ABM sont déjà largement utilisés en écologie~\cite{grimm2005pattern}. Ainsi, nous choisissons un modèle dynamique de chaine trophique (modèle proie-prédateur étendu) qui est capable d'inclure des règles comportementales pour les poissons et un environnement spatial hétérogène. Un tel modèle est particulièrement adapté pour l'implémentation du jeu : les comportements des poissons sont influencés par les joueurs tandis que l'écosystème est perturbé par des évènements extérieurs.
}





















