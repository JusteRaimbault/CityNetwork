

%--------------------------------

\newpage


\section{Gamed-based tools as media to transmit freshwater ecology concepts}{Communication scientifique par la gamification}

\label{app:sec:mediationecotox}


% illustration of sci communication



\stars

\textit{Cette section est une collaboration interdisciplinaire avec \noun{Dr. Hélène Serra}}


\stars

%--------------------------------


\subsection{Introduction}{Introduction}

There is an increasing expectation on people to be aware and to get involved in the environmental issues that our world is facing. However, expert knowledge is often required to understand most of these issues. One of the challenges in science today lies in explaining complex issues in a simple and understandable way to an unspecialized audience. Games can turn out to be a good medium for scientific vulgarization. Indeed, the first form of learning we all experienced was by playing. Games are very popular, and from an educational point of view, they present many advantages. They are dynamic and interactive. Therefore, the player engagement increases, as well as its knowledge retention. In addition, the player is immerged into a new world and discovers a virtual environment where he needs to develop strategies and to indentify crucial processes. Those characteristics can be wisely used to spread scientific topics, and gamification has already been proposed as a tool for an easier propagation of scientific thinking [1] such as in pharmacology [2] or geosciences [3]. In this context, our project aims at developing game-based tools to transmit the basic concepts of freshwater ecology. We choose to focus on a classical board game and on a computer based game because they are complementary in the targeted audience (groups versus online gamers) and the possibilities offered, in particular regarding the interactions between players and the system dynamics. 


\subsection{Material and Methods}{}

The general methodology is divided in five steps: (1) selection of species; (2) definition of the instructions (object, game board, rules); (3) incorporation of environmental stressors (biotic and abiotic), (4) design and construction of interfaces (board and computer model); (5) test with players. All steps are necessarily interdependent and are tackled in parallel during the development of the games.

While the board game is inspired by past experiences of player, the computer game is based on a model of simulation of the ecosystem. In order to introduce notions of equilibrium and its perturbations that occur at a larger time scale than on the board game, we propose to implement an agent-based model (ABM) and to couple its dynamics with gaming actions. ABM have already been widely used in ecology [4]. Therefore, we selected a trophic chain dynamic model (extended prey-predator model) that can capture fish behavioral rules and spatially heterogeneous environment. It is particularly suitable for the game implementation: fish behaviors are influenced by players whereas the ecosystem is disturbed by external events. 

