

%----------------------------------------------------------------------------------------

\newpage


\section*{Chapter Conclusion}{Conclusion du Chapitre}


La lecture d'un article ou d'un ouvrage est toujours bien plus éclairante lorsqu'on connait personnellement l'auteur, d'une part car on peut profiter des \emph{private joke} et extrapoler certains développements des narrations qui se doivent synthétique (même si l'art de l'écriture est justement d'essayer de transmettre la majorité de ces éléments, l'ambiance en quelque sorte), et d'autre part car la personnalité a des implications complexes sur la manière d'appréhender la nature de la connaissance et une certaine structure a priori du monde. Pour cela, la connaissance scientifique serait très probablement moins riche si elle était produite par des machines aux capacités cognitives équivalentes, aux connaissances et experiences empiriques subjectives équivalentes et aussi diverses que celles humaines, mais qui auraient été programmées pour minimiser l'impact de leur personnalité et de leur convictions sur l'écriture et la communication (toujours en supposant qu'elles aient une certaine forme de données et fonctions plus ou moins équivalentes). Dans ces laboratoires de recherche dignes de \emph{Blade Runner}, nous doutons que la production d'une pensée complexe serait effectivement possible, puisqu'il manquerait à ces machines justement la \emph{Rationalité Evolutive} développée en~\ref{sec:epistemology}, et nous doutons fortement que celle-ci puisse être produite du moins dans l'état des connaissances actuelles en intelligence artificielle. Le but de ce chapitre était donc ``de faire connaissance'' sur les points de positionnements incontournables pour l'ensemble de notre réflexion. Ceux-ci en sont d'autant plus en rien superflus car conditionnent très fortement certaines directions de recherche. Notre positionnement sur la reproductibilité développé en~\ref{sec:reproducibility} impliquent certains choix de modélisation, notamment l'utilisation univoque de plateformes ouvertes, de workflow et d'implémentations ouverts ; il implique aussi un choix de données qui se doivent au maximum d'être accessibles ou rendues accessibles, et donc certains d'objets et d'ontologie, ou plutôt le non-choix de certains : nos problématiques pourraient être mobilisées sur des données d'entreprise fines tout en gardant une cohérence avec l'approche théorique et thématique (la théorie évolutive a largement mobilisé ce type d'étude comme par exemple~\cite{paulus2004coevolution}), mais la relative fermeture de ce type de données ne les rend pas utilisables dans notre démarche. Ensuite, notre positionnement sur le rôle du calcul intensif et les besoins d'exploration des modèles~\ref{sec:computation} est source de l'ensemble des expériences numériques et des méthodologies utilisées ou développées. Enfin, notre positionnement épistémologique~\ref{sec:epistemology} percole dans l'ensemble de notre travail, et permet de poser les premières briques pour des formalisations théoriques plus systématiques qui seront développées en Chapitre~\ref{ch:theory}.


%\vspace{-1cm} \stars
