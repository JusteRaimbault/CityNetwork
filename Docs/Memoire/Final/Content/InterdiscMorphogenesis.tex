




\newpage

%----------------------------------------------------------------------------------------

\section{An Interdisciplinary Approach to Morphogenesis}{Une Approche Interdisciplinaire de la Morphogenèse}




\bpar{
A first crucial step is a clarification of what is meant by morphogenesis. Initially introduced in biology, its transfert to other field was accompanied by a deformation of associated concepts. We adapt here the text of~\cite{antelope2016interdisciplinary} which proposes an interdisciplinary entry on morphogenesis. As an essential building block of our constructions, it is indeed necessary to give it a rigorous and clear skeleton. We make the choice of a crossed approach, in the spirit of an applied perspectivism as introduced in section~\ref{sec:epistemo-position}, to obtain concepts as generic and broad as possible.
}{
Une première étape essentielle est la clarification de ce qui est entendu par le terme de morphogenèse. Initialement introduit en biologie, son transfert à d'autres champs s'est accompagné d'une déformation des concepts associés. Nous adaptons et traduisons ici le texte de~\cite{antelope2016interdisciplinary} qui propose une entrée interdisciplinaire sur la morphogenèse. Brique essentielle de nos constructions, il est en effet crucial de lui donner une armature rigoureuse et claire. Nous prenons le parti d'une vision croisée, dans l'idée d'un perspectivisme appliqué comme introduit en section~\ref{sec:epistemo-position}, pour obtenir des concepts aussi génériques et larges que possible.
}



\paragraph{Objective}{Objectif}



\paragraph{Context}{Contexte}




\subsection{Reviews}{Revues}

\subsubsection*{Developmental Biology}{Biologie du Développement}


\subsubsection*{Artificial Intelligence}{Intelligence Artificielle}



\subsubsection*{Territorial Sciences}{Sciences Territoriales}


\subsubsection*{Social Science and Psychology}{Sciences Sociales et Psychologie}


\subsubsection*{Others}{Autres}


\paragraph{A mathematical approach}{Une Approche Mathématique}


\bpar{
Ren{\'e} Thom, in \emph{Structural stability and Morphogenesis}~\cite{thom1974stabilite} has developed a theory of system dynamics, the ``catastrophe theory'', that studies in deep the impact of topological structure of phase space manifolds on a system dynamics. Let $M$ a differentiable manifold, in which system state $(m,\dot{m})$ is embedded. We assume the existence of a closed set $K$, called \emph{Catastrophe set}. The topological type of $K$ is indeed endogenously determined by system dynamics (in simple cases, it refers to the "classical" type of attractors/fixed points usually known: points, limit cycles). When $m$ encounters $K$, the system follows a \emph{qualitative} change in its form, what constitutes the basis of \emph{morphogenesis}. This abstract theory of morphogenesis is independent of the nature of the system studied, its main contribution being to classify local catastrophes that occur during morphogenesis. Differentiation and richness of patterns have thus a geometrical explanation through the topological types of catastrophes. Thom notes that at this time, the study of form has mainly be the focus of biology, but that many applications could be done in physics and geomorphology for example. He formulated the hypothesis that it is because it implies discontinuities and self-organisation, to which mathematicians were repulsive, that it was not applied easily to various fields. We can link this to the rise of complexity approaches, with complexity paradigms that slowly spreaded in various disciplines, and the study of morphogenesis seem to have followed.
}{
Ren{\'e} Thom a développé dans \emph{Stabilité Structurelle et Morphogenèse}~\cite{thom1974stabilite} une théorie de la dynamique des systèmes, la théorie des catastrophes, qui étudie en profondeur l'impact de la structure topologique des variétés de l'espace des phases sur les dynamiques du système. Soit $M$ une variété différentiable, dans laquelle l'état du système $(m,\dot{m})$ est embarqué. On suppose l'existence d'un ensemble fermé $K$ appelé \emph{Ensemble de Catastrophe}. Le type topologique de $K$ est en fait déterminé de manière endogène par la dynamique du système (dans les cas simples, il réfère au types ``classiques'' d'attracteurs/points fixes que l'on connait usuellement : points et cycles limites). Quand $m$ traverse $K$, le système encontre un changement \emph{qualitatif} dans sa forme, ce qui constitue la base de la \emph{morphogenèse}. Cette théorie abstraite de la morphogenèse est indépendante de la nature du système étudié, sa contribution principale étant de classifier les catastrophes locales qui occurrent lors de la morphogenèse. La différentiation et la richesse des motifs ont ainsi une explication géométrique à travers les types topologiques des catastrophes.
% TODO reclarify link between dynamics and form in Thom's theory
}



\paragraph{Autopoiesis and Morphogenesis}{Autopoièse et Morphogenèse}


\bpar{
The notion of \emph{autopoiesis} expresses the ability for a system to reproduce itself. A basic characterization is a semi-permeable boundary produced within the system and the ability to reproduce its components. A more general definition is proposed by Bourgine and Stewart in~\cite{bourgine2004autopoiesis}: \textit{``An autopoietic system is a network of processes that produces the components that reproduce the network, and that also regulates the boundary conditions necessary for its ongoing existence as a network''}. The notion of dynamical processes is key, and could be linked to Thom's theory of morphogenesis. They furthermore introduce a definition of cognition (trigger actions as function of sensory inputs to ensure viability), and of living organism as autopoietic and cognitive, both notions being distinct~\cite{bitbol_autopoiesis_2004}. In that frame, for example, the arbotron~\cite{jun2005formation} is cognitive but not autopoietic. An example of link between autopoiesis and morphogenesis is shown in~\cite{niizato2010model}, where a type of Physarum organism has to play both on cell mobility and form evolution to be able to collect the food necessary for its survival. At this stage, we can postulate a strict inclusion from autopoietic systems, morphogenetic systems to self-organizing systems.
}{
La notion d'\emph{autopoièse} exprime la capacité d'un système à s'auto-reproduire. Une caractérisation basique est une frontière semi-perméable produite par le système et la capacité à reproduire ses composants. Une définition plus générale est proposée par \noun{Bourgine} et \noun{Stewart} dans~\cite{bourgine2004autopoiesis} :
}



\subsection{Synthesis}{Synthèse}




\subsection{Discussion}{Discussion}










