


% Chapter 

%\chapter{Positioning}{Positionnements} % Chapter title
\chapter{Positionnements}


\label{ch:positioning} % For referencing the chapter elsewhere, use \autoref{ch:name} 

%----------------------------------------------------------------------------------------

%\headercit{}{}{}

\bigskip


Toute activité de recherche serait, selon certains acteurs de celle-ci, nécessairement politisée, de par pour commencer le choix de ses objets. Ainsi, \noun{Ripoll} alerte contre l'illusion d'une recherche objective et les dangers de la technocratie~\cite{ripoll2017jig}. Nous ne rentrerons pas dans ces débats bien trop vastes pour être traités même en un chapitre, puisqu'il rejoignent des thèmes de sciences politiques, d'éthique, de philosophie, liés par exemple à la gouvernance scientifique, à l'insertion de la science dans la société, à la responsabilité scientifique.


 Il est clair que même des sujets a priori intrinsèquement objectifs, comme la physique des particules et des hautes énergies, ont des implications regardant d'une part les choix de leur financements et les externalités associées (par exemple, l'existence du CERN a largement contribué au développement du calcul distribué), mais d'autre part aussi les applications potentielles des découvertes qui peuvent avoir des répercussions sociales considérables. En biologie, l'éthique est au coeur des principes fondateurs des disciplines, comme en témoignent les débats soulevés par l'émergence de la biologie synthétique~\cite{gutmann2011ethics}. Les tenants d'approche prudentes dans celle-ci se recoupent avec la biologie intégrative, or les sciences intégratives défendues par \noun{Paul Bourgine}, mises en oeuvre par l'intermédiaire du campus digital Unesco CS-DC\footnote{\url{https://www.cs-dc.org/}}, ont typiquement la responsabilité sociale et l'implication citoyenne au coeur de leur cercle vertueux. En sciences humaines et sociales, comme les recherches interagissent avec les objets étudiés (en quelque sorte l'idée des \emph{interactive kind} de \noun{Hacking}~\cite{hacking1999social})%\comment[SR]{num de page et-ou citation en note de bas de page pour définir le concept pour le lecteur, ref plus précise en géographie/sys ville comme ex Batty 1976 ??}
 , les implications politiques et sociales de la recherche sont bien évidemment indiscutables.
 
%Là où il y aurait matière à discussion, et nous y reviendrons en ouverture~\ref{ch:opening} car il s'agira d'une des questions ouvertes posées par notre recherche et sa démarche dans leur ensemble, serait sur la compatibilité des méthodes systématiques et \emph{evidence-based} avec les sciences sociales, autrement dit dans quelle mesure peut-on s'extraire de certains dogmatismes encore plus marqués lors de l'usage de théorie politiques\footnote{\noun{Monod} montre par exemple les désastres liés aux ``niaiseries épistémologiques'' découlant de l'application littérale de la dialectique matérialiste marxiste à l'épistémologie du vivant.}.
   
Nous nous placerons ici à un niveau épistémologique, c'est-à-dire à des réflexions sur la nature et le contenu des connaissances scientifiques au sens large, c'est-à-dire co-construites et validées au sein d'une communauté imposant certains critères de scientificité~\cite{morin1991methode}, bien sûr évolutifs puisque nous nous positionnerons pour la systématisation de certains. Mais donc, même en restant à ce niveau, des prises de positions sont nécessaires, celles-ci pouvant être épistémologiques, méthodologiques, thématiques. Ces dernières ont déjà été ébauchées dans les deux chapitres précédents par les choix des objets d'étude, des problématiques, et seront renforcées à mesure de la progression.
% pour finalement être synthétisées en Chapitre~\ref{ch:theory}.
 
 
Nous proposons ainsi ici un exercice relativement original mais que nous jugeons nécessaire pour une lecture plus fluide de la suite. Il consiste en le développement précis de certains positionnements qui ont une influence particulière dans notre démarche de recherche.

Dans une première section (\ref{sec:computation}), nous précisons notre position au regard des modèles de simulation. Après avoir détaillé les fonctions que nous prêterons aux modèles, nous argumentons sous forme d'essai pour un usage raisonné des données massives et du calcul intensif, et illustrons notre positionnement par rapport à l'exploration des modèles par une étude de cas méthodologique pour l'exploration de la sensibilité des modèles aux conditions initiales.

Dans une deuxième section (\ref{sec:reproducibility}), nous développons des exemples pour illustrer le besoin et la difficulté de reproductibilité, ainsi que les liens avec des nouveaux outils pouvant la favoriser mais aussi la mettre en danger. Nous illustrons la question d'ouverture des données et d'exploration interactive par une étude de cas empirique des flux de trafic en Ile-de-France.

Enfin, la dernière section (\ref{sec:epistemology}) explicite modestement des positions épistémologiques, notamment concernant le courant dans lequel nous nous plaçons, la complexité des objets en sciences sociales, et la nature de la complexité de manière générale.

Le lecteur très familier avec les ``commandements'' de \noun{Banos}~\cite{banos2013pour} pourra trouver dans les deux premières sections des illustrations pratiques originales de ceux-ci, notre positionnement étant principalement dans leur lignée.

%\comment[SR]{Théoreme de l'entonnoir de banos, ne vaudrait il mieux pas partir du plus large et finir sur le plus reserré ? Ce qui permettrait de dire en quoi les deux premières sections actuelles (reproductibilité, calcul) servent un tout plus général (complexité en géo) ?}[plus difficile a lire comme cela]


\stars


\textit{Ce chapitre est composé de divers travaux. La première section est inédite pour ses deux premières parties, et pour sa dernière partie reprend des idées présentées dans \cite{cottineau2017initial}. La deuxième section rend compte pour sa première partie du contenu théorique de \cite{raimbault2016cautious}, et reprend \cite{raimbault2017investigating} pour l'illustration empirique. La troisième section reprend dans sa première partie les bases épistémologiques de \cite{raimbault:halshs-01505084} approfondies par \cite{raimbault2017applied}, est inédite pour sa deuxième partie et rend compte de \cite{raimbault2017complex} pour sa dernière partie.
}






