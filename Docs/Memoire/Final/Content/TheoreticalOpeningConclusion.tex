


%----------------------------------------------------------------------------------------

\newpage


\section*{Chapter Conclusion}{Conclusion du Chapitre}


Dans une logique de lecture linéaire, cette ouverture par l'introduction d'un cadre théorique, devrait avoir synthétisé et rassuré sur les questions ouvertes a priori réglées dans leur majorité - seul la conclusion pouvant encore apporter une chute dans la narration. Il s'agit d'un malentendu, et le lecteur qui voudrait être rassuré aurait du s'arrêter à la fin de la troisième partie, à la fin duquel nous avions fait un tour relativement conséquent des approches proposées.


Ce chapitre ouvre en fait un gouffre, et fait prendre conscience que la portée des connaissances est extrêmement embryonnaire. Pour donner une allégorie, nous serions un peu dans la situation du périhélie de Mercure et du spectre de l'atome qui étaient des détails négligeables pour la physique classique à la fin du 19ème siècle, et ont mené aux gigantesques développements au cours du 20ème que sont la physique quantique et la relativité générale.

Les questions soulevées par chacun des niveaux sont fondamentales pour l'étude des systèmes territoriaux complexes mais aussi des systèmes complexes en général. La théorie proposée en~\ref{sec:theory} pointe à nouveau la question de la non-stationnarité spatio-temporelle dans un contexte multi-échelle, que nous postulons cruciale mais peu explorée dans le cas des systèmes territoriaux. Nous distinguons également la difficulté d'intégration de théories existantes ce qui implique une compréhension des processus de couplage des modèles.

Ce problème est au coeur du cadre formel développé par la suite~\ref{app:sec:csframework}, qui soulève aussi des questions d'imbrication d'échelles. Le problème d'obtenir une structure algébrique cohérente avec une action de monoïde sur les données implique une intégration de la théorie de \noun{Krob}, ce qui questionne plus généralement l'intégration des approches d'ingénierie système (systèmes complexes ``industriel'') avec celles de systèmes complexes naturels.

La possibilité de théorie intégratives est soulevée par l'introduction du cadre de connaissance~\ref{sec:knowledgeframework}, qui pose également des problèmes plus généraux de production des connaissances et de nature de la complexité que nous avions brièvement abordé d'un point de vue épistémologique en~\ref{sec:epistemology}.


Nous proposons de synthétiser une partie de ces diverses questions ouvertes dans un projet de recherche cohérent sur un long terme mais incluant des premières pistes concrètes immédiates, que nous présenterons en ouverture.




\stars
