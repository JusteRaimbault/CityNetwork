


%----------------------------------------------------------------------------------------


\newpage



\section{Epistemological Positioning}{Positionnement Epistémologique}\label{sec:epistemology}

%----------------------------------------------------------------------------------------



%%%%%%%%%%%%%%%%%%%%%%
\subsection{Cognitive Approach and Perspectivism}{Approche cognitive et Perspectivisme}



\bpar{
Our epistemological positioning relies on a cognitive approach to science, given by Giere in~\cite{giere2010explaining}. The approach focuses on the role of cognitive agents as carriers and producers of knowledge. It has been shown to be operational by \cite{giere2010agent} that studies an agent-based model of science. These ideas converge with Chavalarias' Nobel Game~\cite{chavalarias2016s} that tests through a stylized model the balance between exploration and falsification in the collective scientific enterprise. This epistemological positioning has been presented by Giere as \emph{scientific perspectivism}~\cite{giere2010scientific}, which main feature is to consider any scientific entreprise as a \emph{perspective} in which \emph{agents} use \emph{media} (models) to represent something with a certain purpose. To make it more concrete, we can position it within Hacking's ``check-list'' of constructivism~\cite{hacking1999social}, a practical tool to position an epistemological position within a simplified three dimensional space which dimensions are different aspects on which realist approaches and constructivist approach generally diverge: first the contingency (path-dependency of the knowledge construction process) is necessary in the pluralist perspectivist approach, secondly the ``degree of constructivism'' is quite high because agents produce knowledge, and finally the stability of theories depends on the complex interaction between the agents and their perspectives. It was presented for these reasons as an intermediate and alternative way between absolute realism and skeptical constructivism~\cite{brown2009models}. The \emph{perspective} plays a central role in the framework.
}{
Notre positionnement épistémologique se fonde sur une approche cognitive de la science, introduite par \noun{Giere} dans~\cite{giere2010explaining}. L'approche se concentre sur le rôle des agents cognitifs comme porteurs et producteurs de la connaissance. Son opérationalité a été montrée par \cite{giere2010agent} qui étudie un modèle basé-agent de la science. Ces idées convergent avec le jeu Nobel de \noun{Chavalarias}~\cite{chavalarias2016s} qui teste de manière stylisée l'équilibre entre exploration et falsification \comment[AB]{est-ce vraiment le bon couple ? A vérifier} dans l'entreprise scientifique collective. Ce positionnement épistémologique a été présenté par \noun{Giere} comme \emph{perspectivisme scientifique}~\cite{giere2010scientific}, dont la caractéristique principale est de considérer toute entreprise scientifique comme une \emph{perspective} dans laquelle des \emph{agents} utilisent des \emph{media} (modèles) pour représenter quelque chose dans un certain but. Pour concrétiser, nous pouvons le positionner sur la \emph{check-list} du constructivisme de \noun{Hacking}~\cite{hacking1999social}, un outil pratique pour positionner une position épistémologique dans un espace simplifié à trois dimensions dans lequel les dimensions sont différents aspects sur lesquels les approches réalistes et constructivistes généralement divergent : d'abord la contingence (dépendance au chemin du processus de construction de connaissances) est nécessaire l'approche perspectiviste qui est pluraliste, deuxièmement le ``degré de constructivisme'' est assez haut car les agents produisent la connaissance, et enfin la stabilité des théories dépend des interactions complexes entre les agents et leur perspectives \comment[AB]{reprendre plus clairement}. Cela a pour ces raisons été présenté comme un chemin intermédiaire et alternatif entre le réalisme absolu et le constructivisme sceptique~\cite{brown2009models}. la notion de \emph{perspective} jouera un rôle fondamental dans le cadre développé en~\ref{sec:knowledgeframework}.
}

Cette approche mettant l'emphase sur l'auto-organisation, nous la voyons totalement compatible avec une vision anarchiste de la science comme défendue par \noun{Feyerabend}~\cite{feyerabend1993against}. Celui-ci émet des doutes sur l'intérêt de l'anarchisme politique mais introduit l'\emph{anarchisme scientifique}, qu'il ne faut pas comprendre comme un refus total de toute méthode ``objective'', mais d'une autorité et légitimité artificielle que certaines méthodes ou courants scientifique pourraient vouloir prendre. Il démontre par une analyse précise des travaux de Galilée que la plupart de ses résultats étaient basés sur des croyances et que la plupart n'étaient pas accessibles avec les outils et méthodes de l'époque, et postule qu'il devrait en être de même pour certains travaux contemporains. Il n'y a donc pas de \emph{perspective} objectivement plus légitimes que d'autres dans la mesure de leurs validation par des faits et des pairs - et même dans ces cas la légitimité doit pouvoir être discutée, car la remise en question est un fondement de la connaissance. Cela correspond exactement à la pluralité des perspectives que nous défendons. L'auto-organisation et l'émergence des connaissance nécessitent un certain anarchisme pour échapper aux préconceptions cadrant par le haut \comment[AB]{a revoir}. En effet, les positions anarchistes ont trouvé un écho très cohérent dans les différents courants de la complexité, de la cybernétique à l'auto-organisation au cours du 20ème siècle~\cite{duda2013cybernetics}. Notre cadre de connaissances développé en~\ref{sec:knowledgeframework} illustre cette émergence de la connaissance. De plus, notre volonté de réflexivité et de donner à notre travail des pistes de lecture diverses au delà de la linéarité (voir \ref{app:reflexivity}), illustre l'application de ces principes. Les recommandations méthodologiques et les positionnements donnés précédemment dans ce chapitre pourraient sonner comme totalitaires s'ils étaient assénés de manière sèche sans contexte, mais ceux-ci sont en fait tout le contraire puisqu'ils découlent d'un dynamique récente de science ouverte qui a bien émergé par le bas, conséquence en partie de l'ouverture et de la pluralité.

% concerning the cs framework : what is deconstructivism ; position our approach in regard ? Feyerabend as a confirmation ? - eventually search for that and develop later.






%%%%%%%%%%%%%%%%%%%%%%
\subsection{From Life to Culture}{De la Vie à la Culture}

Le parallèle entre les systèmes sociaux et les systèmes biologiques est souvent fait, parfois de manière plus qu'imagée comme par exemple pour la théorie du \emph{Scaling} de \noun{West} qui applique des équations de croissance similaires à partir des lois d'échelle, avec des conclusions inverses tout de même concernant la relation entre taille et rythme de vie~\cite{bettencourt2007growth}. Les relations d'échelle ne tiennent plus lorsqu'on essaye de les appliquer à une fourmi seule, et il faut alors l'appliquer à la fourmilière entière qui est alors l'organisme en question. En ajoutant la propriété de cognition, on confirme qu'il s'agit du niveau pertinent, puisque celle-ci possède des propriétés cognitives avancées, comme la résolution de problèmes d'optimisation spatiaux, ou la réponse rapide à une perturbation extérieure. Les organisations sociales humaines, les villes, peuvent-elles être vues comme des organismes ? \noun{Banos} file dans~\cite{banos2013pour} la métaphore de la \emph{fourmilière urbaine} mais rappelle que le parallèle s'arrête assez vite. Nous allons voir cependant dans quelle mesure certains concepts de l'épistémologie de la biologie peuvent être utiles pour comprendre les systèmes sociaux que nous nous proposons d'étudier. Nous nous basons sur la contribution fondamentale de \noun{Monod} dans~\cite{monod1970hasard}, qui tente de développer les principes épistémologiques cruciaux pour l'étude du vivant. Ainsi, les organismes vivants répondent à trois propriétés essentielles qui permettent des les différencier d'autres systèmes : (i) la téléonomie, c'est à dire qu'il s'agit ``d'objets doués d'un projet'', projet qui se reflète dans leur structure et dans celles des artefacts qu'ils produisent\footnote{à ne pas confondre avec la téléologie, propres aux animismes, qui consiste à prêter un projet ou un sens à l'univers} ; (ii) l'importance des processus morphogénétiques dans leur constitution (voir~\ref{sec:interdiscmorphogenesis}) ; (iii) la propriété de reproduction invariante de l'information définissant leur structure. \noun{Monod} esquisse de plus en conclusion des pistes pour une théorie de l'évolution culturelle. La téléonomie est essentielle dans les structures sociales, puisque toute organisation essaye de satisfaire un ensemble d'objectifs, même si en général elle n'y parviendra pas et que ceux-ci co-évolueront avec l'organisation. Un aspect divergent est cette notion de multi-objectif qui est typique des systèmes complexes socio-techniques. Ensuite, nous postulons que la notion de morphogenèse est un outil essentiel pour comprendre ces systèmes, avec une définition très proche de celle utilisée en biologie. Un travail approfondi pour donner cette définition est fait en~\ref{sec:interdiscmorphogenesis}, que nous résumerons en l'existence de processus relativement autonomes guidant la croissance du système et impliquant des relations causales circulaires entre forme et fonction qui témoignent d'une architecture émergente. Pour des systèmes sociaux, isoler le système est plus difficile et la notion de frontière sera moins stricte que pour un système biologique, mais on retrouvera bien ce lien entre forme et fonction, comme par exemple la structure d'une organisation ayant un impact sur ses fonctionnalités. Enfin, la reproduction de l'information est au coeur de l'évolution culturelle, par la transmission de la culture et la \emph{mémétique}, la différence étant que le rapport d'échelle de temps entre la fréquence de transmission et les processus de croisement et de mutation ou d'autres processus non mémétiques de production culturelle est très faible, alors qu'elle est de plusieurs ordres de magnitude en biologie. \cite{2017arXiv170305917G} propose un modèle de réseau auto-catalytique pour la cognition, qui expliquerait l'apparition de l'évolution culturelle par des processus analogues à ceux s'étant produit à l'apparition de la vie, c'est à dire une transition permettant au molécules de s'auto-entretenir et s'auto-reproduire, les représentations mentales faisant office de molécules. Cet exemple montre bien que le parallèle n'est pas toujours absurde. Mais si les processus à l'origine sont analogues, la nature de l'évolution est bien différente par la suite, comme le montre \cite{vanderLeeuw2009}, les critères darwiniens d'évolution n'étant pas suffisant pour expliquer l'évolution de nos sociétés organisées. Il s'agit d'un degré de complexité supérieur\comment[AB]{vraiment ? Eléments factuels à avancer} et le rôle des flux d'information est crucial (voir le rôle de la complexité informationnelle dans la sous-section suivante). Enfin, l'un des points sur lequel il s'agit d'être attentif, est la plus grande difficulté de définir les niveaux d'émergence pour les systèmes sociaux : \cite{roth2009reconstruction} souligne le risque de tomber dans des cul-de-sac ontologiques car les niveaux ont été mal définis. Il soutient qu'il faut d'une manière générale penser au-delà de la seule dichotomie micro-macro qui est utilisée pour caricaturer les notions d'émergence faible, mais que les ontologies doivent souvent être multi-niveaux et impliquant de multiples niveaux intermédiaires.

\comment{\cite{Mesoudi25072017}}




%----------------------------------------------------------------------------------------

\subsection{Nature of Complexity and Knowledge Production}{Nature de la Complexité et Production de Connaissances}

Un aspect de la production de connaissance sur des Systèmes Complexes, auquel nous nous heurtons plusieurs fois ici (voir chapitre~\ref{ch:theory}), et qui semble être récurrent voire inévitable, est un certain niveau de réflexivité (et qui serait inhérent aux systèmes complexes en comparaison aux systèmes simples, comme nous le développerons plus loin). Nous entendons par là à la fois une réflexivité pratique, c'est à dire la nécessité d'élever le niveau d'abstraction, comme le besoin de reconstruire de manière endogène les disciplines dans lesquelles une réflexion cherche à se positionner comme proposé en \ref{sec:quantepistemo}, ou de réfléchir à la nature épistémologique de la modélisation lors de l'élaboration d'un modèle comme en \ref{sec:csframework}, mais également une réflexivité théorique en le sens que les appareils théoriques ou les concepts produits peuvent s'appliquer de manière récursive à eux-mêmes. Cette constatation pratique fait echo à des débats épistémologiques anciens questionnant la possibilité d'une connaissance objective de l'univers qui serait indépendante de notre structure cognitive, ou bien la nécessité d'une ``rationalité évolutive'' impliquant que notre système cognitif, produit de l'évolution, reflète les processus complexes ayant conduit à son émergence, et que toute structure de connaissance sera par conséquent réflexive\footnote{Nous remercions D. Pumain d'avoir pointé cette vue alternative du problème que nous allons développer par la suite}. Nous ne prétendons pas ici apporter une réponse à une question aussi vaste et vague telle quelle, mais proposons un lien potentiel entre cette réflexivité et la nature de la complexité.


\paragraph{Complexity and Complexities}{Complexité et Complexités}

Ce qui est entendu par complexité d'un système mène souvent à des malentendus car celle-ci peut être qualifiée selon différentes dimensions et visions. Nous distinguons d'une part la complexité au sens d'émergence faible et d'autonomie entre les différents niveaux d'un système, et sur laquelle différentes positions peuvent être développées comme dans \cite{deffuant2015visions}. Nous ne rentrerons pas dans une granularité plus fine, la vision de la complexité sociale donnant encore plus de fil à retordre au démon de Laplace, peut être par exemple comprise par une émergence plus forte, la nature des systèmes ne jouant pas de rôle dans notre reflexion.\comment[AB]{je vois l’idée mais c’est loin d’être clair ==> à clarifier} D'autre part, nous distinguons deux autres ``types'' de complexité, la complexité computationnelle et la complexité informationnelle, qui peuvent être vues comme des mesures de complexité, mais qui ne sont pas directement équivalentes à l'émergence, puisqu'il n'existe pas de lien systématique entre les trois. On peut par exemple imaginer utiliser un modèle de simulation, pour lequel les interactions entre agents élémentaires se traduisent par un message codé au niveau supérieur: il est alors possible en exploitant les degré de liberté de minimiser la quantité d'information contenue dans le message. Les différentes langues demandent des efforts cognitifs différents et compressent différemment l'information, ayant différents niveau de complexité mesurables~\cite{febres2013complexity}. De même, des artefacts architecturaux sont le résultat d'un processus d'évolution naturelle puis culturelle et peuvent témoigner plus ou moins de cette trajectoire. D'autres caractérisations conceptuelles ou opérationnelles de la complexité existent, et il est clair que la communauté scientifique n'a pas convergé sur une définition unique~\cite{chu2008criteria}\footnote{Dans une approche en un sens réflexive, \cite{chu2008criteria} propose de continuer d'explorer les différentes approches existantes, comme des proxys de la complexité dans le cas d'un essentialisme, ou comme des notions à part entière. La complexité devrait émerger d'elle même de l'interaction entre ces différentes approches étudiant la complexité (d'où la réflexivité).}. Nous proposons de nous concentrer sur ces trois concepts en particulier, pour lesquels les relations ne sont déjà pas évidentes. En effet, les liens entre ces trois types de complexité ne sont pas systématiques, et dépendent du type de système. Des liens épistémologiques peuvent néanmoins être introduits. Nous traitons ceux entre émergence et les deux autre complexités, étant donné que le lien entre complexité computationnelle et complexité informationnelle est assez bien compris et relève de problématiques de compression de l'information et de traitement du signal, ou encore de cryptographie.

\comment[JR]{Gell-mann : effective complexity : AIC of CAS observing an other CAS : prelude to perspectivism ? check link between information and Kolmogorov complexity - add Kolmogorov complexity in computational complexity.}


\paragraph{Computational Complexity}{Complexité computationnelle et émergence}



Différents indices suggèrent une certaine nécessité de complexité computationnelle pour avoir émergence dans des systèmes complexes, tandis que réciproquement un certain nombre de systèmes complexes adaptatifs sont dotés de capacités de calcul élevées. 

Le ``paradoxe'' du chat de Schrödinger n'en est un que sous une perspective réductionniste, c'est à dire si l'on suppose que la superposition d'états peut se propager à travers les niveaux successifs et qu'il n'y aurait pas émergence, au sens de constitution d'un niveau supérieur autonome.\comment[AB]{trop elliptique là aussi : on ne voit pas clairement le lien avec el gato. Sois plus précis} Cette vision intuitive a récemment été démontrée rigoureusement par \cite{2014arXiv1403.7686B} qui prouve que l'acceptation de $\mathbf{P}\neq\mathbf{NP}$ implique une séparation qualitative entre le niveau quantique microscopique et le niveau d'observation macroscopique. En d'autres termes, la complexité computationnelle est suffisante pour la présence d'émergence. A priori, cette séparation effective des échelles n'implique pas que le niveau inférieur ne joue pas un rôle crucial, puisque \cite{vattay2015quantum} prouve que les propriétés de criticalité quantiques sont typiques des molécules du vivant, sans qu'il n'y ait a priori de spécificité pour la vie dans cette détermination complexe par les échelles inférieures : \cite{2016arXiv161102269V} a introduit une nouvelle approche liant théories quantiques et relativité générale dans laquelle il est montré que la gravité est un phénomène émergent et que la dépendance au chemin dans la déformation de l'espace de base introduit un terme supplémentaire au niveau macroscopique, qui permet d'expliquer les déviations attribuées jusqu'alors à la ``matière noire''.

Dans le sens inverse, le lien entre complexité computationnelle et émergence est mis en valeur par les questions liées à la nature de la computation~\cite{moore2011nature}. Des automates cellulaires, qui sont par ailleurs cruciaux pour la compréhension de divers systèmes complexes, ont été montrés Turing-complets\footnote{un système est Turing-complet s'il est capable de calculer les mêmes fonctions qu'une machine de Turing}\comment[AB]{definition}[a finir] (comme le Jeu de la Vie). Des organismes sans système nerveux central sont capables de résoudre des problèmes décisionnels difficiles~\cite{reid2016decision}. Un algorithme à base de fourmis est montré par~\cite{Pintea2017} comme résolvant un Problème du Voyageur de Commerce Généralisé (GTSP), problème NP-difficile. Ce lien fondamental avait été envisagé par \noun{Turing}, puisqu'au delà de ses contributions fondamentales à l'informatique moderne, il s'était intéressé à la morphogenèse et a tenté de produire des modèles chimiques d'explication de celle-ci~\cite{turing1952chemical} (qui étaient très loin de effectivement l'expliquer - elle n'est toujours pas bien comprise aujourd'hui, voir~\ref{sec:interdiscmorphogenesis} - mais dont les contributions conceptuelles ont été fondamentales, notamment pour la notion de réaction-diffusion).


\comment{\cite{tovsic2017boolean} lower bound for computationnal complexity of simple ABM when adding interactions with the environment.}


\comment{\cite{2017arXiv170404231E} quantum computation reduces drastically memory needed}

\paragraph{Informational Complexity and Emergence}{Complexité informationnelle et émergence}

La complexité informationnelle, ou la quantité d'information contenue dans un système et la manière dont celle-ci est stockée, entretient également des liens fondamentaux avec l'émergence. L'information est équivalente à l'entropie d'un système et donc à son degré d'organisation - c'est ce qui a permis de résoudre le paradoxe apparent du Démon de Maxwell qui serait capable de diminuer l'entropie d'un système isolé et donc contredire la deuxième loi de la thermodynamique : celui-ci utilise en fait l'information sur les positions et vitesses des molécules du système, et son action compense la perte d'entropie par sa captation d'information\footnote{Le démon de Maxwell est plus qu'une construction intellectuelle : \cite{cottet2017observing} implémente un démon expérimentalement au niveau quantique.}. Cette notion d'accroissement local de l'entropie a été étudiée largement par \noun{Chua} sous la forme du \emph{Local Activity Principle}, qui est introduit comme un troisième principe de la thermodynamique, permettant d'expliquer par des arguments mathématiques l'auto-organisation pour une certaine classe de systèmes complexes typiquement impliquant des équations de réaction-diffusion~\cite{mainzer2013local}. La manière dont l'information est stockée et compressée est essentielle pour la vie, puisque l'ADN est bien un système de stockage d'information, dont le rôle à différents niveaux bien loin d'être compris complètement. La complexité culturelle témoigne également d'un stockage de l'information à différents niveaux, par exemple au sein des individus mais aussi des artefacts et des institutions, et des flux d'information relevant nécessairement des deux autres types de complexité. Les flux d'information sont essentiels pour l'auto-organisation dans un système multi-agent. Les comportements collectifs de poissons ou d'oiseau sont des exemples typiques utilisés pour illustrer l'émergence et font partie des cas d'école de systèmes complexes. On commence cependant seulement à comprendre comment ces flux structurent le système, et quels sont les motifs spatiaux de transfert d'information au sein d'un \emph{flock} par exemple : \cite{crosato2017informative} introduit des premiers résultats empiriques avec l'entropie de transfert pour des poissons et pose les bases méthodologiques de ce type d'étude.




\paragraph{Knowledge production}{Production de connaissances}

Nous avons à présent la matière suffisante pour en venir à la réflexivité. Il est possible de positionner la production de connaissances à l'intersection des interactions entre types de complexité développées ci-dessus. Tout d'abord, la connaissance telle que nous l'envisageons ne peut se passer d'une construction collective, et implique donc un encodage et une transmission de l'information : il s'agit à un autre niveau de toutes les problématiques liées à la communication scientifique. La production de connaissances nécessite donc cette première interaction entre complexité computationnelle et complexité informationnelle. Le lien entre complexité informationnelle et émergence est mobilisé si on considère l'établissement de connaissances comme un processus morphogénétique. Il est montré en~\ref{sec:interdiscmorphogenesis} que le lien entre forme et fonction est fondamental en psychologie : nous pouvons l'interpréter comme un lien entre information et sens, puisque la sémantique d'un objet cognitif ne peut se passer d'une fonction. \noun{Hofstader} rappelle dans~\cite{hofstadter1980godel} l'importance des symboles à différents niveaux pour l'émergence d'une pensée, qui consistent à un niveau intermédiaire en des signaux. Enfin, la dernière relation entre complexité computationnelle et émergence est celle qui nous permet d'affirmer qu'on s'intéresse particulièrement à une production de connaissance sur des systèmes complexes, les deux premiers pouvant s'appliquer à tout type de connaissance. Comme ces systèmes sont généralement multi-niveaux, ou présentent au moins un certain niveau de complexité computationnelle, leur connaissance se doit de la capturer, puisque même des modèles \emph{simples} devront capturer leur complexité de manière conceptuelle et impliquer une structure conceptuelle sous-jacente complexe, même si celle-ci n'est pas explicitement explorée.\comment[AB]{ton raisonnement pourrait être clarifié un peu} Ainsi, toute \emph{connaissance du complexe} embrasse non seulement toutes les complexités mais aussi leur relations, dans son contenu et dans sa nature : elle doit nécessairement avoir un certain degré de réflexivité pour alors être cohérente. On peut tenter d'étendre à la réflexivité en tant que réflexion sur le positionnement disciplinaire : suivant \noun{Pumain} dans~\cite{pumain2005cumulativite}, la complexité d'une approche est également liée à la diversité des points de vue nécessaire pour la construire. Pour atteindre ce nouveau type de complexité, qui serait une dimension supplémentaire liée à la connaissance des systèmes complexes, la réflexivité doit être au coeur de la démarche. \cite{read2009innovation} rappelle que l'innovation a été rendue possible quand les sociétés ont été capables de produire et diffuser de l'information sur leur propre structure, c'est à dire quand elles ont pu atteindre un certain niveau de réflexivité. La \emph{connaissance du complexe} serait donc le produit et le support de sa propre évolution grâce à la réflexivité qui a joué un rôle fondamental dans l'évolution du système cognitif : on pourrait ainsi suggérer de rassembler ces considérations, comme proposé par \noun{Pumain}, sous une nouvelle notion épistémologique de \emph{Rationalité Evolutive}. Pour conclure, notons qu'étant donné la loi de la \emph{requisite complexity}, proposée par \cite{gershenson2015requisite} comme extension de la \emph{requisite variety} d'\noun{Ashby}\footnote{L'un des principes cruciaux de la cybernétique, la \emph{requisite variety} postule que pour contrôler un système ayant un certain nombre d'états, le contrôleur doit avoir au moins autant d'états. \noun{Gershenson} propose une extension conceptuelle à la complexité, qui peut être justifiée par exemple par \cite{allen2017multiscale} qui introduit la \emph{requisite variety} multi-échelle, démontrant la compatibilité avec une théorie de la complexité basée sur la théorie de l'information.}, la \emph{connaissance du complexe} devra nécessairement être \emph{connaissance complexe}. Cet autre point de vue renforce la nécessité de la réflexivité, puisque suivant \noun{Morin} (voir par exemple voir le Tome 4 de la méthode centré sur la production de connaissance~\cite{morin1991methode}), la \emph{Connaissance de la Connaissance} est centrale dans l'établissement d'une pensée complexe.


%\comment[AB]{personnellement je ne suis pas très fan de cette idée de « connaissance complexe » ou même de « pensée complexe » (ok je suis un mauvais morinien !). }[(JR) mal formule peu etre, ``connaissance du complexe'' $\rightarrow$ cf dernière phrase : serait equivalent selon le point de vue du controle ; rejoint Morin]

% Remarques : 
% portugali semantic information : chaud à introduire.
% check paper Valentina ``Practical Reflexivity''
% lire Morin sur la pensée complexe





%%%%%%%%%%%%%%%%%%%%%%
\subsection*{Practical implications}{Conséquences pratiques}


Pour conclure cette section épistémologique, nous proposons de synthétiser l'ensemble des idées introduites sous forme de manifestations concrètes en découlant directement, et qui conditionneront fortement l'ensemble de la forme et de la sémantique de la connaissance introduite par la suite. Ces directions (que nous n'irons pas jusqu'à nommer principes car seulement à l'état d'ébauche) peuvent être regroupées en trois grandes familles : pratiques de modélisation, pratique de la Science Ouverte, et épistémologie. Sur le plan des pratiques de modélisation, dans chaque section se dégagent différents axes plus ou moins complémentaires :

\begin{itemize}
	\item La modélisation, qui sera dans la majorité des cas équivalente à la simulation, doit être comprise comme un instruments de connaissance indirect sur des processus au sein d'un système complexe ou sur la structure de celui-ci (d'après la sous-section sur ``pourquoi modéliser''), et les modèles devront nécessairement être complexes (d'après la réflexion sur les différents types de complexité) au sens qu'il capturent un phénomène d'émergence faible, tout en respectant des exigences de parcimonie.
	\item L'exploration des modèles est partie intégrante de l'entreprise de modélisation (voir reproductibilité), et le calcul intensif est un élément clé pour explorer efficacement les modèles de simulation (voir calcul intensif). Les méthodes d'analyse de sensibilité doivent être questionnées et étendues si besoin (comme l'illustre l'exemple de la sensibilité à l'espace).
	\item Comme suggéré par le positionnement perspectiviste, le couplage de modèles devra jouer un rôle crucial dans la capture de la complexité.
\end{itemize}

Pour la Science Ouverte, on peut extraire les points suivants :

\begin{itemize}
	\item La nécessité de l'ensemble des démarches liées à la Science Ouverte pour parvenir à la construction de modèles toujours plus complexes, vers la co-construction de modèles par différentes disciplines.
	\item Dans ce cadre, l'ouverture complète du code source, ainsi que sa lisibilité sont cruciaux. L'explicitation complète du modèle dans le compte-rendu scientifique, ainsi qu'une documentation du code auto-suffisante, sont deux aspects de celle-ci.
	\item La question des données ouvertes n'est pas négociable dans ce cadre. La quasi-totalité de nos traitements est basée sur des données initialement ouverte, et lorsque ce n'est pas le cas nous travaillons à un niveau agrégé auquel on peut fournir les données. Les jeux de données construits sont ouverts.
	\item Concernant les méthodes d'exploration interactive, qui sont un pendant de l'ouverture de la Science, nous en développons un certain nombre, mais restons limités par rapport au pré-requis idéal qui devrait rendre celles-ci totalement compatibles avec une démarche reproductible.
\end{itemize}

Enfin, sur le point épistémologique, on peut également tirer des implications ``pratiques'' qui seront bien évidemment plus implicites dans notre démarche, mais pas moins structurantes :

\begin{itemize}
	\item Notre inspiration sera essentiellement interdisciplinaire et cherchera à croiser les différents points de vue.
	\item Les différents domaines de connaissance (notion que nous préciserons en~\ref{sec:knowledgeframework}, mais qu'on peut comprendre pour l'instant au sens des domaines théorique, empirique et de la modélisation introduits par~\cite{livet2010}) sont indissociables pour toute démarche de production scientifique, et nous les mobiliserons de manière fortement dépendante.
	\item Notre démarche devra comprendre un certain niveau de réflexivité.
	\item La construction d'une connaissance complexe (\cite{morin1991methode}) est ni inductive ni déductive, mais constructive dans l'idée d'une morphogenèse de la connaissance : il peut par exemple être délicat d'identifier clairement des ``verrous scientifiques'' précis puisque cette métaphore suppose qu'il faut débloquer un problème déjà construit, et de même de faire rentrer notions, concepts, objet ou modèles dans des cadres analytiques stricts, en les catégorisant selon une classification fixe, alors que l'enjeu est de comprendre si la construction des catégories est pertinente. Le faire a posteriori relève d'une négation de la circularité et de la récursivité de la production de connaissance. L'élaboration de modes de compte-rendus rendant compte du caractère diachronique et des propriétés évolutives de celle-ci est un problème ouvert.
\end{itemize}






\stars












