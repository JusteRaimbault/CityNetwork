


%----------------------------------------------------------------------------------------


\newpage



\section{Epistemological Positioning}{Positionnement Epistémologique}\label{sec:epistemology}

%----------------------------------------------------------------------------------------



%%%%%%%%%%%%%%%%%%%%%%
\subsection{Cognitive Approach and Perspectivism}{Approche cognitive et Perspectivisme}


\comment[JR]{Pour une science anarchiste (Feyerabend) ; compatibilité avec le Perspectivisme de Giere et pourquoi celui-ci est particulièrement adapté aux paradigmes de la complexité ; multiplicité des lectures de la thèse (voir annexe réflexivité, au delà d'une lecture linéaire)}




%%%%%%%%%%%%%%%%%%%%%%
\subsection{From Life to Culture}{De la Vie à la Culture}
% De Monod à ?
% Le Démon de ? (ref Démon de Laplace dans tous ses états)



\comment[JR]{compatibilité avec Monod sur la majorité des points ; divergences propres aux sciences sociales par rapport à la bio ? - notamment sur la morphogenèse. on en prend une définition ``unifiée'' qui convient bien à nos problématiques.}







%----------------------------------------------------------------------------------------

\subsection{Nature of Complexity and Knowledge Production}{Nature de la Complexité et Production de Connaissances}


% Contenu de la présentation au colloque Geodivercity

\comment[JR]{deuxième niveau de complexité lié à degré de reflexivité de la théorie ? mais qu'est ce alors reflexif. comparer fourmillière à société humaine ? Knowledge of the complex at the intersection, donc nécessairement reflexif ? douteux, à creuser.}


\comment[JR]{check paper Valentina ``Practical Reflexivity''}

Un aspect de la production de connaissance sur des Systèmes Complexes, auquel nous nous heurtons plusieurs fois ici (voir chapitre~\ref{ch:theory}), et qui semble être récurrent voire inévitable, est une certaine réflexivité. Nous entendons par là à la fois une réflexivité pratique, c'est à dire la nécessité d'élever le niveau d'abstraction, comme le besoin de reconstruire de manière endogène les disciplines dans lesquelles une réflexion cherche à se positionner comme proposé en \ref{sec:quantepistemo}, ou de réfléchir à la nature épistémologique de la modélisation lors de l'élaboration d'un modèle comme en \ref{sec:csframework}, mais également une réflexivité théorique en le sens que les appareils théoriques ou les concepts produits peuvent s'appliquer de manière récursive à eux-mêmes. Cette constatation pratique fait echo à des débats épistémologiques anciens questionnant la possibilité d'une connaissance objective de l'univers qui serait indépendante de notre structure cognitive, ou bien la nécessité d'une ``rationalité évolutive'' impliquant que notre système cognitif, produit de l'évolution, reflète les processus complexes ayant conduit à son émergence, et que toute structure de connaissance sera par conséquent réflexive\footnote{Nous remercions D. Pumain d'avoir pointé cette vue alternative du problème que nous allons développer par la suite}. Nous ne prétendons pas ici apporter une réponse à une question aussi vaste et vague telle quelle, mais proposons un lien potentiel entre cette réflexivité et la nature de la complexité.


\paragraph{Complexity and Complexities}{Complexité et Complexités}

Ce qui est entendu par complexité d'un système mène souvent à des malentendus car celle-ci peut être qualifiée selon différentes dimensions et visions. Nous distinguons d'une part la complexité au sens d'émergence faible et d'autonomie entre les différents niveaux d'un système, et sur laquelle différentes positions peuvent être développées comme dans \cite{deffuant2015visions}. Nous ne rentrerons pas dans une granularité plus fine, la vision de la complexité sociale donnant encore plus de fil à retordre au démon de Laplace, peut être par exemple comprise par une émergence plus forte, la nature des systèmes ne jouant pas de rôle dans notre reflexion. D'autre part, nous distinguons deux autres ``types'' de complexité, la complexité computationnelle et la complexité informationnelle, qui peuvent être vues comme des mesures de complexité, mais qui ne sont pas directement équivalentes à l'émergence, puisqu'il n'existe pas de lien systématique entre les trois. On peut par exemple imaginer utiliser un modèle de simulation, pour lequel les interactions entre agents élémentaires se traduisent par un message codé au niveau supérieur: il est alors possible en exploitant les degré de liberté de minimiser la quantité d'information contenue dans le message (ce qui serait en pratique inutile car il y a des moyens plus simples de simuler un bruit blanc). Les différentes langues demandent des efforts cognitifs différents et compressent différemment l'information. % TODO cit. lang. compression
De même, des artefacts architecturaux sont le résultat d'un processus d'évolution naturelle puis culturelle et peuvent témoigner plus ou moins de cette trajectoire. Ainsi, les liens entre ces trois types de complexité ne sont pas systématiques, et dépendent du type de système. Des liens épistémologiques peuvent néanmoins être introduits.


\paragraph{Computational Complexity}{Complexité computationnelle et émergence}

Le ``paradoxe'' du chat de Schrödinger n'en est un que si l'on prend une vision réductionniste, c'est à dire si l'on suppose que la superposition d'états peut se propager à travers les niveaux successifs et qu'il n'y aurait pas émergence, c'est à dire constitution d'un niveau supérieur autonome. Cette vision intuitive a récemment été démontrée rigoureusement par \cite{2014arXiv1403.7686B} qui prouve que l'acceptation de $\mathbf{P}\neq\mathbf{NP}$ implique une séparation qualitative entre le niveau quantique microscopique et le niveau d'observation macroscopique. En d'autres termes, la complexité computationnelle est suffisante pour avoir émergence.


\cite{2016arXiv161102269V}

\cite{vattay2015quantum}


\paragraph{Computational and Informational Complexity}{Complexité computationnelle et informationnelle}

\comment[JR]{how is compressed (and transmitted ?) information : crucial for life - DNA is info storage ; cultural complexity is a higher level of info storage}

Information is entropy - third law of thermodynamic - démon de Maxwell : énergie conservée car info sur le système.


Hofstader sur meaning - lien information et sens ?


Système territoriaux intelligents ? slime mould compute..


\paragraph{Information Complexity and Emergence}{Complexité informationelle et émergence}

\comment[JR]{Nature of computation : slime mould, automates cellulaires etc.}

\comment[JR]{How do information flows organise the system : cf. Thèse E Crosato ; morphogenesis}




\paragraph{Knowledge production}{Production de connaissances}

\comment[JR]{knowledge prod at the intersection of three interactions : then necessary reflexive ? WHY ? point still to clarify.}















