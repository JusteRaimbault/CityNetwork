


%----------------------------------------------------------------------------------------


\newpage



\section{Epistemological Positioning}{Positionnement Epistémologique}\label{sec:epistemology}

%----------------------------------------------------------------------------------------



%%%%%%%%%%%%%%%%%%%%%%
\subsection{Cognitive Approach and Perspectivism}{Approche cognitive et Perspectivisme}



\bpar{
Our epistemological positioning relies on a cognitive approach to science, given by Giere in~\cite{giere2010explaining}. The approach focuses on the role of cognitive agents as carriers and producers of knowledge. It has been shown to be operational by \cite{giere2010agent} that studies an agent-based model of science. These ideas converge with Chavalarias' Nobel Game~\cite{chavalarias2016s} that tests through a stylized model the balance between exploration and falsification in the collective scientific enterprise. This epistemological positioning has been presented by Giere as \emph{scientific perspectivism}~\cite{giere2010scientific}, which main feature is to consider any scientific entreprise as a \emph{perspective} in which \emph{agents} use \emph{media} (models) to represent something with a certain purpose. To make it more concrete, we can position it within Hacking's ``check-list'' of constructivism~\cite{hacking1999social}, a practical tool to position an epistemological position within a simplified three dimensional space which dimensions are different aspects on which realist approaches and constructivist approach generally diverge: first the contingency (path-dependency of the knowledge construction process) is necessary in the pluralist perspectivist approach, secondly the ``degree of constructivism'' is quite high because agents produce knowledge, and finally the stability of theories depends on the complex interaction between the agents and their perspectives. It was presented for these reasons as an intermediate and alternative way between absolute realism and skeptical constructivism~\cite{brown2009models}. The \emph{perspective} plays a central role in the framework.
}{
Notre positionnement épistémologique se fonde sur une approche cognitive de la science, donnée par \noun{Giere} dans~\cite{giere2010explaining}. L'approche se concentre sur le rôle des agents cognitifs comme porteurs et producteurs de la connaissance. Elle a été montrée opérationnelle par \cite{giere2010agent} qui étudie un modèle basé-agent de la science. Ces idées convergent avec le jeu Nobel de \noun{Chavalarias}~\cite{chavalarias2016s} qui teste de manière stylisée l'équilibre entre exploration et falsification dans l'entreprise scientifique collective. Ce positionnement épistémologique a été présenté par \noun{Giere} comme \emph{perspectivisme scientifique}~\cite{giere2010scientific}, dont la caractéristique principale est de considérer toute entreprise scientifique comme une \emph{perspective} dans laquelle des \emph{agents} utilisent des \emph{media} (modèles) pour représenter quelque chose dans un certain but. Pour concrétiser, nous pouvons le positionner sur la ``check-list'' du constructivisme de \noun{Hacking}~\cite{hacking1999social}, un outil pratique pour positionner une position épistémologique dans un espace simplifié à trois dimensions dans lequel les dimensions sont différents aspects sur lesquels les approches réalistes et constructivistes généralement divergent : d'abord la contingence (dépendance au chemin du processus de construction de connaissances) est nécessaire l'approche perspectiviste qui est pluraliste, deuxièmement le ``degré de constructivisme'' est assez haut car les agents produisent la connaissance, et enfin la stabilité des théories dépend des interactions complexes entre les agents et leur perspectives. Cela a pour ces raisons été présenté comme un chemin intermédiaire et alternatif entre le réalisme absolu et le constructivisme sceptique~\cite{brown2009models}. la notion de \emph{perspective} jouera un rôle fondamental dans le cadre développé en~\ref{sec:knowledgeframework}.
}


\comment[JR]{Pour une science anarchiste (Feyerabend) ; compatibilité avec le Perspectivisme de Giere et pourquoi celui-ci est particulièrement adapté aux paradigmes de la complexité ; multiplicité des lectures de la thèse (voir annexe réflexivité, au delà d'une lecture linéaire)}

% Compatible avec une \textit{science anarchiste} à la Feyerabend 
% \cite{feyerabend1993against} : auto-organisation et émergence des connaissances

% concerning the cs framework : what is deconstructivism ; position our approach in regard ? Feyerabend as a confirmation ?

\cite{duda2013cybernetics}

% se positioner dans le réalisme/construct. : fait avec Hacking




%%%%%%%%%%%%%%%%%%%%%%
\subsection{From Life to Culture}{De la Vie à la Culture}
% De Monod à ?
% Le Démon de ? (ref Démon de Laplace dans tous ses états)

Le parallèle entre les systèmes sociaux et les systèmes biologiques est souvent fait, parfois de manière plus qu'imagée comme par exemple pour la théorie du \emph{Scaling} de \noun{West} qui applique des équations de croissance similaires à partir des lois d'échelle, avec des conclusions inverses tout de même concernant la relation entre taille et rythme de vie~\cite{bettencourt2007growth}. Les relations d'échelle ne tiennent plus lorsqu'on essaye de les appliquer à une fourmi seule, et il faut alors l'appliquer à la fourmilière entière qui est alors l'organisme en question. En ajoutant la propriété de cognition, on confirme qu'il s'agit du niveau pertinent, puisque celle-ci possède des propriétés cognitives avancées, comme la résolution de problèmes d'optimisation spatiaux, ou la réponse rapide à une perturbation extérieure. Les organisations sociales humaines, les villes, peuvent-elles être vues comme des organismes ? \noun{Banos} file la métaphore de la \emph{fourmilière urbaine} mais rappelle que le parallèle s'arrête assez vite.

\comment[JR]{compatibilité avec Monod sur la majorité des points ; divergences propres aux sciences sociales par rapport à la bio ? - notamment sur la morphogenèse. on en prend une définition ``unifiée'' qui convient bien à nos problématiques.}


\cite{2017arXiv170305917G} propose un modèle de réseau auto-catalytique pour la cognition, qui expliquerait l'apparition de l'évolution culturelle par des processus analogues à ceux s'étant produit à l'apparition de la vie, c'est à dire une transition permettant au molécules de s'auto-entretenir et s'auto-reproduire, les représentations mentales faisant office de molécules. Cet exemple montre bien que le parallèle n'est pas toujours absurde.



\paragraph{Emergence}{Emergence}

\cite{roth2009reconstruction} : rethink levels ; weak emergence not as simple as micro-macro - dead-end ontologies





%----------------------------------------------------------------------------------------

\subsection{Nature of Complexity and Knowledge Production}{Nature de la Complexité et Production de Connaissances}


% Contenu de la présentation au colloque Geodivercity

\comment[JR]{deuxième niveau de complexité lié à degré de reflexivité de la théorie ? mais qu'est ce alors reflexif. comparer fourmillière à société humaine ? Knowledge of the complex at the intersection, donc nécessairement reflexif ? douteux, à creuser.}


\comment[JR]{check paper Valentina ``Practical Reflexivity''}

Un aspect de la production de connaissance sur des Systèmes Complexes, auquel nous nous heurtons plusieurs fois ici (voir chapitre~\ref{ch:theory}), et qui semble être récurrent voire inévitable, est une certaine réflexivité. Nous entendons par là à la fois une réflexivité pratique, c'est à dire la nécessité d'élever le niveau d'abstraction, comme le besoin de reconstruire de manière endogène les disciplines dans lesquelles une réflexion cherche à se positionner comme proposé en \ref{sec:quantepistemo}, ou de réfléchir à la nature épistémologique de la modélisation lors de l'élaboration d'un modèle comme en \ref{sec:csframework}, mais également une réflexivité théorique en le sens que les appareils théoriques ou les concepts produits peuvent s'appliquer de manière récursive à eux-mêmes. Cette constatation pratique fait echo à des débats épistémologiques anciens questionnant la possibilité d'une connaissance objective de l'univers qui serait indépendante de notre structure cognitive, ou bien la nécessité d'une ``rationalité évolutive'' impliquant que notre système cognitif, produit de l'évolution, reflète les processus complexes ayant conduit à son émergence, et que toute structure de connaissance sera par conséquent réflexive\footnote{Nous remercions D. Pumain d'avoir pointé cette vue alternative du problème que nous allons développer par la suite}. Nous ne prétendons pas ici apporter une réponse à une question aussi vaste et vague telle quelle, mais proposons un lien potentiel entre cette réflexivité et la nature de la complexité.


\paragraph{Complexity and Complexities}{Complexité et Complexités}

Ce qui est entendu par complexité d'un système mène souvent à des malentendus car celle-ci peut être qualifiée selon différentes dimensions et visions. Nous distinguons d'une part la complexité au sens d'émergence faible et d'autonomie entre les différents niveaux d'un système, et sur laquelle différentes positions peuvent être développées comme dans \cite{deffuant2015visions}. Nous ne rentrerons pas dans une granularité plus fine, la vision de la complexité sociale donnant encore plus de fil à retordre au démon de Laplace, peut être par exemple comprise par une émergence plus forte, la nature des systèmes ne jouant pas de rôle dans notre reflexion. D'autre part, nous distinguons deux autres ``types'' de complexité, la complexité computationnelle et la complexité informationnelle, qui peuvent être vues comme des mesures de complexité, mais qui ne sont pas directement équivalentes à l'émergence, puisqu'il n'existe pas de lien systématique entre les trois. On peut par exemple imaginer utiliser un modèle de simulation, pour lequel les interactions entre agents élémentaires se traduisent par un message codé au niveau supérieur: il est alors possible en exploitant les degré de liberté de minimiser la quantité d'information contenue dans le message (ce qui serait en pratique inutile car il y a des moyens plus simples de simuler un bruit blanc). Les différentes langues demandent des efforts cognitifs différents et compressent différemment l'information, ayant différents niveau de complexité mesurables~\cite{febres2013complexity}. De même, des artefacts architecturaux sont le résultat d'un processus d'évolution naturelle puis culturelle et peuvent témoigner plus ou moins de cette trajectoire. Ainsi, les liens entre ces trois types de complexité ne sont pas systématiques, et dépendent du type de système. Des liens épistémologiques peuvent néanmoins être introduits.


\paragraph{Computational Complexity}{Complexité computationnelle et émergence}

Le ``paradoxe'' du chat de Schrödinger n'en est un que si l'on prend une vision réductionniste, c'est à dire si l'on suppose que la superposition d'états peut se propager à travers les niveaux successifs et qu'il n'y aurait pas émergence, c'est à dire constitution d'un niveau supérieur autonome. Cette vision intuitive a récemment été démontrée rigoureusement par \cite{2014arXiv1403.7686B} qui prouve que l'acceptation de $\mathbf{P}\neq\mathbf{NP}$ implique une séparation qualitative entre le niveau quantique microscopique et le niveau d'observation macroscopique. En d'autres termes, la complexité computationnelle est suffisante pour avoir émergence.


\cite{2016arXiv161102269V}

\cite{vattay2015quantum}


\paragraph{Computational and Informational Complexity}{Complexité computationnelle et informationnelle}

\comment[JR]{how is compressed (and transmitted ?) information : crucial for life - DNA is info storage ; cultural complexity is a higher level of info storage}

Information is entropy - third law of thermodynamic - démon de Maxwell : énergie conservée car info sur le système.


Hofstader sur meaning - lien information et sens ?


Système territoriaux intelligents ? slime mould compute..


\paragraph{Information Complexity and Emergence}{Complexité informationelle et émergence}

\comment[JR]{Nature of computation : slime mould, automates cellulaires etc.}

\comment[JR]{How do information flows organise the system : cf. Thèse E Crosato ; morphogenesis}




\paragraph{Knowledge production}{Production de connaissances}

\comment[JR]{knowledge prod at the intersection of three interactions : then necessary reflexive ? WHY ? point still to clarify.}


% from csframework : \noun{Pumain} suggère en effet dans~\cite{pumain2005cumulativite} une nouvelle approche de la complexité qui serait profondément ancrée dans les sciences sociales et qui serait ``mesurée par la diversité des disciplines nécessaires pour élaborer une notion''
%  -> introducing new dimensions / measures of complexity ?












