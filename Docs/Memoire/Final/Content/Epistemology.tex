


%----------------------------------------------------------------------------------------


\newpage



\section{Epistemological Positioning}{Positionnement Epistémologique}\label{sec:epistemo-position}

%----------------------------------------------------------------------------------------



%%%%%%%%%%%%%%%%%%%%%%
\subsection{Cognitive Approach and Perspectivism}{Approche cognitive et Perspectivisme}


\comment[JR]{Pour une science anarchiste (Feyerabend) ; compatibilité avec le Perspectivisme de Giere et pourquoi celui-ci est particulièrement adapté aux paradigmes de la complexité ; multiplicité des lectures de la thèse (voir annexe réflexivité, au delà d'une lecture linéaire)}




%%%%%%%%%%%%%%%%%%%%%%
%\subsection{}{}
% De Monod à ?
% Le Démon de ? (ref Démon de Laplace dans tous ses états)



\comment[JR]{compatibilité avec Monod sur la majorité des points ; divergences propres aux sciences sociales par rapport à la bio ? - notamment sur la morphogenèse. on en prend une définition ``unifiée'' qui convient bien à nos problématiques.}







%----------------------------------------------------------------------------------------

\subsection{Nature of Complexity and Knowledge Production}{Nature de la Complexité et Production de Connaissances}


% Contenu de la présentation au colloque Geodivercity

\comment[JR]{deuxième niveau de complexité lié à degré de reflexivité de la théorie ? mais qu'est ce alors reflexif. comparer fourmillière à société humaine ? Knowledge of the complex at the intersection, donc nécessairement reflexif ? douteux, à creuser.}


\comment[JR]{check paper Valentina ``Practical Reflexivity''}

Un aspect de la production de connaissance sur des Systèmes Complexes, auquel nous nous heurtons plusieurs fois ici (voir chapitre~\ref{ch:theory}), et qui semble être récurrent voire inévitable, est une certaine réflexivité. Nous entendons par là à la fois une réflexivité pratique, c'est à dire la nécessité d'élever le niveau d'abstraction, comme le besoin de reconstruire de manière endogène les disciplines dans lesquelles une réflexion cherche à se positionner comme proposé en \ref{sec:quantepistemo}, ou de réfléchir à la nature épistémologique de la modélisation lors de l'élaboration d'un modèle comme en \ref{sec:csframework}, mais également une réflexivité théorique en le sens que les appareils théoriques ou les concepts produits peuvent s'appliquer de manière récursive à eux-mêmes. Cette constatation pratique fait echo à des débats épistémologiques anciens questionnant la possibilité d'une connaissance objective de l'univers qui serait indépendante de notre structure cognitive, ou bien la nécessité d'une ``rationalité évolutive'' impliquant que notre système cognitif, produit de l'évolution, reflète les processus complexes ayant conduit à son émergence, et que toute structure de connaissance sera par conséquent réflexive\footnote{Nous remercions D. Pumain d'avoir pointé cette vue alternative du problème que nous allons développer par la suite}. Nous ne prétendons pas ici apporter une réponse à une question aussi vaste et vague telle quelle, mais proposons un lien potentiel entre cette refléxivité et la nature de la complexité.












