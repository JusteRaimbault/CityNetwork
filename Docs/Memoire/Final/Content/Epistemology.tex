


%----------------------------------------------------------------------------------------


\newpage



\section{Epistemological positioning}{Positionnement épistémologique}\label{sec:epistemology}

%----------------------------------------------------------------------------------------


\bpar{
The last section of this chapter aims at clarifying our epistemological positioning, since it has only been sketched at different points previously. Such a positioning is never harmless, since it strongly conditions the approaches, experiments and the interpretation of results: as \cite{morin1980methode} recalls, a positioning that pretends to be objective by rejecting any subjective component is much more biased than a conscious subjective approach.
}{
La dernière section de ce chapitre vise à clarifier notre positionnement épistémologique, celui-ci ayant déjà été ébauché à plusieurs occasions précédemment. Un tel positionnement n'est jamais anodin, puisqu'il conditionne fortement les démarches, les expériences et l'interprétation des résultats : comme le souligne~\cite{morin1980methode}, un positionnement qui se dit objectif en rejetant toute subjectivité est bien plus biaisé qu'une approche subjective consciente.
}


\bpar{
The points we wish to develop can be put into both a vertical perspective in terms of levels of abstraction and in a perspective of scientific domains: linearly, we first give the general epistemological context (typical to history of science, at a medium abstraction level), then switch at a less generic level to conceptually precise our particular objects (epistemology of the living and of the social), and finally take a broader perspective at the level of knowledge production itself (epistemology of complexity).
}{
Les points que nous souhaitons développer se placent dans une logique à la fois verticale de niveau d'abstraction et dans une logique de domaines scientifiques : dans l'ordre, nous posons d'abord le contexte épistémologique général (propre à l'histoire des sciences, à un niveau d'abstraction moyen), pour descendre en généralité et préciser conceptuellement nos objets particuliers (épistémologie du vivant et du social), pour finalement tout remettre en perspective au niveau de la production de connaissance elle-même (épistémologie de la complexité).
}



%%%%%%%%%%%%%%%%%%%%%%
\subsection{Cognitive approach and perspectivism}{Approche cognitive et perspectivisme}



\bpar{
Our epistemological positioning relies on a cognitive approach to science, given by Giere in~\cite{giere2010explaining}. The approach focuses on the role of cognitive agents as carriers and producers of knowledge. It has been shown to be operational by \cite{giere2010agent} that studies an agent-based model of science. These ideas converge with \noun{Chavalarias}' Nobel Game~\cite{chavalarias2016s} which tests through a stylized model the balance between exploration and falsification in the collective scientific enterprise.
}{
Notre positionnement épistémologique se fonde sur une approche cognitive de la science, introduite par \noun{Giere} dans~\cite{giere2010explaining}. L'approche se concentre sur le rôle des agents cognitifs comme porteurs et producteurs de la connaissance. Son caractère opérationnel a été montré par \cite{giere2010agent} qui étudie un modèle multi-agents de la science. Ces idées convergent avec le jeu Nobel de \noun{Chavalarias}~\cite{chavalarias2016s} qui teste de manière stylisée l'équilibre entre production de nouvelles théories et tentative de falsification de théories existantes dans l'entreprise scientifique collective.
}




\bpar{
This epistemological positioning has been presented by \noun{Giere} as \emph{scientific perspectivism}~\cite{giere2010scientific}, which main feature is to consider any scientific entreprise as a \emph{perspective} in which \emph{agents} use \emph{media} (models) to represent something with a certain purpose. To make it more concrete, we can position it within Hacking's ``check-list'' of constructivism~\cite{hacking1999social}, a practical tool to position an epistemological position within a simplified three dimensional space which dimensions are different aspects on which realist approaches and constructivist approach generally diverge: first the contingency (path-dependency of the knowledge construction process) is necessary in the pluralist perspectivist approach which assumes parallel paths of knowledge construction. Secondly the ``degree of constructivism'' is quite high because agents produce knowledge. Finally, concerning the endogenous or exogenous explanation of the stability of theories, this stability depends on the complex interaction between the agents and their perspectives, and is thus strongly endogenous, close to the positioning of constructivism. It was presented for these reasons as an intermediate and alternative way between absolute realism and skeptical constructivism~\cite{brown2009models}. The concept of \emph{perspective} will play thus a central role in the framework developed in~\ref{sec:knowledgeframework}.

}{
Ce positionnement épistémologique a été présenté par \noun{Giere} comme \emph{perspectivisme scientifique}~\cite{giere2010scientific}, dont la caractéristique principale est de considérer toute entreprise scientifique comme une \emph{perspective} dans laquelle des \emph{agents} utilisent des \emph{media} (modèles) pour représenter quelque chose dans un certain but. Pour comprendre ses principes de manière plus concrète, nous pouvons le positionner sur la \emph{check-list} du constructivisme de \noun{Hacking}~\cite{hacking1999social}, un outil pratique pour situer une position épistémologique. Celle-ci suppose un espace simplifié tri-dimensionnel dans lequel les dimensions sont différents aspects sur lesquels les approches réalistes et constructivistes généralement divergent. Le premier point est le niveau de contingence (dépendance au chemin du processus de construction de connaissances) : celle-ci est nécessaire dans l'approche perspectiviste qui est pluraliste et suppose des chemins parallèles de construction de connaissance. Le deuxième point mesure un ``degré de constructivisme'', qui est assez haut en perspectivisme car les agents produisent la connaissance. Enfin, le dernier point qui concerne l'explication endogène ou exogène de la stabilité des théories, est fortement du côté du constructivisme, puisque cette stabilité dépend des interactions complexes entre les agents et leur perspectives et donc totalement endogène. Le perspectivisme a pour ces raisons été présenté comme un chemin intermédiaire et alternatif entre le réalisme absolu et le constructivisme sceptique~\cite{brown2009models}. Le concept de \emph{perspective} jouera pour nous un rôle fondamental dans le cadre développé en~\ref{sec:knowledgeframework}.
}


\bpar{
Since this approach puts the emphasis on auto-organization, we consider it to be fully compatible with an anarchist view of science as advocated by~\cite{feyerabend1993against}. He formulates doubts on the relevance of political anarchism but introduces \emph{scientific anarchism}, which must not be understood as a full refusal of any ``objective'' method, but of an artificial authority and legitimacy that some scientific methods or currents would like to impose. He demonstrates through a precise analysis of Galileo's work that most of his results were based on beliefs and that most were not accessible with the current tools and methods at that time, and postulates that a similar logic should apply to contemporary works. There is thus no \emph{perspective} that is objectively more legitimate than others as soon as they are evidence-based and peer review validated - and even in this case legitimacy should be questionable, since questioning is one foundation of knowledge. It corresponds exactly to the plurality of perspectives we defend.
}{
Cette approche mettant l'emphase sur l'auto-organisation, nous la voyons totalement compatible avec une vision anarchiste de la science comme défendue par~\cite{feyerabend1993against}. Celui-ci émet des doutes sur l'intérêt de l'anarchisme politique mais introduit l'\emph{anarchisme scientifique}, qu'il ne faut pas comprendre comme un refus total de toute méthode ``objective'', mais d'une autorité et légitimité artificielle que certaines méthodes ou courants scientifiques pourraient vouloir prendre. Il démontre par une analyse précise des travaux de Galilée que la plupart de ses résultats étaient basés sur des croyances et que la plupart n'étaient pas accessibles avec les outils et méthodes de l'époque, et postule qu'il devrait en être de même pour certains travaux contemporains. Il n'y a donc pas de \emph{perspective} objectivement plus légitime que d'autres dans la mesure de leurs validation par des faits et des pairs - et même dans ces cas la légitimité doit pouvoir être discutée, car la remise en question est un fondement de la connaissance. Cela correspond exactement à la pluralité des perspectives que nous défendons.
}


\bpar{
Assuming an auto-organization and emergence of knowledge can be interpreted as a priority given to the \emph{bottom-up} construction of paradigms, trying to take some distance with preconceptions or dogmas that impose a top-down view. In other words, it is similar to practicing the scientific anarchism proposed by \noun{Feyerabend}. Indeed, anarchist positioning have found a very relevant echo in the different currents of complexity, from cybernetics to self-organization during the 20th century~\cite{duda2013cybernetics}. Our knowledge framework developed in~\ref{sec:knowledgeframework} illustrates this emergence of knowledge. Moreover, our will for reflexivity and to give to this work diverse reading paths beyond linearity (see Appendix~\ref{app:reflexivity}), shows the application of these principles. Methodological recommendations and positioning given previously in this chapter could sound as totalitarian if they were given roughly out of context, but these are indeed exactly the contrary since they sprout from a recent dynamic of open science which is well bottom-up founded, and in part a consequence of opening and plurality.
}{
Supposer auto-organisation et émergence des connaissances peut être interprété comme une priorité donnée à la construction des paradigmes \emph{par le bas} (\emph{bottom-up}), en tentant de se distancer des préconceptions ou dogmes cadrant par le haut. En d'autres termes, il s'agit de pratiquer l'anarchisme scientifique prôné par \noun{Feyerabend}. En effet, les positions anarchistes ont trouvé un écho très cohérent dans les différents courants de la complexité, de la cybernétique à l'auto-organisation au cours du 20ème siècle~\cite{duda2013cybernetics}. Notre cadre de connaissances développé en~\ref{sec:knowledgeframework} illustre cette émergence de la connaissance. De plus, notre volonté de réflexivité et de donner à notre travail des pistes de lecture diverses au delà de la linéarité (voir Annexe~\ref{app:reflexivity}), montre l'application de ces principes. Les recommandations méthodologiques et les positionnements donnés précédemment dans ce chapitre pourraient sonner comme totalitaires s'ils étaient assénés de manière sèche sans contexte, mais ceux-ci sont en fait tout le contraire puisqu'ils découlent d'une dynamique récente de science ouverte qui a bien émergé par le bas, conséquence en partie de l'ouverture et de la pluralité.
}

% Concept of ``Deconstructivism'' ? does not seen to exist.






%%%%%%%%%%%%%%%%%%%%%%
\subsection{From life to culture}{De la Vie à la Culture}

\subsubsection{Biological systems and social systems}{Systèmes biologiques et systèmes sociaux}


\bpar{
The parallel between social and biological systems is not rare, sometimes more from an analogy perspective as for example in \noun{West}'s \emph{Scaling} theory which applies similar growth equations starting from scaling laws, with however inverse conclusions concerning the relation between size and pace of life~\cite{bettencourt2007growth}. Scaling relations do not hold when we try to apply them to a single ant, and they must be applied to the whole ant colony which is then the organism studied. When adding the property of cognition, we confirm that it is the relevant level, since the colony shows advanced cognitive properties, such as the resolution of spatial optimization problems, or the quick answer to an external perturbation. Human social organizations, cities, could be seen as organisms ? \cite{banos2013pour} extends the metaphor of the \emph{urban anthill} but recalls that the parallel stops quickly. We will however see to what extent some concepts from the epistemology of biology can be useful to understand social systems that we propose to study.
}{
Le parallèle entre les systèmes sociaux et les systèmes biologiques est souvent fait, parfois de manière plus qu'imagée comme par exemple pour la théorie du \emph{Scaling} de \noun{West} qui applique des équations de croissance similaires à partir des lois d'échelle, avec des conclusions inverses tout de même concernant la relation entre taille et rythme de vie~\cite{bettencourt2007growth}. Les relations d'échelle ne tiennent plus lorsqu'on essaye de les appliquer à une fourmi seule, et il faut alors l'appliquer à la fourmilière entière qui est l'organisme en question. En ajoutant la propriété de cognition, on confirme qu'il s'agit du niveau pertinent, puisque celle-ci possède des propriétés cognitives avancées, comme la résolution de problèmes d'optimisation spatiaux, ou la réponse rapide à une perturbation extérieure. Les organisations sociales humaines, les villes, peuvent-elles être vues comme des organismes ? \cite{banos2013pour} file la métaphore de la \emph{fourmilière urbaine} mais rappelle que le parallèle s'arrête assez vite. Nous allons voir cependant dans quelle mesure certains concepts de l'épistémologie de la biologie peuvent être utiles pour comprendre les systèmes sociaux que nous nous proposons d'étudier.
}


\bpar{
We start from the fundamental contribution of \noun{Monod} in~\cite{monod1970hasard}, which aims at developing crucial epistemological principles for the study of life. Thus, living organisms answer to three essential properties that differentiate them from other systems: (i) the teleonomy , i.e. the property that these are ``objects with a project'', project that is reflected in their structure and the structure of artifacts they produce\footnote{That must not be mistaken with teleology, typical of animist thoughts, that consists in giving a project or a meaning to the universe.}; (ii) the importance of morphogenetic processes in their constitution (see~\ref{sec:interdiscmorphogenesis}); (iii) the property of the invariant reproduction of information defining their structure. \noun{Monod} furthermore sketches in conclusion some paths towards a theory of cultural evolution. Teleonomy is crucial in social structures, since any organization aims at satisfying a set of objectives, even if in general it will not succeed and the objectives will co-evolve with the organization. This notion of multi-objective optimization is typical of complex socio-technical systems, and will be more crucial than for biological systems.
}{
Nous nous basons sur la contribution fondamentale de \noun{Monod} dans~\cite{monod1970hasard}, qui tente de développer les principes épistémologiques cruciaux pour l'étude du vivant. Ainsi, les organismes vivants répondent à trois propriétés essentielles qui permettent des les différencier d'autres systèmes : (i) la téléonomie, c'est-à-dire qu'il s'agit ``d'objets doués d'un projet'', projet qui se reflète dans leur structure et dans celles des artefacts qu'ils produisent\footnote{Qu'il ne faut pas confondre avec la téléologie, propres aux animismes, qui consiste à prêter un projet ou un sens à l'univers.} ; (ii) l'importance des processus morphogénétiques dans leur constitution (voir~\ref{sec:interdiscmorphogenesis}) ; (iii) la propriété de reproduction invariante de l'information définissant leur structure. \noun{Monod} esquisse de plus en conclusion des pistes pour une théorie de l'évolution culturelle. La téléonomie est essentielle dans les structures sociales, puisque toute organisation essaye de satisfaire un ensemble d'objectifs, même si en général elle n'y parviendra pas et que ceux-ci co-évolueront avec l'organisation. Cette notion d'optimisation multi-objectif est typique des systèmes complexes socio-techniques, et y sera plus cruciale que pour les systèmes biologiques.
}


\bpar{
Moreover, we postulate that the concept of morphogenesis is an essential tool to understand these systems, with a definition very similar to the one used in biology. A more thorough work to build this definition is done in~\ref{sec:interdiscmorphogenesis}, that we will sum up as the existence of relatively autonomous processes guiding the growth of the system and implying causal circular relations between form and function, that witness an emergent architecture. For social systems, isolating the system is more difficult and the notion of boundary will be less struct than for a biological system, but we will indeed find this link between form and function, such as for example the structure of an organization that impacts its functionalities.
}{
Ensuite, nous postulons que le concept de morphogenèse est un outil essentiel pour comprendre ces systèmes, avec une définition très proche de celle utilisée en biologie. Un travail approfondi pour donner cette définition est fait en~\ref{sec:interdiscmorphogenesis}, que nous résumerons en l'existence de processus relativement autonomes guidant la croissance du système et impliquant des relations causales circulaires entre forme et fonction qui témoignent d'une architecture émergente. Pour des systèmes sociaux, isoler le système est plus difficile et la notion de frontière sera moins stricte que pour un système biologique, mais on retrouvera bien ce lien entre forme et fonction, comme par exemple la structure d'une organisation ayant un impact sur ses fonctionnalités.
}


\bpar{
Finally, the reproduction of information is at the core of cultural evolution, through the transmission of culture and \emph{memetics}, the difference being that the ratio of scales between the frequency of transmission and mutation and cross-over processes or other non-memetic processes of cultural production is relatively low, whereas is many orders of magnitude in biology.
}{
Enfin, la reproduction de l'information est au coeur de l'évolution culturelle, par la transmission de la culture et la \emph{mémétique}, la différence étant que le rapport d'échelles de temps entre la fréquence de transmission et les processus de croisement et de mutation ou d'autres processus non mémétiques de production culturelle est relativement faible, alors qu'elle est de plusieurs ordres de grandeur en biologie.
}


\bpar{
An example shows that the parallel is not always absurd : \cite{2017arXiv170305917G} proposes an auto-catalytic network model for cognition, that would explain the apparition of cultural evolution through processes that are analogous to the ones that occurred at the apparition of life, i.e. a transition allowing the molecules to be self-sustained and to self-reproduce, mental representations being the analogous of molecules.
}{
Un exemple illustre que le parallèle n'est pas toujours absurde : \cite{2017arXiv170305917G} propose un modèle de réseau auto-catalytique pour la cognition, qui expliquerait l'apparition de l'évolution culturelle par des processus analogues à ceux s'étant produit à l'apparition de la vie, c'est-à-dire une transition permettant au molécules de s'auto-entretenir et s'auto-reproduire, les représentations mentales faisant office de molécules.
}


\bpar{
But even if processes are at the origin analogous, the nature of evolution is then quite different, as show \cite{vanderLeeuw2009}, darwinian criteria for evolution being not sufficient to explain the evolution of our organized societies. This is a complexity of a different nature in which the role of information flows is crucial (see the role of informational complexity in the next subsection).
}{
Mais si les processus à l'origine sont analogues, la nature de l'évolution est bien différente par la suite, comme le montrent \cite{vanderLeeuw2009}, les critères darwiniens d'évolution n'étant pas suffisant pour expliquer l'évolution de nos sociétés organisées. Il s'agit d'une complexité de nature différente dans laquelle le rôle des flux d'information est crucial (voir le rôle de la complexité informationnelle dans la sous-section suivante). 
}


\bpar{
One point that also must retain our attention is the greater difficulty to define levels of emergence for social systems: \cite{roth2009reconstruction} underlines the risk to fall into ontological dead-ends if levels were badly defined. He argues that more generally we must go past the single dichotomy micro-macro that is used as a caricature of the concepts of weak emergence, and that ontologies must often be multi-level and imply multiple intermediate levels.
}{
L'un des points sur lequel il s'agit également d'être attentif est la plus grande difficulté de définir les niveaux d'émergence pour les systèmes sociaux : \cite{roth2009reconstruction} souligne le risque de tomber dans des cul-de-sac ontologiques car les niveaux ont été mal définis. Il soutient qu'il faut d'une manière générale penser au-delà de la seule dichotomie micro-macro qui est utilisée pour caricaturer les concepts d'émergence faible, et que les ontologies doivent souvent être multi-niveaux et impliquant de multiples niveaux intermédiaires.
}


\bpar{
This last question must also be put into perspective with the problem of the existence of strong emergence in social structures, that in sociological terms corresponds to the idea of the existence of ``collective beings''~\cite{angeletti2015etres}. \noun{Morin} indeed distinguishes living systems of the second type (multi-cellular) and of the third type (social structures), but precises that the \emph{subjects} of the latest are necessarily unachieved\cite{morin1980methode} (p.~852). Thus, emergences from the biological to the social are analogous by stay fundamentally different.
}{
Cette dernière question est aussi à mettre en perspective avec le problème de l'existence d'émergence forte dans les structures sociales, qui en termes sociologiques correspond à l'idée de l'existence ``d'êtres collectifs''~\cite{angeletti2015etres}. \noun{Morin} distingue d'ailleurs les systèmes vivants du second type (multi-cellulaire) et du troisième type (structures sociales), mais précise que les \emph{sujets} de ces derniers sont nécessairement inachevés~\cite{morin1980methode} (p.~852). Ainsi, les émergences du biologique au social sont analogues mais restent fondamentalement différentes.
}



\subsubsection{Co-evolution}{Co-évolution}



\bpar{
This positioning on biological and social systems finds a direct echo for the concept of co-evolution. It indeed comes from biology, where it was developed following the concept of evolution, to be used more recently in social sciences and humanities. To what extent the concept was transfered ? Is there a parallel similar to the one between biological evolution and cultural evolution ? We propose, in order to answer these questions, to develop a brief multidisciplinary point of view on co-evolution\footnote{The approach here is slightly different from the one lead in~\ref{sec:interdiscmorphogenesis} in the case of morphogenesis, that will be \emph{interdisciplinary} in the sens that it aims at integrating approaches, whereas we stay here in an overview of concepts and thus more in a \emph{multidisciplinary} approach. The concept of \emph{co-evolution} being key for our empirical work in the following, we will therefore give an original characterization to it, and make the choice to not go into an integrative syncretism for this concept, but indeed to approach it from a \emph{geographical point of view}, and even more precisely in the frame of territorial systems. We could postulate a congruence between the empirical and modeling specialization and the one for theory, reading our process of knowledge production in a particular profile of knowledge domains dynamics (see~\ref{sec:knowledgeframework}).}. We will in the following review a broad spectrum of disciplines, starting from biology where the concept originated to progressively come to disciplines closer to territorial sciences.
}{
Ce positionnement sur les systèmes biologiques et sociaux trouve un écho immédiat pour le concept de co-évolution. Il provient en effet de la biologie, où il a été développé à la suite de celui d'évolution, pour être utilisé plus récemment en sciences humaines et sociales. Dans quelle mesure le concept a-t-il été transféré ? Retrouve-t-on un parallèle similaire à celui entre évolution biologique et évolution culturelle ? Nous proposons pour répondre à ces questions d'apporter un bref point de vue multidisciplinaire sur la co-évolution\footnote{La démarche ici est légèrement différente de celle que nous mènerons en~\ref{sec:interdiscmorphogenesis} dans le cas de la morphogenèse, qui sera \emph{interdisciplinaire} au sens où elle cherchera à intégrer les approches, tandis que nous restons ici dans un aperçu des concepts et donc plutôt dans du \emph{multidisciplinaire}. Le concept de \emph{co-évolution} étant clé pour notre travail empirique par la suite, nous en donnerons alors une caractérisation originale et prenons le parti de ne pas tomber dans le syncrétisme intégrateur pour ce concept, mais bien de l'approcher d'un \emph{point de vue géographique}, et même plus précisément dans le cadre des systèmes territoriaux. On pourrait postuler une congruence entre la spécialisation empirique/de modélisation et celle théorique, plaçant notre processus de production de connaissance dans un profil particulier de dynamiques de domaines de connaissance (voir~\ref{sec:knowledgeframework}).}. Nous passons par la suite en revue un large spectre de disciplines, partant de la biologie où le concept a initialement trouvé son origine pour arriver progressivement à des disciplines en relation avec les sciences du territoire.
}



\subsubsection{Biology}{Biologie}

\bpar{
The concept of co-evolution in biology is an extension of the well-known concept of \emph{evolution}, that can be tracked back to \noun{Darwin}. \cite{durham1991coevolution} (p.~22) recalls the components and systemic structures that are necessary to have evolution\footnote{And in that general context, evolution is not restricted to the biology of life and the presence of genes, but also to physical systems verifying these conditions. We will come back to that later.}.
}{
Le concept de co-évolution en biologie est une extension de celui bien connu d'\emph{évolution}, qui remonte à \noun{Darwin}. \cite{durham1991coevolution} (p.~22) rappelle les composantes et structure systémiques nécessaires pour qu'il y ait évolution\footnote{Et dans ce contexte général l'évolution n'est pas réservée à la biologie du vivant et la présence de gènes, mais aussi à des systèmes physiques vérifiant ces conditions. Nous y reviendrons plus loin.}.
}

\bpar{
\begin{enumerate}
\item Process of \emph{transmission}, implying transmission units and transmission mechanisms.
\item Process of \emph{transformation}, that necessitates sources of variation.
\item Isolation of sub-systems such that the effects of previous processes are observable in differentiations.
\end{enumerate}
}{
\begin{enumerate}
	\item Processus de \emph{transmission}, impliquant des unités de transmission et des mécanismes de transmission.
	\item Processus de \emph{transformation}, nécessitant des sources de variation.
	\item Isolation de sous-systèmes pour que les effets des processus précédents soient observables dans des différentiations.
\end{enumerate}
}

\bpar{
This way, a population submitted to constraints (often conceptually synthesized as a \emph{fitness}) that condition the transmission of the genetic heritage of individuals (transmission), and to random genetic mutations (transformation), will indeed be in evolution in the spatial territories it populates (isolation), and by extension the species to which it can be associated. 
}{
Ainsi, une population soumise à des contraintes (souvent synthétisées conceptuellement comme une \emph{fitness}) qui conditionnent la transmission du patrimoine génétique des individus (transmission), et à des mutation génétiques aléatoires (transformation), sera bien en évolution dans les territoires spatiaux qu'elle occupe (isolation), et par extension l'espèce à laquelle on peut l'associer.
}


\bpar{
Co-evolution is then defined as an evolutionary change in a characteristic of individuals of a population, in response to a change in a second population, which in turn responds by evolution to the change in the first, as synthesized by~\cite{janzen1980coevolution}. This author furthermore highlights the subtlety of the concept and warns against its unjustified uses: the presence of a congruence between two characteristics that seem adapted one to the other does not necessarily imply a co-evolution, since one species could have adapted alone to one characteristic already present in the other.
}{
La co-évolution est alors définie comme un changement évolutionnaire dans une caractéristique des individus d'une population, en réponse à un changement dans une deuxième population qui à son tour répond évolutionnairement au changement de la première, comme synthétisé par~\cite{janzen1980coevolution}. Cet auteur appuie par ailleurs la subtilité du concept et alerte contre ses utilisations injustifiées : la présence d'une congruence de deux caractéristiques qui semblent adaptées l'une à l'autre n'implique pas l'existence d'une co-évolution, l'une des deux espèces ayant pu s'adapter seule à une caractéristique déjà présente de l'autre.
}

\bpar{
This rough presentation partly hides the real complexity of ecosystems: populations are embedded in trophic networks and environments, and co-evolutionary interactions would imply communities of populations from diverse species, as presented by \cite{strauss2005toward} under the appellation of diffuse co-evolution. Similarly, spatio-temporal dynamics are crucial in the realization of these processes: \cite{dybdahl1996geography} study for example the influence of the spatial distribution on patterns of co-evolution for a snail and its parasite, and show that a higher speed of genetic diffusion in space for the parasite drive the co-evolutionary dynamics.
}{
Cette présentation brute de décoffrage mutile dans une certaine mesure la complexité réelle des écosystèmes : les populations s'insèrent dans des réseaux trophiques et des environnements, et les interactions co-évolutionnaires impliqueraient des communautés de populations d'espèces diverses, comme présenté par \cite{strauss2005toward} sous l'appellation de co-évolution diffuse. De même, les dynamiques spatio-temporelles sont cruciales dans la réalisation de ces processus : \cite{dybdahl1996geography} étudient par exemple l'influence de la distribution spatiale sur les motifs de co-évolution pour un escargot et son parasite, et montrent qu'une vitesse de diffusion génétique dans l'espace plus grande pour le parasite conduit les dynamiques de co-évolution.
}

\bpar{
The essential concepts to retain from the biological point of view are thus: (i) existence of evolution processes, in particular transmission and transformation; (ii) in circular schemas between populations in the case of co-evolution; and (iii) in a complex territorial frame (spatio-temporal and environmental in the sense of the rest of the ecosystem).
}{
Les concepts essentiels à retenir du point de vue biologique sont ainsi : (i) existence de processus d'évolution, en particulier transmission et transformation ; (ii) dans des schémas circulaires entre populations dans le cas de la co-évolution ; et (iii) dans un cadre territorial (spatio-temporel et environnemental au sens du reste de l'éco-système) complexe.
}


\subsubsection{Cultural evolution}{Evolution culturelle}


\bpar{
This development on co-evolution was brought by the parallel between biological and social systems. The evolution of culture is theorized within a proper field, and witnesses many co-evolutive dynamics. \cite{Mesoudi25072017} recalls the state of knowledge on the subject and future issues, such as the relation with the cumulative nature of culture, the influence of demography in evolution processes, or the construction of phylogenetic methods allowing to reconstruct branches of past evolutionary trees.
}{
Ce développement sur la co-évolution nous a été amené par le parallèle entre systèmes biologiques et systèmes sociaux. L'évolution de la culture est théorisée est explorée par un champ propre, et n'est pas en reste de dynamiques co-évolutives. \cite{Mesoudi25072017} rappelle l'état des connaissances sur le sujet et les défis à venir, comme la relation avec la nature cumulative de la culture, l'influence de la démographie dans les processus d'évolution, ou la construction de méthodes phylogénétiques permettant de reconstruire des arbres des branchements passés.
}


\bpar{
To give an example, \cite{carrignon2015modelling} introduces a conceptual frame for the co-evolution of culture and commerce in the case of ancient societies for which there are archeological data, and proposes its implementation with a multi-agent model which dynamics are partly validated by the study of stylized facts produced by the model. The co-evolution is here indeed taken in the sense of a mutual adaptation of socio-spatial structures, at comparable time scales, in this more general frame of cultural evolution.
}{
Pour donner un exemple, \cite{carrignon2015modelling} introduit un cadre conceptuel pour la co-évolution de la culture et du commerce dans le cas de sociétés anciennes sur lesquelles on dispose de données archéologiques, et propose son implémentation par un modèle multi-agents dont les dynamiques sont partiellement validées par l'étude des faits stylisés produits par le modèle. La co-évolution est bien prise ici au sens d'adaptation mutuelle de structures socio-spatiales, à des échelles de temps comparables, dans ce cadre plus général d'évolution culturelle.
}


\bpar{
Cultural evolution would even be indissociable from genetic evolution, since \cite{durham1991coevolution} postulates and illustrates a strong link between the two, that would themselves be in co-evolution. \cite{bull2000meme} explores a stylized model including two types of replicant populations (genes and memes) and shows the existence of phase transitions for the results of the genetic evolution process when the interaction with the cultural replicant is strong.
}{
L'évolution culturelle serait même indissociable de l'évolution génétique, puisque \cite{durham1991coevolution} postule et illustre un lien fort entre les deux, qui seraient eux-mêmes en co-évolution. \cite{bull2000meme} explore un modèle stylisé impliquant deux populations de répliquants (les gènes et les memes) et montre l'existence de transitions de phase pour les résultats du processus d'évolution génétique lorsque l'interaction avec le répliquant culturel est forte.
}


%\subsubsection{Artificial Life}{Artificial Life}
% -> in cultural evolution

\subsubsection{Sociology}{Sociologie}


\bpar{
The concept was used in sociology and related disciplines such as organisation studies, following the parallel done before the same way as cultural evolution. In the field of the study of organisations, \cite{volberda2003co} develop a conceptual frame of inter-organisational co-evolution in relation with internal management processes, but deplore the absence of empirical studies aiming at quantifying this co-evolution. In the context of production systems management, \cite{tolio2010species} conceptualize an intelligent production chain where product, process and the production system must be in co-evolution.
}{
Le concept a été utilisé en sociologie et disciplines apparentées comme les études de l'organisation, suivant le parallèle effectué ci-dessus de la même manière que pour l'évolution culturelle. Dans le domaine de l'étude des organisations, \cite{volberda2003co} développent un cadre conceptuel de la co-évolution inter-organisationnelle en relations avec les processus de management internes, mais déplore l'absence d'études empiriques cherchant à quantifier cette co-évolution. Dans le cadre de la gestion des systèmes de production, \cite{tolio2010species} conceptualisent un chaine de production intelligente où produit, processus et système de production doivent être en co-évolution.
}


\subsubsection{Economic geography}{Economie géographique}


\bpar{
In economic geography, the concept of co-evolution has also largely been used. The idea of evolutionary entities in economy comes in opposition to the neo-classical current which remains a majority, but finds a more and more relevant echo~\cite{nelson2009evolutionary}. \cite{schamp201020} proceeds to an epistemological analysis of the use of co-evolution, and opposes the view of a neo-schumpeterian approach to economy which considers the emergence of populations that evolve from micro-economic rules (what would correspond to a direct and relatively isolationist reading of biological evolution) to a systemic approach that would consider the economy as an evolutive system in a global perspective (what would correspond to diffuse co-evolution that we previously developed), to propose a precise characterization that would correspond to the first case, assuming co-evolving \emph{institutions}. The most important for our purpose is that he underlines the crucial aspect of the choice of populations and of considered entities, of the geographical area, and highlights the importance of the existence of causal circular relations.
}{
En économie géographique, le concept de co-évolution a également largement été mobilisée. L'idée d'entités évolutionnaires en économie vient à contre-courant du courant néoclassique qui reste majoritaire, mais trouve un écho de plus en plus pertinent~\cite{nelson2009evolutionary}. \cite{schamp201020} procède à une analyse épistémologique de l'utilisation de la co-évolution, et oppose une approche néo-schumpeterienne de l'économie qui considère l'émergence de populations qui évoluent à partir de règles micro-économiques (qui correspondrait à une lecture directe et relativement isolationniste de l'évolution biologique) à une approche systémique qui considérerait l'économie comme un système évolutif de manière globale (qui correspondrait à l'évolution diffuse que nous avons développé précédemment), pour proposer une caractérisation précise tombant dans le premier cas, qui suppose des \emph{institutions} qui co-évoluent. Le plus important pour notre propos est qu'il souligne l'aspect crucial du choix des population et des entités considérées, de la zone géographique, et appuie l'importance de l'existence de relations causales circulaires.
}


\bpar{
Diverse examples of application can be given. \cite{doi:10.1080/00343400802662658} introduce a conceptual frame to allow to conciliate the evolutionary nature of companies, the theory of clusters and knowledge networks, in which the co-evolution between networks and companies  is central, and which is defined as a circular causality between different characteristics of these subsystems. \cite{colletis2010co} introduces a framework for the co-evolution of territories and technology (questioning for example the role of proximity on innovations), that reveals again the importance of the institutional aspect. The framework proposed by \cite{ter2011co} couples the evolutionary approach to companies, the literature on industries and innovation in clusters, and the approach through complex networks of connexions between the latest in the territorial system.
}{
Il est possible de donner divers exemples d'application. \cite{doi:10.1080/00343400802662658} introduisent un cadre conceptuel pour permettre de concilier nature évolutionnaire des entreprises, théorie des clusters et réseaux de connaissance, dans lequel la co-évolution entre réseaux et entreprises est centrale, et qui est définie comme une causalité circulaire entre différentes caractéristiques de ces sous-systèmes. \cite{colletis2010co} introduit un cadre de co-évolution des territoires et de la technologie (questionnant par exemple le rôle de la proximité pour les innovations), qui révèle l'importance à nouveau de l'aspect institutionnel. Le cadre proposé par \cite{ter2011co} couple la vision évolutionnaire des entreprises, la littérature sur les industries et l'innovation dans les clusters, et l'approche par réseau complexe des connexions entre ces premiers dans le système territorial.
}

\bpar{
In environmental economics, \cite{kallis2007coevolution} show that ``broad'' approaches (that can consider most of co-dynamics as co-evolutive) are opposed to stricter approaches (in the spirit of the definition given by \cite{schamp201020}), and that in any case a precise definition, not necessarily coming from biology, must be given, in particular for the search of an empirical characterization.
}{
En économie environnementale, \cite{kallis2007coevolution} montre que des approches ``larges'' (pouvant considérer la majorité des co-dynamiques comme co-évolutives) s'opposent à des approches plus strictes (dans l'esprit de la définition donnée par \cite{schamp201020}), et que dans tous les cas une définition précise, ne venant pas forcément de la biologie, doit être donnée, en particulier pour la recherche d'une caractérisation empirique.
}



\subsubsection{Geography}{Géographie}


\bpar{
For geography, as we already presented in introduction, the works that are the closest to notions of co-evolution empirically and theoretically are closely linked to the evolutive urban theory. It is not easy to track back in the literature at what time the notion was clearly formalized, but it is clear that it was present since the foundations of the theory as recalls \noun{Denise Pumain} (see~\ref{app:sec:interviews}): the complex adaptive system is composed of subsystems that are interdependent in a complex way, often with circular causalities. The first models indeed include this vision in an implicit way, but co-evolution is not explicitly highlighted of precisely defined, in terms that would be quantifiable or structurally identifiable. \cite{paulus2004coevolution} brings empirical proofs of mechanisms of co-evolution through the study of the evolution of economic profiles of French cities. The interpretation used by~\cite{schmitt2014modelisation} is based on an entry by the evolutive urban theory, and fundamentally consists in a reading of systems of cities as highly interdependent entities.
}{
Pour la géographie, comme nous l'avons déjà présenté en introduction, les travaux les plus proches empiriquement et théoriquement des notions de co-évolution sont étroitement liés à la théorie évolutive des villes. Il n'est pas évident de tracer dans la littérature à quel moment la notion a été clairement formalisée, mais il est évident qu'elle était présente dès les fondements de la théorie comme le rappelle \noun{Denise Pumain} (voir~\ref{app:sec:interviews}) : le système complexe adaptatif est composé de sous-systèmes en interdépendances complexes, souvent circulairement causales. Les premiers modèles incluent bien cette vision de manière implicite, mais la co-évolution n'est pas appuyée explicitement ou définie précisément, en termes qui seraient quantifiables ou identifiables structurellement. \cite{paulus2004coevolution} amène des preuves empiriques de mécanismes de co-évolution par l'étude de l'évolution des profils économiques des villes françaises. L'interprétation utilisée par~\cite{schmitt2014modelisation} repose sur une entrée par la théorie évolutive des villes, et consiste fondamentalement en une lecture des systèmes de villes comme entités fortement interdépendantes.
}


\subsubsection{Physical geography}{Géographie physique}


\bpar{
In the study of landscapes, \cite{sheeren2015coevolution} evoke the co-evolution of landscape and agricultural activities, but in fact do not consider any circular effect of one on the other. Their result show a priori that the evolution of agricultural practices yield an evolution of the landscape, and it is not clear to what extent the conceptual frame of co-evolution, evoked without any more details, is used.
}{
En étude des paysages, \cite{sheeren2015coevolution} parlent de co-évolution du paysage et des activités agricoles, mais ne considèrent en fait pas d'effet circulaires de l'un sur l'autre. A priori, leurs résultats montrent que l'évolution des pratiques agricoles entraine une évolution du paysage, et il n'est ainsi pas clair dans quelle mesure le cadre conceptuel de la co-évolution, mentionné sans plus de détails, est mobilisé.
}


\subsubsection{Physics}{Physique}


\bpar{
Finally, we can mention in an anecdotical way that the term of co-evolution has also been used by physics. Its use for physical systems may induce some debates, depending if we suppose or not that the transmission assumes a transmission of \emph{information}\footnote{Information is defined within the shanonian theory as an occurence probability for a chain of characters. \cite{morin1976methode} shows that the concept of information is indeed far more complex, and that it must be thought conjointly to a given context of the generation of a self-organizing negentropic system, i.e. realizing local decreases in entropy in particular thanks to this information. This type of system is necessarily alive. We will follow here this complex approach to information.}. In the case of a purely physical ontological transmission (\emph{physical beings}), then a large part of physical systems are evolutive. \cite{hopkins2008cosmological} develop a cosmological frame for the co-evolution of cosmic heterogenous objects which presence and dynamics are difficultly explained by more classical theories (some types of galaxies, quasars, supermassive black holes). \cite{antonioni2017coevolution} study the co-evolution between synchronisation and cooperation properties within a Kuramoto oscillators network\footnote{The Kuramoto model studies synchronization within complex systems, by studying the evolution of phases $\theta_i$ coupled by interaction equations $\dot{\vec{\theta}} = \vec{\omega} + \vec{W}\left[\vec{\theta}\right] + \mathbf{B}$ where $\vec{\omega}$ are proper forcing phases and the coupling strength between $i$ and $j$ is given by $\vec{W}_{i} = \sum_j w_{ij} \sin\left(\theta_i - \theta_j\right)$ and $\vec{B}$ is noise.}, showing on the one hand that the concept can be applied to abstract objects, and on the other hand that a complex network of relations between variables can be at the origin of dynamics witnessing circular causalities, i.e. a co-evolution in that sense.
}{
Enfin, on peut noter de manière anecdotique que le terme de co-évolution a également été utilisé par la physique. L'utilisation pour des systèmes physiques peut porter à débat, selon que l'on suppose ou non que la transmission suppose une transmission d'\emph{information}\footnote{L'information est définie dans la théorie shanonienne comme une probabilité d'occurrence d'une chaîne de caractère. \cite{morin1976methode} montre que le concept d'information est en fait bien plus complexe, et qu'il doit être pensé conjointement à un contexte donné de génération d'un système auto-organisateur néguentropique, i.e. réalisant des diminutions locales d'entropie notamment grâce à cette information. Ce type de système est nécessairement vivant. Nous prendrons ici cette vision complexe de l'information.}. Dans le cas d'une transmission ontologique uniquement physique (\emph{êtres physiques}), alors une grande partie des systèmes physiques sont évolutifs. \cite{hopkins2008cosmological} développent un cadre cosmologique pour la co-évolution d'objets cosmiques hétérogènes dont la présence et les dynamiques sont difficilement expliquées par des théories plus classiques (certains types de galaxies, quasars, trous noirs supermassifs). \cite{antonioni2017coevolution} étudient la co-évolution entre des propriétés de synchronisation et de coopération au sein d'un réseau d'oscillateurs de Kuramoto\footnote{Le modèle de Kuramoto s'intéresse à la synchronisation au sein de systèmes complexes, en étudiant l'évolution de phases $\theta_i$ couplée par les équations d'interaction $\dot{\vec{\theta}} = \vec{\omega} + \vec{W}\left[\vec{\theta}\right] + \mathbf{B}$ où $\vec{\omega}$ sont les phases propres de forçage et la force de couplage entre $i$ et $j$ est donnée par $\vec{W}_{i} = \sum_j w_{ij} \sin\left(\theta_i - \theta_j\right)$ et $\vec{B}$ du bruit.}, montrant d'une part que le concept peut être appliqué à des objets abstraits, et d'autre part qu'un réseau de relations complexes entre variables peut être à l'origine de dynamiques présentant des causalités circulaires, c'est-à-dire d'une co-évolution en ce sens.
}

\subsubsection{Synthesis}{Synthèse}


\bpar{
Most of these approaches fit in the theory of complex adaptive systems developed by \noun{Holland}, in particular in~\cite{holland2012signals}: it takes any system as an imbrication of systems of boundaries, that filter signals or objects. Within a given limit, the corresponding subsystem is relatively autonomous from the outside, and is called an \emph{ecological niche}, in a direct correspondence with highly connected communities within trophic or ecological networks. This way, interdependent entities within a niche are said to be co-evolving. We will come back on that approach in our theoretical construction in~\ref{sec:theory} when we will have developed other concepts that are necessary for it.
}{
La plupart de ces approches rentrent dans la théorie des systèmes complexes adaptatifs développée par \noun{Holland}, notamment dans~\cite{holland2012signals} : il voit tout système comme une imbrication de systèmes de limites, filtrant des signaux ou des objets. Au sein d'une limite donnée, le sous-système correspondant est relativement autonome de l'extérieur, est est appelé \emph{niche écologique}, en correspondance directe avec les communautés fortement connectées au sein des réseaux trophiques ou écologiques. Ainsi, des entités interdépendantes au sein d'une niche sont dites en co-évolution. Nous reviendrons sur cette entrée lors de la construction théorique en~\ref{sec:theory} lorsque nous aurons développé d'autres concepts qui lui sont nécessaire.
}


\bpar{
We retain from this multidisciplinary view of co-evolution the fundamental following points, that are precursors of a proper definition of co-evolution that will be given further, concluding the first part.
}{
Nous retenons de cet aperçu multidisciplinaire de la co-évolution les points fondamentaux suivants précurseurs à une définition propre de la co-évolution que nous donnerons plus loin, en conclusion de la première partie.
}


\bpar{
\begin{enumerate}
	\item The presence of \emph{evolution processes} is primary, and their definition is almost always based on the existence of transmission and transformation processes.
	\item Co-evolution assumes entities or systems, belonging to distinct classes, which evolutive dynamics are coupled in a circular causal way. Approaches can differ depending on the assumptions of populations of these entities, singular objects, or components of a global system then in mutual interdependency without a direct circularity.
	\item The delineation of systems and subsystems, both in the ontological space (definition of studied objects), but also in space and time, and their distribution in these spaces, is fundamental for the existence of co-evolutionary dynamics, and it seems in a large number of cases, of their empirical characterization.
\end{enumerate}
}{
\begin{enumerate}
	\item La présence de \emph{processus d'évolution} est primaire, et leur définition se ramène presque toujours à l'existence de processus de transmission et de transformation.
	\item La co-évolution suppose des entités ou systèmes, appartenant à des classes distinctes, dont les dynamiques évolutives sont couplées de manière circulaire causale. Les approches peuvent différer selon l'hypothèse de populations de ces entités, d'objets singuliers, ou de composantes d'un système global alors en interdépendance mutuelle sans qu'il y ait circularité directe. % rq : dit-on qu'il y a coevol dans les cas spurieux
	\item La délimitation des systèmes ou des sous-systèmes, à la fois dans l'espace ontologique (définition des objets étudiés), mais aussi dans l'espace et le temps, ainsi que leur distribution dans ces espaces, est fondamental pour l'existence de dynamiques co-évolutives, et a priori dans un grand nombre de cas, pour leur caractérisation empirique.
\end{enumerate}
}



%----------------------------------------------------------------------------------------

\subsection{Nature of complexity and knowledge production}{Nature de la complexité et production de connaissances}


\bpar{
The two previous epistemological points that we just developed were related respectively related first to the positioning in itself, i.e. the framework to read processes of production of scientific knowledge, and then to the nature of the concepts considered. We propose to again gain in generality compared to the first one and to introduce a development modestly contributing (i.e. in our context) to the \emph{knowledge of knowledge}. The aim is to interrogate the links between complexity and processes of knowledge production.
}{
Les deux premiers points épistémologiques que nous venons de traiter relevaient respectivement du positionnement en lui-même, c'est-à-dire du cadre de lecture des processus de production de connaissance scientifique, puis de la nature des concepts considérés. Nous proposons de monter encore en généralité par rapport au premier et d'introduire un développement contribuant modestement (c'est-à-dire dans notre contexte) à \emph{la connaissance de la connaissance}. Il s'agit d'interroger les liens entre complexité et processus de production de connaissances.
}


\bpar{
One aspect of knowledge production on complex systems, that we encounter several times here (see chapter~\ref{ch:theory}), and that seems to be recurrent and even inevitable, is a certain level of reflexivity (and that would be inherent to complex system in comparison to simple systems, as we will develop further). We mean by this term both a practical reflexivity, i.e. a necessity to increase the level of abstraction, such as the need to reconstruct in an endogenous way the disciplines in which a reflexion aims at positioning as proposed in \ref{sec:quantepistemo}, or to reflect on the epistemological nature of modeling when constructing a model such as in \ref{app:sec:csframework}, but also a theoretical reflexivity in the sense that theoretical apparels or produced concepts can apply recursively to themselves. This practical observation can be related to old epistemological debates questioning the possibility of an objective knowledge of the universe that would be independent of our cognitive structure, somehow opposed to the necessity of an ``evolutive rationality'' implying that our cognitive system, product of the evolution, mirrors the complex processes that led to its emergence, and that any knowledge structure will be consequently reflexive\footnote{We thank here \noun{D. Pumain} to have formulated this alternative view on the problem that we will develop in the following.}. We do not pretend here to bring a response to such a broad and vague question as such, but we propose a potential link between this reflexivity and the nature of complexity. 
}{
Un aspect de la production de connaissance sur des systèmes complexes, auquel nous nous heurtons plusieurs fois ici (voir chapitre~\ref{ch:theory}), et qui semble être récurrent voire inévitable, est un certain niveau de réflexivité (et qui serait inhérent aux systèmes complexes en comparaison aux systèmes simples, comme nous le développerons plus loin). Nous entendons par là à la fois une réflexivité pratique, c'est-à-dire la nécessité d'élever le niveau d'abstraction, comme le besoin de reconstruire de manière endogène les disciplines dans lesquelles une réflexion cherche à se positionner comme proposé en \ref{sec:quantepistemo}, ou de réfléchir à la nature épistémologique de la modélisation lors de l'élaboration d'un modèle comme en \ref{app:sec:csframework}, mais également une réflexivité théorique en le sens que les appareils théoriques ou les concepts produits peuvent s'appliquer de manière récursive à eux-mêmes. Cette constatation pratique fait écho à des débats épistémologiques anciens questionnant la possibilité d'une connaissance objective de l'univers qui serait indépendante de notre structure cognitive, ou bien la nécessité d'une ``rationalité évolutive'' impliquant que notre système cognitif, produit de l'évolution, reflète les processus complexes ayant conduit à son émergence, et que toute structure de connaissance sera par conséquent réflexive\footnote{Nous remercions \noun{D. Pumain} d'avoir pointé cette vue alternative du problème que nous allons développer par la suite.}. Nous ne prétendons pas ici apporter une réponse à une question aussi vaste et vague telle quelle, mais proposons un lien potentiel entre cette réflexivité et la nature de la complexité.
}

\subsubsection{Complexity and complexities}{Complexité et complexités}


\bpar{
What is meant by complexity of a system often leads to misunderstandings since it can be qualified according to different dimensions and visions. We distinguish first the complexity in the sense of weak emergence and autonomy between the different levels of a system, and on which different positions can be developed as in \cite{deffuant2015visions}. We will not enter a finer granularity, the vision of social complexity giving even more nightmares to the Laplace daemon, and since it can be understood as a stronger emergence (in the sense of weak and strong emergence as developed before in~\ref{sec:computation}). We thus simplify and assume that the nature of systems plays a secondary role in our reflexion, and therefore consider complexity in the sense of an emergence.
}{
Ce qui est entendu par complexité d'un système mène souvent à des malentendus car celle-ci peut être qualifiée selon différentes dimensions et visions. Nous distinguons dans un premier temps la complexité au sens d'émergence faible et d'autonomie entre les différents niveaux d'un système, et sur laquelle différentes positions peuvent être développées comme dans \cite{deffuant2015visions}. Nous ne rentrerons pas dans une granularité plus fine, la vision de la complexité sociale donnant encore plus de fil à retordre au démon de Laplace, et pouvant être par exemple comprise par une émergence plus forte (au sens d'émergence faible et forte développée précédemment en~\ref{sec:computation}). Nous simplifions ainsi et supposons que la nature des systèmes joue un rôle secondaire dans notre reflexion, et considérons la complexité au sens d'une émergence.
}



\bpar{
Moreover, we distinguish two other ``types'' of complexity, namely computational complexity and informational complexity, that can be seen as measures of complexity, but that are not directly equivalent to emergence, since there exists no systematic link between the three. We can for example consider the use of a simulation model, for which interactions between elementary agents translate as a coded message at the upper level: it is then possible by exploiting the degrees of freedom to minimize the quantity of information contained in the message. The different languages require different cognitive efforts and compress the information in a different way, having different levels of measurable complexity~\cite{febres2013complexity}. In a similar way, architectural artefacts are the result of a process of natural and cultural evolution, and witness more or less this trajectory.
}{
D'autre part, nous distinguons deux autres ``types'' de complexité, la complexité computationnelle et la complexité informationnelle, qui peuvent être vues comme des mesures de complexité, mais qui ne sont pas directement équivalentes à l'émergence, puisqu'il n'existe pas de lien systématique entre les trois. On peut par exemple imaginer utiliser un modèle de simulation, pour lequel les interactions entre agents élémentaires se traduisent par un message codé au niveau supérieur : il est alors possible en exploitant les degrés de liberté de minimiser la quantité d'information contenue dans le message. Les différentes langues demandent des efforts cognitifs différents et compressent différemment l'information, ayant différents niveau de complexité mesurables~\cite{febres2013complexity}. De même, des artefacts architecturaux sont le résultat d'un processus d'évolution naturelle puis culturelle et peuvent témoigner plus ou moins de cette trajectoire.
}


\bpar{
Numerous other conceptual or operational characterizations of complexity exist, and it is clear that the scientific community has not converged on a unique definition~\cite{chu2008criteria}\footnote{In an approach that is in a way reflexive, \cite{chu2008criteria} proposes to continue exploring the different existing approaches, as proxies of complexity in the case of an essentialism, or as concepts in themselves. The complexity should emerge naturally from the interaction between these different approaches studying complexity, hence the reflexivity.}. We propose to focus on these three concepts in particular, for which the relations are already not evident.
}{
De nombreuses autres caractérisations conceptuelles ou opérationnelles de la complexité existent, et il est clair que la communauté scientifique n'a pas convergé sur une définition unique~\cite{chu2008criteria}\footnote{Dans une approche en un sens réflexive, \cite{chu2008criteria} propose de continuer d'explorer les différentes approches existantes, comme des proxys de la complexité dans le cas d'un essentialisme, ou comme des concepts à part entière. La complexité devrait émerger d'elle même de l'interaction entre ces différentes approches étudiant la complexité, d'où la réflexivité.}. Nous proposons de nous concentrer sur ces trois concepts en particulier, pour lesquels les relations ne sont déjà pas évidentes.
}


\bpar{
Indeed, links between these three types of complexity are not systematic, and depend on the type of system. Epistemological links can however be introduced. We will develop the links between emergence and the two other complexities, since the link between computational complexity and informational complexity is relatively well explored, and corresponds to issues in the compression of information and signal processing, or moreover in cryptography.
}{
En effet, les liens entre ces trois types de complexité ne sont pas systématiques, et dépendent du type de système. Des liens épistémologiques peuvent néanmoins être introduits. Nous traitons ceux entre émergence et les deux autres complexités, étant donné que le lien entre complexité computationnelle et complexité informationnelle est assez bien compris et relève de problématiques de compression de l'information et de traitement du signal, ou encore de cryptographie.
}


\subsubsection{Computational complexity and emergence}{Complexité computationnelle et émergence}



\bpar{
Different clues suggest a certain necessity of computational complexity to have emergence in complex systems, whereas reciprocally a certain number of adaptive complex systems have high computational capabilities.
}{
Différents indices suggèrent une certaine nécessité de complexité computationnelle pour avoir émergence dans des systèmes complexes, tandis que réciproquement un certain nombre de systèmes complexes adaptatifs sont dotés de capacités de calcul élevées.
}


\bpar{
A first link where computational complexity implies emergence is suggested by an algorithmic study of fundamental problems in quantum physics. Indeed, \cite{2014arXiv1403.7686B} shows that the resolution of the Schrödinger equation with any Hamiltonian is a NP-hard and NP-complete problem, and thus that the acceptation of $\mathbf{P}\neq\mathbf{NP}$ implies a qualitative separation between the microscopic quantum level and the macroscopic level of the observation. Therefore, it is indeed the complexity (here in the sense of their computation) of interactions in a system and its environment that implies the apparent collapse of the wave function, what rejoins the approach of \noun{Gell-Mann} by quantum decoherence~\cite{gell1996quantum}, which explains that probabilities can only be associated to decoherent histories (in which correlations have led the system to follow a trajectory at the macroscopic scale)\footnote{The \emph{Quantum Measurement Problem} arises when we consider a microscopic wave function giving the state of a system that can be the superposition of several states, and consists in a theoretical paradox, on the one hand the measures being always deterministic whereas the system has probabilities for states, and on the other hand the issue of the non-existence of superposed macroscopic states (collapse of the wave function). As reviewed by~\cite{schlosshauer2005decoherence}, different epistemological interpretations of quantum physics are linked to different explanations of this paradox, including the ``classical'' Copenhagen one which attributes to the act of observation the role of collapsing the wave function. \noun{Gell-Mann} recalls that this interpretation is not absurd since it is indeed the correlations between the quantum object and the world that product the decoherent history, but that it is far more specific, and that the collapse happens in the emergence itself: the cat is either dead or living, but not both, before we open the box.}. The paradox of the Schrödinger cat appears then as a fundamentally reductionist perspective, since it assumes that the superposition of states can propagate through the successive levels and that there would be no emergence, in the sense of the constitution of an autonomous upper level. In other terms, the work of \cite{2014arXiv1403.7686B} suggests that computational complexity is sufficient for the presence of emergence\footnote{This effective separation of scales does not a priori imply that the lower level does not play a crucial role, since \cite{vattay2015quantum} proves that the properties of quantum criticality are typical of molecules of the living, without a priori any specificity for life in this complex determination by lower scales: \cite{2016arXiv161102269V} has recently introduced a new approach linking quantum theories and general relativity in which it is shown that gravity is an emergent phenomenon and that path-dependency in the deformation of the original space introduces a supplementary term at the macroscopic level, that allows to explain deviations attributed up to now to \emph{dark matter}.}.
}{
Un premier lien où complexité computationnelle implique émergence est suggéré par un examen algorithmique des problèmes fondamentaux de la physique quantique. En effet, \cite{2014arXiv1403.7686B} démontre que la résolution de l'équation de Schrödinger avec Hamiltonien quelconque est un problème NP-difficile et NP-complet, et donc que l'acceptation de $\mathbf{P}\neq\mathbf{NP}$ implique une séparation qualitative entre le niveau quantique microscopique et le niveau d'observation macroscopique. Ainsi, c'est bien la complexité (ici au sens de leur calcul) des interactions au sein du système et de son environnement qui explique l'apparente réduction du paquet d'onde, ce qui rejoint l'approche de \noun{Gell-Mann} par la décohérence quantique~\cite{gell1996quantum}, qui explique que des probabilités ne peuvent être associées qu'aux histoires décohérentes (dans lesquelles les corrélations ont fait prendre une trajectoire au système à l'échelle macroscopique)\footnote{Le \emph{Problème de la Mesure Quantique} se pose lorsqu'on considère une fonction d'onde microscopique donnant l'état d'un système pouvant être superposition de plusieurs états, et consiste en un paradoxe théorique, les mesures étant toujours déterministes alors que le système a des probabilité d'états d'une part, et le problème de la non-existence d'états macroscopiques superposés (réduction du paquet d'onde) d'autre part. Comme revu par~\cite{schlosshauer2005decoherence}, différentes interprétations épistémologiques de la physique quantiques sont liées à différentes explications de ce paradoxe, dont celle ``classique'' de Copenhague qui donne à l'acte d'observation le rôle de reduction du paquet d'onde. \noun{Gell-Mann} précise que cette interprétation n'est pas absurde puisque c'est bien les corrélations entre l'objet quantique et le monde qui produisent l'histoire décohérente, mais qu'elle est bien trop spécifique, et que la réduction a lieu dans l'émergence elle-même : le chat est bien mort ou vivant, mais pas les deux, avant que l'on ouvre la boîte.}. Le paradoxe du chat de Schrödinger nous apparait ainsi comme une perspective fondamentalement réductionniste, puisqu'il suppose que la superposition d'états peut se propager à travers les niveaux successifs et qu'il n'y aurait pas émergence, au sens de constitution d'un niveau supérieur autonome. En d'autres termes, le travail de \cite{2014arXiv1403.7686B} suggère que la complexité computationnelle est suffisante pour la présence d'émergence\footnote{A priori, cette séparation effective des échelles n'implique pas que le niveau inférieur ne joue pas un rôle crucial, puisque \cite{vattay2015quantum} prouve que les propriétés de criticalité quantiques sont typiques des molécules du vivant, sans qu'il n'y ait a priori de spécificité pour la vie dans cette détermination complexe par les échelles inférieures : \cite{2016arXiv161102269V} a introduit une nouvelle approche liant théories quantiques et relativité générale dans laquelle il est montré que la gravité est un phénomène émergent et que la dépendance au chemin dans la déformation de l'espace de base introduit un terme supplémentaire au niveau macroscopique, qui permet d'expliquer les déviations attribuées jusqu'alors à la \emph{matière noire}.}.
}


%  bizarre : epistemo de la QM pas bien developpée ? (que physiciens cloisonnés ? du coup philo de comptoir ? et conflits avec informaticiens ? check si Moore en parle)



\bpar{
Reciprocally, the link between computational complexity and emergence is revealed by questions linked to the nature of computation~\cite{moore2011nature}. Cellular automatons, that are moreover crucial for the understanding of several complex systems, have been shown as Turing-complete\footnote{A system is said to be Turing-complete if it is able to compute the same functions than a Turing machine, commonly accepted as all what is ``computable'' (\noun{Church}'s thesis). We recall that a Turing machine is a finite automaton with an infinite writing band~\cite{moore2011nature}.}, such as the Game of Life~\cite{beer2004autopoiesis}\footnote{There even exists a programming language allowing to code in the \emph{Game of Life}, available at \url{https://github.com/QuestForTetris}. Its genesis finds its origin in a challenge posted on \emph{codegolf} aiming at the conception of a Tetris, and ended in an extremely advanced collaborative project.}. Some organisms without a central nervous system are capable of solving difficult decisional problems~\cite{reid2016decision}. An ant-based algorithm is shown by~\cite{Pintea2017} as solving a Generalized Travelling Salesman Problem (GTSP), problem which is NP-difficult. This fundamental link had already been conceived by \noun{Turing}, since beyond his fundamental contributions to contemporary computer science, he studied morphogenesis and tried to produce chemical models to explain it~\cite{turing1952chemical} (that were far from actually explaining it - it is still not well understood today, see~\ref{sec:interdiscmorphogenesis} - but which conceptual contributions were fundamental, in particular for the notion of reaction-diffusion). We moreover know that a minimum of complexity in terms of constituting interactions in a particular case of agent-based system (models of boolean networks), and thus in terms of possible emergences, implies a lower bound on computational complexity, which becomes significant as soon as interactions with the environment are added~\cite{tovsic2017boolean}.
}{
Dans le sens inverse, le lien entre complexité computationnelle et émergence est mis en valeur par les questions liées à la nature de la computation~\cite{moore2011nature}. Des automates cellulaires, qui sont par ailleurs cruciaux pour la compréhension de divers systèmes complexes, ont été montrés Turing-complets\footnote{Un système est Turing-complet s'il est capable de calculer les mêmes fonctions qu'une machine de Turing, communément accepté comme l'ensemble du ``calculable'' (thèse de \noun{Church}). Pour mémoire, une machine de Turing est un automate fini à bande d'écriture infinie~\cite{moore2011nature}.}, comme le Jeu de la Vie~\cite{beer2004autopoiesis}\footnote{Il existe même un langage de programmation permettant de programmer en \emph{Game of Life}, disponible à \url{https://github.com/QuestForTetris}. Sa genèse trouve son origine dans un défi posté sur \emph{codegolf} ayant pour but la conception d'un Tetris, et a abouti à un projet collaboratif extrêmement avancé.}. Des organismes sans système nerveux central sont capables de résoudre des problèmes décisionnels difficiles~\cite{reid2016decision}. Un algorithme à base de fourmis est montré par~\cite{Pintea2017} comme résolvant un Problème du Voyageur de Commerce Généralisé (GTSP), problème NP-difficile. Ce lien fondamental avait déjà été envisagé par \noun{Turing}, puisqu'au delà de ses contributions fondamentales à l'informatique moderne, il s'était intéressé à la morphogenèse et a tenté de produire des modèles chimiques d'explication de celle-ci~\cite{turing1952chemical} (qui étaient très loin de effectivement l'expliquer - elle n'est toujours pas bien comprise aujourd'hui, voir~\ref{sec:interdiscmorphogenesis} - mais dont les contributions conceptuelles ont été fondamentales, notamment pour la notion de réaction-diffusion). On sait par ailleurs qu'un minimum de complexité en termes d'interactions constituantes dans un cas particulier de système basé sur les agents (modèles de réseaux booléens), et donc d'émergences possibles, implique une borne inférieure sur la complexité computationnelle, qui devient conséquente dès que les interactions avec l'environnement sont ajoutées~\cite{tovsic2017boolean}.
}


%\comment{\cite{2017arXiv170404231E} quantum computation reduces drastically memory needed}


\subsubsection{Informational complexity and emergence}{Complexité informationnelle et émergence}


\bpar{
Informational complexity, or the quantity of information contained in a system and the way it is stored, also bears some fundamental links with emergence. Information is equivalent to the entropy of a system and thus to its degree of organisation - this what allows to solve the apparent paradox of the Maxwell Daemon that would be able to diminish the entropy of an isolated system and thus contradict the second law of thermodynamics: it indeed uses the information on positions and velocities of molecules of the system, and its action balances to loss of entropy through its captation of information\footnote{The Maxwell Daemon is more than an intellectual construction: \cite{cottet2017observing} implements experimentally a daemon at the quantic level.}.
}{
La complexité informationnelle, ou la quantité d'information contenue dans un système et la manière dont celle-ci est stockée, entretient également des liens fondamentaux avec l'émergence. L'information est équivalente à l'entropie d'un système et donc à son degré d'organisation - c'est ce qui a permis de résoudre le paradoxe apparent du Démon de Maxwell qui serait capable de diminuer l'entropie d'un système isolé et donc contredire la deuxième loi de la thermodynamique : celui-ci utilise en fait l'information sur les positions et vitesses des molécules du système, et son action compense la perte d'entropie par sa captation d'information\footnote{Le démon de Maxwell est plus qu'une construction intellectuelle : \cite{cottet2017observing} implémente un démon expérimentalement au niveau quantique.}.
}


\bpar{
This notion of local increase in entropy has been largely studied by \noun{Chua} under the form of the \emph{Local Activity Principle}, which is introduced as a third principle of thermodynamics, allowing to explain with mathematical arguments the self-organization for a certain class of complex systems that typically involve reaction-diffusion equations~\cite{mainzer2013local}.
}{
Cette notion d'accroissement local de l'entropie a été étudiée largement par \noun{Chua} sous la forme du \emph{Local Activity Principle}, qui est introduit comme un troisième principe de la thermodynamique, permettant d'expliquer par des arguments mathématiques l'auto-organisation pour une certaine classe de systèmes complexes typiquement impliquant des équations de réaction-diffusion~\cite{mainzer2013local}.
}



\bpar{
The way information is stored and compressed is essential for life, since the ADN is indeed an information storage system, which role at different levels is far from being fully understood. Cultural complexity also witnesses of an information storage at different levels, for example within individuals but also within artefacts and institutions, and information flows that necessarily deal with the two other types of complexities. Information flows are essential for self-organization in a multi-agent system. Collective behaviors of fishes or birds are typical examples used to illustrate emergence and belong to the canonic examples of complex systems. We only begin to understand how these flows structure the system, and what are the spatial patterns of information transfer within a \emph{flock} for example: \cite{crosato2017informative} introduce first empirical results with transfer entropy for fishes and lay the methodological basis of this kind of studies. 
}{
La manière dont l'information est stockée et compressée est essentielle pour la vie, puisque l'ADN est bien un système de stockage d'information, dont le rôle à différents niveaux est bien loin d'être compris complètement. La complexité culturelle témoigne également d'un stockage de l'information à différents niveaux, par exemple au sein des individus mais aussi des artefacts et des institutions, et des flux d'information relevant nécessairement des deux autres types de complexité. Les flux d'information sont essentiels pour l'auto-organisation dans un système multi-agents. Les comportements collectifs de poissons ou d'oiseaux sont des exemples typiques utilisés pour illustrer l'émergence et font partie des cas d'école de systèmes complexes. On commence cependant seulement à comprendre comment ces flux structurent le système, et quels sont les motifs spatiaux de transfert d'information au sein d'un \emph{flock} par exemple : \cite{crosato2017informative} introduisent des premiers résultats empiriques avec l'entropie de transfert pour des poissons et posent les bases méthodologiques de ce type d'étude.
}



\subsubsection{Knowledge production}{Production de connaissances}


\bpar{
We know have enough material to come to reflexivity. It is possible to position knowledge production at the intersection of interactions between types of complexity developed above. First of all, knowledge as we consider it can not be dissociated from a collective construction, and implies thus an encoding and a transmission of information: it is at an other level all problematics linked to scientific communication. The production of knowledge thus necessitates this first interaction between computational complexity and informational complexity. The link between informational complexity and emergence is introduced if we consider the establishment of knowledge as a morphogenetic process. It is shown in~\ref{sec:interdiscmorphogenesis} that the link between form and function is fundamental in psychology: we can interpret it as a link between information and meaning, since semantics of a cognitive object can not be considered without a function. \noun{Hofstader} recalls in~\cite{hofstadter1980godel} the importance of symbols at different levels for the emergence of a thought, that consist in signals at an intermediate level. Finally, the last relation between computational complexity and emergence is the one allowing us a positioning in particular on knowledge production on complex systems, the previous links being applicable to any type of knowledge.
}{
Nous avons à présent la matière suffisante pour en venir à la réflexivité. Il est possible de positionner la production de connaissances à l'intersection des interactions entre types de complexité développées ci-dessus. Tout d'abord, la connaissance telle que nous l'envisageons ne peut se passer d'une construction collective, et implique donc un encodage et une transmission de l'information : il s'agit à un autre niveau de toutes les problématiques liées à la communication scientifique. La production de connaissances nécessite donc cette première interaction entre complexité computationnelle et complexité informationnelle. Le lien entre complexité informationnelle et émergence est mobilisé si on considère l'établissement de connaissances comme un processus morphogénétique. Il est montré en~\ref{sec:interdiscmorphogenesis} que le lien entre forme et fonction est fondamental en psychologie : nous pouvons l'interpréter comme un lien entre information et sens, puisque la sémantique d'un objet cognitif ne peut se passer d'une fonction. \noun{Hofstader} rappelle dans~\cite{hofstadter1980godel} l'importance des symboles à différents niveaux pour l'émergence d'une pensée, qui consistent à un niveau intermédiaire en des signaux. Enfin, la dernière relation entre complexité computationnelle et émergence est celle qui nous permet d'affirmer qu'on s'intéresse particulièrement à une production de connaissance sur des systèmes complexes, les deux premiers pouvant s'appliquer à tout type de connaissance.
}


\bpar{
Therefore, any \emph{knowledge of the complex} embraces not only all complexities and their relations in its content, but also in its nature as we just showed. The structure of knowledge in terms of complexity is analog to the structure of systems its studies. We postulate that this structural correspondence implies a certain recursivity, and thus a certain level of \emph{reflexivity} (in the sens of knowledge of itself and its own conditions).
}{
Ainsi, toute \emph{connaissance du complexe} embrasse non seulement toutes les complexités et leur relations dans son contenu, mais aussi dans sa nature comme nous venons de montrer. La structure de la connaissance en termes de complexité est analogue à la structure des systèmes qu'elle étudie. Nous postulons que cette correspondance structurelle implique une certaine récursivité, et donc un certain niveau de \emph{réflexivité} (au sens de connaissance d'elle-même et de ses propres conditions). 
}


%Comme ces systèmes sont généralement multi-niveaux, ou présentent au moins un certain niveau de complexité computationnelle, la connaissance de ceux-ci se doit de la capturer, puisque même des modèles \emph{simples} devront capturer leur complexité de manière conceptuelle et impliquer une structure conceptuelle sous-jacente complexe, même si celle-ci n'est pas explicitement explorée. %anticipation sur la requisite complexity ?


\bpar{
We can try to extend to reflexivity in terms of a reflexion on the disciplinary positioning: following \cite{pumain2005cumulativite}, the complexity of an approach is also linked to the diversity of viewpoints that are necessary to construct it. To reach this new type of complexity\footnote{For which links with the previous types naturally appear: for example, \cite{gell1995quark} considers the effective complexity as an \emph{Algorithmic Information Content} (close to Kolmogorov complexity) of a Complex Adaptive System \emph{which is observing an other} Complex Adaptive System, what gives their importance to informational and computational complexities and suggests the importance of the observational viewpoint, and by extension of their combination - what furthermore must be related to the perspectivist approach of complex sciences presented above.}, that would be a supplementary dimension linked to the knowledge of complex systems, reflexivity must be at the core of the approach. \cite{read2009innovation} recall that innovation has been made possible when societies reached the ability to produce and diffuse innovation on their own structure, i.e when they were able to reach a certain level of reflexivity. The \emph{knowledge of the complex} would thus be the product and the support of its own evolution thanks to reflexivity which played a fundamental role in the evolution of the cognitive system: we could thus suggest to gather these considerations, as proposed by \noun{Pumain}, as a new epistemological notion of \emph{evolutive rationality}.
}{
On peut tenter d'étendre à la réflexivité en tant que réflexion sur le positionnement disciplinaire : suivant \cite{pumain2005cumulativite}, la complexité d'une approche est également liée à la diversité des points de vue nécessaires pour la construire. Pour atteindre ce nouveau type de complexité\footnote{Pour laquelle des liens avec les types précédents apparaissent naturellement : par exemple, \cite{gell1995quark} considère la complexité effective comme le \emph{Contenu d'Information Algorithmique} (proche de la complexité de Kolmogorov) d'un Système Complexe Adaptatif \emph{observant un autre} Système Complexe Adaptatif, ce qui donne son importance aux complexités informationnelle et computationnelle et suggère l'importance du point de vue d'observation, et par extension de la combinaison de ceux-ci - ce qui est par ailleurs à mettre en relation avec l'approche perspectiviste des sciences complexes présentée précédemment.}, qui serait une dimension supplémentaire liée à la connaissance des systèmes complexes, la réflexivité doit être au coeur de la démarche. \cite{read2009innovation} rappellent que l'innovation a été rendue possible quand les sociétés ont été capables de produire et diffuser de l'information sur leur propre structure, c'est-à-dire quand elles ont pu atteindre un certain niveau de réflexivité. La \emph{connaissance du complexe} serait donc le produit et le support de sa propre évolution grâce à la réflexivité qui a joué un rôle fondamental dans l'évolution du système cognitif : on pourrait ainsi suggérer de rassembler ces considérations, comme proposé par \noun{Pumain}, sous une nouvelle notion épistémologique de \emph{rationalité évolutive}.
}


\bpar{
To conclude, we can remark that given the law of \emph{requisite complexity}, proposed by \cite{gershenson2015requisite} as an extension of \emph{requisite variety}~\cite{ashby1991requisite}\footnote{One of the crucial principles of cybernetics, the \emph{requisite variety}, postulates that to control a system having a certain number of states, the controller must have at least as much states. \noun{Gershenson} proposes a conceptual extension of complexity, which can be justified for example by \cite{allen2017multiscale} which introduce the multi-scale \emph{requisite variety}, showing the compatibility with a theory of complexity based on information theory.}, the \emph{knowledge of the complex} will necessarily have to be a \emph{complex knowledge}. This other point of view reinforces the necessity of reflexivity, since following \noun{Morin} (see for example \cite{morin1991methode} on the production of knowledge), the \emph{knowledge of knowledge} is central in the construction of a complex thinking.
}{
Pour conclure, notons qu'étant donné la loi de la \emph{requisite complexity}, proposée par \cite{gershenson2015requisite} comme extension de la \emph{requisite variety}~\cite{ashby1991requisite}\footnote{L'un des principes cruciaux de la cybernétique, la \emph{requisite variety}, postule que pour contrôler un système ayant un certain nombre d'états, le contrôleur doit avoir au moins autant d'états. \noun{Gershenson} propose une extension conceptuelle à la complexité, qui peut être justifiée par exemple par \cite{allen2017multiscale} qui introduisent la \emph{requisite variety} multi-échelle, démontrant la compatibilité avec une théorie de la complexité basée sur la théorie de l'information.}, la \emph{connaissance du complexe} devra nécessairement être \emph{connaissance complexe}. Cet autre point de vue renforce la nécessité de la réflexivité, puisque suivant \noun{Morin} (voir par exemple \cite{morin1991methode} sur la production de connaissance), la \emph{connaissance de la connaissance} est centrale dans l'établissement d'une pensée complexe.
}



%\comment[AB]{personnellement je ne suis pas très fan de cette idée de « connaissance complexe » ou même de « pensée complexe » (ok je suis un mauvais morinien !). }[(JR) mal formule peu etre, ``connaissance du complexe'' $\rightarrow$ cf dernière phrase : serait equivalent selon le point de vue du controle ; rejoint Morin]

% Remarques : 
% portugali semantic information : chaud à introduire.
% check paper Valentina ``Practical Reflexivity''
% lire Morin sur la pensée complexe





%%%%%%%%%%%%%%%%%%%%%%
\subsubsection*{Practical implications}{Conséquences pratiques}


\bpar{
To conclude this epistemological section, we propose to synthesize all the ideas introduced as concrete manifestations that directly yield from them, and that strongly condition all the forms and semantics of knowledge introduced in the following. These directions (that we will not go up to name principles since they are only at the state of sketch) can be grouped into three large families: modeling practices, Open Science practices, and epistemology. On the domain of modeling practices, in each section emerge different axis that are more or less complementary:
}{
Pour conclure cette section épistémologique, nous proposons de synthétiser l'ensemble des idées introduites sous forme de manifestations concrètes en découlant directement, et qui conditionneront fortement l'ensemble de la forme et de la sémantique de la connaissance introduite par la suite. Ces directions (que nous n'irons pas jusqu'à nommer principes car seulement à l'état d'ébauche) peuvent être regroupées en trois grandes familles : pratiques de modélisation, pratique de la science ouverte, et épistémologie. Sur le plan des pratiques de modélisation, dans chaque section se dégagent différents axes plus ou moins complémentaires :
}

\bpar{
\begin{itemize}
	\item Modeling, which will be in most cases equivalent to simulation, must be understood as an indirect tool of knowledge on processes within a complex system or on its structure (according to the section on ``why modeling''), and models will necessarily have to be complex (following the reflexion on the different types of complexity) in the sense that they capture a phenomenon of weak emergence, but still respecting constraints of parsimony.
	\item The exploration of models is fully contained in the modeling enterprise (see reproducibility), and intensive computation is a cornerstone to efficiently explore simulation models (see intensive computation). Sensitivity analysis methods must be questioned and extended if needed (as illustrates the example of the sensitivity to space).
	\item As suggested by the perspectivist positioning, the coupling of models will have to play a crucial role in the capture of complexity.
\end{itemize}
}{
\begin{itemize}
	\item La modélisation, qui sera dans la majorité des cas équivalente à la simulation, doit être comprise comme un instrument de connaissance indirect sur des processus au sein d'un système complexe ou sur la structure de celui-ci (d'après la sous-section sur ``pourquoi modéliser''), et les modèles devront nécessairement être complexes (d'après la réflexion sur les différents types de complexité) au sens qu'il capturent un phénomène d'émergence faible, tout en respectant des exigences de parcimonie.
	\item L'exploration des modèles est partie intégrante de l'entreprise de modélisation (voir reproductibilité), et le calcul intensif est un élément clé pour explorer efficacement les modèles de simulation (voir calcul intensif). Les méthodes d'analyse de sensibilité doivent être questionnées et étendues si besoin (comme l'illustre l'exemple de la sensibilité à l'espace).
	\item Comme suggéré par le positionnement perspectiviste, le couplage de modèles devra jouer un rôle crucial dans la capture de la complexité.
\end{itemize}
}

\bpar{
Concerning open science, we can extract the following points:
}{
Pour la science ouverte, on peut extraire les points suivants :
}

\bpar{
\begin{itemize}
	\item The necessity of all measures linked to open science to allow the construction always more complex models, towards the co-construction of models by different disciplines.
	\item In this frame, the full opening of source code, together with its readability are crucial. The complete explicitation of the model in the scientific reporting, and a self-sustaining code documentation, are two aspect of it.
	\item The question of open data is not negotiable in that frame. The quasi-totality of our treatments is based on initially open data, and when it is not the case we work at an aggregated level for which data can be opened. Constructed open data are open.
	\item Concerning the methods of interactive exploration, which are an aspect of opening science, we develop some, but stay limited compared to the ideal requirement that these should be fully compatible with a reproducible approach.
\end{itemize}
}{
\begin{itemize}
	\item La nécessité de l'ensemble des démarches liées à la science ouverte pour parvenir à la construction de modèles toujours plus complexes, vers la co-construction de modèles par différentes disciplines.
	\item Dans ce cadre, l'ouverture complète du code source, ainsi que sa lisibilité sont cruciaux. L'explicitation complète du modèle dans le compte-rendu scientifique, ainsi qu'une documentation du code auto-suffisante, sont deux aspects de celle-ci.
	\item La question des données ouvertes n'est pas négociable dans ce cadre. La quasi-totalité de nos traitements est basée sur des données initialement ouvertes, et lorsque ce n'est pas le cas nous travaillons à un niveau agrégé auquel on peut fournir les données. Les jeux de données construits sont ouverts.
	\item Concernant les méthodes d'exploration interactive, qui sont un pendant de l'ouverture de la science, nous en développons un certain nombre, mais restons limités par rapport au pré-requis idéal qui devrait rendre celles-ci totalement compatibles avec une démarche reproductible.
\end{itemize}
}



\bpar{
Finally, from the epistemological point of view, we can also find ``practical'' implications that will naturally be more implicit in our approach, but not less structuring:
}{
Enfin, sur le point épistémologique, on peut également tirer des implications ``pratiques'' qui seront bien évidemment plus implicites dans notre démarche, mais pas moins structurantes :
}


\bpar{
\begin{itemize}
	\item Our inspiration will essentially be interdisciplinary and will aim at combining different points of view.
	\item Different knowledge domains (notion that we will precise in~\ref{sec:knowledgeframework}, but that we can understand for now in the sense of theoretical, modeling and empirical domains introduced by~\cite{livet2010}) can not be dissociated for any approach of scientific production, and we will use them in a strongly dependent way.
	\item Our approach will have to imply a certain level of reflexivity.
	\item The construction of a complex knowledge (\cite{morin1991methode}) is neither inductive nor deductive, but constructive in the idea of a morphogenesis of knowledge: it can be for example difficult to clearly identify precise ``scientific deadlocks'' since this metaphor assumes that an already constructed problem has to be unlocked, and even to constrain notions, concepts, objects or models in strict analytical frameworks, by categorizing them following a fixed classification, whereas the issue is to understand if the construction of categories is relevant. Doing it a posteriori is similar to a negation of the circularity and recursivity of knowledge production. The elaboration of ways to report that translate the diachronic character and the evolutive properties of it is an open problem.
\end{itemize}
}{
\begin{itemize}
	\item Notre inspiration sera essentiellement interdisciplinaire et cherchera à croiser les différents points de vue.
	\item Les différents domaines de connaissance (notion que nous préciserons en~\ref{sec:knowledgeframework}, mais qu'on peut comprendre pour l'instant au sens des domaines théorique, empirique et de la modélisation introduits par~\cite{livet2010}) sont indissociables pour toute démarche de production scientifique, et nous les mobiliserons de manière fortement dépendante.
	\item Notre démarche devra comprendre un certain niveau de réflexivité.
	\item La construction d'une connaissance complexe (\cite{morin1991methode}) est ni inductive ni déductive, mais constructive dans l'idée d'une morphogenèse de la connaissance : il peut par exemple être délicat d'identifier clairement des ``verrous scientifiques'' précis puisque cette métaphore suppose qu'il faut débloquer un problème déjà construit, et de même de faire rentrer notions, concepts, objet ou modèles dans des cadres analytiques stricts, en les catégorisant selon une classification fixe, alors que l'enjeu est de comprendre si la construction des catégories est pertinente. Le faire a posteriori relève d'une négation de la circularité et de la récursivité de la production de connaissance. L'élaboration de modes de compte-rendu rendant compte du caractère diachronique et des propriétés évolutives de celle-ci est un problème ouvert.
\end{itemize}
}





\stars












