


%----------------------------------------------------------------------------------------


\newpage



\section{Epistemological Positioning}{Positionnement Epistémologique}\label{sec:epistemology}

%----------------------------------------------------------------------------------------


La dernière section de ce chapitre vise à clarifier notre positionnement épistémologique, celui-ci ayant déjà été ébauché à plusieurs occasions précédemment. Un tel positionnement n'est jamais anodin, puisqu'il conditionne fortement les démarches, les expériences et l'interprétation des résultats : comme le souligne~\cite{morin1980methode}, un positionnement qui se dit objectif en rejetant toute subjectivité est bien plus biaisé qu'une approche subjective consciente.

Les points que nous souhaitons développer se placent dans une logique à la fois verticale de niveau d'abstraction et dans une logique de domaines scientifique : dans l'ordre, nous posons d'abord le contexte épistémologique général (relevant de l'histoire des sciences, à un niveau d'abstraction moyen), pour descendre en généralité pour préciser conceptuellement nos objets particuliers (épistémologie du vivant et du social), pour finalement tout remettre en perspective au niveau de la production de connaissance elle-même (épistémologie de la complexité).




%%%%%%%%%%%%%%%%%%%%%%
\subsection{Cognitive Approach and Perspectivism}{Approche cognitive et Perspectivisme}



\bpar{
Our epistemological positioning relies on a cognitive approach to science, given by Giere in~\cite{giere2010explaining}. The approach focuses on the role of cognitive agents as carriers and producers of knowledge. It has been shown to be operational by \cite{giere2010agent} that studies an agent-based model of science. These ideas converge with Chavalarias' Nobel Game~\cite{chavalarias2016s} that tests through a stylized model the balance between exploration and falsification in the collective scientific enterprise.
}{
Notre positionnement épistémologique se fonde sur une approche cognitive de la science, introduite par \noun{Giere} dans~\cite{giere2010explaining}. L'approche se concentre sur le rôle des agents cognitifs comme porteurs et producteurs de la connaissance. Son caractère opérationnel a été montré par \cite{giere2010agent} qui étudie un modèle basé-agent de la science. Ces idées convergent avec le jeu Nobel de \noun{Chavalarias}~\cite{chavalarias2016s} qui teste de manière stylisée l'équilibre entre production de nouvelles théories et tentative de falsification de théories existantes dans l'entreprise scientifique collective.
}




\bpar{
This epistemological positioning has been presented by \noun{Giere} as \emph{scientific perspectivism}~\cite{giere2010scientific}, which main feature is to consider any scientific entreprise as a \emph{perspective} in which \emph{agents} use \emph{media} (models) to represent something with a certain purpose. To make it more concrete, we can position it within Hacking's ``check-list'' of constructivism~\cite{hacking1999social}, a practical tool to position an epistemological position within a simplified three dimensional space which dimensions are different aspects on which realist approaches and constructivist approach generally diverge: first the contingency (path-dependency of the knowledge construction process) is necessary in the pluralist perspectivist approach, secondly the ``degree of constructivism'' is quite high because agents produce knowledge, and finally the stability of theories depends on the complex interaction between the agents and their perspectives. It was presented for these reasons as an intermediate and alternative way between absolute realism and skeptical constructivism~\cite{brown2009models}. The \emph{perspective} plays a central role in the framework.

}{
Ce positionnement épistémologique a été présenté par \noun{Giere} comme \emph{perspectivisme scientifique}~\cite{giere2010scientific}, dont la caractéristique principale est de considérer toute entreprise scientifique comme une \emph{perspective} dans laquelle des \emph{agents} utilisent des \emph{media} (modèles) pour représenter quelque chose dans un certain but. Pour comprendre ses principes de manière plus concrète, nous pouvons le positionner sur la \emph{check-list} du constructivisme de \noun{Hacking}~\cite{hacking1999social}, un outil pratique pour situer une position épistémologique. Celle-ci suppose un espace simplifié tri-dimensionnel dans lequel les dimensions sont différents aspects sur lesquels les approches réalistes et constructivistes généralement divergent. Le premier point est le niveau de contingence (dépendance au chemin du processus de construction de connaissances) : celle-ci est nécessaire dans l'approche perspectiviste qui est pluraliste et suppose des chemins parallèles de construction de connaissance. Le deuxième point mesure un ``degré de constructivisme'', qui est assez haut en perspectivisme car les agents produisent la connaissance. Enfin, le dernier point qui concerne l'explication endogène ou exogène de la stabilité des théories, est fortement du côté du constructivisme, puisque cette stabilité dépend des interactions complexes entre les agents et leur perspectives et donc totalement endogène. Le perspectivisme a pour ces raisons été présenté comme un chemin intermédiaire et alternatif entre le réalisme absolu et le constructivisme sceptique~\cite{brown2009models}. la notion de \emph{perspective} jouera pour nous un rôle fondamental dans le cadre développé en~\ref{sec:knowledgeframework}.
}


Cette approche mettant l'emphase sur l'auto-organisation, nous la voyons totalement compatible avec une vision anarchiste de la science comme défendue par~\cite{feyerabend1993against}. Celui-ci émet des doutes sur l'intérêt de l'anarchisme politique mais introduit l'\emph{anarchisme scientifique}, qu'il ne faut pas comprendre comme un refus total de toute méthode ``objective'', mais d'une autorité et légitimité artificielle que certaines méthodes ou courants scientifique pourraient vouloir prendre. Il démontre par une analyse précise des travaux de Galilée que la plupart de ses résultats étaient basés sur des croyances et que la plupart n'étaient pas accessibles avec les outils et méthodes de l'époque, et postule qu'il devrait en être de même pour certains travaux contemporains. Il n'y a donc pas de \emph{perspective} objectivement plus légitimes que d'autres dans la mesure de leurs validation par des faits et des pairs - et même dans ces cas la légitimité doit pouvoir être discutée, car la remise en question est un fondement de la connaissance. Cela correspond exactement à la pluralité des perspectives que nous défendons.


Supposer auto-organisation et l'émergence des connaissances peut être interprété comme une priorité donnée à la construction des paradigmes \emph{par le bas} (\emph{bottom-up}), en tentant de se distancer des préconceptions ou dogmes cadrant par le haut. En d'autres termes, il s'agit de pratiquer l'anarchisme scientifique prôné par \noun{Feyerabend}. En effet, les positions anarchistes ont trouvé un écho très cohérent dans les différents courants de la complexité, de la cybernétique à l'auto-organisation au cours du 20ème siècle~\cite{duda2013cybernetics}. Notre cadre de connaissances développé en~\ref{sec:knowledgeframework} illustre cette émergence de la connaissance. De plus, notre volonté de réflexivité et de donner à notre travail des pistes de lecture diverses au delà de la linéarité (voir Appendice~\ref{app:reflexivity}), illustre l'application de ces principes. Les recommandations méthodologiques et les positionnements donnés précédemment dans ce chapitre pourraient sonner comme totalitaires s'ils étaient assénés de manière sèche sans contexte, mais ceux-ci sont en fait tout le contraire puisqu'ils découlent d'un dynamique récente de science ouverte qui a bien émergé par le bas, conséquence en partie de l'ouverture et de la pluralité.

% Concept of ``Deconstructivism'' ? does not seen to exist.






%%%%%%%%%%%%%%%%%%%%%%
\subsection{From Life to Culture}{De la Vie à la Culture}

\subsubsection{Biological systems and social systems}{Systèmes biologiques et systèmes sociaux}

Le parallèle entre les systèmes sociaux et les systèmes biologiques est souvent fait, parfois de manière plus qu'imagée comme par exemple pour la théorie du \emph{Scaling} de \noun{West} qui applique des équations de croissance similaires à partir des lois d'échelle, avec des conclusions inverses tout de même concernant la relation entre taille et rythme de vie~\cite{bettencourt2007growth}. Les relations d'échelle ne tiennent plus lorsqu'on essaye de les appliquer à une fourmi seule, et il faut alors l'appliquer à la fourmilière entière qui est alors l'organisme en question. En ajoutant la propriété de cognition, on confirme qu'il s'agit du niveau pertinent, puisque celle-ci possède des propriétés cognitives avancées, comme la résolution de problèmes d'optimisation spatiaux, ou la réponse rapide à une perturbation extérieure. Les organisations sociales humaines, les villes, peuvent-elles être vues comme des organismes ? \cite{banos2013pour} file la métaphore de la \emph{fourmilière urbaine} mais rappelle que le parallèle s'arrête assez vite. Nous allons voir cependant dans quelle mesure certains concepts de l'épistémologie de la biologie peuvent être utiles pour comprendre les systèmes sociaux que nous nous proposons d'étudier.


Nous nous basons sur la contribution fondamentale de \noun{Monod} dans~\cite{monod1970hasard}, qui tente de développer les principes épistémologiques cruciaux pour l'étude du vivant. Ainsi, les organismes vivants répondent à trois propriétés essentielles qui permettent des les différencier d'autres systèmes : (i) la téléonomie, c'est à dire qu'il s'agit ``d'objets doués d'un projet'', projet qui se reflète dans leur structure et dans celles des artefacts qu'ils produisent\footnote{Qu'il ne faut pas confondre avec la téléologie, propres aux animismes, qui consiste à prêter un projet ou un sens à l'univers.} ; (ii) l'importance des processus morphogénétiques dans leur constitution (voir~\ref{sec:interdiscmorphogenesis}) ; (iii) la propriété de reproduction invariante de l'information définissant leur structure. \noun{Monod} esquisse de plus en conclusion des pistes pour une théorie de l'évolution culturelle. La téléonomie est essentielle dans les structures sociales, puisque toute organisation essaye de satisfaire un ensemble d'objectifs, même si en général elle n'y parviendra pas et que ceux-ci co-évolueront avec l'organisation. Cette notion de multi-objectif qui est typique des systèmes complexes socio-techniques, et y sera plus cruciale que pour les systèmes biologiques.

Ensuite, nous postulons que la notion de morphogenèse est un outil essentiel pour comprendre ces systèmes, avec une définition très proche de celle utilisée en biologie. Un travail approfondi pour donner cette définition est fait en~\ref{sec:interdiscmorphogenesis}, que nous résumerons en l'existence de processus relativement autonomes guidant la croissance du système et impliquant des relations causales circulaires entre forme et fonction qui témoignent d'une architecture émergente. Pour des systèmes sociaux, isoler le système est plus difficile et la notion de frontière sera moins stricte que pour un système biologique, mais on retrouvera bien ce lien entre forme et fonction, comme par exemple la structure d'une organisation ayant un impact sur ses fonctionnalités.


Enfin, la reproduction de l'information est au coeur de l'évolution culturelle, par la transmission de la culture et la \emph{mémétique}, la différence étant que le rapport d'échelle de temps entre la fréquence de transmission et les processus de croisement et de mutation ou d'autres processus non mémétiques de production culturelle est très faible, alors qu'elle est de plusieurs ordres de magnitude en biologie.


Un exemple illustre que le parallèle n'est pas toujours absurde :\cite{2017arXiv170305917G} propose un modèle de réseau auto-catalytique pour la cognition, qui expliquerait l'apparition de l'évolution culturelle par des processus analogues à ceux s'étant produit à l'apparition de la vie, c'est à dire une transition permettant au molécules de s'auto-entretenir et s'auto-reproduire, les représentations mentales faisant office de molécules.


Mais si les processus à l'origine sont analogues, la nature de l'évolution est bien différente par la suite, comme le montre \cite{vanderLeeuw2009}, les critères darwiniens d'évolution n'étant pas suffisant pour expliquer l'évolution de nos sociétés organisées. Il s'agit d'une complexité de nature différente dans laquelle le rôle des flux d'information est crucial (voir le rôle de la complexité informationnelle dans la sous-section suivante). 


L'un des points sur lequel il s'agit également d'être attentif est la plus grande difficulté de définir les niveaux d'émergence pour les systèmes sociaux : \cite{roth2009reconstruction} souligne le risque de tomber dans des cul-de-sac ontologiques car les niveaux ont été mal définis. Il soutient qu'il faut d'une manière générale penser au-delà de la seule dichotomie micro-macro qui est utilisée pour caricaturer les notions d'émergence faible, mais que les ontologies doivent souvent être multi-niveaux et impliquant de multiples niveaux intermédiaires.


Cette dernière question est aussi à mettre en perspective avec le problème de l'existence d'émergence forte dans les structures sociales, qui en terme sociologiques correspond à l'idée de l'existence ``d'êtres collectifs''~\cite{angeletti2015etres}. \noun{Morin} distingue d'ailleurs les systèmes vivants du second type (multi-cellulaire) et du troisième types (structures sociales), mais précise que les \emph{sujets} de ces derniers sont nécessairement inachevés~\cite{morin1980methode} (p.~852). Ainsi, les émergences du biologique au social sont analogues mais restent fondamentalement différentes.



\subsubsection{Co-evolution}{Co-évolution}


Ce positionnement sur les systèmes biologiques et sociaux trouve un écho immédiat pour le concept de co-évolution. Il provient en effet de la biologie, où il a été développé à la suite de celui d'évolution, pour être utilisé plus récemment en sciences humaines et sociales. Dans quelle mesure le concept a-t-il été transféré ? Retrouve-t-on un parallèle similaire à celui entre évolution biologique et évolution culturelle ? Nous proposons pour répondre à ces questions d'apporter un bref point de vue multidisciplinaire sur la co-évolution\footnote{La démarche ici est légèrement différente de celle que nous menerons en~\ref{sec:interdiscmorphogenesis} dans le cas de la Morphogenèse, qui sera \emph{interdisciplinaire} au sens où elle cherchera à intégrer les approches, tandis que nous restons ici dans un aperçu des concepts et donc plutôt dans du \emph{multidisciplinaire} (voir~\ref{app:sec:cybergeo} pour des précision sur le $\ast$-disciplinaire). Le concept de \emph{co-évolution} étant clé pour notre travail empirique par la suite, nous en donnerons alors une caractérisation originale et prenons le parti de ne pas tomber dans le syncrétisme intégrateur pour ce concept, mais bien de l'approcher d'un \emph{point de vue géographique}, et même plus précisément dans le cadre des systèmes territoriaux. On pourrait postuler une congruence entre la spécialisation empirique/de modélisation et celle théorique, plaçant notre processus de production de connaissance dans un profil particulier de dynamiques de Domaines de Connaissance (\ref{sec:knowledgeframework}).}. Nous passons par la suite en revue un large spectre de disciplines, partant de la biologie où le concept a initialement trouvé son origine pour arriver progressivement à des disciplines en relation avec les sciences du territoire.




\subsubsection{Biology}{Biologie}

Le concept de co-évolution en biologie est une extension de celui bien connu d'\emph{évolution}, qui remonte à \noun{Darwin}. \cite{durham1991coevolution} (p.~22) rappelle les composantes et structure systémiques nécessaires pour qu'il y ait évolution\footnote{Et dans ce contexte général l'évolution n'est pas réservée à la biologie du vivant et la présence de gènes, mais aussi à des systèmes physiques vérifiant ces conditions. Nous y reviendrons plus loin.} :

\begin{enumerate}
	\item Processus de \emph{transmission}, impliquant des unités de transmission et des mécanismes de transmission.
	\item Processus de \emph{transformation}, nécessitant des sources de variation.
	\item Isolation de sous-systèmes pour que les effets des processus précédents soient observable dans des différentiations.
\end{enumerate}

Ainsi, une population soumise à des contraintes (souvent synthétisée conceptuellement comme une \emph{fitness}) qui conditionnent la transmission du patrimoine génétique des individus (transmission), et à des mutation génétiques aléatoires (transformation), sera bien en évolution dans les territoires spatiaux qu'elle occupe (isolation), et par extension l'espèce à laquelle on peut l'associer.


La co-évolution est alors définie comme un changement évolutionnaire dans une caractéristique des individus d'une population, en réponse à un changement dans une deuxième population qui à son tour répond évolutionnairement au changement de la première, comme synthétisé par~\cite{janzen1980coevolution}. Cet auteur appuie par ailleurs la subtilité du concept et alerte contre ses utilisations injustifiées : la présence d'une congruence de deux caractéristiques qui semblent adaptées l'une à l'autre n'implique pas l'existence d'une co-évolution, l'une des deux espèces ayant pu s'adapter seule à une caractéristique déjà présente de l'autre.

Cette présentation brute de décoffrage mutile dans une certaine mesure la complexité réelle des écosystèmes : les populations s'insèrent dans des réseaux trophiques et des environnements, et les interactions co-évolutionnaires impliqueraient des communautés de populations d'espèces diverses, comme présenté par \cite{strauss2005toward} sous l'appellation de co-évolution diffuse. De même, les dynamiques spatio-temporelles sont cruciales dans la réalisation de ces processus : \cite{dybdahl1996geography} étudie par exemple l'influence de la distribution spatiale sur les motifs de co-évolution pour un escargot et son parasite, et montre qu'une vitesse de diffusion génétique dans l'espace plus grande pour le parasite conduit les dynamiques de co-évolution.

Les concepts essentiels à retenir du point de vue biologique sont ainsi : (i) existence de processus d'évolution, en particulier transmission et transformation ; (ii) dans des schémas circulaires entre populations dans le cas de la co-évolution ; et (iii) dans un cadre territorial (spatio-temporel et environnemental au sens du reste de l'éco-système) complexe.


\subsubsection{Cultural evolution}{Evolution culturelle}


Ce développement sur la co-évolution nous a été amené par le parallèle entre systèmes biologiques et systèmes sociaux. L'évolution de la culture est théorisée est explorée par un champ propre, et n'est pas en reste de dynamiques co-évolutives. \cite{Mesoudi25072017} rappelle l'état des connaissances sur le sujet et les défis à venir, comme la relation avec la nature cumulative de la culture, l'influence de la démographie dans les processus d'évolution, ou la construction de méthodes phylogénétiques permettant de reconstruire des arbres des branchements passés.

Pour donner un exemple, \cite{carrignon2015modelling} introduit un cadre conceptuel pour la co-évolution de la culture et du commerce dans le cas de sociétés anciennes sur lesquelles on dispose de données archéologique, et propose son implémentation par un modèle basé-agent dont les dynamiques sont partiellement validées par l'étude des faits stylisés produits par le modèle. La co-évolution est bien prise ici au sens d'adaptation mutuelle de structures socio-spatiales, à des échelles de temps comparables, dans ce cadre plus général d'évolution culturelle.


L'évolution culturelle serait même indissociable de l'évolution génétique, puisque \cite{durham1991coevolution} postule et illustre un lien fort entre les deux, qui seraient eux-même en co-évolution. \cite{bull2000meme} explore un modèle stylisé impliquant deux populations de répliquants (les gènes et les memes) et montre l'existence de transitions de phase pour les résultats du processus d'évolution génétique lorsque l'interaction avec le répliquant culturel est forte.


%\subsubsection{Artificial Life}{Artificial Life}
% -> in cultural evolution

\subsubsection{Sociology}{Sociologie}

Le concept a été utilisé en sociologie et disciplines apparentées comme les études de l'organisation, suivant le parallèle effectué ci-dessus de la même manière que pour l'évolution culturelle. Dans le domaine de l'étude des organisations, \cite{volberda2003co} développe un cadre conceptuel de la co-évolution inter-organisationnelle en relations avec les processus de management internes, mais déplore l'absence d'études empiriques cherchant à quantifier cette co-évolution. Dans le cadre de la gestion des systèmes de production, \cite{tolio2010species} conceptualise un chaine de production intelligente où produit, processus et système de production doivent être en co-évolution.


\subsubsection{Economic geography}{Economie géographique}


En Economie Géographique, le concept de co-évolution a également largement été mobilisée. L'idée d'entités évolutionnaires en économie vient à contre-courant du courant néoclassique qui reste majoritaire, mais trouve un écho de plus en plus pertinent~\cite{nelson2009evolutionary}. \cite{schamp201020} procède à une analyse épistémologique de l'utilisation de la co-évolution, et oppose une approche néo-Schumpeterienne de l'Economie qui considère l'émergence de populations qui évoluent à partir de règles micro-économiques (qui correspondrait à une lecture directe et relativement isolationniste de l'évolution biologique) à une approche systémique qui considérerait l'économie comme un système évolutif de manière globale (qui correspondrait à l'évolution diffuse que nous avons développé précédemment), pour proposer une caractérisation précise tombant dans le premier cas, qui suppose des \emph{institutions} qui co-évoluent. Le plus important pour notre propos est qu'il souligne l'aspect crucial du choix des population et des entités considérées, de la zone géographique, et appuie l'importance de l'existence de relations causale circulaires.

Il est possible de donner divers exemples d'application. \cite{doi:10.1080/00343400802662658} introduit un cadre conceptuel pour permettre de concilier nature évolutionnaire des firmes, théorie des clusters et réseaux de connaissance, dans lequel la co-évolution entre réseaux et firmes est centrale, et qui est définie comme une causalité circulaire entre différentes caractéristiques de ces sous-systèmes. \cite{colletis2010co} introduit un cadre de co-évolution des territoires et de la technologie (questionnant par exemple le rôle de la proximité pour les innovations), qui révèle l'importance à nouveau de l'aspect institutionnel. \cite{ter2011co} propose un cadre couplant la vision évolutionnaire des entreprises, la littérature sur les industries et l'innovation dans les clusters, et l'approche par réseau complexe des connexions entre ces premiers dans le système territorial.

En Economie Environnementale, \cite{kallis2007coevolution} montre que des approches ``larges'' (pouvant considérer la majorité des co-dynamiques comme co-évolutives) s'opposent à des approches plus strictes (dans l'esprit de la définition donnée par \cite{schamp201020}), et que dans tous les cas une définition précise, pas forcément venant de la biologie, doit être donnée, en particulier pour la recherche d'une caractérisation empirique.




\subsubsection{Geography}{Géographie}

Pour la géographie, comme nous l'avons déjà présenté en introduction, les travaux les plus proches empiriquement et théoriquement des notions de co-évolution sont étroitement liés à la Théorie Evolutive des Villes. Il n'est pas évident de tracer dans la littérature à quel moment la notion a été clairement formalisée, mais il est évident qu'elle était présente dès les fondements de la théorie comme le rappelle \noun{Denise Pumain} (voir~\ref{app:sec:interviews}) : le système complexe adaptatif est composé de sous-systèmes en interdépendances complexes, souvent circulairement causales. Les premiers modèles incluent bien cette vision de manière implicite, mais la co-évolution n'est pas appuyée explicitement ou définie précisément, en termes qui seraient quantifiables ou identifiables structurellement. \cite{paulus2004coevolution} a amené des preuves empiriques de mécanismes de co-évolution par l'étude de l'évolution des profils économiques des villes françaises. L'interprétation utilisée par~\cite{schmitt2014modelisation} repose sur une entrée par la Théorie Evolutive, et consiste fondamentalement en une lecture des systèmes de villes comme entités fortement interdépendantes.


\subsubsection{Physical Geography}{Géographie Physique}

En étude des paysages, \cite{sheeren2015coevolution} parle de co-évolution du paysage et des activités agricoles, mais ne désigne en fait pas d'effet circulaires de l'un sur l'autre. A priori, leurs résultats montrent que l'évolution des pratiques agricoles entraine une évolution du paysage, et il n'est ainsi pas clair dans quelle mesure le cadre conceptuel de la co-évolution, mentionné sans plus de détails, est mobilisé.


\subsubsection{Physics}{Physique}

Enfin, on peut noter de manière anecdotique que le terme de co-évolution a également été utilisé par la physique. L'utilisation pour des systèmes physiques peut porter à débat, selon que l'on suppose ou non que la transmission suppose un transmission d'\emph{information}\footnote{L'information est définie dans la théorie Shanonienne comme une probabilité d'occurrence d'une chaîne de caractère. \cite{morin1976methode} montre que le concept d'information est en fait bien plus complexe, et qu'il doit être pensé conjointement à un contexte donné de génération d'un système auto-organisateur néguentropique, i.e. réalisant des diminutions locales d'entropie notamment grâce à cette information. Ce type de système est nécessairement vivant. Nous prendrons ici cette vision complexe de l'information.}. Dans le cas d'une transmission ontologique uniquement physique (\emph{êtres physiques}), alors une grande partie des systèmes physiques sont évolutifs. \cite{hopkins2008cosmological} développe un cadre cosmologique pour la co-évolution d'objets cosmiques hétérogènes dont la présence et les dynamiques sont difficilement expliquées par des théories plus classiques (certains types de galaxies, quasars, trous noirs supermassifs). \cite{antonioni2017coevolution} étudie la co-évolution entre des propriétés de synchronisation et de coopération au sein d'un réseau d'oscillateurs de Kuramoto\footnote{Le modèle de Kuramoto s'intéresse à la synchronisation au sein de systèmes complexes, en étudiant l'évolution de phases $\theta_i$ couplée par les équations d'interaction $\dot{\vec{\theta}} = \vec{\omega} + \vec{W}\left[\vec{\theta}\right] + \mathbf{B}$ où $\vec{\omega}$ sont les phases propres de forçage et la force de couplage entre $i$ et $j$ est donnée par $\vec{W}_{i} = \sum_j w_{ij} \sin\left(\theta_i - \theta_j\right)$ et $\vec{B}$ du bruit.}, montrant d'une part que le concept peut être appliqué à des objets abstraits, et d'autre part qu'un réseau de relations complexes entre variables peut être à l'origine de dynamiques présentant des causalités circulaires, c'est à dire d'une co-évolution en ce sens.


\subsubsection{Synthesis}{Synthèse}


La plupart de ces approches rentrent dans la théorie des systèmes complexes adaptatifs développée par \noun{Holland}, notamment dans~\cite{holland2012signals} : il voit tout système comme une imbrication de systèmes de limites, filtrant des signaux ou des objets. Au sein d'une limite donnée, le sous-système correspondant est relativement autonome de l'extérieur, est est appelé \emph{niche écologique}, en correspondance directe avec les communautés fortement connectées au sein des réseaux trophiques ou écologiques. Ainsi, des entités interdépendantes au sein d'une niche sont dites en co-évolution. Nous reviendrons sur cette entrée lors de la construction théorique en~\ref{sec:theory} lorsque nous aurons développé d'autres concepts qui lui sont nécessaire.


Nous retenons de cet aperçu multidisciplinaire de la co-évolution les points fondamentaux suivants précurseurs à une définition propre de la co-évolution que nous donnerons plus loin, en conclusion de la première partie :

\begin{enumerate}
	\item La présence de \emph{processus d'évolution} est primaire, et leur définition se ramène presque toujours à l'existence de processus de transmission et de transformation.
	\item La co-évolution suppose des entités ou systèmes, appartenant à des classes distinctes, dont les dynamiques évolutives sont couplées de manière circulaire causale. Les approches peuvent différer selon l'hypothèse de populations de ces entités, d'objets singuliers, ou de composantes d'un système global alors en interdépendance mutuelle sans qu'il y ait circularité directe. % rq : dit-on qu'il y a coevol dans les cas spurieux
	\item La délimitation des systèmes ou des sous-systèmes, à la fois dans l'espace ontologique (définition des objets étudiés), mais aussi dans l'espace et le temps, ainsi que leur distribution dans ces espaces, est fondamental pour l'existence de dynamiques co-évolutives, et a priori dans un grand nombre de cas, pour leur caractérisation empirique.
\end{enumerate}




%----------------------------------------------------------------------------------------

\subsection{Nature of Complexity and Knowledge Production}{Nature de la Complexité et Production de Connaissances}


Les deux premiers points épistémologiques que nous venons de traiter relevaient respectivement du positionnement en lui-même, c'est à dire du cadre de lecture des processus de production de connaissance scientifique, puis de la nature des concepts considérés. Nous proposons de monter encore en généralité par rapport au premier et d'introduire un développement contribuant modestement (c'est à dire dans notre contexte) à \emph{la Connaissance de la Connaissance}. Il s'agit d'interroger les liens entre complexité et processus de production de connaissance.



Un aspect de la production de connaissance sur des Systèmes Complexes, auquel nous nous heurtons plusieurs fois ici (voir chapitre~\ref{ch:theory}), et qui semble être récurrent voire inévitable, est un certain niveau de réflexivité (et qui serait inhérent aux systèmes complexes en comparaison aux systèmes simples, comme nous le développerons plus loin). Nous entendons par là à la fois une réflexivité pratique, c'est à dire la nécessité d'élever le niveau d'abstraction, comme le besoin de reconstruire de manière endogène les disciplines dans lesquelles une réflexion cherche à se positionner comme proposé en \ref{sec:quantepistemo}, ou de réfléchir à la nature épistémologique de la modélisation lors de l'élaboration d'un modèle comme en \ref{app:sec:csframework}, mais également une réflexivité théorique en le sens que les appareils théoriques ou les concepts produits peuvent s'appliquer de manière récursive à eux-mêmes. Cette constatation pratique fait echo à des débats épistémologiques anciens questionnant la possibilité d'une connaissance objective de l'univers qui serait indépendante de notre structure cognitive, ou bien la nécessité d'une ``rationalité évolutive'' impliquant que notre système cognitif, produit de l'évolution, reflète les processus complexes ayant conduit à son émergence, et que toute structure de connaissance sera par conséquent réflexive\footnote{Nous remercions D. Pumain d'avoir pointé cette vue alternative du problème que nous allons développer par la suite}. Nous ne prétendons pas ici apporter une réponse à une question aussi vaste et vague telle quelle, mais proposons un lien potentiel entre cette réflexivité et la nature de la complexité.


\subsubsection{Complexity and Complexities}{Complexité et Complexités}

Ce qui est entendu par complexité d'un système mène souvent à des malentendus car celle-ci peut être qualifiée selon différentes dimensions et visions. Nous distinguons dans un premier temps la complexité au sens d'émergence faible et d'autonomie entre les différents niveaux d'un système, et sur laquelle différentes positions peuvent être développées comme dans \cite{deffuant2015visions}. Nous ne rentrerons pas dans une granularité plus fine, la vision de la complexité sociale donnant encore plus de fil à retordre au démon de Laplace, peut être par exemple comprise par une émergence plus forte (au sens d'émergence faible et forte développée précedemment en~\ref{sec:computation}). Nous simplifions ainsi et supposons que la nature des systèmes joue un rôle secondaire dans notre reflexion, et considérons la complexité au sens d'une émergence.


D'autre part, nous distinguons deux autres ``types'' de complexité, la complexité computationnelle et la complexité informationnelle, qui peuvent être vues comme des mesures de complexité, mais qui ne sont pas directement équivalentes à l'émergence, puisqu'il n'existe pas de lien systématique entre les trois. On peut par exemple imaginer utiliser un modèle de simulation, pour lequel les interactions entre agents élémentaires se traduisent par un message codé au niveau supérieur: il est alors possible en exploitant les degré de liberté de minimiser la quantité d'information contenue dans le message. Les différentes langues demandent des efforts cognitifs différents et compressent différemment l'information, ayant différents niveau de complexité mesurables~\cite{febres2013complexity}. De même, des artefacts architecturaux sont le résultat d'un processus d'évolution naturelle puis culturelle et peuvent témoigner plus ou moins de cette trajectoire.

De nombreuses autres caractérisations conceptuelles ou opérationnelles de la complexité existent, et il est clair que la communauté scientifique n'a pas convergé sur une définition unique~\cite{chu2008criteria}\footnote{Dans une approche en un sens réflexive, \cite{chu2008criteria} propose de continuer d'explorer les différentes approches existantes, comme des proxys de la complexité dans le cas d'un essentialisme, ou comme des notions à part entière. La complexité devrait émerger d'elle même de l'interaction entre ces différentes approches étudiant la complexité, d'où la réflexivité.}. Nous proposons de nous concentrer sur ces trois concepts en particulier, pour lesquels les relations ne sont déjà pas évidentes.


En effet, les liens entre ces trois types de complexité ne sont pas systématiques, et dépendent du type de système. Des liens épistémologiques peuvent néanmoins être introduits. Nous traitons ceux entre émergence et les deux autre complexités, étant donné que le lien entre complexité computationnelle et complexité informationnelle est assez bien compris et relève de problématiques de compression de l'information et de traitement du signal, ou encore de cryptographie.


\subsubsection{Computational Complexity}{Complexité computationnelle et émergence}



Différents indices suggèrent une certaine nécessité de complexité computationnelle pour avoir émergence dans des systèmes complexes, tandis que réciproquement un certain nombre de systèmes complexes adaptatifs sont dotés de capacités de calcul élevées.


Un premier lien où complexité computationnelle implique émergence est suggéré par un examen algorithmique des problèmes fondamentaux de la Physique Quantique. En effet, \cite{2014arXiv1403.7686B} démontre que la résolution de l'équation de Schrödinger avec Hamiltonien quelconque est un problème NP-difficile et NP-complet, et donc que l'acceptation de $\mathbf{P}\neq\mathbf{NP}$ implique une séparation qualitative entre le niveau quantique microscopique et le niveau d'observation macroscopique. Ainsi, c'est bien la complexité (ici au sens de leur calcul) des interactions au sein du système et de son environnement qui explique l'apparente réduction du paquet d'onde, ce qui rejoint l'approche de \noun{Gell-Mann} par la décohérence quantique~\cite{gell1996quantum}, qui explique que des probabilités ne peuvent être associées qu'aux histoires décohérentes (dans lesquelles les correlations ont fait prendre une trajectoire au système à l'échelle macroscopique)\footnote{Le \emph{Problème de la Mesure Quantique} se pose lorsqu'on considère une fonction d'onde microscopique donnant l'état d'un système pouvant être superposition de plusieurs états, et consiste en un paradoxe théorique, les mesures étant toujours déterministes alors que le système a des probabilité d'états d'une part, et le problème de la non-existence d'états macroscopiques superposés (réduction du paquet d'onde). Comme revu par~\cite{schlosshauer2005decoherence}, différentes interprétations épistémologiques de la physique quantiques sont liées à différentes explications de ce paradoxe, dont celle ``classique'' de Copenhague qui donne à l'acte d'observation le rôle de reduction du paquet d'onde. \noun{Gell-Mann} précise que cette interprétation n'est pas absurde puisque c'est bien les correlations entre l'objet quantique et le monde qui produisent l'histoire décohérente, mais qu'elle est bien trop spécifique, et que la réduction a lieu dans l'émergence elle-même : le chat est bien mort ou vivant, mais pas les deux, avant que l'on ouvre la boîte.}. Le paradoxe du chat de Schrödinger nous apparait ainsi comme une perspective fondamentalement réductionniste, puisqu'il suppose que la superposition d'états peut se propager à travers les niveaux successifs et qu'il n'y aurait pas émergence, au sens de constitution d'un niveau supérieur autonome. En d'autres termes, le travail de \cite{2014arXiv1403.7686B} suggère que la complexité computationnelle est suffisante pour la présence d'émergence.\footnote{A priori, cette séparation effective des échelles n'implique pas que le niveau inférieur ne joue pas un rôle crucial, puisque \cite{vattay2015quantum} prouve que les propriétés de criticalité quantiques sont typiques des molécules du vivant, sans qu'il n'y ait a priori de spécificité pour la vie dans cette détermination complexe par les échelles inférieures : \cite{2016arXiv161102269V} a introduit une nouvelle approche liant théories quantiques et relativité générale dans laquelle il est montré que la gravité est un phénomène émergent et que la dépendance au chemin dans la déformation de l'espace de base introduit un terme supplémentaire au niveau macroscopique, qui permet d'expliquer les déviations attribuées jusqu'alors à la \emph{matière noire}.}

%  bizarre : epistemo de la QM pas bien developpée ? (que physiciens cloisonnés ? du coup philo de comptoir ? et conflits avec informaticiens ? check si Moore en parle)


Dans le sens inverse, le lien entre complexité computationnelle et émergence est mis en valeur par les questions liées à la nature de la computation~\cite{moore2011nature}. Des automates cellulaires, qui sont par ailleurs cruciaux pour la compréhension de divers systèmes complexes, ont été montrés Turing-complets\footnote{Un système est Turing-complet s'il est capable de calculer les mêmes fonctions qu'une machine de Turing, communément accepté comme l'ensemble du ``calculable'' (thèse de \noun{Church}). Pour mémoire, une machine de Turing est un automate fini à bande d'écriture infinie~\cite{moore2011nature}.}, comme le Jeu de la Vie~\cite{beer2004autopoiesis}\footnote{Il existe même un langage de programmation permettant de programmer en \emph{Game of Life}, disponible à \url{https://github.com/QuestForTetris}. Sa genèse trouve son origine dans un défi posté sur \emph{codegolf} ayant pour but la conception d'un Tetris, et a abouti à un projet collaboratif extrêmement avancé.}. Des organismes sans système nerveux central sont capables de résoudre des problèmes décisionnels difficiles~\cite{reid2016decision}. Un algorithme à base de fourmis est montré par~\cite{Pintea2017} comme résolvant un Problème du Voyageur de Commerce Généralisé (GTSP), problème NP-difficile. Ce lien fondamental avait déjà été envisagé par \noun{Turing}, puisqu'au delà de ses contributions fondamentales à l'informatique moderne, il s'était intéressé à la morphogenèse et a tenté de produire des modèles chimiques d'explication de celle-ci~\cite{turing1952chemical} (qui étaient très loin de effectivement l'expliquer - elle n'est toujours pas bien comprise aujourd'hui, voir~\ref{sec:interdiscmorphogenesis} - mais dont les contributions conceptuelles ont été fondamentales, notamment pour la notion de réaction-diffusion). On sait par ailleurs qu'un minimum de complexité en termes d'interactions constituantes dans un cas particulier de système basé-agent (modèles de réseaux booléens), et donc d'émergences possibles, implique une borne inférieure sur la complexité computationnelle, qui devient conséquente dès que les interactions avec l'environnement sont ajoutées~\cite{tovsic2017boolean}.


%\comment{\cite{2017arXiv170404231E} quantum computation reduces drastically memory needed}


\subsubsection{Informational Complexity and Emergence}{Complexité informationnelle et émergence}

La complexité informationnelle, ou la quantité d'information contenue dans un système et la manière dont celle-ci est stockée, entretient également des liens fondamentaux avec l'émergence. L'information est équivalente à l'entropie d'un système et donc à son degré d'organisation - c'est ce qui a permis de résoudre le paradoxe apparent du Démon de Maxwell qui serait capable de diminuer l'entropie d'un système isolé et donc contredire la deuxième loi de la thermodynamique : celui-ci utilise en fait l'information sur les positions et vitesses des molécules du système, et son action compense la perte d'entropie par sa captation d'information\footnote{Le démon de Maxwell est plus qu'une construction intellectuelle : \cite{cottet2017observing} implémente un démon expérimentalement au niveau quantique.}.

Cette notion d'accroissement local de l'entropie a été étudiée largement par \noun{Chua} sous la forme du \emph{Local Activity Principle}, qui est introduit comme un troisième principe de la thermodynamique, permettant d'expliquer par des arguments mathématiques l'auto-organisation pour une certaine classe de systèmes complexes typiquement impliquant des équations de réaction-diffusion~\cite{mainzer2013local}.


La manière dont l'information est stockée et compressée est essentielle pour la vie, puisque l'ADN est bien un système de stockage d'information, dont le rôle à différents niveaux bien loin d'être compris complètement. La complexité culturelle témoigne également d'un stockage de l'information à différents niveaux, par exemple au sein des individus mais aussi des artefacts et des institutions, et des flux d'information relevant nécessairement des deux autres types de complexité. Les flux d'information sont essentiels pour l'auto-organisation dans un système multi-agent. Les comportements collectifs de poissons ou d'oiseau sont des exemples typiques utilisés pour illustrer l'émergence et font partie des cas d'école de systèmes complexes. On commence cependant seulement à comprendre comment ces flux structurent le système, et quels sont les motifs spatiaux de transfert d'information au sein d'un \emph{flock} par exemple : \cite{crosato2017informative} introduit des premiers résultats empiriques avec l'entropie de transfert pour des poissons et pose les bases méthodologiques de ce type d'étude.




\subsubsection{Knowledge production}{Production de connaissances}

Nous avons à présent la matière suffisante pour en venir à la réflexivité. Il est possible de positionner la production de connaissances à l'intersection des interactions entre types de complexité développées ci-dessus. Tout d'abord, la connaissance telle que nous l'envisageons ne peut se passer d'une construction collective, et implique donc un encodage et une transmission de l'information : il s'agit à un autre niveau de toutes les problématiques liées à la communication scientifique. La production de connaissances nécessite donc cette première interaction entre complexité computationnelle et complexité informationnelle. Le lien entre complexité informationnelle et émergence est mobilisé si on considère l'établissement de connaissances comme un processus morphogénétique. Il est montré en~\ref{sec:interdiscmorphogenesis} que le lien entre forme et fonction est fondamental en psychologie : nous pouvons l'interpréter comme un lien entre information et sens, puisque la sémantique d'un objet cognitif ne peut se passer d'une fonction. \noun{Hofstader} rappelle dans~\cite{hofstadter1980godel} l'importance des symboles à différents niveaux pour l'émergence d'une pensée, qui consistent à un niveau intermédiaire en des signaux. Enfin, la dernière relation entre complexité computationnelle et émergence est celle qui nous permet d'affirmer qu'on s'intéresse particulièrement à une production de connaissance sur des systèmes complexes, les deux premiers pouvant s'appliquer à tout type de connaissance.

Ainsi, toute \emph{connaissance du complexe} embrasse non seulement toutes les complexités et leur relations dans son contenu, mais aussi dans sa nature comme nous venons de montrer. La structure de la connaissance en termes de complexité est analogue à la structure des systèmes qu'elle étudie. Nous postulons que cette correspondance structurelle implique une certaine récursivité, et donc un certain niveau de \emph{réflexivité} (au sens de connaissance d'elle-même et de ses propres conditions). 

%Comme ces systèmes sont généralement multi-niveaux, ou présentent au moins un certain niveau de complexité computationnelle, la connaissance de ceux-ci se doit de la capturer, puisque même des modèles \emph{simples} devront capturer leur complexité de manière conceptuelle et impliquer une structure conceptuelle sous-jacente complexe, même si celle-ci n'est pas explicitement explorée. %anticipation sur la requisite complexity ?


On peut tenter d'étendre à la réflexivité en tant que réflexion sur le positionnement disciplinaire : suivant \noun{Pumain} dans~\cite{pumain2005cumulativite}, la complexité d'une approche est également liée à la diversité des points de vue nécessaire pour la construire. Pour atteindre ce nouveau type de complexité\footnote{Pour laquelle des liens avec les types précédents apparaissent naturellement : par exemple, \cite{gell1995quark} considère la complexité effective comme le \emph{Contenu d'Information Algorithmique} (proche de la complexité de Kolmogorov) d'un Système Complexe Adaptatif \emph{observant un autre} Système Complexe Adaptatif, ce qui donne son importance aux complexités informationnelle et computationnelle et suggère l'importance du point de vue d'observation, et par extension de la combinaison de ceux-ci - ce qui est par ailleurs à mettre en relation avec l'approche perspectiviste des sciences complexes présentée précédemment.}, qui serait une dimension supplémentaire liée à la connaissance des systèmes complexes, la réflexivité doit être au coeur de la démarche. \cite{read2009innovation} rappelle que l'innovation a été rendue possible quand les sociétés ont été capables de produire et diffuser de l'information sur leur propre structure, c'est à dire quand elles ont pu atteindre un certain niveau de réflexivité. La \emph{connaissance du complexe} serait donc le produit et le support de sa propre évolution grâce à la réflexivité qui a joué un rôle fondamental dans l'évolution du système cognitif : on pourrait ainsi suggérer de rassembler ces considérations, comme proposé par \noun{Pumain}, sous une nouvelle notion épistémologique de \emph{Rationalité Evolutive}.


Pour conclure, notons qu'étant donné la loi de la \emph{requisite complexity}, proposée par \cite{gershenson2015requisite} comme extension de la \emph{requisite variety}~\cite{ashby1991requisite}\footnote{L'un des principes cruciaux de la cybernétique, la \emph{requisite variety} postule que pour contrôler un système ayant un certain nombre d'états, le contrôleur doit avoir au moins autant d'états. \noun{Gershenson} propose une extension conceptuelle à la complexité, qui peut être justifiée par exemple par \cite{allen2017multiscale} qui introduit la \emph{requisite variety} multi-échelle, démontrant la compatibilité avec une théorie de la complexité basée sur la théorie de l'information.}, la \emph{connaissance du complexe} devra nécessairement être \emph{connaissance complexe}. Cet autre point de vue renforce la nécessité de la réflexivité, puisque suivant \noun{Morin} (voir par exemple \cite{morin1991methode} sur la production de connaissance), la \emph{Connaissance de la Connaissance} est centrale dans l'établissement d'une pensée complexe.




%\comment[AB]{personnellement je ne suis pas très fan de cette idée de « connaissance complexe » ou même de « pensée complexe » (ok je suis un mauvais morinien !). }[(JR) mal formule peu etre, ``connaissance du complexe'' $\rightarrow$ cf dernière phrase : serait equivalent selon le point de vue du controle ; rejoint Morin]

% Remarques : 
% portugali semantic information : chaud à introduire.
% check paper Valentina ``Practical Reflexivity''
% lire Morin sur la pensée complexe





%%%%%%%%%%%%%%%%%%%%%%
\subsubsection*{Practical implications}{Conséquences pratiques}


Pour conclure cette section épistémologique, nous proposons de synthétiser l'ensemble des idées introduites sous forme de manifestations concrètes en découlant directement, et qui conditionneront fortement l'ensemble de la forme et de la sémantique de la connaissance introduite par la suite. Ces directions (que nous n'irons pas jusqu'à nommer principes car seulement à l'état d'ébauche) peuvent être regroupées en trois grandes familles : pratiques de modélisation, pratique de la Science Ouverte, et épistémologie. Sur le plan des pratiques de modélisation, dans chaque section se dégagent différents axes plus ou moins complémentaires :

\begin{itemize}
	\item La modélisation, qui sera dans la majorité des cas équivalente à la simulation, doit être comprise comme un instruments de connaissance indirect sur des processus au sein d'un système complexe ou sur la structure de celui-ci (d'après la sous-section sur ``pourquoi modéliser''), et les modèles devront nécessairement être complexes (d'après la réflexion sur les différents types de complexité) au sens qu'il capturent un phénomène d'émergence faible, tout en respectant des exigences de parcimonie.
	\item L'exploration des modèles est partie intégrante de l'entreprise de modélisation (voir reproductibilité), et le calcul intensif est un élément clé pour explorer efficacement les modèles de simulation (voir calcul intensif). Les méthodes d'analyse de sensibilité doivent être questionnées et étendues si besoin (comme l'illustre l'exemple de la sensibilité à l'espace).
	\item Comme suggéré par le positionnement perspectiviste, le couplage de modèles devra jouer un rôle crucial dans la capture de la complexité.
\end{itemize}

Pour la Science Ouverte, on peut extraire les points suivants :

\begin{itemize}
	\item La nécessité de l'ensemble des démarches liées à la Science Ouverte pour parvenir à la construction de modèles toujours plus complexes, vers la co-construction de modèles par différentes disciplines.
	\item Dans ce cadre, l'ouverture complète du code source, ainsi que sa lisibilité sont cruciaux. L'explicitation complète du modèle dans le compte-rendu scientifique, ainsi qu'une documentation du code auto-suffisante, sont deux aspects de celle-ci.
	\item La question des données ouvertes n'est pas négociable dans ce cadre. La quasi-totalité de nos traitements est basée sur des données initialement ouverte, et lorsque ce n'est pas le cas nous travaillons à un niveau agrégé auquel on peut fournir les données. Les jeux de données construits sont ouverts.
	\item Concernant les méthodes d'exploration interactive, qui sont un pendant de l'ouverture de la Science, nous en développons un certain nombre, mais restons limités par rapport au pré-requis idéal qui devrait rendre celles-ci totalement compatibles avec une démarche reproductible.
\end{itemize}

Enfin, sur le point épistémologique, on peut également tirer des implications ``pratiques'' qui seront bien évidemment plus implicites dans notre démarche, mais pas moins structurantes :

\begin{itemize}
	\item Notre inspiration sera essentiellement interdisciplinaire et cherchera à croiser les différents points de vue.
	\item Les différents domaines de connaissance (notion que nous préciserons en~\ref{sec:knowledgeframework}, mais qu'on peut comprendre pour l'instant au sens des domaines théorique, empirique et de la modélisation introduits par~\cite{livet2010}) sont indissociables pour toute démarche de production scientifique, et nous les mobiliserons de manière fortement dépendante.
	\item Notre démarche devra comprendre un certain niveau de réflexivité.
	\item La construction d'une connaissance complexe (\cite{morin1991methode}) est ni inductive ni déductive, mais constructive dans l'idée d'une morphogenèse de la connaissance : il peut par exemple être délicat d'identifier clairement des ``verrous scientifiques'' précis puisque cette métaphore suppose qu'il faut débloquer un problème déjà construit, et de même de faire rentrer notions, concepts, objet ou modèles dans des cadres analytiques stricts, en les catégorisant selon une classification fixe, alors que l'enjeu est de comprendre si la construction des catégories est pertinente. Le faire a posteriori relève d'une négation de la circularité et de la récursivité de la production de connaissance. L'élaboration de modes de compte-rendus rendant compte du caractère diachronique et des propriétés évolutives de celle-ci est un problème ouvert.
\end{itemize}






\stars












