


%-------------------------


\section[From Paris to Zhuhai][De Paris à Zhuhai]{From Paris to Zhuhai}{De Paris à Zhuhai}

\label{sec:casestudies}

%-------------------------




% collection concrete d'interactions land-use/transportation ; cas particulier du Grand Paris et du Pearl River Delta


%%%%%%%%%%%%%%%%%%%%%%%%
\subsection[Greater Paris][Grand Paris]{Greater Paris: History and current Issues}{Le Grand Paris : histoire et enjeux}


La région parisienne est une bonne illustration de la complexité des interactions entre réseaux de transports et territoires, au cours du temps et à l'échelle intermédiaire d'une région métropolitaine globalement mono-centrique.

\cite{gilli2009paris} propose en 2009 un diagnostic de la situation institutionelle de la région parisienne, et des pistes pour une approche couplée entre gouvernance et aménagement. La préfiguration de ``l'instauration d'un acteur collectif métropolitain'' correspond à la métropole du Grand Paris qui sera inaugurée 7 ans plus tard








%%%%%%%%%%%%%%%%%%%%%%%%
\subsection[Pearl River Delta][Le Delta de la Rivière des Perles]{Pearl River Delta: new urban regimes and mega-city regions}{Le Delta de la Rivière des Perles : nouveaux régimes urbains et Mega-City Regions}


\todo{some ``comparable'' maps would be useful : ask Chenyi most precise data on PRD : territorial variables and transportation networks ?}

Parler du pont et des bifurcations induites (cf intro chap 5)

Si la notion de megalopolis peut être tracée jusqu'à \noun{Gottmann}~\cite{gottmann1964megalopolis}, et qu'elle est à l'origine de celle de Mega-city Region consacrée par \noun{Hall}~\cite{hall2006polycentric}, il est clair que cette dernière est toujours plus d'actualité avec l'apparition récente de nouveaux régimes, notamment par l'urbanisation croissante dans des pays à forte croissance et en mutation très rapide comme la Chine~\cite{swerts2015megacities}.







%%%%%%%%%%%%%%%%%%%%%%%%
\subsection{Comparability of case studies}{Comparabilité des études de cas}

% déjà introduire l'ergodicité, non stationarité dans le temps et l'espace.











