


%-------------------------

\newpage

\section[From Paris to Zhuhai][De Paris à Zhuhai]{From Paris to Zhuhai}{De Paris à Zhuhai}

\label{sec:casestudies}

%-------------------------


Nous approfondissons dans cette section des cas d'étude géographique à l'échelle métropolitaine, que nous choisissons très différents pour montrer la diversité des situations possibles mais aussi les motifs récurrents généraux qui pourraient se dégager. Il s'agit de la métropole du Grand Paris, et de la mega-région urbaine du Delta de la Rivière des Perles dans le sud de la Chine.


%-------------------------

%%%%%%%%%%%%%%%%%%%%%%%%
\subsection[Greater Paris][Grand Paris]{Greater Paris: History and current Issues}{Le Grand Paris : histoire et enjeux}


La région parisienne est une bonne illustration de la complexité des interactions entre réseaux de transports et territoires, au cours du temps et à l'échelle intermédiaire d'une région métropolitaine globalement monocentrique. \cite{gilli2005bassin} rappelle l'importance de l'hinterland du Bassin Parisien et l'importance de ne pas considérer l'hypercentre de manière isolée.




\paragraph{Grand Paris Transportation Network}{Réseau de Transport du Grand Paris}

le développement du réseau de transport du Grand Paris : in\cite{beauguitte:halshs-01068589}

\paragraph{Governance}{Gouvernance}

\cite{gilli2009paris} propose en 2009 un diagnostic de la situation institutionnelle de la région parisienne, et des pistes pour une approche couplée entre gouvernance et aménagement. La préfiguration de ``l'instauration d'un acteur collectif métropolitain'' correspond à la métropole du Grand Paris qui sera inaugurée 7 ans plus tard, puisque le conseil métropolitain est mis en place fin 2016. Son effectivité concrète reste quasi-nulle au moment de l'écriture, confirmant une certaines inertie des structures de gouvernance, qui a nécessairement un impact sur celle des réseaux de transport. 


\cite{Padeiro2012}, \cite{PADEIRO201344}









%-------------------------


%%%%%%%%%%%%%%%%%%%%%%%%
\subsection[Pearl River Delta][Le Delta de la Rivière des Perles]{Pearl River Delta: new urban regimes and mega-city regions}{Le Delta de la Rivière des Perles : nouveaux régimes urbains et Mega-City Regions}


\comment{Some ``comparable'' maps would be useful : ask Chenyi most precise data on PRD : territorial variables and transportation networks ?}

Parler du pont et des bifurcations induites (cf intro chap 5)

Si la notion de megalopolis peut être tracée jusqu'à \noun{Gottmann}~\cite{gottmann1964megalopolis}, et qu'elle est à l'origine de celle de Mega-city Region consacrée par \noun{Hall}~\cite{hall2006polycentric}, il est clair que cette dernière est toujours plus d'actualité avec l'apparition récente de nouveaux régimes, notamment par l'urbanisation croissante dans des pays à forte croissance et en mutation très rapide comme la Chine~\cite{swerts2015megacities}.







%%%%%%%%%%%%%%%%%%%%%%%%
\subsection{Comparability of case studies}{Comparabilité des études de cas}

% déjà introduire l'ergodicité, non stationarité dans le temps et l'espace.

La possibilité de transfert des modèles urbains est délicate, et la particularité Est-asiatique a déjà été montrée pour la structure économique, et comment celle-ci ne peut être interprétée de manière simple par une séparation des processus microscopiques et macroscopiques comme certaines lectures rapides et idéologiquement orientée ont pu le faire, comme la vision de la Banque Mondiale~\cite{amsden1994isn}.










\stars




