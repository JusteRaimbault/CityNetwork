


%-------------------------

\newpage

\section[From Paris to Zhuhai][De Paris à Zhuhai]{From Paris to Zhuhai}{De Paris à Zhuhai}

\label{sec:casestudies}

%-------------------------


Nous approfondissons dans cette section des cas d'étude géographique à l'échelle métropolitaine, que nous choisissons très différents pour montrer la diversité des situations possibles mais aussi les motifs récurrents généraux qui pourraient se dégager. Il s'agit de la métropole du Grand Paris, et de la mega-région urbaine du Delta de la Rivière des Perles dans le sud de la Chine.


%-------------------------

%%%%%%%%%%%%%%%%%%%%%%%%
\subsection[Greater Paris][Grand Paris]{Greater Paris: History and current Issues}{Le Grand Paris : histoire et enjeux}


La région parisienne est une bonne illustration de la complexité des interactions entre réseaux de transports et territoires, au cours du temps et à l'échelle intermédiaire d'une région métropolitaine globalement monocentrique. \cite{gilli2005bassin} rappelle l'importance de l'hinterland du Bassin Parisien et l'importance de ne pas considérer l'hypercentre de manière isolée. Si la moyenne couronne possède un certain niveau de polycentricité, notamment grâce à l'effet des villes nouvelles qui sont d'importants pôles d'emplois locaux~\cite{berroir2005contribution} qui même s'il a rapidement divergé des intentions planificatrices~\cite{es119}, est bien réel, le Bassin Parisien étendu peut être également lu ayant un certains nombre de centres importants à une heure de Paris : Chartres, Orléans, Rouen, Reims et Lille grâce à la grande vitesse.


\cite{Padeiro2012}, \cite{PADEIRO201344}


\paragraph{Governance}{Gouvernance}

\cite{gilli2009paris} propose en 2009 un diagnostic de la situation institutionnelle de la région parisienne, et des pistes pour une approche couplée entre gouvernance et aménagement. La préfiguration de ``l'instauration d'un acteur collectif métropolitain'' correspond à la métropole du Grand Paris qui sera inaugurée 7 ans plus tard, puisque le conseil métropolitain est mis en place fin 2016. Son effectivité concrète reste quasi-nulle au moment de l'écriture, confirmant une certaines inertie des structures de gouvernance, qui a nécessairement un impact sur celle des réseaux de transport. La mise en place de ce nouveau niveau de gouvernance a été disséquée plus récemment toujours par \noun{Gilli} dans~\cite{gilli2014gouverner}, où il la situe dans un contexte plus large socio-économique et urbain, en quelque sorte un diagnostic territorial qui explique certains aspects de ce besoin de mutation. En perte de vitesse sur le plan de l'aménagement, mais aussi sur le plan social au vu d'inégalités socio-économiques locales très fortes, la métropole a besoin de se réinventer, et se nouveau souffle se cristallise naturellement dans le Grand Paris, c'est à dire ``l'avenir de Paris est sa banlieue''. Cette initiative se concrétise par la convergence d'une auto-organisation des élus locaux, et d'une redéfinition du rôle de l'état, voulue centralisatrice jusqu'en 2012 puis laissant la place libre à la gouvernance métropolitaine avec le nouveau gouvernement, même si les projets lancés et les financements restent les mêmes dans les grandes lignes : le projet du Grand Paris Express est un compromis entre la solution voulue par l'état et celle poussée par la région.




\paragraph{Grand Paris Transportation Network}{Réseau de Transport du Grand Paris}

L'histoire du développement du réseau de transport de la métropole francilienne est rappelée dans~\cite{beauguitte:halshs-01068589}. La particularité centralisatrice française a conduit à une structure particulière du réseau ferré à l'échelle nationale, mais aussi à celle régionale. La domination de Paris a en effet fortement marqué la structuration du réseau de transport au cours des différentes périodes historiques où il a subit des évolutions conséquentes. Avant 1975, la distribution de l'accessibilité est clairement centralisée et le centre de Paris fortement congestionné.





\paragraph{Impact of the Grand Paris Express}{Impact du Grand Paris Express}

Les impacts immédiats d'une nouvelle de transport en terme d'accessibilité concernent généralement des territoires bien plus larges que les zones où la ligne et ses stations sont implantées : les motifs d'accessibilité sont dus aux propriétés topologiques du réseau et celles-ci sont fortement discontinues en fonction de la structure du graphe. Illustrons le cas des lignes du Grand Paris Express et de leur impact direct sur l'accessibilité régionale. 







%-------------------------


%%%%%%%%%%%%%%%%%%%%%%%%
\subsection[Pearl River Delta][Le Delta de la Rivière des Perles]{Pearl River Delta: new urban regimes and mega-city regions}{Le Delta de la Rivière des Perles : nouveaux régimes urbains et Mega-City Regions}



\bpar{}{
Si la notion de megalopolis peut être tracée jusqu'à \noun{Gottmann}~\cite{gottmann1964megalopolis}, et qu'elle est à l'origine de celle de \emph{Mega-city Region} (MCR) consacrée par \noun{Hall}~\cite{hall2006polycentric}, il est clair que cette dernière est toujours plus d'actualité avec l'apparition récente de nouveaux régimes, notamment par l'urbanisation accélérée dans des pays à forte croissance économique et en mutation très rapide comme la Chine~\cite{swerts2015megacities}. Le second cas que nous développons ici rentre dans cette catégorie : le Delta de la Rivière des Perles est une des illustrations classiques de la structure d'une MCR fortement polycentrique. Historiquement initialement composé de Guangzhou uniquement, le développement de Hong-Kong puis la mise en place Zones Economiques Spéciales (\cn{经济特区}) dans le cadre des politiques d'ouverture de \noun{Deng Xioaping}, a conduit à un développement extrêmement rapide de Shenzhen, et dans une moindre mesure de Zhuhai. La province du Guangdong dans lequel le PRD se situe intégralement a actuellement le plus fort PIB régional de Chine, et la MCR regroupe une population d'environ 60 millions (les estimations fluctuant fortement selon la définition prise de la MCR et la prise en compte de la population flottante). Le phénomène de migration des campagnes est très présent dans la région et une ville comme Dongguan a par exemple basé son économie sur des manufactures employant ces travailleurs migrants.
}


\bpar{}{
\cite{Ye2014200} analyse les actions de gouvernance métropolitaine à l'échelle de centres de la MCR, et plus particulièrement comment les communes de Guangzhou et Foshan ont progressivement accru leur coopération pour former une zone métropolitaine intégrée, pouvant ainsi fortement influencer le développement des transports par exemple et permettant la mise en place d'un réseau connecté. Une forte tension entre des processus émergents par le bas, et un dirigisme d'état relativement fort en Chine, se répercutant de l'Etat central, au gouvernement provincial jusqu'aux gouvernements locaux, a permis la mise en place d'une telle structure. La compétition avec les autre villes de la MCR reste très forte, et la logique d'intégration de la MCR est partiellement guidée par la région seulement. La nature particulière des ZES de Shenzhen et Zhuhai, liée aux relations privilégiées avec les Zones Administratives Spéciales de Hong-Kong et Macao, qui n'ont été réintégrées à la République Populaire qu'à la fin du millénaire et conservent un certain niveau d'indépendance en termes de gouvernance, complique encore les jeux d'acteurs au sein de la région.
\cite{} % gaudin liao liao
Il n'existe pas d'autorité d'organisation des transports au niveau de la MCR, et chaque commune gère indépendamment le réseau local, tandis que les connections entre villes sont assurées par le réseau de train national.
}


\paragraph{Impact of the Zhuhai-Hong-Kong-Macao bridge}{Impact du Pont Zhuhai-Hong-Kong-Macao}

Un projet iconique d'infrastructure de transport dans la région est le pont fermant l'embouchure du Delta, reliant Zhuhai et Macao à Hong-Kong. En réalité un Pont-tunnel, celui-ci fait une cinquantaine de kilomètres, en faisant le plus long du monde. L'ouverture au traffic a été retardée de plusieurs années et est prévue pour fin 2017. Le changement de motifs d'accessibilité peut potentiellement induire de fortes bifurcations dans les trajectoires des villes.





%-------------------------


%%%%%%%%%%%%%%%%%%%%%%%%
\subsection{Comparability of case studies}{Comparabilité des études de cas}


La possibilité de transfert des modèles urbains est délicate, et la particularité Est-asiatique a déjà été montrée pour la structure économique, et comment celle-ci ne peut être interprétée de manière simple par une séparation des processus microscopiques et macroscopiques comme certaines lectures rapides et idéologiquement orientée ont pu le faire, comme la vision de la Banque Mondiale~\cite{amsden1994isn}. La comparabilité de systèmes urbains est une question ouverte au centre des enjeux de la Théorie Evolutive Urbaine, et est par exemple liée au caractère ergodique de ces systèmes : si la trajectoire d'une ville dans le temps capture l'ensemble des états urbains possibles, alors les différentes villes sont différentes manifestations du même processus stochastique à différentes périodes, et un ensemble de villes permettrait d'avoir une idée des trajectoires temporelles. Intuitivement ce n'est pas le cas, et la Théorie Evolutive postule en effet la non-ergodicité~\cite{pumain2012urban}, que nous étudierons plus en détail en~\ref{sec:staticcorrelations}.







\stars




