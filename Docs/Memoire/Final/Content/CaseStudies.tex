


%-------------------------

\newpage

\section[From Paris to Zhuhai][De Paris à Zhuhai]{From Paris to Zhuhai}{De Paris à Zhuhai}

\label{sec:casestudies}

%-------------------------


Nous approfondissons dans cette section des cas d'étude géographique à l'échelle métropolitaine, que nous choisissons très différents pour montrer la diversité des situations possibles mais aussi les motifs récurrents généraux qui pourraient se dégager. Il s'agit de la métropole du Grand Paris, et de la mega-région urbaine du Delta de la Rivière des Perles dans le sud de la Chine.


%-------------------------

%%%%%%%%%%%%%%%%%%%%%%%%
\subsection[Greater Paris][Grand Paris]{Greater Paris: History and current Issues}{Le Grand Paris : histoire et enjeux}


La région parisienne est une bonne illustration de la complexité des interactions entre réseaux de transports et territoires, au cours du temps et à l'échelle intermédiaire d'une région métropolitaine globalement monocentrique. \cite{gilli2005bassin} rappelle l'importance de l'hinterland du Bassin Parisien et l'importance de ne pas considérer l'hypercentre de manière isolée. Si la moyenne couronne possède un certain niveau de polycentricité, notamment grâce à l'effet des villes nouvelles qui sont d'importants pôles d'emplois locaux~\cite{berroir2005contribution} qui même s'il a rapidement divergé des intentions planificatrices~\cite{es119}, est bien réel, le Bassin Parisien étendu peut être également lu ayant un certains nombre de centres importants à une heure de Paris : Chartres, Orléans, Rouen, Reims et Lille grâce à la grande vitesse.


\cite{Padeiro2012}, \cite{PADEIRO201344}


\paragraph{Grand Paris Transportation Network}{Réseau de Transport du Grand Paris}

L'histoire du développement du réseau de transport de la métropole francilienne est rappelée dans~\cite{beauguitte:halshs-01068589}.

\paragraph{Governance}{Gouvernance}

\cite{gilli2009paris} propose en 2009 un diagnostic de la situation institutionnelle de la région parisienne, et des pistes pour une approche couplée entre gouvernance et aménagement. La préfiguration de ``l'instauration d'un acteur collectif métropolitain'' correspond à la métropole du Grand Paris qui sera inaugurée 7 ans plus tard, puisque le conseil métropolitain est mis en place fin 2016. Son effectivité concrète reste quasi-nulle au moment de l'écriture, confirmant une certaines inertie des structures de gouvernance, qui a nécessairement un impact sur celle des réseaux de transport. La mise en place de ce nouveau niveau de gouvernance a été disséquée plus récemment toujours par \noun{Gilli} dans~\cite{gilli2014gouverner}, où il la situe dans un contexte plus large socio-économique et urbain, en quelque sorte un diagnostic territorial qui explique certains aspects de ce besoin de mutation. En perte de vitesse sur le plan de l'aménagement, mais aussi sur le plan social au vu d'inégalités socio-économiques locales très fortes, la métropole a besoin de se réinventer, et se nouveau souffle se cristallise naturellement dans le Grand Paris, c'est à dire ``l'avenir de Paris est sa banlieue''. Cette initiative se concrétise par la convergence d'une auto-organisation des élus locaux, et d'une redéfinition du rôle de l'état, voulue centralisatrice jusqu'en 2012 puis laissant la place libre à la gouvernance métropolitaine avec le nouveau gouvernement, même si les projets lancés et les financements restent les mêmes dans les grandes lignes : le projet du Grand Paris Express est un compromis entre la solution voulue par l'état et celle poussée par la région.











%-------------------------


%%%%%%%%%%%%%%%%%%%%%%%%
\subsection[Pearl River Delta][Le Delta de la Rivière des Perles]{Pearl River Delta: new urban regimes and mega-city regions}{Le Delta de la Rivière des Perles : nouveaux régimes urbains et Mega-City Regions}



Si la notion de megalopolis peut être tracée jusqu'à \noun{Gottmann}~\cite{gottmann1964megalopolis}, et qu'elle est à l'origine de celle de Mega-city Region consacrée par \noun{Hall}~\cite{hall2006polycentric}, il est clair que cette dernière est toujours plus d'actualité avec l'apparition récente de nouveaux régimes, notamment par l'urbanisation croissante dans des pays à forte croissance et en mutation très rapide comme la Chine~\cite{swerts2015megacities}.


Parler du pont et des bifurcations induites (cf intro chap 5)





%-------------------------


%%%%%%%%%%%%%%%%%%%%%%%%
\subsection{Comparability of case studies}{Comparabilité des études de cas}


La possibilité de transfert des modèles urbains est délicate, et la particularité Est-asiatique a déjà été montrée pour la structure économique, et comment celle-ci ne peut être interprétée de manière simple par une séparation des processus microscopiques et macroscopiques comme certaines lectures rapides et idéologiquement orientée ont pu le faire, comme la vision de la Banque Mondiale~\cite{amsden1994isn}. La comparabilité de systèmes urbains est une question ouverte au centre des enjeux de la Théorie Evolutive Urbaine, et est par exemple liée au caractère ergodique de ces systèmes : si la trajectoire d'une ville dans le temps capture l'ensemble des états urbains possibles, alors les différentes villes sont différentes manifestations du même processus stochastique à différentes périodes, et un ensemble de villes permettrait d'avoir une idée des trajectoires temporelles. Intuitivement ce n'est pas le cas, et la Théorie Evolutive postule en effet la non-egodicité~\cite{pumain2012urban}, que nous étudierons plus en détail en~\ref{}







\stars




