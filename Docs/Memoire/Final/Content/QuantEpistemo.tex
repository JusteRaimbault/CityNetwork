

\newpage

%----------------------------------------------------------------------------------------

\section{An epistemological Approach}{Une Approche Epistémologique}

\label{sec:quantepistemo}


%----------------------------------------------------------------------------------------


\bpar{
A corollary of the thematic background introduced in chapter~\ref{ch:thematic} is the need of an understanding of involved disciplines themselves to be able to build integrated heterogeneous models. The potentialities of couplings and integrations are greatly determined by existing approaches and corresponding gaps. This implies an advanced epistemological study in each field, that we propose to tackle in a systematic and quantitative way. This deliberate choice may shadow elaborated epistemological considerations but fits our purpose of preliminary investigations for the construction of models, as it may reveal investigation directions.
}{
Un corolaire de la matière thématique introduite en chapitre~\ref{ch:thematic} est le besoin d'une compréhension des disciplines impliquées elles-même pour être en mesure de construire des modèles hétérogènes intégrés. Les possibilités de couplage et d'intégration sont hautement déterminées par les approches existantes et les lacunes correspondantes qui ont été exposées dans la section précédente~\ref{sec:modelingsa}. Cela implique une étude épistémologique avancée dans chaque champ, que nous proposons de mener de manière quantitative et systématique. Ce choix délibéré pourrait occulter des considérations épistémologiques élaborées mais suit notre objectif d'investigations préliminaires pour la construction de modèles, en révélant potentiellement des directions de recherche.
}


\bpar{
We describe and explore first a systematic review exploration algorithm, that retrieve corpuses of references through iterative semantic extraction. We describe then briefly possible extended bibliometrics by presenting an external example of application. We finally suggest possible development directions towards unsupervised data and text-mining.
}{
Nous décrivons et explorons d'abord un algorithme de revue systématique algorithmique, qui reconstruit des corpus de références par une extraction sémantique itérative. Nous procédons ensuite à une analyse de réseaux, couplant réseau de citation et réseau sémantique, pour préciser les contours des disciplines impliquées. Nous suggérons finalement des possibles extensions vers de l'apprentissage non-supervisé et la fouille de texte complets pour une extraction automatique de la structure de modèles par exemple.
}





%----------------------------------------------------------------------------------------



\subsection{Algorithmic Systematic Review}{Revue Systématique Algorithmique}


\bpar{
A broad bibliographical study suggests a scarcity of quantitative models of simulation integrating both network and urban growth. This absence may be due to diverging interests of concerned disciplines, resulting in a lack of communication.  We propose to proceed to an algorithmic systematic review to give quantitative elements of answer to this question. A formal iterative algorithm to retrieve corpuses of references from initial keywords, based on text-mining, is developed and implemented. We study its convergence properties and do a sensitivity analysis. We then apply it on queries representative of the specific question, for which results tend to confirm the assumption of disciplines compartmentalization.
}{
Une étude bibliographique étendue suggère une rareté des modèles quantitatifs de simulation qui intègrent à la fois la croissance urbaine et la croissance des réseaux. Cette absence pourrait être due aux intérêts divergents des disciplines concernées qui induiraient un manque de communication. Nous proposons de procéder à une revue de la littérature systématique et algorithmique pour donner des éléments de réponse quantitatifs à cette question. Un algorithme itératif formel pour construire des corpus de références à partir de mots-clés initiaux, basé sur l'analyse textuelle, est développé et mis en oeuvre. Nous étudions ses propriétés de convergence et procédons à une analyse de sensibilité. Nous l'appliquons ensuite à des requêtes représentatives de notre question spécifique, pour lesquelles les résultats tendent à confirmer l'hypothèse d'isolation des disciplines.
}



\subsubsection{In search of models of co-evolution}{En recherche de modèles de co-évolution}


\bpar{
Transportation networks and urban land-use are known to be strongly coupled components of urban systems at different scales~\cite{bretagnolle2009organization}. One common approach is to consider them as co-evolving, avoiding misleading interpretations such as the myth of structural effect of transportation infrastructures~\cite{offner1993effets}. A question rapidly arising is the existence of models endogeneizing this co-evolution, i.e. taking into account simultaneous urban and network growth. We try to answer it using an algorithmic systematic review. We propose in this section, after a brief state of the art of existing literature, to present such an approach by formalizing the algorithm, which results are then presented and discussed. 
}{
Les réseaux de transport et l'usage du sol urbain sont connus pour être des composantes fortement couplées des systèmes urbains à différentes échelles~\cite{bretagnolle2009organization}.\comment{(Florent) c'est une affirmation très forte : pourquoi fermer le débat à ce stade ?}
 Une approche commune est de les considérer comme étant en co-évolution, tout en évitant les interprétations trompeuses comme le mythe des effets structurants des infrastructures de transport~\cite{offner1993effets}. Une question qui se présente rapidement est l'existence de modèles endogénéisant cette co-évolution, i.e. prenant en compte simultanément la croissance urbaine et celle du réseau. Nous essayons d'y répondre par une revue systématique algorithmique. Nous proposons dans cette section, après un état de l'art rapide de la littérature existante, de développer cette approche en formalisant l'algorithme, dont les résultats sont ensuite présentés et discutés.
}


\subsubsection{Modeling Interactions between Urban Growth and Network Growth}{Modéliser les Interactions entre croissance urbaine et croissance des réseaux}

Nous avons revu selon divers point de vue les efforts de modélisation des interactions entre territoires et réseaux dans la section précédente~\ref{sec:modelingsa}.



\subsubsection{Bibliometric Analysis}{Analyse Bibliométrique}


\bpar{
Literature review is a crucial preliminary step for any scientific work and its quality and extent may have a dramatic impact on research quality. Systematic review techniques have been developed, from qualitative review to quantitative meta-analyses allowing to produce new results by combining existing studies \cite{rucker2012network}. Ignoring some references can even be considered as a scientific mistake in the context of emerging information systems~\cite{lissacksubliminal}. We aim to take advantage of such techniques to tackle our issue.
Indeed, observing the form of the bibliography obtained in previous section raises some hypothesis. It is clear that all components are present for co-evolutive models to exist but different concerns and objectives seem to stop it. As it was shown by \cite{commenges:tel-00923682} for the concept of mobility, for which a ``small world of actors'' relatively closed invented a notion ad hoc, using models without accurate knowledge of a more general scientific context, we could be in an analog case for the type of models we are interested in. Restricted interactions between scientific fields working on the same objects but with different purposes, backgrounds and at different scales, could be at the origin of the relative absence of co-evolving models. 
While most of studies in bibliometrics rely on citation networks \cite{2013arXiv1310.8220N} or co-autorship networks \cite{2014arXiv1402.7268S}, we propose to use a less explored paradigm based on text-mining introduced in~\cite{chavalarias2013phylomemetic}, that obtain a dynamic mapping of scientific disciplines based on their semantic content. For our question, it has a particular interest, as we want to understand content structure of researches on the subject. We propose to apply an algorithmic method described in the following. The algorithm proceeds by iterations to obtain a stabilized corpus from initial keywords, reconstructing scientific semantic landscape around a particular subject.
}{
Avec l'avènement des nouveaux moyens techniques et des nouvelles sources de données, la revue de littérature classique tend à se coupler à des revues automatiques. Des techniques de revue systématique ont été développées, des revues qualitatives aux meta-analyses quantitatives qui permettent de produire des nouveaux résultats par combinaison d'études existantes~\cite{rucker2012network}. Passer sous silence certaines références peut même être considéré comme une erreur scientifique dans le contexte de l'émergence des systèmes d'information qui par l'accès plus aisé à l'information rend difficilement justifiable l'omission de références clés~\cite{lissacksubliminal}. Nous proposons de tirer parti de telles techniques pour traiter notre problème.
En effet, l'observation de la bibliographie obtenue dans la section précédente soulève une hypothèse. On peut postuler sans risques à partir de la revue précédente~\ref{sec:modelingsa} Il semble clair que toutes les briques sont présentes pour l'existence de modèles co-évolutifs mais des questionnements et objectifs différents semblent la stopper. Comme montré par~\cite{commenges:tel-00923682} pour le concept de mobilité, pour lequel un ``petit monde d'acteurs'' relativement fermé, en l'occurence les corsards des Ponts, a inventé une notion ad hoc, utilisant des modèles sans connaissance préalable d'un contexte scientifique plus général. On pourrait se trouver dans un cas similaire pour le type de modèles auxquels on s'intéresse. Des interactions restreintes entre des champs scientifiques travaillant sur les mêmes objets mais avec des objectifs et contextes divergents, et à des échelles différentes, pourrait être à l'origine de l'absence de modèles co-évolutifs. Tandis que la majorité des études en bibliométrie se reposent sur les réseaux de citation~\cite{2013arXiv1310.8220N} ou les réseaux de co-auteurs~\cite{2014arXiv1402.7268S}, nous proposons d'utiliser un paradigme moins exploré, basé sur l'analyse textuelle, introduit par~\cite{chavalarias2013phylomemetic}, qui obtient une cartographie dynamique des disciplines scientifiques en se basant sur leur contenu sémantique. \comment{(Florent) tu ne dis pas clairement pourquoi ses méthodes classiques ne suffisent pas}
 La méthode est particulièrement adaptée pour notre étude puisque nous voulons comprendre la structure du contenu des recherches sur le sujet. Nous appliquons une approche algorithmique décrite par la suite. L'algorithme procède par itérations pour obtenir un corpus stabilisé à partir de mots-clés initiaux, reconstruisant l'horizon sémantique scientifique autour d'un sujet donné.
}


\paragraph{Description of the Algorithm}{Description de l'Algorithme}


\bpar{
Let $A$ be an alphabet, $A^{\ast}$ corresponding words and $T = \cup_{k\in \mathbb{N}} {A^{\ast}}^k$ texts of finite length on it. A reference is for the algorithm a record with text fields representing title, abstract and keywords. Set of references at iteration $n$ will be denoted $\mathcal{C} \subset T^3$. We assume the existence of a set of keywords $\mathcal{K}_n$, initial keywords being $\mathcal{K}_0$. An iteration goes as follows :

\begin{enumerate}
\item A raw intermediate corpus $\mathcal{R}_n$ is obtained through a catalog request providing previous keywords $\mathcal{K}_{n-1}$.
\item Overall corpus is actualized by $\mathcal{C}_n = \mathcal{C}_{n-1} \cup \mathcal{R}_n$.
\item New keywords $\mathcal{K}_n$ are extracted from corpus through Natural Language Processing treatment, given a parameter $N_k$ fixing the number of keywords.
\end{enumerate}

The algorithm stops when corpus size becomes stable or a user-defined maximal number of iterations has been reached. Fig.~\ref{fig:quantepistemo:algo} shows the global workflow.
}{
Soit $A$ un alphabet, \comment{(Florent) dans quelle théorie mathématique ?}[just a set of symbols]
 $A^{\ast}$ les mots correspondants et $T = \cup_{k\in \mathbb{N}} {A^{\ast}}^k$ les textes de longueur finie sur celui-ci. Ce qu'on nomme une référence est pour l'algorithme un enregistrement avec des champs textuels représentant le titre, le résumé et les mots-clés. L'ensemble de références à l'itération $n$ sera noté $\mathcal{C} \subset T^3$.\comment{(Florent) pourquoi ?}[car titre, résumé, mots clés - on aggrege dans tous les cas pour l'extraction] Nous supposons l'existence d'un ensemble de mots-clés $\mathcal{K}_n$, les mots-clés initiaux étant $\mathcal{K}_0$. \comment{(Florent) si j'ai bien compris, c'est l'utilisateur qui rentre $\mathcal{K}_0$, c'est commode non ?}[c'est plutôt dans l'esprit de cet algo là. on pourrait aussi partir d'un corpus ; mais ça c'est plutot la partie suivante avec le réseau de citations.]
 Une itération procède de la manière suivante :

\begin{enumerate}
\item Un corpus intermédiaire brut $\mathcal{R}_n$ est obtenu par une requête à un catalogue \comment{(Florent) notamment tu dois avoir énormément de bruit, comment le gères tu ?}
 auquel on fourni les mots-clés précédents $\mathcal{K}_{n-1}$.
\item Le corpus total est actualisé par $\mathcal{C}_n = \mathcal{C}_{n-1} \cup \mathcal{R}_n$. \comment{(Florent) pas clair que sont $\mathcal{C}_0$ et $\mathcal{K}_0$}
\item Les nouveaux mot-clés $\mathcal{K}_n$ sont extraits du corpus par Traitement du Language Naturel (NLP), étant donné un paramètre $N_k$ fixant le nombre de mot-clés. \comment{(Florent) variable à chaque pas ?}[non]
\end{enumerate}

L'algorithme termine quand la taille du corpus devient stable ou quand un nombre maximal d'itérations défini par l'utilisateur est atteint. La figure~\ref{fig:quantepistemo:algo} synthétise le processus général.
}


%%%%%%%%%%%%%%%%%%%%%%%%%%%%
\begin{figure}
\centering
\includegraphics[width=\textwidth]{Figures/QuantEpistemo/schema_algo}
\caption[Systematic review algorithm workflow][Architecture de l'algorithme de revue systématique]{Global workflow of the algorithm, including implementation details: catalog request is done through Mendeley API; final state of corpuses are RIS files.}{Architecture globale de l'algorithme, incluant des détails d'implémentation : la requête au catalogue est faite via l'API Mendeley ; les corpus finaux sont sous forme de fichiers RIS.}
\label{fig:quantepistemo:algo}
\end{figure}
%%%%%%%%%%%%%%%%%%%%%%%%%%%%



%%%%%%%%%%%%%%%%%%%%%%%%%%%%
\paragraph{Results}{Résultats}

%\paragraph{Implementation}{Implémentation}


\bpar{
Because of the heterogeneity of operations required by the algorithm (references organisation, catalog requests, text processing), it was found a reasonable choice to implement it in Java. Source code is available on the Github repository of the project\footnote{at \texttt{https://github.com/JusteRaimbault/CityNetwork/tree/master/Models/Biblio/AlgoSR}}. Catalog request, consisting in retrieving a set of references from a set of keywords, is done using the Mendeley software API \cite{mendeley} as it allows an open access to a large database. Keyword extraction is done by Natural Language Processing (NLP) techniques, following the workflow given in \cite{chavalarias2013phylomemetic}, calling a Python script that uses \cite{bird2006nltk}.
}{
De par l'hétérogénéité des opérations requises par l'algorithme (organisation des références, requêtes au catalogue, analyse textuelle), le language Java s'est présenté comme une alternative raisonnable. Le code source est disponible sur le dépôt ouvert du projet\comment{(Florent) lien ?}\footnote{à l'adresse \texttt{https://github.com/JusteRaimbault/CityNetwork/tree/master/Models/Biblio/AlgoSR}}. Les requêtes au catalogue, qui consistent à récupérer un ensemble de références à partir d'un ensemble de mots-clés, sont faites via l'API du logiciel Mendeley~\cite{mendeley} qui permet un accès ouvert à une base de données conséquente. L'extraction des mots-clés est effectuée par techniques d'Analyse Textuelle (NLP) selon le processus donné dans~\cite{chavalarias2013phylomemetic}, via un script Python qui utilise~\cite{bird2006nltk}.
}


%\paragraph{Convergence and Sensitivity Analysis}{Convergence et analyse de sensibilité}


\bpar{
A formal proof of algorithm convergence is not possible as it will depend on the empirical unknown structure of request results and keywords extraction. We need thus to study empirically its behavior. Good convergence properties but various sensitivities to $N_k$ were found as presented in Fig.~\ref{fig:quantepistemo:sensitivity}. We also studied the internal lexical consistence of final corpuses as a function of keywords number. As expected, small number yields more consistent corpuses, but the variability when increasing stays reasonable.
}{
Une preuve formelle de convergence de l'algorithme n'est guère envisageable puisque qu'elle dépendra de la structure empirique inconnue des résultats de requête et d'extraction de mots-clés. Il est donc nécessaire d'étudier le comportement de l'algorithme de manière empirique. Comme présenté en figure~\ref{fig:quantepistemo:sensitivity}, l'algorithme a de bonnes propriétés de convergence mais diverse sensibilités à $N_k$. Nous étudions également la cohérence lexicale interne des corpus finaux et fonction du nombre de mots-clés. Comme attendu, des valeurs faibles produisent des corpus plus cohérents, mais la variabilité lorsque qu'elles augmentent reste raisonnable.
}



%%%%%%%%%%%%%%%%%%%%%%%%%%%%
\begin{figure}
\centering
\includegraphics[width=\textwidth]{Figures/QuantEpistemo/explo}
\medskip
\includegraphics[width=0.8\textwidth]{Figures/QuantEpistemo/lexicalConsistence_MeanSd}
\caption[Convergence and sensitivity analysis of systematic review algorithm]{Convergence and sensitivity analysis. Left : Plots of number of references as a function of iteration, for various queries linked to our theme (see further), for various values of $N_k$ (from 2 to 30). We obtain a rapid convergence for most cases, around 10 iterations needed. Final number of references appears to be very sensitive to keyword number depending on queries, what seems logical since encountered landscape should strongly vary depending on terms. Right : Mean lexical consistence and standard error bars for various queries, as a function of keyword number. Lexical consistence is defined though co-occurrences of keywords by, with $N$ final number of keywords, $f$ final step, and $c(i)$ co-occurrences in references, $k = \frac{2}{N(N-1)}\cdot \sum_{i,j \in \mathcal{K}_f}{\left| c(i) - c(j) \right|}$. The stability confirms the consistence of final corpuses.}{\comment{(Florent) illisible}}
\label{fig:quantepistemo:sensitivity}
\end{figure}
%%%%%%%%%%%%%%%%%%%%%%%%%%%%




\bpar{
Once the algorithm is partially validated, we apply it to our question. We start from five different initial requests that were manually extracted from the various domains identified in the bibliography (that are ``city system network'', ``land use transport interaction'', ``network urban modeling'', ``population density transport'', ``transportation network urban growth''). We take the weakest assumption on parameter $N_k=100$, as it should less constrain reached domains. After having constructed corpuses, we study their lexical distances as an indicator to answer our initial question. Large distances would go in the direction of the assumption made above, i.e. that discipline self-centering may be at the origin of the lack of interest for co-evolutive models. We show in Table~\ref{tab:quantepistemo:lexical} values of relative lexical proximity, that appear to be significantly low, confirming this assumption.
}{
Lorsque l'algorithme a été partiellement validé, \comment{(Florent) avec quels mots clés as tu validé empiriquement la convergence de l'algo?}
 on peut l'appliquer à notre question. Nous partons de cinq différentes requêtes initiales qui ont été manuellement extraites des divers domaines identifiés dans la bibliographie (qui sont ``city system network'', ``land use transport interaction'', ``network urban modeling'', ``population density transport'', ``transportation network urban growth''). \comment{(Florent) pourquoi ce mots là (par ex pas coevolution ?}
 Nous prenons l'hypothèse la plus faible pour le paramètre $N_k=100$, \comment{(Florent) pourquoi est ce weak ?}
  au sens où les domaines atteints devraient être moins restreints. Après avoir construit les corpus, nous étudions leur cohérence lexicale comme un indicateur de réponse à notre question initiale. De grande distances devraient confirmer l'hypothèse formulée ci-dessus, i.e. que des disciplines auto-centrées pourraient être à l'origine d'un manque d'intérêt pour des modèles co-évolutifs. La table~\ref{tab:quantepistemo:lexical} montre les valeurs de la proximité lexicale relative, qui est significativement basse, \comment{(Florent) comment peut on en juger ?}[c'est un des flaws, on n'a pas de null model..]
   confirmant notre hypothèse.
}




%%%%%%%%%%%%%%%%%%%%%%%%%%%%
\begin{table}
\centering
\includegraphics[width=\textwidth]{Figures/QuantEpistemo/corpusesDistances}
\caption[Stationary lexical proximities][Proximités lexicales stationnaires]{Symmetric matrix of lexical proximities between final corpuses, defined as the sum of overall final keywords co-occurrences between corpuses, normalized by number of final keywords (100). We obtain very low values, confirming that corpuses are significantly far. Size of final corpuses is given as $W$.}{Matrice symétrique des proximités lexicales entre les corpus finaux, définies comme la somme des co-occurrences totale de mots-clés finaux entre corpus, normalisé par le nombre de mots-clés finaux (100). Les valeurs obtenues sont considérablement faibles, ce qui confirme que les corpus sont éloignés de manière significative. La taille des corpus finaux est donnée par $W$. \comment{(Arnaud) a developper}}
\label{tab:quantepistemo:lexical}
\end{table}
%%%%%%%%%%%%%%%%%%%%%%%%%%%%




\bpar{
Possible developments can include the construction of citation networks through an automatic access to Google Scholar that provides backward citations. The confrontation of inter-cluster coefficients on the citation network for the different corpuses with lexical consistence are an essential aspect of a further validation of our results.
}{
Les développements possibles incluent la construction de réseaux de citation via un accès automatique à Google Scholar qui fournit les citations entrantes. La confrontation des coefficients inter-clusters pour le réseau de citations entre les différents corpus avec la cohérence lexicale est un aspect clé d'une validation approfondie des résultats.
}



\bpar{
The disturbing absence of models simulating the co-evolution of transportation networks and urban land-use, confirmed through a state-of-the-art covering many domain, may be due to the absence of communication between scientific disciplines studying different aspects of that problems. We have proposed an algorithmic method to give elements of answers through text-mining-based corpus extraction. First numerical results seem to confirm the assumption. However, such a quantitative analysis should not be considered alone, but rather come as a back-up for qualitative studies that will be the object of further work, such as the one lead in~\cite{commenges:tel-00923682}, in which questionnaires with historical actors of modeling provide highly relevant information.
}{
L'absence peu explicable a priori de modèles qui simulent la co-évolution des réseaux de transport et de l'usage du sol urbain, qui se confirme à première vue par un état de l'art couvrant des domaines disparates, pourrait être due à l'absence de communication entre les disciplines scientifiques étudiant différents aspects du problème. Nous avons proposé une méthode algorithmique pour donner des éléments de réponse par l'extraction de corpus basée sur l'analyse textuelle. Les premiers résultats numériques semblent confirmer l'hypothèse. Cependant, une telle analyse quantitative ne doit pas être considérée seule, mais devrait plutôt venir comme soutien à des études qualitatives qui peuvent être l'objet de développements futurs, comme celle menée dans~\cite{commenges:tel-00923682}, dans laquelle des questionnaires avec des acteurs historiques fournit des informations extrêmement pertinentes.
\comment{(Florent) tu l'as déjà dit; il y a d'autres arguments à mobiliser (comme les cas d'application potentiels de tes modèles) - sur des temporalités si longues qu'ils ne bénéficient pas ou peu de financements ad hoc}
}




%--------------------------------------------------------------



\subsection[Indirect Bibliometrics][Bibliométrie indirecte]{Indirect bibliometrics through Complex Network analysis}{Bibliométrie Indirecte par Analyse de Réseaux Complexes}

\label{subsec:indirectbibliometrics}


\bpar{
As described before, semantic analysis of final corpus does not contain all the information on disciplinary compartmentation nor on patterns of propagation of scientific knowledge as the ones contained in citation networks for example. Furthermore, data collection in the previous algorithm is subject to convergence towards self-consistent themes because of the proper structure of the method. It may give more insight about scientific social patterns of ontological choices in modeling to study communities in broader networks, that would more correspond to disciplines (or sub-disciplines depending on granularity level). We propose to reconstruct disciplines around our thematic, to obtain a more precise view of interdisciplinarity and the scientific landscape on our subject.
}{
Comme décrit précédemment, l'analyse sémantique des corpus finaux ne contient pas la totalité de l'information sur les liens entre disciplines ni sur les motifs de propagation de la connaissance scientifique comme ceux contenus dans les réseaux de citations par exemple. De plus, la collection des données dans l'algorithme précédent est sujette à convergence vers des thèmes relativement auto-cohérents de par la structure propre de la méthode. On pourrait obtenir plus d'information sur les motifs sociaux de choix ontologiques pour la modélisation en étudiant les communautés dans des réseaux plus larges, ce qui correspondrait plus à des disciplines (ou des sous-disciplines selon le niveau de granularité). Nous proposons de reconstruire les disciplines autour de notre thématique, pour obtenir une vue plus précise de l'interdisciplinarité et du paysage scientifique sur notre sujet. 
}





%%%%%%%%%%%%%%%%%%%%%%%%





%%%%%%%%%%%%%%%%%%%%%%%%
\subsection{Discussion}{Discussion}




\subsubsection{Towards modeling purpose and context automatic extraction}{Vers une modélisation des thèmes et une extraction automatique du contexte}


\bpar{
A possible direction to strengthen our quantitative epistemological analysis would be to work on full textes related to the modeling of interaction between networks and territories, with the aim to automatically extract thematics within articles. The idea would be to perform some kind of automatized modelography, with possible features to be extracted that would be ontologies, model architecture or structures, scales, or even typical parameter values. It is not clear to what degree structure of models can be extracted from their description in papers and it surely depends on the discipline considered. For example in a framed field such as transportation planning, using a pre-defined ontology (in the sense of dictionary) and a fuzzy grammar could be efficient to extract information as the discipline is relatively formatted. In theoretical and quantitative geography, beyond the barrier of language, information organisation is surely less subject to unsupervised data-mining because of the more literary nature of the discipline : synonyms and figures of speech are generally the norm in good level human sciences writing, fuzzing a possible generic structure of knowledge description. 
}{
Une direction possible pour renforcer cette analyse en épistémologie quantitative serait de travailler sur les textes complets des références contenant des efforts de modélisations des interactions entre réseaux et territoires, avec le but d'extraire automatiquement les thématiques des articles. Des méthodes plus adaptées pour les long texte que celle utilisée ici incluent par exemple l'Allocation Latente de Dirichlet~\cite{blei2003latent}. L'idée serait de procéder à une sorte de modélographie automatique, pour extraire des caractéristiques telle les ontologies, l'architecture ou la structure des modèles, les échelles ou même des valeurs typiques des paramètres. Il n'est pas clair dans quelle mesure la structure des modèles peut être extraite de leur description dans un article, et cela dépend sûrement de la discipline considérée. Par exemple dans champ relativement cadré comme la planification des transports, l'utilisation d'une ontologie pré-définie (dans le sens d'un dictionnaire) et d'une grammaire floue pourrait être efficace vu les conventions assez strictes dans la discipline. En géographie théorique et quantitative, au delà de la barrière du language, l'organisation de l'information est sûrement plus délicate à appréhender par de l'apprentissage non-supervisé à cause de la nature plus littéraire de la discipline : les synonymes et les figures de style sont généralement la norme pour l'écriture d'un bon niveau en sciences humaines, rendant plus floue une possible structure générique de la description des connaissances.
}


%Depending on extended results of the two previous sections and on thematic requirements (huge need of knowledge on precise models structure, that may appear when trying to construct more specialized operational models), this project may be conducted with more or less investment.




\subsubsection{Reflexivity}{Réflexivité}


\bpar{
The methodology developed here is particularly interesting since it is reflexive, i.e. it can be used on our work itself. Therefore, an other application will be the reflexivity of our thesis : we attend to proceed to similar analysis on our proper bibliography (and possibly its evolution, available via \texttt{git} history), to understand our patterns of knowledge, possible gaps or unveil unexpected developments. The detailed development is done in Appendix~\ref{app:reflexivity}.
}{
La méthodologie que nous avons développé ici est particulièrement intéressante puisqu'elle offre des potentialités de réflexivité, c'est à dire qu'elle peut être utilisée pour étudier notre approche elle-même. Une de ses applications, hors de celle à la revue scientifique Cybergeo dans la perspective de Science Ouverte (voir Appendice~\ref{}) % appendix cybergeo
, sera à notre propre corpus de références, dans le but de révéler des possibles  directions de recherche ou problématiques exotiques. Il est éventuellement possible de le faire de manière dynamique, grâce à l'historique de \texttt{git} qui permet de récupérer n'importe quelle version de la bibliographie à une date donnée sur les trois ans écoulés. Il s'agira aussi de comprendre nos motifs de production de connaissance afin de contribuer à~\ref{sec:knowledgeframework}. Le développement détaillé est fait en Appendice~\ref{app:reflexivity}.
}





\stars



