



%----------------------------------------------------------------------------------------

\newpage


\section{Contributions and Perspectives}{Contributions et Perspectives}

\label{sec:contributions}

%----------------------------------------------------------------------------------------



%%%%%%%%%%%%%%%%%%%%%%%%%%%%%
\subsection{Defining co-evolution}{Définition de la co-évolution}


% résultats empiriques, théoriques et méthodes

% -> done in multiple knowledge domains



%%%%%%%%%%%%
% On accessibility 

% -> sur remise en cause formulation statique de l'accessibilité

%De tels problèmes opérationnels confirment la complexité de l'intégration des réseaux de transports et des territoires dans dynamiques des systèmes territoriaux, et nous devrons donner dans notre travail des éléments de réponse pour une définition de l'accessibilité qui intégrerait les dynamiques territoriales intrinsèques.
% idem
%Il faut cependant rester prudent sur son usage inconditionnel. Plus précisément, il peut s'agir d'une construction qui ignore une partie conséquente des dynamiques territoriales. L'accessibilité pourrait\comment[AB]{conditionnel ambigu : preciser} de même manière que la mobilité être une construction sociale\comment[FL]{1) gratuit 2) ce n'est pas la meme chose} et n'avoir que peu de fondement théorique, puisqu'il s'agit en grande partie d'un outil de modélisation et de planning.\comment[FL]{a reprendre ; non} 




\paragraph{Spatial scales ans non-stationarity}{Echelles spatiales et non-stationnarité}

% - correlations spatiales / multi-scalarité





\paragraph{Co-évolution régimes}{Régimes de co-évolution}

% - causality regimes -> co-evolution regimes


pourquoi serait-ce un bon proxy ? perspective vers réseaux causaux
\cite{seth2005causal}
\cite{castellacci2013dynamics}






%\subsubsection{Spatio-temporal diffusion}{Diffusion spatio-temporelle}

\bpar{
The application of our approach must be lead carefully regarding the choice of scales, processes and objects of study. Typically, it will be not adapted to the quantification of spatio-temporal processes for which the temporal scale of diffusion if of the same order than the estimation window, as our stationarity assumption here stays basic. We could propose to proceed to estimations on moving windows but it would then require the elaboration of a spatial correspondence technique to follow the propagation of phenomena. An example of concrete application that would have a strong thematic impact would be a characterization of a fundamental component of the Evolutive Urban Theory that is the hierarchical diffusion of innovation between cities~\cite{pumain2010theorie}. This would be done by analyzing potential spatio-temporal dynamics of patents classifications such as the one introduced by~\cite{10.1371/journal.pone.0176310}. We also underline that these are rather open methodological questions, for which a concretisation is the potential link between the non-ergodic properties of urban systems~\cite{pumain2012urban} and a wave-based characterization of these processes.
}{
L'application de notre approche doit être menée précautionneusement concernant le choix des  échelles, processus et objets d'étude. Typiquement, elle ne sera pas du tout adaptée à la quantification de processus spatio-temporels dont l'échelle temporelle de diffusion est de l'ordre de celle de la fenêtre d'estimation : l'hypothèse de stationnarité est basique. On peut proposer de procéder à des estimations par fenêtres glissantes, mais il faudrait ensuite élaborer une technique de correspondance spatiale pour traquer la propagation des phénomènes. Un exemple d'application concrète à l'impact thématique fort serait une caractérisation d'une composante fondamentale de la Théorie Evolutive des Villes, la diffusion hiérarchique de l'innovation entre les villes~\cite{pumain2010theorie}, en analysant les potentielles dynamiques spatio-temporelles des classifications de brevets comme celle introduite par~\cite{10.1371/journal.pone.0176310}. Il faut noter toutefois qu'il s'agit de questions méthodologiques relativement ouvertes, dont une des manifestations est le lien potentiel entre le caractère non-ergodique des systèmes urbains~\cite{pumain2012urban} et une caractérisation ondulatoire de ces processus.
\cite{co2002evolution} diffusion de l'innovation
}


%\subsubsection{Geographically Weighted Regression}{Regression Géographique Pondérée}

\bpar{
An other direction for developments and potential applications can be found when going to a more local scale, by exploring an hybridation with Geographically Weighted Regression techniques~\cite{brunsdon1998geographically}. The determination by cross-validation of Akaike criterion of an optimal spatial scale for the performance of these models, as done by~\cite{2017arXiv170607467R} in a multi-modeling fashion, could be adapted in our case to determine a local optimal scale on which lagged correlations would be the most significant, what would allow to tackle the question of non-stationarity by a mostly spatial approach.
}{
Une autre direction de développement et d'applications potentiels se révèle en se tournant vers l'échelle plus locale, et d'explorer une hybridation avec les techniques de Regression Géographique Pondérée~\cite{brunsdon1998geographically}. La détermination par validation croisée ou Critère d'Akaike d'une portée spatiale optimale pour la performance de ce type de modèles pourrait être adaptée dans notre cas pour déterminer une échelle locale optimale sur laquelle les correlations retardées sont les plus significatives, ce qui permettrait de s'extraire du problème de la non-stationnarité prioritairement par l'aspect spatial.
}



%%%%%%%%%%%%%%%
%\subsection{Conclusion}{Conclusion}
%%%%%%%%%%%%%%%

%We have introduced a generic method of Granger causality on territorial spatio-temporal data, and shown its potentialities and operational nature with synthetic data and on a real case. We postulate that the simple methodological apparel is am asset for a certain level of generality, but that the application to complex case studies exhibiting circular causalities demonstrate the high potential to contribute to the understanding of dynamics for this type of co-evolutive systems.

%Nous avons proposé une méthode générique de causalité de Granger sur des données territoriales, puis démontré sa potentialité et son caractère opérationnel sur données synthétiques et un cas réel. Nous postulons que l'appareillage méthodologique simple est un atout pour une certaine généralité, mais que l'application à ces cas complexes présentant des causalités circulaires a un fort potentiel de contribution à la compréhension des dynamiques de ce type de systèmes co-évolutifs.






\paragraph{Empirical applicability}{Applicabilité empirique}

Nos différents cas d'étude empiriques témoignent de la difficulté voire de l'impossibilité de mettre en place les méthodes testées sur des données synthétiques ou uniquement théorique. Prenons l'illustration de la méthode des régimes de causalité : sur les données d'Ile-de-France en~\ref{sec:casestudies}, sur une échelle temporelle courte et une portée spatiale restreinte, son application suggère l'existence de différents régimes. Sur les données sud-africaines en~\ref{sec:causalityregimes}, on n'est pas capable de classifier les relations entre différentes variables, notamment à cause de l'autocorrélation de l'accessibilité, en on dévoie la méthode à l'étude unique d'un sens de causalité entre croissance de population et croissance de temps moyen de trajet, ce qui donne toutefois des résultats concluants. Enfin, dans le cas de la France en~\ref{}, qu'on peut qualifier de pire au regard de l'esprit initial de la méthode, le signal obtenu est très faible, avec quasiment aucune corrélation significative pour la majorité des dates de 1836 à 1946. On dégage toutefois les résultats intéressant d'échelle intermédiaire de stationnarité spatiale, ainsi que d'une échelle de stationnarité temporelle pour les relations à longue distance. Ainsi en pratique, la méthode est bien loin de fonctionner comme attendu, et les résultats peuvent provenir d'analyses annexes ou préliminaires.

Dans le cas des analyses des corrélations statiques, qui pourraient ouvrir une porte à une analyse fine et des corrélations significatives, on a déjà vu que l'absence de données temporelles empêche toute perspective d'analyse dans ce sens. En résumé, la co-évolution est si difficile à caractériser empiriquement, car (i) soit il n'y a effectivement aucune dynamique apparente, c'est à dire que les variables observables sont assimilables à du bruit (ce cas rejoint une grande partie de la littérature qui conclut à des dynamiques au cas par cas) ; (ii) les données sont très pauvres et malgré des indices suggérant l'existence de régimes de co-évolution, ceux-ci sont difficile à caractériser.


Dans cette perspective, et au regard des résultats plus concluants obtenus par l'intermédiaire de la modélisation, une direction future de recherche, bien au delà de la portée de notre travail de par l'envergure de l'entreprise, consiste en l'élaboration de méthodes hybrides qui viseraient à compléter les données manquantes par l'intermédiaires de modèles développés. Plus précisément, % a developer - en lien avec multi-echelles.



%%%%%%%%%%%%%%%%%%%%%%%%%%%%%
\subsection{Systems of Cities and the macroscopic scale}{Systèmes de villes et échelle macroscopique}

% - faits stylises obtenus
% - comparaison simpopnet ; portugali ; baptiste
% - premiere fois network dans modele systeme de villes ; premiere fois coevolution observee.
% - comparaison hypotheses/ constats empiriques (cf espace geo et Bretagnolle)



%%%%%%%%
%% Interaction Gibrat


%%%%%%%%%%%%%%%%%%%%%%%%%%%
\paragraph{Theoretical implications}{Implications théoriques}


\bpar{
Our results support the hypothesis that physical transportation networks are necessary to explain the morphogenesis of territorial systems, in the sense that some aspects are fully contained within networks and cannot be approximated by abstract proxies. We showed indeed on a relatively simple case that the integration of physical networks into some models effectively increase their explanative power even when controlling for overfitting. This can be understood as a direction to expand Pumain's Evolutive Urban Theory~\citep{pumain1997pour}, that consider networks as carriers of interactions in systems of cities but do not put particular emphasis on their physical aspect and the possible spatial patterns resulting from it such as bifurcations or network induced differentiations. The development of a sub-theory focusing on these aspect is an interesting direction suggested by our empirical and modeling results.
}{
Nos résultats soutiennent l'hypothèse que les réseaux de transports physique sont nécessaire pour expliquer la morphogenèse des systèmes territoriaux, au sens où certains aspects sont entièrement contenus\comment[FL]{trop fort} dans les réseaux et ne peuvent pas être approchés par des proxy\comment[FL]{cela tu ne le montres pas} abstraits. Nous avons montré en effet sur un cas relativement simple que l'intégration des réseaux physiques dans certains modèles améliore effectivement leur pouvoir explicatif même lorsqu'on contrôle pour l'overfitting. Cela peut être compris comme une direction pour étendre la Théorie Evolutive des Villes de \noun{Pumain}~\cite{pumain1997pour}, qui considère les réseaux comme médiateurs des interactions dans les systèmes de villes mais ne met pas d'accent précis sur leur aspect physique et les possibles motifs spatiaux en résultant comme des bifurcations ou des différenciations induites par le réseau. Le développement d'une sous-théorie se concentrant sur ces aspects est une direction intéressante suggérée par ces résultats empiriques et de modélisation. Nous explorerons cette piste en section~\ref{sec:theory}.
}

\paragraph{Urban System Specificity}{Spécificité du système urbain}


\bpar{
The model has not yet been tested on other urban systems and other temporalities, and further work should investigate which conclusions we obtained here are specific to the French Urban System on this periods, and which are more general and could be more generic in system of cities. Applying the model to other system of cities also recalls the difficulty of defining Urban Systems. In our case, a strong bias should arise from considering France only, as Lille must be highly influenced by Brussels for example. The extent and scale of such models is always a delicate subject. We rely here on the administrative coherence and the consistence of the database, but sensitivity to system definition and extent should also be further tested.
}{
Le modèle n'a pas encore été testé sur d'autres systèmes urbains et d'autres étendues temporelles, et les développements futurs devront étudier quelles conclusions obtenues ici sont spécifiques au système de villes français sur ces périodes, et lesquelles sont plus générales et pourraient être plus génériques dans les systèmes de villes. L'application du modèle à d'autres systèmes de villes rappelle également la difficulté de définir les systèmes urbains. Dans notre cas, une fort biais doit être induit par le fait de considérer la France seule, comme Lille doit être fortement influencée par Bruxelles par exemple.\comment[FL]{c'est loin d'etre le seul exemple (on parle de metropoles transfrontalieres)} L'étendue et l'échelle de tels modèles est toujours un sujet délicat. Nous reposons ici sur la cohérence administrative et celle de la base de données, mais la sensibilité à la définition du système et à son étendue doivent encore être testés.
}





%%%%%%%%%%%%
%% Macro Coevol


\paragraph{Multi-layer network}{Réseau multi-couches}


\bpar{
Specifically-designed database of the highway networks containing its full genesis from 1950 to 2015).
}{
La considération d'un seul mode de transport pour le système réel est bien sûr réductrice, et une direction immediate de développement est d'une part le test du modele avec des matrices de distance réelles pour d'autres types de réseaux, comme le réseau autoroutier qui a connu un essor considerable en France entre 1950 et 1999. Cette application nécessite la mise en place d'une base dynamique pour la croissance du réseau couvrant 1950 à 2015, les bases classiques (IGN ou OpenStreetMap n'integrant pas la date d'ouverture des tronçons). Une extension naturelle du modele consisterait alors en la mise en place d'un réseau multi-couches, approche typique pour représenter des systèmes de transport multi-modaux~\cite{gallotti2014anatomy}. Chaque couche du réseau de transport devrait avoir une dynamique co-evolutive avec les populations, avec possiblement l'existence d'une dynamique inter-couches.
}



\paragraph{Physical Network}{Réseau physique}



Ces extensions sont aussi l'objet de~\cite{mimeur:tel-01451164}, qui produit des résultats intéressants quant à l'influence de la centralisation de la décision d'investissement dans le réseau sur les formes finales, mais garde des populations statiques et ne produit pas de modèle de co-évolution. De même, le choix des indicateurs pour quantifier la distance du réseau simulé à un réseau réel est un problème délicat dans ce contexte : des indicateurs comme le nombre d'intersections pris par~\cite{mimeur:tel-01451164} relève de la modélisation procédurale et non d'indicateurs de structure. C'est probablement pour la même raison que~\cite{schmitt2014modelisation} ne s'intéresse qu'aux trajectoires de population et pas aux indicateurs de réseau : la conjonction et l'ajustage des dynamiques de population et de réseau à des échelles différentes semble être un problème difficile.





%%%%%%%%%%%%%%%%%%%%%%%%%%%%%
\subsection{Territories and the macroscopic scale}{Territoires et échelle mesoscopiques}


% - apport des modèles par rapport a modeles existants.




%%%%%%%%
%% Density Morphogenesis

\paragraph{Integration into a multi-scale growth model}{Intégration dans un modèle de croissance multi-scalaire}

\bpar{
The question of the generic character of the model is also open: would it work as well when trying to reproduce Urban Forms on very different systems such as the United States or China. A first interesting development would be to test it on these systems and at slightly different scales (1km cell for example). Finally, we believe that a significant insight into the non-stationarity of Urban Systems would be allowed by its integration into a multi-scale growth model. Urban growth patterns have been empirically shown to exhibit multi-scale behavior~\cite{zhang2013identifying}. Here at the meso-scale, total population and growth rates are fixed by exogenous conditions of processes occurring at the macro-scale. It is particularly the aim of spatial growth models such as the Favaro-pumain model~\cite{favaro2011gibrat} to determine such parameters through relations between cities as agents. One would condition the morphological development in each area to the values of the parameters determined at the level above. In that setting, one must be careful of the role of the bottom-up feedback: would the emerging urban form influence the macroscopic behavior in its turn ? Such multi-scale complex model are promising but must be considered carefully.
}{
La question du caractère générique du modèle est également ouverte, c'est à dire s'il fonctionnerait de la même manière pour reproduire des formes urbaines sur des systèmes très différents comme les Etats-Unis ou la Chine. Un premier développement intéressant serait de le tester sur ces systèmes et à des échelles légèrement différentes (cellules de taille 1km par exemple). Enfin, nous pensons qu'un gain de connaissance important concernant la non-stationnarité des systèmes urbains serait rendu possible par son intégration dans un modèle de croissance multi-échelles. Les motifs de croissance urbaine ont été prouvés empiriquement exhibant un comportement multi-échelle~\cite{zhang2013identifying}. Ici à l'échelle mesoscopique, la population totale et le taux de croissance sont fixés par les conditions exogènes de processus se produisant à l'échelle macroscopique. C'est particulièrement le but des modèles spatiaux de croissance comme le modèle Favaro-Pumain~\cite{favaro2011gibrat} de déterminer de tels paramètres par les relations entre villes comme agents. On pourrait conditionner le développement morphologique de chaque zone aux valeurs des paramètres déterminés au niveau supérieur. Dans ce contexte, il faudrait être prudent sur le rôle de la retroaction bottom-up: la forme urbaine émergente devrait-elle influence le comportement macroscopique à son tour ? De tels modèles complexes multi-scalaires sont prometteurs mais doivent être considérés avec précaution.
}


%  determination of effective independent dimensions of the urban system at this scale ?













