



%----------------------------------------------------------------------------------------

\newpage


\section{Contributions and Perspectives}{Contributions et Perspectives}

\label{sec:contributions}

%----------------------------------------------------------------------------------------


Nous proposons à présent de passer en revue nos contributions au regard des différents cadres existants revus en première partie, et de suggérer des perspectives ouvertes. Nous le faisons dans la logique de notre problématique générale : dans un premier temps nos apports sur la définition et la caractérisation de la co-évolution, dans un second temps les différentes approches de modélisation de celle-ci.



%%%%%%%%%%%%%%%%%%%%%%%%%%%%%
\subsection{Definition and characterisation of co-evolution}{Définition et caractérisation de la co-évolution}


L'étape de définition et de caractérisation de la co-évolution se repose sur des résultats empiriques, théoriques et méthodologiques.


\subsubsection{Conceptual definition}{Définition conceptuelle}

L'une de nos contributions principales est la construction d'une définition de la co-évolution au sein des systèmes territoriaux. Comme développé en~\ref{sec:epistemology}, la géographie utilise ce concept de manière très floue, tandis que des disciplines où son usage pourrait sembler plus mature comme l'économie géographique évolutionnaire ne s'accordent pas sur un usage précis~\cite{schamp201020}.

Nous précisons ainsi la définition qui en est prise dans la théorie évolutive des villes (voir par exemple \cite{paulus2004coevolution}), en gardant compatibilité. Notre définition repose en effet sur trois axes :
\begin{enumerate}
	\item existence de processus de transformation des composantes du système territorial (\emph{evolution}\footnote{Sachant qu'on peut établir une correspondance faible avec reproduction et mutation, notamment dans le cas de composantes socio-économiques ``simples'' pour lesquelles les principes de l'évolution culturelle s'appliquent, mais que la correspondance devient conceptuelle quand les entités considérées sont plus complexes, comme justement dans notre cas des villes et des réseaux de transport.}) ;
	\item modalités de co-évolution à différents niveaux : local, population, système\footnote{Qui sont hiérarchiquement nécessaires : une relation au niveau de la population en implique une au niveau des individus, et la vue systémique implique une co-évolution au niveau des populations.} ;
	\item modularité en sous-systèmes territoriaux : les entités territoriales sont à la fois le support et l'objet de la co-évolution.
\end{enumerate}


Notre apport par rapport à la littérature géographique mobilisant le concept est une clarification, qui permet par ailleurs la mise en place dans certains cas d'une caractérisation empirique. \cite{paulus2004coevolution} ou \cite{bretagnolle1998space} partent du postulat que la co-évolution existe nécessairement au sein des systèmes de villes, entre villes ou entre villes et réseaux de transport. Notre approche laisse une entrée à une vérification empirique et étend également l'application aux territoires de manière plus générale.

En positionnement par rapport à la littérature en économie géographique (voir~\cite{schamp201020}), notre approche permet une vision fondamentalement multi-échelles, et donc plus facilement compatible avec les positionnements géographiques comme celui de la théorie évolutive des villes.

Enfin, nous avons étudié particulièrement le concept dans le cadre des interactions entre réseaux de transports et territoires : nous montrons qu'il s'agit d'un type de système territorial pour lequel il est particulièrement pertinent et opérationnel. Nous pouvons par là même revisiter le débat des effets structurants : la congruence de \cite{offner1993effets} peut être soit une corrélation fortuite, soit un vrai effet de co-évolution au niveau de la population. Une manifestation locale (lien local ``attendu'' entre deux entités) peut mais n'a pas de raison particulière de se manifester en tant que co-evolution au niveau de la population des entités (et donc il n'y a bien sûr aucun ``effet systématique''). Mais qualifier les approches de cette question de ``mystification scientifique'' \cite{offner1993effets} relève du réductionnisme scientifique, que notre approche contribue à dépasser.





\subsubsection{Spatial scales and non-stationarity}{Echelles spatiales et non-stationnarité}

Une contribution empirique, permettant d'apporter des pistes pour la caractérisation de la co-évolution, est issue du travail mené en~\ref{sec:staticcorrelations}. L'existence de différentes échelles spatiales observables dans les corrélations statiques entre caractéristiques du territoire et celles du réseau, ainsi que la non-stationnarité spatiale de celles-ci, suggère la vérification du dernier point de notre définition, à savoir l'existence de sous-systèmes territoriaux au sein desquels la co-évolution pourrait se manifester.




\subsubsection{Co-évolution régimes}{Régimes de co-évolution}

Notre contribution fondamentale en termes de caractérisation de la co-évolution est la méthode des régimes de causalité développée en~\ref{sec:causalityregimes}. Nous suggérons que selon les régimes observables, certains sont en effet des régimes de co-évolution, puisque présentant des relations causales circulaires observées statistiquement au niveau d'une population. Il s'agit ainsi d'une caractérisation empirique du niveau intermédiaire de co-évolution, qui est par ailleurs particulièrement intéressant puisque coincide avec les sous-systèmes territoriaux\footnote{Qui donne alors toute sa puissance à l'approche par la morphogenèse, en faisant le lien comme nous l'avons déjà suggéré et le développerons par la suite, avec la notion de niche écologique~\cite{holland2012signals}.}.

Nous pensons que notre mesure est un relativement bon proxy d'une co-évolution, puisque son application s'oriente vers l'étude des réseaux causaux~\cite{seth2005causal}, c'est à dire un ensemble de relations dirigées entre variables. \cite{castellacci2013dynamics} applique par exemple une méthode similaire à la notre, mais étendue à un réseau de variables, pour quantifier la co-évolution entre innovation et capacité d'absorption des territoires.

Notre approche est à remettre en perspective avec la vue de \noun{Diderot} présentée en~\ref{ch:thematic} : s'il existe une niche dans laquelle on isole des relations en effet circulaires, alors sur le temps long le drift d'évolution par rapport à d'autres niches les entrainera sur des trajectoires bien différentes\footnote{Nous avons par ailleurs considéré ce cas de manière indirecte dans les modèles, lorsqu'ils sont calibrés sur le temps long sur des périodes successives : l'évolution des paramètres correspond à des dynamiques évolutives sur le temps long.}. D'où l'importance de notre cadre général multi-scalaire, qui permet par ailleurs la considération du système plus globalement, et au sein duquel la mise en réseau des sous-systèmes complexifiera alors les relations de co-évolution\footnote{Il y aurait sur ce point une plus grande complexité des systèmes territoriaux par rapport aux systèmes biologiques ``simples'', c'est à dire ceux dans lesquels des niches écologiques sont clairement identifiables et isolables, dans le cas où la mise en réseau entre sous-systèmes est limitée.}.





\subsubsection{Empirical applicability}{Applicabilité empirique}

Nos différents cas d'étude empiriques témoignent toutefois de la difficulté de mettre en place les méthodes testées sur des données synthétiques ou uniquement théorique. L'application de la méthode des régimes de causalité donne des résultats très divers. Sur les données d'Ile-de-France en~\ref{sec:casestudies}, à une échelle temporelle courte et une portée spatiale restreinte, son application suggère l'existence de différents régimes. Sur les données sud-africaines en~\ref{sec:causalityregimes}, on n'est pas capable de classifier les relations entre différentes variables, notamment à cause de l'autocorrélation de l'accessibilité, mais la méthode permet l'étude d'un sens de causalité entre croissance de population et croissance de temps moyen de trajet, ce qui donne toutefois des résultats concluants. Enfin, dans le cas de la France en~\ref{sec:macrocoevol}, le signal obtenu est très faible, avec quasiment aucune corrélation significative pour la majorité des dates de 1836 à 1946. On dégage toutefois les résultats intéressant d'échelle intermédiaire de stationnarité spatiale, ainsi que d'une échelle de stationnarité temporelle pour les relations à longue distance. Ainsi en pratique, l'application de la méthode est à considérer au cas par cas, et les résultats peuvent provenir d'analyses annexes ou préliminaires.


Dans le cas des analyses des corrélations statiques, qui pourraient ouvrir une porte à une analyse fine et des corrélations significatives, on a déjà vu que l'absence de données temporelles empêche toute perspective d'analyse dans ce sens.

En résumé, la co-évolution reste difficile à caractériser empiriquement, car (i) soit il n'y a effectivement aucune dynamique apparente, c'est à dire que les variables observables sont assimilables à du bruit (ce cas rejoint une grande partie de la littérature qui conclut à des dynamiques au cas par cas) ; (ii) les données sont très pauvres et malgré des indices suggérant l'existence de régimes de co-évolution, ceux-ci sont difficile à caractériser.



\subsubsection{Perspectives}{Perspectives}

\bpar{
The application of our approach must be lead carefully regarding the choice of scales, processes and objects of study. Typically, it will be not adapted to the quantification of spatio-temporal processes for which the temporal scale of diffusion if of the same order than the estimation window, as our stationarity assumption here stays basic. We could propose to proceed to estimations on moving windows but it would then require the elaboration of a spatial correspondence technique to follow the propagation of phenomena.
}{
L'application de notre méthode des régimes de causalité doit être menée précautionneusement concernant le choix des échelles, processus et objets d'étude. Typiquement, elle ne sera pas du tout adaptée à la quantification de processus spatio-temporels dont l'échelle temporelle de diffusion est de l'ordre de celle de la fenêtre d'estimation : l'hypothèse de stationnarité est basique. Nous pouvons proposer de procéder à des estimations par fenêtres glissantes, mais il faudrait ensuite élaborer une technique de correspondance spatiale pour traquer la propagation des phénomènes.
}


\bpar{
An example of concrete application that would have a strong thematic impact would be a characterization of a fundamental component of the Evolutive Urban Theory that is the hierarchical diffusion of innovation between cities~\cite{pumain2010theorie}. This would be done by analyzing potential spatio-temporal dynamics of patents classifications such as the one introduced by~\cite{10.1371/journal.pone.0176310}. We also underline that these are rather open methodological questions, for which a concretisation is the potential link between the non-ergodic properties of urban systems~\cite{pumain2012urban} and a wave-based characterization of these processes.
}{
Un exemple d'application concrète à l'impact thématique fort serait une caractérisation d'une composante fondamentale de la théorie évolutive des villes, la diffusion hiérarchique de l'innovation entre les villes~\cite{pumain2010theorie}, en analysant les potentielles dynamiques spatio-temporelles des classifications de brevets comme celle introduite par~\cite{10.1371/journal.pone.0176310}, pour revisiter des analyses comme \cite{co2002evolution} avec un point du vue plus fin à la fois géographiquement et pour la caractérisation de l'innovation. Il faut noter toutefois qu'il s'agit de questions méthodologiques relativement ouvertes, dont une des manifestations est le lien potentiel entre le caractère non-ergodique des systèmes urbains~\cite{pumain2012urban} et une caractérisation ondulatoire de ces processus.
}




\bpar{
An other direction for developments and potential applications can be found when going to a more local scale, by exploring an hybridation with Geographically Weighted Regression techniques~\cite{brunsdon1998geographically}. The determination by cross-validation of Akaike criterion of an optimal spatial scale for the performance of these models, as done by~\cite{2017arXiv170607467R} in a multi-modeling fashion, could be adapted in our case to determine a local optimal scale on which lagged correlations would be the most significant, what would allow to tackle the question of non-stationarity by a mostly spatial approach.
}{
Une autre direction de développement et d'applications potentiels se révèle en se tournant vers l'échelle plus locale, et d'explorer une hybridation avec les techniques de Regression Géographique Pondérée~\cite{brunsdon1998geographically}. La détermination par validation croisée ou Critère d'Akaike d'une portée spatiale optimale pour la performance de ce type de modèles, comme fait en~\ref{sec:staticcorrelations} et en~\ref{sec:energyprice}, pourrait être adaptée dans notre cas pour déterminer une échelle locale optimale sur laquelle les correlations retardées sont les plus significatives, ce qui permettrait de s'extraire du problème de la non-stationnarité prioritairement par l'aspect spatial.
}



%%%%
% -- ON HOLD --
% Dans cette perspective, et au regard des résultats plus concluants obtenus par l'intermédiaire de la modélisation, une direction future de recherche, bien au delà de la portée de notre travail de par l'envergure de l'entreprise, consiste en l'élaboration de méthodes hybrides qui viseraient à compléter les données manquantes par l'intermédiaires de modèles développés. Plus précisément, % a developer - en lien avec multi-echelles.





%%%%%%%%%%%%%%%%%%%%%%%%%%%%%
\subsection{Systems of Cities and the macroscopic scale}{Systèmes de villes et échelle macroscopique}


Nous détaillons à présent nos contributions obtenues par l'intermédiaire de la modélisation, selon les deux axes complémentaires suivis. Dans un premier temps, nous considérons la co-évolution des territoires et des réseaux de transport au sein des systèmes de villes, à l'échelle macroscopique.


\subsubsection{Network effects}{Effets de réseau}

\bpar{
Our results support the hypothesis that physical transportation networks are necessary to explain the morphogenesis of territorial systems, in the sense that some aspects are fully contained within networks and cannot be approximated by abstract proxies. We showed indeed on a relatively simple case that the integration of physical networks into some models effectively increase their explanative power even when controlling for overfitting. This can be understood as a direction to expand Pumain's Evolutive Urban Theory~\citep{pumain1997pour}, that consider networks as carriers of interactions in systems of cities but do not put particular emphasis on their physical aspect and the possible spatial patterns resulting from it such as bifurcations or network induced differentiations. The development of a sub-theory focusing on these aspect is an interesting direction suggested by our empirical and modeling results.
}{
Nos résultats de la section~\ref{sec:interactiongibrat} soutiennent l'hypothèse que les réseaux de transport sont nécessaires pour expliquer la morphogenèse des systèmes territoriaux, au sens où certaines dimensions sont contenues dans les réseaux. Nous avons montré en effet sur un cas relativement simple que l'intégration des réseaux physiques dans certains modèles améliore effectivement leur pouvoir explicatif même lorsqu'on contrôle pour l'overfitting. Cela peut être compris comme une direction pour étendre la théorie évolutive des villes, qui considère les réseaux comme médiateurs des interactions dans les systèmes de villes mais ne met pas d'accent précis sur leur aspect physique et les possibles motifs spatiaux en résultant comme des bifurcations ou des différenciations induites par le réseau. Le développement d'une sous-théorie se concentrant sur ces aspects est une direction intéressante suggérée par ces résultats empiriques et de modélisation. Nous explorerons cette piste en section~\ref{sec:theory}.
}


\subsubsection{Co-evolution at the macroscopic scale}{Co-évolution à l'échelle macroscopique}

% - faits stylises obtenus
% - comparaison simpopnet ; portugali ; baptiste
% - premiere fois calibré dans modele systeme de villes ; premiere fois coevolution observee.
% - comparaison hypotheses/ constats empiriques (cf espace geo et Bretagnolle)
%%%%%%
% synthetic stylized facts
%\begin{enumerate}
%	\item On révèle l'existence d'une échelle spatiale intermédiaire permettant l'évolution de niches relativement indépendantes, correspondant à un niveau de complexité des trajectoires maximal.
%	\item Les corrélations retardées mettent en évidence au moins trois régimes différents d'interaction, que l'on interprète comme un régime d'adaptation, un régime de co-évolution direct et un régime de co-évolution indirecte.
%\end{enumerate}


Concernant la co-évolution en elle-même, à l'échelle du système de villes, notre contribution principale est la compréhension globale des trajectoires et régimes possibles dans un modèle de co-évolution simple. Nous retrouvons les faits stylisé typiques comme le renforcement de la hiérarchie pour certains paramètres d'auto-renforcement comme obtenu par~\cite{baptistemodeling}. Il s'agit à notre connaissance de la première fois qu'un modèle de co-évolution entre transport et villes dans un système de villes est exploré systématiquement, que ses régimes potentiels de co-évolution sont déterminés. Nous montrons également que le modèle de~\cite{schmitt2014modelisation} est moins flexible et ne produit pas de situation de co-évolution à proprement parler.

Pour l'application au cas réel du système de ville français, c'est également à notre connaissance la première fois qu'un tel modèle est calibré sur données observées. Il est difficile de dire si les processus de co-évolution sont effectivement observables, puisqu'au contraire de \cite{bretagnolle2003vitesse} nous ne trouvons pas de relation significative entre croissance des villes et accessibilité. Cette application reste donc stylisée et non opérationnelle dans un premier temps.




\subsubsection{Perspectives}{Perspectives}


\paragraph{Urban System Specificity}{Spécificité du système urbain}


\bpar{
The model has not yet been tested on other urban systems and other temporalities, and further work should investigate which conclusions we obtained here are specific to the French Urban System on this periods, and which are more general and could be more generic in system of cities. Applying the model to other system of cities also recalls the difficulty of defining Urban Systems. In our case, a strong bias should arise from considering France only, as Lille must be highly influenced by Brussels for example. The extent and scale of such models is always a delicate subject. We rely here on the administrative coherence and the consistence of the database, but sensitivity to system definition and extent should also be further tested.
}{
Nos modèles macroscopiques n'ont pas encore été testés sur d'autres systèmes urbains et d'autres étendues temporelles, et les développements futurs devront étudier quelles conclusions obtenues ici sont spécifiques au système de villes français sur ces périodes, et lesquelles sont plus générales et pourraient être plus génériques dans les systèmes de villes. L'application du modèle à d'autres systèmes de villes rappelle également la difficulté de définir les systèmes urbains. Dans notre cas, une fort biais doit être induit par le fait de considérer la France seule, comme Lille doit être fortement influencée par Bruxelles par exemple. L'étendue et l'échelle de tels modèles est toujours un sujet délicat. Nous reposons ici sur la cohérence administrative et celle de la base de données, mais la sensibilité à la définition du système et à son étendue doivent encore être testés.
}

% \comment[FL]{exemple de Lille : c'est loin d'etre le seul exemple (on parle de metropoles transfrontalieres)}




\paragraph{Multi-layer network}{Réseau multi-couches}


\bpar{
Specifically-designed database of the highway networks containing its full genesis from 1950 to 2015).
}{
La considération d'un seul mode de transport pour le système réel est bien sûr réductrice, et une direction immediate de développement est d'une part le test du modele avec des matrices de distance réelles pour d'autres types de réseaux, comme le réseau autoroutier qui a connu un essor considerable en France entre 1950 et 1999. Cette application nécessite la mise en place d'une base dynamique pour la croissance du réseau couvrant 1950 à 2015, les bases classiques (IGN ou OpenStreetMap n'integrant pas la date d'ouverture des tronçons). Une extension naturelle du modele consisterait alors en la mise en place d'un réseau multi-couches, approche typique pour représenter des systèmes de transport multi-modaux~\cite{gallotti2014anatomy}. Chaque couche du réseau de transport devrait avoir une dynamique co-evolutive avec les populations, avec possiblement l'existence d'une dynamique inter-couches.
}



\paragraph{Physical Network}{Réseau physique}



Ces extensions sont aussi l'objet de~\cite{mimeur:tel-01451164}, qui produit des résultats intéressants quant à l'influence de la centralisation de la décision d'investissement dans le réseau sur les formes finales, mais garde des populations statiques et ne produit pas de modèle de co-évolution. De même, le choix des indicateurs pour quantifier la distance du réseau simulé à un réseau réel est un problème délicat dans ce contexte : des indicateurs comme le nombre d'intersections pris par~\cite{mimeur:tel-01451164} relève de la modélisation procédurale et non d'indicateurs de structure. C'est probablement pour la même raison que~\cite{schmitt2014modelisation} ne s'intéresse qu'aux trajectoires de population et pas aux indicateurs de réseau : la conjonction et l'ajustage des dynamiques de population et de réseau à des échelles différentes semble être un problème difficile.





%%%%%%%%%%%%%%%%%%%%%%%%%%%%%
\subsection{Territories and the macroscopic scale}{Territoires et échelle mesoscopiques}

Nous proposons à présent de développer nos contributions pour la modélisation de la co-évolution des territoires et des réseaux de transport à l'échelle mesoscopique. 

\subsubsection{Urban morphogenesis}{Morphogenèse urbaine}

% - apport thematique du cadre de la morphogenese

Dans un premier temps, le cadre conceptuel de la morphogenèse développé en~\ref{sec:interdiscmorphogenesis} est un apport thématique propre pour la modélisation urbaine : nous appuyons le rôle crucial de la forme urbaine, et de son lien fort avec la fonction urbaine. Ce cadre devrait permettre par ailleurs de mieux situer des modèles de morphogenèse comme celui de \cite{bonin2014modelisation} (qui est à notre connaissance l'un des seuls modèles se présentant comme morphogénétiques ayant les fondements théoriques requis) dans un cadre interdisciplinaire.

Il permet également de considérer de façon cohérente des sous-systèmes territoriaux, puisque la recherche de règles morphogénétiques est conjointe à la définition de limites plus ou moins précises au sous-système considéré. Ce point rejoint remarquablement l'isolation géographique requise pour la co-évolution, et nous ferons la jonction théorique par la suite en~\ref{sec:theory}.



\subsubsection{Modeling co-evolution with morphogenesis}{Modélisation de la co-évolution par morphogenèse}

% - apport des modèles par rapport a modeles existants.
% - comparaison barthelemy par ex

L'apport de notre modèle de co-évolution par morphogenèse est multiple et au moins les points suivants sont à noter :
\begin{itemize}
	\item comparaison de multiples heuristiques de génération au sein d'un modèle de co-évolution
	\item calibration sur données morphologiques observées
	\item calibration au premier et au second ordre
	\item étude des régimes de co-évolution produit par un tel modèle.
\end{itemize}

L'ontologie couplée distribution de population et réseau permet le couplage fort entre forme et fonction, et justement de considérer des processus de co-évolution. En comparaison à \cite{barthelemy2009co} qui ne considère que le réseau, notre modèle permet plus de flexibilité dans les processus pris en compte.

\subsubsection{Towards modeling governance}{Vers une modélisation de la gouvernance}

% - positionnement Lutecia ?

Enfin, le modèle Lutecia est également une contribution fondamentale vers la prise en compte de processus plus complexes impliqués dans la co-évolution, comme la gouvernance du système de transport. Comme nous l'avons déjà indiqué, \cite{Xie2011} introduit un modèle économique théorique s'intéressant à une problématique similaire, et \cite{xie2011governance} développe une application simplifiée sur réseau synthétique. Nous allons plus loin en considérant une intégration à un modèle entièrement dynamique d'interaction entre transport et usage du sol, et implémentons une application stylisée au cas réel du Delta de la Rivière des Perles. Ce modèle ouvre la porte à une nouvelle génération de modèles, pouvant être potentiellement opérationnels dans le cas de systèmes régionaux à très grande vitesse d'évolution comme dans le cas Chinois.



\subsubsection{Perspectives}{Perspectives}


\bpar{
The question of the generic character of the model is also open: would it work as well when trying to reproduce Urban Forms on very different systems such as the United States or China. A first interesting development would be to test it on these systems and at slightly different scales (1km cell for example).
}{
La question du caractère générique du modèle de morphogenèse est également ouverte, c'est à dire s'il fonctionnerait de la même manière pour reproduire des formes urbaines sur des systèmes très différents comme les Etats-Unis ou la Chine. Un premier développement intéressant serait de le tester sur ces systèmes et à des échelles légèrement différentes (cellules de taille 1km par exemple).
}


\paragraph{Integration into a multi-scale growth model}{Intégration dans un modèle de croissance multi-scalaire}



\bpar{
Finally, we believe that a significant insight into the non-stationarity of Urban Systems would be allowed by its integration into a multi-scale growth model. Urban growth patterns have been empirically shown to exhibit multi-scale behavior~\cite{zhang2013identifying}. Here at the meso-scale, total population and growth rates are fixed by exogenous conditions of processes occurring at the macro-scale. It is particularly the aim of spatial growth models such as the Favaro-pumain model~\cite{favaro2011gibrat} to determine such parameters through relations between cities as agents. One would condition the morphological development in each area to the values of the parameters determined at the level above. In that setting, one must be careful of the role of the bottom-up feedback: would the emerging urban form influence the macroscopic behavior in its turn ? Such multi-scale complex model are promising but must be considered carefully.
}{
Enfin, nous pensons qu'un gain de connaissance important concernant la non-stationnarité des systèmes urbains serait rendu possible par son intégration dans un modèle de croissance multi-échelles. Les motifs de croissance urbaine ont été prouvés empiriquement exhibant un comportement multi-échelle~\cite{zhang2013identifying}. Ici à l'échelle mesoscopique, la population totale et le taux de croissance sont fixés par les conditions exogènes de processus se produisant à l'échelle macroscopique. C'est particulièrement le but des modèles spatiaux de croissance comme le modèle Favaro-Pumain~\cite{favaro2011gibrat} de déterminer de tels paramètres par les relations entre villes comme agents. On pourrait conditionner le développement morphologique de chaque zone aux valeurs des paramètres déterminés au niveau supérieur. Dans ce contexte, il faudrait être prudent sur le rôle de la retroaction bottom-up: la forme urbaine émergente devrait-elle influence le comportement macroscopique à son tour ? De tels modèles complexes multi-scalaires sont prometteurs mais doivent être considérés avec précaution.
}


%  determination of effective independent dimensions of the urban system at this scale ?


% projet postdoc -> conclusion







\stars









