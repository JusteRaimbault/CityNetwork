

%-------------------------

\newpage


\section{Fieldwork elements}{Elements de terrain}

\label{sec:qualitative}


%-------------------------


Cette section propose d'illustrer la problématique des interactions entre réseaux de transports et territoires, et plus particulièrement leur complexité et la diversité des situations possibles déjà perceptibles de manière subjective à l'échelle microscopique, par des exemples concrets de terrain. Le terrain géographique majoritaire est le Delta de la Rivière des Perles en Chine (\cn{珠江三角洲}), dans la province du Guangdong (\cn{广东}), que nous avons décrit ci-dessus, et plus particulièrement en grande partie la ville de Zhuhai (\cn{珠海}). Dans le cadre du projet européen Medium, visant à une approche interdisciplinaire de la soutenabilité pour les villes Chinoises en se concentrant sur les villes moyenne, cette ville a été choisie comme cas d'étude.


%-------------------------

%%%%%%%%%%%%%%%
\subsection[Floating Observation][Observation Flottante]{An Experiment in Floating Observation}{Une Experience en Observation Flottante}


\bpar{The devil is in the details, and transportation systems are in particular the materialization of this image. What some will see as detail contains the majority of information for others. Logically trapped in a scientific information bubble, despite all}{
Si le diable est dans les détails, les systèmes de transport entre autres sont l'allégorie de cette adage. Ce que certains appellent détail contient la majorité de l'information pour d'autres. Logiquement enfermés dans une bulle scientifique, malgré toutes les volontés développées en introduction, on tâchera de rester conscient de la nature et la portée de la connaissance produite ici. Ce que nous pourrions appeler détail, lors de l'étude de l'accessibilité d'un réseau de transport par exemple, tel des impressions ressenties par les usagers ou les relations sociales induites par les situations découlant des dynamiques du systèmes, seront le centre du questionnement pour un anthropologue ou sociologue. Une telle connaissance, qui trouverait certainement une place dans nos problématiques, est hors de notre portée de par l'absence de \emph{terrain} de longue durée. Nous proposons toutefois ici d'ébaucher une entrée qualitative d'un certain type, pour suggérer une façon de compléter nos connaissances.
}


\bpar{}{
L'entrée prise suit la méthode \emph{d'observation flottante}, introduite à l'interface de l'anthropologie et la sociologie par~\cite{petonnet1982observation}, avec l'ambition de fonder une anthropologie urbaine, au sens de l'étude des comportements humains au sein d'un environnement urbain. Il ne s'agit pas exactement de la même idée que l'anthropologie de l'espace de \emph{Choay}~\cite{choay2009pour} qui explore la direction inverse, c'est à dire le propre des sociétés humaines de façonner l'espace, et la capacité de construire un environnement bâti à différentes échelles par l'architecture et l'urbanisme. Répondant à un besoin de mouvement que le sédentaire éprouve facilement, le chercheur se place au centre du processus de production de connaissances, nous citons, en ``rest[ant] en toute circonstance vacant et disponible, à ne pas mobiliser l'attention sur un objet précis, mais à la laisser flotter afin que les informations la pénètrent sans filtre, sans a priori, jusqu'à ce que des points de repère, des convergences, apparaissent et que l'on parvienne alors à découvrir des règles sous-jacentes''. Sans s'y méprendre et considérer la méthode comme une négligence méthodologique, nous y voyons une opportunité d'un accès rapide et à faible coût dans le monde du qualitatif, tout en restant conscient de sa portée très limitée. La méthode peut servir d'étude préliminaire pour fixer des protocoles et grilles précises d'entretien : elle est par exemple utilisée justement au sujet du transport par~\cite{de2012deplacements}.
}



\bpar{}{
Les mouvements pendulaires à échelle moyenne sont nécessairement vécus d'une façon particulière en comparaison à d'autres lieux géographiques et à d'autres échelles sur le même lieu. Et si une façon d'appréhender des faits stylisés particuliers était alors d'effectuer l'analogue d'une étude de perturbation sur le système, mais en prenant comme référentiel l'observateur lui-même ? Il s'agirait de faire porter un choc sur une situation ``d'équilibre'', puis de se laisser flotter au gré du courant pour appréhender la réaction et certains mécanismes qu'il aurait été difficile de considérer en suivant sa routine. Une expérience naturelle causée par une perturbation des transports (qui en région francilienne est bien courante) est un événement déclencheur de ``naufrages'' de l'observation, au sens où le chercheur peut capturer des situations et réactions individuelles particulières.
}


\paragraph{Systemisation tentative}{Tentative de systématisation}

\bpar{

}{
Notre méthodologie est relativement simple : se laisser errer dans les transports en commun, avec ou sans but et de manière ou non aléatoire, mais en essayant sur chaque trajet de maximiser les opportunités de mise en situation ou de capture d'évènement. La répétition de l'expérience visera également à maximiser la couverture spatiale, temporelle, de situation. Une production traçable est en théorie nécessaire à chaque itération, qu'il s'agisse de description factuelle, de description perçue, de semi-synthèse. Celle-ci permet a posteriori de voir les stratifications successives du vécu et des expériences d'observation progressivement raffinées dans leur contexte, et de tracer ainsi la genèse des idées induites. Nous faisons le choix de retranscrire l'aspect subjectif, voir maximiser celui-ci, dans les synthèses générales des observations, afin d'appuyer cet aspect en contraste avec la suite de notre travail qui sera relativement déconnecté du sujet menant la recherche.
}



\bigskip

\begin{figure}[h!]
\begin{mdframed}
Le ciel est gris et les visages fermés, Oxmo avait tristement raison, ce Soleil du Nord n'avait de lumière que le nom. L'initié ne saura s'y tromper et ressentira au fond de lui-même cette banale routine d'un aller-retour quotidien en RER. Il ne cherchera ni à maudire les planifications successives dont les stratifications temporelles ont laissé décanter cette organisation territoriale incongrue, ni à se prendre à rêver d'une trajectoire de vie alternative puisque choisir c'est un peu mourir et qu'il ne se sent pas une âme de Phoenix aujourd'hui. Peut être que la beauté de la ville est finalement dans ces tensions qui la façonnent à tous les niveaux et dans tous les domaines, ces paradoxes qui deviennent cadre de vie au point d'asséner quotidiennement une vérité. Cette philosophie de couloir de métro, le francilien en fait son cheval de bataille car après tout s'il vit en ville il doit bien la connaître. Encore un rail cassé sur le A, ``tout cela est mal géré, et ce réseau est mal conçu'' vocifère un utilisateur journalier, s'improvisant expert en planification ; d'autres plus patients prennent leur mal en patience mais se présentent tout aussi connaisseurs d'une illusoire vision d'ensemble d'un territoire aux multiples visages. Ces usagers \emph{sont} pourtant le système, de manière concrète à leur échelle d'espace et de temps, par induction et émergence aux échelles supérieures. La fourmi est supposée ne pas avoir conscience de l'intelligence collective dont elle est une des composantes fondamentales. Ils n'ont de la même manière que peu de perception de l'auto-désorganisation dont ils sont la source, peut-être la cause, et qui très sûrement subissent les désagréments de ses dynamiques. Se laisser flotter dans les transports franciliens est une expérience intemporelle. Presque thérapeutique parfois, quand l'un commence à perdre son optimisme quant à l'intérêt d'une vie urbaine, une excursion aléatoire en métro rappelle rapidement la richesse et la diversité qui sont un des plus grand succès des villes. C'est cette variété apparente de profils que le chercheur retiendra principalement de ces errements dont la méthodologie est de ne pas avoir de méthodologie, et il gardera à l'esprit qu'il n'existe pas d'échelle où un traitement spécifique de chaque objet géographiques n'est pas nécessaire : en quelque stations sur la ligne 4 le profil des quartiers et donc des usagers change profondément et souvent sans transition au moins trois fois, comme sur la ligne 13 nord où les motifs horaires soulignent d'autant plus de dures réalités socio-économiques qui sont en fait géographiques dans cet \emph{espace produit} de la métropole. Lorsqu'il s'agit de modéliser, prendre en compte les limites de toute tentative de généralisation est d'autant plus cruciale comme chaque modèle est un équilibre fragile entre spécificité et généralité.

\medskip

\noun{Encadré : } \textit{Une expérience en observation flottante en région parisienne}
\end{mdframed}
\end{figure}


\bigskip


\begin{figure}[h!]
\begin{mdframed}
%Le trajet sera long. La perturbation choisie est la simulation de l'événement malencontreux, ``我的护照丢了,我得去法国的领事馆在广州。''. 
\cn{我的护照丢了,我得去法国的领事馆在广州。}

\noun{Encadré : } \textit{Une expérience en observation flottante, Guangdong, Zhuhai}
\end{mdframed}
\end{figure}





%%%%%%%%%%%%%%%%%%%%%
\subsection{Transportation and local contrasts}{Transports et contrastes locaux}

% TOD Hong-Kong vs Zhuhai

% schémas



One main axis of my fieldwork research in China was to get a subjective view of multiple aspects and layers of the complex and ever-changing public transportation system, in Zhuhai as an illustration for local transportation but also at larger scales in China. Indeed, transportation network modeling or data analysis such as accessibility studies or land-use transport interaction models I develop as the main focus of my thesis, generally fail to grasp microscopic aspects that can become crucial when it comes to the effective use of the network. For example, multimodality can be made effective in practice through self-organized informal transportation modes that solve the ``last-mile'' kilometer problem that seems to be often forgotten in the planning of newly developped areas in China. Or in the contrary, practical details such as ticket booking or check-in delays can drastically change use patterns. I made several trips to understand how the High-speed rail (HSR) network works. Since 2008, China has achieved the largest HSR network starting from nothing, with a huge success and currently saturated lines. These answer to primary demand patterns in terms of city size, but other dimensions have been taken into account in their planning, such as the development of tourism. This way, the Guangzhou-Shenyang line has seen the construction of stations specific to the development of tourism, such as Yangshuo in Guanxi province that has seen its frequentation strongly rising. One year after the station opened, the road link with the city is still in construction, but most of trains stop in the week-end – more than one per hour, and are full more than two weeks in advance. New patterns of mobility must be induced by this new offer, as shown by a Guangzhou inhabitant I interviewed in Yangshuo, that was coming with co-workers just for the week-end as a ``team-building'' trip financed by their startup in information technology. I observed a similar strategy must have been employed for the line Chengdu-Emeishan, which principal objective for now is to desserve the highly visited touristic destination of Leshan and Emeishan, although the missing link from Leshan to Shenyang is already well advanced and will complete the direct link between Guangzhou and Changdu. The complexity of the network is increased by the diversity of missions and congestion, and maximal potential speeds seem to be not systematically exploited. Regional branches such as the Guangzhou-Zhuhai line can be interpreted in-between a long distance service and regional proximity transportation, depending on the modulation of stop distribution. On top of that still exists the classical train network, and some connections require for now to use both and urban transportation, such as Zhuhai-Hong-Kong that I experimented by terrestrial transport only. The local urban network and real estate development operations are planned in close conjunction with the new train network : Zhuhai new tramway, of which a single line is today open and in test, aims at participating to a “Transit-oriented development” (TOD) approach of Urban Development which aims at promoting the use of public transport and a city with less cars, as claimed for example by the High-Tech Zone planning committee in charge of the development around Zhuhai North station. Observing the surroundings of Tangjia station, also constructed in the same spirit, the anti-urban atmosphere and unpractical setting can lead to question the effectiveness of the approach and wonder if it is not more a kind of self-fulfilling prophecy, as suggested by the advertisements for new real estate to sell highlighting the role of the train line. Other field observations, such as in Hong-Kong new territories, witness of an efficient and well achieved TOD, with the smart combination of heavy transit and local light rail, together with high urban densities around stations. These observations recall the complexity of urban trajectories coupled with network development, and how one must be careful before drawing any general conclusion from particular cases.





(1)Tangjia HSR station in Zhuhai, with a huge advertisement for real estate suggesting the importance of the train connection, what can also be used as an argument for higher prices 
(2) HSR line in Zhuhai, deserted bus station and new real estate project in a difficultly accessible area 
(3) Yangshuo station, in the middle of nowhere on the High-speed line, recalling the “betterave-TGV” stations typical to France
(4) Advertisement for HSR in Sichuan, at Chengdu international airport station on the line to Leshan and Emeishan



%%%%%%%%%%%%%%%%%%%%%
\subsection{Urbanism analysis}{Analyse Urbanistique}

% maps, perception de l'espace etc. par exemple alentours de gares


pub TOD in Zhuhai near BeiZhan -- develop on that // in the fieldwork report








%%%%%%%%%%%%%%%%%%%%%
%\subsection{Interviews}{Entretiens}
% chaud pour interviews






