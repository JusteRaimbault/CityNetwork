



%----------------------------------------------------------------------------------------

\newpage


\section*{Chapter Conclusion}{Conclusion du Chapitre}



Une question générale relativement ouverte concernant les systèmes urbains et celle du \emph{lien entre forme et fonction}. Si, dans certains cas et à certaines échelles, celui-ci est aisément extricable, il ne semble pas exister de règle générale ni de théorie répondant à ce problème fondamental. Les futures villes intelligentes seront-elles capables de totalement déconnecter la forme de la fonction comme le suppose~\cite{batty2017age} ?

Si on se place à l'échelle d'un système de ville ou d'une méga-région urbaine, pour lesquels la forme se manifestera dans les positions relatives à la fois géographique, mais aussi selon des réseaux multi-couches, des villes selon leur spécialisations, ou dans la localisation fine des différents types d'activité dans la région et les liens formés par le réseau de transport, on peut supposer au contraire que les nouvelles formes urbaines seront liées de manière toujours plus intriquées et complexes avec leurs fonctions, à différentes échelles et selon différentes dimensions.

La notion de morphogenèse, que nous avons définie et explorée partiellement, semble être bonne candidate pour lier forme et fonction puisque cette hypothèse fait partie intégrante de sa définition construite en~\ref{sec:interdiscmorphogenesis}. Un modèle simple comme celui étudié en~\ref{sec:densitygeneration} intègre ce paradigme sans pouvoir offrir d'interprétation possible puisque les fonctions sont implicites dans les processus considérés. En couplant le modèle au réseau de transport comme fait en~\ref{sec:correlatedsyntheticdata}, on introduit explicitement des notions de fonctions puisque par exemple l'accessibilité se met à jouer un rôle, mais aussi parce que le réseau est une fonction en lui-même.

Ces paradigmes seront utilisés par la suite pour modéliser la co-évolution dans une perspective correspondante en~\ref{sec:mesocoevolmodel}, c'est à dire à l'échelle mesoscopique avec les mêmes hypothèses de processus autonomes et de sous-système bien défini. On poussera la reflexion du rôle des fonctions et d'une forme urbaine multi-dimensionnelle dans l'étude du modèle Lutecia en~\ref{sec:lutecia}, qui intègrera la gouvernance du système de transport et les relations entre actifs et emplois dans une région métropolitaine.




\stars



