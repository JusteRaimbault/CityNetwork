

%----------------------------------------------------------------------------------------

\newpage

\section*{Synthesis of modeled processes}{Synthèse des processus modélisés}



Nous proposons de synthétiser les processus pris en compte par les modèles parcourus lors de la modélographie, afin de procéder à un effort similaire à celui concluant l'approche thématique du chapitre~\ref{ch:thematic}. Nous ne pouvons ni avoir une vision exhaustive (comme déjà précisé lors de la description de la méthodologie de la modélographie) ni rendre compte avec grande précision de chaque modèle en détail, puisque quasiment chacun est unique dans son ontologie. L'exercice de synthèse permet ainsi de s'extraire de ces limites et prendre un certain recul, et avoir ainsi un aperçu sur les \emph{processus modélisés}\footnote{En gardant en tête les choix de selection, qui emmènent par exemple à ne pas avoir les processus de mobilité dans cette synthèse.}.



%%%%%%%%%%%%%
\begin{table}%[h!]
%\rotatebox{90}{
\caption[Synthesis of processes included in models][Synthèse des processus modélisés]{\textbf{Synthesis of processes included in models.}\label{tab:modelography:processes}}{\textbf{Synthèse des processus modélisés.} Ceux-ci sont classés par échelle, type de modèle et discipline. \label{tab:modelography:processes}}
\begin{tabular}{|l|p{5cm}|p{5cm}|p{5cm}|}
\hline
 & Réseaux $\rightarrow$ Territoires & Territoires $\rightarrow$ Réseaux & Réseaux $\leftrightarrow$ Territoires\\ \hline
\multirow{2}{*}{Micro} &
\textbf{Economie : } marché immobilier, relocalisation, marché de l'emploi & NA & \textbf{Informatique : } croissance spontanée \\\cline{2-2}
& \textbf{Planification : } régulations, développement & & \\\hline
& \textbf{Economie : } marché immobilier, coût du transport, aménités & \textbf{Economie : } croissance du réseau, offre et demande & \textbf{Economie : } investissements, relocalisations, offre et demande, planification du réseau\\\cline{2-4}
\multirow{2}{*}{Meso}& \textbf{Géographie : } usage du sol, centralité, étalement urbain, effets de réseau & \textbf{Transports : } investissements, niveau de gouvernance & \textbf{Géographie : } usage du sol, croissance du réseau, diffusion de population \\\cline{2-3}
& \textbf{Planification/transports : } accessibilité, usage du sol, relocalisation, marché immobilier & \textbf{Physique : } corrélations topologiques, hiérarchie, congestion, optimisation locale, maintenance du réseau & \\\hline
& \textbf{Economie : } croissance économique, marché, usage du sol, agglomération, dispersion, compétition & \textbf{Economie : } interactions entre villes, investissements & \textbf{Economie : } offre et demande \\ \cline{2-4}
\multirow{2}{*}{Macro} & \textbf{Géographie : } accessibilité, interaction entre villes, relocalisation, histoire politique & \textbf{Géographie : } interactions entre villes, rupture de potentiel & \textbf{Transports : } couverture du réseau \\\cline{2-3}
& \textbf{Transports : } accessibilité, marché immobilier & \textbf{Transports : } planification de réseau & \\\hline
\end{tabular}
%}
\end{table}
%%%%%%%%%%%%%

% \makecell{\textbf{Physique : } Corrélations topologiques, hiérarchie, congestion, optimisation locale, maintenance du réseau\\\textbf{Transports : } Investissements, Niveau de gouvernance\\\textbf{Economie : } Croissance du réseau, offre et demande }

% Micro & Motifs de mobilité & Congestion du réseau ; Externalités négatives & Mobilité et structure sociale 
% Meso & Relocalisations ; Effets locaux des infrastructures & Rupture de potentiel & Planification métropolitaine ; TOD
% Macro & Interactions entre villes ; Effet tunnel & Différenciation hiérarchique de l'accessibilité &

%\multirow{3}{*}{Territoires $\rightarrow$}& Economie des Réseaux & Moyenne & Mesoscopique & Explication & Rôle de processus économiques & Economie, Gouvernance\\\cline{2-7}


La table~\ref{tab:modelography:processes} propose cette synthèse à partir des 145 articles issus de la modélographie et pour lesquels une classification de type était possible, c'est-à-dire qu'il existait un modèle rentrant dans la typologie développée en~\ref{sec:modelingsa}. Être complètement exhaustif relèverait d'une opération de méta-modélisation interdisciplinaire qui est bien hors de la portée de notre travail\footnote{Il s'agirait pour cela d'avoir des correspondances entre les ontologies, sans lesquelles on se retrouverait avec au moins autant de processus que de modèles, même au sein d'une discipline. Il n'en existe à notre connaissance pas entre deux disciplines seulement. Une piste pour une approche formelle est donnée en~\ref{app:sec:csframework}.}, et la liste donnée ici est indicative.


Nous retrouvons les correspondances entre disciplines, échelles et types de modèles obtenues dans la modélographie en~\ref{sec:modelography}. Nous retirons les enseignement principaux suivants, en écho au tableau de synthèse obtenu en fin du Chapitre~\ref{ch:thematic} (Table~\ref{tab:thematic:processes}) :
\begin{enumerate}
	\item La dichotomie des ontologies et des processus pris en compte entre les échelles et entre les types est autant manifeste ici dans les modèles que dans les processus en eux-même\footnote{Puisqu'on a plus détaillé cette étude, elle parait même plus forte aussi, une plus grande précision permettant alors de séparer des catégories abstraites.} Nous postulons qu'il existe bien des processus différents aux différentes échelles, et nous prendrons le parti d'étudier différentes échelles.
	\item Le cloisonnement des disciplines démontré en~\ref{sec:quantepistemo} se retrouve qualitativement dans cette synthèse : il est évident qu'elles divergent originellement dans leurs différentes épistémologies fondatrices. Nous tacherons d'intégrer des paradigmes de différentes disciplines, tout en prenant en compte les limites imposées par les principes de modélisation que nous présenterons en~\ref{sec:computation} (par exemple, la parcimonie des modèles limite nécessairement l'intégration d'ontologies hétérogènes).
	\item Un décalage important entre cette synthèse et celle des processus est la quasi absence ici de modèles intégrant des processus de gouvernance. Il s'agira d'une piste à explorer.
	\item Au contraire, une très bonne correspondance s'établit entre les modèles géographiques des systèmes urbains et les positionnement théoriques de la théorie évolutive des villes. Cette adéquation, plus difficile à retrouver pour l'ensemble des autres approches revues, nous suggère également de suivre cette piste.
\end{enumerate}






%----------------------------------------------------------------------------------------

\newpage


\section*{Chapter Conclusion}{Conclusion du Chapitre}

%La réflexivité, au sens de la reflexion de la recherche sur les facteurs influençant son contenu et sa propre structure, semble dans notre cas être nécessaire pour une appréhension claire des enjeux thématiques, méthodologiques et plus généralement scientifiques liés au

Les processus que nous cherchons à modéliser étant multi-scalaires, hybrides et hétérogènes, les angles d'approches et questionnements possibles sont nécessairement extrêmement variés, complémentaires et riches. Il pourrait s'agir d'une caractéristique fondamentale des systèmes socio-techniques, que \noun{Pumain} formule dans~\cite{pumain2005cumulativite} comme ``une nouvelle mesure de complexité'', qui serait liée aux nombre de point de vue nécessaires pour appréhender un système à un niveau donné d'exhaustivité. Cette idée rejoint la position de \emph{perspectivisme appliqué} que~\ref{app:sec:csframework} formalise et qui est implicitement présente dans l'investigation des relations entre économie et géographie développée en~\ref{app:sec:ecogeo}. Ainsi, la modélisation des interactions entre réseaux et territoires peut être reliées à un ensemble très large de disciplines et d'approches revues en section~\ref{sec:modelingsa}.

Afin de mieux comprendre le paysage scientifique environnant, et quantifier les rôles ou poids relatifs de chacune, nous avons procédé à une série d'analyse en épistémologie quantitative en~\ref{sec:quantepistemo}. Une première analyse préliminaire basée sur une revue systématique algorithmique suggère un certain cloisonnement des domaines. Cette conclusion est confirmée par l'analyse d'hyperréseau couplant réseau de citations et réseau sémantique, qui permet également de dessiner plus finement les contours disciplinaires, à la fois sur leur relations directes (citations) mais aussi leur proximité scientifique pour les termes et méthodes utilisées. On peut alors utiliser le corpus constitué et cette connaissance des domaines pour une revue systématique semi-automatique en~\ref{sec:modelography}, qui permet de constituer un corpus de travaux traitant directement du sujet, qui est ensuite inspecté intégralement, permettant de lier caractéristique des modèles au différents domaines. Nous avons alors à ce stade une idée assez précise de ce qui se fait, pourquoi et comment.


L'enjeu reste de déterminer les pertinences relatives de certaines approches ou ontologies, ce qui sera le but des deux chapitres de la deuxième partie. Nous concluons d'abord cette première partie par un chapitre de discussion~\ref{ch:positioning}, éclairant des points nécessaires à clarifier avant une entrée dans le vif du sujet.






\stars
