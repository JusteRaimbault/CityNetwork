


%----------------------------------------------------------------------------------------

\newpage


\section*{Chapter Conclusion}{Conclusion du Chapitre}

La réflexivité semble dans notre cas être nécessaire pour une appréhension claire des enjeux thématiques, méthodologiques et plus généralement scientifiques liés au processus que nous cherchons à modéliser : ceux-ci étant multi-scalaires, hybrides et hétérogènes, les angles d'approches et questionnements possibles sont nécessairement extrêmement variés, complémentaires et riche. Il pourrait s'agir d'une caractéristique fondamentale des systèmes socio-techniques, que \noun{Pumain} formule dans~\cite{pumain2005cumulativite} comme ``une nouvelle mesure de complexité'', qui serait liée aux nombre de point de vue nécessaires pour appréhender un système à un niveau donné d'exhaustivité. Cette idée rejoint la position de \emph{perspectivisme appliqué} que la section~\ref{sec:csframework} formalise et qui est implicitement présente dans l'investigation des relations entre Economie et Géographie développée en~\ref{app:sec:ecogeo}. Ainsi, la modélisation des interactions entre réseaux et territoires peut être reliées à un ensemble très large de disciplines et d'approches revues en section~\ref{sec:modelingsa}. Afin de mieux comprendre le paysage scientifique environnant, et quantifier les rôles ou poids relatifs de chacune, nous avons procédé à une série d'analyse en épistémologie quantitative en~\ref{sec:quantepistemo}. Une première analyse préliminaire basée sur une revue systématique algorithmique suggère un certain cloisonnement des domaines. Cette conclusion est confirmée par l'analyse d'hyperréseau couplant réseau de citation et réseau sémantique, qui permet également de dessiner plus finement les contours disciplinaires, à la fois sur leur relations directes (citations) mais aussi leur proximité scientifique pour les termes et méthodes utilisées. On peut alors utiliser le corpus constitué et cette connaissance des domaines pour une revue systématique semi-automatique et une meta-analyse en~\ref{sec:modelography}, qui permet de constituer un corpus de travaux traitant directement du sujet, qui est ensuite inspecté intégralement, permettant de lier caractéristique des modèles au différents domaines. On a alors à ce stade une idée assez précise de ce qui ce fait, pourquoi et comment. L'enjeu reste de déterminer les pertinences relatives de certaines approches ou ontologies, ce qui sera le but des trois chapitres de la deuxième partie. Nous concluons d'abord cette première partie par un chapitre de discussion~\ref{ch:positioning}, éclairant des points nécessaires à clarifier avant une entrée dans le vif du sujet.






\stars
