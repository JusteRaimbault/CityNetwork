

%\chapter{Modeling Interactions between Networks and Territories}{Modéliser les Interactions entre Réseaux et Territoires} % Chapter title
\chapter{Modéliser les interactions entre réseaux et territoires}


\label{ch:modelinginteractions}

%----------------------------------------------------------------------------------------




La littérature empirique et thématique, ainsi que les cas d'études développés précédemment, semblent converger vers un consensus sur la complexité des relations entre réseaux de transport et territoires. Dans certaines configurations et à certaines échelles, il est possible de mettre en valeur des relations circulaires causales entre dynamiques territoriales et dynamiques des réseaux de transports. Nous désignons leur existence par le concept de \emph{co-évolution}. Il semble difficile d'introduire des explications simples ou systématiques de ces dynamiques, comme le rappelle par exemple les débats autour des effets structurants des infrastructures~\cite{offner1993effets}.


Par ailleurs, les multiples situations géographiques suggère une forte dépendance au contexte, donnant une pertinence au travail de terrain et aux études ciblées. Or l'explication géographique et la compréhension des processus est très vite limitée dans cette approche, et intervient un besoin d'un certain niveau de généralisation. C'est sur un tel point que la Théorie Evolutive des Villes se concentre en particulier, puisqu'elle permet de combiner des schémas et modèles généraux aux particularités géographiques. Au contraire, certaines théories issues de la physique s'appliquant à l'étude des systèmes urbains~\cite{west2017scale} peuvent être plus difficiles à accepter pour les géographes de par leur positionnement d'universalité qui est à l'opposé de leurs épistémologies habituelles.


Dans tous les cas, le \emph{medium} qui permet de gagner en généralité sur les processus et structures des systèmes est toujours le modèle (voir \ref{sec:knowledgeframework} pour un développement des domaines de connaissance et du rôle du modèle). Comme le rappelle \noun{J.P. Marchand}\footnote{Communication personnelle, Mai 2017.}, ``\textit{notre génération a compris qu'il y avait une co-évolution, la votre cherche à la comprendre}'', ce qui appuie le pouvoir de compréhension apporté par la modélisation et la simulation que nous jugeons être encore aujourd'hui à très fort potentiel de développement (voir notre positionnement scientifique en~\ref{ch:positioning}).

Sans développer pour le moment les nombreuses fonctions que peut avoir un modèle, nous nous baserons sur la position de \noun{Banos} qui soutient que ``modéliser c'est apprendre'', et suivant notre positionnement dans une science des systèmes complexes suggéré en introduction, nous ferons ainsi de la \emph{modélisation des interactions entre réseaux et territoires} notre principal sujet d'étude, outil, objet\footnote{Même si dans une relecture à la lumière de~\ref{sec:knowledgeframework} ce positionnement n'a pas de sens puisque notre démarche contenait déjà des modèles à partir du moment où elle était scientifique.}. Ce chapitre doit être pris comme un ``état de l'art'' des démarches de modélisation des interactions entre réseaux et territoires. Il vise en particulier à capturer différentes dimensions des connaissances : pour cela, nous mobiliserons des analyses en épistémologie quantitative.


Dans une première section~\ref{sec:modelingsa}, nous passons en revue de manière interdisciplinaire les modèles pouvant être concernés, même de loin, sans a priori d'échelle temporelle ou spatiale, d'ontologies, de structure, ou de contexte d'application. Cet aperçu est possible par les entrées disciplinaires diverses révélées au chapitre précédent : par exemple géographie, géographie des transports, planification. Cet aperçu suggère des structures de connaissances assez indépendantes et des disciplines ne communiquant que rarement.

Nous procédons dans~\ref{sec:quantepistemo} à une revue systématique algorithmique, qui correspond à une reconstruction par exploration itérative d'un paysage scientifique. Ses résultats tendent à confirmer ce cloisonnement. L'étude est complétée par une analyse de réseau multi-couches, combinant réseau de citation et réseau sémantique issu d'analyse textuelle, qui permet de mieux cerner les relations entre disciplines, leur champs lexicaux et leur motifs d'interdisciplinarité.


Cette étude permet la constitution d'un corpus utilisé pour la modélographie (typologie de modèles) et la méta-analayse (caractérisation de cette typologie) effectuée en dernière section~\ref{sec:modelography}. Celle-ci dissèque la nature d'un certain nombre de modèles et la relie au contexte disciplinaire, ce qui pose les bases et le cadre précis des efforts de modélisation qui seront développés par la suite.




%Les modèles de changement d'usage du sol très appliqués en planification sont tout autant concernés que des modèles totalement abstraits issus de la biologie ou de la physique, que des approches intégrées en géographie ou spécifiques en économie\comment[FL]{pas utile ici tu detailleras en 2.1 : ici affirmer les objectifs (quelle information en retirer) et les moyens (par ez meme automatique tu as bien un (des) points d'entrees $\rightarrow$ lesquels et pourquoi?}.



\stars


\textit{Ce chapitre est inédit pour sa première section ; reprend dans sa deuxième section le texte traduit de~\cite{raimbault2015models}, puis pour sa deuxième partie la méthodologie introduite par \cite{raimbault2016indirect} et développée dans~\cite{raimbault2017exploration} ainsi que les outils de \cite{bergeaud2017classifying} ; et enfin est inédit pour sa dernière partie.}





