

%\chapter{Modeling Interactions between Networks and Territories}{Modéliser les Interactions entre Réseaux et Territoires} % Chapter title
\chapter{Modéliser les Interactions entre Réseaux et Territoires}


\label{ch:modelinginteractions}

%----------------------------------------------------------------------------------------




Si la littérature empirique et thématique, ainsi que les cas d'études développés précédemment, semblent converger vers un consensus sur la complexité des relations entre réseaux et territoires, et dans certaines configurations et à certaines échelles de relations circulaires causales entre dynamiques territoriales et dynamiques des réseaux de transports que l'on se proposera de désigner par \emph{co-évolution}, ceux-ci semblent diverger sur toute explication potentiellement simple ou systématique, comme le rappelle par exemple les débats autour des effets structurants des infrastructures~\cite{offner1993effets}. Au contraire, les multiples situations géographiques poussent à privilégier des études ciblées très fortement dépendantes du contexte et du travail de terrain. Or l'explication géographique et la compréhension des processus est très vite limitée dans cette approche, et intervient un besoin d'un certain niveau d'abstraction et de généralisation. C'est sur un tel point que la Théorie Evolutive des Villes est absolument remarquable, puisqu'elle arrive à combiner des schémas et modèles généraux aux particularités géographiques, et en tire même parti, tandis que certaines théories issues de la physique comme la Théorie du Scaling de \noun{West}~\cite{west2017scaling} peuvent être plus difficile à digérer pour les géographes de par leur positionnement d'universalité qui est à l'opposé de leurs épistémologies habituelles. Dans tous les cas, le \emph{medium} qui permet de gagner en généralité sur les processus et structures des systèmes est toujours le \emph{modèle} (voir \ref{sec:knowledgeframework} pour un développement des domaines de connaissance et du rôle du modèle). Comme le rappelle \noun{J.P. Marchand}~\cite{raimbault2017entretiens}, ``notre génération a compris qu'il y avait une co-évolution, la votre cherche à la comprendre'', ce qui appuie le pouvoir de compréhension apporté par la modélisation et la simulation qui pourraient être aujourd'hui à leur balbutiements. Sans développer les innombrables fonctions que peut avoir un modèle, nous nous baserons sur l'adage de \noun{Banos} qui soutient que ``modéliser c'est apprendre'', et suivant notre positionnement dans une science des systèmes complexes suggéré en introduction, nous ferons ainsi de la \emph{modélisation des interactions entre réseaux et territoires} notre principal sujet d'étude, outil, objet (même si dans une lecture rigoureuse de~\ref{sec:knowledgeframework} ce positionnement n'a pas de sens puisque notre démarche contenait déjà des modèles à partir du moment où elle était scientifique). Ce chapitre peut être vu comme un ``état de l'art'' des démarches de modélisation des interactions entre réseaux et territoires, mais vise à être aussi objectif et exhaustif que possible: pour cela, nous mobiliserons des analyses en épistémologie quantitative. Dans une première section~\ref{sec:modelingsa}, nous revoyons de manière interdisciplinaire les modèles pouvant être concernés, même de loin, sans a priori d'échelle temporelle ou spatiale, d'ontologies, de structure, ou de contexte d'application. Les modèles de changement d'usage du sol très appliqués en planification sont tout autant concernés que des modèle totalement abstraits issus de la biologie ou de la physique, que des approches intégrées en géographie ou spécifiques en économie. Cet aperçu suggère des structures de connaissances assez indépendantes et des disciplines ne communiquant que rarement. Nous procédons à une revue systématique algorithmique dans~\ref{sec:quantepistemo} pour reconstruire leur paysage scientifique, dont les résultats tendent à confirmer ce cloisonnement. L'étude est complétée par une analyse d'hyperréseau, combinant réseau de citation et réseau sémantique issu d'analyse textuelle, qui permet de mieux cerner les relations entre disciplines, leur champs lexicaux et leur motifs d'interdisciplinarité. Cette étude permet la constitution du corpus utilisé pour la modélographie et la meta-analayse effectuée en dernière section~\ref{sec:modelography}, qui dissèque la nature d'un certain nombre de modèles et la relie au contexte disciplinaire, ce qui pose les bases et le cadre précis des efforts de modélisation qui seront développés par la suite.






\stars


\textit{Ce chapitre est inédit pour sa première section ; reprend dans sa deuxième section le texte traduit de~\cite{raimbault2015models}, puis pour sa deuxième partie la méthodologie de \cite{raimbault2016indirect}, les outils de \cite{bergeaud2017classifying} et des passages de~\cite{}; et est enfin inédit pour sa dernière partie.}








