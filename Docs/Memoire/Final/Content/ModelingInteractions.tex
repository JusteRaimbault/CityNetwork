

%\chapter{Modeling Interactions between Networks and Territories}{Modéliser les Interactions entre Réseaux et Territoires} % Chapter title
\chapter{Modéliser les Interactions entre Réseaux et Territoires}


\label{ch:modelinginteractions}

%----------------------------------------------------------------------------------------




Si la littérature empirique et thématique, ainsi que les cas d'études développés précédemment\comment[FL]{NB: comme je commence par lire des 2 je ne les ai pas en tete}, semblent converger vers un consensus sur la complexité des relations entre réseaux et territoires\comment[FL]{mots qu'il conviendra d'avoir definis precedemment}, et\comment[FL]{faire deux phrases} dans certaines configurations et à certaines échelles de relations circulaires causales entre dynamiques territoriales et dynamiques des réseaux de transports\comment[FL]{pourquoi dans la premiere partie de la phrase il ny a pas transports et la oui?} que l'on se proposera de désigner par \emph{co-évolution}\comment[FL]{faire une tournure de phrase plus simple pour definir le not coevolution}, ceux-ci semblent diverger sur toute explication potentiellement simple\comment[FL]{mal dit} ou systématique, comme le rappelle par exemple les débats autour des effets structurants des infrastructures~\cite{offner1993effets}\comment[FL]{qu'il convient ici d'expliciter en 2.3 - $\phi$ : qui dit quoi dans ce debat ?}. Au contraire, les multiples situations géographiques poussent à privilégier des études ciblées très fortement dépendantes du contexte et du travail de terrain\comment[FL]{phrase un peu rapide tu donnes l'impression que tu tranches le debat}. Or l'explication géographique et la compréhension des processus est très vite limitée dans cette approche, et intervient un besoin d'un certain niveau d'abstraction et de généralisation\comment[FL]{or justement il convient d'expliciter ce debat theorie vs empirisme (ou autre)$\rightarrow$dans la phrase en pointilles tu parles d'un besoin d'abstraction, ce nest pas le terme scientifique}. C'est sur un tel point que la Théorie Evolutive des Villes se concentre particulièrement\comment[FL]{il faut absolument se departir d'un ton enthousiaste non scientifique. tu te proposes d'appliquer une theorie/un cadre analytique a une question - ca marche plus ou moins, cest tout, il ne doit pas y avoir d'a priori ou de preference de ton cote}, puisqu'elle arrive à combiner des schémas et modèles généraux aux particularités géographiques, et en tire même parti, tandis\comment[FL]{faire deux phrases} que certaines théories issues de la physique comme la Théorie du Scaling de \noun{West}~\cite{west2017scaling} peuvent être plus difficile à digérer pour les géographes\comment[FL]{tu pars deja dans de l'interpretation epistemo : il faut séparer les choses, d'abord de quoi s'agit-il ?} de par leur positionnement d'universalité qui est à l'opposé de leurs épistémologies habituelles. Dans tous les cas, le \emph{medium} qui permet de gagner en généralité sur les processus et structures des systèmes est toujours le \emph{modèle}\comment[FL]{l'italique ne joue pas le meme role pour les 2 mots cela prete a confusion} (voir \ref{sec:knowledgeframework} pour un développement des domaines de connaissance et du rôle du modèle). Comme le rappelle \noun{J.P. Marchand}~\cite{raimbault2017entretiens}, ``notre génération a compris qu'il y avait une co-évolution, la votre cherche à la comprendre''\comment[FL]{TB citation a mettre en italique}, ce qui appuie le pouvoir de compréhension apporté par la modélisation et la simulation qui pourraient être aujourd'hui à leur balbutiements\comment[FL]{ok avec le sens de cette phrase mais le conditionnel sonne mou. si tu as des arguments avance les, sinon dis clairement auelle est ta position}. Sans développer les innombrables fonctions que peut avoir un modèle, nous nous baserons sur l'adage\comment[FL]{terme non scientifique} de \noun{Banos} qui soutient que ``modéliser c'est apprendre'', et suivant notre positionnement dans une science des systèmes complexes suggéré en introduction, nous ferons ainsi de la \emph{modélisation des interactions entre réseaux et territoires} notre principal sujet d'étude, outil, objet (même si dans une lecture rigoureuse de~\ref{sec:knowledgeframework} ce positionnement n'a pas de sens puisque notre démarche contenait déjà des modèles à partir du moment où elle était scientifique\comment[FL]{tu ne peux pas parler d'une lecture rigoureuse de tes propres propos ! supprime la ()}). Ce chapitre peut être vu comme un ``état de l'art'' des démarches de modélisation des interactions entre réseaux et territoires. Il vise en particulier à être aussi objectif et exhaustif que possible\comment[FL]{mal amene, cest souvent le but non ?} : pour cela, nous mobiliserons des analyses en épistémologie quantitative. Dans une première section~\ref{sec:modelingsa}, nous passons en revue de manière interdisciplinaire les modèles pouvant être concernés, même de loin, sans a priori d'échelle temporelle ou spatiale, d'ontologies, de structure, ou de contexte d'application. Les modèles de changement d'usage du sol très appliqués en planification sont tout autant concernés que des modèles totalement abstraits issus de la biologie ou de la physique, que des approches intégrées en géographie ou spécifiques en économie\comment[FL]{pas utile ici tu detailleras en 2.1 : ici affirmer les objectifs (quelle information en retirer) et les moyens (par ez meme automatique tu as bien un (des) points d'entrees $\rightarrow$ lesquels et pourquoi?}. Cet aperçu suggère des structures de connaissances assez indépendantes et des disciplines ne communiquant que rarement\comment[FL]{B}. Nous procédons à une revue systématique algorithmique\comment[FL]{est-ce une facon standard de nommer cela ?}[(JR) l'exploration iterative de la facon dont elle est faite n'a jamais ete faite a ma connaissance, j'introduis donc une ``nouvelle'' façon de faire.] dans~\ref{sec:quantepistemo} pour reconstruire leur paysage scientifique, dont les résultats tendent à confirmer ce cloisonnement\comment[FL]{c'est interessant bien sur mais il faut dire pourquoi tu vises cela. a mon avis il vaut mieux commencer par des exemples de manieres dont la litterature sci. prend en compte ces interactions, dans differentes disciplines, avant d'atteindre l'epistemologie quantitative.}. L'étude est complétée par une analyse d'hyperréseau, combinant réseau de citation et réseau sémantique issu d'analyse textuelle, qui permet de mieux cerner les relations entre disciplines, leur champs lexicaux et leur motifs d'interdisciplinarité. Cette étude permet la constitution du corpus utilisé pour la modélographie et la meta-analayse\comment[FL]{mots qui nont pas encore ete introduits} effectuée en dernière section~\ref{sec:modelography}, qui dissèque la nature d'un certain nombre de modèles et la relie au contexte disciplinaire, ce qui pose les bases et le cadre précis des efforts de modélisation qui seront développés par la suite.






\stars


\textit{Ce chapitre est inédit pour sa première section ; reprend dans sa deuxième section le texte traduit de~\cite{raimbault2015models}, puis pour sa deuxième partie la méthodologie de \cite{raimbault2016indirect}, les outils de \cite{bergeaud2017classifying} et des passages de~\cite{}; et est enfin inédit pour sa dernière partie.}

\comment[FL]{mettre bout a bout tous ces passages (non classiques, mais que je trouve beinvenus, en toute fin d'intro generale)}[à rediscuter, je trouve ca plus adapté pour chaque chapitre dans le cadre d'une ``these a papiers'' (meme si c'est est pas une officiellement).]






