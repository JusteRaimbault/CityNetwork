





%\chapter*{Part IV Conclusion}{Conclusion de la Partie IV}
\chapter*{Conclusion de la Partie IV}


% to have header for non-numbered introduction
\markboth{Conclusion}{Conclusion}


%\headercit{}{}{}


%Perspectives pour la co-évolution : recensement systematique des données, benchamrks plus systématiques des modèles ; etc ; exploration des modeles ; plus d'interdisc ; communication etc : justifie ces cadres.



% - complexité disciplinaire des objets pour la co-evol ? physique \subset bio \subset socio / eco \subset geo 
% - difficulté de la reflexivité : lien avec meta-modeling (theorie intégrative ?)
% - besoin de plus d'interdisciplinarité, de communication
% - Vers une géographie intégrée ?
% !! ne pas repeter ce qui sera dit en ouverture




Cette partie a donc permis de placer notre travail dans un cadre plus global et de gagner en réflexivité. Un premier chapitre (\ref{ch:micro}) s'est attelé à explorer empiriquement des manifestations des interactions entre territoires et réseaux de transport à d'autres échelles et suivant d'autres ontologies que celle considérées jusque là pour la co-évolution. A l'échelle microscopique, nous montrons l'aspect non-stationnaire des flux de traffic pour l'Ile-de-France et que l'Equilibre Utilisateur Statique n'est pas vérifié en pratique. Ensuite, en étudiant un aspect économique relevant simultanément des territoires et du réseau routier, à savoir les marchés locaux des prix du carburant, nous montrons l'existence d'échelles endogènes, retrouvons la non-stationnarité des processus, ainsi que la superposition de processus de gouvernance et de processus locaux.

Un second chapitre (\ref{ch:theory}) nous conduit par la suite à une mise en perspective théorique. Après avoir mis en perspective nos contributions, nous articulons de manière théorique la théorie évolutive et la morphogenèse, par l'intermédiaire d'une approche de la co-évolution par les niches écologiques. Enfin, nous introduisons un cadre de connaissance qui est appliqué à notre travail et apporte un gain en réflexivité.



\subsection*{A global take}{Un aperçu global}


Une relecture de la thèse à la lumière de l'articulation théorique proposée en~\ref{sec:theory} nous confirme que (i) l'approche morphogénétique était naturellement induite par la contrainte de niche écologique dans la définition de la co-évolution ; (ii) la théorie évolutive des villes est ainsi précisée pour le cas précis de la co-évolution ; (iii) les systèmes territoriaux doivent intrinsèquement induire de tels processus, puisqu'ils sont à la fois support et objets de ceux-ci. La question de la nécessité des réseaux pour représenter les systèmes territoriaux reste ouverte, et nous l'avons postulée dans notre construction théorique. Nos résultats suggèrent la pertinence de leur prise en compte, et ouvrent la question d'une démonstration de ce postulat.


Ensuite, une relecture par les domaines de connaissance permet de mieux comprendre l'articulation entre les différentes composantes : les constructions conceptuelles et empiriques de la première partie permettent une définition de la co-évolution, puis la mise en place de méthodes et de modèles en deuxième partie, qui en retour alimentent ces autres domaines en troisième partie. Nous proposons une analyse quantitative brève de ces dynamiques en~\ref{app:reflexivity}. Ainsi, l'interdépendance dans le cheminement, donnée par le diagramme en introduction (Encadré~\ref{frame:intro:organisation}), est en fait bien plus complexe et pas forcément linéaire. Une deuxième lecture de notre monographie sera ainsi bien plus riche, par émergence des liens implicites.



\subsection*{Complexity and reflexivity}{Complexité et réflexivité}


Nous ne pouvons toutefois que constater la difficulté d'une posture réflexive, qui est reliée en filigrane à la difficulté d'une vraie interdisciplinarité et de la construction de théories intégratives. Nous préciserons pour conclure des pistes de recherche dans ces directions.




\stars



