
%----------------------------------------------------------------------------------------

\newpage


\section*{Chapter Conclusion}{Conclusion du Chapitre}


Cette entrée macroscopique dans les processus de co-évolution visait à les comprendre (i) au sein d'un système de villes, c'est à dire de manière agrégée et à un niveau abstrait ; et (ii) sur une échelle temporelle longue, de l'ordre de la centaine d'années. Les processus considérés sont : croissance des villes entrainée par les interactions qui dépendent du réseau, effet de flux au second ordre sur ces croissances (que nous n'avons pas exploré ici), effet de retroaction des flux sur les distance dans le réseau de manière seuillée (ce dernier étant raffiné avec un effet de la topologie du réseau dans le cas de SimpopNet). Nous démontrons dans un premier temps, par exploration systématique du modèle SimpopNet, que celui-ci est très sensible à la configuration spatiale, suggérant que les conclusions potentielles sur des processus devront toujours être contextualisées. Nous montrons également que celui-ci ne produit pas à proprement parler de co-évolution au sens de circularités causales entre réseau et villes, mais plutôt d'une adaptation des villes au réseau. Notre modèle exploré par la suite permet quant à lui, au prix d'une abstraction du réseau, de révéler de manière synthétique d'une part une échelle intermédiaire de complexité maximale suggérant l'émergence de sous-systèmes régionaux, permis par des valeurs intermédiaire de la distance d'interaction et des valeurs fortes du seuil de rétroaction pour le réseau ; d'autre part l'existence d'au moins trois régimes de causalité, dont deux comprennent des causalités circulaires. L'étude des données réelles pour le système de ville français confirme bien l'existence de l'échelle régionale, ainsi que d'une échelle temporelle de stationnarité courte d'une vingtaine d'années, mais très peu de liens significatifs à celle-ci, en contradiction avec la littérature existante. La calibration du modèle sur données réelles reproduit bien les motifs connus de croissance du réseau ferré, et révèlent un ``effet TGV'' plus récemment, en restant modeste sur la portée des conclusions vu la faible qualité de la calibration. Nous suggérons un développement avec réseau physique, qui permet de faire le lien avec les ontologies que nous allons explorer par la suite en chapitre~\ref{ch:mesocoevolution} : la co-évolution à l'échelle mesoscopique, en appuyant sur le rôle de la forme et de la fonction, et donc des mécanismes précis de développement du réseau.
\comment[AB]{mieux mettre en valeur}



\stars
