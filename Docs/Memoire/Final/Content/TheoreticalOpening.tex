


% Chapter 

%\chapter{Theoretical Framework}{Cadre Théorique} % Chapter title
\chapter{Cadre Théorique}

% opening is not a good formulation ?
% this part should make a "meta-synthesis" (include here theories ? and research project for the future - lines of research and fundamental questions)

\label{ch:theory} % For referencing the chapter elsewhere, use \autoref{ch:name} 

%----------------------------------------------------------------------------------------



%\headercit{Your theory is crazy, but not enough to be true.\comment{(Florent) rigolo mais le rapport avec le sujet est discutable}}{Niels Bohr}{}
% shitty citation, find a better one.

\bigskip


\bpar{
Theory is a key element of any scientific construction, especially in Human Sciences in which object definition and questioning are more open but also determining for research directions. We develop in this chapter a self-consistent theoretical background. It naturally emerges from thematic considerations of previous chapter, empirical explorations done in chapter~\ref{ch:empirical} and modeling experiments conducted in chapter~\ref{ch:modeling}, as a linear structure of knowledge is not appropriate to translate the type of scientific entreprise we are conducting, typically in the spirit of \noun{Sanders} in~\cite{livet2010} for which the simultaneous conjonction of empirical, conceptual and modeling domains is necessary for the emergence of knowledge. This theoretical construction is however presented to be understood independently, and is used as a structuring skeleton for the rest of the thesis.
}{
La théorie est un élément essentiel de toute construction scientifique, en particulier en Sciences Humaines pour lesquelles la définition des objets et questions de recherche sont plus ouverts mais aussi plus déterminants des directions de recherche alors prises. \comment{(Florent) pourquoi commencer comme cela ? c'est très logique je pense si tu présentes ta thèse comme la recherche d'uen théorie unifiée, mais c'est probablement un objectif difficile à atteindre}
 Nous développons dans ce chapitre un cadre théorique autonome. Il émerge naturellement des considérations thématiques du chapitre précédent, des explorations empiriques faites dans le chapitre~\ref{ch:empirical} et des experiences de modélisation conduites dans le chapitre~\ref{ch:modeling}
}


\bpar{
We propose first to construct the \emph{geographical theory} that will pose the studied objects and their meaning in the real world (their ontology), with their interrelations. This yields precise assumptions that will be sought to be confirmed or proven false in the following. Staying at a thematic level appears however to be not enough to obtain general guidelines on the type of methodologies and the approaches to use. More precisely, even if some theories imply a more natural use of some tools\footnote{to give a rough example, a theory emphasizing the complexity of relations between agents in a system will conduct generally to use agent-based modeling and simulation tools, whereas a theory based on macroscopic equilibrium will favorise the use of exact mathematical derivations.}, at the subtler level of contextualization in the sense of the approach taken to implement the theory (as models or empirical analysis), the freedom of choice may mislead into unappropriated techniques or questionings (see \cite{raimbault2016cautious} on the example of incautious use of big data and computation). We develop therefore in a second section a theoretical framework at a meta-level, aiming to give a vision and framing for modeling socio-technical systems.
}{
Nous proposons d'abord de construire une \emph{Théorie Géographique} \comment{(Florent) qu'est ce que cela veut dire ?} \comment{(Arnaud) pfiou => cadre théorique}
qui fixera les objets étudiés et leur nature réelle (leur ontologie), ainsi que leur interrelations. Celle-ci permettra de produire des hypothèses précises qu'on cherchera à confirmer ou infirmer par la suite. Rester à un niveau thématique apparaît cependant ne pas être suffisant pour obtenir des lignes directrices générales sur le type de méthodologies et d'approches à utiliser. Plus précisément, même si certaines théories impliquent un usage plus naturel de certains outils\footnote{pour donner un exemple basique, une théorie mettant l'emphase sur la complexité des relations entre agents dans un système conduira généralement à utiliser de la modélisation basée agent et des outils de simulation, tandis qu'une théorie basée sur un équilibre macroscopique favorisera l'usage de dérivations mathématiques exactes.}, au niveau plus subtil de la mise en contexte au sens de l'approche prise pour implémenter la théorie (comme modèles ou analyses empiriques), la liberté de choix \comment{(Florent) de qui ? préciser que là tu parles de sciences sociales} peut conduire à l'utilisation de techniques inappropriées ou des questionnements inadaptés (voir \cite{raimbault2016cautious} pour un example sur l'usage inconsidéré des données massives et du calcul). Nous développons pour cela dans une seconde section un cadre théorique à un niveau méta, ayant pour but de donner une vision et un cadre pour la modélisation des systèmes socio-techniques. \comment{(Florent) je ne suis pas sûr de comprendre cette dernière phrase}
}



