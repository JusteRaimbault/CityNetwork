


% Chapter 

%\chapter{Theoretical Framework}{Cadre Théorique} % Chapter title
\chapter{Cadre Théorique}


\label{ch:theory} % For referencing the chapter elsewhere, use \autoref{ch:name} 

%----------------------------------------------------------------------------------------


%\headercit{}{}{}


\bigskip


\bpar{
Theory is a key element of any scientific construction, especially in Human Sciences in which object definition and questioning are more open but also determining for research directions. We develop in this chapter a self-consistent theoretical background. It naturally emerges from thematic considerations of previous chapter, empirical explorations done in chapter~\ref{ch:empirical} and modeling experiments conducted in chapter~\ref{ch:modeling}, as a linear structure of knowledge is not appropriate to translate the type of scientific entreprise we are conducting, typically in the spirit of \noun{Sanders} in~\cite{livet2010} for which the simultaneous conjonction of empirical, conceptual and modeling domains is necessary for the emergence of knowledge. This theoretical construction is however presented to be understood independently, and is used as a structuring skeleton for the rest of the thesis.
}{
La théorie est un élément essentiel de toute construction scientifique, en particulier en Sciences Humaines pour lesquelles la définition des objets et questions de recherche sont plus ouverts mais aussi plus déterminants des directions de recherche alors prises. L'esprit de notre travail n'est pas de produire une théorie unifiée, mais des pistes pour des \emph{Théories Intégrées}, c'est à dire s'appuyant sur une intégration horizontale et verticale au sens de la feuille de route~\cite{2009arXiv0907.2221B}, mais aussi permettant une intégration des domaines de connaissance et une réflexivité, au sens qui seront précisés en section~\ref{sec:knowledgeframework}. Nous développons dans ce chapitre un cadre théorique à plusieurs niveaux. Il émerge naturellement de l'interaction des différentes composantes de la connaissance développées jusqu'ici. Dans sa partie thématique, il s'agit donc d'une clarification et unification d'hypothèse ainsi que de conclusions éparses.
}


\bpar{
We propose first to construct the \emph{geographical theory} that will pose the studied objects and their meaning in the real world (their ontology), with their interrelations. This yields precise assumptions that will be sought to be confirmed or proven false in the following. Staying at a thematic level appears however to be not enough to obtain general guidelines on the type of methodologies and the approaches to use. More precisely, even if some theories imply a more natural use of some tools\footnote{to give a rough example, a theory emphasizing the complexity of relations between agents in a system will conduct generally to use agent-based modeling and simulation tools, whereas a theory based on macroscopic equilibrium will favorise the use of exact mathematical derivations.}, at the subtler level of contextualization in the sense of the approach taken to implement the theory (as models or empirical analysis), the freedom of choice may mislead into unappropriated techniques or questionings (see \cite{raimbault2016cautious} on the example of incautious use of big data and computation). We develop therefore in a second section a theoretical framework at a meta-level, aiming to give a vision and framing for modeling socio-technical systems.
}{
Nous proposons d'abord de construire une \emph{Théorie Géographique}, en quelque sorte un cadre théorique même si nous postulons qu'une Théorie propre a une plus grande portée de par son intégration forte avec les autres domaines de connaissance, qui fixera les objets étudiés et leur nature réelle (leur ontologie), ainsi que leur interrelations. Celle-ci permettra de produire des hypothèses précises qu'on cherchera à confirmer ou infirmer par la suite. Rester à un niveau thématique apparaît cependant ne pas être suffisant pour obtenir des lignes directrices générales sur le type de méthodologies et d'approches à utiliser. Plus précisément, même si certaines théories impliquent un usage plus naturel de certains outils\footnote{pour donner un exemple basique, une théorie mettant l'emphase sur la complexité des relations entre agents dans un système conduira généralement à utiliser de la modélisation basée agent et des outils de simulation, tandis qu'une théorie basée sur un équilibre macroscopique favorisera l'usage de dérivations mathématiques exactes.}, au niveau plus subtil de la mise en contexte au sens de l'approche prise pour implémenter la théorie (comme modèles ou analyses empiriques), la liberté de choix d'objets et d'approches en sciences sociales peut conduire à l'utilisation de techniques inappropriées ou des questionnements inadaptés (voir la section~\ref{sec:computation} pour l'example de l'usage inconsidéré des données massives et du calcul). Nous développons pour cela dans une seconde section (\ref{sec:csframework}) un cadre théorique à un niveau plus abstrait, visant à formaliser les entreprises de modélisation dans une certaine structure algébrique afin de capturer des articulations fondamentales entre diverses approches. Enfin, nous élaborons dans une dernière section (\ref{sec:knowledgeframework}) un cadre de connaissances appliqué visant à expliciter des processus de production de connaissance sur les systèmes complexes. Celui-ci est illustré par une analyse fine de la genèse de la Théorie Evolutive des Villes, puis est ensuite appliqué de manière réflexive à l'ensemble de notre travail.
}



\bpar{

}{
Ce chapitre sera éventuellement le plus délicat à la lecture, d'une part car il est fortement dépendant de la majorité des points thématiques traités précédemment et devrait être lu progressivement selon les concepts introduits (on touche encore aux limitations de la présentation linéaire), et d'autre part car les constructions théoriques introduites sont à un niveau d'abstraction progressif : en quelque sorte, chaque théorie est un cadre méta pour la précédente. On touche alors la question de la réflexivité, et dans quelle mesure celles-ci peuvent s'appliquer à elles-mêmes, en gardant à l'esprit que la séparation entre les niveaux n'est pas directement évidente : par exemple le cadre formel pour les systèmes socio-techniques pourrait être appliqué comme une formalisation du cadre de connaissances. Dans tous les cas, il faut comprendre la démarche à la fois comme une synthèse et comme une ouverture.
}





\stars

\textit{La première section de ce chapitre reprend un court passage de~\cite{raimbault2017knowledge} ; la deuxième est entièrement inédite. La troisième a été proposée par~\cite{raimbault:halshs-01505084} puis dévelopée et appliquée dans~\cite{raimbault2017knowledge}, et son application réflexive a été présentée par~\cite{raimbault2017co}.}






