



%\chapter*{Part III Introduction}{Introduction de la Partie III}
\chapter*{Introduction de la Partie III}


% to have header for non-numbered introduction
\markboth{Introduction}{Introduction}


%\headercit{}{}{}




\comment{à ce stade, expliquer lien entre les différents modèles : utiliser appendice unified framework urban grwoth}

Les rationelles meso-macro font echo à Gibrat-Simon.

Ontologies : dans le macro, villes fixes, pas de nouvelles ville, mais nouveaux liens de réseau. Meso : tout évolue.







\begin{table}
\begin{tabular}[6pt]{m{4cm}|c|c|c|c}
Processus & Analyse Empirique & Echelles & Type \textit{find a typology of processes} & Modèle \\\hline
Attachement préférentiel & & Croissance Urbaine & & \\\hline
Diffusion/Etalement & & Forme Urbaine & &\\\hline
Accessibilité  & & Réseau / Ville & & \\\hline
Gouvernance des Transports & & & &\\\hline
Flux direct  & & & &\\\hline
Flux indirect/Effet tunnel \textit{c'est le même processus, vu sous un angle différent : l'effet tunnel est l'absence de nw feedback} & & & &\\\hline
Centralité de proximité (accessibilité : generalisation) & & & &\\\hline
Centralité de Chemin (correspond aux flux indirect : différents niveaux de généralité / sous-processus-sous-classif ?) & & & &\\\hline
Proximité au réseau & & & &\\\hline
Distance au centre (similar to agrégation ?) & & & RBD &\\\hline
\end{tabular}
\caption{Description des différents processus pris en compte dans les modèles de co-évolution}{}
\end{table}


\comment{faire le même tableau pour les modèles existants : vue plus large de l'ensemble des processus. pour chacun de ces modèles et de nos modèles, lister tous les processus potentiels ; faire une typologie ensuite. Q : typologie différente d'une pure empirique ? a creuser, et peut être intéressant dans le cadre du knowledge framework, comme illustration coevol connaissances.}


\comment{justifier ici poruquoi pas modèle très fins sur processus eco par exemple (//Levinson) : prix à payer pour être accross scales, disciplines et avoir vraiment de la coevol ? pour ces premières étapes oui. à justifier}

\paragraph{}{Echelles et processus}

Partant des hypothèses tirées des enseignements empiriques et théoriques, on postulera \emph{a priori} que certaines échelles privilégient certains processus, par exemple que la forme urbaine aura une influence au niveaux micro et mesoscopiques, tandis que les motifs émergeant des flux agrégés entre villes au sein d'un système se manifesteront au niveau macroscopique. Toutefois la distinction entre échelles n'est pas toujours si claire et certains processus tels la centralité ou l'accessibilité sont de bons candidats pour jouer un rôle à plusieurs échelles\footnote{on entend ici par ``jouer un rôle'' avoir une autonomie propre à l'échelle correspondante, c'est à dire qu'ils émergent \emph{faiblement}  des niveaux inférieurs.}
 : il s'agira par la modélisation d'également tester ce postulat, par comparaison des processus nécessaires et/ou suffisants dans les familles de modèles à différentes échelles que nous allons mettre en place, en gardant à l'esprit des possibles développements vers des modèles multi-scalaires dans lesquels ces processus intermédiaires joueraient alors un rôle crucial.



We expect to product \emph{models of coevolution}, \comment{(Florent) expliciter la différence avec ce que tu as fait jusque là}
 with the emphasis on processes of coevolution, to directly confront the theory. They will be necessary a flexible family because of the variety of scales and concrete cases we can include and we already began to explore in preliminary studies. Processes already studied can serve either as a thematic bases for a reuse as building bricks in a multi-modeling context, or as methodological tools such as synthetic data generator for synthetic control. Finally, we mean by operational models hybrid models, in the sense of semi-parametrized or semi-calibrated on real datasets or on precise stylized facts extracted from these same datasets. This point is a requirement to obtain a thematic feedback on geographical processes and on theory.













