



%\chapter*{Introduction}{Introduction}

% to have header for non-numbered introduction
%\markboth{Introduction}{Introduction}


%\headercit{}{}{}


\todo{à ce stade, expliquer lien entre les différents modèles : utiliser appendice unified framework urban grwoth}



\todo{Plutôt en Conclusion Partie III : Towards operational Models : what is possible ; what is desirable ; etc.}




\bigskip




\begin{table}
\begin{tabular}[6pt]{c|c|c|c|c}
Processus & Analyse Empirique & Echelles & Type \textit{find a typology of processes} & Modèle \\\hline
Attachement préférentiel & & Croissance Urbaine & & \\\hline
Diffusion/Etalement & & Forme Urbaine & &\\\hline
Accessibilité  & & Réseau / Ville & & \\\hline
Gouvernance des Transports & & & &\\\hline
Flux direct  & & & &\\\hline
Flux indirect/Effet tunnel \textit{c'est le même processus, vu sous un angle différent : l'effet tunnel est l'absence de nw feedback} & & & &\\\hline
Centralité de proximité (accessibilité : generalisation) & & & &\\\hline
Centralité de Chemin (correspond aux flux indirect : différents niveaux de généralité / sous-processus-sous-classif ?) & & & &\\\hline
Proximité au réseau & & & &\\\hline
Distance au centre (similar to agrégation ?) & & & RBD &\\\hline
\end{tabular}
\caption{Description des différents processus pris en compte dans les modèles de co-évolution}{}
\end{table}


\comment{faire le même tableau pour les modèles existants : vue plus large de l'ensemble des processus. pour chacun de ces modèles et de nos modèles, lister tous les processus potentiels ; faire une typologie ensuite. Q : typologie différente d'une pure empirique ? a creuser, et peut être intéressant dans le cadre du knowledge framework, comme illustration coevol connaissances.}




\paragraph{}{Echelles et processus}

Partant des hypothèses tirées des enseignements empiriques et théoriques, on postulera \emph{a priori} que certaines échelles privilégient certains processus, par exemple que la forme urbaine aura une influence au niveaux micro et mesoscopiques, tandis que les motifs émergeant des flux agrégés entre villes au sein d'un système se manifesteront au niveau macroscopique. Toutefois la distinction entre échelles n'est pas toujours si claire et certains processus tels la centralité ou l'accessibilité sont de bons candidats pour jouer un rôle à plusieurs échelles\footnote{on entend ici par ``jouer un rôle'' avoir une autonomie propre à l'échelle correspondante, c'est à dire qu'ils émergent \emph{faiblement}  des niveaux inférieurs.}% TODO somewhere clarification and discussion on definitions of emergence, Bedau etc.
 : il s'agira par la modélisation d'également tester ce postulat, par comparaison des processus nécessaires et/ou suffisants dans les familles de modèles à différentes échelles que nous allons mettre en place, en gardant à l'esprit des possibles développements vers des modèles multi-scalaires dans lesquels ces processus intermédiaires joueraient alors un rôle crucial. % TODO : multi-scale modeling : who does some and how much ?








% introduction de la partie 

\section*{A Roadmap for an Operational Family of Models of Coevolution}{Vers des Modèles Opérationnels de Coevolution} % Chapter title

\label{ch:operational}

%----------------------------------------------------------------------------------------

As previously stated, one of our principal aims is the validation of the network necessity assumption, that is the differentiating point with a classic evolutive urban theory. To do so, toy-model exploration and empirical analysis will not be enough as hybrid models are generally necessary to draw effective and well validated conclusions. We briefly give an overview of planned work in the following, that will be the conclusion of this Memoire.



%----------------------------------------------------------------------------------------



\section*{Objectives}{Objectifs}


We expect to product \emph{models of coevolution}, \comment{(Florent) expliciter la différence avec ce que tu as fait jusque là}
 with the emphasis on processes of coevolution, to directly confront the theory. They will be necessary a flexible family because of the variety of scales and concrete cases we can include and we already began to explore in preliminary studies. Processes already studied can serve either as a thematic bases for a reuse as building bricks in a multi-modeling context, or as methodological tools such as synthetic data generator for synthetic control. Finally, we mean by operational models hybrid models, in the sense of semi-parametrized or semi-calibrated on real datasets or on precise stylized facts extracted from these same datasets. This point is a requirement to obtain a thematic feedback on geographical processes and on theory.


%----------------------------------------------------------------------------------------

\section*{Case Studies}{Cas d'étude}

Currently we expect to work on the following case studies to build these hybrid models :

\begin{itemize}
\item Dynamical data for Bassin Parisien should allow to parametrize and calibrate a model at this temporal and spatial scale.
\item On larger scales, South African dataset of \noun{Baffi} will along empirical analysis also be used to parametrize hybrid co-evolution models.
\item A possibility that is not currently set up (and that may however be difficult because of a disturbing closed-data policy among a frightening large number of scientists !) is the exploitation of French railway growth dataset (with population dataset) used in~\cite{bretagnolle:tel-00459720}, that would also provide an interesting case study on other regimes, scales and transportation mode.
\end{itemize}



%----------------------------------------------------------------------------------------



\section*{Roadmap}{Feuille de Route}


We give the following (non-exhaustive and provisory) roadmap for modeling explorations (theoretical and empirical domains being still explored conjointly) :

\begin{enumerate}
\item Complete the exploration of independent and weak coupled urban growth and network growth processes (all models presented in chapter~\ref{ch:modeling}), in order to know precisely involved mechanisms when they are virtually isolated, and to obtain morphogenesis scales.
\item Go further into the exploration of toy-model of non conventional processes such as governance network growth heuristic to pave the road for a possible integration of such modules in hybrid models.
\item Build a Marius-like generic infrastructure that implement the theory in a family of models that can be declined into diverse case studies.
\item Launch it and adapt it on these case studies.
\end{enumerate}

Next steps would be too hypothetical if formulated, we propose thus to proceed iteratively in our construction of knowledge and naturally update this roadmap constantly.

\bigskip
\bigskip
\bigskip

\textit{ - La route est longue mais la voie est libre.}














