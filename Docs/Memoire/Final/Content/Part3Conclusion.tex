





%\chapter*{Part III Conclusion}{Conclusion de la Partie III}
\chapter*{Conclusion de la Partie III : une vue complète de la co-évolution}


% to have header for non-numbered introduction
\markboth{Conclusion}{Conclusion}


%\headercit{}{}{}


Cette partie a ainsi donné des premiers éléments d'exploration de différentes entrées sur la modélisation de la co-évolution. Nous avons exploré dans le chapitre~\ref{ch:macrocoevolution} un modèle de co-évolution à l'échelle macroscopique, qui permet l'isolation de nombreux régimes de causalité, qu'on peut alors nommer régimes de co-évolution pour ceux présentant des causalités circulaires, et qui est calibré sur le système de villes français. Nous montrons ainsi que des mécanismes et une représentation simple permettent déjà de capturer synthétiquement et empiriquement la co-évolution à cette échelle.

Nous avons ensuite exploré des modèles à une échelle plus grande, impliquant une complexité croissante. Un modèle de co-évolution par morphogenèse permet de coupler la forme urbaine (distribution de la population et topologie du réseau) à une abstraction des fonctions urbaines (mesures de centralité et d'accessibilité dans le réseau). Les différentes heuristiques d'évolution du réseau qui ont été testées se révèlent complémentaires pour s'approcher de configurations réelles. Enfin, nous avons introduit des pistes pour la prise en compte des processus de gouvernance dans l'évolution des réseaux de transport.




\subsection*{Processes in models}{Processus modélisés}

%\comment{justifier ici poruquoi pas modèle très fins sur processus eco par exemple (//Levinson) : prix à payer pour être accross scales, disciplines et avoir vraiment de la coevol ? pour ces premières étapes oui. à justifier}

Les modèles que nous avons développés l'ont été dans une logique de parcimonie, tout en cherchant à effectivement capturer des processus de co-évolution à différentes échelles et en s'encrant dans des disciplines variées : ces contraintes se paient par un prix en raffinement des mécanismes intégrés. Nous reviendrons sur ce compromis en~\ref{sec:contributions}.

\subsection*{A full view on co-evolution}{Une vue complète de la co-évolution}

% complementarite de la vision conceptuelle/empirique/modelisation : exemples de conclusions fondamentales / adequations / non-adequations pour chaque

Nous avons à ce stade apporté des éléments de réponse aux deux axes de notre problématique générale (comment définir et caractériser la co-évolution, et comment la modéliser). Il est remarquable de noter que ceux-ci s'articulent dans les trois domaines de connaissance du conceptuel (définition), de l'empirique (caractérisation) et de la modélisation (modèles). Ces trois aspects s'auto-génèrent l'un l'autre, et notre point de vue forme une véritable trinité, c'est-à-dire un concept à la fois unique et triple, dans lequel aucune des approches ne peut être ignorée (de la manière dont le fait \cite{morin2001methode} pour l'anthropologie complexe).

Ainsi, les modèles contiennent l'aspect individuel de la co-évolution (interactions réciproques entre entités), et dans certains cas l'aspect statistique au niveau d'une population. Cette conclusion est rendue possible par l'outil de caractérisation opérationnelle, celui-ci permettant par ailleurs de renforcer la pertinence de la définition.




\subsection*{Perspectives}{Perspectives}

Notre point de vue sur la co-évolution a bien entendu été réducteur et limité, puisque l'état actuel de nos modes de production de connaissance est encore loin d'une intégration paradigmatique de la complexité~\cite{morin1991methode}, et que toute tentative d'appréhension d'un système complexe combine habilement analyse et synthèse, réductionnisme et holisme, modularité et interdépendance. Afin d'enrichir notre point de vue, nous proposons finalement un chapitre d'ouverture.








