





%\chapter*{Part III Conclusion}{Conclusion de la Partie III}
\chapter*{Conclusion de la Partie III}


% to have header for non-numbered introduction
\markboth{Conclusion}{Conclusion}


%\headercit{}{}{}






Towards operational Models : what is possible ; what is desirable ; etc.


\subsubsection*{A Roadmap for an Operational Family of Models of Coevolution}{Vers des Modèles Opérationnels de Coevolution} % Chapter title

%----------------------------------------------------------------------------------------

As previously stated, one of our principal aims is the validation of the network necessity assumption, that is the differentiating point with a classic evolutive urban theory. To do so, toy-model exploration and empirical analysis will not be enough as hybrid models are generally necessary to draw effective and well validated conclusions. We briefly give an overview of planned work in the following, that will be the conclusion of this Memoire.








%----------------------------------------------------------------------------------------

%\subsubsection*{Case Studies}{Cas d'étude}

%$\rightarrow$ potential application cases ?

%Currently we expect to work on the following case studies to build these hybrid models :

%\begin{itemize}
%\item Dynamical data for Bassin Parisien should allow to parametrize and calibrate a model at this temporal and spatial scale.
%\item On larger scales, South African dataset of \noun{Baffi} will along empirical analysis also be used to parametrize hybrid co-evolution models.
%\item A possibility that is not currently set up (and that may however be difficult because of a disturbing closed-data policy among a frightening large number of scientists !) is the exploitation of French railway growth dataset (with population dataset) used in~\cite{bretagnolle:tel-00459720}, that would also provide an interesting case study on other regimes, scales and transportation mode.
%\end{itemize}






