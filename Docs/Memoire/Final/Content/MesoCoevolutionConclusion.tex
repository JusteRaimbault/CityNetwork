
%----------------------------------------------------------------------------------------

\newpage


\section*{Chapter Conclusion}{Conclusion du Chapitre}


\bpar{
This second entry on co-evolution models, at the mesoscopic scale, has been the occasion to explore the coupling between urban form and functions through the coupling between territory and network. In comparison with macroscopic models, processes that are taken here into account are much more varied and complementary.
}{
Cette deuxième entrée sur les modèles de co-évolution, à l'échelle mesoscopique, a été l'occasion d'explorer le couplage entre forme urbaine et fonctions au travers du couplage entre territoire et réseau. En comparaison avec les modèles macroscopiques, les processus pris en compte ici sont beaucoup plus variés et complémentaires.
}


\bpar{
A first morphogenesis model includes different heuristics for network growth, which are necessary and complementary to capture all the possible range of generated network configurations. We show that the model is able to resemble observed situations
}{
Un premier modèle de morphogenèse inclut différentes heuristiques pour la croissance du réseau, qui sont nécessaires et complémentaires pour capturer toute l'étendue possible des configurations de réseau générées. Nous montrons que le modèle est capable de se rapprocher de situation observées, pour la forme territoriale, la topologie du réseau, ainsi que pour les corrélations statiques entre ces indicateurs, tout en nécessitant un compromis entre ces différents objectifs. En termes de régimes de causalité, et donc de capture de dynamiques co-évolutives, le modèle est capable d'en capturer dans certaines situations précises, mais on tire de cette expérience une leçon fondamentale pour les modèles de co-évolution : une fidélité des processus ou des configurations statiques doit se faire au prix de la flexibilité des régimes dynamiques produits. Cela peut être un effet structurel des modèles, ou plus intéressant, une restriction des régimes existants dans les situations réelles.
}


Nous avons ensuite fait le pari d'introduire un modèle plus complexe, incluant une ontologie pour les processus de gouvernance pour l'évolution du réseau de transport. Nous menons des premières expériences de validation du modèle sur données synthétiques, et proposons une application au cas du Delta de la Rivière des Perles, renouvelant le regard que nous en avons apporté en~\ref{sec:casestudies}. Nous montrons par exemple qu'il est possible d'extrapoler des paramètres liés au niveau de collaboration entre acteurs. Cette section permet ainsi d'introduire une nouvelle façon de considérer la co-évolution, prenant en compte l'intégralité du cadre conceptuel développé en~\ref{ch:thematic}, et ouvre également de nombreuses perspectives de recherche.




\stars
