
%----------------------------------------------------------------------------------------

\newpage


\section*{Chapter Conclusion}{Conclusion du Chapitre}


Cette deuxième entrée sur les modèles de co-évolution, à l'échelle mesoscopique, a été l'occasion d'explorer le couplage entre forme urbaine et fonctions au travers du couplage entre territoire et réseau. En comparaison aux modèles macroscopiques, les processus pris en compte ici sont beaucoup plus variés et complémentaires.

Un premier modèle de morphogenèse inclut différentes heuristiques pour la croissance du réseau, qui sont nécessaires et complémentaires pour capturer toute l'étendue possibles des configurations de réseau générées. Nous montrons que le modèle est capable de se rapprocher de situation observées, pour la forme urbaine, le réseau, ainsi que pour les corrélations statiques entre ces indicateurs, tout en nécessitant un compromis entre ces différents objectifs. En termes de régimes de causalité, et donc de capture de dynamiques co-évolutives, le modèle est capable d'en capturer dans certaines situations précises, mais on tire de cette expérience une leçon fondamentale pour les modèles de co-évolution : une fidélité des processus ou des configurations statiques doit se faire au prix de la flexibilité des régimes dynamiques produits. Cela peut être un effet structurel des modèles, ou plus intéressant, une restriction des régimes existants dans les situations réelles. Cela ouvre ainsi des avenues pour la recherche future.


Nous avons ensuite fait le pari d'introduire un modèle plus complexe, incluant une ontologie pour les processus de gouvernance pour l'évolution du réseau de transport. Nous menons des premières expériences de validation du modèle sur données synthétiques, et proposons une application au cas du Delta de la Rivière des Perles, renouvelant le regard que nous en avons apporté en~\ref{sec:casestudies}. Nous montrons par exemple qu'il est possible d'extrapoler des paramètres liés au niveau de collaboration entre acteurs. Cette section permet ainsi d'introduire une nouvelle façon de considérer la co-évolution, prenant en compte l'intégralité du cadre conceptuel développé en~\ref{ch:thematic}, et ouvre également de nombreuses perspectives de recherche.




\stars
