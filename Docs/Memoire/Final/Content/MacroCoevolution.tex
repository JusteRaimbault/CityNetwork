% Chapter 

%\chapter{Co-evolution at the macro-scale}{Co-évolution à l'Echelle Macroscopique}
\chapter{Co-évolution à l'échelle macroscopique}


\label{ch:macrocoevolution} 

%----------------------------------------------------------------------------------------


Les dynamiques couplées des territoires et des réseaux peuvent être appréhendées à l'échelle macroscopique au moyen d'une approche par les interactions, comme nous l'avons montré au chapitre~\ref{ch:evolutiveurban}. Le pouvoir explicatif est alors different de celui des modèles économiques classiques et concerne d'autre types de processus, basés sur les interactions à des échelles d'espace plus grandes et des échelles de temps plus longues. Dans ce cadre, les réseaux de transports et les systèmes de villes co-évoluent sur le temps long.


Dans quelle mesure la construction du lien ferroviaire par le tunnel sous la Manche a-t-elle pu conforter le pouvoir économique de Londres ou renforcer ses interactions avec ses proches voisins Européens, et dans quelle mesure les évènements politiques recents peuvent-ils conduire a une modification des trajectoires économiques puis par conséquent à une modification des motifs de transports par une rétroaction de la demande ? D'une façon similaire, dans quelle mesure les projets de lignes à grande vitesse sur la côte Est des Etats-Unis et dans le corridor Californien sont-ils coordonnés aux dynamiques régionales, et s'ils sont effectivement réalisés, dans quelle mesure peuvent-ils influencer les trajectoires du système de villes ?


Nous avons déjà étudié des questions analogues dans le cas de l'Afrique du Sud de manière empirique en~\ref{sec:causalityregimes}, et nous proposons dans ce chapitre d'y faire écho du point de vue de la modélisation, en introduisant les processus de co-evolution dans les modèles d'interactions déjà développés.


Pour donner une idée de la nature des enseignements qu'il est possible de tirer d'une telle approche, nous commençons en~\ref{sec:macrocoevolexplo} par une exploration systématique du modèle SimpopNet, approche la plus avancée en termes de modélisation de la co-evolution des villes et des réseaux de transport à cette échelle, comme établi au chapitre~\ref{ch:modelinginteractions}. Cela nous permet également d'introduire les indicateurs adaptés pour l'évaluation des trajectoires des systèmes de villes.


Nous décrivons ensuite en~\ref{sec:macrocoevol} le modèle générique de co-évolution, qui est testé sur des données synthétiques à deux niveaux de détail pour la représentation du réseau, puis sur le système de villes français.



\stars


\textit{Ce chapitre va paraître prochainement comme~\cite{raimbault2018unveiling} pour sa première section. La deuxième section reprend les résultats de~\cite{raimbault2017macro} pour les données synthétiques, et va paraître prochainement comme~\cite{raimbault2018models}.
}


%----------------------------------------------------------------------------------------


% Pour rappeler les idées sous-jacentes de manière synthétique, en echo au point de vue par la morphogenèse développé en Chapitre~\ref{ch:morphogenesis} qui au contraire se concentre sur les règles autonomes au sein des sous-systèmes a une échelle intermédiaire\comment[FL]{cela sera pour le chap8 (wrapup)}, le principe dans cette ontologie est de raffiner\comment[FL]{sens} le rôle des interactions en capturant\comment[AB]{suppr. en particulier, nous avons montré qu'il est possible de capturer} les variations propres\comment[AB]{des interactions ? preciser} dans des processus abstraits endogènes simples.










