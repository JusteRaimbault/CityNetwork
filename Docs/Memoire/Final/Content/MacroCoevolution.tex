% Chapter 

%\chapter{Co-evolution at the macro-scale}{Co-évolution à l'Echelle Macroscopique}
\chapter{Co-évolution à l'Echelle Macroscopique}


\label{ch:macrocoevolution} 

%----------------------------------------------------------------------------------------

% chapter introduction


Les dynamiques des systèmes territoriaux à l'échelle macroscopique peuvent être partiellement comprises par une approche par les interactions, comme montré en Chapitre~\ref{ch:evolutiveurban}. Pour rappeler les idées sous-jacentes de manière synthétique, en echo au point de vue par la morphogenèse développé en Chapitre~\ref{ch:morphogenesis} qui au contraire se concentre sur les règles autonomes au sein des sous-systèmes a une échelle intermédiaire, le principe dans cette ontologie est de raffiner le role des interactions en capturant les variations propres dans des processus abstraits endogènes simples. Le pouvoir explicatif est alors different des modèles économiques plus classiques et concerne d'autre types de processus, bases sur les interactions à des échelles d'espace plus grandes et des échelles de temps plus longues. Le role des réseaux de transports dans ce cadre est crucial, comme suggéré par les résultats préliminaires obtenus précédemment. Dans quelle mesure la construction du lien ferroviaire par le tunnel sous la Manche a pu conforter le pouvoir économique de Londres ou renforcer ses interactions avec ses proches voisins Européens, et dans quelle mesure les évènements politiques recents peuvent-il conduire a une modification des trajectoires économiques puis par conséquent a une modification des motifs de transports par une retroaction de la demande ? D'une façon similaire, les projets de lignes à grande vitesse sur la cote Est des Etats-Unis et dans le corridor Californien sont-ils une consequence attendue des dynamiques régionale ou un choix de gouvernance plus difficile a cerner, et s'ils sont realises malgré le contexte politique plus difficile, dans quelle mesure influenceront-ils les trajectoires du système de ville ? Nous avons deja étudié des questions analogues dans le cas de l'Afrique du Sud de manière empirique en~\ref{sec:causalityregimes}, et nous proposons dans ce chapitre d'éclairer celles-ci à un plus grand niveau de généralité par la modélisation, en introduisant les processus de co-evolution dans les modèles d'interactions deja développés. Pour donner une idée de la nature des conclusions qu'il est possible de tirer d'une telle approche, nous commençons en~\ref{sec:macrocoevolexplo} par une exploration systématique du modele SimpopNet, état de l'art en modélisation de la co-evolution au sein des systèmes de villes, comme établi en Chapitre~\ref{ch:modelinginteractions}. Cela permet également d'introduire les indicateurs adaptes pour la comprehension des trajectoires des systèmes de villes en termes de dynamiques co-évolutives. Nous décrivons ensuite en~\ref{sec:macrocoevol} le modèle générique de co-evolution, qui est testé sur données synthétiques à deux niveaux de genericité pour le réseau, puis sur le système de ville français de manière à pouvoir comparer avec les modèles statiques précédemment étudiés.
%Enfin, nous décrivons en~\ref{sec:simpopsino} le cas d'application potentiel au système de ville Chinois et ses enjeux particuliers.




\stars


\textit{Ce chapitre est inédit pour sa premiere section. La deuxième section reprend les résultats de~\cite{} % medium conference
pour les données synthétiques, et va paraitre prochainement de manière synthétique comme~\cite{}. % chapitre Rozenblat
%La dernière section est également inédite.
}


%----------------------------------------------------------------------------------------













