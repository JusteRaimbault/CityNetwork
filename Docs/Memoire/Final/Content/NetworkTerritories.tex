


\newpage


%-------------------------------


\section{Territories and networks}{Territoires et réseaux}

\label{sec:networkterritories}


%-------------------------------


\bpar{
We begin by constructing more precisely the concepts we will use. This construction helps to understand how the concepts of territory and network are rapidly in strong interaction, implying an ontological importance of interactions between corresponding objects. We will see that territories imply the existence of networks, but that reciprocally they are also influenced by them. A refined focus on properties of transportation networks allow to progressively a precise vision of \emph{co-evolution}, that we will take up to there in its preliminary sense given before, i.e. the existence of circular causal relationships between transportation networks and territories.
}{
Nous commençons par une construction plus précise des concepts mobilisés, qui permet de comprendre comment les concepts de territoire et de réseau sont rapidement en interdépendance forte, impliquant une importance ontologique des interactions entre les objets correspondants. Nous verrons que les territoires impliquent l'existence de réseaux, mais que réciproquement ceux-ci les influencent également. Un développement plus particulier sur les propriétés des réseaux de transport permet d'amener progressivement une vision précise de la \emph{co-évolution}, que nous prendrons jusque là dans son sens préliminaire donné précédemment, c'est-à-dire l'existence de relations causales circulaires entre réseaux de transports et territoires.
}



\subsection{Territories and networks, closely linked since their definition}{Territoires et Réseaux, intimement liés dès leur définition}



\subsubsection{Territories: an approach by systems of cities}{Territoires : une approche par les systèmes de villes}


\bpar{
The concept\footnote{We will use the term \emph{concept} for constructed knowledge, more than \emph{notion}, which following \cite{raffestion1978construits} is closer to an empirical information.} of \emph{territory}, that we introduced before through cities and systems of cities, will be central to our reasoning and must be depthen and enriched. In ecology, a territory corresponds to a spatial extent occupied by a group of agents or more generally an ecosystem \cite{tilman1997spatial}. Territories of human societies imply supplementary dimensions, for example through the importance of their semiotic representations\footnote{In the sense signs marking the territory and their meaning, but also their representations, as maps for example.}. These play a significant role in the emergence of social constructions, which genesis is profoundly linked to the one of urban systems. According to~\cite{raffestin1988reperes}, the \emph{Human Territoriality} is the ``conjonction of a territorial process with an informational process'', what means that the physical occupation and exploitation of space by human societies can not be dissociated from the representations (cognitive and material) of these territorial processes, driving in return its further evolutions.
}{
Le concept\footnote{Nous utiliserons le terme \emph{concept} pour des connaissances construites, plutôt que celui de \emph{notion}, qui suivant~\cite{raffestin1978construits} est plus proche d'une information empirique.} de \emph{territoire}, que nous avons introduit précédemment par ceux de ville et de système de ville, sera central à nos raisonnements et nécessite d'être approfondi et enrichi. En écologie, un groupe d'agents ou plus généralement un écosystème occupe une certaine étendue spatiale~\cite{tilman1997spatial}, qu'on peut identifier comme notion de territoire. Les territoires des sociétés humaines impliquent des dimensions supplémentaires, par exemple par l'importance de leur représentations sémiotiques\footnote{C'est-à-dire des signes marquants les territoires et leur sens, mais aussi leur représentations, cartographiques par exemple.}. Celles-ci jouent un rôle significatif dans l'émergence des constructions sociétales, dont la genèse est profondément liée à celle des systèmes urbains. Selon~\cite{raffestin1988reperes}, la \emph{Territorialité Humaine} est ``la conjonction d'un processus territorial avec un processus informationnel'', ce qui implique que l'occupation physique et l'exploitation de l'espace par les sociétés humaines sont complémentaires des représentations (cognitives et matérielles) de ces processus territoriaux, qui influent en retour sur leur évolution.
}



\bpar{
In other words, as soon as social constructions are implied in the constitution of human settlements, concrete and abstract social structures will play a role in the evolution of territories, and these two objects will be intimately binded. Examples of such links are for example the propagation of information and representations, political processes, or the conjunction or disjunction between lived and perceived territory. A territory is thus understood as a social structure organized in space, which includes its concrete abstract artifacts.
}{
En d'autres termes, à partir de l'instant où les constructions sociales déterminent la constitution des établissements humains, les structures sociales abstraites et concrètes joueront un rôle dans l'évolution des territoires, et ces deux objets seront intimement liés. Des exemples de tels liens se retrouvent à travers la propagation d'informations et de représentations, par des processus politiques, ou encore par la correspondance plus ou moins effective entre territoire vécu et territoire perçu. Un territoire est ainsi compris comme une structure sociale organisée dans l'espace, qui comprend ses artefacts concrets et abstraits.
}


\bpar{
This approach of the territory rejoin the preliminary definition we took and reinforces it. The approach of \noun{Raffestin} insists on the role of cities as places of power (in the sense of a place gathering decision processes and of socio-economic control) and of wealth creation through social and economical exchanges and interactions\footnote{An interaction will be taken in its broader meaning, as a reciprocal action of several entities one on the other. It can be physical, informational, transform the entities, etc. See \cite{morin1976methode} for a complete and complex construction of the concept, closely linked with the concept of organisation.}. The city has however no existence without its hinterland, that can be interpreted as the \emph{territory of a city}\footnote{Although an exact correspondance between territories and cities is probably only a simplification of reality, since territories can be entangled at different scales, along different dimensions. A reading through central places typical of \noun{Christaller}~\cite{banos2011christaller} gives a conceptual idea of this correspondance. Functional definitions such as \emph{Insee}'s urban areas, that defines the area around a center above a critical size (10000 jobs) by the cities for which a minimal threshold of actives work in that center (40\%) - see \url{https://www.insee.fr/fr/metadonnees/definition/c2070}, is a possible approach. The sensitivity of the properties of the urban system to these parameters is tested by~\cite{2015arXiv150707878C}. The definition of the city is therefore intimely linked to the one of territories, and the definition of the urban system to the set of territories.}. This correspondence sheds a light on all territories from the point of view of systems of cities, as developed by the evolutive urban theory~\cite{pumain2010theorie}. This theory interprets cities as complex self-organized systems, which act as mediators of social change: for example, innovation cycles initialize within cities and propagate between them (see~\ref{app:sec:patentsmining} for an empirical entry on the notion of innovation). It yield a vision of the territory as a space of flows, what will introduce the notion of network as we will see further. Cities are furthermore seen as competitive agents that co-evolve \cite{paulus2004coevolution}, what already suggests the importance of co-evolution for territorial dynamics.
}{
Cette approche du territoire rejoint la définition préliminaire que nous en avions prise, et vient alors la renforcer. L'approche de \noun{Raffestin} insiste sur le rôle des villes comme lieu de pouvoir (au sens d'un lieu rassemblant des processus décisionnels et de contrôle socio-économique) et de création de richesse au travers des échanges et interactions\footnote{Une interaction sera comprise dans son sens le plus général, comme une action réciproque de plusieurs entités l'une sur l'autre. Celle-ci peut être physique, informationnelle, transformer les entités, etc. Voir~\cite{morin1976methode} pour une construction complète et complexe du concept, en lien intime avec celui d'organisation.} (sociaux, économiques). La ville n'a cependant pas d'existence sans son hinterland, ce que nous pouvons interpréter comme le \emph{territoire d'une ville}\footnote{Même si une correspondance exacte entre territoires et villes n'est probablement qu'une simplification de la réalité, puisque les territoires peuvent s'entremêler à différentes échelles, selon différentes dimensions. Une lecture par lieux centraux de type \noun{Christaller}~\cite{banos2011christaller} permet de se faire une image conceptuelle de cette correspondance. Des définitions fonctionnelles comme celles des aires urbaines de l'Insee, qui définit l'aire autour d'un pôle dépassant une taille critique (10000 emplois) par les communes dont un seuil minimal d'actifs travaillent dans le pôle (40\%) - voir \url{https://www.insee.fr/fr/metadonnees/definition/c2070}, est une approche possible. La sensibilité des propriétés du système urbain à ces paramètres est testée par~\cite{2015arXiv150707878C}. La définition de la ville est alors intimement liée à celle de ses territoires, et celle du système urbain à l'ensemble des territoires.}. Cette correspondance permet de lire l'ensemble des territoires au prisme du système de villes, comme développé par la théorie évolutive des villes~\cite{pumain2010theorie}. Celle-ci interprète les villes comme des systèmes complexes auto-organisés, qui agissent comme des médiateurs du changement social : par exemple, les cycles d'innovation s'initialisent au sein des villes et se propagent entre elles (voir~\ref{app:sec:patentsmining} pour une entrée empirique sur la notion d'innovation). Cela permet de comprendre le territoire comme un espace des flux, ce qui permettra d'introduire la notion de réseau comme nous le verrons plus loin. Les villes sont par ailleurs vues comme des agents compétitifs qui co-évoluent~\cite{paulus2004coevolution}, ce qui permet de préfigurer également l'importance de la co-évolution pour les dynamiques territoriales.
}



% approche historique et point de vue complementaires

\bpar{
We have thus two complementary approaches of the territory that allow us to consider human territories structured by systems of cities\footnote{These complementary views on the territory can also be enriched with an historical perspective. \cite{di1998espace} gives an historical analysis of the different conceptions of space (that lead in particular to the lived space, the social space and the classical space of geography) and shows how their combination yields what \noun{Raffestin} describes as territories. \cite{giraut2008conceptualiser} recalls the different recent uses that have been done of the concept of territory, from cultural geography where it was used more as a scientific fashion, to geopolitics where it is a very specific term linked to governance structures, to uses where it is more an abstract concept, and highlights therein the interdisciplinary aspect of an object capturing a certain level of complexity of the systems studied.}.
}{
On a ainsi deux approches complémentaires du territoire qui nous permettent de considérer des territoires humains structurés par les systèmes de villes\footnote{Ces visions complémentaires du territoire peuvent également être enrichies par une perspective historique. \cite{di1998espace} procède à une analyse historique des différentes conceptions de l'espace (qui aboutissent entre autres à l'espace vécu, l'espace social et l'espace classique de la géographie) et montre comment leur combinaison forme ce que \noun{Raffestin} décrit comme territoires. \cite{giraut2008conceptualiser} rappelle les différents usages récents qui ont été faits du concept de territoire, de la géographie culturelle où il a plus été utilisé par effet de mode, à la géopolitique où c'est un terme bien spécifique lié aux structures de gouvernance, en passant par des utilisations où il sert plus de concept abstrait, et dégage l'aspect interdisciplinaire d'un objet capturant une certaine complexité des systèmes étudiés.}.
}



\bpar{
Moreover, a central aspect of human settlements that were studied in geography for a long time, and that relates directly to the concept of territory, is the one of \emph{networks}. We will detail their definition and show how switching from one to the other is intrinsic to the approaches we take on these.
}{
Par ailleurs, un aspect central des établissements humains qui a une longue tradition d'étude en géographie, et qui est directement relié au concept de territoire, est celui des \emph{réseaux}. Nous allons préciser leur définition et voir comment le passage de l'un à l'autre est intrinsèque aux approches que nous en prenons.
}



\subsubsection{Definition of networks}{Définition des réseaux}


% definition des réseaux de manière generale


\bpar{
A \emph{network} must be understood in the broad sense of the establishment of relations between entities of a system, that can be seen as abstract relations, links, interactions. \cite{hagget1970network} postulates that the existence of a network is necessarily linked to the existence of flows\footnote{Flows are defined as a material exchange (people, goods, raw materials) or immaterial (information) between two entities.}, and recalls the topological representation as a graph of any geographical system in which flows circulate between entities or places that are abstracted as nodes, linked by edges. Edges of the graph have then a \emph{capacity}, which translate their ability to transport flows (that can be defined in a similar way as an \emph{impedance}). The topological analysis already unveils a certain number of system properties, but \cite{hagget1970network} precises the importance of the network spatialization, included in the properties of its nodes (localization) and of its links (localization, impedance), for the understanding of dynamics within the network (flows) or of the network itself (network growth). This specificity is recalled by~\cite{barthelemy2011spatial} which puts into perspective empirical domains that relate to spatial networks, some network growth models, and some models of processes within networks: for example, topological structures, or diffusion processes will be strongly constrained by the spatial dimension.
}{
Un \emph{réseau} doit être compris au sens large d'une mise en relation entre entités d'un système, qui peuvent être vus comment relations abstraites, liens, interactions. \cite{haggett1970network} postule que l'existence d'un réseau est nécessairement liée à celle de flux\footnote{On définit le flux comme un échange matériel (personnes, marchandises, matières premières) ou immatériel (information) entre deux entités.}, et rappelle la représentation topologique sous forme de graphe de tout système géographique dans lequel circulent des flux entre des entités ou des lieux qui sont abstraits sous la forme de noeuds, reliés par des liens. Les liens du réseau disposent alors d'une \emph{capacité}, qui traduit leur capacité à transporter les flux (qui peut également être définie de manière équivalence comme \emph{impédance}). L'analyse topologique révèle déjà un certain nombre de propriétés du système, mais \cite{haggett1970network} précise l'importance de la spatialisation du réseau, incluse dans les propriétés de ses noeuds (localisation) et de ses liens (localisation, impédance), pour la compréhension des dynamiques dans le réseau (flux) ou du réseau lui-même (croissance du réseau). Cette spécificité est rappelée par~\cite{barthelemy2011spatial} qui met en perspective les domaines empiriques concernés par les réseaux spatiaux, certains modèles de croissance de réseau, et certains modèles de processus dans les réseaux : par exemple, les structures topologiques, ou les processus de diffusion seront très contraints par le caractère spatial.
}



% Les territoires impliquent des réseaux potentiels, selon Dupuy


\bpar{
To study more thoroughly the concept of network by focusing on its strong interdependency with the concept of territory, we follow~\cite{dupuy1987vers} which proposes elements for ``a territorial theory of networks'' inspired by the concrete case of an urban transportation network. This theory distinguishes \emph{real networks}\footnote{Real networks include a category that can be described as concrete, material or physical networks - we will use these terms in an interchangeable manner in the following, to which transportation networks belong; other categories such as social networks are also real networks that we will not study.} and \emph{virtual networks}, that are themselves induced partly by the territorial configuration. Real networks are the materialization of virtual networks. More precisely, a territory is characterized by strong spatio-temporal discontinuities induced by the non-uniform distribution of agents and ressources. These discontinuities naturally induce a network of of potential interactions between the elements of the territorial system, namely agents and ressources. \cite{dupuy1987vers} designates these potential interactions as \emph{transactional projects}. These induce the notion of \emph{potential of interaction}, i.e. a property of space from which the interactions derive\footnote{Given any vectorial field of class $\mathcal{C}^1$ on $\mathbb{R}^3$, the \noun{Helmoltz} theorem yields a vector potential and a scalar potential from which this field derives as a rotational and a gradient. It justifies in the particular case of such a viewpoint the correspondence between an interaction field between agents and a potential field.}. For example nowadays people need to access the ressource of employments, economic exchanges operate between different territories that can be more or less specialized in different types of production.
}{
Pour approfondir le concept de réseau en appuyant sur sa forte interdépendance avec celui de territoire, nous reprenons~\cite{dupuy1987vers} qui propose des éléments pour une ``théorie territoriale des réseaux'' s'inspirant du cas concret d'un réseau de transport urbain. Cette théorie distingue les \emph{réseaux réels}\footnote{Les réseaux réels contiennent une catégorie qu'on peut désigner comme réseaux concrets, matériels ou physiques - nous utiliserons ces termes de manière interchangeable par la suite, à laquelle les réseaux de transport appartiennent ; d'autres catégories comme les réseaux sociaux sont également des réseaux réels sur lesquels nous ne nous attarderons pas.} et les \emph{réseaux virtuels}, eux-mêmes induits entre autres par la configuration territoriale. Les réseaux réels sont la matérialisation de réseaux virtuels. Plus précisément, un territoire est caractérisé par de fortes discontinuités spatio-temporelles induites par la distribution non-uniforme des agents et des ressources. Ces discontinuités induisent naturellement un réseau d'interactions potentielles entre les éléments du système territorial, notamment des agents et des ressources. \cite{dupuy1987vers} désigne ces interactions potentielles comme \emph{projets transactionnels}. Celles-ci induisent la notion de \emph{potentiel d'interaction}, c'est-à-dire une propriété de l'espace dont les interactions dérivent\footnote{Étant donné tout champ vectoriel de classe $\mathcal{C}^1$ sur $\mathbb{R}^3$, le théorème d'\noun{Helmoltz} fournit un potentiel vecteur et un potentiel scalaire dont ce champ dérive par rotationnel et gradient. Cela justifie dans le cas particulier d'un tel point de vue formel le passage d'un champ d'interactions entre agents à un champ de potentiel.}. Par exemple, de nos jours les actifs ont besoin d'accéder à la ressource qu'est l'emploi, et des échanges économiques s'effectuent entre les différents territoires qui peuvent être plus ou moins spécialisés dans les productions de différents types.
}



\subsubsection{From networks to real networks}{Des réseaux aux réseaux réels}

% Les réseaux potentiels se transforment en réseaux réels sous certaines conditions.
%. -> effet des territoires sur les réseaux



\bpar{
In some cases, a potential network is materialized into a real network. The underlying question is then to determine if the potential field of territories is partly at the origin of this materialization, if it is totally independent, or if the dynamic of the two is strongly coupled, in other terms in co-evolution. The materialization will generally result of the combination of economic and geographical constraints with demand patterns, in a non-linear way. Such a process is not immediate, leading to strong non-stationarity and path-dependancy effects\footnote{Spatial non-stationarity consists in the dependancy of the covariance structure of processes to space, whereas path-dependency corresponds to the fact that trajectories taken in the past strongly influence the current trajectories of the system.}: the extension of an existing network will depend on previous configurations, and depending on involved time scales, the logic and even the nature of operators, i.e. agents participating to its production, may have evolved.
}{
Dans certains cas, un réseau potentiel peut se matérialiser en réseau réel. La question sous-jacente est alors de savoir si le champ de potentiel des territoires est en partie à l'origine de cette matérialisation, si celle-ci est totalement indépendante, ou si la dynamique des deux est fortement couplée, en d'autres termes en co-évolution. La matérialisation résultera généralement de la combinaison de contraintes économiques et géographiques avec des motifs de demande, de manière non-linéaire. Un tel processus est loin d'être immédiat, et conduit à de forts effets de non-stationnarité et de dépendance au chemin\footnote{La non-stationnarité spatiale consiste en la dépendance de la structure de covariance des processus à l'espace, tandis que la dépendance au chemin traduit le fait que les trajectoires prises par le passé influencent fortement les trajectoires actuelles du système.} : l'extension d'un réseau existant dépendra de la configuration précédente, et selon les échelles de temps impliquées, la logique et même la nature des opérateurs, c'est-à-dire des agents participant à sa production, peut avoir évolué.
}




\bpar{
Examples of concrete trajectories can be quite varied: \cite{kasraian2015development} show for example, in the case of Randstad on long time, a first period during which the railway network has developed to follow urban development, whereas opposite effects has been more recently observed. At a urban scale on long time, the path-dependency is shown for Boston by~\cite{block2012hysteresis} since the built environment and the distribution of population appear as highly dependant of past tramway lines even when they do not exist anymore: the way the transportation line changes the urban space acts on immediate dynamics but also on a longer time through reinforcement effects or because of the inertia of the built environment for example. 
}{
Les exemples de trajectoires concrètes peuvent être très variées : \cite{kasraian2015development} montrent par exemple dans le cas de la Randstad sur le temps long, une première période pendant laquelle le réseau ferré s'est développé pour suivre le développement urbain, tandis que des effets inverses ont été constatés plus récemment. À une échelle urbaine sur le temps long, la dépendance au chemin est montrée pour Boston par~\cite{block2012hysteresis} puisque l'environnement bâti et la distribution de la population apparaissent comme fortement dépendants des lignes de tramway antérieures même lorsqu'elles n'existent plus : la façon dont la ligne de transport change l'espace urbain s'opère dans les dynamiques immédiates mais aussi sur le temps long par des effets de renforcement ou à cause de l'inertie du bâti par exemple.
}

\bpar{
Therefore, the existence of a human territory necessarily imply the presence of abstract interaction networks, and concrete networks are crucial for the transport of people and ressources (including communication networks as information is a crucial ressource~\cite{morin1976methode}), but the processes through which they are established are difficult to identify generally. Our ontological choice of positioning within \noun{Dupuy}'s theory, gives a privileged place to the relations between networks and territories, since it induces in the construction of the objects themselves a complex entanglement between these.
}{
Ainsi, l'existence d'un territoire humain implique nécessairement la présence de réseaux d'interactions abstraites, et les réseaux concrets sont cruciaux pour transporter les individus et les ressources (incluant les réseaux de communication puisque l'information est une ressource essentielle~\cite{morin1976methode}), mais les processus d'établissement de ceux-ci sont difficiles à identifier de manière générale. Notre choix ontologique de positionnement dans la théorie de \noun{Dupuy}, donne une place privilégiée aux relations entre réseaux et territoires, puisqu'il induit dans la construction des objets même une imbrication complexe entre ceux-ci.
}


% le contexte socio-eco/techno conditionne fortement la facon dont les réseaux agissent sur les territoires.

\bpar{
The status of the network in relation with the territory is moreover highly conditioned by the socio-economical and technological context. Following~ \cite{duranton1999distance}, a factor influencing the form of pre-industrial cities was the performance of transportation networks. Technological progresses, leading to a decrease in transportation costs, have inducted a regime change, what conducted to a preponderance of land markets in shaping cities (and thus a role of transportation network since they influence prices through accessibility), and more recently to the rising importance of telecommunication networks what induced a ``tyranny of proximity'', since a physical presence can not be replaced by virtual communications \cite{duranton1999distance}.
}{
Le statut du réseau par rapport au territoire est d'autre part fortement conditionné par le contexte socio-économique et technologique. Selon \cite{duranton1999distance}, un facteur influençant la forme des villes pré-industrielles était la performance des réseaux de transport. Les progrès technologiques, conduisant à une baisse des coûts de transport, ont induit un changement de régime, ce qui a mené à une prépondérance du marché foncier dans la formation des villes (et par conséquent un rôle des réseaux de transport qui déterminent les prix par l'accessibilité), et plus récemment à une importance croissante des réseaux de télécommunication ce qui a induit une ``tyrannie de la proximité'' puisque la présence physique n'est pas remplaçable par une communication virtuelle \cite{duranton1999distance}.
}



% Transition

\bpar{
This territorial approach to networks seems natural in geography, since networks are studied conjointly with geographical objects they connect, in opposition to theoretical works on complex networks which study them in a relatively disconnected way from their thematic background~\cite{ducruet2014spatial}.
}{
Cette approche territoriale des réseaux semble naturelle en géographie, puisque les réseaux sont étudiés conjointement avec des objets géographiques qu'ils connectent, en opposition aux travaux théoriques sur les réseaux complexes qui les étudient de manière relativement déconnectée de leur fond thématique~\cite{ducruet2014spatial}.
}



\subsubsection{Networks shaping territories ?}{Des réseaux qui façonnent les territoires ?}

% Effets des réseaux sur les territoires ? -> approfondir le debat des effets structurants

\bpar{
However networks are not only a material manifestation of territorial processes, but play their role in these processes since their evolution may influence the evolution of territories in return. Here comes an intrinsic difficulty: it is far from evident to attribute territorial mutations to an evolution of the network, and reciprocally the materialization of a network to precise territorial dynamics. Different exogenous factors are furthermore important, such as the price of energy or existing technologies in the case of the effect of the network on territories for example. In the case of \emph{technical networks}, an other designation of concrete networks given in~\cite{offner1996reseaux}, many examples of such feedbacks can be found: an increased accessibility may shape urban growth, or the interconnectivity of different transportation networks allows a significant extension of mobility ranges. At a smaller scale, changes in accessibility may induce relocalizations of different urban components. These retroactions of networks on territories does not necessarily act on concrete components: \cite{claval1987reseaux} shows that transportation and communication networks contribute to the collective representation of a territory by acting on the sentiment to belong to the territory, that can then play a crucial role in the emergence of a strongly coherent regional dynamic. We first develop with more details the possible influences of networks on territories.
}{
Cependant les réseaux ne sont pas seulement une manifestation matérielle de processus territoriaux, mais jouent également leur rôle dans ces processus puisque leur évolution peut influencer l'évolution des territoires en retour. Il emerge alors une difficulté intrinsèque : il n'est pas évident d'attribuer des mutations territoriales à une évolution du réseau et réciproquement la matérialisation d'un réseau à des dynamiques territoriales précises. Différents facteurs exogènes rentrent par ailleurs en compte, comme le prix de l'énergie ou les technologies existantes dans le cas de l'effet du réseau sur les territoires par exemple. Dans le cas des \emph{réseaux techniques}, une autre désignation des réseaux concrets donnée dans~\cite{offner1996reseaux}, de nombreux exemples de tels retroactions peuvent être mis en évidence : une accessibilité accrue peut être un facteur favorisant la croissance urbaine, ou bien l'interconnexion de différents réseaux de transport permet une extension significative de la portée des déplacements. À une plus petite échelle, des changements de l'accessibilité peuvent induire des relocalisations de différentes composantes urbaines. Ces rétroactions des réseaux sur les territoires n'agissent pas nécessairement sur des composantes concretes : \cite{claval1987reseaux} montre que les réseaux de transport et de communication contribuent à la représentation collective d'un territoire en agissant sur un sentiment d'appartenance, qui peut alors jouer un rôle crucial dans l'émergence d'une dynamique régionale fortement cohérente. Développons d'abord plus en détail les possibles influences des réseaux sur les territoires.
}


\bpar{
The confusion on possible simple causal relationships has fed a scientific debate that is still active nowadays. The underlying question relies on more or less deterministic attributions of impacts to transportation infrastructures or to a new transportation mode on territorial transformations. Precursors of such a reasoning can be tracked back in the twenties: \noun{McKenzie}, from the Chicago school, mentions in~\cite{burgess1925city} some ``modifications of forms of transportation and communication as determining factors of growth and decline cycles [of territories]'' (p.~69). Methodologies to identify what is then called \emph{structuring effects} of transportation networks has been developed for planning in the seventies: \cite{bonnafous1974methodologies} situates the concept of structuring effect in the perspective of using the transportation offer as a planning tool (the alternatives are the development of an offer to answer to a congestion of the network, and the simultaneous development of associated offer and planning). These authors identify from an empirical viewpoint direct effects of a novel offer on the behavior of agents, on transportation flows and possible inflexions on socio-economic trajectories of concerned territories. \cite{bonnafous1974detection} develop a method to identify such effects through the modification of the class of cities in a typology established a posteriori. More recently, \cite{bonnafous2014observatoires} recalls that the institution of \emph{permanent observatories} for territories makes such analyses more robust, allowing a continuous monitoring of the territories that are the most concerned by the extent of a new infrastructure.
}{
La confusion autour de possibles relations causales simples a nourri un débat scientifique encore actif aujourd'hui. La question sous-jacente repose sur des attributions plus ou moins déterministes d'impact d'infrastructures ou d'un nouveau mode de transport sur des transformations territoriales. Nous pouvons trouver des précurseurs de ce raisonnement dès les années 1920 : \noun{McKenzie}, de l'école de Chicago, parle dans~\cite{burgess1925city} des ``modifications des formes du transport et de la communication comme facteurs déterminants des cycles de croissance et de déclin [des territoires]'' (p.~69). Des méthodologies pour identifier ce qui est alors nommé \emph{effets structurants} des réseaux de transport ont été développées pour la planification dans les années 1970 : \cite{bonnafous1974methodologies} situe le concept d'effet structurant dans le cadre d'une logique d'utilisation de l'offre de transport comme outil d'aménagement (les alternatives étant le développement d'une offre pour répondre à une congestion du réseau, et le développement simultané d'une offre et d'un aménagement associé). Ces auteurs identifient du point de vue empirique des effets directs d'une nouvelle offre sur le comportement des agents, sur les flux de transport et des possibles inflexions sur les trajectoires socio-économiques des territoires concernés. \cite{bonnafous1974detection} développent une méthode pour identifier de tels effets par modifications de la classe des communes dans une typologie établie a posteriori. Plus récemment, \cite{bonnafous2014observatoires} rappelle que la mise en place \emph{d'observatoires permanents} des territoires permet de rendre plus robustes ce type d'analyse, en permettant un suivi continu de l'évolution des territoires les plus concernés par l'emprise d'une nouvelle infrastructure.
}


\bpar{
According to \cite{offner1993effets} which follows ideas already given by~\cite{franccois1977autoroutes} for example, a not reasoned and out-of-context use of these methods has then been developed by planners and politicians which generally used them to justify transportation projects in a technocratic manner: through the argument of a direct effect of a new infrastructure on local development (for example economic), politics are able to ask for subsidies and to legitimate their action in front of the people. \cite{offner1993effets} insists on the necessity of a critical positioning on these issues, recalling that there exists no scientific demonstration of an effect that would be systematic. A special issue of the journal \emph{L'Espace Géographique}~\cite{espacegeo2014effets} on that debate recalled that on the one hand misconceptions and misuses were still greatly present in operational and planning communities, which can be explained for example by the need to justify public actions, and on the other hand that a scientific understanding of relations between networks and territories is still in construction. \noun{A. Bonnafous} (interview on the 09/01/2018, see Appendix~\ref{app:sec:interviews}) gives the current example of the project of the Seine-Nord-Europe canal\footnote{The canal project links the Oise at Compiègne to the Dunkerque-Escault canal in the north, see \url{https://www.canal-seine-nord-europe.fr/Projet}.} as a transportation project for which traffic previsions were largely overestimated and that politics of concerned territories have largely instrumentalized.
}{
Selon \cite{offner1993effets} qui reprend des idées déjà évoquées par~\cite{franccois1977autoroutes} par exemple, il s'est par la suite développé un usage non raisonné et hors contexte de ces méthodes par les planificateurs et les politiques qui les mobilisaient généralement pour justifier des projets de transports de manière technocratique : par l'argument d'un effet direct d'une nouvelle infrastructure sur le développement local (par exemple économique), les élus sont en mesure de demander des financements et de légitimer leur action auprès des contribuables. \cite{offner1993effets} insiste sur la nécessité d'un positionnement critique sur ces enjeux, rappelant qu'il n'existe pas de démonstration scientifique d'un effet qui serait systématique. Une édition spéciale de l'Espace Géographique sur ce débat~\cite{espacegeo2014effets} a rappelé d'une part que de telles croyances était encore largement présentes aujourd'hui dans les milieux opérationnels de la planification, ce qui peut s'expliquer par exemple par le besoin de justifier l'action publique, et d'autre part qu'une compréhension scientifique des relations entre réseaux et territoires est encore en pleine construction. \noun{A. Bonnafous} (entretien du 09/01/2018, voir Annexe~\ref{app:sec:interviews}) donne l'exemple actuel du projet du canal Seine-Nord-Europe\footnote{Le projet de canal relie l'Oise à Compiègne au canal Dunkerque-Escault au nord, voir \url{https://www.canal-seine-nord-europe.fr/Projet}.} comme projet de transport pour lequel les prévisions de trafic ont été largement surestimées et que les élus des territoires concernés ont largement instrumentalisé.
}


\bpar{
An other concrete illustration in the actuality gives an idea of this instrumentalization: debates in July of 2017 concerning the opening of the \emph{LGV Bretagne} and the \emph{LGV Sud-Ouest} have shown the full ambiguity of positions, conceptions, imaginaries both of politics but also of the public: worries on the speculation on real estate in stations neighborhoods, questionings on daily mobility but also social mobility\footnote{See for example \url{http://www.liberation.fr/futurs/2017/07/02/immobilier-plus-de-parisiens-comment-les-bordelais-voient-l-arrivee-de-la-lgv_1580776}, or \url{http://www.lemonde.fr/big-browser/article/2017/10/24/a-bordeaux-une-fronde-anti-parisiens-depuis-l-ouverture-de-la-ligne-a-grande-vitesse_5205282_4832693.html} for an immediate reaction of diverse local actors, witnessing at least an impact on representations. For example, people in Bordeaux seem to fear the arrival of Parisians searching for cheaper housing and better living conditions, what could increase prices in the surroundings of the station.}. The complexity and the reach of these subjects show well the difficulty of a systematic understanding of effects of transportation on territories.
}{
Une autre illustration concrète d'actualité permet de se faire une image de cette instrumentalisation : les débats en juillet 2017 relatifs à l'ouverture des LGV Bretagne et Sud-Ouest ont montré toute l'ambiguïté des positions, des conceptions, des imaginaires à la fois des politiques mais aussi du public : inquiétude quant à la spéculation sur l'immobilier dans les quartiers de gare, questionnements des pratiques de mobilité quotidienne mais aussi sociale\footnote{Voir par exemple \url{http://www.liberation.fr/futurs/2017/07/02/immobilier-plus-de-parisiens-comment-les-bordelais-voient-l-arrivee-de-la-lgv_1580776}, ou \url{http://www.lemonde.fr/big-browser/article/2017/10/24/a-bordeaux-une-fronde-anti-parisiens-depuis-l-ouverture-de-la-ligne-a-grande-vitesse_5205282_4832693.html} pour une réaction ``à chaud'' de divers acteurs locaux, témoignant d'un impact au minimum sur les représentations. Par exemple, les Bordelais semblent craindre l'arrivée de Parisiens en recherche d'un logement moins cher et de meilleures conditions de vie, ce qui pourrait augmenter les prix aux environs de la gare.}. La complexité et la portée des sujets montrent bien la difficulté d'une compréhension systématique d'effets du transport sur les territoires.
}



\subsubsection{An integrative approach: Territorial Systems}{Une vision intégrative : les Systèmes Territoriaux}


\bpar{
This overview as an introduction, from territories to networks, allows us thus to clarify our approach of territorial systems that will be underlying all the following. Taking into account diverse potential feedbacks of networks for the understanding of territories is suggested when coming back to the citation by Diderot that introduced the subject, in the sense that we must consider neither the network nor territories as independent systems that would influence themselves through one directional causal relations, but as strongly coupled components of a broader system, and thus being in a circular causal relationship. Depending on components and the scale that are considered, different manifestations of these will be observable, and there will exist some cases where there is apparently the influence of one on the other, other where influences are simultaneous, or moreover others where no relationship can be observed in a significant way. 
}{
Cet aperçu introductif, des territoires aux réseaux, nous permet ainsi de clarifier notre approche des systèmes territoriaux qui sera sous-jacente dans l'ensemble de la suite. Une prise en compte des diverses rétroactions potentielles des réseaux pour la compréhension des territoires est suggérée par un retour à la citation de Diderot ayant introduit le sujet, au sens où il ne faut pas considérer le réseau ni les territoires comme des systèmes indépendants qui s'influenceraient soit l'un soit l'autre par des relations causales en sens unique, mais comme des composantes fortement couplées d'un système plus large, et donc étant en relations causales circulaires. Selon les composantes ainsi que l'échelle considérées, différentes manifestations de celles-ci pourront être observables, et il existera des cas où il y a apparemment influence de l'une sur l'autre, d'autres où les influences sont simultanées, ou encore d'autres où aucune relation n'est observable de manière significative.
}


\bpar{
Since we have highlighted the role of networks in several aspects of territorial dynamics, we propose a definition of territorial systems that explicitly includes them. We consider a \emph{Territorial System} as a \emph{human territory that contains both interactions networks and real networks}. Real networks, and more particularly concrete networks\footnote{Which are as we previously saw materialized real networks.}, are an entire component of the system, influencing evolution processes, through multiple feedbacks with other components at many spatial and temporal scales. 
}{
Comme nous avons mis en exergue le rôle des réseaux dans de nombreux aspects des dynamiques territoriales, nous proposons une définition des systèmes territoriaux les incluant explicitement. Nous considérons un \emph{Système Territorial} comme un \emph{territoire humain qui contient à la fois des réseaux d'interactions et des réseaux réels}. Les réseaux réels, et plus particulièrement les réseaux concrets\footnote{Qui comme nous l'avons vu précédemment sont des réseaux réels matérialisés.}, sont une composante à part entière du système, jouant dans les processus d'évolution, au travers de multiples rétroactions avec les autres composantes à plusieurs échelles spatiales et temporelles.
}



\bpar{
The network is not necessarily a component in itself of the territory, but indeed of the \emph{Territorial System} in our sense\footnote{This ontological choice is not innocent and reinforces the dialectic between networks and territories. Starting from the distant past where physical networks did not exist, the emergence of a human territory, that we assume equivalent to a network of interactions, induces the establishment of the complex diachronic dialectic between physical networks and human territories. We can thus read the genesis of a territorial system as a morinian loop~\cite{morin1976methode}, in which we enter by the initial territory and which then loops from the physical network to territorial components to produce the territorial system (thus the territory in most cases) in the following recursive way:\\Initial territory $\rightarrow$ Territory $=$ \tikzmark{Territorial} configuration $\rightarrow$ Physical \tikzmark{network}\arrow{network}{Territorial}\\}. This view rejoins the positioning of \cite{dupuy1985systemes} which introduces the territory as the ``product of a dialectic'' between territorial components and networks. We remark the semantic shortcut to designate components of the territorial system that are not the network and which interact with it, through the term of territory. These depend on ontologies and scales considered, as we will see in the following, and can span from microscopic agents to cities themselves. As we will also see in the following (see~\ref{sec:modelingsa}), there exists some paradigms in which this simplification is not done, such as in the particular case of interactions between transportation and land-use where entities are specific. But it is done if we stay in a more general framework, as witnesses one of the reference works on the subject~\cite{offner1996reseaux}\footnote{When \cite{amar1985essai} proposes a conceptual model of network morphogenesis, he designates the territorial components as ``The World'', what does not solve the semantic issue. The choice to keep the term of territory, within the territory, suggests a recursivity, and thus a complexity in the generativity of the system~\cite{morin1976methode}. The use of the concept of morphogenesis starting from chapter~\ref{ch:morphogenesis} suggests that this recursivity would not be spurious, but indeed intrinsic to the problem.}. We will similarly postulate this semantic simplification, when designating by \emph{interactions between networks and territories} or \emph{co-evolution between networks and territories}, the interactions or the co-evolution between physical networks and components they connect, within a territorial system and thus a territory.
}{
Le réseau n'est pas nécessairement une composante en tant que telle du territoire, mais bien du \emph{Système Territorial} en notre sens\footnote{Ce choix ontologique n'est pas anodin et appuie la dialectique entre réseaux et territoires. Partant de l'époque lointaine où les réseaux physiques n'existaient pas, l'émergence d'un territoire humain, que nous supposons équivalent à un réseau d'interactions, induit la mise en place de la dialectique diachronique complexe entre réseaux physiques et territoires humains. On peut ainsi lire la genèse du système territorial comme une boucle morinienne~\cite{morin1976methode}, dans laquelle on entre par le territoire initial puis qui se boucle du réseau physique aux composantes territoriales pour former le système territorial (donc le territoire dans la majorité des cas) de la manière récursive suivante :\\Territoire initial $\rightarrow$ Territoire $=$ \tikzmark{Configuration} territoriale$\rightarrow$ Réseau \tikzmark{physique}\arrow{physique}{Configuration}\\}. Cette vision rejoint le positionnement de \cite{dupuy1985systemes} qui introduit le territoire comme ``produit d'une dialectique'' entre composantes territoriales et réseaux. Notons le raccourci sémantique pour désigner les composantes du système territorial qui ne sont pas les réseaux et qui interagissent avec celui-ci, par le terme de territoire. Celles-ci dépendent des ontologies et des échelles considérées, comme nous le verrons par la suite, et peuvent aller des agents microscopiques aux villes elle-mêmes. Comme nous le verrons aussi par la suite (voir~\ref{sec:modelingsa}), il existe des paradigmes où ce raccourci n'est pas fait, comme dans le cas particulier des interactions entre transport et usage du sol où les entités sont spécifiques. Mais il est fait si nous restons dans un cadre plus général, comme en témoigne l'un des ouvrages de référence sur le sujet~\cite{offner1996reseaux}\footnote{Lorsque \cite{amar1985essai} propose un modèle conceptuel de morphogenèse des réseaux, il désigne les composantes territoriales par ``Le Monde'', ce qui n'apporte pas de solution au problème sémantique. Le parti pris de garder le territoire, au sein du territoire, suggère une récursivité, et donc une complexité dans la générativité du système~\cite{morin1976methode}. La mobilisation du concept de morphogenèse à partir du chapitre~\ref{ch:morphogenesis} suggère que cette récursivité serait plus que fortuite, mais bien intrinsèque au problème.}. Nous assumerons également ce raccourci de langage, en désignant par \emph{interactions entre réseaux et territoires} ou \emph{co-évolution entre réseaux et territoires}, les interactions ou la co-évolution entre les réseaux physiques et les composantes qu'ils relient, au sein d'un système territorial et donc d'un territoire.
}




%\subsection{Transportation Networks}{Réseaux de Transport}
\subsection{Transportation networks, specific carriers of interactions}{Les réseaux de transport, catalyseurs privilégiés des interactions}


\bpar{
We now precise the particular case of transportation networks and develop associated specific concepts that will play an important role in the precision of our problematic.
}{
Nous précisons à présent le cas particulier des réseaux de transport et développons des concepts spécifiques associés qui joueront un rôle prépondérant dans la précision de notre problématique.
}


\subsubsection{Characteristics and specificities of transportation networks}{Caractéristiques et spécificités des réseaux de transport}


\bpar{
Central to the already evoked debates on structuring effects of networks, transportation networks play a significant role in the evolution of territories, but it is of course out of question to give them deterministic causal effects. We will generally use the term of transportation network to designate the functional entity allowing a movement of agents and resources within and between territories\footnote{We designate thus simultaneously the infrastructure, but also its exploitation conditions, the rolling stock, the exploitation agents.}. Even if other types of networks are also strongly implicated in the evolution of territorial systems (see for example the debates on the impact of communication networks on the localization of economic activities), transportation networks condition other types of networks (logistic, commercial exchanges, concrete social interactions to give a few examples) and are a privileged entry regarding patterns of territorial evolution, in particular in our contemporary societies for which transportation networks play a crucial role~\cite{bavoux2005geographie}. We will therefore focus in the following only on transportation networks.
}{
Centraux aux discussions déjà évoquées sur les effets structurants des réseaux, les réseaux de transports jouent un rôle significatif dans l'évolution des territoires, mais il n'est évidemment pas question de leur attribuer des effets causaux déterministes. Nous parlerons de manière générale de réseau de transport pour désigner l'entité fonctionnelle permettant un déplacement des agents et des ressources au sein et entre les territoires\footnote{On désigne ainsi à la fois l'infrastructure, mais aussi ses conditions d'exploitation, le matériel roulant, les agents exploitants.}. Même si d'autres types de réseaux sont également fortement impliqués dans l'évolution des systèmes territoriaux (voir par exemple les débats sur l'impact des réseaux de communication sur la localisation des activités économiques), les réseaux de transport conditionnent d'autres types de réseaux (logistique, échanges commerciaux, interactions sociales concrètes pour donner quelques exemples) et sont une entrée privilégiée en rapport aux motifs d'évolution territoriale, en particulier dans nos sociétés contemporaines pour lesquelles les réseaux de transport jouent un rôle crucial~\cite{bavoux2005geographie}. Nous nous concentrerons ainsi par la suite uniquement sur les réseaux de transport.
}




\bpar{
The development of the French high speed rail network is an illustration of the role of transportation networks on policies of territorial development. Presented as a new era of railway transportation, it consisted in a top-down planning of totally novel lines, relatively independent through they two times higher speed, as~\cite{zembri1997fondements} puts it. High speed has been defended by political actors among other things as central for the development.
The weak integration of these new networks with the existing network and with local territories is now understood as a structural weakness~\cite{zembri1997fondements} (i.e. that is a consequence of network structure such as it was planned in the \emph{Scéma Directeur} of 1990), and negative impacts on some territories, such as the suppression of intermediate stops on classical lines used by the TGV, what contributes to an increase of the tunnel effect\footnote{The tunnel effect designates the process of telescoping the territory traversed by the infrastructure, when it is not accessible from this territory.} have been shown~\cite{zembri2008contribution}. A review done in~\cite{bazin2011grande} confirms that no general conclusions on local effects of a connection to a high speed line could be drawn, although it keeps a strong place in imaginaries of politics\footnote{But particular conclusions exist in some cases: for example a positive effect of the LGV Sud-Est on the touristic intensity in intermediate medium-sized cities such as Montbard or Beaune~\cite{bonnafous1987regional}; or the positioning of Lille as an European metropolis in which the connexions to the LGV have played a role~\cite{giblin2004lille}.}. The development of different high speed lines takes place in very different territorial contexts, and it is in any case difficult to interpret processes out of context: for example, the LGV Nord and LGV Est lines are situated within European scales that are broader than for the LGV Bretagne opened in July 2017\footnote{The LGV Nord line links Paris to Lille then Calais (entirely opened in 1997), and is used for the link with London, Brussels, Amsterdam. The LGV Est line links Paris to Strasbourg (partially opened in 2007, fully in 2016) and allows to serve Luxembourg and Germany. The LGV Bretagne line, opened in 2017, is the branch of the LGV Ouest towards Rennes and its service is uniquely to Britanny~\cite{zembri2010new}.}. The effects of the opening of a line can extend beyond the directly concerned territories: \cite{l2014contribution} show through the use of indicators from \emph{Time Geography}\footnote{The \emph{Time Geography}, introduced by the Swedish geographer \noun{Hägerstrand}, focuses mainly on trajectories of individuals in time and space, and of their implications in interactions with the environment~\cite{chardonnel2007time}.} (measuring an available working time in the context of a return journey within the day) that the Tours-Bordeaux line has potential impacts in the North and East of France. These examples illustrate well the way transportation networks can have effects both directly and indirectly, positive or negative, at different scales, or no effect at all on territorial dynamics.
}{
Le développement du réseau français à grande vitesse est une illustration du rôle des réseaux de transport sur les politiques de développement territorial. Présenté comme une nouvelle ère de transport sur rail, il s'agit d'une planification au niveau de l'État de lignes totalement nouvelles et relativement indépendantes de par leur vitesse deux fois plus élevée, selon la lecture de~\cite{zembri1997fondements}. La grande vitesse a été défendue par les acteurs politiques entre autres comme central pour le développement. L'articulation faible de ces nouveaux réseaux avec le réseau classique et avec les territoires locaux est à présent observé comme une faiblesse structurelle~\cite{zembri1997fondements} (c'est-à-dire conséquence de la structure du réseau tel qu'il a été planifié dans le Schéma Directeur de 1990), et des impacts négatifs sur certains territoires, comme par la suppression de dessertes intermédiaires sur les lignes classiques empruntées par le TGV, qui contribue à un accroissement de l'effet tunnel\footnote{L'effet tunnel désigne le processus de télescopage du territoire traversé par une infrastructure, celle-ci n'étant utilisable à partir de celui-ci.} ont été montrés~\cite{zembri2008contribution}. Une revue faite dans~\cite{bazin2011grande} confirme qu'aucune conclusion générale sur des effets locaux d'une connection à une ligne à grande vitesse ne peut être tirée, bien que ce sésame garde une place conséquente dans les imaginaires des élus\footnote{Mais des conclusions particulières existent dans certains cas : par exemple un effet positif de la LGV Sud-Est sur la fréquentation touristique de villes moyennes intermédiaires comme Montbard ou Beaune~\cite{bonnafous1987regional} ; ou le positionnement de Lille comme métropole européenne dans lequel les connexions LGV ont joué~\cite{giblin2004lille}.}. Le développement des différentes Lignes à Grande Vitesse s'inscrit dans des contextes territoriaux très différents, et il est dans tous les cas délicat d'interpréter des processus en les sortant de leur contexte : par exemple, les lignes LGV Nord et LGV Est s'inscrivent dans des échelles européennes plus vastes que la LGV Bretagne ouverte en juillet 2017\footnote{La ligne LGV Nord relie Paris à Lille puis Calais (ouverte entièrement en 1997), et s'inscrit dans la liaison avec Londres, Bruxelles et Amsterdam. La LGV Est relie Paris à Strasbourg (ouverte partiellement en 2007, puis entièrement en 2016) et permet de desservir le Luxembourg et l'Allemagne. La LGV Bretagne, ouverte en 2017, est le tronçon de la LGV Ouest vers Rennes et sa desserte est uniquement bretonne~\cite{zembri2010new}.}. Les effets de l'ouverture d'une ligne peuvent s'étendre au delà des seuls territoires directement concernés : \cite{l2014contribution} montrent par l'utilisation d'indicateurs issus de la \emph{Time Geography}\footnote{La \emph{Time Geography}, introduite par le géographe suédois \noun{Hägerstrand}, s'intéresse majoritairement aux trajectoires des individus dans le temps et l'espace, et de leurs implications dans les interactions avec l'environnement~\cite{chardonnel2007time}.} (mesurant une quantité de temps de travail disponible dans le cadre d'un aller-retour journalier) que la ligne Tours-Bordeaux a des répercussions potentielles dans le Nord et l'Est de la France. Ces exemples illustrent la manière dont les réseaux de transport peuvent avoir des effets à la fois directs et indirects, positifs ou négatifs, et à différentes échelles, ou bien aucun effet sur les dynamiques territoriales.
}





\subsubsection{Processes depending on scales}{Des processus dépendant des échelles}


\bpar{
The question of concerned temporal and spatial scales has until now been tackled only on a secondary plan compared to the concepts introduced. We propose now to integrate them to our reasoning in a structural way, i.e. guiding the developments of new concepts. Therefore, the concepts of \emph{Mobility}, \emph{Accessibility}\footnote{The accessibility, as we will see, can be defined at different scales, but we will use this term in a privileged way for accessibility landscapes at the metropolitan scale.}, and \emph{Structural Dynamics on long time}, correspond each to decreasing scales in time and space: intra-urban and daily, metropolitan and decennial, regional (in the broad and flexible sense of the range of a system of cities) and centennial. The correspondence we postulate here between time scales and spatial scales, far from being an evidence, will be shown during the development of each of these concepts. However, to take into account multiple scales is important, as shows \cite{RIETVELD1994329} with a review of economic approaches to interactions, which insists on the difference between intra-urban and intra-regional: at a large scale, different methods (models or qualitative approaches) give very different results concerning the impact of the infrastructure stock, whereas at a small scale, the positive impact of the global stock on productivity can not a priori been discussed.
}{
La question des échelles temporelles et spatiale concernées a été jusqu'ici abordée de manière auxiliaire aux concepts introduits. Nous proposons à présent de les intégrer de manière structurelle à notre raisonnement, c'est-à-dire guidant le développements de nouveaux concepts. Ainsi, les concepts de \emph{Mobilité}, d'\emph{Accessibilité}\footnote{L'accessibilité, comme nous le verrons, se définit à plusieurs échelles, mais nous privilégierons ce terme pour les paysages d'accessibilité à l'échelle métropolitaine.}, puis de \emph{Dynamique structurelle sur le temps long}, correspondent chacun à des échelles de temps et d'espace décroissantes : intra-urbain et journalier, métropolitain et décennal, régional (au sens large et flexible de la portée d'un système de villes) et centennal. La correspondance que nous postulons ici entre échelles de temps et échelles d'espace, loin d'être évidente, sera montrée lors du développement de chacun de ces concepts. Par contre, la prise en compte d'échelles multiples est importante, comme le montre \cite{RIETVELD1994329} par une revue des approches économiques des interactions, qui appuie la différence entre l'intra-urbain et l'intra-régional : à grande échelle, différentes méthodes (modèles ou approches qualitatives) donnent des résultats très différents quant à l'impact du stock d'infrastructure, tandis qu'à petite échelle, l'impact positif du stock global sur la productivité est a priori non discutable.
}



\subsubsection{Transportation and mobility}{Transports et mobilité}


\bpar{
The notion of mobility and all the associated approaches capture partly our questionings at a large scale. We will define mobility in a broad manner as a movement of territorial agents in space and time. It is related to use patterns of transportation networks. \cite{hall2005reconsidering} introduces a theoretical framework that yields a typology of mobility practices. In particular, he shows a rapid decrease of the frequency of journeys with spatial range and duration, and thus that ``micro-micro'' patterns (for the daily temporal scale and the intra-urban spatial scale), that we designate as \emph{daily mobility}, correspond to the most of journeys. It does not however mean an absence of link with other scales: on the one hand mobility patterns are very strongly conditioned by the distribution of activities as illustrate~\cite{lee2015relating}, but are on the other hand correlated to the social structure~\cite{camarero2008exploring}, that evolve both at time scales of a different magnitude (larger than a decade, thus at least one order in magnitude). Therefore, infrastructure and superstructure determine mobility practices, giving an important role to transportation networks in these.
}{
La notion de mobilité et l'ensemble des approches associées capturent en partie nos questionnements à grande échelle. Nous définirons la mobilité de manière générale comme un déplacement d'agents territoriaux dans l'espace et le temps. Elle relève des motifs d'utilisation des réseaux de transport.  \cite{hall2005reconsidering} introduit un cadre théorique permettant une typologie des pratiques de mobilité. En particulier, il montre une décroissance rapide de la fréquence des déplacements avec la portée spatiale et la durée, et donc que les motifs ``micro-micro'' (pour échelle temporelle journalière et échelle spatiale intra-urbaine), qu'on désigne par \emph{mobilité quotidienne}, sont majoritaires. Cela ne signifie pas pour autant une absence de lien avec d'autres échelles : d'une part les motifs de mobilité sont très fortement conditionnés par la distribution des activités comme l'illustrent~\cite{lee2015relating}, mais également corrélés à la structure sociale~\cite{camarero2008exploring}, qui évoluent tous deux à des échelles de temps d'un ordre différent (supérieur à la dizaine d'année, donc au moins un ordre de grandeur de différence). Ainsi, infrastructure et superstructure déterminent pratiques de mobilité, donnant un rôle important aux réseaux de transports dans celles-ci.
}



\bpar{
Reciprocally, use patterns of transportation networks are the product of daily mobility dynamics, and they adapt to it, while inducing relocations of actives and employments: there exists a co-evolution between transportation and territorial components at the microscopic and mesoscopic scales, which are objects of study in themselves. For example, \cite{fusco2004mobilite} unveils an influence\footnote{Which is interpreted as causal in the sense of bayesian networks.} of mobility on the urban structure, whereas the offer in infrastructure and its properties have however simultaneous effects on mobility and on the urban structure. In the case of freeway networks, \cite{faivr2003} recalls the necessity to construct a framework going beyond the logic of structuring effects on long times, and exhibits also interactions at a large scale that are typical of mobility on which more systematic conclusions can be established, such as an evolution of mobility practices implying a different use of the transportation network. We have thus at a large scale a first strong interdependency between transportation networks and territories, a first scale of co-evolution.
}{
Réciproquement, les motifs d'utilisation des réseaux de transport sont le produit des dynamiques de mobilité quotidiennes, et ceux-ci s'y adaptent, tout en induisant des relocalisations des actifs et emplois : il existe une co-évolution entre transports et composantes territoriales aux échelles microscopiques et mesoscopiques, qui sont un objet d'étude à part entière. Par exemple, \cite{fusco2004mobilite} révèle une influence\footnote{Qui est interprétée comme causale au sens des réseaux Bayesiens.} de la mobilité sur la structure urbaine, l'offre d'infrastructure et ses propriétés ayant cependant des effets simultanément sur la mobilité et sur la structure urbaine. Dans le cas des réseaux autoroutiers, \cite{faivr2003} rappelle la nécessité de construire un cadre d'analyse dépassant la logique des effets structurants sur le temps long, et montre également des interactions à grande échelle propres à la mobilité sur lesquelles des conclusions plus systématiques peuvent être établies, comme une évolution des pratiques de mobilité impliquant une utilisation différente du réseau de transport. Nous avons donc à grande échelle une première interdépendance forte entre réseaux de transports et territoires, une première échelle de co-évolution.
}



\bpar{
It is important to keep in mind the strong contingency of concepts we use here. The co-construction of the concept of mobility with technical solutions that model it with an operational purpose, has been illustrated by~\cite{commenges:tel-00923682} for the French context, which reveals among other things an application of frameworks and methods imported from the United States which were not well adapted to the French context. This contingency means that even the choice of concepts depends of broader conditions than their direct utility, and suggests a global systemic insertion within the \emph{Territorial System}.
}{
Il est important de garder à l'esprit la forte contingence des concepts mobilisés ici. La co-construction du concept de mobilité et des solutions techniques modélisant celle-ci dans un but opérationnel, a été illustrée par~\cite{commenges:tel-00923682} pour le contexte français, qui révèle entre autres une application peu adaptée au contexte français de cadres et méthodes importés des États-Unis. Cette contingence signifie que le choix des concepts même dépend de déterminants plus larges que leur utilité directe, et suggère une inscription systémique globale dans le \emph{Système Territorial}.
}

% Mobility as a service https://maas-alliance.eu/homepage/what-is-maas/


\bpar{
Finally, we have to remark that our approach of mobility is necessarily in a way reductionist, and overshadows socio-economic problematics for example: following \cite{remy2000metropolisation} mobility is indeed a ``virtual field'', i.e. it increases the potentialities offered to individuals, but in a way strongly dependent to the social class and to the socio-economic status. Indeed, mobility practices and political measures acting on transportation are closely linked and can lead to high socio-spatial inequalities in access to urban amenities \cite{gallez2015mobilite} (p.~236). Mobility practices will be indeed indirectly studied in an empirical preliminary study of traffic flows in~\ref{sec:reproducibility}, but we will not be able to treat of their socio-economic aspect: we must stay conscious that this aspect is not taken into account in our work. 
}{
Enfin, nous devons noter que notre approche de la mobilité est nécessairement réductrice, et occulte par exemple des problématiques socio-économiques : selon \cite{remy2000metropolisation} la mobilité est bien un ``champ de virtualité'', c'est-à-dire qu'elle accroit les potentialités offertes aux individus, mais de manière fortement dépendante à la classe sociale et au statut socio-économique. En effet, les pratiques de mobilité et les mesures politiques agissant sur le transport sont en lien étroit et peuvent conduire à de fortes inégalités socio-spatiales d'accès au aménités urbaines \cite{gallez2015mobilite} (p.~236). Les pratiques de mobilité seront bien abordées indirectement dans une étude empirique préliminaire des flux de trafic en~\ref{sec:reproducibility}, mais nous ne serons pas en mesure d'aborder leur aspect socio-économique : il faut ainsi rester conscient que cet aspect n'est pas pris en compte dans notre travail.
}

% en fait pourrait etre une approche entiere (cf Morency) : dans quelle mesure relocs emergent ; erreur des Luti de considerer uniquement gain de temps.



\subsubsection{Transportation and accessibility}{Transports et accessibilité}


% lien entre transports et accessibilite ; potentielle implication de l'accessibilite dans les transformations territoriales.

\bpar{
The concept of \emph{accessibility} is fundamental to our question, since it is positioned at the exact crossroad of networks and territories. Based on the ability to access a place through a transportation network (that can take into account the speed, the difficulty to travel), it is generally defined as a spatial interaction potential\footnote{And often generalized as a \emph{functional accessibility}, for example employments accessible to the actives in one place. Spatial interaction potentials that are expressed in gravity laws can also be understood in the same way.}~\cite{bavoux2005geographie}. It was initially introduced in this form by~\cite{hansen1959accessibility}, with the aim to be applied to planning. Various formulations and formalizations of corresponding indicators have been proposed. It was shown that these enter the same theoretical frame. Indeed, \cite{weibull1976axiomatic} develops an axiomatic approach to accessibility, i.e. proposing to characterize it starting from a minimal number of fundamental hypothesis (axioms). \cite{miller1999measuring} takes the same frame and shows that it includes three classical ways to view accessibility. These are respectively the one based on \emph{Time Geography} and constraints, the one on utility measures for the user, and the one on an average travel time. Corresponding measures are derived within an unified mathematical framework, what allows both a theoretical and operational link between approaches of the concept that are a priori different.
}{
Le concept d'\emph{accessibilité} est fondamental pour notre question, puisqu'il se positionne à la croisée même des réseaux et des territoires. Basée sur la possibilité d'accéder un lieu par un réseau de transport (pouvant prendre en compte la vitesse, la difficulté de se déplacer), elle est généralement définie comme un potentiel d'interaction spatiale\footnote{Et souvent généralisée comme une \emph{accessibilité fonctionnelle}, par exemple les emplois accessibles aux actifs d'un lieu. Les potentiels d'interaction spatiale s'exprimant dans les lois gravitaires peuvent aussi être compris de cette façon.}~\cite{bavoux2005geographie}. Elle a été introduite sous cette forme initialement par~\cite{hansen1959accessibility}, dans un but d'application à la planification. Diverses formulations et formalisations d'indicateurs correspondants ont été proposées. Il a été montré que celles-ci rentrent dans le même cadre théorique. En effet, \cite{weibull1976axiomatic} développe une approche axiomatique de l'accessibilité, c'est-à-dire proposant de la caractériser à partir d'un nombre minimal d'hypothèses fondamentales (les axiomes). \cite{miller1999measuring} reprend ce cadre et montre qu'il englobe trois façons classiques de comprendre l'accessibilité. Celles-ci sont respectivement celle basée sur la \emph{Time Geography} et les contraintes, celle sur les mesures d'utilité pour l'utilisateur, et celle sur un temps de trajet moyen. Les mesures correspondantes sont dérivées dans un cadre mathématique unifié, ce qui permet un lien à la fois théorique et opérationnel entre des approches du concept a priori différentes.
}


\bpar{
We can first see to what extent accessibility patterns induce a evolution of the network. This concept is often used as a planning tool or as an explicative variable for the localization of agents, since it is for example a good indicator of the quantity of people concerned by a transportation project. 
}{
Nous pouvons voir dans un premier temps dans quelle mesure des motifs d'accessibilité induisent une évolution du réseau. Ce concept est souvent utilisé comme un outil de planification ou comme une variable explicative de localisation des agents, puisqu'il s'agit par exemple d'un bon indicateur pour la quantité de personnes affectées par un projet de transport.
}

\bpar{
Recent debates on the planning of \emph{Grand Paris Express}~\cite{mangin2013paris}, this new metropolitan transportation infrastructure planned for the next twenty years, has revealed the opposition between a vision of accessibility as necessary to open up disadvantaged territories, and a vision of accessibility as a driver of economic development for already dynamic areas, both being not necessarily compatible since they correspond to different transportation corridors. One was initially defended by the state in the perspective of competitive clusters, the other by the region in a perspective of territorial equity. These two logics answer naturally to different objective at various levels, and the chosen solution must be a compromise. We will come back on this precise example of the greater Paris in details in the following.
}{
Les débats récents sur la planification du \emph{Grand Paris Express}~\cite{mangin2013paris}, cette nouvelle infrastructure de transport métropolitaine planifiée pour les vingts prochaines années, a révélé l'opposition entre une vision de l'accessibilité comme nécessaire pour désenclaver des territoires désavantagés, et une vision de l'accessibilité comme moteur du développement économique pour des zones déjà dynamiques, les deux n'étant pas forcément compatibles car correspondent à des corridors de transport différents. L'un était initialement porté par l'Etat dans la perspective des pôles de compétitivité, l'autre par la région dans une perspective d'équité territoriale. Ces deux logiques répondent bien sûr à des objectifs différents à plusieurs niveaux, et la solution choisie doit former un compromis. Nous reviendrons sur cet exemple précis du Grand Paris en détails par la suite.
}

\bpar{
This example allows us to suggest an effect of patterns of potential on network evolution: even if this goes through complex social structures (we will also come back on this point in details further), there exists numerous situations where a growth of the transportation network (that can correspond to a topological evolution, i.e. the addition of a link, but also an evolution of link capacities) is directly or indirectly induced by a distribution of the accessibility~\cite{zhang2007economics}. This phenomenon can concern fundamental modifications of the networks or minor modifications: \cite{rouleau1985villages} studies the evolution on long times (from 1800 to 1980) of satellite villages around Paris that have progressively been integrated to its urban fabric and shows both a persistence of the roads and parcels frame, but also local evolutions answering to a logic of connectivity for example, while being part of a more complex evolution context (as in the case of Haussmann). We will designate this abstract process of an answer of the network to a connectivity demand as \emph{potential breakdown}\footnote{In analogy with the phenomenon of \emph{dieletric breakdown} which corresponds to the breakthrough of electrical current in a insulator when the difference of electrical potential is too high.}.
}{
Cet exemple permet de suggérer un effet des motifs de potentiels sur l'évolution du réseau : même si celui-ci passe par des structures sociales complexes (nous y reviendrons aussi en détail plus loin), il existe de nombreuses situations où une croissance du réseau de transport (qui peut se manifester par une évolution topologique, c'est-à-dire l'ajout d'un lien, mais aussi une évolution des capacités des liens) est directement ou indirectement induite par une distribution d'accessibilité~\cite{zhang2007economics}. Ce phénomène peut concerner des modifications fondamentales du réseau comme des modifications mineures : \cite{rouleau1985villages} étudie l'évolution sur le temps long (de 1800 à 1980) des villages satellites à Paris qui ont été progressivement intégrés à son tissu urbain et montre à la fois une persistance de la trame viaire et parcellaire, mais aussi des évolutions locales répondant à des logiques de connectivité par exemple, tout en s'inscrivant dans un cadre d'évolution globale plus complexe (comme dans le cas d'Haussmann). Nous désignerons ce processus abstrait de réponse du réseau à une demande de connectivité par \emph{rupture de potentiel}\footnote{En analogie avec le phénomène de \emph{dieletric breakdown}, ou décharge partielle, qui correspond au passage du courant dans un isolant quand la différence de potentiel électrique est trop grande.}.
}

\bpar{
An other significant process is the impact of an evolution of accessibility through relocations on network use patterns, and more particularly congestion, inducing a modification of capacity (flow that can be carried by network links): this phenomenon is shown in the case of Beijing by~\cite{yang2006transportation}, which unveils modification of the impedance (effective speed in the transportation network) up to 30\%. This can be put in correspondence with processes linked to mobility, even if we are more within meso-meso scales here, i.e. an evolution of the network and relocations on time scales of the order of the decade (the network being slower, of the order of two decades), and on spatial metropolitan scales\footnote{Which correspond to spatial extents from 100 to 200km, but to various urban realities. A metropolis will be a city of importance in a system of cities at a small scale, and will be seen with its functional territory (for example Paris and a consequent part of Ile-de-France). The emergence of new metropolitan forms, such as \emph{Mega-city-regions} which are composed by metropolis of comparable sizes, on a small spatial extent, and with strong interactions, makes this question of the scale more complicated. We will come back on these objects in~\ref{sec:casestudies}.}.
}{
Un autre processus significatif est l'impact d'une évolution de l'accessibilité par relocalisations sur les motifs d'utilisation du réseau, et particulièrement la congestion, induisant une modification de la capacité (flux pouvant être porté par les liens du réseau) : ce phénomène est montré dans le cas de Beijing par~\cite{yang2006transportation}, qui révèle des modifications d'impédance (vitesse effective dans le réseau routier) allant jusqu'à 30\%. Il peut être mis en correspondance avec les processus liés à la mobilité, même si on se situe ici plutôt dans des échelles meso-meso, c'est-à-dire une évolution du réseau et des relocalisations sur des temporalités de l'ordre de la dizaine d'année (le réseau étant plus lent, de l'ordre de la vingtaine d'années), et sur des échelles spatiales métropolitaines\footnote{Qui correspondent à des étendues spatiales de 100 à 200km, mais à diverses réalités urbaines. Une métropole sera une ville d'importance dans un système de villes à petite échelle, et sera vue avec son territoire fonctionnel (par exemple Paris et une grande partie de l'Ile-de-France). L'émergence de nouvelles formes métropolitaines, comme les \emph{Mega-city-regions} qui sont composées de métropoles de taille comparable, sur une faible étendue spatiale, et en très forte interaction, complique cette question de l'échelle. Nous reviendrons sur ces objets en~\ref{sec:casestudies}.}.
}


\bpar{
Reciprocally, an evolution of the network implies an immediate reconfiguration of the spatial distribution of accessibilities (in the sense of all existing approaches, since all take the network into account), and also potentially of territorial transformations on a longer time: we finally come back to the debate of structuring effects we already commented on. We have seen that accessibility co-evolves\footnote{The concept applies a priori at different scales, what will be confirmed by the more precise definition we will take at the end of this first part.} with mobility practices, what suggests an effect at this scale. Concerning relocations and distribution of populations, there exists some cases where it is indeed possible to attribute some territorial dynamics to network growth, that we will develop in the following.
}{
Réciproquement, une évolution du réseau implique une reconfiguration immédiate de la distribution spatiale des accessibilités (au sens de l'ensemble des approches existantes, puisque toutes mobilisent le réseau), et aussi potentiellement des transformations territoriales sur une plus longue durée : on rejoint finalement le débat des effets structurants que nous avons déjà commenté. Nous avons vu que l'accessibilité co-évolue\footnote{Le concept s'applique a priori à diverses échelles, ce qui sera confirmé par la définition plus précise que nous prendrons à la fin de cette première partie.} avec les pratiques de mobilité, ce qui suppose un effet à cette échelle. Concernant les relocalisations et la distribution des populations, il existe des cas où il est en effet possible d'attribuer à la croissance du réseau des dynamiques des territoires, que nous allons développer par la suite.
}

\bpar{
\cite{duranton2012urban} show thus at a medium time scale of 20 years for the United States, through the use of instrumental variables\footnote{The method of instrumental variables aims at unveiling causal relations between an explicative and an explicated variable. The choice of a third variable, called the instrumental variable, must be done such that it influences only the explicative variable but not the explicated variable, in a sense an exogenous shock.}, that accessibility growth in a city causes the growth of employments. On a similar time scale, but at the spatial scale of the country for Sweden, \cite{johansson1993infrastructure} show the local accessibility (``intra-regional'') and global accessibility (``inter-regional'') explains the growth of production and of the productivity of companies. \cite{doi:10.1080/01441647.2016.1168887} proceed to a systematic review of empirical studies of impacts at a medium term of transportation infrastructures, and show that an urban densification at the proximity of new infrastructures is highly probable, being residential in the case of a railway infrastructure and for employments and industrial activity in the case of a road infrastructure\footnote{The studies reviewed cover mainly the second half of the 20th century and Europe, the United States and East Asia. It is important to keep in mind that even if they are relatively general, conclusions must always be contextualized.}. Similarly, it is possible to show strong effects of the presence of infrastructures for particular types of land-use: \cite{nilsson2016measuring} show it for example for fast foods in two cities in the United States, by showing statistically that the access to an important infrastructure induces a spatial aggregation of commerces. 
}{
\cite{duranton2012urban} montrent ainsi à une échelle de temps moyenne de 20 ans pour les États-unis, par l'utilisation de variables instrumentales\footnote{La méthode des variables instrumentales permet de dégager des relations causales entre une variable explicative et une variable expliquée. Le choix d'une troisième variable, appelée variable instrumentale, doit être fait tel que celle-ci n'influence que la variable explicative mais pas la variable expliquée, en quelque sorte un choc exogène.}, que la croissance de l'accessibilité dans une ville cause une croissance de l'emploi. Sur une échelle temporelle similaire, mais à l'échelle spatiale du pays pour la Suède, \cite{johansson1993infrastructure} montre que l'accessibilité locale (``intra-régionale'') et globale (``inter-régionale'') explique la croissance de la production et la productivité des entreprises. \cite{doi:10.1080/01441647.2016.1168887} procèdent à une revue systématique des études empiriques des impacts à moyen terme des infrastructures de transport, et montrent qu'une densification urbaine à proximité des nouvelles infrastructures est très probable, celle-ci étant résidentielle dans le cas d'une infrastructure ferroviaire et pour les emplois et l'activité industrielle et commerciale dans le cas d'une infrastructure routière\footnote{Les études revues couvrent majoritairement la seconde moitié du 20ème siècle et l'Europe, les Etats-Unis et l'Asie de l'Est. Il est donc important de garder à l'esprit que même relativement générales, les conclusions doivent toujours être contextualisées.}. De même, il est possible de montrer des effets forts de la présence d'infrastructures pour des types particuliers d'usage du sol : \cite{nilsson2016measuring} l'illustrent par exemple pour les fast food dans deux villes aux Etats-Unis, en montrant statistiquement que l'accès à une infrastructure importante induit une agrégation spatiale des commerces.
}

\bpar{
The latest examples suggest the potential existence of effects of accessibility, and thus of the network, on territorial dynamics. In some cases, structuring effects are thus present. But these are always links to the precise context and also to scales. This allows us to make the transition to concepts linked by dynamics of urban systems on long times.
}{
Ces derniers exemples suggèrent l'existence potentielle d'effets de l'accessibilité, et donc du réseau, sur les dynamiques territoriales. Dans certain cas, les effets structurants sont ainsi présents. Mais ceux-ci sont toujours liés au contexte précis ainsi qu'aux échelles. Cela nous permet de faire la transition vers les concepts liés aux dynamiques des systèmes urbains sur le temps long.
}





\subsubsection{Transportation and urban systems}{Transports et systèmes urbains}

\bpar{
The third conceptual entry on interactions between networks and territories, and which will be particularly linked to the idea of co-evolution, is the one of urban systems, at a small spatial scale and on long times. We will designate the concept by \emph{structural dynamics of the urban system}.
}{
La troisième entrée conceptuelle sur les interactions entre réseaux et territoires, et qui sera particulièrement liée à l'idée de co-évolution, est celle par les systèmes urbains, à petite échelle spatiale et sur le temps long. Nous désignerons le concept par \emph{dynamique structurelle du système urbain}.
}



\bpar{
The evolutive urban theory considers systems of cities as systems of systems at multiple scale, from the intra-urban microscopic level, to the macroscopic level of the whole system, through the mesoscopic level of the city~\cite{pumain2008socio}. These systems are complex, dynamical, and adaptive: their components \emph{co-evolve} and the system answers to internal or external perturbations by modifying its structure and its dynamics. We will largely develop the multiple implications of this approach all along our work, and retain here processes of interactions between cities. These interactions consist in material or informational exchanges, and the diffusion of innovation is therein a crucial component~\cite{pumain2010theorie}. These are necessarily carried by physical networks, and more particularly transportation networks. We expect thus from a theoretical point of view strong interdependencies between cities and transportation networks at these scales, i.e. a co-evolution.
}{
La théorie évolutive des villes considère les systèmes de villes comme des systèmes de systèmes à de multiples échelles, du niveau microscopique intra-urbain, au niveau macroscopique du système entier, par le niveau mesoscopique de la ville~\cite{pumain2008socio}. Ces systèmes sont complexes, dynamiques, et adaptatifs : leur composants \emph{co-évoluent} et le système répond à des perturbations intérieures ou extérieures par des modifications de sa structure et de sa dynamique. Nous développerons longuement les multiples implications de cette approche tout au long de notre travail, et retenons ici les processus d'interactions entre villes. Ces interactions consistent en des échanges informationnels ou matériels, et la diffusion de l'innovation en est une composante cruciale~\cite{pumain2010theorie}. Elles sont nécessairement portées par les réseaux physiques, et plus particulièrement les réseaux de transport. On s'attend ainsi du point de vue théorique à une interdépendance forte entre villes et réseaux de transport à ces échelles, c'est-à-dire à une co-évolution.
}


\bpar{
From the empirical point of view, it has already been shown: \cite{bretagnolle:tel-00459720} reveals an increasing correlation in time between urban hierarchy and the hierarchy of temporal accessibility for the French railway network (which is a priori clearer for this measure than for integrated measures of accessibility that are prone to auto-correlation as we will see in~\ref{sec:causalityregimes}). This correlation is a witness of positive feedbacks between urban ranks and network centralities. Different regimes in space and times has been identified: for the evolution of the French railway network, a first phase of adaptation of the network to the existing urban configuration was followed by a phase of co-evolution, in the sense that causal relations became difficult to identify. The impact of the contraction of space-time by networks on patterns of growth potential had already been shown for Europe with an exploratory analysis in~\cite{bretagnolle1998space}.
}{
Du point de vue empirique, celle-ci a déjà été mise en valeur : \cite{bretagnolle:tel-00459720} souligne une corrélation croissante dans le temps entre la hiérarchie urbaine et la hiérarchie de l'accessibilité temporelle pour le réseau ferroviaire français (a priori plus claire pour cette mesure que pour les mesures intégrées d'accessibilité soumises à l'auto-corrélation comme nous le verrons en~\ref{sec:causalityregimes}). Celle-ci est un marqueur de rétroactions positives entre le rang urbain et la centralité de réseau. Différents régimes dans le temps et l'espace ont été identifiés : pour l'évolution du réseau ferroviaire français, une première phase d'adaptation du réseau à la configuration urbaine existante a été suivie par une phase de co-évolution, au sens où les relations causales sont devenues difficiles à identifier. L'impact de la contraction de l'espace-temps par les réseaux sur le potentiel de croissance des villes avait déjà été montré pour l'Europe par des analyses exploratoires dans~\cite{bretagnolle1998space}.
}

\bpar{
Modeling results by~\cite{bretagnolle2010comparer}, and more particularly the different parametrizations of the Simpop2 model\footnote{The generic structure of the Simpop2 model is the following~\cite{pumain2008socio}: cities are characterized by their population ad their wealth; they product goods according to their economic profile; interactions between cities produce exchanges, determined by the offer and demand functions; populations evolve according to wealth after exchanges.}, show that the evolution of the railway network in the United States has followed a rather different dynamic, without hierarchical diffusion, shaping locally urban growth in some cases. This particular context of conquest of a space empty of infrastructures implies a specific regime for the territorial system. Other contexts reveal different impacts of the network at short and long term: \cite{berger2017locomotives} study the impact of the construction of the Swedish railway network on the growth of urban populations, from 1800 to 2010, and find an immediate causal effect of the accessibility increase on population growth, followed on long times of a strong inertia for population hierarchy. In each case, we indeed observe the existence of \emph{structural dynamics} on long times, which correspond to the slow dynamics of the urban system structure, and witness in that sense of \emph{structuring effects on long times} as~\cite{pumain2014effets} puts it.
}{
Les résultats de modélisation par~\cite{bretagnolle2010comparer}, et plus particulièrement les paramétrisations différentes du modèle Simpop2\footnote{La structure générique du modèle Simpop2 est la suivante~\cite{pumain2008socio} : les villes sont caractérisées par leur population et leur richesse ; produisent des biens selon leur profil économique ; les interactions entre villes produisent des échanges, déterminés par les fonctions d'offre et demande ; les populations évoluent selon la richesse après échanges.}, montrent que l'evolution du réseau ferroviaire aux États-unis a suivi une dynamique bien différente, sans diffusion hiérarchique, donnant forme localement à la croissance urbaine dans certains cas. Ce contexte particulier de conquête d'un espace vierge d'infrastructures implique un régime spécifique pour le système territorial. D'autres contextes révèlent des impacts différents du réseau à court et long terme : \cite{berger2017locomotives} étudient l'impact de l'établissement du réseau ferroviaire suédois sur la croissance des populations urbaines, de 1800 à 2010, et trouvent un effet causal immédiat de la croissance de l'accessibilité sur la croissance de la population, suivi sur le temps long d'une forte inertie de la hiérarchie des populations. Dans chaque cas, on a bien existence de \emph{dynamiques structurelles} sur le temps long, qui correspondent aux dynamiques lentes de la structure du système urbain, et témoignent en ce sens d'\emph{effets structurants sur le temps long} comme le souligne~\cite{pumain2014effets}.
}




\bpar{
We must be careful to differentiate the latest from the structuring effects previously mentioned which are subject to debates. At the level of the urban system, it is relevant to globally follow trajectories that were possible, and locally the effect has necessarily a probabilistic aspect. Moreover, we insist on the role of path-dependency for trajectories of urban systems: for example the existence in France of a previous system of cities and network (postal roads) has strongly influenced the development of the railway network, or as \cite{berger2017locomotives} showed for Sweden. The same way, \cite{doi:10.1068/b39089} highlight the importance of historical events in coupled dynamics of the road network and territories, historical shocks that can be seen as exogenous and inducing bifurcations of the system that accentuate the effect of path-dependency. Therefore, for these structural dynamics on long times, forecasting can difficultly be considered.
}{
Il s'agit bien de différencier ces derniers des effets structurants sujets des débats mentionnés précédemment. Au niveau du système urbain, il est pertinent de suivre globalement des trajectoires qui étaient possibles, et localement l'effet a nécessairement un aspect probabiliste. D'autre part, il faut mettre l'accent sur le rôle de la dépendance au chemin pour les trajectoires des systèmes urbains : par exemple la présence en France d'un système préalable de villes et de réseau (routes postales) a fortement influencé le développement du réseau ferré, ou comme \cite{berger2017locomotives} l'ont montré pour la Suède. De même, \cite{doi:10.1068/b39089} soulignent l'importance des évènements historiques dans les dynamiques couplées du réseau routier et des territoires, choc historiques pouvant être vus comme exogènes et induisant des bifurcations du système qui accentuent l'effet de la dépendance au chemin. Ainsi, pour ces dynamiques de structure sur le temps long, des prévisions ne sont guère envisageables.
}

\bpar{
This third approach allowed us to unveil a complementary point of view on co-evolution, at an other scale.
}{
Cette troisième approche nous a permis de dégager un point de vue complémentaire de la co-évolution, à une autre échelle.
}


%%%%%
% digression with no future, or eventually in a DynSys-ABM opening ?
%\bpar{
%(different \emph{regimes} in the sense of settlement systems transitions introduced in the current ANR Research project TransMonDyn, as a transition can be understood as a change of stationarity for meta-parameters of a general dynamic). In terms of dynamical systems formulation, it is equivalent to ask if dynamics of attractors (long time scale components) obey similar equations as the position and nature of attractors for a stochastic dynamical system that give its current regime, in particular if it is in a divergent state (positive local Liapounov exponent) or is converging towards stable mechanisms~\cite{sanders1992systeme}.
%}{
%, c'est à des \emph{régimes} différents au sens des transitions des systèmes de peuplement\comment[FL]{concept non connu}, puisqu'une transition entre deux régimes peut être comprise comme un changement de stationnarité des méta-paramètres d'une dynamique plus générale. En termes de systèmes dynamiques, cela revient à se demander si les dynamiques des ensembles de catastrophes (composantes à plus grandes échelles temporelles) obéissent à des équations similaires à la position et nature des attracteurs pour un système dynamique stochastique qui donne son régime courant, en particulier si le système est dans un état local divergent (exposant de Liapounov local positif\comment[FL]{ce n'est pas comprehensible}) ou en train de converger vers des mécanismes stables\comment[FL]{c'est flou}~\cite{sanders1992systeme}.
%}



%\subsubsection{Scaling laws}{Lois d'échelles}
\subsubsection{Links between scales suggested by Scaling Laws}{Des liens entre échelles suggérés par les Lois d'Échelle}

% avant la transition : rappeler que le schema par échelles est simplifié, mais grille de lecture permettant de construire.




\bpar{
Our framework with successive scales, that yield a reasonable correspondence between spatial and temporal scales, and also to associate the corresponding concepts, does naturally not capture the full range of possible processes: these that would fundamentally be multi-scalar, for example by implying the emergence of their own intermediate level, are not evoked. These are important and we will come back to them below. First we propose to establish a conceptual link between scales by the intermediary of \emph{scaling laws} (that we understand in the general sense given in introduction). This link aims in particular at going beyond a reductionist reading through the compartmentalization of scales.
}{
Notre grille de lecture par échelles progressives, qui permet de dégager une assez bonne correspondance entre échelle spatiale et temporelle, ainsi que d'y associer les concepts adaptés, ne capture bien sûr pas l'ensemble des processus possibles : ceux qui seraient fondamentalement multi-échelles, par exemple en impliquant l'émergence de leur propre niveau intermédiaire, ne sont pas évoqués. Ceux-ci sont importants et nous y reviendrons ci-dessous. Dans un premier temps, nous proposons d'effectuer un lien conceptuel entre les échelles par l'intermédiaire des \emph{lois d'échelles} (que nous comprenons au sens général donné en introduction). Ce lien permet en particulier de dépasser une lecture réductrice par cloisonnement d'échelle.
}

\bpar{
Transportation networks are by essence hierarchical, this property depending on scales they are embedded in, and leading to the emergence of scaling laws for their properties. For example, \cite{10.1371/journal.pone.0102007} show empirical scaling properties for a consequent number of metropolitan areas across the world. Indeed, scaling laws reveal the presence of hierarchy within a system, as for size hierarchy for systems of cities expressed by Zipf's law~\cite{nitsch2005zipf} or other urban scaling laws~\cite{arcaute2015constructing,bettencourt2016urban}, what suggests a particular structure for these systems. We can expect to find it again in interaction processes themselves. Transportation network topology follows such laws for the distribution of its local measures such as centrality~\cite{samaniego2008cities}, these being directly linked to accessibility patterns at different scales. Furthermore, network topology is among the factors inducing the hierarchy of use, since it influences congestion negative externalities, in relation with the spatial distribution of land-use~\cite{Tsekeris20131}. Thus, considering scaling laws for transportation networks, and more generally for territorial systems, is first a signature of the complexity of these systems, and secondly yields an implicit link between scales.
}{
Les réseaux de transport sont par essence hiérarchiques, cette propriété dépendant des échelles dans lesquelles ils sont intégrés, et se manifestant par l'émergence de lois d'échelles pour leurs propriétés. Par exemple, \cite{10.1371/journal.pone.0102007} montrent empiriquement des propriétés de loi d'échelle pour un nombre conséquent d'aires métropolitaines à travers la planète. Or les lois d'échelle révèlent la présence de hiérarchies dans un système, comme pour la hiérarchie de tailles dans les systèmes de villes exprimée par la loi de Zipf~\cite{nitsch2005zipf} ou d'autres lois d'échelles urbaines~\cite{arcaute2015constructing,bettencourt2016urban}, ce qui suggère une structure particulière pour ces systèmes. Nous pouvons nous attendre à la retrouver dans les processus d'interaction eux-mêmes. La topologie du réseau de transport suit de telles lois pour la distribution de ses mesures locales comme la centralité~\cite{samaniego2008cities}, celles-ci étant directement liées au motifs d'accessibilité à différentes échelles. De plus, la topologie du réseau fait partie des facteurs induisant la hiérarchie d'usage, se retrouvant dans les externalités négatives de congestion, en relation avec la distribution spatiale de l'usage du sol~\cite{Tsekeris20131}. Ainsi, la considération des lois d'échelles pour les réseaux de transport, et plus généralement pour les systèmes territoriaux, est dans un premier temps une signature de la complexité de ces systèmes, et permet dans un second temps un lien implicite entre les échelles.
}



%%%%%%
% transition : synthese breve et preliminaire des processus mis en evidence


\subsubsection{Scales: a synthesis}{Echelles : synthèse}


\bpar{
To recall our framework by scales, we propose the Table~\ref{tab:networkterritories:scales}. Designations and orders of magnitude of temporal and spatial scales are of course indicative, such as key concept that are indeed the ones that allowed us to enter these scales. We also give references that illustrate corresponding conceptual frameworks. This table will however be useful to keep in mind  the typical scales to which we refer.
}{
Pour rappeler notre cadre de lecture par échelles, nous proposons la Table~\ref{tab:networkterritories:scales}. Les appellations ainsi que les ordres de grandeur des échelles temporelles et spatiales sont évidemment indicatifs, de même que les concepts clés qui sont en fait ceux qui nous ont permis d'entrer dans ces échelles. Nous donnons également des références illustrant des cadres conceptuels correspondant. Ce tableau nous sera toutefois utile pour garder à l'esprit les échelles typiques auxquelles nous ferons référence.
}


%%%%%%%%%%%%%
\begin{table}
\caption[Synthesis of the approach by scales][Synthèse de la lecture par échelle des interactions entre réseaux de transport et territoires]{\textbf{Synthesis of the approach by scales of interactions between transportation networks and territories.} References give a possible theoretical frame for each scale.\label{tab:networkterritories:scales}}{\textbf{Synthèse de la lecture par échelle des interactions entre réseaux de transport et territoires.} Les références donnent un cadre théorique possible pour chaque échelle.\label{tab:networkterritories:scales}}
\bpar{
\begin{tabular}{|p{1.7cm}|p{3.7cm}|p{3.7cm}|p{3.7cm}|p{2.7cm}|}\hline
	Scale & Spatial scale & Temporal scale & Concept & Reference \\ \hline
	Micro & Intra-urban (10km) & daily (1d) & Mobility practices & \cite{hall2005reconsidering} \\ \hline
	Meso & Metropolitan (100km) & Decade (10y) & Metropolitan reconfiguration & \cite{wegener2004land} \\\hline
	Macro & Regional (500km) & Century (100y) & Structural dynamic on long times & \cite{pumain1997pour} \\\hline
\end{tabular}
}{
\begin{tabular}{|p{1.7cm}|p{3.7cm}|p{3.7cm}|p{3.7cm}|p{2.7cm}|}\hline
	Echelle & Echelle spatiale & Echelle temporelle & Concept & Référence \\ \hline
	Micro & Intra-urbaine (10km) & Journalière (1j) & Pratiques de mobilité & \cite{hall2005reconsidering} \\ \hline
	Meso & Métropolitaine (100km) & Décade (10ans) & Reconfiguration métropolitaine & \cite{wegener2004land} \\\hline
	Macro & Régionale (500km) & Siècle (100ans) & Dynamique structurelle sur le temps long & \cite{pumain1997pour} \\\hline
\end{tabular}
}
\end{table}
%%%%%%%%%%%%%





\subsubsection{Processus: a synthesis}{Processus : synthèse}

\bpar{
At this stage, we can already propose a preliminary of the interaction processes we introduced. A more exhaustive typology will be possible at the end of this chapter.
}{
A ce stade, nous pouvons d'ores et déjà proposer une synthèse préliminaire des processus d'interaction que nous avons introduit. Une typologie plus exhaustive sera possible à l'issue du chapitre.
}

\bpar{
Thus, territorial components can act on networks by:
}{
Ainsi, des composantes territoriales peuvent agir sur les réseaux de transport par :
}

\bpar{
\begin{itemize}
	\item Impact of mobility patterns on impedances and capacities
	\item Potential breakdown, emergence of centralities
	\item Hierarchical selection of accessibility
	\item Systemic structural effects and bifurcations 
\end{itemize}
}{
\begin{itemize}
	\item Impact des motifs de mobilité sur les impédances et les capacités
	\item Rupture de potentiel, émergence de centralités
	\item Sélection hiérarchique de l'accessibilité
	\item Effets systémiques structurels et bifurcations
\end{itemize}
}

\bpar{
Reciprocally, processes where network properties act on territories include:
}{
Réciproquement, des processus où les propriétés des réseaux agissent sur les territoires incluent :
}

\bpar{
\begin{itemize}
	\item Relocations induced by mobility constraints
	\item Land-use changes due to a transportation infrastructure
	\item Accessibility patterns induced by networks, that can induce relocations
	\item Interactions between territories carried by network, including the tunnel effect when these are telescoped
\end{itemize}
}{
\begin{itemize}
	\item Relocalisations induite par des contraintes de mobilité
	\item Changement d'usage du sol dû à une infrastructure de transport
    \item Motifs d'accessibilité induits par les réseaux, pouvant induire des relocalisations
	\item Interactions entre territoires portées par les réseaux, incluant l'effet tunnel lorsque celles-ci sont télescopées
\end{itemize}
}

\bpar{
These different processes do not all have the same level of abstraction neither the same scales. We have furthermore hidden some processes already evoked, within which the coupling is stronger and for which the circularity is already present in the ontology, such as processes linked to planning. We will now detail these, what will allow us then to refine the list above and to present it as a typology after having enriched it with empirical studies.
}{
Ces différents processus n'ont pas tous le même statut d'abstraction ni les mêmes échelles. Nous avons de plus volontairement occulté des processus déjà évoqués, au sein desquels le couplage est plus fort et pour lesquels la circularité est déjà présente dans l'ontologie, comme les processus liés à la planification. Nous allons détailler à présent ceux-ci, ce qui nous permettra par la suite de raffiner la liste ci-dessus et de la présenter sous forme de typologie après l'avoir enrichie par des études empiriques.
}



%%%%%%%%%%%%%%%%%%%
\subsection{From interactions to co-evolution}{Des interactions à la co-évolution}


\bpar{
At this stage, we have identified processes of interaction between transportation networks and territories that play a significant role in the complexity of territorial systems. In the frame of our preliminary definition of a territorial system, this question can be reformulated as the study of networked territorial systems with an emphasis on the role of transportation networks. We have seen that the extent of spatial and temporal scales spans from daily mobility (micro-micro) to processes on long time in systems of cities (macro-macro), with the possibility of intermediate combinations. The precision of scales that are particularly relevant will be the subject of most of preliminaries (Part 1) and of foundations (Part 2), until chapter~\ref{ch:morphogenesis} that concludes foundations. We now extend this list and give concrete examples in terms of the complexity of interactions.
}{
A ce stade, nous avons identifié des processus d'interaction entre réseaux de transport et territoires jouant un rôle significatif dans la complexité des systèmes territoriaux. Dans le cadre de l'approche d'un système territorial par la définition donnée lors de la construction première des concepts, cette question peut être reformulée comme l'étude de systèmes territoriaux réticulaires, avec une emphase sur le rôle des systèmes de transports. Nous avons vu que l'étendue des échelles spatiales et temporelles va de celle de la mobilité quotidienne (micro-micro) à des processus sur le temps long dans les systèmes de villes (macro-macro), avec la possibilité de combinaisons intermédiaires. La précision des échelles particulièrement pertinentes fera l'objet de la majorité des préliminaires (Partie 1) et des fondations (Partie 2), jusqu'au chapitre~\ref{ch:morphogenesis} qui conclura les fondations. Étendons à présent cette liste et donnons des exemples concrets précisant la complexité des interactions.
}



\subsubsection{Importance of the geographical context}{Importance du contexte géographique}


\bpar{
The contextualization of our question in a particular frame reveals the importance of taking into account the geographical context. The exemple of mountain territories, where constraints on ressources and travel are stronger, shows the richness of possible situations when a generic frame is put in context of a particular case.
}{
La mise en contexte de notre question dans un cadre bien particulier révèle l'importance de la prise en compte du contexte géographique. L'exemple des territoires de montagne, où les contraintes de ressources et de déplacement sont fortes, montre la richesse des situations possibles lorsqu'un schéma générique est mis en contexte dans un cas particulier.
}

\bpar{
For example, on comparable French mountain territories, \cite{berne2008ouverture} shows that reactions to a same context of evolution of the transportation network can lead to very different territorial dynamics, some territories highly benefiting of the increased accessibility, others in the contrary becoming more closed. In the same frame, these possible opposed processes are scrutinized with more details by~\cite{bernier2007dynamiques}, for which he proposes a typology based on the opening potential both of territorial dynamics and network dynamics: for example, a territory can exhibit rich opportunities to be attractive, such as touristic opportunities, but keep a low accessibility. Reciprocally, he gives the illustration of custom constraints that can impede the opening potential of a performant infrastructure.
}{
Par exemple, sur des territoires de montagne français comparables, \cite{berne2008ouverture} montre que les réactions à un même contexte d'évolution du réseau de transport peuvent mener à des dynamiques territoriales très diverses, certains trouvant de forts bénéfices de l'accessibilité accrue, d'autres au contraire devenant plus fermés. Dans le même cadre, ces potentiels processus antagonistes sont examinés plus en détail par~\cite{bernier2007dynamiques}, pour lesquels il propose une typologie basée sur le potentiel d'ouverture à la fois des dynamiques territoriales et des dynamiques des réseaux : par exemple, un territoire peut présenter de riches opportunités d'attractivité, comme des opportunités touristiques, tout en gardant une faible accessibilité. Réciproquement, il donne l'illustration des contraintes douanières pouvant freiner le potentiel d'ouverture d'une infrastructure performante.
}


\bpar{
Similarly to approaches considering systems of cities, \cite{torricelli2002traversees} shows how it is possible in that context to establish a link between the nature of transportation flows and the local development of the urban system: cities in the mountains have first emerged as waypoints on paths to mountain passes, then have lost their importance when roads came into existence. The construction of railways gave them a new dynamic, through tourism and industry, and finally freeways has more recently inducted a loss of urban structure through peri-urbanization for example. Thus, structural dynamics on long time are particular, as a consequence of the geographical context.
}{
En écho aux approches par systèmes de villes, \cite{torricelli2002traversees} montre comment dans ce contexte il est possible de faire un lien entre nature des flux de transport et développement local du système urbain : les villes de montagne ont d'abord émergé comme point de passage sur les chemins de col, puis ont perdu de leur importance avec l'avènement des routes. L'arrivée du chemin de fer a pu les re-dynamiser, par le tourisme et l'industrie, et enfin l'autoroute a encore plus récemment induit une déstructuration par des effets de périurbanisation par exemple. Ainsi, les dynamiques structurelles sur le temps long sont particulières, en conséquence du contexte géographique.
}




\subsubsection{Planification processes}{Processus de planification}

\bpar{
As we already suggested, potential impacts of territorial dynamics on networks imply processes at different levels. This way, infrastructure projects are generally planned\footnote{We will use the term \emph{planning} in general, territorial or urban, of an infrastructure project, when a project and its plan is willingly elaborated by a planning stakeholder, with an aim at transforming space according to some motivations depending on the stakeholder and on its interactions with other stakeholders.}, in order to fulfil some objectives fixed generally by institutional actors. These objects bring progressively the concept of governance, but let first give some illustrations of planned projects. 
}{
Comme nous l'avons déjà suggéré, les potentiels impacts des dynamiques territoriales sur les réseaux impliquent des processus à plusieurs niveaux. Ainsi, les projets d'infrastructure sont généralement planifiés\footnote{Nous parlerons de \emph{planification} en général, urbaine, territoriale, d'un projet d'infrastructure, pour désigner la conception volontaire d'un projet et d'un plan par un acteur d'aménagement, dans le but de transformer l'espace selon certaines motivations propres à l'acteur et à ses interactions avec les autres acteurs.}, afin de répondre à certains objectifs fixés par des acteurs souvent institutionnels. Ces objets nous amènent progressivement vers le concept de gouvernance, mais prenons d'abord un instant pour illustrer des projets planifiés.
}

\bpar{
The example of the failure in the planning of the Ciudad Real airport in Spain shows that the answer to a planned infrastructure is far from systematic. The explanations to it are probably a complex combination of diverse factors, difficult to disentangle. \cite{otamendi2008selection} predicted before the opening of the airport a complex management due to the dimension of expected flows and proposed a suited model, but the order of magnitude of effective flows where closer to thousands than millions that were planned and the airport rapidly closed. It is complicated to know the reason of the failure, if it is an optimism of the regional level of polycentricity (the airport is halfway between Madrid and Seville), the lack of construction of a train station on the high speed line, or just purely economical factors.
}{
L'exemple de l'échec de planification de l'aéroport de Ciudad Real en Espagne montre que la réponse à une infrastructure planifiée n'est pas systématique. Les explications à celui-ci découlent très probablement d'une combinaison complexe de multiples facteurs, difficiles à séparer. \cite{otamendi2008selection} prédisaient avant l'ouverture de l'aéroport une gestion complexe due à la dimension des flux attendus et proposaient un modèle approprié, or les ordres de grandeurs de flux effectifs étaient plus proches des milliers que des millions planifiés et l'aéroport a rapidement fermé. Il est compliqué de savoir la raison de l'échec, s'il s'agit de l'optimisme quand au polycentrisme régional (l'aéroport est à mi-chemin de Madrid et Séville), la non-réalisation de la gare sur la ligne à grande vitesse, ou des facteurs purement économiques.
}


\bpar{
\cite{heddebaut:hal-01355621}\footnote{The possible pun with the ambiguous title on the existence of the ``Tunnel effect'' recalls the effect through which an infrastructure traversing a territory has no interaction with it.} show for the impact of infrastructures on the long term, in the case of the Channel tunnel\footnote{Put into service in 1994 between Calais in France and Folkestone in the United Kingdom, this railway underwater tunnel with a length of 50km establishes a physical link between the continent and the UK.}, through an analysis of investments and political actions in time, that the effect effectively observed for the Nord-Pas-de-Calais region such as a gain in centrality and in European visibility, are in strong distorsion with the initial justifications of the project, and that the renewing of stakeholders implies that the project is not accompanied on the long time what makes its impact more uncertain. We rejoin the idea advocated by \cite{espacegeo2014effets} according to which some ``structure effects'' effectively exist but that these can be observed on the long time in terms of the dynamic of the system for which a short time local vision does not make much sense. At the intra-urban scale, \cite{fritsch2007infrastructures} takes the example of the tramway in Nantes to show, through a localized study of urban transformations in the nieghborhood of a new line, that urban densification dynamics are far from what was expected from deciders and planners, i.e. a strong correspondence between the proximity to the line and a densification.
}{
\cite{heddebaut:hal-01355621}\footnote{Le possible jeu de mot par le titre ambigu sur l'existence du ``Tunnel effect'' rappelle l'effet tunnel, qui réside en la non-interaction d'une infrastructure sur un territoire le traversant sans s'y arrêter.} montrent pour l'impact des infrastructures sur le long terme, dans le cas du tunnel sous la Manche\footnote{Mis en service en 1994 entre Calais en France et Folkestone au Royaume-uni, ce tunnel ferroviaire sous-marin de 50km permet une liaison physique entre l'Europe continentale et le Royaume-uni.}, par une analyse des investissements et des politiques dans le temps, que les effets effectivement constatés pour la région Nord-Pas-de-Calais comme un gain de centralité et de visibilité au niveau Européen, sont en fort décalage avec les discours justifiant le projet, et que le renouvellement des acteurs implique un non-accompagnement du projet sur le long terme, rendant son impact plus hasardeux. Nous rejoignons l'idée défendue par \cite{espacegeo2014effets} selon laquelle des ``effets de structure'' effectivement existent mais que ceux-ci se manifestent sur le temps long en termes de dynamiques systémiques pour lesquelles une vision locale courte n'a que peu de sens. A l'échelle intra-urbaine, \cite{fritsch2007infrastructures} prend l'exemple du tramway de Nantes pour montrer, par une étude localisée des transformations urbaines à proximité d'une nouvelle ligne, que les dynamiques de densification urbaine sont en décalage avec ce qu'en attendaient les élus et planificateurs, c'est-à-dire une association forte entre proximité à la ligne et densification.
}

\bpar{
These exemples confirm that the understanding of effects of territories on infrastructures imply to take into the concept of \emph{governance}.
}{
Ces exemples confirment que la compréhension des effets des territoires sur les infrastructures impliquent la prise en compte du concept de \emph{gouvernance}.
}




\subsubsection{Governance}{Gouvernance}


\bpar{
The development of a transportation network necessitate actors disposing of both concrete and economic capabilities to proceed to the construction, and furthermore having the legitimacy to lead this development. This must thus necessarily be actors of the social superstructure, that can be different levels of public governance, sometimes associated with private actors. The concept of \emph{governance}, that we understand as the management of an organisation with common ressources with targets linked to the interest of the concerned community (these can be defined in different ways, for example in a \emph{top-down} manner by governance actors or in a \emph{bottom-up} manner by consulting the agents concerned with the decision), is then crucial to understand the evolution of transportation projects and thus of transportation networks. We will use the term of \emph{territorial governance} when decision imply directly or indirectly components of territorial systems. 
}{
Le développement d'un réseau de transport nécessite des acteurs disposant à la fois des moyens concrets et économiques de mener à bien la construction, et d'autre part ayant la légitimité de mener ce développement. Il s'agit donc nécessairement d'acteurs de la superstructure sociale, pouvant être différents niveaux de pouvoirs publics, parfois associés à des acteurs privés. Le concept de \emph{gouvernance}, que nous comprenons comme la gestion d'une organisation disposant de ressources communes dans des buts liés à l'intérêt de la communauté concernée (pouvant être définis de différentes façons, par exemple de manière \emph{top-down} par les acteurs de gouvernance ou de manière \emph{bottom-up} par consultation des agents concernés par la décision), est alors essentiel pour comprendre l'évolution des projets de transports et donc des réseaux de transport. Nous parlerons de \emph{gouvernance territoriale} lorsque les décisions concernent directement ou indirectement des composantes de systèmes territoriaux. 
}

\bpar{
For example, \cite{offner2000territorial} illustrates the difficulties posed by the deregulation of some networked public services concerning the territorial competences of authorities, and proposes the emergence of a new local regulation for a new compromise between networks and territories.
}{
Par exemple, \cite{offner2000territorial} illustre les difficultés posées par la dérégulation de certains services publics en réseau quant aux compétences territoriales des autorités, et propose l'émergence d'une régulation locale pour un nouveau compromis entre réseaux et territoires.
}

\bpar{
Some aspects of territorial governance can have a significant impact on the development of transportation infrastructures. We can illustrate some for particular cases of the application of \emph{urban models}\footnote{In the sense of planning, i.e. conceptual generic schemas acting as a guide to the planification.}. \cite{deng2007potential} show in the case of Chinese cities that new directives in terms of housing can significantly deteriorate the performance of infrastructures, and that specific actions must be taken to anticipate these negative externalities. These concern in particular the dispositions in terms of \emph{Transit Oriented Development} (TOD). TOD is a particular approach to urban planning that aims at articulating the development of public transportation and urban development. It can be understood as a voluntary co-evolution by developers (administrative authorities and/or planning authorities), in which the articulation is thought and planned. We will come back on TOD during empirical studies in the following.
}{
Certains aspects de la gouvernance territoriale peuvent avoir un impact déterminant sur le développement des infrastructures de transport. Illustrons ceux-ci pour des cas particuliers d'application de \emph{modèles urbains}\footnote{Au sens de la planification, c'est-à-dire de schémas conceptuels génériques permettant de guider une démarche de planification.}. \cite{deng2007potential} montrent dans le cas des villes Chinoises que les nouvelles directives en terme de logement peuvent fortement détériorer la performance des infrastructures, et que des dispositions spécifiques doivent être prises pour anticiper ces externalités négatives. Celles-ci concernent notamment les dispositions en termes de \emph{Transit Oriented Development} (TOD). Le TOD est une approche particulière de l'aménagement urbain visant à articuler développement de l'offre de transport en commun et développement urbain. Il s'agit en quelque sorte d'une co-évolution volontaire de la part des développeurs (autorités administratives et/ou de planification), dans laquelle l'articulation est pensée et planifiée. Nous reviendrons sur le TOD lors d'études empiriques par la suite.
}


\bpar{
These concepts are not new, since they were for example implicit in the planning of new towns in \emph{Ile-de-France}, under a different form since these were strongly zoned (i.e. planned into relatively isolated mono-functional areas) and dependant on the automotive for some districts~\cite{es119}. \cite{l2012ville} give an example of an European project that has explored some implementations of TOD paradigms: planning details such as a quality of the network for active mobility modes at a short range are crucial for the concretization of principles. For example, \cite{lhostis:hal-01179934} use a multi-criteria analysis\footnote{In the frame of decision making for the planning of transportation infrastructures, multi-criteria analysis is an alternative to cost-benefit analysis (that compare projects by aggregating a generalized cost) which allows to take into account multiple dimensions, that are often contradictory (for example construction cost and robustness for a network), and obtain optimal solutions in the Pareto sense.} to understand determining factors in the selection	of stations for the planned city, including urban density and access time to stations. \cite{LIU2014120} show that even if some planning policies do not directly take a positioning as such, particularly in France, they exhibit very similar characteristics as shows the case of Lille.
}{
Ces concepts ne sont pas nouveaux, puisqu'ils étaient implicites par exemple dans l'aménagement des villes nouvelles en Ile-de-France, sous une forme différente car celles-ci étaient également fortement zonées (c'est-à-dire planifiées en zones relativement cloisonnées et mono-fonctionnelles) et dépendantes de l'automobile pour certains quartiers~\cite{es119}. \cite{l2012ville} donnent un exemple de projet européen ayant exploré des mises en pratiques de paradigmes du TOD : des détails d'aménagement comme un réseau de qualité pour les modes actifs à courte portée sont cruciaux pour une concrétisation des principes. Par exemple, \cite{lhostis:hal-01179934} utilisent une analyse multi-critères\footnote{Dans le cadre de l'aide à la décision pour la planification des infrastructures de transport, l'analyse multi-critère est une alternative aux analyses coût-bénéfices (qui comparent des projets en agrégeant un coût généralisé) qui permet de prendre en compte de multiples dimensions, souvent contradictoires (par exemple coût de construction et robustesse pour un réseau), et obtenir des solutions optimales au sens de Pareto.} pour comprendre les facteurs déterminants dans la sélection des stations de la ville planifiée, incluant densité urbaine et temps d'accès aux stations. \cite{LIU2014120} montrent que si certaines politiques de planification, en particulier en France, ne se réclament pas directement de cette approche, leurs caractéristiques sont très similaires comme le révèle le cas de Lille.
}


\bpar{
The articulation between transportation and urban planning must often be operated in a strongly coupled manner to attain the expected objectives, even more when the project is specialized: \cite{larroque2002paris} recall the case of the SK metro in Noisy-le-Grand which unveils a case of a complete dependency of the functionality of the transport to local development. In order to serve a project of a office complex, a specific line with a lightweight equipment is constructed to make a link with the RER station of Mont-d'Est. The real estate project will fail whereas the line is inaugurated in 1993, it will be first regularly maintained and then abandoned without having never been opened to the public. 
}{
L'articulation entre transport et aménagement doit souvent être opérée de façon fortement couplée pour parvenir aux objectifs recherchés, d'autant plus que le projet est spécialisé : \cite{larroque2002paris} rappellent l'anecdote du metro SK de Noisy-le-Grand qui montre un cas de dépendance complète de la fonctionnalité du transport à l'aménagement local. Pour desservir un projet de complexe de bureaux, une ligne spécifique avec une matériel roulant léger est construite pour faire le lien avec la gare RER de Mont-d'Est. Le projet immobilier avortera alors que la ligne est inaugurée en 1993, celle-ci sera d'abord entretenue régulièrement puis laissée a l'abandon sans jamais avoir été ouverte au public.
}



\bpar{
Therefore, governance processes, that manifest themselves in different ways, such as planning, or more particularly as TOD, play an important role in interactions between transportation networks and territories. These add up to our panorama, being of a particular type since they imply their own level of emergence and a strong autonomy.
}{
Ainsi, les processus de gouvernance, qui peuvent se décliner de plusieurs manières, comme ceux de planification, ou plus spécifiques de TOD, jouent un rôle important dans les interactions entre réseaux de transports et territoires. Ceux-ci s'ajoutent à notre panorama, étant d'un type particulier car impliquant leur propre niveau d'émergence et une forte autonomie.
}







\subsubsection{Co-evolution of networks and territories}{Co-évolution des réseaux de transport et des territoires}


\bpar{
This progressive construction allowed us to highlight the complexity of interactions between networks and territories, what suggests the relevance of the particular ontology of \emph{co-evolution} as we defined in introduction. \cite{levinson2011coevolution} insists on the difficulty of understanding the co-evolution between transport and land-use in terms of circular causalities, partly because of the different time scales implied, but also because of the heterogeneity of components. \cite{offner1993effets} uses the term of congruence, that can be understood as systemic dynamics implying correlations that can be spurious or not, that would be a preliminary vision of co-evolution.
}{
Cette construction progressive nous a permis de souligner la complexité des interactions entre réseaux et territoires, ce qui suggère la pertinence de l'ontologie particulière de la \emph{co-évolution} comme nous l'avons définie en introduction. \cite{levinson2011coevolution} insiste sur la difficulté de la compréhension de la co-évolution entre transport et usage du sol en termes de causalités circulaires, en partie à cause des différentes échelles de temps impliquées, mais aussi par l'hétérogénéité des composantes. \cite{offner1993effets} parle de congruence, qu'on peut comprendre comme une dynamique systémique impliquant des corrélations fortuites ou non, ce qui serait une vision préliminaire de la co-évolution.
}



\bpar{
The necessity to go past reducing approaches of structuring effects, together with the capture of the complexity of interactions between networks and territories through their co-evolution, is confirmed by the case of economic effects of high speed lines: \cite{Blanquart2017} proceeds to a both empirical and theoretical review, including grey literature, of studies of this specific case, and concludes, beyond the direct effects linked to the construction on which there is a consensus, that proper effects on a long time seem to be random. This witnesses in fact complex local situations, a large number of conjunctural aspects playing a role in the production of effects, that can then not be attributed to transport only. This review confirms moreover the gap between political and technical narratives preceding transportation projects and the effective posterior analysis, revealed by~\cite{bazin:hal-00615196}. \cite{bazin2007evolution} conduct also a targeted study of the real estate market in Reims in anticipation to the arrival of the \emph{TGV Est}. Through a diachronic analysis for each year between 1999 and 2005, for each district, of the real estate prices and the origin of buyers (locals or from the region of Paris), they conclude that only very localized operations can be directly linked to the TGV, the whole market following a global independent dynamic.
}{
La nécessité de dépasser les approches réductrices des effets structurants, tout en capturant la complexité des interactions entre réseaux et territoires par leur co-évolution, est confirmée par le cas des effets économiques des lignes à grande vitesse : \cite{Blanquart2017} procèdent à une revue à la fois théorique et empirique, incluant la littérature grise, des études de ce cas spécifique, et conclut, au delà des retombées directes liées à la construction sur lesquelles il y a consensus, que les effets propres sur un temps plus longs paraissent aléatoires. Cela témoigne en fait de situations locales complexes, un grand nombre d'aspects conjoncturels entrant en jeu dans la production d'effets, qu'on ne peut alors pas attribuer seulement au transport. Cette revue confirme par ailleurs le décalage entre les discours politiques et techniques prévalant aux projets de transports et les analyses effectives a posteriori, révélé par~\cite{bazin:hal-00615196}. \cite{bazin2007evolution} mènent d'autre part à une étude ciblée du marché immobilier à Reims en anticipation de l'arrivée du TGV Est. En procédant à une analyse diachronique pour chaque année entre 1999 et 2005, par quartier, des prix immobiliers et de la provenance des acheteurs (franciliens ou locaux), ils concluent que seulement des opérations très localisées pouvaient être directement reliées au TGV, l'ensemble du marché répondant à une dynamique globale indépendante.
}




\stars


\bpar{
Thus, our constructive overview, broad and conceived as circular, of interactions between transportation networks and territories, confirms the relevance of the concept of \emph{co-evolution} on the one hand, but suggests on the other hand a more thorough investigation and clarification for it.
}{
Ainsi, notre aperçu constructif, large et voulu circulaire, des interactions entre réseaux de transports et territoires, confirme la pertinence de ce concept de \emph{co-évolution} d'une part, mais suggère d'autre part un approfondissement et une clarification de celui-ci.
}

\bpar{
We have therefore seen in this section that (i) the concept of territory naturally yields the concept of network; (ii) reciprocally, networks can transform territories, following different processes more or less established depending on scales; (iii) there exists a large number of cases and of particular processes for which the relation between networks and territories is imbricated, and for which we can use the term of \emph{co-evolution}.
}{
Nous avons ainsi vu dans cette section que (i) le concept de territoire amène naturellement celui de réseau ; (ii) réciproquement, le réseau peut transformer les territoires, selon différents processus plus ou moins établis selon les échelles ; (iii) qu'il existe un grand nombre de cas et de processus particuliers pour lesquels la relation entre réseau et territoire est imbriquée, et où on peut parler de \emph{co-évolution}.
}

\bpar{
We will aim in the following section at studying more thoroughly in an empirical way various aspects evoked here, to put into perspective and refine the questions we aim at answering here.
}{
Nous nous appliquerons dans la section suivante à approfondir de manière empirique différents aspects abordés ici, pour une mise en situation et un raffinement des questions que nous nous posons.
}



\stars




