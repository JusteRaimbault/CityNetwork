


\newpage


%-------------------------------


\section{Territories and Networks}{Territoires et Réseaux}

\label{sec:networkterritories}


%-------------------------------


Nous commençons par une construction plus précise des concepts mobilisés, qui permet de comprendre comment les concepts de territoire et de réseau sont rapidement en interdépendance forte, impliquant une importance ontologique des interactions entre les objets correspondants. Un développement plus particulier sur les propriétés des réseaux de transport permet d'amener progressivement une vision précise de la \emph{co-évolution}, que nous prendrons jusque là dans son sens préliminaire donné précédemment, c'est à dire 


\subsection{Territories and Networks : There and Back Again}{Territoires et Réseaux : \emph{There and Back Again}}


\paragraph{Territories}{Territoires}


\bpar{
The notion of territory can be taken as a basis to explore the scope of geographical objects we will study. In Ecology, a territory corresponds to a spatial extent occupied by a group of agents or more generally an ecosystem. \emph{Human Territories} are far more complex in the sense of semiotic representations of these that are a central part in the emergence of societies. For \noun{Raffestin} in~\cite{raffestin1988reperes}, the so-called \emph{Human Territoriality} is the ``conjonction of a territorial process with an informational process'', what means that physical occupation and exploitation of space by human societies is not dissociable from the representations (cognitive and material) of these territorial processes, driving in return its further evolutions. In other words, as soon as social constructions are assumed in the constitution of human settlements, concrete and abstract social structures will play a role in the evolution of the territorial system, through e.g. propagation of information and representations, political processes, conjonction or disjonction between lived and perceived territory.
}{
Le concept\footnote{Nous utiliserons le terme \emph{concept} pour des connaissances construites, plutôt que celui de \emph{notion}, qui suivant~\cite{raffestin1978construits} est plus proche d'une information empirique. Cette distinction peut être mise en perspective avec les domaines de connaissance théorique et empirique de~\cite{livet2010}, que nous approfondissons en~\ref{sec:knowledgeframework}.} de \emph{Territoire}, que nous avons introduit précédemment par ceux de Ville et de Système de Ville, sera central à nos raisonnements et nécessite d'être approfondi et enrichi. En Ecologie Spatiale, un groupe d'agents ou plus généralement un écosystème occupe une certaine étendue spatiale~\cite{tilman1997spatial}, qu'on peut identifier comme notion de territoire. Les \emph{Territoires Humains} impliquent des dimensions supplémentaires, par exemple par l'importance de leur représentations sémiotiques\footnote{c'est à dire des signes marquants les territoires et leur sens, mais aussi leur représentations, cartographiques par exemple}. Celles-ci jouent un rôle significatif dans l'émergence des constructions sociétales, dont la genèse est profondément liée à celle des systèmes urbains. Selon~\cite{raffestin1988reperes}, la \emph{Territorialité Humaine} est ``la conjonction d'un processus territorial avec un processus informationnel'', ce qui implique que l'occupation physique et l'exploitation de l'espace par les sociétés humaines sont complémentaires des représentations (cognitives et matérielles) de ces processus territoriaux, qui influent en retour sur leur évolution. En d'autres termes, à partir de l'instant où les constructions sociales déterminent la constitution des établissements humains, les structures sociales abstraites et concrètes joueront un rôle dans l'évolution des territoires, et ces deux objets seront intimement liés. Des exemples de tels liens se retrouvent à travers la propagation d'informations et de représentations, par des processus politiques, ou encore par la correspondance plus ou moins effective entre territoire vécu et territoire perçu. Une illustration concrète est donnée par une vision simplifiée de la construction de la Métropole du Grand Paris, qui témoigne des ajustements successifs des territoires administratifs (émergence d'un nouveau niveau de gouvernance), des territoires fonctionnels (partiellement par l'évolution des possibilités d'accessibilité), des territoires perçus (dépassement de l'opposition Paris-banlieue), des territoires vécus (nouvelles pratiques de mobilité ou de mobilité résidentielle potentiellement induites par les dynamiques territoriales et du nouveau réseau de transport), l'ensemble de ces processus étant liés de manière complexe et ceux-ci étant loin d'être systématiques.
}



% complementarité du point de vue de la theorie evolutive : boucler la boucle ?

\bpar{
Although this approach does not explicitly give the condition for the emergence of a seminal system of aggregated settlements (i.e. the emergence of cities), it insists on the role of these that become places of power and of creation of wealth through exchange. But the city has no existence without its hinterland and the territorial system can not be summarized by its cities as a system of cities. There is however compatibility on this subsystem between \noun{Raffestin} approach to territories and \noun{Pumain}'s evolutive theory of urban systems~\cite{pumain2010theorie}, in which cities are viewed as an auto-organized complex dynamical systems, and act as mediators of social changes : for example, cycles of innovation occur within cities and propagate between them. Cities are thus competitive agents that co-evolve (in the sense given before). The territorial system can be understood as a spatially organized social structure, including its concrete and abstract artifacts. A imaginary free-of-man spatial extent with potential ressources will not be a territory if not inhabited, imagined, lived, and exploited, even if the same ressources would be part of the corresponding habited territorial system. Indeed, what is considered as a ressource (natural or artificial) will depend on the corresponding society (e.g. of its practices and technological potentialities).
}{
Bien que cette approche ne donne pas de conditions explicites pour l'émergence d'un système séminal d'établissements agrégés (c'est à dire l'émergence des villes, qui est une question centrale pour des théories crédibles des systèmes urbains), elle insiste sur leur rôle comme lieu de pouvoir et de création de richesse au travers des échanges.\comment[FL]{tous ces termes doivent etre definis} Mais la ville n'a pas d'existence sans son hinterland et le système territorial peut difficilement être résumé par ses villes, comme un système de villes.\comment[FL]{question sous-jacente : quel est le territoire d'une ville ? merite un developpement specifique} En se restreignant à ce sous-système, il y a toutefois compatibilité entre la théorie des territoires de \noun{Raffestin} et la théorie évolutive des villes de \noun{Pumain}~\cite{pumain2010theorie},\comment[FL]{dans cette section on ne sait pas pourquoi tu parles de cela} qui interprète les villes comme des systèmes complexes dynamiques auto-organisés, qui agissent comme des médiateurs du changement social : par exemple, les cycles d'innovation s'initialisent au sein des villes et se propagent entre elles\comment[FL]{je ne vois pas ce que cela apporte : va moins vite} (voir~\ref{app:sec:patentmining} pour une entrée sur la notion d'innovation). Les villes sont ainsi des agents compétitifs qui co-évoluent (au sens donné précédemment).\comment[FL]{trop affirmatif c'est une grille de lecture possible} Le système territorial peut ainsi être compris comme une structure sociale organisée dans l'espace, qui comprend ses artefacts concrets et abstraits. Une étendue spatiale imaginaire avec des ressources potentielles qui n'aurait jamais connu de contact avec l'humain ne pourra pas être un territoire si elle n'est pas habitée, imaginée, vécue, exploitée, même si ces ressources pourraient être potentiellement exploitée le cas échéant.\comment[FL]{tu as deja defini territoire $\rightarrow$ ne pas y revenir} En effet, ce qui est considéré comme une ressource (naturelle ou artificielle) dépendra de la société (par exemple de ses pratiques et de ses capacité technologiques).
}



% approche historique et point de vue complementaires

\bpar{

}{
Ces visions complémentaires du territoire peuvent enrichies par une perspective historique. \cite{di1998espace} procède à une analyse historique des différentes conceptions de l'espace (qui aboutissent entre autre à l'espace vécu, l'espace social et l'espace classique de la géographie) et montre comment leur combinaison forme ce que \noun{Raffestin} décrit comme territoires. \cite{giraut2008conceptualiser} rappelle les différents usages récents qui ont été faits de la notion de territoire, de la géographie culturelle où il a plus été utilisé par effet de mode, à la géopolitique où c'est un terme bien spécifique lié aux structures de gouvernance, en passant par des utilisations où il sert plus de concept, et dégage l'avantage d'un objet interdisciplinaire capturant une certaine complexité des systèmes étudiés, ce qui confirme la pertinence de la notion dans notre cas.
}




% transition

\bpar{
A crucial aspect of human settlements that were studied in geography for a long time, and that relate with the previous notion of territory, are \emph{networks}. Let see how we can switch from one to the other and how their definition may be indissociable.
}{
Un aspect central des établissements humains qui a une longue tradition d'étude en géographie, et qui est directement relié au concept de territoire, est celui des \emph{réseaux}. Nous allons préciser leur définition et voir comment le passage de l'un à l'autre est inévitable.
}



\paragraph{Networks}{Réseaux}


% definition des réseaux de manière generale


Un \emph{réseau} doit être compris au sens très large de motifs de connectivité entre entités d'un système, qui peuvent être vus comment relations, liens, interactions. \cite{haggett1970network} postule que l'existence d'un réseau est nécessairement liée à celle de flux, et rappelle la représentation topologique sous forme de graphe de tout système géographique dans lequel circulent des flux entre des entités ou des lieux qui sont abstraits sous la forme de noeuds, reliés par des liens. L'analyse topologique révèle déjà un certain nombre de propriétés du système, mais \cite{haggett1970network} précise l'importance de la spatialisation du réseau, incluse dans les propriétés de ses noeuds (localisation) et de ses liens (localisation, impédance), pour la compréhension des dynamiques dans le réseau (flux) ou du réseau lui-même (croissance du réseau). Cette spécificité a été rappelée par~\cite{barthelemy2011spatial} qui met en perspective les domaines empiriques concernés par les réseaux spatiaux, certains modèles de croissance de réseau, et certains modèles de processus dans les réseaux (par exemple de diffusion).


% Les territoires impliquent des réseaux potentiels, selon Dupuy


\bpar{
We paraphrase \noun{Dupuy} in~\cite{dupuy1987vers} when he proposes elements for ``a territorial theory of networks'' based on the concrete case of Urban Transportation Networks. This theory sees \emph{real networks} (i.e. concrete networks, including transportation networks) as the materialization of \emph{virtual networks}. More precisely, a territory is characterized by strong spatio-temporal discontinuities induced by the non-uniform distribution of agents and ressources. These discontinuities naturally induce a network of ``transactional projects'' that can be understood as potential interactions between elements of the territorial system (agents and/or ressources). For example today, people need to access the ressource of employments, economic exchanges operate between specialized production territories.
}{
Pour approfondir le concept de réseau en appuyant sur sa forte interdépendance avec celui de Territoire, nous reprenons~\cite{dupuy1987vers} qui propose des éléments pour une ``théorie territoriale des réseaux'' s'inspirant du cas concret d'un réseau de transport urbain. Cette théorie distingue les \emph{réseaux réels} (auxquels appartiennent les réseaux concrets, incluant les réseaux matériels et donc les réseaux de transport, les réseaux sociaux étant également des réseaux réels sur lesquels nous ne nous attarderons pas) et les \emph{réseaux virtuels}, eux-même induits entre autre par la configuration territoriale. Les réseaux réels sont la matérialisation de réseaux virtuels. Plus précisément, un territoire est caractérisé par de fortes discontinuités spatio-temporelles induites par la distribution non-uniforme des agents et des ressources. Ces discontinuités induisent naturellement un réseau d'interactions\footnote{} potentielles entre les éléments du système territorial, notamment des agents et des ressources. \cite{dupuy1987vers} désigne ces interactions potentielles comme \emph{projets transactionnels}. Cela justifie de manière thématique l'utilisation de modèles de potentiels\comment[FL]{de quoi parles tu ? c'est enigmatique} pour capturer les interactions entre agents\comment[FL]{c'est la premiere fois que tu parles d'interaction, c'est dommage que cela arrive comme cela}, que ce soit à l'échelle des villes ou au sein de celles-ci. Par exemple, de nos jours les actifs se doivent d'accéder à la ressource qu'est l'emploi, et des échanges économiques s'effectuent entre les différents territoires spécialisés dans les productions de différents types. Une distribution spatiale d'agents suffit à introduire des interactions potentielles.
}



% Les réseaux potentiels se transforment en réseaux réels sous certaines conditions.
%. -> effet des territoires sur les réseaux

\bpar{
The potential interaction network is concretized as offer adapts to demand, and results of the combination of economic and geographical constraints with demand patterns, in a non-linear way through agents designed as \emph{operators}. This process is not immediate, leading to strong non-stationarity and path-dependance effects : the extension of an existing network will depend on previous configuration, and depending on involved time scales, the logic and even the nature of operators may have evolved. \noun{Raffestin} points out in his preface of~\cite{offner1996reseaux} that a geographical theory articulating space, network and territories had never been consistently formulated. It appears to still be the case today, but the theory developed just before is a good candidate, even if it stays at a conceptual level. The presence of a human territory necessarily imply the presence of abstract interaction networks and concrete networks used for transportation of people and ressources (including communication networks as information is a crucial ressource). Depending on regime in which the considered system is, the respective role of different networks may be radically different. Following \noun{Duranton} in \cite{duranton1999distance}, pre-industrial cities were limited in growth because of limitations of transportation networks. Technological progresses have lead to the end of these limitations and the preponderance of land markets in shaping cities (and thus a role of transportation network as shaping prices through accessibility), and recently to the rising importance of telecommunication networks that induce a ``tyranny of proximity'' as physical presence is not replaceable by virtual communication.
}{
 Le réseau d'interaction potentiel est concrétisé quand l'offre s'adapte à la demande\comment[AB]{me semble une generalite trop grande}, et résulte en la combinaison de contraintes économiques et géographiques avec les motifs de demande, de manière non-linéaire via des agents qu'on peut désigner comme \emph{opérateurs}\comment[FL]{c'est theorique a ce stade $\rightarrow$ pourquoi ce vocabulaire ?}. Un tel processus est loin d'être immédiat, et conduit à de forts effets de non-stationnarité et de dépendance au chemin\comment[AB]{definir les 2 concepts} : l'extension d'un réseau existant \comment[FL]{la tu es deja dans le raisonnement de ta these mais le lecteur ne le sait pas} dépendra de la configuration précédente, et selon les échelles de temps impliquées, la logique et même la nature des opérateurs peut avoir évolué. Les exemples de trajectoires concrètes peuvent être très variées : \cite{kasraian2015development} montre par exemple dans le cas de la Randstad sur le temps long, un premier régime d'adaptation\comment[FL]{sens ?} du réseau ferré au développement urbain suivi par des effets inverses plus récemment. A une échelle urbaine sur le temps long, la dépendance au chemin \comment[FL]{sens ?} est montrée pour Boston par~\cite{block2012hysteresis} puisque l'environnement bâti et la distribution de la population sont montrés fortement dépendants des lignes de tramway passé même lorsqu'elles n'existent plus. \noun{Raffestin} souligne dans sa préface de~\cite{offner1996reseaux} qu'une théorie géographique articulant espaces, réseaux et territoires n'a jamais été formulée de manière cohérente\comment[FL]{plutot dans la conclusion generale}, chaque approche ayant une vision réduite à certaines composantes seulement et ne visant pas à construire une théorie globalement cohérente. Il semble que c'est toujours le cas aujourd'hui, même si la théorie évoquée ci-dessus semble être un bon candidat bien qu'elle reste à un niveau conceptuel. La présence d'un territoire humain implique nécessairement \comment[FL]{non pas d'accord mais surtout : pourquoi est-ce important ?} la présence de réseaux d'interactions abstraites et de réseaux concrets utilisés pour transporter les individus et les ressources (incluant les réseaux de communication puisque l'information est une ressource essentielle). 
}





% le contexte socio-eco/techno conditionne fortement la facon dont les réseaux agissent sur les territoires.

\bpar{
}{
Le statut du réseau par rapport au territoire est fortement conditionné par le contexte socio-économique et technologique. Selon \noun{Duranton}~\cite{duranton1999distance}, un facteur influençant la forme des villes pré-industrielles était la performance des réseaux de transport. Les progrès technologiques ont induit un changement de régime, ce qui a mené à une prépondérance du marché foncier dans la formation des villes (et par conséquent un rôle des réseaux de transport qui déterminent les prix par l'accessibilité), et plus récemment à une importance croissante des réseaux de télécommunication ce qui a induit une ``tyrannie de la proximité'' puisque la présence physique n'est pas remplaçable par une communication virtuelle. On s'attendra ainsi à l'existence de multiples processus d'interaction, potentiellement superposables de manière complexe.
}



% Transition

\bpar{
This territorial approach to networks seems natural in geography, since networks are studied conjointly with geographical objects with an underlying theory, in opposition to network science that studies brutally spatial networks with few thematic background~\cite{ducruet2014spatial}.
}{
Cette approche territoriale des réseaux semble naturelle en géographie, puisque les réseaux sont étudiés conjointement avec des objets géographiques qu'ils connectent, en opposition à la science des réseaux qui étudie les réseaux spatiaux de manière relativement déconnectée de leur fond thématique~\cite{ducruet2014spatial}. \comment[AB]{pas d'accord $\rightarrow$ vrai pour sa composante \emph{statistique} mais pas dans ses applications en sociologie, archi, histoire ou geo par exemple $\implies$ pas une question de ``science des reseaux'' donc}
}





\paragraph{Networks shaping territories ?}{Des réseaux qui façonnent les territoires ?}

% Effets des réseaux sur les territoires ? -> approfondir le debat des effets structurants


\bpar{
However networks are not only a material manifestation of territorial processes, but play their part in these processes as they evolution may shape territories in return. In the case of \emph{technical networks}, an other designation of real networks given in~\cite{offner1996reseaux}, many examples of such feedbacks can be found : the interconnectivity of transportation networks allows multi-scalar mobility patterns, thus shaping the lived territory. At a smaller scale, changes in accessibility may result in an adaptation of a functional urban space. Here emerges again an intrinsic difficulty : it is far from evident to attribute territorial mutations to some network evolutions and reciprocally materialization of a network to precise territorial dynamics. Coming back to Diderot should help, in the sense that one must not consider network nor territories as independent systems that would have causal relationships but as strongly coupled components of a larger system. The confusion on possible simple causal relationships has fed a scientific debate that is still active. Methodologies to identify so-called \emph{structural effects} of transportation networks were proposed by planners in the seventies~\cite{bonnafous1974detection,bonnafous1974methodologies}. It took some time for a critical positioning on unreasoned and decontextualized use of these methods by planners and politics generally to technocratically justify transportation projects, that was first done by \noun{Offner} in~\cite{offner1993effets}. Recently the special issue~\cite{espacegeo2014effets} on that debate recalled that on the one hand misconceptions and misuses were still greatly present in operational and planning milieus as~\cite{crozet:halshs-01094554} confirmed, and on the other hand that a lot of scientific progresses still need to be made to understand relations between networks and territories as \noun{Pumain} highlights that recent works gave evidence of systematic effects on very long time scales (as e.g. the work of \noun{Bretagnolle} on railway evolution, that shows a kind of structural effect in the necessity of connectivity to the network for cities to ``stay in the game'', but that is not fully causal as not sufficient).
}{
Cependant les réseaux ne sont pas seulement une manifestation matérielle de processus territoriaux, mais jouent également leur rôle dans ces processus comme leur évolution peut influencer l'évolution des territoires en retour.\comment[FL]{B - en parler avant pour fixer les idees} Dans le cas des \emph{réseaux techniques}, une autre désignation des réseaux réels\comment[AB]{bof} donnée dans~\cite{offner1996reseaux}, de nombreux exemples de tels retroactions peuvent être mis en évidence : une accessibilité accrue peut être un facteur favorisant la croissance urbaine, ou bien l'interconnexion des réseaux de transport permet des motifs de mobilité multi-échelles formant ainsi le territoire vécu. A une plus petite échelle, des changements de l'accessibilité peuvent induire l'adaptation d'un espace fonctionnel urbain. Il emerge alors une difficulté intrinsèque : il n'est pas évident d'attribuer des mutations territoriales à une évolution du réseau et réciproquement la matérialisation d'un réseau à des dynamiques territoriales précises.\comment[FL]{c'est le coeur de ta these : mieux mettre en valeur} Revenir à la citation de Diderot\comment[FL]{laquelle ?} devrait aider à ce point, au sens où il ne faut pas considérer le réseau ni les territoires comme des systèmes indépendants qui s'influenceraient mutuellement par des relations causales\comment[FL]{sens ?}, mais comme des composantes fortement couplées\comment[FL]{sens ?} d'un système plus large. La confusion autour de possibles relations causales simples a nourri un débat scientifique encore actif aujourd'hui. Les méthodologies pour identifier ce qui est nommé \emph{effets structurants} des réseaux de transport ont été proposées par les planificateurs dans les années 1970~\cite{bonnafous1974detection,bonnafous1974methodologies}. Il aura fallu un certain temps pour un positionnement critique sur l'usage non raisonné et hors contexte de ces méthodes par les planificateurs et les politiques qui les mobilisaient généralement pour justifier des projets de transports de manière technocratique.\comment[FL]{c'est elliptique. je crois voir ce que tu veux dire mais il faut ancrer davantage ton propos} Cela a été fait en premier par \noun{Offner} dans~\cite{offner1993effets}. Une édition spéciale relativement récente de l'Espace Géographique sur ce débat~\cite{espacegeo2014effets} a rappelé d'une part que les mauvaises interprétations et les mauvais usages\comment[FL]{la encore : il faut du concret} étaient encore largement présents aujourd'hui dans les milieux opérationnels de la planification comme confirmé par~\cite{crozet:halshs-01094554}, et d'autre part qu'il faudrait encore une certaine quantité de progrès\comment[FL]{formulation curieuse} scientifiques pour comprendre en profondeur les relations entre réseaux et territoires. Les débats récents en juillet 2017 relatifs à l'ouverture des LGV Bretagne et Sud-Ouest ont montré toute l'ambiguïté des positions, des conceptions, des imaginaires à la fois des politiques mais aussi du public : spéculation dans les quartiers de gare, questionnements des pratiques de mobilité quotidienne mais aussi sociale\footnote{voir par exemple \url{http://www.liberation.fr/futurs/2017/07/02/immobilier-plus-de-parisiens-comment-les-bordelais-voient-l-arrivee-de-la-lgv_1580776}, ou \url{http://www.lemonde.fr/big-browser/article/2017/10/24/a-bordeaux-une-fronde-anti-parisiens-depuis-l-ouverture-de-la-ligne-a-grande-vitesse_5205282_4832693.html} pour une réaction ``à chaud'' de divers acteurs locaux, témoignant d'un impact au minimum sur les représentations.}. La complexité et la portée des sujets montrent bien la difficulté d'une compréhension systématique d'effets du transport sur les territoires. \noun{Pumain} souligne que des travaux récents ont révélé des effets systématiques\comment[FL]{def} sur de très longues échelles temporelles, comme par exemple le travail de \noun{Bretagnolle} sur l'évolution des chemins de fer, qui montre une sorte d'effet structurel\comment[FL]{est ce different des effets structurants} sur la nécessité de connexion au réseau des villes, afin de rester actives, mais qui n'est ni suffisant ni totalement causal. Certaines trajectoires de villes correspondent à un renforcement de la hiérarchie par une accessibilité accrue, tandis que des villes non connectées seront a priori hors-jeu pour une période considérable\comment[FL]{c'est caricatural : etre pedagogique ce n'est pas etre emphatique}, avec de grandes fluctuations pouvant conduire à une relative indépendance du taux de croissance\comment[FL]{de quoi ?} et de l'augmentation d'accessibilité pour certains cas.}





\bpar{
At a macroscopic level typical patterns of interaction emerge, but microscopic trajectories of the system are essentially chaotic : the understanding of coupled dynamics strongly depends on the scale considered. At a small scale it seems indeed impossible to show systematic behavior, as \noun{Offner} pointed out. For example, on comparable French mountain territories, \cite{berne2008ouverture} shows that reactions to a same context of evolution of the transportation network can lead to very different reactions of territories, some finding a huge benefit in the new connectivity, whereas others become more closed. These potential retroactions of networks on territories does not necessarily act on concrete components : \noun{Claval} shows in~\cite{claval1987reseaux} that transportation and communication networks contribute to the collective representation of territories by acting on territorial belonging feeling.

}{
 A un niveau macroscopique\comment[AB]{pas clair // petite echelle} des motifs typiques d'interaction émergent, mais les trajectoires microscopiques du systèmes sont essentiellement chaotiques\comment[AB]{meilleur terme ?} : la compréhension des dynamiques couplées dépend fortement de l'échelle considérée. A une petite échelle il est difficile de s'attendre à identifier facilement des comportements systématiques\comment[FL]{sens ?}, comme le rappelle \noun{Offner}. Par exemple, sur des territoires de montagne français comparables, \cite{berne2008ouverture} montre que les réactions à un même contexte d'évolution du réseau de transport peuvent mener à des dynamiques territoriales très diverses, certains trouvant de forts bénéfices par la nouvelle connectivité\comment[FL]{sens}, d'autres au contraire devenant plus fermés. Ces potentiels processus antagonistes sont examinés plus en détails par~\cite{bernier2007dynamiques} toujours dans le cadre des territoires de montagne\comment[FL]{tu en parles deux fois : organiser differement}, pour lesquels il propose un typologie basée sur le potentiel d'ouverture à la fois des dynamiques territoriales et des dynamiques des réseaux : par exemple, un territoire peut avoir un fort potentiel touristique, et donc d'ouverture, mais avoir une faible accessibilité. Au contraire, des contraintes douanières\comment[FL]{tu prends des exemples trop variés, au risque d'etre imprecis sur la plupart} par exemple peuvent freiner le potentiel d'ouverture d'une infrastructure performante. Ces retroactions potentielles des réseaux sur les territoires n'agissent pas nécessairement sur des composantes concretes : \noun{Claval} montre dans~\cite{claval1987reseaux} que les réseaux de transport et de communication contribuent à la représentation collective d'un territoire en agissant sur un sentiment d'appartenance, qui peut alors jouer un rôle crucial dans l'émergence d'une dynamique régionale fortement cohérente. A l'échelle mesoscopique, on peut montrer des effets forts de la présence d'infrastructures pour des types particuliers d'usage du sol : \cite{nilsson2016measuring} l'illustre par exemple pour les fast food dans deux villes aux Etats-Unis.\comment[FL]{en montrant quoi ?}
}

\comment[AB]{je pense que la lecture Peter Hagget et de Peter Gould te fera pas de mal $\rightarrow$ j'essaye de les retrouver :)}



\paragraph{Territorial Systems}{Systèmes Territoriaux}


\bpar{
This detour from territories, to networks and back again, allows us to give a preliminary definition of a territorial system that will be the basis of our following theoretical considerations. As we emphasized the role of networks, the definition takes it into account.
\textbf{Preliminary Definition.} \textit{A territorial system is a human territory to which both interaction and real networks can be associated. Real 
 networks are a component of the system, involved in evolution processes, through multiples feedbacks with other components at various spatial and temporal scales.}
 This reading of territorial systems is conditional to the existence of networks and may discard some human territories, but it is a deliberate choice that we justify by previous considerations, and that drives our subject towards the study of interactions between networks and territories.
}{
Cet aperçu introductif, des territoires aux réseaux, nous permet de clarifier notre approche des systèmes territoriaux qui sera sous-jacente dans l'ensemble de la suite. Comme nous avons mis en exergue le rôle des réseaux dans de nombreux aspects des dynamiques territoriales, nous proposons une définition des systèmes territoriaux les incluant explicitement. Nous considérons un Système Territorial comme un territoire humain auquel peuvent être associés\comment[FL]{ok mais il faut etre plus constructif} à la fois des réseaux d'interactions et des réseaux réels\comment[AB]{mieux definir. reels $\rightarrow$ concret}. Les réseaux réels, et plus particulièrement les réseaux concrets, sont une composante à part entière du système, jouant dans les processus d'évolution, au travers de multiples retroactions avec les autres composantes à plusieurs échelles spatiales et temporelles. Cette lecture des systèmes territoriaux est conditionnée à l'existence des réseaux et pourrait écarter certains territoires humains, mais il s'agit d'un choix délibéré justifié par les considérations précédentes, et qui précise notre sujet vers l'étude des interactions entre réseaux et territoires. Le réseau n'est pas une composante en tant que telle du territoire, mais bien du système territorial en notre sens.\comment[FL]{pourquoi opposer ? on peut dire que nw $\in$ territoires et aussi $\in$ systeme territorial. plus largement, je ne vois pas encore en quoi cette distinction va t'etre utile.} Il faut aussi garder à l'esprit que le transport en lui-même est différent des réseaux de transport\comment[FL]{en quoi estce un argument ?}, puisqu'il correspond à l'utilisation de ceux-ci par les agents territoriaux. Dans une grande partie des approches que nous décrirons par la suite, et typiquement les approches appliquée en planification urbaine, la modélisation du transport s'axe sur des question de demande, d'offre, de congestion, c'est à dire à des échelles relatives à la mobilité, et est liée au réseau mais ne se concentre pas directement sur celui-ci\comment[FL]{ce n'est pas clair $\rightarrow$ la croissane du reseau ?, l'usage du reseau ? la croissance de l'usage du reseau ?} comme notre positionnement propose\comment[AB]{$\simeq$}.
}


\subsection{Transportation Networks}{Réseaux de Transport}


\paragraph{The particularity of transportation networks}{La particularité des réseaux de transport}



\bpar{
Already evoked in relation to the question of structural effects of networks, transportation networks play a determining role in the evolution of territories. Although other types of networks are also strongly involved in the evolution of territorial systems (see e.g. the discussions of impacts of communication networks on economic activities), transportation networks shape many other networks (logistics, commercial exchanges, social concrete interactions to give a few) and are prominent in territorial evolution patterns, especially in our recent societies that has become dependent of transportation networks~\cite{bavoux2005geographie}. The development of French High Speed Rail network is a good illustration of the impact of transportation networks on territorial development policies. Presented as a new era of railway transportation, a top-down planning of totally novel lines was introduced as central for developments~\cite{zembri1997fondements}. The lack of integration of these new networks with existing ones and with local territories is now observed as a structural weakness and negative impacts on some territories have been shown~\cite{zembri2008contribution}. A review done in~\cite{bazin2011grande} confirms that no general conclusions on local effects of High Speed lines connection can be drawn although it keeps a strong place in imaginaries. These are examples of how transportation networks have both direct and indirect impacts on territorial dynamics. Integrated planning, in the sense of a joint planning of transportation infrastructures and urban development, considers the network as a determining component of the territorial system. Parisian \emph{Villes Nouvelles} are such a case, that witnesses of the complexity of such planning actions that generally do not lead to the desired effect~\cite{es119}. Recent projects as~\cite{l2012ville} have try to implement similar ideas but we have now not enough temporal scope to judge their success in effectively producing an integrated territory. Transportation networks are anyway at the center of these approaches of urban territories. We will focus in our work on transportation networks for the various reasons given here.
}{
Déjà évoqués dans le cas des effets structurants des réseaux, les réseaux de transports jouent un rôle central dans l'évolution des territoires, mais il n'est évidemment pas question de leur attribuer des effets causaux déterministes. Même si d'autres types de réseaux sont également fortement impliqués dans l'évolution des systèmes territoriaux (voir e.g. les débats sur l'impact des réseaux de communication sur la localisation des activités économiques), les réseaux de transport conditionnent d'autres types de réseaux (logistique, échanges commerciaux, interactions sociales concrètes pour donner quelques exemples) et semblent dominer\comment[FL]{question tres controversée} dans les motifs d'évolution territoriale, en particulier dans nos sociétés contemporaines pour lesquelles les réseaux de transport jouent un rôle privilégié~\cite{bavoux2005geographie}. Le développement du réseau français à grande vitesse est une illustration de l'impact des réseaux de transport sur les politiques de développement territorial. Présenté comme une nouvelle ère de transport sur rail, une planification par le haut\comment[FL]{sens?} de lignes totalement nouvelles et indépendantes\comment[FL]{non elles ne le sont pas vraiment} de par leur vitesse deux fois plus élevée, a été défendu par les acteurs politiques entre autres comme central pour le développement~\cite{zembri1997fondements}. Le manque d'intégration\comment[FL]{sens} de ces nouveaux réseaux avec l'existant et avec les territoires locaux est à présent observé comme une faiblesse structurelle\comment[FL]{sens} et des impacts négatifs sur certains territoires\comment[FL]{de quel type ? etre plus explicite} ont été prouvés~\cite{zembri2008contribution}. Une revue faite dans~\cite{bazin2011grande} confirme qu'aucune conclusion générale sur des effets locaux d'une connection à une ligne à grande vitesse ne peut être tirée, bien que ce sésame garde une place conséquente dans les imaginaires des élus. Ces exemples illustrent la manière dont les réseaux de transport peuvent avoir des effets à la fois directs et indirects sur les dynamiques territoriales. Le développement des différentes Lignes à Grande Vitesse s'inscrit dans des contextes territoriaux très différents, et il est dans tous les cas délicat de penser pouvoir interpréter des processus hors de ceux-ci\comment[FL]{mal tourne mais oui c'est une remarque importante} : par exemple, les lignes LGV Nord et LGV Est s'inscrivent dans des échelles européennes plus vastes que la LGV Bretagne ouverte en juillet 2017. Les effets de l'ouverture d'une ligne peuvent s'étendre au delà des seuls territoires directement concernés : \cite{l2014contribution} montre par l'utilisation d'indicateurs issus de la \emph{Time Geography} (mesurant une quantité de temps de travail disponible dans le cadre d'un aller-retour journalier) que la ligne Tours-Bordeaux a des répercussions potentielles dans le Nord et l'Est de la France. La planification intégrée, au sens d'une planification coordonnée entre les infrastructures de transport et le développement urbain, considère le réseau comme une composante déterminante du système territorial. Les Villes Nouvelles parisiennes sont un tel cas qui témoigne de la complexité de ces actions de planification qui le plus souvent ne mènent pas au effets initialement désirés~\cite{es119}. Des projets récents comme~\cite{l2012ville} ont tenté d'implémenter des idées similaires, mais le recul manque encore pour juger de leur succès à produire un territoire effectivement intégré. On sait que sur des échelles de temps relativement courtes allant de l'année à la dizaine d'année, les effets observés sur les mobilités quotidiennes et mobilité résidentielles peuvent être significatifs.\comment[FL]{c'est ce genre de choses qu'il faut creuser} Les réseaux de transports sont dans tous les cas au centre de ces approches des territoires urbains. Nous nous concentrerons par la suite sur les réseaux de transport de manière générale pour toutes ces raisons évoquées ici.
}



\paragraph{Transportation and Accessibility}{Transports et Accessibilité}


% lien entre transports et accessibilite ; potentielle implication de l'accessibilite dans les transformations territoriales.

\bpar{
Reformulate positioning. \cite{miller1999measuring} on three different way to approach accessibility : time-geography and constraints, user utility based measures, and transportation time. It derives measures for each in perspective of \noun{Weibull}'s axiomatic frameworks and reconcile the three in a way.
The notion of accessibility comes rapidly when considering transportation networks. Based on the possibility to access a place through a transportation network (including transportation speed, difficulty of travel), it is generally described as a potential of spatial interaction\footnote{and often generalized as \emph{functional accessibility}, for example employments accessible for actives at a location. Spatial interaction potentials ruling gravity law can also been understood this way.}~\cite{bavoux2005geographie}. This object is often used as a planning tool or as an explicative variable of agents localisation for example. One has to be however careful on its unconditional use. More precisely, it may be a construction that misses a consistent part of territorial dynamics. The mystification of the notion of \emph{mobility} was shown by \noun{Commenges} in~\cite{commenges:tel-00923682}, which proved than most of debates on modeling mobility and corresponding notions were mostly made-of by transportation administrators of \emph{Corps des Ponts} who roughly imported ideas from the United States without adaptation and reflexion fit to the totally different French context. Accessibility may be such a social construct and have no theoretical root since it is mostly a modeling and planning tool. Recent debates on the planification of \emph{Grand Paris Express}~\cite{confMangin}, a totally novel metropolitan transportation infrastructure planned to be built in the next twenty years, have revealed the opposition between a vision of accessibility as a right for disadvantaged territories against accessibility as a driver of economic development for already dynamic areas, both being difficultly compatible since corresponding to very different transportation corridors. Such operational issues confirm the complexity of the role of transportation networks in the dynamics of territorial systems, and we shall give in our work elements of response to a definition of accessibility that would integrate intrinsic territorial dynamics.
}{
La notion d'accessibilité émerge naturellement\comment[AB]{rapidement ?} lorsqu'on s'intéresse aux réseaux de transport.\comment[FL]{phrase toute faite} Basée sur la possibilité d'accéder un lieu par un réseau de transport (pouvant prendre en compte la vitesse, la difficulté de se déplacer), elle est généralement définie comme un potentiel d'interaction spatiale\footnote{et souvent généralisée comme une \emph{accessibilité fonctionnelle}, par exemple les emplois accessibles aux actifs d'un lieu. Les potentiels d'interaction spatiaux\comment[AB]{spatial/spatiaux} s'exprimant dans les lois de gravité\comment[AB]{gravitaires ?} peuvent aussi être compris de cette façon.}~\cite{bavoux2005geographie}. Une approche axiomatique\comment[FL]{sens} a été proposée par~\cite{miller1999measuring}, en mettant en valeur trois façons de comprendre l'accessibilité, basées sur la \emph{Time Geography} et les contraintes, les mesures d'utilité basée sur l'utilisateur, et le temps de transport\comment[AB]{(...)}, les mesures correspondantes étant dérivées dans un cadre mathématique unifié\comment[FL]{et alors ?}. Ce concept est souvent utilisé comme un outil de planification ou comme une variable explicative de localisation des agents par exemple, puisqu'il s'agit par exemple d'un bon indicateur pour la quantité de personnes impactées par un projet de transport. Il faut cependant rester prudent sur son usage inconditionnel. Plus précisément, il peut s'agir d'une construction qui ignore une partie conséquente des dynamiques territoriales. La co-construction de la notion de \emph{mobilité}\comment[FL]{def} et des solutions techniques impliquant les solutions de modélisation mais aussi la production de l'infrastructure, a été montrée par \noun{Commenges} dans~\cite{commenges:tel-00923682} pour le contexte français\comment[FL]{trop rapide}. Il révèle qu'une partie des débats sur la modélisation de la mobilité et les notions correspondantes étaient majoritairement construites de manière ad-hoc par les administrateurs de transports issus du \emph{Corps des Ponts} qui importaient les outils et méthodes des Etats-Unis sans mener de reflexion approfondie pour leur adaptation au contexte français\comment[FL]{exagere}. L'accessibilité pourrait\comment[AB]{conditionnel ambigu : preciser} de même être une construction sociale\comment[FL]{1) gratuit 2) ce n'est pas la meme chose} et n'avoir que peu de fondement théorique, puisqu'il s'agit en grande partie d'un outil de modélisation et de planning.\comment[FL]{a reprendre ; non} Les débats récents sur la planification du \emph{Grand Paris Express}~\cite{confMangin}, cette nouvelle infrastructure de transport métropolitaine planifiée pour les vingts prochaines années, a révélé l'opposition entre une vision de l'accessibilité comme un droit\comment[FL]{?} pour les territoires désavantagés, contre l'accessibilité comme un moteur du développement économique pour des zones déjà dynamiques, les deux étant difficilement compatibles\comment[FL]{pourquoi ? a preciser} car correspondent à des corridors de transport très différents : l'un initialement porté par l'Etat dans la perspective des pôles de compétitivité, l'autre par la région dans une perspective d'équité territoriale. De tels problèmes opérationnels confirment la complexité du rôle des réseaux de transports dans les\comment[FL]{formulation causale implicitement} dynamiques des systèmes territoriaux, et nous devrons donner dans notre travail des éléments de réponse pour une définition de l'accessibilité qui intégrerait les dynamiques territoriales intrinsèques.
}

\paragraph{Scales and Hierarchies}{Echelles et Hierarchies}\comment[FL]{titre $\simeq$}

\bpar{
An incontournable aspect of transportation networks that we will need to take into account in further developments is hierarchy. Transportation networks are by essence hierarchical, depending on scales they are embedded in. \cite{10.1371/journal.pone.0102007} showed empirical scaling properties for public transportation networks for a consequent number of metropolitan areas across the world, and scaling laws reveal the presence of hierarchy within a system, as for size hierarchy for system of cities expressed by Zipf's law~\cite{nitsch2005zipf} or other urban scaling laws~\cite{2013arXiv1301.1674A,2015arXiv151000902B}. Transportation network topology has been shown to exhibit such scaling also for the distribution of its local measures such as centrality~\cite{samaniego2008cities}. Hierarchy seems to play a particular role on interaction processes, as \noun{Bretagnolle}~\cite{bretagnolle:tel-00459720} highlighted an increasing correlation in time between urban hierarchy and network hierarchy for French railway network, marker of positive feedbacks between urban rank and network centralities. Different regimes in space and times were identified: for French railway network evolution e.g., a first phase of adaptation of the network to the existing urban configuration was followed by a phase of co-evolution i.e. in the sense that causal relations became difficult to identify. The impact of space-time contraction by the network on patterns of growth potential had already been shown for Europe with an exploratory analysis in~\cite{bretagnolle1998space}. Railway evolution in the United States followed a different pattern, without hierarchical diffusion, shaping locally urban growth. It emphasizes the presence of path-dependance for trajectories of urban systems: the presence in France of a previous city system and network (postal roads) strongly shaped railway development, whereas its absence in the US lead to a completely different story. An open question is if generic processes underlie both evolutions, each being different realizations with different initial conditions and different meta-parameters (different \emph{regimes} in the sense of settlement systems transitions introduced in the current ANR Research project TransMonDyn, as a transition can be understood as a change of stationarity for meta-parameters of a general dynamic). In terms of dynamical systems formulation, it is equivalent to ask if dynamics of attractors (long time scale components) obey similar equations as the position and nature of attractors for a stochastic dynamical system that give its current regime, in particular if it is in a divergent state (positive local Liapounov exponent) or is converging towards stable mechanisms~\cite{sanders1992systeme}. To answer this question together with a disentangling of co-evolution processes for that regime, \cite{bretagnolle:tel-00459720} proposes modeling as a constructive element of answer. We will see in next section how modeling can bring knowledge about territorial processes.
}{
Un aspect des réseaux de transport qu'il est important de considérer est la notion de hiérarchie\comment[FL]{pourquoi ?}. Les réseaux de transport sont par essence hiérarchiques, cette propriété dépendant des échelles dans lesquelles ils sont intégrés. \cite{10.1371/journal.pone.0102007}\comment[FL]{il y a des travaux plus anciens, reprendre} montre empiriquement des propriétés de loi d'échelle pour un nombre conséquent d'aires métropolitaines à travers la planète, et les lois d'échelle révèlent la présence de hiérarchies dans un système\comment[FL]{$\simeq$}, comme pour la hiérarchie de tailles dans les systèmes de villes exprimée par la loi de Zipf~\cite{nitsch2005zipf} ou d'autres lois d'échelle urbaines~\cite{2013arXiv1301.1674A,2015arXiv151000902B}. La topologie du réseau de transport suit de telles lois pour la distribution de ses mesures locales comme la centralité~\cite{samaniego2008cities}, celles-ci étant directement liées au motifs d'accessibilité à différentes échelles : cette notion est donc nécessaire pour les comprendre, mais induit aussi des choix d'échelles pour préciser la définition de ceux-ci. De plus, la topologie du réseau fait partie des facteurs induisant la hiérarchie d'usage, se retrouvant dans les externalités négatives de congestion, en relation avec la distribution spatiale de l'usage du sol~\cite{Tsekeris20131}. La hiérarchie joue un rôle particulier dans les processus d'interaction. \cite{bretagnolle:tel-00459720} souligne ainsi une corrélation croissante dans le temps entre la hiérarchie urbaine et la hiérarchie de l'accessibilité temporelle pour le réseau ferroviaire français (a priori plus claire pour cette mesure que pour les mesures intégrées d'accessibilité soumises à l'auto-corrélation comme nous le verrons en~\ref{sec:causalityregimes}).\comment[FL]{c'est une discussion un peu faible ce n'est pas parce que les hierarchient coevoluent que la notion de ``hierarchie'' jour un role} Celle-ci est un marqueur de rétroactions positives entre le rang urbain et la centralité de réseau. Différents régimes dans le temps et l'espace ont été identifiés : pour l'évolution du réseau ferroviaire français, une première phase d'adaptation du réseau à la configuration urbaine existante a été suivie par une phase de co-évolution, au sens où les relations causales sont devenues difficiles à identifier. L'impact de la contraction de l'espace-temps par les réseaux sur le potentiel de croissance des villes avait déjà été montré pour l'Europe par des analyses exploratoires dans~\cite{bretagnolle1998space}. L'evolution du réseau ferroviaire aux Etats-unis a suivi une dynamique bien différente, sans diffusion hiérarchique, donnant forme localement à la croissance urbaine dans certains cas sans effet systématique toujours\comment[AB]{reformuler} : ce contexte particulier de conquête d'un espace vierge d'infrastructures implique un régime particulier au système territorial comme le montrent les paramétrisations différentes du modèle Simpop2 résumées dans~\cite{bretagnolle2010comparer}. Cela met l'emphase sur la présence de dépendances au chemin pour les trajectoires des systèmes urbains, que nous retrouverons régulièrement par la suite : la présence en France d'un système préalable de villes et de réseau (routes postales) a fortement influencé le développement du réseau ferré, tandis que son absence aux Etats-Unis a conduit à une histoire complètement différente\comment[FL]{tu en as deja parle (routes postales)}. La question reste ouverte si \comment[FL]{reformuler} des processus génériques sont implicites aux deux évolutions, chacun correspondant à des réalisations différentes avec des conditions initiales et des méta-paramètres différents\comment[FL]{def ?}, c'est à des \emph{régimes} différents au sens des transitions des systèmes de peuplement\comment[FL]{concept non connu}, puisqu'une transition entre deux régimes peut être comprise comme un changement de stationnarité des méta-paramètres d'une dynamique plus générale. En termes de systèmes dynamiques, cela revient à se demander si les dynamiques des ensembles de catastrophes (composantes à plus grandes échelles temporelles) obéissent à des équations similaires à la position et nature des attracteurs pour un système dynamique stochastique qui donne son régime courant, en particulier si le système est dans un état local divergent (exposant de Liapounov local positif\comment[FL]{ce n'est pas comprehensible}) ou en train de converger vers des mécanismes stables\comment[FL]{c'est flou}~\cite{sanders1992systeme}. Pour répondre à cette question en même temps que l'isolation des processus\comment[FL]{pourquoi se fixer de tels objectifs ?} de co-évolution pour ce régime, \cite{bretagnolle:tel-00459720} propose la modélisation comme élément de réponse constructif. Nous verrons dans le chapitre suivant comme la modélisation peut être source de connaissance sur les processus territoriaux. 
}

\comment[AB]{AERER}


\paragraph{Transportation and Mobility}{Transports et Mobilité}


\bpar{}{
La notion de mobilité et l'ensemble des approches associées, peuvent capturer nos questionnements à échelle fine : les motifs d'utilisation des réseaux de transport sont le produit des dynamiques de mobilité quotidiennes, et ceux-ci s'y adaptent, tout en induisant des relocalisations des actifs et emplois : il existe une co-évolution entre transports et composantes territoriales aux échelles microscopiques et mesoscopiques, qui sont un objet d'étude à part entière. \cite{fusco2004mobilite} révèle par exemple une relation causale de la mobilité sur la structure urbaine, l'offre d'infrastructure et ses propriétés ayant cependant des effets joints\comment[FL]{def ?} à la fois sur la mobilité et sur la structure urbaine. Dans le cas des réseaux autoroutiers, \cite{faivr2003} rappelle la nécessité de construire un cadre d'analyse dépassant la logique des effets structurants sur le temps long, et montre également des interactions à petite échelle propres à la mobilité\comment[FL]{sens ?} sur lesquelles des conclusions plus systématiques peuvent être établies, comme une évolution des pratiques de mobilité impliquant une utilisation différente du réseau de transport.\comment[FL]{ce sont des points importants : ne pas etre aussi elliptique}
}


%%%%%%%%%%%%%%%%%%%
\subsection{Interactions between transportation networks and territory}{Interactions entre Réseaux et Territoires}


\bpar{
At this state of progress, we have naturally identified a research subject that seems to take a significant place in the complexity of territorial systems, that is the study of interactions between transportation networks and territories. In the frame of our preliminary definition of a territorial system, this question can be reformulated as the study of networked territorial systems with an emphasize on the role of transportation networks in system evolution processes.
}{
A ce stade, nous avons identifié que les processus d'interaction\comment[FL]{on en a toujours pas une vision exhaustive} entre réseaux de transport et territoires jouent un rôle significatif dans la complexité des systèmes territoriaux. Dans le cadre de l'approche d'un système territorial par la définition donnée ci-dessus\comment[FL]{dans quelle section}, cette question peut être reformulée comme l'étude de systèmes territoriaux réticulaires, avec une emphase sur le rôle des systèmes de transports. On a vu que l'étendue des échelles spatiales et temporelles va de celle de la mobilité quotidienne (micro-micro)\comment[FL]{ces parenthese sont tres rapides $\rightarrow$ pourquoi mob que micro-micro ; etc} à des processus sur le temps long dans les systèmes de villes (macro-macro), avec la possibilité de combinaisons intermédiaires. La précision des échelles particulièrement pertinentes fera l'objet de la majorité des préliminaires (Partie 1) et des fondations (Partie 2), jusqu'au Chapitre~\ref{ch:morphogenesis} qui conclura les fondations. Donnons à présent des exemples concrets clarifiant la complexité des interactions et la nécessité de considérer une co-évolution.
}



\paragraph{Diversity of interactions}{Diversité des interactions}

\bpar{}{
\cite{heddebaut:hal-01355621} montre pour l'impact des infrastructures sur le long terme, dans le cas du tunnel sous la Manche\comment[FL]{c'est quoi ?}, que les effets effectivement constatés pour la région Nord-Pas-de-Calais comme un gain de centralité et de visibilité au niveau Européen, sont en fort décalage avec les discours justifiant le projet, et que les retombées économiques directes locales se sont rapidement estompées \comment[FL]{un peu rapide : comment est faite la ``preuve'' ? on ne peut pas critiquer si on ne sait rien}: on rejoint l'idée défendue par \noun{Bretagnolle} dans \cite{espacegeo2014effets}\comment[AB]{cit ?} selon laquelle des ``effets de structure'' effectivement existent mais que ceux-ci se manifestent sur le temps long en termes de dynamiques systémiques pour lesquelles une vision locale courte n'a que peu de sens. Le possible jeu de mot par le titre ambigu sur l'existence du ``Tunnel effect''\comment[FL]{existe aussi en francais}[(JR) jeu de mot sur le titre du papier] rappelle l'effet tunnel, qui réside en la non-interaction d'une infrastructure sur un territoire le traversant sans s'y arrêter. A l'échelle intra-urbaine, \cite{fritsch2007infrastructures} prend l'exemple du Tramway de Nantes pour montrer que les dynamiques de densification urbaine sont bien en dessous des anticipations\comment[FL]{phrase floue} des élus et planificateurs. \cite{doi:10.1080/01441647.2016.1168887} procède à une revue systématique des études empiriques des impacts à moyen terme des infrastructures de transport, et montre qu'une densification urbaine à proximité des nouvelles infrastructures est relativement systématique\comment[FL]{mal dit}, résidentielle dans le cas d'une infrastructure ferroviaire et pour les emplois et l'activité industrielle et commerciale pour le réseau routier. Pour reprendre l'exemple bien particulier des villes de montagne, \cite{torricelli2002traversees} montre comment dans ce contexte il est possible de faire un lien entre nature des flux de transport et développement local du système urbain : les villes de montagne ont d'abord émergé comme point de passage sur les chemins de col, puis ont perdu de leur importance avec l'avènement des routes. L'arrivée du chemin de fer a pu les re-dynamiser, par le tourisme et l'industrie, et enfin l'autoroute a encore plus récemment induit une déstructuration par des effets de périurbanisation par exemple. Les effets sont naturellement différents selon l'échelle d'observation, comme \cite{RIETVELD1994329} le montre dans une revue des approches économiques des interactions\comment[FL]{plutot : mettre clairement en evidence la diversite des approches disciplinaires sur cette question}, notamment l'importance de différencier l'intra-urbain et l'intra-régional. Les effets des territoires sur les infrastructures sont plus complexes. L'exemple de l'échec de planification de l'aéroport de Ciudad Real en Espagne montre que la réponse d'une infrastructure planifiée n'est absolument pas systématique. \cite{otamendi2008selection} prédisait avant l'ouverture de l'aéroport une gestion complexe due à la dimension des flux attendus et propose un modèle approprié, or les ordres de grandeurs de flux effectifs étaient plus proches des milliers que des millions planifiés et l'aéroport a rapidement fermé. Il est difficile de savoir la raison de l'échec, s'il s'agit de l'optimisme quand au polycentrisme régional (l'aéroport est à mi-chemin de Madrid et Séville), la non-réalisation de la gare sur la ligne à grande vitesse, ou des facteurs purement économiques. Il s'agit très probablement d'une combinaison complexe de multiples facteurs, difficiles à séparer.\comment[FL]{cest dommage de raconter cet exemple pertinent comme cela : un lecteur exterieur n'attendra pas la derniere phrase qui permet de comprendre tout le debut : il passera au paragraphe suivant. [FORME]}
}

\comment[AB]{Aerer}



\paragraph{Planning and Governance}{Gouvernance et Planification}

\bpar{}{
Certains aspects de la gouvernance territoriale peuvent avoir un impact déterminant sur le développement des infrastructures de transport : \cite{deng2007potential} montre dans le cas des villes Chinoises que les nouvelles directives en terme de logement peuvent fortement détériorer la performance des infrastructures, et que des dispositions spécifiques en termes de \emph{Transit Oriented Development} (TOD) doivent être prises pour anticiper ces externalités négatives. Le TOD est une approche particulière de l'aménagement urbain visant à articuler développement de l'offre de transport en commun et développement urbain. Il s'agit en quelque sorte d'une co-évolution volontaire\comment[FL]{de la part de qui ?}, dans laquelle l'articulation est pensée et planifiée. Ces concepts ne sont pas nouveaux, puisqu'ils étaient implicites par exemple dans l'aménagement des villes nouvelles, sous une forme différente puisque celles-ci étaient également fortement zonées\comment[FL]{sens ?} et dépendantes de l'automobile pour certains quartiers. \cite{l2012ville} est un exemple de projet européen ayant exploré des mises en pratiques de paradigmes du TOD : des détails d'aménagement comme un réseau de qualité pour les modes actifs à courte portée sont cruciaux pour une concrétisation des principes. Par exemple, \cite{lhostis:hal-01179934}\comment[AB]{L'Hostis $\subset$ Jury ?} utilise une analyse multi-critères\comment[FL]{sens ?} pour comprendre les facteurs déterminants dans la sélection des stations de la ville planifiée, incluant densité urbaine et temps d'accès aux stations. \cite{LIU2014120} montre que si certaines politiques de planification, en particulier en France, ne se réclament pas directement de cette approche, leurs caractéristiques sont très similaires comme le révèle le cas de Lille. L'articulation entre transport et aménagement doit être opérée de façon fortement couplée, d'autant plus que le projet est spécialisé : \cite{larroque2002paris} rappelle l'anecdote du metro SK de Noisy-le-Grand montre un cas de dépendance complète de la fonctionnalité du transport à l'aménagement local. Pour desservir un projet de complexe de bureau, une ligne spécifique avec une matériel roulant léger est construite pour faire le lien avec la gare RER de Mont-d'Est. Le projet immobilier avortera alors que la ligne est inaugurée en 1993, celle-ci sera d'abord entretenue régulièrement puis laissée a l'abandon sans jamais avoir été ouverte au public.
}




\paragraph{Co-evolution of networks and territories}{Co-évolution des réseaux et des territoires}


\bpar{}{
La complexité des interactions entre réseaux et territoires nécessite de se placer dans une ontologie\comment[FL]{def ontologie ?} particulière, celle d'une \emph{co-évolution}.\comment[FL]{c'est central donc cela doit etre beaucoup plus tot} \cite{levinson2011coevolution} souligne la difficulté de la compréhension de la co-évolution entre transport et usage du sol en termes de causalités circulaires, en partie à cause des différentes échelles de temps impliquées, mais aussi par l'hétérogénéité des composantes. \cite{offner1993effets} parle de congruence, qu'on peut comprendre comme une dynamique systémique impliquant des corrélations fortuites ou non, à lier avec la vision systémique de l'époque, ce qui serait une vision préliminaire de la co-évolution. La nécessité de dépasser les approches réductrices des effets structurants, tout en capturant la complexité des interactions entre réseaux et territoires par leur co-évolution, est confirmée par le cas des effets économiques des trains à grande vitesse : \cite{Blanquart2017} procède à une revue à la fois théorique et empirique, incluant la littérature grise, des études de ce cas spécifiques, et conclut, au delà des retombées directes liées à la construction sur lesquelles il y a consensus, à des effets en apparence aléatoires si les sujets sont considérés hors contexte, témoignant de situation locales bien plus complexes, un grand nombre d'aspect conjoncturels entrant en jeu dans la production d'effets, qu'on ne peut alors pas attribuer seulement au transport : il y a bien co-évolution entre les différentes composantes du système. Cette revue confirme le décalage entre les discours politiques et techniques prévalant aux projets de transports et les analyses effectives a posteriori révélée par~\cite{bazin:hal-00615196}. \cite{bazin2007evolution} avait d'autre part procédé à une étude ciblée du marché immobilier à Reims en anticipation de l'arrivée du TGV Est, et avait conclu que seul des opérations très localisées pouvaient être directement reliées au TGV, l'ensemble du marché répondant à une dynamique globale indépendante.\comment[FL]{interessant. meme si la aussi developper le node d'administration de la preuve serait pertinent} Ainsi, cette notion de co-évolution que nous préciserons par la suite est une bonne candidate pour capturer la complexité de relations circulaires causales.\comment[FL]{1) ce n'est pas clair en quoi cette phrase conclut cette question ; 2) tu dois absolument definir bien en amont ce terme de coevolution}
}



\stars




