


\newpage


%-------------------------------


\section{Territories and Networks}{Réseaux et Territoires}

\label{sec:networkterritories}


%-------------------------------



\subsection{Territories and Networks : There and Back Again}{Une circularité naturelle}

\paragraph{Human Territories}{Territorialité Humaine}


\bpar{
The notion of territory can be taken as a basis to explore the scope of geographical objects we will study.
 In Ecology, a territory corresponds to a spatial extent occupied by a group of agents or more generally an ecosystem. \emph{Human Territories} are far more complex in the sense of semiotic representations of these that are a central part in the emergence of societies. 
  For \noun{Raffestin} in~\cite{raffestin1988reperes}, the so-called \emph{Human Territoriality} is the ``conjonction of a territorial process with an informational process'', what means that physical occupation and exploitation of space by human societies is not dissociable from the representations (cognitive and material) of these territorial processes, driving in return its further evolutions. In other words, as soon as social constructions are assumed in the constitution of human settlements, concrete and abstract social structures will play a role in the evolution of the territorial system, through e.g. propagation of information and representations, political processes, conjonction or disjonction between lived and perceived territory. Although this approach does not explicitly give the condition for the emergence of a seminal system of aggregated settlements (i.e. the emergence of cities), it insists on the role of these that become places of power and of creation of wealth through exchange. But the city has no existence without its hinterland and the territorial system can not be summarized by its cities as a system of cities. There is however compatibility on this subsystem between \noun{Raffestin} approach to territories and \noun{Pumain}'s evolutive theory of urban systems~\cite{pumain2010theorie}, in which cities are viewed as an auto-organized complex dynamical systems, and act as mediators of social changes : for example, cycles of innovation occur within cities and propagate between them. Cities are thus competitive agents that co-evolve (in the sense given before). The territorial system can be understood as a spatially organized social structure, including its concrete and abstract artifacts. A imaginary free-of-man spatial extent with potential ressources will not be a territory if not inhabited, imagined, lived, and exploited, even if the same ressources would be part of the corresponding habited territorial system. Indeed, what is considered as a ressource (natural or artificial) will depend on the corresponding society (e.g. of its practices and technological potentialities). A crucial aspect of human settlements that were studied in geography for a long time, and that relate with the previous notion of territory, are \emph{networks}. Let see how we can switch from one to the other and how their definition may be indissociable.
}{
Une entrée possible dans l'ensemble des objets géographiques que nous proposons d'étudier est la notion de territoire.\comment{(Florent) et un objet de recherche en lui meme}
 En Ecologie, un territoire correspond à l'étendue spatiale occupée par un groupe d'agent ou plus généralement un écosystème. Les \emph{Territoires Humains} sont extrêmement plus complexes de par l'importance de leur représentations sémiotiques, qui jouent un rôle significatifs dans l'émergence des constructions sociétales.\comment{(Florent)pas besoin ni interet de se positionner sur emergence des societes}
  Selon \noun{Raffestin} dans~\cite{raffestin1988reperes}, la \emph{Territorialité Humaine} est ``la conjonction d'un processus territorial avec un processus informationnel'', ce qui implique que l'occupation physique et l'exploitation de l'espace par les sociétés humaines n'est pas dissociable \comment{(Florent) ou est complémentaire ?}
   des représentations (cognitives et matérielles) de ces processus territoriaux, qui influent en retour leur évolution. En d'autres termes, à partir de l'instant où les constructions sociales déterminent la constitution des établissements humains, les structures sociales abstraites et concrètes joueront un role dans l'évolution des systèmes territoriaux, par exemple à travers la propagation d'informations et de représentations, par des processus politiques, ou encore par la correspondance effective entre territoire vécu et territoire perçu.  \comment{(Florent)donner exemples concrets serait pédagogique (ex metropole grd paris, cf articles)}
    Bien que cette approche ne donne pas de conditions explicites pour l'émergence d'un système séminal d'établissements agrégés (c'est à dire l'émergence des villes), \comment{(Florent)pourquoi cette interrogation particulière ?}
     elle insiste sur leur role comme lieu de pouvoir et de création de richesse au travers des échanges. Mais la ville n'a pas d'existence sans son hinterland et le système territorial peut difficilement être résumé par ses villes, comme un système de villes. En se restreignant à ce sous-système, il y a toutefois compatibilité entre la théorie de territoires de \noun{Raffestin} et la théorie évolutive des villes de \noun{Pumain}~\cite{pumain2010theorie}, qui interprète les villes comme des systèmes complexes dynamiques auto-organisés,\comment{(Arnaud) self-organized ?} qui agissent comme des médiateurs du changement social : par exemple, les cycles d'innovation s'initialisent au sein des villes et se propagent entre elles.  \comment{(Florent)tres pertinent bien sur, mais va aborder la question de l'innovation dans la thèse ?}
      Les villes sont ainsi des agents \comment{(Arnaud) entities ?} compétitifs qui co-évoluent (au sens donné précédemment). Le système territorial peut ainsi être compris comme une structure sociale organisée dans l'espace, qui comprend ses artefacts concrets et abstraits. Une étendue spatiale imaginaire avec des ressources potentielles qui n'aurait jamais connu de contact avec l'humain ne pourra pas être un territoire si elle n'est pas habitée, imaginée, vécue, exploitée, même si ces ressources pourraient être potentiellement exploitée le cas échéant. En effet, ce qui est considéré comme une ressource (naturelle ou artificielle) dépendra de la société (par exemple de ses pratiques et de ses capacité technologiques). \cite{di1998espace} procède à une analyse historique des différentes conceptions de l'espace (qui aboutissent entre autre à l'espace vécu, l'espace social et l'espace classique de la géographie) et montre comment leur combinaison forme ce que \noun{Raffestin} décrit comme territoires.
       Un aspect central des établissements humains qui a une longue tradition d'étude en géographie, et qui est directement relié à la notion de territoire, est celui des \emph{réseaux}. Nous allons voir comment le passage de l'un à l'autre est inévitable et leur définition indissociable.
       \comment{(Florent)structure generale de l'argumentaire tb, mais devrait expliquer plus en détail ce qu'on appelle réseau (avant de détailler les différents réseaux réel/virtuel, les réseaux ont une inscription spatiale}
}


\paragraph{A Territorial Theory of Networks}{Une théorie territoriale des réseaux}


\bpar{
We paraphrase \noun{Dupuy} in~\cite{dupuy1987vers} when he proposes elements for ``a territorial theory of networks'' based on the concrete case of Urban Transportation Networks. This theory sees \emph{real networks} (i.e. concrete networks, including transportation networks) as the materialization of \emph{virtual networks}. More precisely, a territory is characterized by strong spatio-temporal discontinuities induced by the non-uniform distribution of agents and ressources. These discontinuities naturally induce a network of ``transactional projects'' that can be understood as potential interactions between elements of the territorial system (agents and/or ressources). For example today, people need to access the ressource of employments, economic exchanges operate between specialized production territories. At any time period, potential interactions existed\footnote{even when nomadism was still the rule, spatially dynamic networks of potential interactions necessarily existed, but should have less chance to materialize into concrete routes.% bib on that ?
}. The potential interaction network is concretized as offer adapts to demand, and results of the combination of economic and geographical constraints with demand patterns, in a non-linear way through agents designed as \emph{operators}. This process is not immediate, leading to strong non-stationarity and path-dependance effects : the extension of an existing network will depend on previous configuration, and depending on involved time scales, the logic and even the nature of operators may have evolved. \noun{Raffestin} points out in his preface of~\cite{offner1996reseaux} that a geographical theory articulating space, network and territories had never been consistently formulated. It appears to still be the case today, but the theory developed just before is a good candidate, even if it stays at a conceptual level. The presence of a human territory necessarily imply the presence of abstract interaction networks and concrete networks used for transportation of people and ressources (including communication networks as information is a crucial ressource). Depending on regime in which the considered system is, the respective role of different networks may be radically different. Following \noun{Duranton} in \cite{duranton1999distance}, pre-industrial cities were limited in growth because of limitations of transportation networks. Technological progresses have lead to the end of these limitations and the preponderance of land markets in shaping cities (and thus a role of transportation network as shaping prices through accessibility), and recently to the rising importance of telecommunication networks that induce a ``tyranny of proximity'' as physical presence is not replaceable by virtual communication. This territorial approach to networks seems natural in geography, since networks are studied conjointly with geographical objects with an underlying theory, in opposition to network science that studies brutally spatial networks with few thematic background~\cite{ducruet2014spatial}.
}{
Nous paraphrasons \noun{Dupuy} dans~\cite{dupuy1987vers} lorsqu'il propose des éléments pour une ``théorie territoriale des réseaux'' basée sur le cas concret d'un réseau de transport urbain. Cette théorie présente les \emph{réseaux réels} (i.e. les réseaux concrets % TODO différent de réseaux matériels ?
, incluant les réseaux de transport) comme la matérialisation de \emph{réseaux virtuels}. \comment{(Florent)dans un second temps seulement, à ce stade ``de qui viennent les réseaux'' n'est pas une question cruciale, c'est la question réseau/espace/human settlements qui doit être au coeur}
 Plus précisément, un territoire est caractérisé par de fortes discontinuités spatio-temporelles induites par la distribution non-uniforme des agents \comment{(Arnaud) Ontology}
  et des ressources. Ces discontinuités induisent naturellement un réseau de ``projets transactionnels'' \comment{(Florent)pquoi guillemets?}
 qui peuvent être compris comme des interactions potentielles entre les éléments du système territorial % TODO remarque : cf Chenyi modèles de potentiels -> pertinence de cette approche, même au regard du modèle macro ?
(agents et/ou ressources). Par exemple, de nos jours les actifs se doivent d'accéder à la ressource qu'est l'emploi, et des échanges économiques s'effectuent entre les différents territoires spécialisés dans les productions de différents types. En tout temps des interactions potentielles ont existé\footnote{même quand le nomadisme devait encore être la règle, des réseaux d'interactions potentielles dynamiques dans l'espace ont du exister, mais devaient avoir moins de chance de se matérialiser en des routes matérielles.} Le réseau d'interaction potentiel est concrétisé quand l'offre s'adapte à la demande, et résulte en la combinaison de contraintes économiques et géographiques avec les motifs de demande, de manière non-linéaire via des agents qu'on peut désigner comme \emph{opérateurs}. Un tel processus est loin d'être immédiat, et conduit à de forts effets de non-stationarité et de dépendance au chemin  \comment{(Florent)Une strategie à adopter serait d'abord de decrire de facon basique, avec exemples concrets, la complexité des interactoins réseau/espace/settelements, puis de rappeler CS et proprietes, puis de decrire lesquelles de ces propriétés presentes dans ces interactions, lequelles modèles vont essayer de reproduire et pquoi.}
 : l'extension d'un réseau existant dépendra de la configuration précédente, et selon les échelles de temps impliquées, la logique et même la nature des opérateurs peut avoir évolué. \noun{Raffestin} souligne dans sa préface de~\cite{offner1996reseaux} qu'une théorie géographique articulant espaces, réseaux et territoires n'a jamais été formulée de manière cohérente. \comment{(Florent)redire les ecueils qui sont perçus par Raffestin}
 Il semble que c'est toujours le cas aujourd'hui, même si la théorie évoquée ci-dessus semble être un bon candidat bien qu'elle reste à un niveau conceptuel. La présence d'un territoire humain implique nécessairement la présence de réseaux d'interactions abstraites et de réseaux concrets utilisés pour transporter les individus et les ressources (incluant les réseaux de communication puisque l'information est une ressource essentielle). Selon le régime dans lequel le système considéré se trouve, le rôle respectif du réseau peut être radicalement différent. Selon \noun{Duranton}~\cite{duranton1999distance}, les villes pré-industrielles étaient limitées en croissance de par les limitations des réseaux de transport. Les progrès technologiques ont permis de les surmonter  \comment{(Florent)trop simplificateur}
et à mené à la prépondérance du marché foncier dans la formation des villes (et par conséquent un rôle des réseaux de transport qui déterminent les prix par l'accessibilité), et plus récemment à une importance croissante des réseaux de télécommunication ce qui a induit une ``tyrannie de la proximité'' puisque la présence physique n'est pas remplaçable par une communication virtuelle. Cette approche territoriale des réseaux semble naturelle en géographie, puisque les réseaux sont étudiés conjointement avec des objets géographiques auxquels est associée une théorie, en opposition à la science des réseaux qui étudie brutalement les réseaux spatiaux avec peu de fond thématique~\cite{ducruet2014spatial}. \comment{(Florent)derniere phrase pas claire} \comment{(Arnaud)  Ajouter noms ? (biblio ?}
}

\paragraph{Networks shaping territories ?}{Des réseaux qui façonnent les territoires ?}

% how do network shape territories : boundaries, scales, etc.
% example : \cite{l2012ville} bahn-ville, volontary coevol ? // idem villes nouvelles


\bpar{
However networks are not only a material manifestation of territorial processes, but play their part in these processes as they evolution may shape territories in return. In the case of \emph{technical networks}, an other designation of real networks given in~\cite{offner1996reseaux}, many examples of such feedbacks can be found : the interconnectivity of transportation networks allows multi-scalar mobility patterns, thus shaping the lived territory. At a smaller scale, changes in accessibility may result in an adaptation of a functional urban space. Here emerges again an intrinsic difficulty : it is far from evident to attribute territorial mutations to some network evolutions and reciprocally materialization of a network to precise territorial dynamics. Coming back to Diderot should help, in the sense that one must not consider network nor territories as independent systems that would have causal relationships but as strongly coupled components of a larger system. The confusion on possible simple causal relationships has fed a scientific debate that is still active. Methodologies to identify so-called \emph{structural effects} of transportation networks were proposed by planners in the seventies~\cite{bonnafous1974detection,bonnafous1974methodologies}. It took some time for a critical positioning on unreasoned and decontextualized use of these methods by planners and politics generally to technocratically justify transportation projects, that was first done by \noun{Offner} in~\cite{offner1993effets}. Recently the special issue~\cite{espacegeo2014effets} on that debate recalled that on the one hand misconceptions and misuses were still greatly present in operational and planning milieus as~\cite{crozet:halshs-01094554} confirmed, and on the other hand that a lot of scientific progresses still need to be made to understand relations between networks and territories as \noun{Pumain} highlights that recent works gave evidence of systematic effects on very long time scales (as e.g. the work of \noun{Bretagnolle} on railway evolution, that shows a kind of structural effect in the necessity of connectivity to the network for cities to ``stay in the game'', but that is not fully causal as not sufficient). At a macroscopic level typical patterns of interaction emerge, but microscopic trajectories of the system are essentially chaotic : the understanding of coupled dynamics strongly depends on the scale considered. At a small scale it seems indeed impossible to show systematic behavior, as \noun{Offner} pointed out. For example, on comparable French mountain territories, \cite{berne2008ouverture} shows that reactions to a same context of evolution of the transportation network can lead to very different reactions of territories, some finding a huge benefit in the new connectivity, whereas others become more closed. These potential retroactions of networks on territories does not necessarily act on concrete components : \noun{Claval} shows in~\cite{claval1987reseaux} that transportation and communication networks contribute to the collective representation of territories by acting on territorial belonging feeling.
}{
Cependant les réseaux ne sont pas seulement une manifestation matérielle de processus territoriaux, mais jouent également leur rôle dans ces processus comme leur évolution peut influencer l'évolution des territoires en retour. Dans le cas des \emph{réseaux techniques}, une autre désignation des réseaux réels donnée dans~\cite{offner1996reseaux}, de nombreux exemples de tels retroactions peuvent être mis en évidence : l'interconnexion des réseaux de transport permet des motifs de mobilité multi-échelles, \comment{(Florent)chose plus basiques à dire en premier (favorise croissance urbaine)}
 formant ainsi le territoire vécu. A une plus petite échelle, des changements de l'accessibilité peuvent induire l'adaptation d'un espace fonctionnel urbain. Il emerge alors une difficulté intrinsèque : \comment{(Florent) TB mais en parler avant, c'est cela le coeur}
  il est loin d'évident d'attribuer des mutations territoriales à une évolution du réseau and réciproquement la matérialisation d'un réseau à des dynamiques territoriales précises. Revenir à la citation de Diderot devrait aider à ce point, au sens où il ne faut pas considérer le réseau ni les territoires comme des systèmes indépendants qui s'influenceraient mutuellement par des relations causales, mais comme des composantes fortement couplées d'un système plus large. La confusion autour de possibles relations causales simples a nourri un débat scientifique encore actif aujourd'hui. Les méthodologies pour identifier ce qui est nommé \emph{effets structurants} des réseaux de transport ont été proposées par les planificateurs dans les années 1970~\cite{bonnafous1974detection,bonnafous1974methodologies}. Il aura fallu un certain temps pour un positionnement critique sur l'usage non raisonné et decontextualisé de ces méthodes par les planificateurs et les politiques qui les mobilisaient généralement pour justifier des projets de transports de manière technocratique. Cela a été fait en premier par \noun{Offner} dans~\cite{offner1993effets}. Récemment un édition spéciale du même journal sur ce débat~\cite{espacegeo2014effets} a rappelé d'une part que les mauvaises interprétations et les mauvais usages étaient encore largement présent aujourd'hui dans les milieux opérationnels de la planification comme~\cite{crozet:halshs-01094554} confirme, et d'autre part qu'il faudrait encore une certaine quantité de progrès scientifique pour comprendre en profondeur les relations entre réseaux et territoires. Les débats récents en juillet 2017 relatifs à l'ouverture des LGV Bretagne et Sud-Ouest ont montré toute l'ambiguïté des positions, des conceptions, des imaginaires à la fois des politiques mais aussi du public : refus du financement d'élus qui s'attendaient au prolongement vers Toulouse et l'Espagne, spéculation dans les quartiers de gare, questionnements des pratiques de mobilité quotidienne mais aussi sociale. La complexité et la portée des sujets montre bien la difficulté d'une compréhension systématique d'effets du transport sur les territoires. \noun{Pumain} souligne que des travaux récents ont révélé des effets systématiques sur de très longues échelles temporelles (comme e.g. le travail de \noun{Bretagnolle} sur l'évolution des chemins de fer, qui montre une sorte d'effet structurel sur la nécessité de connexion au réseau des villes, afin de rester actives, mais qui n'est ni suffisant ni totalement causal). \comment{(Florent)développer ce genre de categories macro c'est très interessant}
    A un niveau macroscopique des motifs typiques d'interaction émergent, mais les trajectoires microscopiques du systèmes sont essentiellement chaotiques : la compréhension des dynamiques couplées dépend fortement de l'échelle considérée. A une petite échelle il est peu raisonnable de vouloir montrer des comportement systématiques, comme le rappelle \noun{Offner}. Par exemple, sur des territoires de montagne français comparables, \cite{berne2008ouverture} montre que les réactions à un même contexte d'évolution du réseau de transport peut mener à des réactions territoriales très diverses, certains trouvant de forts bénéfices par la nouvelle connectivité, d'autres au contraire devenant plus fermés. Ces retroactions potentielles des réseaux sur les territoires n'agit pas nécessairement sur des composantes concretes : \noun{Claval} montre dans~\cite{claval1987reseaux} que les réseaux de transport et de communication contribuent à la représentation collective d'un territoire en agissant sur un sentiment d'appartenance.  \comment{(Florent) la encore de second ordre, a ressortir pour lutetia} % TODO interesting, put that into perspective // DPR ?
}



\paragraph{Territorial Systems}{Systèmes Territoriaux}


\bpar{
This detour from territories, to networks and back again, allows us to give a preliminary definition of a territorial system that will be the basis of our following theoretical considerations. As we emphasized the role of networks, the definition takes it into account.
}{
Ce voyage des territoires aux réseaux, et retour, nous permet d'esquisser une définition préliminaire d'un système territorial sur laquelle se basera les considérations théoriques suivantes.  \comment{(Florent)si c'est autant au coeur, présenter avant}
Comme nous avons mis en exergue le rôle des réseaux, la définition se doit de les prendre en compte.
}



\bigskip


\bpar{
\textbf{Preliminary Definition.} \textit{A territorial system is a human territory to which both interaction and real networks can be associated. Real \comment{(Arnaud)  a préciser (cf réseaux sociaux)}
 networks are a component of the system, involved in evolution processes, through multiples feedbacks with other components at various spatial and temporal scales.}
}{
\textbf{Définition provisoire.} \textit{Un Système Territorial est un territoire humain auquel peuvent être associés à la fois un réseau d'interactions et un réseau réel. Les réseaux réels sont une composante à part entière du système, jouant dans les processus d'évolution, au travers de multiples retroactions avec les autres composantes à plusieurs échelles spatiales et temporelles.}
}

 \comment{(Florent) feedback : propriété, pas def ; plus une axiomatique qu'une demo ?}

\bigskip



\bpar{
This reading of territorial systems is conditional to the existence of networks and may discard some human territories, but it is a deliberate choice that we justify by previous considerations, and that drives our subject towards the study of interactions between networks and territories.
}{
Cette lecture des systèmes territoriaux est conditionnée à l'existence des réseaux et pourrait écarter certains territoires humains, mais il s'agit d'un choix délibéré justifié par les considérations précédentes, et qui précise notre sujet vers l'étude des interactions entre réseaux et territoires. \comment{(Florent) formulé comme ça, on peut penser que network pas inclus dans territoire}
}



\subsection{Transportation Networks}{Réseaux de Transport}


\paragraph{The particularity of transportation networks}{La particularité des réseaux de transport}



\bpar{
Already evoked in relation to the question of structural effects of networks, transportation networks play a determining role in the evolution of territories. Although other types of networks are also strongly involved in the evolution of territorial systems (see e.g. the discussions of impacts of communication networks on economic activities), transportation networks shape many other networks (logistics, commercial exchanges, social concrete interactions to give a few) and are prominent in territorial evolution patterns, especially in our recent societies that has become dependent of transportation networks~\cite{bavoux2005geographie}. The development of French High Speed Rail network is a good illustration of the impact of transportation networks on territorial development policies. Presented as a new era of railway transportation, a top-down planning of totally novel lines was introduced as central for developments~\cite{zembri1997fondements}. The lack of integration of these new networks with existing ones and with local territories is now observed as a structural weakness and negative impacts on some territories have been shown~\cite{zembri2008contribution}. A review done in~\cite{bazin2011grande} confirms that no general conclusions on local effects of High Speed lines connection can be drawn although it keeps a strong place in imaginaries. These are examples of how transportation networks have both direct and indirect impacts on territorial dynamics. Integrated planning, in the sense of a joint planning of transportation infrastructures and urban development, considers the network as a determining component of the territorial system. Parisian \emph{Villes Nouvelles} are such a case, that witnesses of the complexity of such planning actions that generally do not lead to the desired effect~\cite{es119}. Recent projects as~\cite{l2012ville} have try to implement similar ideas but we have now not enough temporal scope to judge their success in effectively producing an integrated territory. Transportation networks are anyway at the center of these approaches of urban territories. We will focus in our work on transportation networks for the various reasons given here.
}{
Déjà évoqués dans le cas des effets structurants des réseaux, les réseaux de transports jouent un rôle central dans l'évolution des territoires, mais il n'est évidemment pas question de leur attribuer des effets causaux déterministes. Même si d'autres types de réseaux sont également fortement impliqués dans l'évolution des systèmes territoriaux (voir e.g. les débats sur l'impact des réseaux de communication sur la localisation des activités économiques), les réseaux de transport conditionnent d'autres types de réseaux (logistique, échanges commerciaux, interactions sociales concrètes pour donner quelques exemples) and semblent dominer dans les motifs d'évolution territoriale, en particulier dans nos sociétés contemporaines qui sont devenues dépendantes des réseaux de transport~\cite{bavoux2005geographie}. Le développement du réseau français à grande vitesse est une illustration pertinente de l'impact des réseaux de transport sur les politiques de développement territorial. Présenté comme une nouvelle ère de transport sur rail, une planification par le haut de lignes totalement nouvelles et indépendantes de par leur vitesse deux fois plus élevée, a été présenté comme central pour le développement~\cite{zembri1997fondements}. Le manque d'intégration de ces nouveaux réseaux avec l'existant et avec les territoires locaux est à présent observé comme une faiblesse structurelle et des impacts négatifs sur certains territoires ont été prouvés~\cite{zembri2008contribution}. Une revue faite dans~\cite{bazin2011grande} confirme qu'aucune conclusion générale sur des effets locaux d'une connection à une ligne à grande vitesse ne peut être tirée, bien que ce sésame garde une place conséquente dans les imaginaires des élus. Ces exemples illustrent comment les réseaux de transport peuvent avoir des effets à la fois directs et indirects sur les dynamiques territoriales. Le développement des différentes Lignes à Grande Vitesse s'inscrit dans des contextes territoriaux très différents, et il est dans tous les cas délicat de penser pouvoir interpréter des processus hors de ceux-ci : par exemple, les lignes LGV Nord et LGV Est s'inscrivent dans des échelles européennes plus vastes que la LGV Bretagne ouverte en juillet 2017. La planification intégrée, au sens d'une planification coordonnée entre les infrastructures de transport et le développement urbain, considère le réseau comme une composante déterminante du système territorial. % TODO : note pub TOD in Zhuhai near BeiZhan -- develop on that // in the fieldwork report
Les Villes Nouvelles parisiennes sont un tel cas qui témoigne de la complexité de ces actions de planification qui le plus souvent ne mène pas au effets initialement désirés~\cite{es119}. Des projets récents comme~\cite{l2012ville} ont tenté d'implémenter des idées similaires, mais il manque pour l'instant de recul pour juger de leur succès à produire un territoire effectivement intégré. \comment{(Florent) dans le detail, quels sont les ordres de grandeur des temps pour que les réseaux puissent avoir un effet ?}
 Les réseaux de transports sont dans tous les cas au centre de ces approches des territoires urbains. Nous nous concentrerons par la suite sur les réseaux de transport  \comment{(Florent)tous ?} pour toutes ces raisons évoquées ici.
}



\paragraph{Transportation and Accessibility}{Transports et Accessibilité}

% critic of accessibility as a planning tool : danger of not taking into account socio-eco dynamics and coupled dynamics (coevol) - cit Hadri mobility as a constructed notion.


% TODO : reformulate positioning ?
\cite{miller1999measuring} on three different way to approach accessibility : time-geography and constraints, user utility based measures, and transportation time. It derives measures for each in perspective of \noun{Weibull}'s axiomatic frameworks and reconcile the three in a way.

\bpar{
The notion of accessibility comes rapidly when considering transportation networks. Based on the possibility to access a place through a transportation network (including transportation speed, difficulty of travel), it is generally described as a potential of spatial interaction\footnote{and often generalized as \emph{functional accessibility}, for example employments accessible for actives at a location. Spatial interaction potentials ruling gravity law can also been understood this way.}~\cite{bavoux2005geographie}. This object is often used as a planning tool or as an explicative variable of agents localisation for example. One has to be however careful on its unconditional use. More precisely, it may be a construction that misses a consistent part of territorial dynamics. The mystification of the notion of \emph{mobility} was shown by \noun{Commenges} in~\cite{commenges:tel-00923682}, which proved than most of debates on modeling mobility and corresponding notions were mostly made-of by transportation administrators of \emph{Corps des Ponts} who roughly imported ideas from the United States without adaptation and reflexion fit to the totally different French context. Accessibility may be such a social construct and have no theoretical root since it is mostly a modeling and planning tool. Recent debates on the planification of \emph{Grand Paris Express}~\cite{confMangin}, a totally novel metropolitan transportation infrastructure planned to be built in the next twenty years, have revealed the opposition between a vision of accessibility as a right for disadvantaged territories against accessibility as a driver of economic development for already dynamic areas, both being difficultly compatible since corresponding to very different transportation corridors. Such operational issues confirm the complexity of the role of transportation networks in the dynamics of territorial systems, and we shall give in our work elements of response to a definition of accessibility that would integrate intrinsic territorial dynamics.
}{
La notion d'accessibilité surgit rapidement lorsqu'on s'intéresse aux réseaux de transport. Basée sur la possibilité d'accéder un lieu par un réseau de transport (pouvant prendre en compte la vitesse, la difficulté de se déplacer), elle est généralement définie comme un potentiel d'interaction spatiale\footnote{et souvent généralisée comme une \emph{accessibilité fonctionnelle}, par exemple les emplois accessibles aux actifs d'un lieu. Les potentiels d'interaction spatiaux s'exprimant dans les lois de gravité peuvent aussi être compris de cette façon.}~\cite{bavoux2005geographie}. Cet objet est souvent utilisé comme un outil de planification ou comme une variable explicative de localisation des agents par exemple.  \comment{(Florent) dire d'abrd à quoi peut servir} Il faut cependant rester prudent sur son usage inconditionnel. Plus précisément, il peut s'agir d'une construction qui ignore une partie conséquente des dynamiques territoriales. La mystification  \comment{(Florent) trop fort, Hadri montre que étude et prod de l'infra sont pas indep, mais pas de myst} \comment{(Arnaud) Contexte français}
 de la notion de \emph{mobilité} a été montrée par \noun{Commenges} dans~\cite{commenges:tel-00923682}, qui révèle que la majorité des débats sur la modélisation de la mobilité et les notions correspondantes était majoritairement construites de manière ad-hoc par les administrateurs de transports issus du \emph{Corps des Ponts}  \comment{(Florent) lecture trop rapide}
  qui importaient brutalement les outils et méthodes des Etats-Unis sans adaptation ni reflexion adaptée au contexte français. L'accessibilité pourrait de même être une construction sociale et n'avoir que peu de fondement théorique, puisqu'il s'agit en grande partie d'un outil de modélisation et de planning. Les débats récents sur la planification du \emph{Grand Paris Express}~\cite{confMangin}, \comment{(Florent) interessant : à creuser}
   cette nouvelle infrastructure de transport métropolitaine planifiée pour les vingts prochaines années, a révélé l'opposition entre une vision de l'accessibilité comme un droit pour les territoires désavantagés, contre l'accessibilité comme un moteur du développement économique pour des zones déjà dynamiques, les deux étant difficilement compatibles car correspondent à des couloirs de transport très différents. De tels problèmes opérationnels confirment la complexité du rôle des réseaux de transports dans les dynamiques des systèmes territoriaux, et nous devrons donner dans notre travail des éléments de réponse pour une définition de l'accessibilité qui intégrerait les dynamiques territoriales intrinsèques.
}

\paragraph{Scales and Hierarchies}{Echelles et Hierarchies}


% \cite{10.1371/journal.pone.0102007}
% \cite{Tsekeris20131} : congestion related to land-use


\bpar{
An incontournable aspect of transportation networks that we will need to take into account in further developments is hierarchy. Transportation networks are by essence hierarchical, depending on scales they are embedded in. \cite{10.1371/journal.pone.0102007} showed empirical scaling properties for public transportation networks for a consequent number of metropolitan areas across the world, and scaling laws reveal the presence of hierarchy within a system, as for size hierarchy for system of cities expressed by Zipf's law~\cite{nitsch2005zipf} or other urban scaling laws~\cite{2013arXiv1301.1674A,2015arXiv151000902B}. Transportation network topology has been shown to exhibit such scaling also for the distribution of its local measures such as centrality~\cite{samaniego2008cities}. Hierarchy seems to play a particular role on interaction processes, as \noun{Bretagnolle}~\cite{bretagnolle:tel-00459720} highlighted an increasing correlation in time between urban hierarchy and network hierarchy for French railway network, marker of positive feedbacks between urban rank and network centralities. Different regimes in space and times were identified: for French railway network evolution e.g., a first phase of adaptation of the network to the existing urban configuration was followed by a phase of co-evolution i.e. in the sense that causal relations became difficult to identify. The impact of space-time contraction by the network on patterns of growth potential had already been shown for Europe with an exploratory analysis in~\cite{bretagnolle1998space}. Railway evolution in the United States followed a different pattern, without hierarchical diffusion, shaping locally urban growth. It emphasizes the presence of path-dependance for trajectories of urban systems: the presence in France of a previous city system and network (postal roads) strongly shaped railway development, whereas its absence in the US lead to a completely different story. An open question is if generic processes underlie both evolutions, each being different realizations with different initial conditions and different meta-parameters (different \emph{regimes} in the sense of settlement systems transitions introduced in the current ANR Research project TransMonDyn, as a transition can be understood as a change of stationarity for meta-parameters of a general dynamic). In terms of dynamical systems formulation, it is equivalent to ask if dynamics of attractors (long time scale components) obey similar equations as the position and nature of attractors for a stochastic dynamical system that give its current regime, in particular if it is in a divergent state (positive local Liapounov exponent) or is converging towards stable mechanisms~\cite{sanders1992systeme}. To answer this question together with a disentangling of co-evolution processes for that regime, \cite{bretagnolle:tel-00459720} proposes modeling as a constructive element of answer. We will see in next section how modeling can bring knowledge about territorial processes.
}{
Un aspect incontournable des réseaux de transport que nous devrons prendre en compte dans nos développements futurs et la hiérarchie. Les réseaux de transport sont par essence hiérarchique, dépendant des échelles dans lesquelles ils sont intégrés. \cite{10.1371/journal.pone.0102007} montre empiriquement des propriétés de loi d'échelle pour un nombre conséquent d'aires métropolitaines à travers la planète, et les lois d'échelle révèlent la présence de hiérarchie dans un système, comme pour la hiérarchie de taille dans les systèmes de villes exprimée par la loi de Zipf~\cite{nitsch2005zipf} ou d'autres lois d'échelle urbaines~\cite{2013arXiv1301.1674A,2015arXiv151000902B}. La topologie du réseau de transport a été montrée suivre de telles lois pour la distribution de ses mesures locales comme la centralité~\cite{samaniego2008cities}. \comment{(Florent) tb mais comment relie à partie juste avant ?}
 La hiérarchie semble jouer un rôle particulier dans les processus d'interaction, comme \noun{Bretagnolle}~\cite{bretagnolle:tel-00459720} a souligné une correlation croissante dans le temps entre la hiérarchie urbaine et la hiérarchie de l'accessibilité temporelle pour le réseau ferroviaire français, \comment{(Florent) tb mais séparé entre réseau et pop ; pourquoi pas regarder la hiérarchie de l'access ?}
  marqueur de retroactions positives entre le rang urbain et la centralité de réseau. Différents régimes dans le temps et l'espace ont été identifiés : pour l'évolution du réseau ferroviaire français e.g., une première phase d'adaptation du réseau à la configuration urbaine existante a été suivie par une phase de co-évolution i.e. au sens où les relations causales sont devenues difficiles à identifier. L'impact de la contraction de l'espace-temps par les réseaux sur le potentiel de croissance des villes avait déjà été montré pour l'Europe par des analyses exploratoires dans~\cite{bretagnolle1998space}. L'evolution du réseau ferroviaire aux Etats-unis a suivi une dynamique bien différente, sans diffusion hiérarchique, donnant forme localement à la croissance urbaine. \comment{(Florent) un peu rapide mais dans l'autre sens : cela n'a pas marché partout mais contexte particulier de la conquete de l'ouest est intéressant à souligner}
   Cela met l'emphase sur la présence de dépendance au chemin \comment{(Florent) en parler avant si c'est le coeur du projet}
   pour les trajectoires des systèmes urbains : la présence en France d'un système préalable de villes et de réseau (routes postales) a fortement influencé le développement du réseau ferré, tandis que son absence aux Etats-unis a conduit à une histoire complètement différente. Une question ouverte est si des processus génériques sont implicites aux deux évolutions, chacun correspondant à des réalisations différentes avec des conditions initiales et des méta-paramètres différentes (des \emph{régimes} différents au sens des transitions des systèmes de peuplement introduites dans le projet de recherche courant ANR TransMonDyn, puisque une transition peut être comprise comme un changement de stationnarité des méta-paramètres \comment{(Florent) trop rapide ce n'est pas compréhensible en l'état}
    d'une dynamique générale). En termes de systèmes dynamiques, cela revient à se demander si les dynamiques des ensembles de catastrophe (composantes à plus grandes échelles temporelles) obéissent à des équations similaires que la position et nature des attracteurs pour un système dynamique stochastique qui donnent son régime courant, en particulier si le système est dans un état local divergent (exposant de Liapounov local positif) ou en train de converger vers des mécanismes stables~\cite{sanders1992systeme}. Pour répondre à cette question en même temps que l'isolation des processus de co-évolution pour ce régime, \cite{bretagnolle:tel-00459720} propose la modélisation comme élément de réponse constructif. Nous verrons dans le chapitre suivante comme la modélisation peut être source de connaissance sur les processus territoriaux. 
}


\paragraph{Transportation and Mobility}{Transports et Mobilité}

% TODO : ajouter au moins un paragraphe sur la mobilité, faire le schéma avec Thèse Eugenia ?
sur la mobilité : nos questionnements à une autre échelle ? cf \cite{fusco2004mobilite} relations causales



%%%%%%%%%%%%%%%%%%%
\subsection{Interactions between transportation networks and territory}{Interactions entre Réseaux et Territoires}

% a subsection detailing what we call interaction, giving several examples, what scales in time and space : precise more the subject in some way !

At this state of progress, we have naturally identified a research subject that seems to take a significant place in the complexity of territorial systems, that is the study of interactions between transportation networks and territories. In the frame of our preliminary definition of a territorial system, this question can be reformulated as the study of networked territorial systems with an emphasize on the role of transportation networks in system evolution processes.

\comment{(Florent) ok : à quelles échelles de temps et d'espace se place t'on (même un intervalle)}

- ici donner des exemples concrets -

Gaelle Lesteven Metro toulouse

\comment{(Florent) aéroport MCR : Ciudad Real}



\paragraph{Co-evolution of networks and territories}{Co-évolution des réseaux et des territoires}

On se place déjà dans l'idée d'une co-evolution - introduire le concept selon exemples empiriques et littérature. 

\cite{offner1993effets} parle de congruence - à lier avec vision systémique de l'époque - serait une vision préliminaire de la co-évolution.








