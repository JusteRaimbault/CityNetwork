



%----------------------------------------------------------------------------------------

\newpage


\section[SimpopNet Exploration][Exploration de SimpopNet]{SimpopNet Exploration}{Exploration de SimpopNet}

%----------------------------------------------------------------------------------------

\todo{éventuellement ici pas seulement explorer simpopnet mais aussi le Portugali par exemple ?}

\subsection{Context}{Contexte}

Quelle différentiel de connaissances obtenues peut s'observer, de la description conceptuelle ou thématique d'un modèle, à sa formalisation mathématique, son implémentation, son exploration systématique, jusqu'à son exploration approfondie à l'aide de meta-heuristiques spécifiques ? Notre postulat, qui découle à la fois de notre positionnement (voir \autoref{ch:positioning} sur la simulation) et d'expériences dont les modèles déroulés précédemment font partie, est que celui-ci est important, mais surtout de nature \emph{qualitative}, c'est à dire que la nature même des connaissances subit des transitions abruptes lors de l'avancée de la démarche dans ce continuum. Le modèle SimpopNet introduit par~\cite{schmitt2014modelisation}, qui est à notre connaissance l'unique modèle de co-évolution dans une perspective de la théorie évolutive des villes, est un exemple d'une telle démarche préliminaire qui nécessite d'être creusée, par exemple par l'exploration systématique.


\subsection{Model Description}{Description du Modèle}


\subsection{Results}{Résultats}




