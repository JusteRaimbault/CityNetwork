



%----------------------------------------------------------------------------------------

\newpage


\section[SimpopNet Exploration][Exploration de SimpopNet]{SimpopNet Exploration}{Exploration de SimpopNet}

%----------------------------------------------------------------------------------------

% éventuellement ici pas seulement explorer simpopnet mais aussi le Portugali par exemple ?  pas le temps !
 

%%%%%%%%%%%%%%%%%%%%%%%
\subsection{Context}{Contexte}

Quelle différentiel de connaissances obtenues peut s'observer, de la description conceptuelle ou thématique d'un modèle, à sa formalisation mathématique, son implémentation, son exploration systématique, jusqu'à son exploration approfondie à l'aide de meta-heuristiques spécifiques ? Notre postulat, qui découle à la fois de notre positionnement (voir \autoref{ch:positioning} sur la simulation) et d'expériences dont les modèles déroulés précédemment font partie, est que celui-ci est important, mais surtout de nature \emph{qualitative}, c'est à dire que la nature même des connaissances subit des transitions abruptes lors de l'avancée de la démarche dans ce continuum. Le modèle SimpopNet introduit par~\cite{schmitt2014modelisation}, qui est à notre connaissance l'unique modèle de co-évolution dans une perspective de la théorie évolutive des villes, est un exemple d'une telle démarche préliminaire qui nécessite d'être creusée, par exemple par l'exploration systématique.


\subsubsection{Model Description}{Description du Modèle}




%%%%%%%%%%%%%%%%%%%%%%%
\subsection{Methodology}{Méthode}

\subsubsection{Spatial configuration}{Configuration spatiale}

space matterisation

\subsubsection{Indicators}{Indicateurs}

Un aspect toujours subtil de l'étude des modèles de simulation est la définition d'indicateurs pertinents, surtout dans le cas de modèles synthétiques où il n'est pas possible de produire des sorties directement liées aux données par exemple. Des faits stylisés très généraux, comme vouloir produire une hiérarchie urbaine ou une hiérarchie de réseau, sont relativement limités. Dans le cas de la hiérarchie particulièrement, les lois obtenues dévient d'une loi d'échelle et il est discutable d'utiliser uniquement la pente d'un ajustement brutal. De plus, la hiérarchie est produite mécaniquement par la majorité des modèles incluant des processus d'agrégation. Il faut donc des indicateurs plus élaborés pour comprendre les dynamiques du système.


Pour se concentrer sur la capacité du modèle à produire des trajectoires à la fois diverses et complexes, et par exemple sa capacité à produire des bifurcations qui se traduiraient par inversions de range, nous proposons les indicateurs suivant pour une variable $X_i(t)$ définie sur chacune des villes et dans le temps (qui pourra être la population ou des mesures de centralité par exemple) :

\begin{itemize}
  \item Indicateurs basiques : hiérarchie, entropie, statistiques descriptives, de la distribution dans le temps
  \item Corrélation de rang initial-final, qui traduit les changements dans la hiérarchie : $\rho\left[X_i(t=0),X_i(t=t_f)\right]$
  \item Diversity des trajectoires, qui capture la diversité de forme des séries temporelles, avec $\tilde{X}_i(t)\in \left[0;1\right]$ les trajectoires mises à l'échelle individuellement,
\[
\frac{2}{N\cdot(N-1)}\sum_{i<j} \left(\frac{1}{T}\int_{t} \left(\tilde{X}_i(t) - \tilde{X}_j(t)\right)^2 \right)^{\frac{1}{2}}
\]
\item Complexité moyenne des trajectoires, la ``complexité'' d'une trajectoire étant donnée simplement par son nombre de points d'inflexion
\item Corrélations en fonction de la distance, pour comprendre la manière dont l'effet de la distance est traduit au niveau macroscopique : 
\[
\hat{\rho}_d\left[(X(\vec{x}_1,Y(\vec{x}_2))|||\vec{x}_1-\vec{x}_2||\sim d\right]
\]
\item Corrélations retardées entre les variations, pour identifier des motifs de causalité entre les variables $X$ et $Y$ : \[
\hat{\rho}_{\tau}\left[\Delta X(t),\Delta Y(t-\tau)\right]
\]
\end{itemize}




%%%%%%%%%%%%%%%%%%%%%%%
\subsection{Results}{Résultats}










