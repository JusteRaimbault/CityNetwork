


% Chapter 

%\chapter{Urban Morphogenesis}{Morphogenèse Urbaine} % Chapter title
\chapter{Morphogenèse Urbaine}

\label{ch:morphogenesis} % For referencing the chapter elsewhere, use \autoref{ch:name} 

%----------------------------------------------------------------------------------------

%\headercit{}{}{}

%\bigskip


Il est bien établi en géographie l'importance des relations spatiales et de la mise en réseau, comme le formule \noun{Tobler} par sa ``première loi de la géographie''~\cite{tobler2004first}. Nous l'avons mis en évidence pour les relations entre réseaux et territoires par exemple en section~\ref{sec:interactiongibrat}. Toutefois, les travaux sur la non-stationnarité et la non-ergodicité, ainsi que la mise en valeur d'échelles locales endogènes, suggèrent une certaine pertinence à l'idée de sous-système relativement indépendant, au sens où il serait possible d'isoler certaines règles locales régissant celui-ci étant fixés certains paramètres exogène capturant justement les relations avec d'autres sous-systèmes. Cette question porte à la fois sur l'échelle d'espace, de temps, mais aussi sur les éléments concernés. Reprenons un exemple concret de terrain déjà évoqué en Chapitre~\ref{ch:thematic}








%----------------------------------------------------------------------------------------









