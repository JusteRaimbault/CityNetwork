


%----------------------------------------------------------------------------------------

\newpage


\section*{Chapter Conclusion}{Conclusion du Chapitre}

% A clarification of scales for models, through empirical analysis of case studies. The micro-scale has been shown to have chaotic properties in the case of traffic, by \cite{raimbault2017investigating} in which empirically equilibrium assumptions were tested. At an intermediate scale, using micro-data for gas stations and socio-economic data for counties, \cite{raimbault2017cost} shows that endogenous scales of processes linking transportation and socio-economic properties can be extracted through Geographically Weighted Regressions, and these correspond roughly to a mesoscopic scale (intra-state) and a macroscopic scale (inter-state). A third case study confirms the relevance of working at least at such a level of aggregation, by showing a significant impact of recently planned transportation projects on population and Real Estate transactions for the Greater Paris Region.


%----------------------------------------------------------------------------------------


Cette collection d'études empiriques nous permet à la fois d'illustrer par des cas concrets nos considérations générales sur les réseaux et territoires, mais aussi de clarifier les échelles et ontologies qu'il nous est pertinent d'utiliser. L'échelle microscopique dans le temps et l'espace, pour les objets du traffic routier ici, présente des dynamiques 









\stars

