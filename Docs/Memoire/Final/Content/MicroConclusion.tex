


%----------------------------------------------------------------------------------------

\newpage


\section*{Chapter Conclusion}{Conclusion du Chapitre}



%----------------------------------------------------------------------------------------


Ces études empiriques nous permettent à la fois d'ouvrir par des cas concrets nos études de la co-évolution entre réseaux et territoires. Elle permettent également de confirmer les échelles et ontologies qu'il est pertinent d'utiliser.

Comme développé par~\ref{sec:transportationequilibrium}, l'échelle microscopique dans le temps et l'espace, pour les objets du traffic routier ici, présente des dynamiques chaotiques, rendant peu réaliste l'intégration de cette échelle dans des modèles qui rendraient comptes d'interactions à de plus grandes échelles. Si cet aspect est pris en compte, c'est généralement sous la forme de congestion, qui est agrégée à une échelle supérieure et pour laquelle soit les conséquences des propriétés chaotiques ont été lissées (ce qui peut être un problème pour les modèles d'équilibre), soit elles sont calibrées empiriquement et l'échelle inférieure n'a donc pas d'ontologie dans le modèle. Ce parti a en effet été pris dans nos modèles impliquant un transport routier.

Ensuite dans~\ref{sec:energyprice}, toujours concernant le réseau de transport routier, mais selon le point de vue d'un ancrage nodal dans les territoires par les stations essence, en relation avec diverses caractéristiques socio-économiques de ces territoires, nous démontrons d'une part l'existence d'échelles endogènes, correspondant à l'échelle mesoscopique et l'échelle macroscopique, et d'autre part la complexité des processus d'interaction mis en jeu de par leur non-stationnarité déjà démontrée en~\ref{sec:staticcorrelations} mais aussi par la superposition d'effets territoriaux locaux à des effets liés à la gouvernance.


Nous confirmons ainsi les choix de modélisation de séparer les échelles, les modèles macroscopiques visant à capturer la non-stationnarité en regardant la dynamique à un niveau supérieur en étudiant des variables simples, les modèles mesoscopiques visant à traduire les processus de morphogenèse locaux. Ceux-ci seront introduits dans le chapitre suivant. Par ailleurs, les processus de gouvernance mis en évidence ici ont fait l'objet d'une attention particulière dans la modélisation proposée en~\ref{sec:lutecia}.








\stars


% La dernière section~\ref{sec:grandparisrealestate} permet de conforter ces conclusions de par l'existence d'effet causaux significatif à une échelle mesoscopique dans le temps et dans l'espace.
% L'existence d'effets causaux nous confortent dans la recherche de régimes de causalité dans les modèles de co-évolution, comme introduits en~\ref{sec:causalityregimes}, ce qui sera fait en chapitre~\ref{ch:mesocoevolution}.


