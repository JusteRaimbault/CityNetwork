

%\chapter*{Opening}{Ouvertures}

\chapter*{Ouvertures}

\label{ch:opening}

%----------------------------------------------------------------------------------------

\newpage

\section*{Thematic and General Perspectives}{Perspectives Thématiques et Générales}



%


\subsection*{Specific Developments}{Développement Spécifiques}

% list here ideas that were not evoked in particular sections, but worth mentioning as each could easily bring significant knowledge.


Le mode de communication scientifique actuel est loin d'être optimal 
% cit peer-review broken ; paper format ?
et les initiatives se multiplient pour proposer des modèles alternatifs : la revue post-publication en est une, l'utilisation de systèmes de contrôle de version et de dépôts publics une autre, ou la publication éclair de pistes de recherche (Journal of Brief Ideas).
 % : Journal of Design and Science ; Journal mec Paris Sud (CEA ?)
Les descriptions courtes de pistes de recherche sont souvent reléguées à la discussion ou la conclusion des articles, qui s'écrivent de manière conventionnelle, souvent avec un biais pour justifier a posteriori l'intérêt de \emph{sa nouvelle méthode} qu'il faut malheureusement vendre. On fait alors des plans sur la comète, propose des développements ayant peu de rapport, ou des domaines d'application \emph{qui auront un impact} (lire qui sont à la mode ou qui reçoivent le plus de financements à la période de l'écriture). % note : lien avec science anarchiste ?
Ce manuscrit tombe bien évidemment partiellement sous ces critiques, et encore plus les articles qui lui sont associés.


Nous proposons dans cette section un exercice pas forcément conventionnel : proposer des idées et développements possibles, en s'efforçant de concrétiser les questions de recherche et/ou points techniques autant que possible, afin que ceux-ci ne s'apparentent pas à une bouteille à la mer.



\subsection*{Quantitative Epistemology}{Epistémologie Quantitative}

% - full-text mining ?
% - integrated platform -> mention here CybergeoNetworks ?
%



\subsection*{Multi-scale Models}{Modèles Multi-scalaires}

% coupling between scales.




\subsection*{Towards Operational Models}{Vers des Modèles Opérationnels}

% Concrete operationalization of models : is it desirable ? what needed to reach it ?




%%%%%%%%%%%%%%%%%
% Other possible developments


%Other targeted projects such as the exploration of an hybrid macro-economic/accessibility-based model to explore transportation companies line implementation strategies are still at the state of ideas and are not described here.



% section : heterogenous bike-sharing
% a priori will not do this -

%\section{Shared Transportation System}{Système de Transport en Partage}





%\subsection{Towards simple models of network morphogenesis}{Vers des modèles simples de morphogenèse de réseau}


%An interdisciplinary project that was just launched with a Physicist \noun{Lagesse}, an Architect \noun{Hachi} and a Computer Scientist \noun{Dugue} aims at finding consistent models of urban street network morphogenesis, regarding urban design particularities, geographical rules and complex network indicators feedbacks. Models of network morphogenesis were already discuss here and the aim of this project is to gain insight from the interdisciplinary vision to explore the potentiality of such models. In the frame of our thesis, it is logically situated within the morphogenesis theoretical part and network growth modeling heuristics.




% -- from old empirical intro --




%One does not simply \emph{try} to model something. On that point personal experience confirms indeed that point, as I remember as an early Master student giving in to the call of incautious agent-based modeling, \comment{(Florent) pas forcément approprié dans une thèse}
% naively thinking that integrated models of any aspect of an urban system could be constructed, producing numerous NetLogo code lines to build a gaz factory with unfounded internal processes, an extremely poor external validation and no internal validation. This was a try and therefore a step towards the dark side of models bricolage. The construction of a computational model of simulation is a rigorous exercise that one can not improvise, as much as statistical modeling. Recent progresses in the field~\cite{banos2013pour} help to that purpose, and modular model construction and validation is one tool useful to avoid becoming lost in shady places.







%----------------------------------------------------------------------------------------

\newpage


%%%%%%%%%%%%%%%%%%%%%%%%
\section*{Towards a Research Program}{Vers un Programme de Recherche}

\label{sec:researchprogram}


\subsection*{For an Alternative Integrated Geography}{Pour une Géographie Intégrée Alternative}

% develop here position for a renewal of TQG : additional three dimensions in the knowledge framework ; position at the core of fundamental CS - cannot ignore fundamental questions.


% sur le quanti-quali (from empirical intro)
%As this quote suggests, a purely quantitative view of the world makes no sense without qualitative counterbalancing. More precisely, we argue that the \textit{clich{\'e}} of an opposition between quantitative and qualitative analysis is an illusion. No distinct boundary exists between both. We propose to call quantitative any process involving computation by a Turing machine, whereas the qualitative will be for us the modeling design process and its interpretations. \comment{(Florent) je ne sais pas si je rangerais l'interprétation dans le qualitatif ; ok pour dire (même si connait rien en machine de Turing) que certaines observations via ``Turing'' peuvent s'appeler quantitatives. mais dans un cas comme dans l'autre, ensuite, il faut interpreter}
% Therefore both are necessarily closely interlaced in any of our approaches. In particular concerning the construction and the validation or refutation of our theory, empirical analysis on real case studies, implying the extraction and qualification of stylized facts, follows that schema.




\comment{sur l'evidence-based : même le subjectif est objectif en un sens ? question d'honneteté et d'intégrité intellectuelle - lié nature connaissance, à developper. arreter les arnaques quel que soit le type de méthode, rigueur et reproducibilité à mettre en place.}


\bpar{
}{
Comme déjà souligné en citant \noun{Rey}, les bouleversements techniques et méthodologiques qu'une discipline peut subir sont souvent accompagnés de profondes mutations épistémologiques, voire de la nature même de la discipline. Il est impossible de juger si l'état actuel des connaissances est transitoire, et s'il l'est quelle est le régime stable qui terminerait la transition s'il en existe un. La spéculation est le seul moyen de lever partiellement le voile, sachant que celle-ci sera nécessairement auto-réalisatrice : proposer des visions ou des programmes de recherche oriente les moyens et questions. L' incomplétude théorique en physique, lorsqu'il s'agit par exemple de lier relativité générale et physique quantique, c'est à dire le microscopique stochastique au macroscopique déterministe, orientent les visions du futur de la discipline qui elle-même conditionnent les actions concrètes qui dans ce domaine sont indispensables (financement du CERN ou de l'interféromètre d'ondes gravitationnelles spatial LISA). En géographie, même si les investissements techniques sont incomparables, ceux-ci existent (accès aux moyens de calcul, financement de laboratoires intégrés, etc.) et sont déterminés également par les perspectives pour la discipline. Nous proposons ici une vision et un manifeste d'une nouvelle géographie, qui est déjà en train de se faire et dont les bases sont solidement construites petit à petit. L'aventure de l'ERC Geodivercity en est l'allégorie, d'autant plus qu'elle a confirmé la plupart des directions professées par \noun{Banos}~\cite{banos2017knowledge}. L'intégration de la théorie, de l'empirique, de la modélisation, mais aussi de la technique et de la méthode, n'a jamais été aussi creusée et renforcée que dans les divers développements du projet. Sans l'accès à la grille de calcul et aux nouveaux algorithmes d'exploration permis par OpenMole, les connaissances tirées du modèle SimpopLocal auraient été moindres, mais les développements techniques ont aussi été conduits par la demande thématique.
}


\bpar{
}{
Nous proposons un cadre de connaissances pour les études ayant une composante quantitative, ou plus précisément se posant dans la lignée de la Géographie Théorique et Quantitative (TQG). Ce cadre tente de répondre aux contraintes suivantes : (i) transcender les frontières artificielles entre quantitatif et qualitatif ; (ii) ne pas favoriser de composante particulière parmi les moyens de production de connaissance (aussi divers que l'ensemble des méthodes qualitatives et quantitatives classiques, les méthodes de modélisation, les approches théoriques, les données, les outils), mais bien le développement conjoint de chaque composante. Nous étendons le cadre de connaissances de~\cite{livet2010ontology}, qui consacre le triptyque des domaines empiriques, conceptuels et de la modélisation, en y ajoutant les domaines à part entière que sont les méthodes, les outils (qu'on peut voir comme des proto-méthodes) et les données. Les interactions entre chaque domaine sont détaillées, comme par exemple le passage des méthodes vers les outils qui consiste en l'implémentation, ou le passage de l'empirique aux méthodes comme prospection méthodologique. Toute démarche de production de connaissance, vue comme une \emph{perspective} au sens de~\cite{giere2010scientific}, est une combinaison complexe des six domaines, les fronts de connaissance dans chacun étant en co-évolution. Nous nommons notre cadre de connaissance \emph{Géographie Intégrée}, pour souligner à la fois l'intégration des différents domaines mais aussi des connaissances qualitatives et quantitatives, puisque les deux se fondent dans chacun des domaines.
}

% other illustration : neural networks-deep learning-cuda etc : coevol of methodo, tools, etc (includes Hardware -> as tools or should be a different domain.. ?)
% the CybergeoNetworks is also a good illustration.





%%%%%%%%%%%%%%%%%%%%%%%%
\subsection*{Research Axis}{Axes de Recherche}


\comment{lister les principaux contributeurs etc. ; quoi est compatible avec quoi quest ce quon pourrait coupler etc ; faire analyse epistemo quanti.}

% note : separate "meta"-axis (type epistemo, sci practice etc)
%  and subaxises : fundations of complex systems

% For the application of suited methods : ex not use hierarchical clustering for time-series, not use linear models when not suited, fit well a power law -> on that, try to apply golden standard (Crutchfield) to existing works (cf thèse Olivier) and look if conclusions hold

% link dynamical systems/ABM
% link geosim/spatial stat/economics


\comment{add somewhere something on the link ``more systematic evidence-based''-politics in science - less dogmatism. or what place for evidence-based research in social science ? linked with quanti-quali : BEYOND classical separations, evidence-based and complex systems allow integration, socially responsible, but evidence-based and systematic..}

\comment{in link ``Complexity, Complexities, and Complex Knowledges'', importance of Nature of Complexity ?}

\comment[JR]{evoquer ouverture des cours, formation interdisciplinaire etc. : pas ici, plutot en ouverture finale ?}

\cite{batty2018artificial} artificial intelligence and smart cities


\paragraph{Non-stationarity, non-ergodicity and path-dependancy}{Non-stationnarité, non-ergodicité et dépendance au chemin}




\paragraph{Coupling models and approaches}{Couplage des modèles et approches}

% questions ouvertes sur le couplage, enjeux, def possibles, etc.

% (from weak coupling in CorrelatedSyntheticData)
% these notion of weak / strong coupling are not enough developed or reference-based. ---> find it in literature ? not sure exists like that. --> integarte it in theoretical paper ? or separate working paper.


\comment{different approaches to coupling / coupling to a certain degree using Kolmogorov etc : specific section or insert here ?}


problem du docking/bemchmarking \cite{axtell1996aligning} pistes system of cities et network morphogenesis.





\paragraph{Empowering Models of Simulation with validation and assessment tools}{Construire des outils de validation pour les modèles de simulation}

% -> robustness
% -> empirical AIC

% Note : sur le seed, on est un peu evasifs. etre plus explicite ? (cf Clem spaceMatters)



\paragraph{Experimental and Quantitative epistemology for an effective integration}{Epistémologie quantitative et expérimentale pour une intégration effective}

Le mantra du mariage entre qualitatif et quantitatif est asséné mécaniquement par de nombreux auteurs, mais lorsqu'il s'agit de mise en application, on peut se permettre de soupçonner dans le meilleur des cas une naïveté, dans le pire des cas une hypocrisie. Quel sens à faire semblant de faire des analyses quantitatives en tartinant des pages de régression linéaires dont le $R^2$ ne dépasse pas 0.1 ?
 % find thèse which Renaud was talking about
% Quel sens à simuler à grande échelle des Gaussiennes pour en calculer la moyenne ?\footnote{au moment de l'écriture, l'application étrange était toujours en ligne à \texttt{http://shiny.parisgeo.cnrs.fr/gibratsim/}, onglet simulation, malgré des signalements répétés} % maybe not do bad pub, insist on other noce aspects of the application ? ; surtout ne pas citer cette merde.
 Quel sens à faire semblant de détenir une connaissance qualitative fine pour justifier la mise en place de modèles relevant de l'usine à gaz technocratique ?\footnote{cette remarque est partiellement une auto-critique, puisqu'il faut rappeler le caractère très peu qualitatif de notre travail}

% needs to do experiments to apply knowledge framework.
%  -> exemples : - MigrationDynamics
%                - SimpopSan
%



\paragraph{A Truly Open Science}{Pour une science totalement ouverte}

\comment[JR]{\cite{fecher2014open} five schools in OpenScience : different dimensions or approaches}


\comment{brosser ici directions vers lesquelles travailler ; intégrer faits dans positionements}

% pb of code sharing

La transparence et mise en disponibilité des données brutes ou au moins pré-traitées, et du code informatique produisant les sorties de simulation ou les figures, semble être plutôt l'exception que la règle en géographie. Comme l'assène \noun{Banos} qui y dédie un de ses commandements, ``le modélisateur n'est pas le gardien de la vérité prouvée'', et comme rappelé en chapitre~\ref{ch:methodology}, une reproductibilité parfaite des résultats est nécessaire pour une reconnaissance d'une quelconque valeur par la communauté scientifique, comme une théorie qui ne fournit pas de possibilité de falsification ne peut être considérée comme scientifique comme l'a introduit \noun{Popper}. Des experiences de revue pour \emph{Cybergeo} ont confirmé à l'unanimité ce problème fondamental. Rappelons que la revue \emph{PNAS} exige les données brutes et tableau produisant toute figure, pour prévenir tout biais de visualisation qu'il soit volontaire (ce qui est rédhibitoire et conduit à un signalement) ou non.


% for new modes of publication / communication.

Les observateurs soulevant le caractère détraqué du mode actuel de publication scientifique sont nombreux. Un papier n'est pas un format compréhensible ni vraiment reproductible, et pousse au biais. Comme me le rappelait un ami qui s'est spécialisé de manière admirable dans l'acceptation de papiers extrêmement techniques par des \emph{top-journals} économiques, écrire de façon à être accepté est ``un jeu'' dont les règles sont subtiles et qu'il faut maitriser pour faire carrière. Selon notre positionnement, un tel mode de communication est contraire à l'honnêteté et l'intégrité intellectuelle nécessaires à une science éthique et ouverte. De la même façon que nous soutenons qu'une présentation linéaire d'un travail de thèse est trop fortement réducteur

% explain why linear not good, in networks, various reading grids, etc. propose various reading lines as appendix ?

% self journal of Science : \cite{bon2017novel}














