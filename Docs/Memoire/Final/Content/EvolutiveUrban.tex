


% Chapter 

%\chapter{Evolutive Urban Theory}{Théorie Evolutive Urbaine} % Chapter title
\chapter[Théorie Evolutive Urbaine]{Co-Evolution : une Entrée par la Théorie Evolutive Urbaine}

\label{ch:evolutiveurban} % For referencing the chapter elsewhere, use \autoref{ch:name} 

%----------------------------------------------------------------------------------------


%\headercit{}{}{}

%\bigskip


Si les particularités de chaque cas à l'échelle microscopique sont avérées\comment[FL]{a eviter : cest evident} pour les relations entre territoires et réseaux comme nous l'avons illustré en Chapitre~\ref{ch:thematic}\comment[FL]{faire deux phrases}, de quelle manière ces processus s'agrègent-ils\comment[FL]{pourquoi ce verbe ?} pour faire émerger des caractéristiques de ces relations à d'autres échelles comme l'échelle mesoscopique ou macroscopique, et plus particulièrement les particularités sont-elles toujours la règle\comment[FL]{quel apport ?} ou est-il possible d'extraire certaines régularités, qu'on qualifierait alors de structurelles, et nous permettraient une certaine connaissance des processus impliqués dans les systèmes urbains. Il s'agit de la question fondamentale de \emph{l'universalité des processus}\comment[FL]{il me semblerait plus operant de montrer des processus en particulier pour discuter de leur universalite} dans les systèmes urbains. Une autre caractéristique fondamentale des interactions est l'absence d'équilibre lié à leur complexité\comment[FL]{c'est un postulat fort}[(JR) naturel pour une approche par la complexite] : on peut également se demander s'il existe des échelles spatiales et temporelles sur lesquelles des équilibres seraient raisonnables\comment[FL]{sens ?}, ce qui revient à se poser la question des \emph{échelles de stationnarité des processus}. Ces deux questions ouvertes sont au centre des préoccupations de la \emph{Théorie Evolutive Urbaine}, qui vise à identifier les régularités dans les systèmes de villes\comment[FL]{des processus (ex dynamique demo)} tout en mettant l'emphase sur la particularité de leur éléments et les bifurcations qui en découlent \comment[FL]{mal dit}(voir~\ref{sec:knowledgeframework}). Il est alors légitime d'explorer cette première entrée et les solutions qu'elle propose, par des analyses empiriques et de modélisation, ce qui est l'objet de ce chapitre. Nous étudions d'abord à l'échelle mesoscopique les propriétés de non-stationnarité spatiale entre des manifestations simples des caractéristiques des territoires et des réseaux, capturées dans des indicateurs morphologiques pour chacun, par l'étude des correlations spatiales entre ces indicateurs. Nous introduisons ensuite la dimension temporelle en étudiant la notion de causalité spatio-temporelle dans la section~\ref{sec:causalityregimes}, qui est essentielle d'une part d'un point de vue méthodologique par l'introduction d'une méthode originale permettant dans certains cas de mieux cerner les influences respectives entre réseaux et territoires, mais également d'un point de vue thématique concernant l'existence avérée d'une co-évolution. Les multiples régimes de causalité mis en évidence pour un modèle simple de morphogenèse couplant fortement croissance du réseau et densité témoignent de causalités circulaires qui sont bien des marques d'une co-évolution. Dans le cas d'une non-stationnarité, qui a été mise en valeur par~\ref{sec:staticcorrelations}, ces régimes peuvent alors évoluer dans le temps et l'espace, impliquant alors une co-évolution sur le temps long. L'application au cas du réseau ferroviaire en Afrique du Sud montre que cette multiplicité des régimes existe bien pour des données réelles. Nous explorons enfin dans une dernière section~\ref{sec:interactiongibrat} les possibilité offerte par les modèles d'interaction issus de la théorie évolutive, à une grande échelle spatiale et temporelle, ce qui permet de démontrer l'existence d'effets de réseau de manière indirecte, sans même introduire d'aspects de co-évolution dans un premier temps. Ainsi, nous posons les premières fondations, sur différents aspects des relations entre réseaux et territoires, qui peuvent paraître lointain en lecture rapide, mais qui sont bien reliés en filigrane par les questions fondamentales à laquelle la Théorie Evolutive tente de répondre.




\stars


\textit{Ce chapitre est composé de divers travaux. La première section reprend une partie traduite de~\cite{} pour l'analyse morphologique, puis les résultats présentés par~\cite{raimbault2016cautious} pour l'analyse des correlations ; la deuxième section correspond à la majorité de~\cite{} pour la formulation théorique et l'illustration sur données synthétiques, puis présente les résultats de~\cite{} pour l'application. Enfin la dernière section est une traduction de~\cite{}.}







%----------------------------------------------------------------------------------------


% articulation with theoretical questions
% articulation with modeling

%\comment{(Florent) Entre cette phrase (next) et le détail projet par projet, il serait bien que tu te positionnes sur le cadre analytique général, justement comme tu l'introduis en parlant des objets et des échelles}

%We propose in this chapter various empirical analysis on different objects at different scales. A first section begins the examination of static spatial correlations between morphological measures of population density and road network measures on Europe at a 500m resolution. Applying last section of the methodological chapter should provide information on typical spatial scales of interaction between these indicators \comment{(Florent) à quel niveau se situe l'indicateur ?}
% of territory and network and on dynamical correlations between these. These computation furthermore provide empirical measures on which one model will be calibrated. We then describe a roadmap for statistical analysis on dynamical data of interactions for Bassin Parisien in the last fifty years.








