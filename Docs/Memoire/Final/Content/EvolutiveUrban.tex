


% Chapter 

%\chapter{Evolutive Urban Theory}{Théorie Evolutive Urbaine} % Chapter title
\chapter[Théorie Evolutive Urbaine]{Co-Evolution : une Entrée par la Théorie Evolutive Urbaine}

\label{ch:evolutiveurban} % For referencing the chapter elsewhere, use \autoref{ch:name} 

%----------------------------------------------------------------------------------------


%\headercit{}{}{}


\bigskip




La section~\ref{sec:causalityregimes} est essentielle d'une part d'un point de vue méthodologique par l'introduction d'une méthode originale permettant dans certains cas de mieux cerner les influences respectives entre réseaux et territoires, mais également d'un point de vue thématique concernant l'existence avérée d'une co-évolution : les multiples régimes de causalité mis en évidence pour un modèle simple de morphogenèse couplant fortement croissance du réseau et densité témoignent de causalités circulaires qui sont bien des marques d'une co-évolution. Dans le cas d'une non-stationnarité, qui a été mise en valeur par~\ref{sec:staticcorrelations}, ces régimes peuvent alors évoluer dans le temps et l'espace, impliquant alors une co-évolution sur le temps long. L'application au cas du réseau ferroviaire en Afrique du Sud montre que cette multiplicité des régimes existe bien pour des données réelles. 


\todo{quel niveau de description de la théorie évolutive ici ? insérer l'étude méthodes mixtes en dernier opening : illustration/construction du cadre de connaissances.}


%As this quote suggests, a purely quantitative view of the world makes no sense without qualitative counterbalancing. More precisely, we argue that the \textit{clich{\'e}} of an opposition between quantitative and qualitative analysis is an illusion. No distinct boundary exists between both. We propose to call quantitative any process involving computation by a Turing machine, whereas the qualitative will be for us the modeling design process and its interpretations. \comment{(Florent) je ne sais pas si je rangerais l'interprétation dans le qualitatif ; ok pour dire (même si connait rien en machine de Turing) que certaines observations via ``Turing'' peuvent s'appeler quantitatives. mais dans un cas comme dans l'autre, ensuite, il faut interpreter}
% Therefore both are necessarily closely interlaced in any of our approaches. In particular concerning the construction and the validation or refutation of our theory, empirical analysis on real case studies, implying the extraction and qualification of stylized facts, follows that schema.

% articulation with theoretical questions
% articulation with modeling

%\comment{(Florent) Entre cette phrase (next) et le détail projet par projet, il serait bien que tu te positionnes sur le cadre analytique général, justement comme tu l'introduis en parlant des objets et des échelles}

%We propose in this chapter various empirical analysis on different objects at different scales. A first section begins the examination of static spatial correlations between morphological measures of population density and road network measures on Europe at a 500m resolution. Applying last section of the methodological chapter should provide information on typical spatial scales of interaction between these indicators \comment{(Florent) à quel niveau se situe l'indicateur ?}
% of territory and network and on dynamical correlations between these. These computation furthermore provide empirical measures on which one model will be calibrated. We then describe a roadmap for statistical analysis on dynamical data of interactions for Bassin Parisien in the last fifty years.








