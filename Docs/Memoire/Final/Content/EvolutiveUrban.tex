


% Chapter 

%\chapter{Evolutive Urban Theory}{Théorie Evolutive Urbaine} % Chapter title
\chapter[Théorie Evolutive Urbaine]{Co-Evolution : une Entrée par la Théorie Evolutive Urbaine}

\label{ch:evolutiveurban} % For referencing the chapter elsewhere, use \autoref{ch:name} 

%----------------------------------------------------------------------------------------


%\headercit{}{}{}

%\bigskip


\comment[FL]{faire des schemas simples $\rightarrow$ fil de lecture}

\comment[FL]{concepts ?}

\comment[FL]{modèles a presenter clairement (I/O)}


On a illustré en Chapitre ~\ref{ch:thematic} les particularités de chaque cas\comment[FL]{flou : redonner le contexte} à l'échelle microscopique pour les interactions entre territoires et réseaux de transport. De quelle manière ces processus font-ils émerger des caractéristiques de ces relations à d'autres échelles comme l'échelle mesoscopique ou macroscopique ? Dans quelle mesure est-il possible d'extraire certaines régularités, qu'on qualifierait alors de structurelles\comment[FL]{pourquoi avoir besoin de fixer ce vocabulaire ?}, et qui permettraient une certaine connaissance générale des processus impliqués dans les systèmes urbains. Il s'agit de la question fondamentale de \emph{l'universalité des processus} \comment[FL]{il me semblerait plus operant de montrer des processus en particulier pour discuter de leur universalite} 
dans les systèmes urbains.\comment[FL]{pas certain que ce soit utile ici} Une autre caractéristique fondamentale des interactions est la question de l'existence d'équilibres\comment[FL]{ce n'est pas une caracteristique des interactions mais une caracteristique (propriete) emergente} : on peut également se demander s'il existe des échelles spatiales et temporelles sur lesquelles des équilibres peuvent exister, ce qui revient à se poser la question des \emph{échelles de stationnarité des processus}.\comment[FL]{le probleme avec cette liste est que le lecteur ne peut pas comprendre d'ou proviennent tes questionnements} Ces deux questions ouvertes sont au centre des préoccupations de la \emph{Théorie Evolutive Urbaine}, qui vise à identifier les faits stylisés réguliers dans les systèmes de villes tout en mettant l'accent sur la particularité de leurs éléments et les bifurcations\comment[FL]{def ?} associées (voir~\ref{sec:knowledgeframework}). Il est alors légitime\comment[FL]{formulation maladroite : c'est subjectif} d'explorer cette première entrée et les solutions qu'elle propose, par des analyses empiriques et de modélisation, ce qui est l'objet de ce chapitre. Nous étudions d'abord à l'échelle mesoscopique les propriétés de non-stationnarité spatiale entre des manifestations simples\comment[FL]{cette justification est speciale : en substance je le fais parce que je le fais} des caractéristiques des territoires et des réseaux, capturées dans des indicateurs morphologiques pour chacun, par l'étude des correlations spatiales entre ces indicateurs. Nous introduisons ensuite l'aspect dynamique en étudiant la notion de causalité spatio-temporelle dans la section~\ref{sec:causalityregimes}. Celle-ci est essentielle d'une part d'un point de vue méthodologique par l'introduction d'une méthode originale permettant dans certains cas de mieux cerner les influences respectives entre réseaux et territoires, mais également d'un point de vue thématique concernant l'existence avérée d'une co-évolution. Les multiples \emph{régimes de causalité}\comment[FL]{def} mis en évidence pour un modèle simple de morphogenèse couplant fortement croissance du réseau et densité témoignent de causalités circulaires qui sont bien des marques d'une co-évolution.\comment[FL]{je croyais que le morphogenese c'etait chapitre 5 ? tu vas perdre ton lecteur.} Dans le cas d'une non-stationnarité, qui a été mise en valeur par~\ref{sec:staticcorrelations},\comment[FL]{on est pas encore au 4.1 $\rightarrow$ pas besoin de ce niveau de detail} ces régimes peuvent alors évoluer dans le temps et l'espace, impliquant alors une co-évolution sur le temps long. L'application au cas de la croissance du réseau ferroviaire et des villes en Afrique du Sud montre que cette multiplicité des régimes existe bien sur une situation réelle.\comment[FL]{md} Nous explorons enfin dans une dernière section~\ref{sec:interactiongibrat} les possibilité offerte par les modèles d'interaction issus de la théorie évolutive, à une grande échelle spatiale et temporelle, ce qui permet de démontrer l'existence d'effets de réseau de manière indirecte, sans même introduire d'aspects de co-évolution dans un premier temps. Ainsi, nous façonnons les premières briques, sur différents aspects des interactions et de la coévolution entre réseaux et territoires, qui peuvent paraître lointain en lecture rapide, mais qui sont bien reliés en filigrane par les questions fondamentales à laquelle la Théorie Evolutive tente de répondre.




\stars


\textit{Ce chapitre est composé de divers travaux. La première section reprend une partie traduite de~\cite{raimbault2017calibration} pour l'analyse morphologique, puis les résultats présentés par~\cite{raimbault2016cautious} pour l'analyse des correlations ; la deuxième section correspond à la majorité de~\cite{raimbault2017identification} pour la formulation théorique et l'illustration sur données synthétiques, puis présente les résultats de~\cite{raimbault:halshs-01584914} pour l'application. Enfin la dernière section est une traduction de~\cite{}.}

\comment[FL]{suppr}[(JR) non]



%----------------------------------------------------------------------------------------


\newpage


\subsubsection*{Evolutive Urban Theory}{Théorie Evolutive Urbaine}



Nous avons déjà évoqué divers aspects de la Théorie Evolutive des villes, en relation à la complexité en géographie, puis à certains modèles de systèmes urbains qu'elle a produit. Une synthèse est ici nécessaire pour poser précisément le cadre dans lequel nos développements s'inscriront. Cette théorie a été introduite initialement dans~\cite{pumain1997pour} qui argumente pour une vision dynamique des systèmes de ville, au sein desquels l'auto-organisation est essentielle. Les villes sont des entités spatiales évolutives interdépendantes dont les interrelations font émerger le comportement macroscopique à l'échelle du système de villes. Le système de villes est aussi vu comme un réseau de villes, ce qui renforce sa vision en tant que système complexe. Chaque ville est elle-même un système complexe dans l'esprit de~\cite{berry1964cities}, l'aspect multi-scalaire, au sens d'échelles autonomes mais ayant chacune un rôle spécifique dans les dynamiques du système, étant essentiel dans cette théorie, puisque les agents microscopiques véhiculent les processus d'évolution du système à travers des rétroactions complexes entre les échelles. Le positionnement de cette théorie au regard des Sciences des Systèmes Complexes a plus tard été confirmé~\cite{pumain2003approche}.

Il a été montré que la théorie évolutive fournit une interprétation des lois d'échelle qui sont omniprésentes dans les systèmes urbains, qui découleraient de la diffusion des cycles d'innovation entre les villes~\cite{pumain2006evolutionary}, qui ont par ailleurs été mis en évidence de manière empirique pour plusieurs systèmes urbains~\cite{pumain2009innovation}. La notion de résilience d'un système de villes, induit par le caractère adaptatif des ces systèmes complexes, implique que les villes sont les moteurs et les adaptateurs du changement social~\cite{pumain2010theorie}. Enfin, la dépendance au chemin est source de non-ergodicité (voir definition en~\ref{sec:staticcorrelations}) au sein de ces systèmes, rendant les interprétations ``universelles'' des lois d'échelle développées par les physiciens incompatibles avec la théorie évolutive~\cite{pumain2010theorie}.


La Théorie Evolutive des Villes a été élaborée conjointement avec des modèles de systèmes urbains : par exemple le modèle Simpop2 introduit par~\cite{bretagnolle2006theory} est un modèle basé agent qui prend en compte des processus économiques, et simule sur de longues échelles de temps les motifs de croissance urbaine pour l'Europe et les Etats-unis~\cite{doi:10.1177/0042098010377366}. Les accomplissements les plus récents de la théorie évolutive reposent sur les productions du projet ERC GeoDivercity, présentées dans \cite{pumain2017urban}, qui incluent de progrès avancés à la fois techniques (logiciel OpenMole\footnote{http://openmole.org/}~\cite{reuillon2013openmole}), thématiques (connaissance issue des modèles SimpopLocal~\cite{schmitt2014modelisation} et Marius~\cite{cottineau2014evolution}) et méthodologiques (modélisation incrémentale~\cite{cottineau2015incremental}). Pour une analyse épistémologique par méthode mixtes de la théorie évolutive, qui permet de renforcer cet aperçu bibliographique par une de sa genèse, en quelque sorte de sa \emph{forme}, se référer à~\ref{sec:knowledgeframework} qui l'utilise comme cas d'étude pour construire un cadre de connaissances.

\comment[JR]{mettre plus en valeur les interviews}


\subsubsection*{Implications}{Implications}


Reprenons la substantifique moelle de la Théorie Evolutive, comme synthétisé par \noun{Denise Pumain} elle-même (entretien en~\ref{app:sec:interviews}) : ``\textit{[Il s'agit d']une Théorie Géographique ayant pour ambition de rassembler la plupart des faits stylisés connus sur les villes et leur organisation dans les territoires, dans une perspective hors-équilibre et non statique, en les suivant sur de longues périodes de temps et mettant une emphase sur les facteurs structurants et les bifurcations.}''












