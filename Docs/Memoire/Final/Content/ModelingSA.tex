



%----------------------------------------------------------------------------------------

\newpage

\section{Modeling Interactions}{Modéliser les Interactions}
\label{sec:modelingsa}


%----------------------------------------------------------------------------------------




\subsection{Modeling in Quantitative Geography}{Modélisation en Géographie Quantitative}


\subsubsection{History}{Histoire}

\bpar{
Modeling in Theoretical and Quantitative Geography (TQG), and more generally in Social Science, has a long history on which we can not go further than a general context. \noun{Cuyala} does in~\cite{cuyala2014analyse} an analysis of the spatio-temporal development of French speaking TQG movement and underlines the emergence of the discipline as the combination between quantitative analysis (e.g. spatial analysis or modeling and simulation practices) and theoretical constructions, an integration of both allowing the construction of theories from empirical stylized facts that yield theoretical hypothesis to be tested on empirical data. These approach were born under the influence of the \emph{new geography} in Anglo-saxon countries and Sweden.
}{
La modélisation joue en Géographie Théorique et Quantitative (TQG) un rôle fondamental. \cite{cuyala2014analyse} procède à une analyse spatio-temporelle du mouvement de la Géographie Théorique et Quantitative en langue française et souligne l'émergence de la discipline comme une combinaison d'analyses quantitatives (e.g. analyse spatiale et pratiques de modélisation et de simulation) et de construction théoriques. Cette dynamique est datée à la fin des années 70, et est intimement liée à l'utilisation et l'appropriation des outils mathématiques~\cite{pumain2002role}. L'intégration de ces deux composantes permet la construction de théories à partir de faits stylisés empiriques, qui produisent à leur tour des hypothèses théoriques pouvant être testées sur les données empiriques. Cette approche est née sous l'influence de la \emph{New Geography} dans les pays Anglo-saxons et en Suède. Concernant la modélisation urbaine en elle-même, d'autre champs que la géographie ont proposé des modèles de simulation à peu près à la même période. Par exemple, le modèle de \noun{Lowry}, développé par~\cite{lowry1964model} dans un but appliqué immédiat à la région métropolitaine de Pittsburg, suppose un système d'équations pour la localisation des actifs et des emplois dans différente zones. Des modèles relativement similaires sont toujours utilisés aujourd'hui.
}


%\comment[FL]{manque peut etre quelques phrases avec des modeles ``landmark'' anciens (Lowry 1964 par exemple) mais encore utilises, de plus ne pas perdre le fil du champ applicatif auquel tu te destines}

% Lowry : https://www.rand.org/content/dam/rand/pubs/research_memoranda/2006/RM4035.pdf
% Alonso : http://onlinelibrary.wiley.com/doi/10.1111/j.1435-5597.1967.tb01370.x/epdf
% http://www.sciencedirect.com/science/article/pii/S009411909792074X prefatt-diff ?


\subsubsection{Simulation of models and intensive computation}{Simulation de modèle et calcul intensif}


\bpar{
A broad history of the genesis of models of simulation in geography is done by \noun{Rey} in~\cite{rey2015plateforme} with a particular emphasis on the notion of validation of models. The use of computation for simulation of models is anterior to the introduction of paradigms of complexity, coming back to \noun{H{\"a}gerstrand} and \noun{Forrester}, pioneers of spatial economic models inspired by Cybernetics. With the increase of computational possibilities epistemological transformations have also occurred, with the apparition of explicative models as experimental tools. \noun{Rey} compares the dynamism of seventies when computation centers were opened to geographers to the democratization of High Performance Computing (transparent grid computing, see~\cite{schmitt2014half} for an exemple of the possibilities offered in terms of model validation and calibration, decreasing the computational time from 30 years to one week), that is also accompanied by an evolution of modeling practices~\cite{banos2013pour} and techniques~\cite{10.1371/journal.pone.0138212}.
}{
Une histoire étendue de la genèse des modèles de simulation en géographie est faite par \noun{Rey} dans~\cite{rey2015plateforme} avec une attention particulière pour la notion de validation de modèles (nous reviendrons sur la place de ces aspects dans notre travail en~\ref{ch:positioning}). L'utilisation de ressources de calcul pour la simulation de modèles est antérieure à l'introduction des paradigmes de la complexité actuels, remontant par exemple à \noun{Forrester}, informaticien qui a été pionnier des modèles d'économie spatiale inspirés par la cybernétique\footnote{Celle-ci, ainsi que le courant systémique, sont comme nous l'avons déjà développé précurseurs des paradigmes actuels de la complexité.}. Avec l'augmentation des potentialités de calcul, des transformations épistémologiques ont également suivi, avec l'apparition de models explicatifs comme outils expérimentaux. \noun{Rey} compare le dynamisme des années soixante-dix quand les centres de calcul furent ouverts aux géographes à la démocratisation actuelle du Calcul Haute Performance. Aujourd'hui, cette facilité d'accès consiste entre autres à du calcul sur grille dont l'utilisation est rendue transparente, c'est à dire sans besoin de compétences techniques pointues liées au mécanismes de la distribution des calculs.. Ainsi~\cite{schmitt2014half} donnent un exemple des possibilités offertes en termes de calibration et de validation de modèle, réduisant le temps de calcul nécessaire de 30 ans à une semaine - ces techniques jouent un rôle clé pour les résultats que nous obtiendrons par la suite. Cette évolution est également accompagnée par une évolution des pratiques~\cite{banos2013pour} et techniques~\cite{10.1371/journal.pone.0138212} de modélisation.
}


\bpar{
Modeling (in particular computational models of simulation) is seen by many as a fundamental building brick of knowledge : \cite{livet2010} recalls the combination of empirical, conceptual (theoretical) and modeling domains with constructive feedbacks between each. A model can be an exploration tool to test assumptions, an empirical tool to validate a theory against datasets, an explicative tool to reveal causalities (and thus internal processes of a system), a constructive tool to iteratively build a theory with an iterative construction of an associated model. These are example among others : \noun{Varenne} proposes in~\cite{varenne2010simulations} a refined classifications of diverse functions of a model. We will consider modeling as a fundamental instrument of knowledge on processes within complex adaptive systems, as already evoked, and restraining again our question, will focus on \emph{models involving interactions between transportation networks and territories}.
}{
La modélisation, et en particulier les modèles de simulation, est vue par beaucoup comme une brique fondamentale de la connaissance : \cite{livet2010} rappelle la combinaison des domaines empirique, conceptuel (théorique) et de la modélisation, avec des retroactions constructives entre chaque. Un modèle peut être un outil d'exploration pour tester des hypothèses, un outil empirique pour valider une théorie sur des jeux de données, un outil explicatif pour révéler des causalités et ainsi des processus internes au système, un outil constructif pour construire itérativement une théorie conjointement avec celle des modèles associés. Ce sont des exemples de fonctions parmi d'autres : \cite{varenne2010simulations} propose une classification des diverses fonctions d'un modèle. Nous considérons la modélisation comme un instrument fondamental de connaissance des processus au sein d'un système, plus particulièrement dans notre cas au sein d'un système complexe adaptatif. Nous rappelons ainsi que notre question de recherche s'intéressera aux \emph{modèles dont l'ontologie contient une part non négligeable d'interactions réseaux et territoires}.
}




%%%%%%%%%%%%%%%%%%%%%%%%%%%
\subsection{Modeling Territories and Networks}{Modéliser les territoires et réseaux}


\bpar{
Concerning our precise question of interactions between transportation networks and territories, we propose an overview of existing approaches. Following~\cite{bretagnolle2002time}, the ``\textit{thoughts of specialists in planning aimed to give definitions of city systems, since 1830, are closely linked to the historical transformations of communication networks}''. It is not far from an reversed self-realizing prophecy, in the sense that it is already realized before happening. It implies that ontologies and corresponding models addressed by geographers and planners are closely linked to their current historical preoccupations, thus necessarily limited in scope and purpose. In a perspectivist vision of science~\cite{giere2010scientific} such boundaries are the essence of the scientific entreprise, and as we will argue in chapter~\ref{ch:theory} their combination and coupling in the case of models is a source of knowledge.
}{
Développons à présent un aperçu des différentes approches modélisant des interactions entre réseaux de transport et territoires. Remarquons de manière préliminaire une forte contingence des constructions scientifiques sous-jacentes à celles-ci. En effet, selon~\cite{bretagnolle2002time}, ``\textit{les idées des spécialistes de la planification cherchant à donner des définitions des systèmes de ville, depuis 1830, sont étroitement liées aux transformations des réseaux de communication}''. Le contexte historique (et donc socio-économique et technologique) conditionne fortement les théories formulées. %C'est en quelque sorte la prophétie auto-réalisatrice inversée, au sens où elle est déjà réalisée avant d'être formulée\comment[FL]{sens ?}. % -> notion de coevol des connaissance avec les objets - on revient sur la reflexivité - trop glissant pour développer.
 Cela implique que les ontologies et les modèles correspondants proposés par les géographes et les planificateurs sont fortement liés aux préoccupations historiques courantes, ce qui limite nécessairement leur portée théorique et/ou opérationnelle. Au delà de la question de la définition du système qui joue également un rôle central, on comprend bien l'impact que peut avoir cette influence sur la portée des modèles développés. Dans une vision perspectiviste de la science~\cite{giere2010scientific} de telles limites sont l'essence de l'entreprise scientifique, et comme nous démontrerons dans le chapitre~\ref{ch:theory} leur combinaison et couplage dans le cas de modèles est généralement une source de connaissance.
}

% \comment[FL]{pas par essence mais cela peut l'etre effectivement}[(JR) pas d'accord : opérer un couplage de modele implique un couplage des ontologies et necessairement un accroissement des connaissances (si celui-ci est utile c'est un autre problème)]


L'entrée que nous proposons ici pour dresser un aperçu des modèles est complémentaire de celle prise au Chapitre~\ref{ch:thematic}, en regardant par objet principal (c'est à dire les relations Réseau $\rightarrow$ Territoire, Territoire $\rightarrow$ Réseau et Territoire $\leftrightarrow$ Réseau). Nous avons vu que la correspondance à des échelles temporelles et spatiales n'est pas systématique (voir la typologie provisoire à double entrée des processus). Par contre, celle à des domaines particuliers et à des acteurs l'est plus. Cette revue de littérature est donc orientée dans cette seconde direction.

%\comment[FL]{tu arrives trop tot la dessus - les processus cles ne sont pas ammenes suffisamment clairement : travaille par echelle de temps, d'espace, d'acteurs}[ok developper en echo au 1 - vision par les acteurs / vision par les echelles.]


%\subsubsection{Land-Use Transportation Interaction Models}{Modèles LUTI}
\subsubsection{Territories}{Territoires}


Le courant principal s'intéressant à la modélisation de l'influence du réseau de transport sur les territoires se trouve dans le champ de la planification, à des échelles spatiales et temporelles moyennes (les échelles de l'accessibilité métropolitaine que nous avons développés ci-dessus). Des modèles en géographie à d'autres échelles, comme les modèles Simpop déjà évoqués~\cite{pumain2012multi}, ne supposent pas une ontologie particulière pour le réseau de transport, et sont relativement éloignés de l'idée d'interaction. Nous reviendrons plus loin sur des extensions pertinentes pour notre question. Revoyons pour commencer le contexte des études de planification.


\paragraph{LUTI models}{Modèles LUTI}

\bpar{
A subsequent bunch of literature in modeling interaction between networks and territories can be found in the field of planning, with the so-called \emph{Land-use Transportation Interaction Models}. These works are difficult to be precisely bounded as they may be influenced by various disciplines. For example, from the point of view of Urban Economics, propositions for integrated models have existed for a relatively long term~\cite{putman1975urban}.
}{
Ces approches sont désignées de manière générale comme \emph{modèles d'interaction entre usage du sol et transport} (\emph{LUTI}, pour \textit{Land-Use Transport Interaction}). Il est entendu par usage du sol la répartition des activités territoriales, généralement réparties en typologies plus ou moins précises (par exemple logements, industrie, tertiaire, espace naturel). Ces travaux peuvent être difficiles à cerner car liés à différentes disciplines scientifiques. Leur principe général est de modéliser et simuler l'évolution de la distribution spatiale des activités, en prenant le réseau de transport comme contexte et déterminant significatif des localisations. Par exemple, du point de vue de l'Economie Urbaine, les propositions de tels modèles existent depuis un certain temps : \cite{putman1975urban} rappelle le cadre d'économie urbaine où les principales composantes sont les emplois, la démographie et le transport, et passe en revue des modèles économiques de localisation qui s'apparentent au modèle de \noun{Lowry}.
}


\bpar{
Generally these type of models operate at relatively small temporal and spatial scales. \cite{wegener2004land} reviewed state of the art in empirical and modeling studies on interactions between land-use and transportation. It is positioned in economic, planning and sociological theoretical contexts, and is relatively far from our geographical approach aiming to also understand long-time processes. Seventeen models are compared and classified, none of which implements actually network endogenous evolution on the relatively small time scales of simulation. A complementary review done in \cite{chang2006models} broadens the scope with inclusion of more general classes of models, such as spatial interaction models (including traffic assignment and four steps models), operational research planning models (optimal localisations), micro-based random utility models, and urban market models.
}{
\cite{wegener2004land} donne plus récemment un état de l'art des études empiriques et de modélisation sur ce type d'approche des interactions entre usage du sol et transport. Le positionnement théorique est plutôt proche des disciplines de la socio-économie des transports et de la planification (voir les paysages disciplinaires dressés en~\ref{sec:quantepistemo}). \cite{wegener2004land} compare et classifie dix-sept modèles, parmi lesquels aucun n'inclut une évolution endogène du réseau de transport sur les échelles de temps relativement courtes (de l'ordre de la décade) des simulations. On retrouve bien la correspondance avec les échelles typiquement mesoscopiques établies précédemment. Une revue complémentaire est faite par~\cite{chang2006models}, élargissant le contexte avec l'inclusion de classes plus générales de modèles, comme des modèles d'interactions spatiales (parmi lesquels l'attribution du traffic et les modèles à quatre temps), les modèles de planification basés sur la recherche opérationnelle (optimisation des localisations des différentes activités, généralement résidences et emplois), les modèles microscopiques d'utilité aléatoire, et les modèles de marché foncier.
}




\paragraph{Operational models}{Des modèles opérationnels très variés}


\bpar{The variety of possible models has lead to operational comparisons~\cite{paulley1991overview,wegener1991one}.}
{
La variété des modèles existants a conduit à des comparaisons opérationnelles : \cite{paulley1991overview} rendent compte d'un projet comparant différents modèles appliqués à différentes villes. Leurs résultats permettent d'un part de classifier des interventions en fonction de leur impact sur le niveau d'interaction entre transport et usage du sol, et d'autre part de montrer que l'effet des interventions dépend fortement de la taille de la ville et de ses caractéristiques socio-économiques.
}


\bpar{
More recently, the respective advantages of static and dynamic modeling was investigated in~\cite{kryvobokov2013comparison}.
}{
Les ontologies des processus, et notamment sur la question de l'équilibre, sont aussi variées. Les avantages respectifs d'une approche statique (calcul d'un équilibre statique de la localisation des ménages pour une certaine spécification de leur fonctions d'utilité) et d'une approche dynamique (simulation hors équilibre des dynamiques résidentielles) a été étudié par~\cite{kryvobokov2013comparison}, dans un cadre métropolitain sur des échelles de temps de l'ordre de la décade. Les auteurs montrent que les résultats sont globalement comparables et que chaque modèle a son utilité selon la question posée.% Dans tous les cas, ce type de modèle opère généralement à des échelles temporelles et spatiales relativement faibles.
}



\bpar{
These techniques operate also at small scales and consider at most land-use evolution. \cite{iacono2008models} covers a similar scope with a further emphasis on cellular automata models of land-use change and agent-based models. These type of models are still largely developed and used today, as for example \cite{delons:hal-00319087} which is used for Parisian metropolitan region. The short-term range of application and their operational character makes them useful for planning, what is far from our preoccupation to obtain explicative models for geographical processes.
}{
Différents aspects du même système peuvent être traduits par divers modèles, comme le montre par exemple~\cite{wegener1991one}, et le trafic, les dynamiques résidentielles et d'emploi, l'évolution de l'usage du sol en découlant, influencée aussi par un réseau de transport statique, sont généralement pris en compte. \cite{iacono2008models} couvre un horizon similaire avec un développement supplémentaire sur les modèles à automates cellulaires d'évolution d'usage du sol et les modèles à base d'agents. Les modèles LUTI sont toujours largement étudiés et appliqués, comme par exemple \cite{delons:hal-00319087} qui est utilisé pour la région métropolitaine parisienne. La portée temporelle d'application de ces modèles, de l'ordre de la décade, et leur nature opérationnelle les rend utiles pour la planification, ce qui est assez loin de notre souci d'obtenir des modèles explicatifs de processus géographiques. En effet, il est souvent plus pertinent pour un modèle utilisé en planification d'être lisible comme outil d'anticipation, voire de communication, que d'être fidèle aux processus territoriaux au prix d'une abstraction.
}

%\comment[FL]{ce point est tres important et merite (un dvlpmt ?)}



\paragraph{Perspectives for LUTI models}{Perspectives pour les LUTI}

\bpar{}{
\cite{timmermans2003saga} émet des doutes quant à la possibilité de modèles d'interaction réellement intégrés, c'est à dire produisant des motifs de transports endogènes et se détachant d'artefacts comme l'accessibilité dont l'influence du caractère artificiel reste à établir, notamment à cause du manque de données et une difficulté à modéliser les processus de gouvernance et de planification. Il est intéressant de noter que les priorités actuelles de développement des modèles LUTI semblent centrées sur une meilleure intégration des nouvelles technologies et une meilleur intégration avec la planification et les processus de prise de décision, par exemple via des interfaces de visualisation comme le propose~\cite{JTLU611}. Ils ne cherchent pas à s'étendre à des problématiques de dynamiques territoriales incluant le réseau sur de plus longues échelles par exemple, ce qui confirme la portée et la logique d'utilisation et de développement de ce type de modèles.
}

\bpar{}{
Une généralisation de ce type d'approche à une plus grande échelle, comme celle proposée par \cite{russo2012unifying}, consiste au couplage du LUTI à l'échelle mesoscopique à des modèles macroéconomiques à l'échelle macroscopique. Ceux-ci ne considèrent pas l'évolution du réseau de transport de manière explicite mais s'intéressent seulement aux motifs abstraits d'offre et demande. L'économie urbaine a développé des approches spécifiques similaires dans leur démarche : \cite{masso2000} décrit par exemple un modèle intégré couplant développement urbain, relocalisation et équilibre des flux de transports.
}


\bpar{}{
Ainsi, nous pouvons synthétiser ce type d'approche, qu'on pourra désigner par abus de langage \emph{approche LUTI}, par les caractéristiques fondamentales suivantes : (i) Modèles visant à comprendre une évolution du territoire, dans le contexte d'un réseau de transport donné ; (ii) Modèles dans une logique de planification et d'applicabilité, étant souvent impliqués eux-même dans les prises de décision ; et (iii) Modèles à des échelles moyennes, dans l'espace (métropole) et dans le temps (décade).
}




\subsubsection{Network Growth}{Croissance du Réseau}

Passons à présent au paradigme ``opposé'', centré sur l'évolution du réseau. Il peut sembler incongru de considérer un réseau variable en négligeant les variations du territoire, au regard de l'aperçu de certains des mécanismes potentiels d'évolution revus précédemment (rupture de potentiel, auto-renforcements, planification du réseau) qui se produisent à des échelles de temps majoritairement plus longues que les évolutions territoriales. On verra ici qu'il n'y a pas de paradoxe, vu que (i) soit la modélisation s'intéresse à l'évolution des \emph{propriétés du réseau}, à une courte échelle (micro) pour des processus de congestion, de capacité, de tarification, principalement d'un point de vue économique ; (ii) soit les composantes territoriales jouant en effet sur le réseau sont stables au échelles longues considérés (approches des physiciens).


\bpar{
Network growth can be used to design modeling entreprises that aim to endogenously explain growth of transportation networks, generally from a bottom-up point of view, i.e. by exhibiting local rules that would allow to reproduce network growth over long time scales (generally the road network).
}{
La croissance de réseaux est l'objet de démarches de modélisation qui cherchent à expliquer la croissance des réseaux de transport. Ils prennent généralement un point de vue \emph{bottom-up} et endogène, c'est-à-dire cherchant à mettre en évidence des règles locales qui permettraient de reproduire la croissance du réseau sur de longues échelles de temps (souvent le réseau routier). Comme nous allons le voir, il peut s'agir de la croissance topologique (création de nouveaux liens) ou la croissance des capacités des liens en relation avec leur utilisation, selon les échelles et les ontologies considérées. Nous distinguons pour simplifier des grands courants disciplinaires s'étant intéressé à la modélisation de la croissance des réseaux de transport : ceux-ci sont liés respectivement à l'économie des transports, la physique, la géographie des transport et la biologie.
}


\bpar{
\cite{xie2009modeling} develops a broad review on network growth modeling extending to other fields: transportation geography early developed empirical-based models but which did concentrate on topology reproduction rather than on mechanisms according to~\cite{xie2009modeling}; statistical models on case studies provide mitigated conclusions on causal relations between offer and demand; economists have studied infrastructure provision from both microscopic and macroscopic point of views, generally non-spatial; network science has provided toy-models of network growth based on structural and topological rules rather on rules inspired from real processes.
}{
On rejoint ainsi partiellement la classification de~\cite{xie2009modeling}, qui propose une revue étendue de la modélisation de croissance des réseaux, dans une perspective d'économie des transports mais en élargissant à d'autres champs. Selon~\cite{xie2009modeling}, la géographie des transports a développé très tôt des modèles basés sur des faits empiriques mais qui ont visé à reproduire la topologie plutôt que sur les mécanismes ; les modèles statistiques sur des cas d'étude fournissent des conclusions très mitigées sur les relations causales entre croissance du réseau et demande (la croissance étant dans ce cas conditionnée aux données de demande) ; les économistes ont étudié la production d'infrastructure à la fois d'un point de vue microscopique et macroscopique, généralement non spatialisés ; la science des réseaux a produit des modèles stylisés de croissance de réseau qui se basent sur des règles topologiques et structurelles plutôt que des règles se reposant sur des processus correspondant à des réalités empiriques.
}



\paragraph{Economics}{Economie}

\bpar{
Economists have proposed such models: \cite{zhang2007economics} reviews transportation economics literature on network growth within an endogenous growth theory~\cite{aghion1998endogenous}, recalling the three main features studied by economists on that subject that are road pricing, infrastructure investment and ownership regime, and describes an analytical model combining the three. An other approach not mentioned that we will develop further is biologically inspired network design. We first give some example of economic-based and geometrical-based network growth modeling attempts. \cite{yerra2005emergence} shows through a reinforcement economic model including investment rule based on traffic assignment that local rules are enough to make hierarchy of roads emerge for a fixed land-use.
}{
Les économistes ont proposé des modèles de ce type : \cite{zhang2007economics} passe en revue la littérature en Economie des Transports sur la croissance des réseaux, rappelant les trois aspects principalement traités par les économistes sur le sujet, qui sont la tarification routière, l'investissement en infrastructures et le régime de propriété, et propose finalement un modèle analytique combinant les trois. Ces trois classes de processus relèvent d'une interaction entre les agents économiques microscopiques (utilisateurs du réseau) et les agents de gouvernance. Les modèles peuvent inclure une description détaillée des processus de planification, comme~\cite{levinson2012forecasting} qui combine des enquêtes qualitatives et des statistiques pour paramétrer un modèle de croissance de réseau. \cite{xie2009jurisdictional} compare l'influence relative des processus de croissance centralisés (planification par une structure de gouvernance) et décentralisés (croissance locale ne rentrant pas dans le cadre d'une planification globale). \cite{levinson2003induced} procède à une étude empirique des déterminants de la croissance du réseau routier pour les \emph{Twin Cities} aux Etats-Unis (Minneapolis-Saint-Paul), établissant que les variables basiques (longueur, changement dans l'accessibilité) ont le comportement attendu, et qu'il existe une différence entre les niveaux d'investissement, impliquant que la croissance locale n'est pas affectée par les coûts, ce qui peut correspondre à une équité des territoires en termes d'accessibilité. Ces données sont utilisées par~\cite{zhang2016model} pour calibrer un modèle de croissance de réseau qui superpose les décisions d'investissement aux motifs d'utilisation du réseau. \cite{yerra2005emergence} montre avec un modèle économique basé sur des processus d'auto-renforcement (c'est à dire incluant une rétroaction positive des flux sur la capacité) et incluant une règle d'investissement basée sur l'attribution du trafic, que des règles locales sont suffisantes pour faire émerger une hiérarchie du réseau routier à usage du sol fixé. Une synthèse de ces travaux gravitant autour de \noun{Levinson} est faite dans~\cite{xie2011evolving}.
}

% sur ownership structure et pricing :
% https://sci-hub.cc/https://link.springer.com/article/10.1007/s11067-015-9309-3
% relié à gouvernance, mais trop loin du sujet ici.


\paragraph{Physics}{Physique}

\bpar{
A very similar model in~\cite{louf2013emergence} with simpler cost-benefits obtains the same conclusion.
}{
La physique a introduit récemment des modèles de croissance des réseaux d'infrastructure, em s'inspirant largement de cette littérature économique : un modèle très similaire au dernier cité est donné par~\cite{louf2013emergence} avec des fonctions coûts-bénéfices plus simples mais obtenant une conclusion similaire. Etant donné une distribution de noeuds (villes)\footnote{On se trouve ici dans un cas où l'hypothèse de non-évolution des population des villes tandis que le réseau s'établit itérativement trouve peu de support empirique ou thématique, puisqu'on a montré que réseau et villes avaient des échelles de temps d'évolution comparables. Ce modèle produit donc plus à proprement parler un \emph{réseau potentiel} étant donné une distribution de villes, et il est à interpréter avec précaution.} dont la population suit une loi puissance, deux villes seront connectées par un lien routier si une fonction d'utilité coût-bénéfice, combinant linéairement flux gravitaire potentiel et coût de construction\footnote{Ce qui donne une fonction de coût de la forme $C = \beta / d_{ij}^{\alpha} - d_{ij}$, où $\alpha$ et $\beta$ sont des paramètres}, a une valeur positive. Ces hypothèses locales simples suffisent à faire émerger un réseau complexe et des transitions de phase en fonction du paramètre de poids relatif dans le coût, conduisant à l'apparition de la hiérarchie. \cite{zhao2016population} applique ce modèle de manière itérative pour connecter des zones intra-urbaines, et montre que la prise en compte des populations dans la fonction de coût change significativement les topologies obtenues.
}



\bpar{
Whereas these models based on processes focus on reproducing macroscopic patterns of networks (typically scaling), geometrical optimization models aim to ressemble topologically real networks. \cite{barthelemy2008modeling} proposes a model based on local energy optimization but it stays very abstract and unvalidated. The morphogenesis model given in~\cite{courtat2011mathematics} using local potential and connectivity rules, even if not calibrated, seems to reproduce more reasonably real street patterns. Very close work is done in~\cite{rui2013exploring}.
}{
Une autre classe de modèles, proche dans leur idée des modèles procéduraux, se basent sur des processus d'optimisation géométrique locale, et visent à ressembler à des réseaux réels dans leur topologie. \cite{2016arXiv160906470B} étudie ainsi un modèle de croissance d'arbre appliqué aux pistes de fourmis, dans lequel coût de maintenance et coût de construction influencent tous les deux les choix de nouveau lien. \cite{barthelemy2008modeling} décrit un modèle basé sur une optimisation locale de l'énergie qui génère des réseaux routiers à l'aspect globalement crédible. Le modèle de morphogenèse de~\cite{courtat2011mathematics} qui utilise des potentiels locaux et des règles de connectivité, même s'il n'est pas calibré, reproduit de manière stylisée des motifs réels des réseaux de rues. Un modèle très proche est décrit dans~\cite{rui2013exploring}, tout en incluant des règles supplémentaires pour l'optimisation locale (prise en compte du degré pour la connection de nouveaux liens). La conception optimale de réseau, plutôt pratiquée par l'ingénierie, utilise des paradigmes similaires : \cite{vitins2010patterns} explore l'influence de différentes règles d'une grammaire de formes (notamment les motifs de connection entre les liens de différents niveaux hiérarchiques) sur les performances de réseaux générés par algorithme génétique.
}

% engineering : congestion / capacity
% http://www.sciencedirect.com/science/article/pii/S1877705816003131

%  La simplicité des hypothèses dans ce genre de modèle permet dans certains cas d'inclure des processus qui serait par ailleurs difficile à intégrer : 






%\paragraph{Transport geography}{Géographie des transports}

% la géographie des transports a développé très tôt des modèles basés sur des faits empiriques mais qui ont visé à reproduire la topologie plutôt que sur les mécanismes
% \comment[FL]{point tres important faire une section a part}

% Un peu de geo dans Ducruet :
% https://halshs.archives-ouvertes.fr/file/index/docid/605653/filename/Ducruet_Lugo_SAGE_Handbook_of_Transport_Studies_draft.pdf
% mais sinon remonte majoritairement a Hagget et Chorley, et des travaux contemporains lies a modelisation procedurale






\paragraph{Biological networks}{Réseaux biologiques}


\bpar{
Finally, an interesting and original approach to network growth are biological networks. These belong to the field of morphogenetic engineering pioneered by \noun{Doursat} that aim to design artificial complex system inspired from natural complex systems and in which a control of emerging properties is possible~\cite{doursat2012morphogenetic}. \emph{Physarum Machines}, that are models of a self-organized mould (slime mould) have been shown to provide efficient bottom-up solution to computationally heavy problems such as routing problems~\cite{tero2006physarum} or NP-complete navigation problems such as the Travelling Salesman Problem~\cite{zhu2013amoeba}. It has been shown to produce networks with Pareto-efficient cost-robustness properties~\cite{tero2010rules}, relatively close in shape to real networks (under certain conditions, see~\cite{adamatzky2010road}). This type of models can be of interest for us since auto-reinforcement mechanisms based on flows are analog to mechanisms of link reinforcement in transportation economics.
}{
Enfin, une approche originale et intéressante pour la croissance des réseaux est le réseau biologique. Cette approche appartient au champ de l'ingénierie morphogénétique dont \noun{Doursat} est un pionnier, qui vise à concevoir des systèmes complexes artificiels inspirés de systèmes complexes naturels et sur lesquels un contrôle des propriétés émergentes est possible~\cite{doursat2012morphogenetic}. Les \emph{Machines Physarum}, qui sont des modèles d'une moisissure auto-organisée (\emph{slime mould}) ont été prouvés comme résolvant de manière efficiente et par le bas des problèmes computationnellement lourds comme des problème de routage~\cite{tero2006physarum} ou des problèmes de navigation NP-complets comme le Problème du Voyageur de Commerce~\cite{zhu2013amoeba}, ce qui est porteur de sens au regard des liens entre différents types de complexité développés en~\ref{sec:epistemology}. Ils produisent des réseaux ayant des propriétés de coût-robustesse Pareto-efficientes~\cite{tero2010rules} qui sont typiques des propriétés empiriques des réseaux réels, et de plus relativement proches en forme de ceux-ci (sous certaines conditions, voir~\cite{adamatzky2010road}). Ce type de modèles peut être d'intérêt dans notre cas puisque les processus d'auto-renforcement basés sur les flots sont analogues aux mécanismes de renforcement de lien en économie des transports. Ce type d'heuristique a été testé pour générer le réseau ferré Français par~\cite{mimeur:tel-01451164}, faisant un pont intéressant avec les modèles d'investissement de \noun{Levinson}. Les critères de validation appliqués restent cependant limités, soit à un niveau inadapté aux faits stylisés étudiés (nombre d'intersection ou de branches) soit trop générales pouvant être produit par n'importe quel modèle (longueur totale et pourcentage de population desservie), et relèvent de critère de forme typique de la modélisation procédurale qui ne peuvent que difficilement rendre compte des dynamiques internes d'un système comme développé précédemment. De plus, prendre pour validation externe la production d'un réseau hiérarchique découle d'une exploration incomplète de la structure et du comportement du modèle, puisque celui-ci par ses mécanismes d'attachement préférentiel doit mécaniquement produire une hiérarchie.\comment[FL]{la encore c'est confus : il faut hierarchiser, structurer, expliquer}
}


% bio-inspired design :
% http://journals.sagepub.com/doi/abs/10.1177/2399808317690156
%  cool mais loin de la pb



\paragraph{Procedural modeling}{Modélisation procédurale}


\bpar{
Other tentatives \cite{de2007netlogo,yamins2003growing} are closer to procedural modeling~\cite{lechner2004procedural,watson2008procedural} and therefore not of interest in our purpose as they can difficultly be used as explicative models.
}{
D'autres tentatives comme~\cite{de2007netlogo,yamins2003growing} sont plus proches de la modélisation procédurale~\cite{lechner2004procedural,watson2008procedural} et pour cette raison n'ont pas d'intérêt pour notre cas puisqu'ils peuvent difficilement être utilisés comme modèles explicatifs. La modélisation procédurale génère des structures à la manière des grammaires de forme\footnote{Une grammaire de forme est un système formel (c'est à dire un ensemble de symboles initiaux, les axiomes, et un ensemble de règles de transformation) qui agit sur des objets géométriques. Partant de motifs initiaux, elles permettent de générer des classes d'objets}, mais celle-ci se concentre généralement sur la reproduction fidèle de forme locale, sans tenir compte des propriétés macroscopiques émergentes. Les classifier comme modèles de morphogenèse n'est pas correct et correspond à une incompréhension des mécanismes du \emph{Pattern Oriented Modeling}~\cite{grimm2005pattern}\footnote{Le \emph{Pattern Oriented Modeling} consiste à chercher à expliquer des motifs observés, généralement à plusieurs échelles, dans une démarche \emph{bottom-up}. La modélisation procédurale n'en relève pas, puisqu'elle vise à reproduire et non à expliquer.} d'une part et de l'épistémologie de la Morphogenèse d'autre part (voir~\ref{sec:interdiscmorphogenesis}). Nous utiliserons ce type de modèle (mélange d'exponentielles pour produire une densité de population par exemple) pour générer des données synthétiques initiales uniquement pour faire tourner d'autres modèles complexes (voir~\ref{sec:computation} et \ref{sec:correlatedsyntheticdata}).
}






%%%%%%%%%%%%%%%%%%
\subsection{Modeling co-evolution}{Modéliser la co-évolution}

Nous pouvons à présent nous intéresser aux modèles intégrant dynamiquement le paradigme Territoire $\leftrightarrow$ Réseau.

\subsubsection{Hybrid Modeling}{Modélisation Hybride}

\comment[JR]{faire la distinction coevol / hybride ?}

\bpar{
Models of simulation implementing a coupled dynamic between urban growth and transportation network growth are relatively rare, and always rather poor from a theoretical and thematic point of view.
}{
Nous désignerons largement par modèle hybride les modèles de simulation qui incluent un couplage des dynamiques de la croissance urbaine et du réseau de transport.  Ceux-ci sont relativement rares, et pour la plupart au stade de modèles stylisés. Les efforts étant assez disparates et dans des domaines très variés, il y a peu d'unité dans ces approches, si ce n'est l'abstraction de l'hypothèse d'interdépendance entre réseaux et caractéristiques du territoire dans le temps.
}


\bpar{
A generalization of the geometrical local optimization model described before was developed in~\cite{barthelemy2009co}. As for the road growth model of which it is an extension, no thematic nor theoretical justification of local mechanisms is provided, and the model is furthermore not explored and no geographical knowledge can be drawn from it.
}{
Une généralisation du modèle d'optimisation locale géométrique décrit précédemment a été développé dans~\cite{barthelemy2009co}, et cherche à capturer la co-évolution entre topologie du réseau et densité de ses noeuds. 

 \cite{ding2017heuristic} introduit un modèle de co-évolution entre différentes couches du réseau de transport, et montre l'existence d'un paramètre de couplage optimal en terme d'inégalités de centralité pour la conception d'un réseau : si on assimile le réseau routier à granularité très fine à une distribution de population, ce modèle se rapproche d'un modèle de co-évolution entre réseau de transport et territoire.
}




\bpar{
\cite{levinson2007co} adopts a more interesting economic approach, similar to a four step model (gravity-based origin-destination flows generation, stochastic user equilibrium traffic assignment) including travel cost and congestion, coupled with a road investment module simulating toll revenues for constructing agents, and a land-use evolution module updating actives and employments through discrete choice modeling. The experiments showed that co-evolving network and land uses lead to positive feedbacks reinforcing hierarchy, but are far from satisfying for two reasons: first network topology does not really evolve as only capacities and flows change within the network, what means that more complex mechanisms on longer time scales are not taken into account, and secondly the conclusions are very limited as model behavior is not known since sensitivity analysis is done on few one-dimensional spaces: exhaustive mechanisms stay thus unrevealed as only particular cases are described in the sensitivity analysis.
}{
\cite{levinson2007co} prend une approche économique plus intéressante\comment[FL]{de quel point de vue ?} du point de vue des processus de développement de réseau impliqués, similaire à un modèle à quatre étapes\comment[FL]{il faut expliquer un peu plus en detail} (génération de flux origine-destination basés sur la gravité, attribution du traffic par Equilibre Utilisateur Stochastique) qui inclut coût de transport et congestion, couplé avec un module d'investissement routier qui simule les revenus des péages pour les agents qui construisent, et un module d'évolution d'usage du sol qui simule les relocalisations des actifs et des emplois. Les expériences d'exploration de ce modèle montrent que l'usage du sol et le réseau en co-évolution conduisent à des retroactions positives renforçant les hiérarchies. Elles sont cependant loin d'être satisfaisantes pour deux raisons : d'une part la topologie du réseau n'évolue pas à proprement parler puisque seules les capacités et les flux changent dans le réseau, ce qui signifie que des mécanismes plus complexes (comme la planification de nouvelles infrastructures) sur de plus longues échelles de temps ne sont pas pris en compte, et d'autre part les conclusions sont assez limitées puisque le comportement du modèle n'est pas connu, les analyses de sensibilité étant faites sur un petit nombre d'espaces unidimensionnels \comment[FL]{la aussi c'est une autre discussion}: les mécanismes exhaustifs restent ainsi inconnus comme seuls des cas particuliers sont donnés dans l'analyse de sensibilité. \cite{li2016integrated} a récemment étendu ce modèle par l'ajout de prix immobiliers endogènes et d'une heuristique d'optimisation par algorithme génétique pour les agents décideurs.
}


\bpar{
 From an other point of view, \cite{levinson2005paving} is also presented as a model of co-evolution, but corresponds more to coupled statistical analysis as it relies on a Markov-chain predictive model. \cite{rui2011urban} gives a model in which coupling between land-use and network growth is done in a weak paradigm, land-use and accessibility having no feedback on network topology evolution.
}{
D'un autre point de vue, \cite{levinson2005paving} est aussi présenté comme un modèle de co-évolution mais correspond plus à une analyse statistique couplée \comment[FL]{de quoi ? quest ce qui est couple ?} puisqu'elle repose sur un modèle prédictif à chaîne de Markov. \cite{rui2011urban} décrit un modèle dans lequel le couplage entre usage du sol et la topologie du réseau est fait par un paradigme faible, l'usage du sol et l'accessibilité n'ayant pas de retroaction sur la topologie du réseau\comment[FL]{en parlant comme cela, tu considere comme aquis que dans ce type de modele, on travaille par sous-blocs ayant des liens avec les autres : cest discutable donc a discuter}[(JR) ontologies separees (cf chap 1), donc necessairement decomposition modulaire avec ontologie de couplage (cf chap 9) $\rightarrow$ a positionner en intro de la sous-partie], le modèle d'usage du sol étant conditionné à la croissance du réseau autonome. Ce modèle est mis en perspective avec d'autres modèles d'usage du sol et de croissance de réseau dans~\cite{rui2013urban}\comment[FL]{et alors ?}.
}



\bpar{
 \cite{achibet2014model} describes a co-evolution model at a very small scale (scale of the building), in which evolution of both network and buildings are ruled by a same agent (influenced differently by network topology and population density) what implies a too strong simplification of underlying processes. Finally, a simple hybrid model explored and applied to a toy planning example in~\cite{raimbault2014hybrid}, relies on urban activities accessibility mechanisms for settlement growth with a network adapting to urban shape. The rules for network growth are too simple to capture processes we are interested in, but the model produces at a small scale a broad range of urban shapes reproducing typical patterns of human settlements.
}{
\cite{achibet2014model} décrit un modèle de co-évolution à une très petite échelle (échelle du bâtiment), dans lequel l'évolution du réseau et des bâtiments sont tous les deux régis par un agent commun (qui est influencé différemment par la topologie du réseau et la densité de population) ce qui implique une simplification trop grande des processus sous-jacents.

\cite{ruas2011conception} regles procedurale, echelle micro.

% http://florence.curie.free.fr/pdf/jfsma2010.pdf
% http://florence.curie.free.fr/pdf/agile2010.pdf

% http://geopensim.ign.fr/publication.html

% https://scholar.google.fr/scholar?cites=18028265655543708128&as_sdt=2005&sciodt=0,5&hl=fr

 Enfin, un modèle hybride simple exploré et appliqué à un exemple jouet de planification dans~\cite{raimbault2014hybrid}, repose sur les mécanismes d'accès aux activités urbaines pour la croissance des établissements avec un réseau s'adaptant à la forme urbaine. Les règles pour la croissance du réseau sont trop simples pour capturer des processus plus élaborés qu'une simple connection systématique (comme une rupture de potentiel par exemple), mais le modèle produit à une petite échelle une large gamme de formes urbaines qui reproduisent les motifs typiques des établissements humains. Ce modèle s'inspire de~\cite{moreno2012automate} pour ses mécanismes de base mais permet une génération de formes bien plus larges par la prise en compte des fonctions urbaines.
}


\bpar{
}{
A cette échelle, i.e. urbaine ou métropolitaine, les mécanismes de localisation de population influencée par l'accessibilité couplés à des mécanismes de croissance de réseau optimisant certaines fonctions semblent être la règle pour ces modèles : de la même façon, \cite{wu2017city} couplent un Automate Cellulaire de diffusion de population à un réseau optimisant un coût local dépendant de la géométrie et de la distribution de population. De manière conceptuelle, une certaine forme de couplage fort est utilisé dans~\cite{bigotte2010integrated} qui par une approche de recherche opérationnelle propose un algorithme de design de réseau pour optimiser l'accessibilité aux services, prenant en compte à la fois la hiérarchie du réseau et celle des centres connectés. Enfin, le modèle proposé par~\cite{blumenfeld2010network} peut être vu comme un pont vers les approches de type système urbain, puisqu'il simule les migrations entre villes et la croissance du réseau induite par une rupture de potentiel lorsque les détours sont trop grands.
}


\bpar{}{
A une échelle macroscopique et également plus proche de la modélisation de systèmes urbains que nous développerons dans la section suivante, \cite{baptiste1999interactions} propose de coupler un modèle de croissance urbaine basé sur les migrations (introduit par l'application de la synergétique au système de ville par~\cite{sanders1992systeme}) avec un mécanisme d'auto-renforcement des capacités pour le réseau routier sans modification topologique\footnote{Plus précisément, la topologie du réseau est fixée dans le temps, mais les capacités des liens évoluent. La règle est une augmentation de la capacité lorsque le flux dépasse celle-ci par un seuil donné comme paramètre.}. Sa dernière version est présentée par~\cite{baptistemodeling}. Les conclusions générales qui peuvent être tirées de ce travail sont que ce couplage permet de faire émerger une configuration hiérarchique (mais on sait par ailleurs que des modèles plus simples, un attachement préférentiel uniquement par exemple, permettent de reproduire ce fait stylisé) et que l'ajout du réseau produit un espace moins hiérarchique, permettant à des villes moyennes de bénéficier de la rétroaction du réseau de transport.
}




\subsubsection{Urban Systems Modeling}{Modélisation de Systèmes Urbains}



\bpar{
An approach rather close to our current questioning is the one of integrated modeling of system of cities. In the continuity of Simpop models for city systems modeling, \cite{schmitt2014modelisation} describes the SimpopNet model which aim was precisely to integrate co-evolution processes in system of cities on long time scales, typically via rules for hierarchical network development as a function of cities dynamics coupled with these that depends on network topology. Unfortunately the model was not explored nor further studied, and furthermore stayed at a toy-level. \noun{Cottineau} proposed transportation network endogenous growth as the last building bricks of her Marius productions but it stayed at a conceptual construction stage.
}{
Une approche relativement proche des précédentes, mais ayant des caractéristiques propres, est celle de la modélisation intégrée des systèmes de villes. Dans la continuité des modèles Simpop pour modéliser les systèmes de villes, \cite{schmitt2014modelisation} décrit le modèle SimpopNet qui vise à précisément intégrer les processus de co-évolution dans les systèmes de villes à longue échelle temporelle, typiquement par des règles pour un développement hiérarchique du réseau comme fonction des dynamiques des villes, couplées à celles-ci qui dépendent de la topologie du réseau. Malheureusement le modèle n'a pas été exploré ni étudié de manière plus approfondie, et de plus est resté au niveau de modèle jouet. \cite{cottineau2014evolution} propose une croissance endogène des réseaux de transport comme la dernière brique de construction du cadre de modélisation MARIUS, mais cela reste à un niveau conceptuel puisque cette brique n'a pas encore été spécifiée ni implémentée. Il n'existe à notre connaissance pas de modèle empirique ou appliqué à un cas concret se basant sur une approche de la co-évolution par les systèmes urbains vus par la Théorie Evolutive des Villes.
}

\bpar{
We shall position more in that stream of research in this thesis.
}{
Nous nous positionnerons particulièrement dans cette lignée de recherche dans cette thèse, vu l'importance que prendra la Théorie Evolutive dans notre démarche théorique et de modélisation comme nous le détaillerons par la suite. Typiquement, les hypothèses ontologiques fondamentales telles le rôle des relations et de la configuration spatiales, ou la présence d'un équilibre\comment[FL]{termes a definir} - nous considérons les systèmes urbains comme des systèmes complexes auto-organisés loin de l'équilibre, sont représentatives de cette approche si on les considère conjointement.
}

% L'ensemble des briques est nécessaire pour comprendre les implications de ce positionnement, mais le lecteur pressé pourra directement consulter le chapitre~\ref{ch:theory}\comment[FL]{a eviter} pour une synthèse des implications théoriques à différents niveaux d'abstraction. 


\bpar{}{
On voit bien l'opposition aux principes épistémologiques de l'économie géographique : \cite{fujita1999evolution} introduisent par exemple un modèle évolutionnaire capable de reproduire une hiérarchie urbaine et une organisation typique de la Théorie des Places Centrales~\cite{banos2011christaller}, mais repose toujours sur la notion d'équilibres successifs, et surtout considère un modèle ``à-la-Krugman'' c'est à dire un espace à une dimension, isotrope, et dans lequel les agents sont répartis de manière homogène \comment[AB]{attention, ça n’est pas forcément une limite fondamentale, tout dépend des objectifs. Peut être suffisant en particulier pour étudier l’apparition de ces « particularités » dont tu parles ensuite}[(JR) oui en effet, on rejoint la question de representation des territoires, qu'est ce qui est necessaire a mettre ou non, selon l'objectif poursuivi (// la meme discussion plus loin) // idee de com pour le cist]. Cette approche peut être instructive sur les processus économiques en eux-mêmes mais plus difficilement sur les processus géographiques, puisque ceux-ci impliquent un déroulement des processus économiques dans l'espace géographique dont les particularités spatiales qui ne sont pas prise en compte dans cette approche sont essentielles. Notre travail s'attachera à montrer dans quelle mesure cette structure de l'espace peut être importante et également explicative, puisque les réseaux, et encore plus les réseaux physiques induisent des processus dépendants au chemin spatio-temporel et donc sensibles aux singularités locales et propices aux bifurcations induites par la combinaison de celles-ci et de processus à d'autres échelles (par exemple la centralité induisant un flux).
}




\subsubsection{Co-evolution}{Co-évolution}


Après cet aperçu de la littérature, incluant différents degrés de couplage entre les composantes des réseaux et territoires, nous sommes en mesure de préciser ce que nous entendrons par \emph{modéliser la co-évolution}.

%La vision donnée ici a un but opérationnel, puisqu'il ne nous paraissait pas pertinent\comment[FL]{c'est dommage car en l'etat tu as surtout juxtapose des lectures sans en tirer tout ce qui pouvait etre tire} de donner d'emblée une vision trop théorique et abstraite (qui sera développée en~\ref{sec:theory}).

En Géographie Economique, la notion de co-évolution a également été mobilisée. Ainsi, \cite{doi:10.1080/00343400802662658} introduit un cadre conceptuel pour permettre de concilier nature évolutionnaire des firmes, théorie des clusters et réseaux de connaissance, dans lequel la co-évolution entre réseaux et firmes est centrale, et qui est définie comme une causalité circulaire entre différentes caractéristiques de ces sous-systèmes. L'idée d'entités évolutionnaires en économie vient à contre-courant du courant néoclassique qui reste majoritaire, mais trouve un écho de plus en plus pertinent~\cite{nelson2009evolutionary}. Pour la géographie, les travaux les plus proches empiriquement et théoriquement des notions de co-évolution sont étroitement liés à la Théorie Evolutive des Villes. Il n'est pas évident de tracer dans la littérature à quel moment la notion a été clairement formalisée, mais il est évident qu'elle était présente dès les fondements de la théorie comme le rappelle \noun{Denise Pumain} (voir~\ref{app:sec:interviews}) : le système complexe adaptatif est composé de sous-systèmes en interdépendances complexes, souvent circulairement causales. Les premiers modèles incluent bien cette vision de manière implicite, mais la co-évolution n'est pas appuyée explicitement ou définie précisément, en termes qui seraient quantifiables ou identifiables structurellement. \cite{paulus2004coevolution} a amené des preuves empiriques de mécanismes de co-évolution par l'étude de l'évolution des profils économiques des villes françaises. L'interprétation utilisée par~\cite{schmitt2014modelisation} repose sur une entrée par la Théorie Evolutive, mais n'approfondi pas au delà d'une lecture des systèmes de villes comme entités fortement interdépendantes. Or l'interdépendance est une notion aussi lâche que le fameux ``tout interagit avec tout''\comment[FL]{mal dit}, c'est à dire qu'elle est particulièrement creuse si elle n'est pas quantifiée\comment[FL]{c'est un postulat fort. je ne suis pas certain d'etre d'accord}[(JR) ok c'est mal dit, malentendu sur la notion de quanti - je veux dire une comprehension fine et integree]. Elle permet comme prémisse épistémologique de considérer certaines ontologies et certaines démarches de modélisation, mais ne permet pas de comprendre finement la structure et les processus d'un système. Par exemple, étant donné un réseau topologique d'interaction entre entités et des motifs temporels de propagation correspondants, on peut se demander quels sont les motifs de corrélations statiques et dynamiques correspondants, s'il existe des causalités et à quelles échelles\comment[FL]{a supprimer cet exemple n'eclaire pas la proposition}. Il existe en pratique un grand nombre de ``régimes'' de co-évolution possibles\comment[FL]{discutable}, liés à la structure du réseau écologique de la niche correspondante si on interprète celle-ci de cette façon~\cite{holland2012signals}. L'idée de diffusion hiérarchique de l'innovation dans la théorie évolutive capture par exemple qualitativement certains de ces aspects, mais la quantification des régimes correspondants et donc de la co-évolution reste une question ouverte.

% \comment[AB]{dependance forte : pas d’accord ! les dépendances faibles (interactions indirectes) font également partie du dispositif. Défini ce que tu appelles « entités »}


L'une de nos contributions principales\comment[FL]{c'est un point important qui se trouve un peu dilue ici} est sous forme théorique en~\ref{sec:theory}. Il s'agira de clarifier cette notion et d'en donner une définition précise. A ce stade, l'état de l'art fait ci-dessus témoigne d'une faiblesse de la littérature dans le domaine du couplage fort entre évolution des territoires et croissance des réseaux, vu la portée restreinte et la disparité des travaux revus. Les lacunes à combler sur ce point seraient donc liées à l'introduction de modèles fortement couplés dans le temps plus ou moins multi-processus et multi-échelles, pour lesquels une partie des modèles décrits ci-dessus sont précurseurs.





\stars








