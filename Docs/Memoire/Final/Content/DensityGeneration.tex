



\section{Urban Morphogenesis by Aggregation-diffusion}{Morphogenèse Urbaine par Agrégation-diffusion}



\todo{insert / translate Density paper}




\comment{(Florent) avant de dire ce que tu fais en couplant cela, on a besoin de connaitre un panorama des différentes approches pour modéliser les dynamiques urbaines}


\comment{(Florent) il y a d'autres modèles couplant diffusion et croissance comme Fischer-Skellam (?) \cite{bosch1990velocity} } % TODO check this model.

\comment{(Florent) effective dimension of urban system : sens ?}

\comment{(Florent) $n_d$ est ce un paramètre du modèle ?}

\comment{(Florent) on indicator choice : pourquoi ce choix ? qu'en attends tu ?}

\comment{(Florent) on scala implementation : si la question computationnelle prend de l'importance dans la thèse, il faudra donner de la matière}

\comment{(Florent) on LHS : qu'est ce que cela veut dire : force brute ?}

\comment{(Florent) figure/par on real data : ordre ? real data : lesquelles ? qu'est ce que ça veut dire ? tout le monde n'a pas lu Cottineau :) }

\comment{(Florent) on Moran vs Entropy : pouvait-on prévoir zone impossible ? (dispersion faible incompatible avec forte autocorrelation spatiale}

\comment{(Florent) on PCA objective : pourquoi se fixer cet objectif si particulier ?}

\comment{(Florent) on calibration process : là encore, pourquoi ce choix? tout est discutable : il faut expliciter}

\comment{(Florent) on multi-scale dvlpmt : pas clair ce que tu as en tête ici : je ne sais pas si tu auras le temps de creuser cela, mais pour du multi-scalaire, les schémas sont très aidant car c'est vite difficile à visualiser}




