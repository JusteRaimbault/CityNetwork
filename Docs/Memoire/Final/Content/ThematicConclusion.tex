




%----------------------------------------------------------------------------------------

\newpage


\section*{Chapter Conclusion}{Conclusion du Chapitre}


Les territoires, que nous avons défini comme territoires humains, interagissent de manière complexe avec les réseaux, en particulier ceux de transport, comme montré par les nombreux exemples empiriques ou les constructions théoriques passés en revue. A différentes échelles temporelles typiques, l'année, la décennie et le siècle, correspondent plus ou moins des échelles spatiales : métropolitaine, régionale et système de villes, ainsi que des processus : mobilité, accessibilité et relocalisations, effets systémiques structurels et bifurcations.\comment[FL]{plus haut} Les situations concrètes témoignent de réalités locales déclinés avec différentes nuances, et des processus portant ces processus abstraits avec différents rôles et interactions entre eux. Nous avons dans une première section clarifié cette notion d'interaction\comment[FL]{pas vraiment : hierarchiser, faire preuve de pedagogie.} entre réseaux de transports et territoires, et suggéré une approche par la co-évolution pour tenir compte de cette complexité. Afin de mieux cerner ces notions sur des exemples géographiques concrets, nous avons développé en~\ref{sec:casestudies} deux cas d'étude métropolitain d'actualité, et souligné les certitudes en termes d'impact d'accessibilité pour des projets majeurs d'infrastructures qui s'accompagnent systématiquement d'incertitude en terme de trajectoire du système à plus long terme. Enfin, pour limiter les effets d'intermédiaire qui augmenteraient la distance au terrain géographique réel qui est selon~\cite{lefort2012terrain} déjà très présente dans les études le prenant comme matériau empirique principal, nous proposons en~\ref{sec:qualitative} une excursion par des éléments de terrain dans le Guangdong, Chine.\comment[FL]{a reprendre} A ce stade, ayant introduit l'objet d'étude thématique, nous proposons de restreindre la portée des entrées prises sur le sujet, et s'intéresser plus particulièrement aux approches impliquant une modélisation, faisant le choix d'un rôle fondamental du \emph{modèle} (que nous définirons par la suite) dans la production de connaissance.





\stars


%-------------------------


% no role here, or put that elsewhere

%\section{Research Question}{Question de Recherche}

%To close this thematic touring introducing chapter, we can state a general research question that frames our further theoretical constructions and first modeling attempts. It is roughly the same as the problematic given at the end of previous section, but adding the insight of modeling as the approach to understand these complex systems.

%networked territorial systems with an emphasize on the role of transportation networks in system evolution processes.

%\medskip

%\textbf{General research Question.} \textit{To what extent a modeling approach to territorial systems as networked human territories can help disentangling complexly involved processes ?}

%\comment{(Florent) à mieux détailler et à réduire d'abord }

%\medskip

%This question will be refined by theoretical developments in the next chapter and experiments in the followings.



