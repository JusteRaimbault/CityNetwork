






%\chapter*{Part I Conclusion}{Introduction de la Partie I}
\chapter*{Conclusion de la Partie I}


% to have header for non-numbered introduction
\markboth{Conclusion}{Conclusion}


%\headercit{}{}{}


\comment[JR]{here fix scales and process, and why morphogenesis and evolutive urban theory}




% why does not study mobility et suggestion preliminaire de meso-macro scales only. -> conclusion Partie I

%\bpar{}
%{
%Il faut aussi garder à l'esprit que le transport en lui-même est différent des réseaux de transport\comment[FL]{en quoi estce un argument ?}, puisqu'il correspond à l'utilisation de ceux-ci par les agents territoriaux. Dans une grande partie des approches que nous décrirons par la suite, et typiquement les approches appliquée en planification urbaine, la modélisation du transport s'axe sur des question de demande, d'offre, de congestion, c'est à dire à des échelles relatives à la mobilité, et est liée au réseau mais ne se concentre pas directement sur celui-ci\comment[FL]{ce n'est pas clair $\rightarrow$ la croissane du reseau ?, l'usage du reseau ? la croissance de l'usage du reseau ?} comme notre positionnement propose\comment[AB]{$\simeq$}.
%}


