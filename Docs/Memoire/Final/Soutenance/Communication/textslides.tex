\documentclass[11pt]{article}

% general packages without options
\usepackage{amsmath,amssymb,bbm}

\usepackage[margin=1cm]{geometry}

\begin{document}


\subsection*{Slide Titre}

\begin{itemize}
	\item Presentation, remerciement jury, public
	\item Remercie les rapporteurs pour le travail fourni. J'ai pris bonne note des pistes d'amelioration et des correction suggérées - la version courante en incluant une grande partie, l'ensemble sera pris en compte dans la version finale avec les suggestions formulées lors de la soutenance
\end{itemize}



\subsection*{Slide 1 : exemple introductif}

\begin{itemize}
	\item Introduction du sujet par un exemple concret
	\item Développement récent du réseau à grande vitesse Chinois permet d'observer des interactions effectives entre transport et ville
	\item Par exemple à gauche, promotion de l'utilisation du train et de la proximité à la gare, à droite des quartiers entièrement nouveaux autour des gares.
	\item L'accessibilité a subi des transformations considérables pour cet exemple de Tangjia. Il s'agit du haut de l'iceberg de dynamiques complexes d'interactions, que nous prendrons le parti d'étudier selon un point de vue particulier que nous allons à présent détailler.
\end{itemize}



\subsection*{Slide 2 : problématique}


\begin{itemize}
	\item 
\end{itemize}









%%%%%%%%%%%%%%%%%%%%
%% Biblio
%%%%%%%%%%%%%%%%%%%%
%\tiny

%\begin{multicols}{2}

%\setstretch{0.3}
%\setlength{\parskip}{-0.4em}


%\bibliographystyle{apalike}
%\bibliography{/Users/juste/ComplexSystems/CityNetwork/Biblio/Bibtex/CityNetwork}%,biblio}
%\end{multicols}



\end{document}

