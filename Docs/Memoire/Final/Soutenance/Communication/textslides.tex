\documentclass[11pt]{article}

% general packages without options
\usepackage{amsmath,amssymb,bbm}

\usepackage[margin=1cm]{geometry}

\usepackage[T1]{fontenc}
\usepackage[utf8]{inputenc}


\begin{document}


\subsection*{Slide Titre}

\begin{itemize}
	\item Presentation, remerciement jury, public
	\item Remercie les rapporteurs pour le travail fourni. J'ai pris bonne note des pistes d'amelioration et des correction suggérées - la version courante en incluant une grande partie, l'ensemble sera pris en compte dans la version finale avec les suggestions formulées lors de la soutenance
\end{itemize}



\subsection*{Slide 1 : exemple introductif}

\begin{itemize}
	\item Introduction du sujet par un exemple concret
	\item Développement récent du réseau à grande vitesse Chinois permet d'observer des interactions effectives entre transport et ville
	\item Par exemple à gauche, promotion de l'utilisation du train et de la proximité à la gare, à droite des quartiers entièrement nouveaux autour des gares.
	\item L'accessibilité a subi des transformations considérables pour cet exemple de Tangjia. Il s'agit du haut de l'iceberg de dynamiques complexes d'interactions, que nous prendrons le parti d'étudier selon un point de vue particulier que nous allons détailler par la suite.
\end{itemize}


\subsection*{Slide 2 : parcours}


\begin{itemize}
	\item Il n'est pas inintéressant, toujours en guise d'introduction, d'expliciter mon parcours personnel, ce qui permettra de mieux appréhender les positionnements et les choix que nous développerons.
	\item Partant d'une formation d'ingénieur généraliste, mon intérêt pour l'urbain m'a poussé vers la découverte du monde de l'architecture et de l'urbanisme, puis une formation à l'Ecole des Ponts et Chaussées en parallèle de la formation interdisciplinaire du Master Systèmes Complexes.
	\item Finalement choix de la géographie qui permet justement d'articuler ces deux aspects théorique et thématique dont je positionne l'articulation comme une composante cruciale de mon identité scientifique.
	\item Les illustrations à droite montrent cette évolution progressive dans les modèles développés à différentes périodes : jeu de la vie, morphogenèse hybride, réseau biologique auto-organisé, analyse des dynamiques territoriales en Ile-de-France.
\end{itemize}


\subsection*{Slide 3 : problématique}


\begin{itemize}
	\item Nous pouvons à présent préciser la problématique générale de la thèse et la stratégie adoptée
	\item Un certain nombre d'approches, comme celle de la théorie évolutive des villes élaborée par Denise Pumain, et en particulier les travaux d'Anne Bretagnolle dans le cas des réseaux de transport et des villes, postulent des dynamiques fortement couplées entre ces deux objets, comprises comme co-évolutive, au sens d'interdépendances réciproques fortes entre les différentes composantes du systèmes.
	\item Notre premier axe de recherche est ainsi de tester la pertinence de ce positionnement, en cherchant à construire une définition précise ainsi qu'une méthode de caractérisation empirique de ces dynamiques co-évolutives
	\item Pour de nombreuses raisons que nous ne pourrons détailler intégralement ici, mais qui incluent la pauvreté des données disponibles, les spécificités de chaque cas d'étude considéré, des dynamiques s'opérant sur le temps long, des couplages fort entre composantes rendant leur compréhension couplée plus difficile, la connaissance est rapidement limitée. Nous considérons alors la modélisation comme un instrument de connaissance, permettant de tirer des conclusions indirectes sur les processus, de considérer des systèmes génériques, de tester des hypothèses dans des laboratoires virtuels, à différentes échelles et selon différents aspects.
	\item Notre deuxième axe s'articule ainsi autour de la construction de modèles de co-évolution, incluant la détermination d'échelles et ontologies pertinentes pour cette modélisation. La modélisation est donc un sujet d'étude à part entière, confirmant la dualité de notre travail affirmée précedemment. 
\end{itemize}



\subsection*{Slides 4a, 4b, 4c : lecture par les domaines de connaissance}

\begin{itemize}
	\item Cette articulation selon différentes dimensions de la connaissance peut en fait se généraliser, et une lecture de la thèse au prisme des \emph{Domaines de connaissance}, initialement introduits par Livet, Sanders et Muller pour la construction de modèles de simulation en sciences sociales, permet de mieux positionner notre contribution ainsi que de prendre du recul sur l'organisation des connaissances produites.
	\item Nous voyons ici les trois domaines initiaux, théorique (constructions conceptuelles), empirique (connaissances concrètes) et de la modélisation, tous en interaction forte, illustrés par des éléments de différentes parties de la thèse. Tout au long de notre travail, cette interdépendance est inévitable et même prise comme un atout pour la construction d'une connaissance voulue complexe au sens de Morin.
	\item \textit{slide 4b} Un approfondissement de la lecture par domaines conduit vite à intégrer trois domaines supplémentaires, ceux des méthodes, outils et données, dans lesquels nous avons également été amenés à contribuer. Nous postulons qu'une connaissance complexe nécessite un couplage fort des différents domaines. La suite de la présentation suivra ce prisme de lecture.
	% note : ne pas evoquer reflexivité ici ?
	\item \textit{slide 4c} Pour donner un aperçu d'une mise en perspective plus fine de l'ensemble des contributions principales de la thèse, ce schéma situe les différentes sections à la fois par rapport aux domaines et au axes de la problématique.
\end{itemize}


\subsection*{Slide 5 : positionnement théorique}

\begin{itemize}
	\item Nous commençons ainsi par détailler nos fondations théoriques, qui conditionnent les objets et processus étudiés, permettent de préciser la définition de la co-évolution, et de suggérer des entrées thématiques pertinentes.
	\item Concernant la construction des objets géographiques étudiés, notre entrée sur les villes et les territoires suit celle de la Théorie evolutive, considérant les villes comme systèmes dans des systèmes de villes, et les territoires comme constructions associées. Au sein de ceux-ci émergent des réseaux par l'intermédiaire de la réalisation de projets transactionnels au sens de la théorie territoriale des réseaux de Gabriel Dupuy. Reformulé sous cet angle, l'existence et la nature des processus de co-evolution revient à comprendre les processus précis par lesquels cette materialisation peut s'opérer, et réciproquement dans quelle mesure celle-ci peut influencer les dynamiques territoriales.
	\item Nous proposons alors une définition de la co-évolution, basée sur une perspective multi-disciplinaire, proposant 3 niveaux possible auxquels de processus de co-évolution peuvent opérer : celui des individus, lorsque les caractéristiques de différents éléments s'influencent réciproquement ; celui des populations - qui correspondra au sens statistique et celui originel introduit par la biologie, supposant l'existence de niches au sein desquelles les caractéristiques de deux populations sont mutuellement liées ; un niveau systémique correspondant à des interdépendances généralisées entre agents.
	\item Deux entrées pour répondre à notre problématique sont alors proposées : une entrée par la morphogenèse, que l'on peut comprendre comme l'émergence de la forme et de la fonction d'un système, et qui est suggérée par la notion de niche territoriale au sein de laquelle une coevolution statistique s'opère ; une entrée par la théorie évolutive permettant d'appréhender les systèmes au niveau macroscopique.
\end{itemize}




\subsection*{Slide 6 : méthode de caractérisation}


\begin{itemize}
	\item Elaborant sur ces bases théoriques, celles-ci demandent une vérification empirique, qui demande un travail méthodologique de construction d'une méthode de caractérisation, puisque la littérature n'en propose pas dans le cas des système territoriaux, et qui serait associées à une définition précise.
	\item Le schéma ici décrit le principe de l'application d'une méthode de corrélations spatio-temporelles retardées à des variables caractéritiques des éléments étudiés. Plus précisément, supposons la donnée de variables de mesure pour les dynamiques des territoires et pour celles des réseaux, qui seront par exemple la croissance de la population et celle de l'accessibilité. La corrélation entre les séries temporelles correspondantes, au sein de niches spatiales et sur une fenetre temporelle, étant donné un retard entre les deux séries, permet de caractériser une causalité faible entre les deux variables.
	\item Par exemple, la première ligne montre une corrélation toujours nulle en fonction du retard, et pas de relation entre les variables. La deuxième et la troisième un maximum pour un retard positif ou negatif, et donc un sens de causalité. La dernière nous intéresse particulièrement, puisque une causalité réciproque s'observe, ce qui correspondra à une co-évolution au sens statistique que nous avons défini.
	\item La dernière colonne donne l'interprétation de ces relations sous forme d'un graphe causal.
\end{itemize}



\subsection*{Slide 7 : observations empiriques}


\begin{itemize}
	\item L'application de cette méthode à différents cas d'étude, à différentes échelles temporelles et spatiales, donne des résultats contrastés.
	\item Le premier exemple, dans la colonne de gauche, réalisé en collaboration avec Solène Baffi, montre les graphes de correlations retardees pour les croissances de l'accessibilite et de la population sur le temps long (1930-2000) en Afrique du Sud. Un effet statistiquement significatif montre une inversion du sens de la causalité au cours du temps, suggérant un impact des politiques de ségrégations non seulement sur les dynamiques territoriales, mais aussi sur leur relations avec celles des réseaux, en quelque sorte un effet au second ordre.
	\item Dans le cas métropolitain des quartiers de gare du Grand Paris express, pour lequel nous avons recherché des effets d'anticipation par l'annonce du tracé de cette nouvelle infrastructure dont la construction a récemment débuté. Les variables étudiées sont ici des caractéristiques socio-économiques ainsi que les dynamiques des prix immobiliers. Il est plus difficile de voir des effets significatifs, mais nous observons par exemple une bulle des prix immobiliers autour des gares.
	\item Dans le cas de la France sur le temps long que nous ne montrons pas ici, aucune relation n'est quantifiable. Ainsi, on observe empiriquement une diversité de situations, diversité que les modèles devront chercher à reproduire.
\end{itemize}

\subsection*{Slide 8 : cartographie des disciplines}

\begin{itemize}
	\item Avant de nous attaquer à la construction de modèles, une dernière incursion dans le domaine théorique (ou empirique si on se place à un niveau meta et on considère la science comme objet d'étude en lui-même) est nécessaire pour comprendre la pertinence de notre double entrée et la difficulté de l'entreprise de modélisation.
	\item Nous avons mené un travail d'épistémologie quantitative, c'est à dire de cartographie des connaissances (ce qui est plus général qu'un exercice de bibliométrie), autour des modèles s'intéressant aux relations entre réseaux et territoires.
	\item Sans rentrer dans le détail de la méthode qui se base sur un couplage d'analyse de réseaux de citations et de réseaux sémantiques, nous présentons ici une carte issue de la spatialisation d'un réseau de citations. Nous observons le positionnement relatif des disciplines, de la géographie critique et sciences politiques ainsi que géographie théorique et quantitative à gauche, et la physique à l'opposée, le pont étant fait par l'économie urbaine, la planification, les études d'accessibilité et le transport. Notre positionnement est donc forcément à cheval sur l'ensemble de ces disciplines : nos modèles de systèmes de villes emprunteront à la géographie, ceux de morphogenese à la physique et a l'économie, tandis que les études d'accessibilité jouent un rôle crucial dans les études empiriques que nous menons.
\end{itemize}




\subsection*{Slide 9 : modélisation macro}

% - modele statique revele effets de réseaux
% - modele de co-evolution synthetique exhibe nombreux regimes, dont co-evolution, diversite plus large que modele existant dans la literature
% - deux modeles calibrés sur données fr : extrapolation effet tunnel par exemple








\subsection*{Slide 10 : modélisation meso}


% - complementarité de multiples heuristiques de croissance de reseau
% - calibration au premier et second ordre
% - Lutecia : vers des modèles plus complexes





\subsection*{Slide 11 : Ouvertures}









%%%%%%%%%%%%%%%%%%%%
%% Biblio
%%%%%%%%%%%%%%%%%%%%
%\tiny

%\begin{multicols}{2}

%\setstretch{0.3}
%\setlength{\parskip}{-0.4em}


%\bibliographystyle{apalike}
%\bibliography{/Users/juste/ComplexSystems/CityNetwork/Biblio/Bibtex/CityNetwork}%,biblio}
%\end{multicols}



\end{document}

