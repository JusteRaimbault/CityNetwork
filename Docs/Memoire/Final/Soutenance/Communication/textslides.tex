\documentclass[12pt]{article}

% general packages without options
\usepackage{amsmath,amssymb,bbm}

\usepackage[margin=3cm]{geometry}

\usepackage[T1]{fontenc}
\usepackage[utf8]{inputenc}


\begin{document}


\subsection*{Slide Titre}

\begin{itemize}
	\item Bonjour à toutes et à tous ; je suis Juste Raimbault, doctorant à Géographie-cités et au LVMT.
	\item Je remercie le jury pour l'honneur qu'il me fait d'évaluer ce travail de thèse, et le public pour sa présence.
	\item Je remercie les rapporteurs pour le travail fourni. J'ai pris bonne note des corrections suggérées.
	\item La version courante en inclut une grande partie, et l'ensemble sera pris en compte dans la version finale.
	\item Je vais présenter mon travail de thèse, qui a porté sur la caractérisation et la modélisation de la co-évolution des réseaux de transport et des territoires.
\end{itemize}


\newpage


\subsection*{Parcours}


\begin{itemize}
	\item Pour commencer, il est important d'expliciter mon parcours personnel.
	\item Cela permettra de mieux appréhender certains positionnements et choix qui seront développés par la suite.
	\item Partant d'une formation d'ingénieur généraliste, mon intérêt pour l'urbain m'a poussé vers la découverte du monde de l'architecture et de l'urbanisme, puis vers une formation à l'Ecole des Ponts et Chaussées.
	\item J'ai suivi en parallèle la formation interdisciplinaire du Master Systèmes Complexes, dont le choix m'a été suggéré par la rencontre déterminante de Paul Bourgine et Kashayar Pakdaman, que je remercie aujourd'hui.
	\item Finalement, la logique de mon parcours m'a conduit à une orientation progressive vers les sciences humaines et sociales.
	\item Le choix de la géographie a justement permis d'articuler ces deux aspects théorique et thématique dont j'ai rapidement positionné l'intégration comme une composante cruciale de mon identité scientifique.
	\item Les illustrations à droite montrent cette évolution progressive dans les modèles développés à différentes périodes :
	\begin{itemize}
		\item le jeu de la vie implémenté pour un projet d'informatique ; 
		\item un modèle de morphogenèse hybride développé pendant le master 2 avec Arnaud Banos et René Doursat ;
		\item un réseau biologique auto-organisé développé avec Arnaud Banos et Jorge Gonzalez pendant le master 1;
		\item et enfin une analyse des dynamiques territoriales en Ile-de-France que j'ai mené pendant ma thèse.
	\end{itemize}
\end{itemize}


\newpage



\subsection*{Exemple introductif}



\begin{itemize}
	\item Je propose pour introduire le sujet de présenter un exemple concret issu d'observations de terrain.
	\item Le développement récent du réseau à grande vitesse Chinois permet d'observer des interactions effectives entre réseaux et territoires, ici dans le cas particulier des réseaux de transport et du développement urbain.
	\item Je relate ici le cas du Delta de la Rivière des Perles, et plus particulièrement le quartier de Tangjia sur la commune de Zhuhai.
	\item Ces photos donnent un témoignage personnel. Par exemple à gauche, la photo de la devanture de la gare de Tangjia montre une publicité pour la promotion de l'utilisation du train et de la proximité à la gare, ce qui témoignent d'une volonté d'adaptation des pratiques de mobilité quotidienne ou résidentielle.
	\item Sur la photo de droite on observe des quartiers entièrement nouveaux en construction autour des gares. Ces deux phénomènes d'evolution urbaine semblent induits par l'arrivée du nouveau réseau de transport.
	\item Ce cas particulier n'illustre que le haut de l'iceberg de dynamiques complexes d'interactions, puisque l'existence de relations réciproques, à différentes échelles d'espace et de temps et affectant différentes dimensions des objets concernés, semble être la règle.
	\item Je prends dans ma thèse le parti d'étudier cette complexité selon le point de vue particulier de la co-evolution que je vais expliciter par la suite.
\end{itemize}
	% flottant -> supprimé
	%\item L'accessibilité a subi des transformations considérables pour cet exemple de Tangjia, qui représentait une centralité locale au moins depuis le 17è siècle jusqu'à très récemment, avant de se retrouver noyé dans la méga-région urbaine née de la conjonction d'une conjoncture historique, politique, economique, notamment à l'époque de la Doctrine de la Porte Ouverte de Deng Xiaoping.
	% precision de la problematique



\newpage


\subsection*{Problématique}


\begin{itemize}
	\item Je peux à présent préciser la problématique générale de la thèse et la stratégie que j'ai adoptée pour y répondre.
	\item Un certain nombre d'approches, comme celle de la théorie évolutive des villes élaborée par Denise Pumain, et en particulier les travaux d'Anne Bretagnolle dans le cas des réseaux de transport et des villes, postulent des dynamiques fortement couplées entre ces deux objets. Celles-ci sont comprises comme co-évolutives, au sens d'interdépendances réciproques fortes entre les différentes composantes du système.
	\item Mon premier axe de recherche est ainsi de tester la pertinence de ce positionnement, en cherchant à construire une définition précise ainsi qu'une méthode de caractérisation empirique de ces dynamiques co-évolutives.
	\item Pour de nombreuses raisons que je ne peux détailler intégralement ici, mais qui incluent par exemple les spécificités de chaque cas d'étude considéré, des dynamiques qui s'opèrent sur le temps long, des couplages fort entre composantes rendant leur compréhension plus difficile, la connaissance est rapidement limitée.
	\item Je considère alors la modélisation comme un instrument de connaissance à part entière, permettant de tirer des conclusions indirectes sur les processus, de considérer des systèmes génériques, de tester des hypothèses dans des laboratoires virtuels, à différentes échelles et selon différents aspects.
	\item Le deuxième axe de recherche s'articule ainsi autour de la construction de modèles de co-évolution, incluant la détermination d'échelles et ontologies pertinentes pour cette modélisation.
	\item La modélisation est bien un sujet d'étude à part entière, confirmant la dualité de notre travail affirmée précédemment. 
	\item Nous nous inscrivons ainsi dans une lignée scientifique conséquente de modélisation en géographie quantitative particulièrement développée ici à Géographie-cités et à l'institut des systèmes complexes, notamment récemment dans le cadre du projet ERC Geodivercity.
\end{itemize}

\newpage


\subsection*{Cartographie des disciplines}



\begin{itemize}
	\item Afin de comprendre le positionnement et l'enjeu posé par notre problématique, je propose d'abord une synthèse de l'état de l'art des approches pouvant être reliées à la modélisation des interactions entre réseaux et territoires.
	\item J'ai donc mené un travail d'épistémologie quantitative, dont l'un des aspects est une cartographie des connaissances.
	\item Ayant élaboré une méthode qui se base sur un couplage d'analyse de réseaux de citations et de réseaux sémantiques, j'en présente ici une sortie qui est une carte issue de la spatialisation d'un réseau de citations.
	\item On peut observer le positionnement relatif des disciplines, de la géographie politique ainsi que la géographie quantitative et les études des effets structurants à gauche.
	\item La physique est à l'opposée à droite, le pont étant fait par des travaux en économie urbaine, des études d'accessibilité et en transports.
	\item Mon positionnement est donc nécessairement à cheval sur l'ensemble de ces disciplines : mes modèles de systèmes de villes empruntent à la géographie, ceux de morphogenèse à la physique et a l'économie, tandis que les études d'accessibilité jouent un rôle crucial dans les études empiriques que je mène.
	\item De plus, cette étude me permet de confirmer qu'un très faible nombre d'approches se positionne explicitement comme modélisant une co-évolution, et que celles ci sont disparates dans différents domaines.
\end{itemize}



\newpage


\subsection*{Lecture par les domaines de connaissance}

\textit{slide a}

\begin{itemize}
	\item Je propose à présent de détailler une grille de lecture originale des résultats qui seront présentés par la suite.
	\item L'articulation selon différentes dimensions de la connaissance peut en fait se généraliser, et on peut formuler une lecture de la thèse au prisme des \emph{Domaines de connaissance}.
	\item Celui-ci a été initialement introduit par Livet, Sanders et Muller pour la construction de modèles de simulation en sciences sociales, et permet de prendre du recul sur l'organisation des connaissances produites.
	\item On peut voir ici les trois domaines initiaux, théorique (constructions conceptuelles, ici une partie du cadre ontologique utilisé), empirique (connaissances concrètes, ici une étude d'accessibilité pour le delta de la rivières de perles) et de la modélisation (ici l'un des modèles pour les systèmes de villes appliqué à la France).
\end{itemize}

\textit{slide b}

\begin{itemize}
\item Un approfondissement de la lecture par domaines m'a conduit rapidement à intégrer trois domaines supplémentaires, ceux des méthodes, outils et données. J'ai également été amené à contribuer dans ceux-ci.
	\item L'ensemble de ces domaines sont tous en interaction forte dans mon travail. L'objectif de modèles opérationnels justifie certains compromis théoriques par exemple pour la définition des objets.     
	\item Ces compromis s'illustrent aussi pour l'empirique avec des restrictions dues aux données disponibles.
	\item Ainsi, l'objectif de modélisation est atteint par divers compromis et interactions entre les différents domaines.
\item La suite de la présentation suivra ce prisme de lecture et déclinera mon travail selon ces différents axes.
\end{itemize}


\newpage


\subsection*{Positionnement théorique}


\textit{slide a}

\begin{itemize}
	\item Je peux commencer ainsi par détailler les fondations théoriques.
	\item Elles permettent de préciser la définition de la co-évolution des réseaux et des territoires, et d'y suggérer des entrées thématiques pertinentes.
	\item Concernant la construction des objets géographiques étudiés, mon entrée sur les villes et les territoires suit celle de la Théorie evolutive, considérant les villes comme systèmes dans des systèmes de villes, et les territoires comme les étendues spatiales associées.
	\item Au sein des territoires émergent des réseaux par l'intermédiaire de la réalisation de projets transactionnels au sens de la théorie territoriale des réseaux de Gabriel Dupuy.
	\item Reformulé sous cet angle, l'existence et la nature des processus de co-evolution entre réseaux de transport et territoires revient à comprendre les processus précis par lesquels cette materialisation peut s'opérer, et réciproquement dans quelle mesure celle-ci peut influencer les dynamiques territoriales.
\end{itemize}

\newpage

\textit{slide b}

\begin{itemize}
	\item Je construis alors une définition de la co-évolution, basée sur une perspective multi-disciplinaire.
	\item Je propose 3 niveaux possibles auxquels de processus de co-évolution peuvent opérer :
	\begin{itemize}
	    \item celui des agents, individuels ou collectifs, lorsque les caractéristiques d'un petit nombre d'agents s'influencent réciproquement ;
	    \item celui des populations d'agents - qui correspondra au sens originellement introduit par la biologie, supposant l'existence de niches au sein desquelles les caractéristiques de deux populations d'individus sont mutuellement et circulairement liées;
	    \item le dernier niveau systémique correspondant à des interdépendances généralisées entre agents.
	\end{itemize}
\end{itemize}

\newpage

\textit{slide c}

\begin{itemize}	
	\item Trois entrées que j'ai prises correspondent à ces niveaux et donnent des pistes pour répondre à la problématique : 
		\begin{itemize}
			\item des études empiriques ont permis d'apprehender certains aspects du niveau microscopique;
			\item une entrée par la morphogenèse, que l'on peut comprendre comme l'émergence de la forme et de la fonction d'un système, est suggérée par le concept de niche 
			\item une entrée par la théorie évolutive permettant d'appréhender les systèmes au niveau macroscopique.
		\end{itemize}
	\item Cette présentation s'intéressera particulièrement aux deux dernières entrées.
\end{itemize}


\newpage


\subsection*{Méthode de caractérisation}


\begin{itemize}
	\item Elaborant sur ces bases théoriques, celles-ci appellent une vérification empirique, qui implique alors un travail méthodologique de construction d'une méthode de caractérisation.
	\item En effet, la littérature n'en propose pas dans le cas des systèmes territoriaux qui correspondrait à notre définition prise.
	\item Je me positionne à l'intermédiaire entre des méthodes en géographie qui se basent sur des corrélations simples ou des cartographies, et des méthodes en économie utilisant par exemple des variables instrumentales, dont l'application est conditionnée entre autres à une grande disponibilité de données. J'adapte pour cela des méthodes existantes en analyse de séries temporelles.
	\item Le schéma ici décrit le principe de l'application d'une méthode de corrélations spatio-temporelles retardées à des variables caractéristiques des éléments étudiés.
	\item Plus précisément, supposons la donnée de variables de mesure pour les dynamiques des territoires et pour celles des réseaux, qui seront par exemple la croissance de la population et celle de l'accessibilité.
	\item La corrélation entre les séries temporelles correspondantes peut être estimée avec un retard entre les deux séries temporelles. Celle-ci permet de caractériser une relation d'influence dirigée entre les deux variables, qui je le précise est bien une corrélation et non une causalité. On parlera pour simplifier de causalité faible.
	\item Par exemple, la première ligne montre une corrélation toujours nulle en fonction du retard $\tau$ en abscisse, et donc pas de relation entre les variables. La deuxième et la troisième ligne un maximum pour un retard positif ou négatif, et donc un sens de causalité faible entre les variables. La dernière nous intéresse particulièrement, puisque une causalité réciproque s'observe, ce qui correspondra à une co-évolution au sens statistique de la population que nous avons défini.
	\item La dernière colonne donne l'interprétation de ces relations sous forme d'un graphe entre variables.
\end{itemize}



\newpage

\subsection*{Observations empiriques}


\begin{itemize}
	\item L'application de cette méthode à différents cas d'étude, à différentes échelles temporelles et spatiales, donne des résultats contrastés.
	\item Je présente ici une étude réalisée en collaboration avec Solène Baffi, pour le cas de l'Afrique du Sud au 20ème siècle.
	\item Nous considérons les croissances de l'accessibilité ferroviaire, au sens d'une population accessible à une portée spatiale donnée, et de la population des villes, sur le temps long entre 1930 et 2000.
	\item On peut voir ici les graphes de correlations retardées en fonction du retard, sur differentes fenetres temporelles successives, la couleur déclinant différentes portées spatiales. Les barres d'erreur permettent la prise en compte de la significativité statistique.
	\item Le cas de l'Afrique du Sud est particulier, puisque la mise en place des politiques de segregation de l'apertheid en 1950 ont conduit à des fortes restructurations du systèmes urbain et de transport. 
	\item On peut observer ici un effet statistiquement significatif d'inversion du sens de la causalité faible au cours du temps : le maximum de corrélation pour les longues portees, à un retards négatif avant 1950 passe à un retard positif après 1950, passant d'une dynamique où l'accessibilité induit la population à l'inverse.
	\item Cet effet est en coherence avec les logiques documentées dans la literature de relocalisations forcees et de reseau ferroviaire établi pour maintenir cette segregation.
	\item Meme si elle reste partielle, cette etude suggère donc un impact des politiques de ségrégations non seulement sur les dynamiques territoriales, ce qui était deja connu, mais aussi sur leur relations avec celles des réseaux, en quelque sorte un effet au second ordre.
\end{itemize}


\newpage


\subsection*{Aperçu des contributions en modélisation}

\textit{slide a}

\begin{itemize}
	\item Je peux à présent m'attaquer à mes contributions sur le plan de la modélisation, dont voici un aperçu global.
	\item Ma première ligne de contributions se situe sur le plan des modèles macroscopiques, dans la lignées des modèles Simpop développés dans le cadre de la théorie évolutive des villes.
	\item J'ai introduit une famille de modèles inédite prenant en compte explicitement les processus de co-évolution entre croissance des villes et croissance des réseaux de transport.
	\item Son application à réseau statique permet deja de révéler des effets de réseau par une amélioration de l'ajustement sur le système de villes français.
	\item J'ai par ailleurs montré, par l'application de la méthode de caractérisation empirique, que ces modèles étaient capable de produire une très grande variété de regimes d'interaction, incluant des regimes co-évolutifs.
\end{itemize}


%\newpage

\textit{slide b}

\begin{itemize}
	\item Sur le plan mesoscopique, un premier modele de morphogenèse permet le couplage fort entre evolution de la forme urbaine et croissance du réseau.
	\item Celui-ci se place dans une perspective de multi-modélisation.
	\item Les différents processus inclus se montrent complémentaires pour couvrir l'ensemble de l'espace morphologique généré, ainsi que pour sa calibration sur données réelles.
	\item Une deuxième modele, le modèle Lutecia, initialement introduit par Florent Le Nechet dans sa these en 2010, a été étendu et exploré, afin de comprendre le role des processus de gouvernance pour la croissance du réseau de transport.
\end{itemize}

\textit{slide c}

\begin{itemize}
	\item Je vais à présent détailler plus particulièrement les idées sous-jacentes aux modèles macroscopiques d'interaction.
\end{itemize}


\newpage



\subsection*{Modélisation macro}


\textit{slide a (splitted)}

\begin{itemize}
	\item Je rappelle que les modèles macroscopiques que j'ai introduit se basent sur les paradigmes de modèles d'interaction au sein des systèmes de villes, dans la lignée des modèles Favaro-Pumain ainsi que Marius dévelopé par Clémentine Cottineau. A l'intermédiaire entre un modèle agent et un modèle discret de système dynamique, ils permettent de capturer la complexité via la non-linéarité, en moyennant les effets stochastiques.
	\item On peut considérer les villes et leur populations, localisées dans l'espace.
	\item On considére également le réseau de transport entre celles-ci. Dans cette specification du modele, il n'est pas permis d'evolution topologique mais une evolution des vitesses des liens. 
	\item On a pour l'evolution des populations un terme de croissance endogène correspondant au modèle de Gibrat. Ce modele stipule que les taux de croissance des villes ont une distribution aléatoire indépendante de la taille des villes.
	\item On peut ensuite ajouter un terme de croissance due aux interactions directes, qui correspond a un modele gravitaire. Il s'agit d'une traduction abstraite de flux migratoires entre villes par exemple.
	\item Enfin, un terme de retroactions des flux permet d'inclure un effet du réseau sur la croissance des villes.
	\item Le réseau enfin peut quant à lui croitre en fonction des flux, selon différentes spécifications, comme par exemple une fonction de renforcement : plus le flux dans le lien a été important, plus l'evolution de la vitesse sera grande. Ce renforcement peut être conditionné à un seuil.
\end{itemize}
% 	\item Une première application est faite dans le cas d'un réseau statique, c'est à dire sans co-évolution à proprement parler mais avec des effets de réseau via l'effet des flux sur la croissance des villes. Nous obtenons une amélioration du pouvoir explicatif du modèle pour les données de population françaises (1830-1999), après avoir controlé le suraprentissage par l'application d'une méthode développée spécifiquement pour les modèles de simulation.


\newpage

\textit{slide b}
	
\begin{itemize}
	\item L'application d'une version co-evolutive du modèle à des systèmes de villes synthétiques révèles de nombreuses configurations en termes d'interactions entre variables de territoires et variables de réseau, au sens de la caractérisation empirique par corrélations retardées.
	\item On obtient notamment un nombre important de régimes aux causalités circulaires correspondant à une co-evolution.
	\item Ces graphes montrent les correlations retardées en fonction du retard, pour chaque couple de variable en couleur parmi accessibilité, population et centralité de proximité.
	\item Chaque graphe présente un regime particulier, codé en cartouche par la structure du graphe relationnel entre les variables.
	\item Considérons par exemple le second.
	\item Les forts effets d'autocorrelation induisent une correlation à l'origine non nulle, et la relation entre variable se lit par rapport à celle-ci.
	\item Pour la centralité et l'accessibilité, on a uniquement une correlation plus forte pour $\tau > 0$, et donc la premiere cause la deuxième.
	\item Pour la population et l'accessibilité, l'accessibilité cause la population.
	\item Enfin, le dernier couple montre une relation circulaire (double flèche) avec l'existence de deux maxima, et témoigne d'une co-evolution.
	\item Les autres regimes ici peuvent être interprétés de la même façon, et j'insiste sur le fait que leur diversité est une résultat important produit par le modèle.
	\item Ainsi, cet exemple a illustré une façon de modéliser la co-evolution entre réseaux de transport et territoires, et quel type de resultats on peut espérer en tirer.
\end{itemize}



\newpage

\subsection*{Mise en perspective}

\begin{itemize}
	\item Avant de conclure sur les perspectives ouvertes, je souhaiterais remettre en contexte le coeur de la thèse au regard de son cadre plus global, contenu notamment dans les annexes, qui permettent une articulation plus générale.
	\item Le travail de définition de la co-évolution, qui a requis la mise en place d'un certain niveau de reflexivité, d'approches interdisciplinaires, et de méthodes et d'outils d'analyse d'épistemologie quantitative, s'inscrit au sein de deux axes de travail en épistémologie et épistémologie quantitative.
	\item La caractérisation peut être associée à des travaux thématiques plus larges, qui permettent d'appréhender indirectement d'autres aspects des relations entre territoires et réseaux.
	\item Enfin, la modélisation s'inscrit au sein de l'élaboration de cadres systémiques, à différents niveaux d'abstraction, incluant le cadre de connaissance qui a guidé notre lecture ici.
\end{itemize}

% TODO : typo cadre connaissance num section dans figure

\newpage

\subsection*{Ouvertures}

\textit{slide a}

\begin{itemize}
	\item Les perspectives de recherche ouvertes par ce travail sont diverses, à la fois en ce qui concerne les modèles introduits en eux-même et la modélisation de la co-évolution, mais aussi des perspectives plus larges.
	\item Je rappellerai tout d'abord quelles sont mes contributions principales.
	\item La définition que j'ai proposé de la co-évolution dans le cadre des systèmes territoriaux devrait permettre une relecture de la théorie évolutive, en particulier au prisme de la co-évolution des réseaux de transport et des territoires. Elle ouvre aussi des potentialités de ponts avec l'économie géographique, puisque celle-ci e été mobilisée pour sa construction.
	\item La méthode de caractérisation, malgré ses limites, est bien opérationelle et parcimonieuse en termes de données requises, et également très générique. Elle devrait pouvoir être appliquee dans divers cas en géographie et sciences territoriales.
	\item Enfin, les modèles introduits pourront être réutilisés par diverses disciplines, puisque ceux-ci ont été voulu interdisciplinaires, et sont pensés pour être couples et réutilisés.
\end{itemize}

\newpage

\textit{slide b}

\begin{itemize}
	\item Je me permets enfin de mentionner des développements déjà en cours ou des perspectives plus lointaines.
	\item Dans le cadre de développement de méthodes génériques d'exploration de modèles spatiaux, ici à l'ISC avec l'équipe d'OpenMole que je remercie d'ailleurs de m'avoir accueilli, une partie du modèle lutecia est utilisée comme modèle jouet de test. 
	\item Ceci nécessite un redeveloppement du modèle dans un autre langage, rappellant les problématiques de passage à l'échelle pour l'utilisation de calcul intensif : en effet, malgré les environ 200 ans de calcul utilisés pour la thèse, les explorations sont restées modestes et simples concernant les méthodes appliquées.
	\item Maintenant du côté du développement des méthodes, ce modèle est une excellente opportunité, autant un moyen qu'une fin pour poursuivre la recherche dans ce domaine.
	\item Je propose finalement de formuler un projet à long terme qui s'oriente vers la construction de théories intégratives pour les systèmes territoriaux.
	\item Integrative au sens de la feuille de route des systèmes complexes c'est-à-dire à la fois horizontalement (répondant à des questions transversales) et verticalement (c'est-à-dire multi-échelles).
	\item Une piste pour la construction de telles théories serait le couplage de la théorie evolutive à celle du scaling, par l'intermédiaire du couplage des modèles développés ici, par exemple le couplage du modèle d'interaction au sein des systèmes de villes avec le modèle de morphogenèse.
\end{itemize}
% \item Une comparaison systématique des modèles de croissance urbaines, incluant notre modele de coevolution, mais aussi le modele Favaro-Pumain et le Marius, sur les multiples systèmes de villes dans le monde pour lesquels les donnees standardisees sont issues de l'erc geodivercity, est un developpement important pour la comprehension des processus de croissance urbaine.
% 	\item Le premier projet, celui de mon postdoc en cours, s'attelle au développement de nouvelles méthodes d'exploration des modèles de simulation spatiaux. Trois axes complémentaires concernent la sensibilité aux meta-parametres spatiaux obtenue par l'intermédiaire de la generation de données synthetiques spatiales, la robustesse des methodes d'exploration et de calibration par algorithmes genetiques, notamment à la stochasticité, et la multi-modelisation.
% \item Une reflexion sur le role de l'intelligence artificielle au sein des systemes territoriaux suggere une intergation horizontale, tandis qu'une integration supplementaire, celle des domaines de connaissance, serait permise par le reflexivite qui découle des travaux en épistémologie quantitative.

\bigskip

\begin{itemize}
    \item Je concluerai simplement et modestement, et de la même manière que le mémoire.
	\item Cela permettra de rappeler et d'appuyer l'importance de la science ouverte que j'ai pratique et défendue tout au long de mon travail.
	\item Je suivrai ainsi l'adage de framasoft selon lequel \textit{la route est longue mais la voie est libre}.
\end{itemize}


\bigskip

\textbf{
Je remercie le jury de m'avoir écouté et lui soumets ainsi ce travail en candidature au diplome de docteur en géographie de l'université Paris Diderot.
}










%\subsection*{Modélisation meso}


% - complementarité de multiples heuristiques de croissance de reseau
% - calibration au premier et second ordre
% - Lutecia : vers des modèles plus complexes

%\textit{slide a}

%\begin{itemize}
%	\item A l'échelle mesoscopique, notre paradigme principal de modélisation est la morphogenèse, que nous interprétons comme l'existence d'une architecture émergente correspondant à un couplage fort entre forme et fonction. Le réseau permet justement d'intégrer des propriétés fonctionnelles de l'espace urbain dans les variables locales.
%	\item Un premier modèle de co-évolution de base sur les processus de réaction-diffusion, qui traduisent abstraitement les effets d'agglomération et de congestion au sein des systèmes urbains. Ceux-ci sont régit par un potentiel local qui dépend à la fois des variables morphologique et des variables fonctionelles, incluses dans notre cas par les différentes centralités dans le réseau. Le couplage fort est bouclé par une croissance du réseau s'adpatant à l'évolution de la population, suivant une multi-modélisation de sa croissance par heuristiques complémentaires (incluant ruptures de potentiel deterministe et aleatoire, cout-benefices, croissance de reseau biologique). Ces différentes heuristiques s'averent complementaires pour produire l'espace topologique faisable en termes de configurations de réseau produites. Par ailleurs, le calcul des indicateurs morphologique, de réseau et leur corrélations sur données réelles pour l'ensemble de l'Europe sur fenetre spatiales de taille comparable, permet de calibrer le modèle. Un compromis entre premier ordre (valeur des indicateurs) et second ordre (distance des matrices de correlation) est obtenu, montrant que le modele est capable dans une certaine mesure de capturer simultanement des formes et leur interactions.
%\end{itemize}

%\textit{slide b}

%\begin{itemize}
%	\item Un dernier modèle, le modèle Lutecia, a été introduit afin d'explorer le role des processus de gouvernance dans la croissance du reseau de transport. A une échelle métropolitaine, un modèle Luti stylisé (extension du modèle de Lowry) est couplé à un réseau d'infrastructure dont la croissance est gérée par des acteurs politiques cherchant à maximiser l'accessibilité des zones dont ils ont le contrôle. L'utilisation de theorie des jeux permet de determiner des probabilités de collaboration entre acteurs pour la construction de nouvelles infrastructures. Nous obtenons par exemple differentes structures de reseau emergentes correspondant à différents niveau de collaboration.
%	%Une calibration sur le Delta de la riviere des perles suggere. % - pas besoin
%	\item Ainsi, nous avons modélisé différents aspects de la co-évolution des réseaux de transport et des territoires, à différentes échelles.
%\end{itemize}







%%%%%%%%%%%%%%%%%%%%
%% Biblio
%%%%%%%%%%%%%%%%%%%%
%\tiny

%\begin{multicols}{2}

%\setstretch{0.3}
%\setlength{\parskip}{-0.4em}


%\bibliographystyle{apalike}
%\bibliography{/Users/juste/ComplexSystems/CityNetwork/Biblio/Bibtex/CityNetwork}%,biblio}
%\end{multicols}



\end{document}

