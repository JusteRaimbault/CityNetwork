\documentclass[11pt]{article}

% general packages without options
\usepackage{amsmath,amssymb,bbm}

\usepackage[margin=1cm]{geometry}

\usepackage[T1]{fontenc}
\usepackage[utf8]{inputenc}


\begin{document}


\section*{Questions possibles}

\begin{itemize}
	\item On parle du micro dans la definition ; qu'en est-il dans l'empirique/les modèles ?
	\item Comment est fait/peut être fait le lien entre theorie evolutive et morphogenese ? $\rightarrow$ \textit{ebauche d'une theorie integrative. Vers des theories integratives plus generales ? cf projet}
	\item Qu'est ce qui n'est pas quantifiable / positionnement debat quanti/quali
	\item Multilayer network / multimodal dans les modeles de coevoloution ?
	\item TGV ? (question a cote de la plaque a spotter)
	\item Vision complexe systemique : telonomie des groupes humaines ? $\rightarrow$ \textit{Multi-objectif}
	\item Cas empiriques non comparables ? possibilite d'une typologie ?
	\item idee : baseline : gravity only with many function / learning sur la fonction ? Question : forme de la fonction de gravité
\end{itemize}






\section*{Retours presoutenances}


\subsection*{LVMT}

\subsubsection*{Général}

\begin{itemize}
	\item Niveau de discours trop haut, recentrer, lpus de details techniques, commentaires plus concretes
	\item attention au voc jargonneux
	\item echanger parcours / exemple (idem geocites)
	\item expliciter apports
	\item domaines de connasisance a resituer
	\item plus de contexte sci ?
	\item slide resume grands resultats ? idem liste contributions / modeles
	\item synthetiser l'info
	\item plus positionnement par rapport a literature
	\item raccrocher exemplpe intro au cours pres ?
	\item prise distance, defense methode a part entiere
	\item domaines connaissance : risque, ok si restitution plus simple des resultats
\end{itemize}


\subsubsection*{Spécifique}

\begin{itemize}
	\item preciser que def coevol issue revue interdisc. reformuler individus / population ? niveau $\neq$ echelle
	\item expliciter les echelles
	\item parcours : theorique / empirique
	\item preciser que causalite faible
	\item afrique du sud : explpiciter apartheid
	\item macro : expliciter que mieux que plpus de reigmes
	\item modele meso : expliciter que coupplage difficile
\end{itemize}



\subsection*{Géocités}

\subsubsection*{Général}

\begin{itemize}
	\item rythme parole / duree
	\item trop exemples
	\item termes pas definis
	\item melange des langages : dangereux
	\item deux trois idees fortes en conclusion
	\item domaines connaissance a la fin ? non
	\item trop figures
	\item entites proliferent
	\item faire choix, pas parler de tout
	\item quest ce que le macro/mrso ppermettent de voir ?
	\item ajouter histoire model. dans laquelle s'insere ?
	\item Elaguer ; paysage pplpus clair, mieux developper
	\item attention a l'effet club
\end{itemize}

\subsubsection*{Spécifique}


\begin{itemize}
	\item Afrique du sud : pas donner l'impression que decontextualise
	\item inserer epistemo sur causalites
\end{itemize}





\subsection*{ISC}

\subsubsection*{Général}

\begin{itemize}
	\item Domaines de connaissance : pas specifiquement original ; mieux poser positionnement ; les remettre au debut
	\item Trop long : debit hardcore ; supprimer du texte, elaguer
	\item Dire plus explicitement les contributions (cadre au debut - avec domaines de connaissance)
	\item Selectionner les exemples, moins general description ; un modele plus detaille
	\item Presenter un seul grapphe, tout detailler axes etc.
	\item Schema de positionnement, accompagner la lecture, elements saillants
	\item Resumer en une phrase ? (conclusion ?)
	\item plus detailler un seul exemple d'appplication / un modele.
	\item Preciser qu'on s'interesse pas a micro ?
\end{itemize}



\subsubsection*{Spécifique}

\begin{itemize}
	\item Photos de terrain : etre plus precis
	\item reifier le rapport caracterisation / modele : un graphe coevol macro.
	\item AIC : incomprehensible
\end{itemize}





%%%%%%%%%%%%%%%%%%%%
%% Biblio
%%%%%%%%%%%%%%%%%%%%
%\tiny

%\begin{multicols}{2}

%\setstretch{0.3}
%\setlength{\parskip}{-0.4em}


%\bibliographystyle{apalike}
%\bibliography{/Users/juste/ComplexSystems/CityNetwork/Biblio/Bibtex/CityNetwork}%,biblio}
%\end{multicols}



\end{document}

