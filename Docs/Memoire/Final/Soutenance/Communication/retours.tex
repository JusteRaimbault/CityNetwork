\documentclass[11pt]{article}

% general packages without options
\usepackage{amsmath,amssymb,bbm}

\usepackage[margin=1cm]{geometry}

\usepackage[T1]{fontenc}
\usepackage[utf8]{inputenc}


\begin{document}


\section*{Questions possibles}


\subsection*{Rapporteurs}


\subsubsection*{Josselin}

\begin{itemize}
	\item Dans la meta-analyse : pourquoi ces techniques de regression ?
	\item concevoir, manipuler et generer modele spatiaux de maniere ouverte et exploratoire
	\item Sur epistemo : emergence - de formes ? pas que en fait
	\item SPatioTempCausality : terminologie ? (``regimes'') epistemo causalite/correlation ? methodo autres methodes ?
	\item Agregation-diffusion : proche automate cellulaire ? oui et non, oui si croissance supposée endogene.
	\item COnfig territoriales correlees : sens des correlations dans ce couplage statique ? peut on faire un lien ? $\rightarrow$ souleve questions d'ergodicité !
	\item effet du TGV : comment le voit-on ? $\rightarrow$ pic secondaire, pour pop uniquement.
	\item embrouillamini entre le mesocoevol et le lutecia. Logique de linearite progressive dans la complexite et enchainement ? theses classiques ?
	\item Concept : non-stationarite
	\item Concept : echelles spatiales
	\item Reseaux ``a l'origine de changement de regimes'' sens ?
	\item interet du cote science ouverte (// Morency)
	\item poeme : pourquoi ?
	\item defricheur de nouveau champ de la geographie quantitative : approche generale a defendre
	\item difficulte maths, indicateurs, illustrations
	\item statut des annexes (integration texte principal) $\rightarrow$ ne pas alourdir car deja consequent ; duexieme these ou meta-these ?
	\item positionnement de these par articles
	\item approche exploratoire pronee
\end{itemize}

\textbf{Questions explictes :}

\begin{itemize}
	\item verrous de la geographie theorique et quantitative $\rightarrow$ towards integrated geography ? mais quest ce qui bloque ? methodes, complexite des systemes, particularite et generalite. \textit{synthese ?} systeme complexes spatiaux. cf fondements.
	\item robustesse des modeles spatiaux $\rightarrow$ cf postdoc
	\item stationnarite processus spatio-temporels $\rightarrow$ plus etude empirique, et prise en compte dans modeles. lien GWR et modeles de simulation. Starma. Spatio-temporal datamining. Ergodicite etc.
	\item apprehension des echelles $\rightarrow$ echelles, meta-echelles (niveau) ?
	\item complexite des interactions spatiales $\rightarrow$ dans quelle mesure l'espace rend plus complexe ? bon exemple est le luti stylise. et cf mesocoevol
	\item place de la geographie dans l'interdiscplinarité : cf débat Barthelemy
	\item etc. : ? (concepts du manuscrit)
\end{itemize}


\subsubsection*{Morency}


\begin{itemize}
	\item trop d'elements intermédiaires de la demarche
	\item objectifs spécifiques ; mode sequentiel et reecriture ? pourquoi pas avoir exposé le 8 des le debut ?
	\item dispersé vs ambitieux : a defendre !
	\item synthese des processus (tableau 4) interessante pour identifier limites actuelles
	\item briques sont en fait les fondements ?
	\item Interpretation des indicateurs morphologiques (pas seulement usabilité)
\end{itemize}


\textbf{Questions explicites : }

\begin{itemize}
	\item possible et pertinent d'avoir d'autres metriques de distance (routier ou TC) pour indics morpho, impact sur resultats ? $\rightarrow$ resultats access tres correlee closeness suggere peu impact ? a reflechir car pas regarde moran par ex.
	\item classif indics : illsutrer le vecteur de données ? ; signification des composantes ?
	\item pourquoi Voronoi (heteorgene) et pas zone influecne homogene.
	\item indics reseaux : interpretation valeurs ; differences en valeurs ?
	\item donnees synthetiques : trop hors-sol, devrait soutenir dvlpmt modeles particuliers ; meilleure synthese apprentissages ou associer ces experimentations a cas d'exemple.
	\item modele density : taille de la grille (100x100 ou 1kmx1km ?) defi associés à l'utilisation de cette echelle ? cf scaling en conclusion. (mise a l'echelle ?)
	\item attention macro : pas systeme par grille.
	\item autres options que reseau ferre ?
	\item transferabilte spatiale (sur d'autres cas d'etude)
	\item Pourquoi ne pas migrer au micro ? $\rightarrow$ cf question Paul
	\item equilibre chapitres, dommage pas plus sur la coevol.
	\item limitations de la demarche de recherche ? synthese OK.
	\item pertinence ouverture (plusieurs fois) $\rightarrow$ cf noosphere Morin pour justifier que nos idees ne nous appartiennent pas ?
	\item attention style romancé
	\item toutefois piece de reference
\end{itemize}

remarques techniques a regler version finale. trad figures ?




\subsection*{Questions générales}


\begin{itemize}
	\item On parle du micro dans la definition ; qu'en est-il dans l'empirique/les modèles ?
	\item Comment est fait/peut être fait le lien entre theorie evolutive et morphogenese ? $\rightarrow$ \textit{ebauche d'une theorie integrative. Vers des theories integratives plus generales ? cf projet}
	\item Qu'est ce qui n'est pas quantifiable / positionnement debat quanti/quali
	\item Multilayer network / multimodal dans les modeles de coevoloution ?
	\item TGV ? (question a cote de la plaque a spotter)
	\item Vision complexe systemique : telonomie des groupes humaines ? $\rightarrow$ \textit{Multi-objectif}
	\item Cas empiriques non comparables ? possibilite d'une typologie ?
	\item idee : baseline : gravity only with many function / learning sur la fonction ? Question : forme de la fonction de gravité
	\item Sur la notion de niche / transfert disciplinaire / difference niveau et echelle / influecnec socio disciplinaire (remarque Thomas positionnement systemes complexes) - interdisc vraiment utile ? \textit{le niveau serait echelle ontologique ? correspopndence claire est une question ouverte}
	\item quest ce quon etend par donnees synthetiques ?
	\item Difference entre epistemo et epistemo quanti ?
	\item morphogenesis, morphogenetic engineering. pas assez point de vue de René ? (retour question niche, theorie globale etc).
	\item theorie integrative justifiee par transversalite, et presence du micro ?
	\item multi-echelle : pourquoi / comment - qui en fait vraiment auj ?
	\item evol topologique ou non dans le macrocoevol : difference fondamentale avec le simpopnet d'ou les differences dans les regimes ? on a un peu force l'effet stat de part l'evolution pour l'ensemble des liens ? meriterait des tests empiriaues cibles. train ; mentionner les autoroutes.
	
\end{itemize}


%\subsection*{Questions ciblées}


\subsection*{Divers}

\begin{itemize}
	\item cadre academique / admin ? (cf USPC) anarchisme sceintifique, cf paragraphe Feyerabend !
	\item etudes micro empiriques : place ? etre franc sur les debats sur le plan et la dissection progressive : d'abord echelle/ontologies, puis ouverture empirique, puis explosé.
\end{itemize}








\end{document}


\section*{Retours presoutenances}


\subsection*{LVMT}

\subsubsection*{Général}

\begin{itemize}
	\item Niveau de discours trop haut, recentrer, lpus de details techniques, commentaires plus concretes
	\item attention au voc jargonneux
	\item echanger parcours / exemple (idem geocites)
	\item expliciter apports
	\item domaines de connasisance a resituer
	\item plus de contexte sci ?
	\item slide resume grands resultats ? idem liste contributions / modeles
	\item synthetiser l'info
	\item plus positionnement par rapport a literature
	\item raccrocher exemplpe intro au cours pres ?
	\item prise distance, defense methode a part entiere
	\item domaines connaissance : risque, ok si restitution plus simple des resultats
\end{itemize}


\subsubsection*{Spécifique}

\begin{itemize}
	\item preciser que def coevol issue revue interdisc. reformuler individus / population ? niveau $\neq$ echelle
	\item expliciter les echelles
	\item parcours : theorique / empirique
	\item preciser que causalite faible
	\item afrique du sud : explpiciter apartheid
	\item macro : expliciter que mieux que plpus de reigmes
	\item modele meso : expliciter que coupplage difficile
\end{itemize}



\subsection*{Géocités}

\subsubsection*{Général}

\begin{itemize}
	\item rythme parole / duree
	\item trop exemples
	\item termes pas definis
	\item melange des langages : dangereux
	\item deux trois idees fortes en conclusion
	\item domaines connaissance a la fin ? non
	\item trop figures
	\item entites proliferent
	\item faire choix, pas parler de tout
	\item quest ce que le macro/mrso ppermettent de voir ?
	\item ajouter histoire model. dans laquelle s'insere ?
	\item Elaguer ; paysage pplpus clair, mieux developper
	\item attention a l'effet club
\end{itemize}

\subsubsection*{Spécifique}


\begin{itemize}
	\item Afrique du sud : pas donner l'impression que decontextualise
	\item inserer epistemo sur causalites
\end{itemize}





\subsection*{ISC}

\subsubsection*{Général}

\begin{itemize}
	\item Domaines de connaissance : pas specifiquement original ; mieux poser positionnement ; les remettre au debut
	\item Trop long : debit hardcore ; supprimer du texte, elaguer
	\item Dire plus explicitement les contributions (cadre au debut - avec domaines de connaissance)
	\item Selectionner les exemples, moins general description ; un modele plus detaille
	\item Presenter un seul grapphe, tout detailler axes etc.
	\item Schema de positionnement, accompagner la lecture, elements saillants
	\item Resumer en une phrase ? (conclusion ?)
	\item plus detailler un seul exemple d'appplication / un modele.
	\item Preciser qu'on s'interesse pas a micro ?
\end{itemize}



\subsubsection*{Spécifique}

\begin{itemize}
	\item Photos de terrain : etre plus precis
	\item reifier le rapport caracterisation / modele : un graphe coevol macro.
	\item AIC : incomprehensible
\end{itemize}





%%%%%%%%%%%%%%%%%%%%
%% Biblio
%%%%%%%%%%%%%%%%%%%%
%\tiny

%\begin{multicols}{2}

%\setstretch{0.3}
%\setlength{\parskip}{-0.4em}


%\bibliographystyle{apalike}
%\bibliography{/Users/juste/ComplexSystems/CityNetwork/Biblio/Bibtex/CityNetwork}%,biblio}
%\end{multicols}



\end{document}

