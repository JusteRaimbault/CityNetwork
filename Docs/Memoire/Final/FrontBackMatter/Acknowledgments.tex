% Acknowledgements

%\pdfbookmark[1]{Acknowledgements}{Acknowledgements} % Bookmark name visible in a PDF viewer
\pdfbookmark[1]{Remerciements}{Remerciements}

%\begin{flushright}{\slshape    
%We have seen that computer programming is an art, \\ 
%because it applies accumulated knowledge to the world, \\ 
%because it requires skill and ingenuity, and especially \\
%because it produces objects of beauty.} \\ \medskip
%--- \defcitealias{knuth:1974}{Donald E. Knuth}\citetalias{knuth:1974} \citep{knuth:1974}
%\end{flushright}

%\bigskip

%----------------------------------------------------------------------------------------

\begingroup

\let\clearpage\relax
\let\cleardoublepage\relax
\let\cleardoublepage\relax

%\chapter*{Acknowledgements}
\chapter*{Remerciements}


% \textit{La forme et la fonction se produisent mutuellement de manière complexe. Les remerciements ne feraient-ils finalement pas partie du fond scientifique ? }

% grille

Une grande partie des résultats obtenus dans cette thèse ont été calculés sur l'organisation virtuelle vo.complex-system.eu de l'European Grid Infrastructure (http://www.egi.eu). Nous remercions l'European Grid Infrastructure et ses National Grid Initiatives (France-Grilles en particulier) pour fournir le support technique et l'infrastructure. \\

% medium

Ce travail de recherche a été mené dans le cadre du project MEDIUM (New pathways for sustainable urban development in China’s MEDIUM sized-cities). Nous souhaitons remercier le Centre National de la Recherche Scientifique (CNRS) et l’UMR 8504 Géographie-cités pour leurs soutiens ainsi que les partenaires de MEDIUM, en particulier la Sun-Yat-Sen University. Le projet MEDIUM a été cofinancé par l’Union européenne au titre de l’Action Extérieure de l’UE – Contrat de subvention ICI+/2014/348-005.



% jury




% directeurs

% Ma profonde reconnaissance va naturellement en très grande partie à mes directeurs, qui ont rendu cette aventure possible et ont permis sa forme finale, par un pilotage subtil du système complexe que formait mes objets, mes modèles, mes idées. J'ai rencontré pour la première fois Arnaud Banos en octobre 2012 à l'ISC qu'il dirigeait, alors toujours rue Lhomond. C'était dans le cadre d'une supervision des \emph{Open Problems} du PA Systèmes Complexes, et nous nous étions immergés avec mon collègue Jorge dans le monde du multi-échelle, de l'optimisation multi-objectif, des réseaux biologiques auto-organisés (projet dont l'implémentation originale a d'ailleurs été reprise ici). Ou plutôt jetés inconsciemment à l'eau au risque de se noyer, merci à Arnaud de nous avoir repêchés. Je garde un certain nombre de paradigmes fondamentaux qu'il nous avait transmis dès le premier contact avec la recherche. Cette bifurcation coïncide étrangement avec une autre plus personnelle, peut être ironiquement pour rappeler la place du sujet dont l'objectivité de la recherche ne fait aucun sens.

%Ma première rencontre avec Florent Le Néchet a eu lieu en mars 2014, à la cafétéria des Ponts, pour discuter de ce projet de thèse. Naïvement, je lui présentais mon modèle RBD ainsi que des idées floues sur les ruptures de potentiel. Il a alors immédiatement donné de la profondeur au projet, en évoquant les Mega-city Regions, les nouveaux régimes urbains, la Chine : vision finalement prémonitoire (ou prophétie auto-réalisatrice ?). La richesse de ses idées n'a cessé d'irriguer ce travail mais aussi mes reflexions de manière plus générale. Sans lui, cette thèse n'aurait de géographie que le nom, et je lui suis fortement reconnaissant d'avoir été patient devant mes difficultés à appréhender les Sciences Humaines et Sociales.


% master systemes complexes

% Je tiens également à remercier Paul Bourgine et Kashayar Pakdaman


% formation des ponts



% X / graduate school X



% sficsss

% Je remercie l'ensemble de l'équipe pédagogique et des participants de la SFI Complex Systems Summer School 2016









%%%%%%%%%%%
%% Invitation soutenance

% -> positionnement scientifique / impact potentiel

% personnalités scientifiques
% - A. Bonnafous
% - F. Laurent
% - N. Aveline
% - C. Rozenblat
% - L. D'Acci
% - D. Chavalarias
% - F. Varenne
% - F. Pfaender
% - D. Badariotti
% - L. Sanders
% - M. Barthelemy
% - J.P. Marchand
% - F. Durand-Dastès
% - P. Frankhauser
% - N. Coulombel
% - C. Gallez


% SFICSS
% - European circ eco crew : Joris, Marius, Mario.
% - Matteo
% - Jelena

% LVMT
% -> annonce interne

% Geocités
% -> annonce interne

% ISC
% -> annonce interne

% EMCSS
% - Kashayar Pakdaman, René Doursat
% - anciens du master cs (via Mme Taki ?)
% - Carantino
% - Jorge

% CSSS2013
% - Claire, Nico


% X
% - Heliou(s), Buisson


% Ponts
% - Camille

% Medium
% - collègues Medium si en Europe
% - Elfie, Medhi

% Coauteurs
% (rq : // diagramme Clem : overlap avec potes etc)
% - Marion, Romain, Clem
% - Solène

% eleves
% ?


% potes
% - Anto, Yo (si en fr), Max, Arnaud
% - SFR, Nihal
% - Axel
% - Hélène (et parents)
% - Cinzia
% - Solène, Charline


% autres
% - Hervé












\endgroup