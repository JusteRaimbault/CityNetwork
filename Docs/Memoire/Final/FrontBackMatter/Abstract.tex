% Abstract

%\renewcommand{\abstractname}{Abstract} % Uncomment to change the name of the abstract

\pdfbookmark[1]{Abstract}{Abstract} % Bookmark name visible in a PDF viewer

\begingroup
\let\clearpage\relax
\let\cleardoublepage\relax
\let\cleardoublepage\relax

\chapter*{Abstract}{Résumé}
%Short summary of the contents\dots a great guide by 
%Kent Beck how to write good abstracts can be found here:  
%\begin{center}
%\url{https://plg.uwaterloo.ca/~migod/research/beckOOPSLA.html}
%\end{center}


\bpar{
Territorial systems exhibit complexity at any levels and for most of their aspects. Related disciplines \comment{(Florent) geography, planning, socio, economy}
generally embrace complex systems science approaches to tackle their understanding and the associated dramatic \comment{(Florent)un peu fort ?}
social and environmental issues. Choosing a specific angle of lecture of territories, \comment{(Florent)trop vite ds problématique, développer d'abord aspect dynamique}, idem
it appears, following territorial theories of networks, that real networks play a crucial role in system dynamics, and in particular transportation networks. Taking furthermore a modeling paradigm, we ask to what extent a modeling approach to territorial systems as networked human territories can help disentangling complexly involved processes. We propose to build an associated theory, relying on a vision of human territories as networked, combined with the evolutive urban theory and insights from morphogenesis and co-evolution, that we call a \emph{theory of co-evolutive networked territorial systems}. \comment{(Florent) pas clair, pas forcément nommer}
 It is then embedded into a more general epistemological framework insisting on the notions of emergence and modularity. Quantitative epistemological analysis \comment{(Florent)trop précis}
 confirm the manual literature review and guide research towards co-evolutive models of networks and territories. We search for stylized facts in empirical datasets to also guide model construction. Methodological developments allow to expect information on dynamical processes from static correlations between urban morphology and network shape. The first modeling experiments include a calibrated spatial model of urban growth, giving an insight into theoretical assumption of network necessity. This model is then weakly coupled with a network generation heuristic to explore the space of feasible correlations. It paves the road for both comparison with real correlations and a strongly coupled calibrated model. We also explore novel paradigms such as the role of governance processes in network growth, through a game-theoretic agent-based model. These preliminary results provide the roadmap towards a family of comprehensive operational models of co-evolution between networks and territories that aim to disentangle their circular causalities.
}{

}

\comment{(Florent) trop de concepts dans l'abstract, peut pas apporter qqchse à tous}

\comment{(Florent) commencer par expliquer ce que sont causalités circulaires et pourquoi difficiles à modéliser}

\comment{(Arnaud) complexly : ?}

\comment{(Arnaud) théorie des systèmes territoriaux en réseau co-évolutifs ?}


\endgroup			

\vfill