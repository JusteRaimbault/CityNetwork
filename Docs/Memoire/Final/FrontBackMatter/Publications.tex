% Publications - a page listing research articles written using content in the thesis

\pdfbookmark[1]{Publications}{Publications} % Bookmark name visible in a PDF viewer

\chapter*{Publications} % Publications page text

%Some ideas and figures have appeared previously in the following publications:\\

%\noindent Put your publications from the thesis here. The packages \texttt{multibib} or \texttt{bibtopic} etc. can be used to handle multiple different bibliographies in your document.

%\begin{refsection}[ownpubs]
%    \small
%    \nocite{*} % is local to to the enclosing refsection
%    \printbibliography[heading=none]
%\end{refsection}

%\emph{Attention}: This requires a separate run of \texttt{bibtex} for your \texttt{refsection}, \eg, \texttt{ClassicThesis1-blx} for this file. You might also use \texttt{biber} as the backend for \texttt{biblatex}. See also \url{http://tex.stackexchange.com/questions/128196/problem-with-refsection}.

\bpar{
The following works have an highly overlapping content with this thesis:
}{
Les publications et communications suivantes contiennent la majorité du contenu de cette thèse. Les sources sont précisément mentionnées en introduction de chaque chapitre. Les traductions sont assurées par l'auteur le cas échéant.
}

% NOTE : on self-plagiarism, be careful to precise when extract from a published paper : 
%  - http://academia.stackexchange.com/questions/12342/self-plagiarism-in-phd-thesis
%  - http://academia.stackexchange.com/questions/2029/can-i-use-the-work-in-my-journal-conference-publications-as-chapters-in-my-disse
%  - http://academia.stackexchange.com/questions/149/what-is-a-sandwich-thesis


\section*{Publications}{Publications}


\noindent Raimbault, J. (2018). Indirect Evidence of Network Effects in a System of Cities. \textit{Environment and Planning B, in revision.}

\bigskip


\noindent Raimbault, J. (2018). Calibration of a Density-based Model of Urban Morphogenesis. \textit{PloSONE, in revision.}

\bigskip

\noindent Raimbault, J. (2017). An Applied Knowledge Framework to Study Complex Systems.
\textit{Complex Systems Design \& Management}, Dec 2017, Paris, France. p.31-45.
%\textit{Proceedings of the 8th International Conference on Complex Systems Design \& Management.}
% A. Chapoutout, D. Krob, A. Roussel, F. Stephan, eds. 

\bigskip

\noindent Raimbault, J. (2017). Identification de causalités dans des données spatio-temporelles, \textit{Sageo 2017 Proceedings.}


%\bigskip

%\noindent Antelope, C., Hubatsch, L., Raimbault, J., and Serna, J. M. (2016). An interdisciplinary approach to morphogenesis. Forthcoming in Proceedings of Santa Fe Institute CSSS 2016.

\bigskip

\noindent Bergeaud, A., Potiron, Y., \& Raimbault, J. (2017). Classifying patents based on their semantic content. \textit{PloS one, 12}(4), e0176310.

\bigskip

\noindent Raimbault, J. (2017). A Discrepancy-Based Framework to Compare Robustness Between Multi-attribute Evaluations. In \textit{Complex Systems Design \& Management} (pp. 141-154). Springer International Publishing. 
%\cite{raimbault2017discrepancy}

\bigskip

\noindent Raimbault, J. (2017). Investigating the empirical existence of static user equilibrium. \textit{Transportation Research Procedia}, 22, 450-458. 
%\cite{raimbault2017investigating}


\bigskip


\noindent Raimbault, J. (2016). Generation of Correlated Synthetic Data, \textit{Actes des Journ{\'e}es de Rochebrune 2016.}


\bigskip

\noindent Raimbault, J. (2017). Models coupling urban growth and transportation network growth: An algorithmic systematic review approach. Plurimondi, (17).




\section*{Working Papers}{Documents de travail}



\noindent Raimbault, J. \& Bergeaud, A. (2017). The Cost of Transportation: Spatial Analysis of Fuel Prices in the US, \textit{en revue pour Transportation Research Part A.} arxiv:1706.07467


\bigskip

\noindent Raimbault, J. (2017). Exploration of an Interdisciplinary Scientific Landscape. \textit{En revue pour Scientometrics.} arXiv preprint arXiv:1712.00805.

\bigskip

\noindent Raimbault J., Cottineau C., Le Texier M., Le Nechet F. \& Reuillon R.
 (2017). Space Matters: extending sensitivity analysis to initial spatial conditions in geosimulation models. \textit{En revue pour Environment and Planning B.}


\bigskip

\noindent Banos A., Chasset P.-O., Commenges A. Cottineau C., Pumain D. \& Raimbault J. (2018). Where do you mean? A spatialised bibliometrics approach of a scientific journal production. \textit{En revue pour Journal of Informetrics.}





\section*{Communications}{Communications}


% gopro ?


\noindent Complexity, Complexities and Complex Knowledge, \textit{Geodivercity International Workshop, Paris, October 2017.}


\bigskip


\noindent Modeling the Co-evolution of Urban Form and Transportation Networks, \textit{Conference on Complex Systems 2017, Cancun, Sept. 2017.}

\bigskip

\noindent Raimbault J. \& Baffi S. (2017). Structural Segregation: Assessing the impact of South African Apartheid on Underlying Dynamics of Interactions between Networks and Territories, \textit{ECTQG 2017, York, Sept. 2017}


\bigskip


\noindent Invisible Bridges ? Scientific landscapes around similar objects studied from Economics and Geography perspectives, \textit{ECTQG 2017, York, Sept. 2017}


\bigskip


\noindent Cottineau C., Raimbault J., Le Texier M., Le N{\'e}chet F. \& Reuillon R. (2017). Initial spatial conditions in simulation models: the missing leg of sensitivity analyses?, \textit{Geocomputation 2017, Leeds, Sept. 2017}

\bigskip


\noindent A macro-scale model of co-evolution for cities and transportation networks, \textit{Medium International Conference, Guangzhou, June 2017}


\bigskip

\noindent Losavio C. \& Raimbault J. (2017). Agent-based Modeling of Migrant Workers Residential Dynamics within a Mega-city Region: the Case of Pearl River Delta, China, \textit{Urban China Development International Conference, London, May 2017}


\bigskip

\noindent Co-construire Modèles, Etudes Empiriques et Théories en Géographie Théorique et Quantitative: le cas des Interactions entre Réseaux et Territoires. In \textit{Treizièmes Rencontres de ThéoQuant, Besançon, Mai 2017}


\bigskip

\noindent Un Cadre de Connaissances pour une Géographie Intégrée. In \textit{Journée des jeunes chercheurs de l'Institut de Géographie de Paris, Paris, April 2017}


\bigskip


\noindent Towards a Theory of Co-evolutive Networked Territorial Systems: Insights from Transportation Governance Modeling in Pearl River Delta, China, \textit{MEDIUM Seminar : Sustainable Development in Zhuhai, Guangzhou, Dec 2016.}


\bigskip


\noindent Models of growth for system of cities : Back to the simple, \textit{Conference on Complex Systems 2016, Amsterdam, Sep 2016.}



%Raimbault J., Bergeaud A. and Potiron Y. (2016). Investigating Patterns of Technological Innovation. \textit{Conference on Complex Systems 2016, Amsterdam, Sep 2016.}


\bigskip

\noindent For a Cautious Use of Big Data and Computation. \textit{Royal Geographical Society - Annual Conference 2016 - Session : Geocomputation, the Next 20 Years (1), London, Aug 2016.}


\bigskip

\noindent Indirect Bibliometrics by Complex Network Analysis. \textit{20e Anniversaire de Cybergeo, Paris, May 2016.}


\bigskip

\noindent Raimbault, J. \& Serra, H. (2016). Game-based Tools as Media to Transmit Freshwater Ecology Concepts, \textit{poster corner at SETAC 2016 (Nantes, May 2016).}


\bigskip

\noindent Le Néchet, F. \& Raimbault, J. (2015). Modeling the emergence of metropolitan transport authority in a polycentric urban region, \textit{ECTQG 2015, Bari, Sep 2015).}


\bigskip

\noindent Hybrid Modeling of a Bike-Sharing Transportation System, \textit{poster presented at ICCSS 2015, Helsinki, June 2015.}

\bigskip

\noindent Raimbault, J. \& Gonzales, J. (2015). Application de la Morphog{\'e}n{\`e}se de R{\'e}seaux Biologiques {\`a} la Conception Optimale d'Infrastructures de Transport, \textit{poster presented at Rencontres du Labex Dynamite, Paris, May 2015.}


