


%%%%%%%%%%%%%%%%
%%  Introduction
%%%%%%%%%%%%%%%%


%% Contents
%
%    - General considerations on Complex Systems, positioning etc (thesis in cs science etc)
%    - Thematic introduction, geographical introduction of the subject.
%
%   - precisions on v1 memoire : foreword ?
%
%    - reading precisions : organisation, interdependances etc 
%
%   - reflexive aspect : here ?  


\chapter*{Introduction}




%-------------------------------------------------

\section*{Scientific Context}












\section*{Interdisciplinarity}



\paragraph{Complexity Has Come of Age}

% Q : quote Morin ?

Beyond ``fashionable'' positions that can be the consequence of a blind following~\cite{dirk1999measure}, or more ambivalent, of a marketing strategy as the fight for funds is becoming a huge obstacle for research~\cite{bollen2014funding}, Science of Complexity is taking a hole new place in the academic landscape.
%Its success may have several origins such as unexpected approaches, theoretical and practical promising results, or the recent explosion of computational possibilities. % needed ? say the same further.
As an informal mix of epistemological positions, methods, and fields of applications, it relies on \emph{unconventionnal} paradigms such as the centrality of emergence and self-organization in most of phenomena of the real world, which make it lie on a frontier of knowledge closer of us than we can think, as Laughlin develops in~\cite{laughlin2006different}. \textit{Detail concepts ?}. Such concepts are indeed not new, as they were already enlighted by Anderson~\cite{anderson1972more}. Even cybernetics can be related to complexity by seing it as a bridge between technics and cognitive science~\cite{wiener1948cybernetics}. Later, synergetics~\cite{haken1980synergetics} paved the way for a theoretical approach of collective phenomena in physics. Reasons for the recent growth of works claiming a CS approach may be various. The explosion of computing power is surely one because of the central role of numerical simulations~\cite{varenne2010simulations}. They could also be the related epistemological progresses : apparition of the notion of perspectivism~\cite{giere2010scientific}, finer reflexions around the notion of model~\cite{varenne2013modeliser} [note : beware of a chicken-egg type problem on the relation between scientific and epistemological progress]. The theoretical and empirical potentialities of such approach play surely a role in their success, as confirmed in various domains of application (see~\cite{newman2011complex} for a general survey), as for example Network Science~\cite{barabasi2002linked} ; Neuroscience~\cite{koch1999complexity} ; Social Sciences  ; Geography~\cite{manson2001simplifying}\cite{pumain1997pour} ; Finance with the rising importance of econophysics~\cite{stanley1999econophysics}.




\paragraph{Conflicting Complexities and Cultural Differences}

Yet this scientific evolution that some see as a revolution~\cite{colander2003complexity}, or even as \emph{a new kind of science}~\cite{wolfram2002new}, could face intrinsic difficulties due to behaviors and a-priori of researchers as human beings. More precisely, the need of interdisciplinarity that makes the strength of Complexity Science may be one of its greatest weaknesses, since the highly partitioned structure of science organization has sometimes negative impacts on works involving different disciplines. We do not tackle the issue of overpublication, competition, indexes, which is more linked to a question of open science and its ethics, also of high importance but of an other nature. That barrier we are dealing with and we might struggle to triumph of, is the impact of certains \emph{cultural disciplinary differences} and outcoming conflicts on views and approaches. We shall now develop some concrete example that lead to such considerations when encoutered. They are of many different natures and concern different disciplines, such that it would not be honnest to assume that the issue is not general. Each come from personnal research experience


\textit{Physics reinvents geography.} 



\textit{Economic Geography or Geographical Economics ?}



\textit{Agent-based Modeling in Economy}


\textit{Finance}



The drama of scientific misunderstandings is that they can indeed anhiliate progresses by interpreting as a falsification some work that answers to a totally different question. The example of a recent work on top-income inequalities given in~\cite{aghion2015innovation}, which conclusions are presented as opposed from the one obtained by Piketty~\cite{piketty2013capital}, follows such a scheme.



%\paragraph{Intrinsic Obstacles}
\paragraph{Keep It Complex, Stupid}



\paragraph{Means Are Here, Let Use Them}



















%-------------------------------------------------

\section*{Geographical Objects}





