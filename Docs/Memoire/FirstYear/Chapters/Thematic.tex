

% Chapter 




\chapter{Interactions between Networks and Territories} % Chapter title

\label{ch:thematic} % For referencing the chapter elsewhere, use \autoref{ch:name} 


%%  Thematic chapter framing geographically the subject.
%%   and reviewing state of the art
%%   and why modeling : evolutive theory of urban systems etc ; multimodeling simfamily etc
%%  
%%   Q  : example to introduce theory ?
%
%   Modelography.  (non-exhaustive) : classification according to purpose, theme, scale, etc.
%   Why dynamic models of ``co-evolution''  ?
%   definition of terms, contextualisation, etc.  (le what/where d'Arnaud ; ontology de Anne)



%----------------------------------------------------------------------------------------

%\headercit{If you are embarrassed by the precedence of the chicken by the egg or of the egg by the chicken, it is because you are assuming that animals have always be the way they are}{Denis Diderot}{\cite{diderot1965entretien}}

\headercit{Si la question de la priorit{\'e} de l'\oe{}uf sur la poule ou de la poule sur l'\oe{}uf vous embarrasse, c'est que vous supposez que les animaux ont {\'e}t{\'e} originairement ce qu'ils sont {\`a} pr{\'e}sent.}{Denis Diderot}{\cite{diderot1965entretien}}

\bigskip


This analogy is ideal to evoke the questions of causality and processes in territorial systems. When trying to tackle naively our preliminary question, some observers have qualified the identification of causalities in complex systems as ``chicken and egg'' problems : if one effect appears to cause another and reciprocally, how can one disentangle effective processes ? This vision is often present in reductionist approaches that do not postulate an intrinsic complexity in studied systems. The idea that Diderot suggests is the notion of \emph{co-evolution} that is a central phenomenon in evolutive dynamics of Complex Adaptive Systems as \noun{Holland} develops in~\cite{holland2012signals}. He links the notion of emergence (that is ignored in a reductionist vision), in particular the emergence of structures at an upper scales from the interactions between agents at a given scale, materialized generally by boundaries, that become crucial in the coevolution of agents at any scales : the emergence of one structure will be simultaneous with one other, each exploiting their interrelations and generated environments conditioned by their boundaries. We shall explore these ideas in the case of territorial systems in the following.


This introductive chapter aims to set up the thematic scene, the geographical context in which further developments will root. It is not supposed to be understood as an exhaustive literature review nor the fundamental theoretical basement of our work (the first will be an object of chapter~\ref{ch:quantepistemo} whereas the second will be earlier tackled in chapter~\ref{ch:theory}), but more as narration aimed to introduce typical objects and views and construct naturally research questions.



%-------------------------------

\newpage

\section[Territories and Networks]{Territories and Networks}


\subsection{Territories and Networks : There and Back Again}

\paragraph{Human Territories}

The notion of territory can be taken as a basis to explore the scope of geographical objects we will study. In Ecology, a territory corresponds to a spatial extent occupied by a group of agents or more generally an ecosystem. \emph{Human Territories} are far more complex in the sense of semiotic representations of these that are a central part in the emergence of societies. For \noun{Raffestin} in~\cite{raffestin1988reperes}, the so-called \emph{Human Territoriality} is the ``conjonction of a territorial process with an informational process'', what means that physical occupation and exploitation of space by human societies is not dissociable from the representations (cognitive and material) of these territorial processes, driving in return its further evolutions. In other words, as soon as social constructions are assumed in the constitution of human settlements, concrete and abstract social structures will play a role in the evolution of the territorial system, through e.g. propagation of information and representations, political processes, conjonction or disjonction between lived and perceived territory. Although this approach does not explicitly give the condition for the emergence of a seminal system of aggregated settlements (i.e. the emergence of cities), it insists on the role of these that become places of power and of creation of wealth through exchange. But the city has no existence without its hinterland and the territorial system can not be summarized by its cities as a system of cities. There is however compatibility on this subsystem between \noun{Raffestin} approach to territories and \noun{Pumain}'s evolutive theory of urban systems~\cite{pumain2010theorie}, in which cities are viewed as an auto-organized complex dynamical systems, and act as mediators of social changes : for example, cycles of innovation occur within cities and propagate between them. Cities are thus competitive agents that co-evolve (in the sense given before). The territorial system can be understood as a spatially organized social structure, including its concrete and abstract artifacts. A imaginary free-of-man spatial extent with potential ressources will not be a territory if not inhabited, imagined, lived, and exploited, even if the same ressources would be part of the corresponding habited territorial system. Indeed, what is considered as a ressource (natural or artificial) will depend on the corresponding society (e.g. of its practices and technological potentialities). A crucial aspect of human settlements that were studied in geography for a long time, and that relate with the previous notion of territory, are \emph{networks}. Let see how we can switch from one to the other and how their definition may be indissociable.


\paragraph{A Territorial Theory of Networks}

We paraphrase \noun{Dupuy} in~\cite{dupuy1987vers} when he proposes elements for ``a territorial theory of networks'' based on the concrete case of Urban Transportation Networks. This theory sees \emph{real networks} (i.e. concrete networks, including transportation networks) as the materialization of \emph{virtual networks}. More precisely, a territory is characterized by strong spatio-temporal discontinuities induced by the non-uniform distribution of agents and ressources. These discontinuities naturally induce a network of ``transactional projects'' that can be understood as potential interactions between elements of the territorial system (agents and/or ressources). For example today, people need to access the ressource of employments, economic exchanges operate between specialized production territories. At any time period, potential interactions existed\footnote{even when nomadism was still the rule, spatially dynamic networks of potential interactions necessarily existed, but should have less chance to materialize into concrete routes.% bib on that ?
}. The potential interaction network is concretized as offer adapts to demand, and results of the combination of economic and geographical constraints with demand patterns, in a non-linear way through agents designed as \emph{operators}. This process is not immediate, leading to strong non-stationarity and path-dependance effects : the extension of an existing network will depend on previous configuration, and depending on involved time scales, the logic and even the nature of operators may have evolved. \noun{Raffestin} points out in his preface of~\cite{offner1996reseaux} that a geographical theory articulating space, network and territories had never been consistently formulated. It appears to still be the case today, but the theory developed just before is a good candidate, even if it stays at a conceptual level. The presence of a human territory necessarily imply the presence of abstract interaction networks and concrete networks used for transportation of people and ressources (including communication networks as information is a crucial ressource). Depending on regime in which the considered system is, the respective role of different networks may be radically different. Following \noun{Duranton} in \cite{duranton1999distance}, pre-industrial cities were limited in growth because of limitations of transportation networks. Technological progresses have lead to the end of these limitations and the preponderance of land markets in shaping cities (and thus a role of transportation network as shaping prices through accessibility), and recently to the rising importance of telecommunication networks that induce a ``tyranny of proximity'' as physical presence is not replaceable by virtual communication. This territorial approach to networks seems natural in geography, since networks are studied conjointly with geographical objects with an underlying theory, in opposition to network science that studies brutally spatial networks with few thematic background~\cite{ducruet2014spatial}.


\paragraph{Networks shaping territories}

% how do network shape territories : boundaries, scales, etc.
% example : \cite{l2012ville} bahn-ville, volontary coevol ? // idem villes nouvelles





\paragraph{Territorial Systems}

This detour from territories, to networks and back again, allows us to give a preliminary definition of a territorial system 



\subsection{Transportation Networks}


\paragraph{Deconstructing Accessibility}

% critic of accessibility as a planning tool : danger of not taking into account socio-eco dynamics and coupled dynamics (coevol) - cit Hadri mobility as a constructed notion.


\paragraph{Scales and Hierarchies}




%----------------------------------------------------------------------------------------

\section{Modeling Interactions}


\subsection{Modeling in Quantitative Geography}

% brief reference to the history of TQG ; history of modeling.
%  note : history of future of TQG, London september 2016




\subsection{Modeling Territories and Networks}

% here overview of different approaches
% TODO Q : do it here, not during quant epistemo part ?

\subsubsection{Land-Use Transportation Interaction Models}

A subsequent bunch of literature in modeling interaction between networks and territories can be found in the field of planning, with the so-called \emph{Land-use Transportation Interaction Models}. These works are difficult to be precisely bounded as they may be influenced by various disciplines. For example, from the point of view of Urban Economics, propositions for synthesizing models have existed for a relatively long term~\cite{putman1975urban}. The variety of possible models has lead to operational comparisons~\cite{paulley1991overview,wegener1991one}. More recently, the respective advantages of static and dynamic was investigated in~\cite{kryvobokov2013comparison}.

\cite{chang2006models}

\cite{delons:hal-00319087}

\cite{iacono2008models}

\cite{wegener2004land}


\subsubsection{Urban Systems Modeling}

% differentiate to other hybrid models : SimpopNet and others ? (if exist ?)


\subsubsection{Network Growth}



\subsubsection{Hybrid Modeling}

Models of simulation implementing a coupled dynamic between urban growth and transportation network growth are relatively rare, and always rather poor from a theoretical and thematic point of view. A generalization of the geometrical local optimization model described before was developed in~\cite{barthelemy2009co}. % pb of scales, def of coevol, thematic meaning of assumptions, etc.



\cite{levinson2007co} : economic model of coevolution. % TODO check timescales if are consistent.



\subsection{Sketch of a \emph{Modelography}}





%-------------------------

\newpage

\section{Research Question}







