

% Chapter 




\chapter{Interactions between Networks and Territories} % Chapter title

\label{ch:thematic} % For referencing the chapter elsewhere, use \autoref{ch:name} 


%%  Thematic chapter framing geographically the subject.
%%   and reviewing state of the art
%%   and why modeling : evolutive theory of urban systems etc ; multimodeling simfamily etc
%%  
%%   Q  : example to introduce theory ?
%
%   Modelography.  (non-exhaustive) : classification according to purpose, theme, scale, etc.
%   Why dynamic models of ``co-evolution''  ?
%   definition of terms, contextualisation, etc.  (le what/where d'Arnaud ; ontology de Anne)



%----------------------------------------------------------------------------------------

\headercit{If you are embarrassed by the precedence of the chicken by the egg or of the egg by the chicken, it is because you are assuming that animals have always be the way they are}{Denis Diderot}{\cite{diderot1965entretien}}

%Si la question de la priorité de l'œuf sur la poule ou de la poule sur l'œuf vous embarrasse, c'est que vous supposez que les animaux ont été originairement ce qu'ils sont à présent

\bigskip

This introductive chapter aims to set up the thematic scene, the geographical context in which further developments will root. It is not supposed to be understood as an exhaustive literature review nor the fundamental theoretical basement of our work (the first will be an object of chapter~\ref{ch:quantepistemo} whereas the second will be earlier tackled in chapter~\ref{ch:theory}), but more as narration aimed to introduce typical objects and views and construct naturally research questions.

\section{From Territories to Networks and back}


\paragraph{Human Territories}

The notion of territory can be taken as a basis to explore the scope of geographical objects we will study. In Ecology, a territory corresponds to a spatial extent occupied by a group of agents or more generally an ecosystem. \emph{Human Territories} are far more complex in the sense of semiotic representations of these that are a central part in the emergence of societies. For \noun{Raffestin} in~\cite{raffestin1988reperes}, the so-called \emph{Human Territoriality} is the ``conjonction of a territorial process with an informational process'', what means that physical occupation and exploitation of space by human societies is not dissociable from the representations (cognitive and material) of these territorial processes, driving in return its further evolutions. In other words, 

\paragraph{Towards a Territorial Theory of Networks}

We paraphrase \noun{Dupuy} in~\cite{dupuy1987vers} when he proposes elements for ``a territorial theory of networks'' based on the concrete case of Urban Transportation Networks. 


\noun{Raffestin} points out in his preface of~\cite{offner1996reseaux} that a geographical theory articulating space, network and territories had never been consistently formulated. It appears to still be the case today.



\subsection{Transportation Networks}


\paragraph{Deconstructing Accessibility}

% critic of accessibility as a planning tool : danger of not taking into account socio-eco dynamics and coupled dynamics (coevol) - cit Hadri mobility as a constructed notion.


\paragraph{Scales and Hierarchies}




%----------------------------------------------------------------------------------------

\section{Modeling Interactions}


\subsection{Modeling in Quantitative Geography}

% brief reference to the history of TQG ; history of modeling.
%  note : history of future of TQG, London september 2016




\subsection{Modeling Territories and Networks}

% here overview of different approaches
% TODO Q : do it here, not during quant epistemo part ?

\subsubsection{Land-Use Transportation Interaction Models}




\subsection{Sketch of a \emph{Modelography}}










