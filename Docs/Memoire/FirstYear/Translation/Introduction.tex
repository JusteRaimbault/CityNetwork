

%%% Test of translation to estimate translation speed %%%



\chapter*{Introduction}

% to have header for non-numbered introduction
\markboth{Introduction}{Introduction}

%\headercit{We need to find Banos' tenth modeling law}{Ren{\'e} Doursat}{}
\headercit{C'est quand on donne un coup de pied dans la fourmilière qu'on se rend compte de toute sa complexité.}{Arnaud Banos}{}

\bigskip

``En conséquence d'un problème technique, le trafic est interrompu sur la ligne B du RER pour une durée indéterminée. Plus d'information seront fournies dès que possible''. Il y a des fortes chances pour que quiconque ayant vécu ou passé un peu de temps en région parisienne ait déjà entendu cette annonce glaçante et en ait subi les conséquences pour le reste de la journée. Mais il ne se doute sûrement pas des ramifications des cascades causales induites par cet évènement presque banal. Les systèmes territoriaux, quelles que soient les aspects considérés pour leur définition, seront toujours extrêmement complexes, les interrelations à de nombreuses échelles spatiales et temporelles participant à la production des comportements émergents observés à tout niveau du système. Martin est un étudiant qui fait l'aller-retour journalier entre Paris et Palaiseau and manquera un examen crucial, ce qui aura un impact profond sur sa vie professionnelle : implications à une longue échelle de temps, une petite échelle spatiale et à la granularité de l'agent. Yuangsi était en train de relier les aéroports d'Orly et Roissy dans son voyage de Londres à Pékin et va manquer son avion ainsi que le mariage de sa soeur : grande échelle spatiale, petite échelle de temps, granularité de l'agent. Une pétition collective émerge des voyageurs, conduisant à la création d'une organisation qui mettra la pression sur les autorités pour qu'elles augmentent le niveau de service : échelle temporelle et spatiales mesoscopique, granularité de l'aggregation d'agents. La recherche de cause possible à l'incident conduira à des processus intriqués à diverses échelles, parmi lesquels aucun ne semble être une meilleure explication ; le développement historique du réseau ferroviaire en région parisienne a conditionné les écolutions futures et le RER B a suivi l'ancienne Ligne de Sceaux, le plan de \noun{Delouvrier} pour le développement régional et son execution partielle, sont également des éléments d'explication des faiblesses structurelles du réseau parisien de transports en commun~\cite{gleyze2005vulnerabilite} ; le motifs pendulaires dus à l'organisation territoriale induisent une surcharge de certaines ligne et ainsi nécessairement une augmentation des incidents d'exploitation. La liste pourrait être ainsi continuée un certain temps, chaque approche apportant sa vision mature correspondant à un corpus de connaissances scientifiques dans des disciplines diverses comme la géographie, l'économie urbains, les transports. Cette anecdote amusante est suffisante pour faire ressentir la complexité des systèmes territoriaux. Notre but ici est de se plonger dans cette complexité, et en particulier donner un point de vue original sur l'étude des relations entre réseaux et territoires. Le choix de cette position sera largement discuté dans une partie thématique, nous nous concentrons à présent sur l'originalité du point de vue que nous allons prendre.






