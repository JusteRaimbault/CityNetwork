

%%% Test of translation to estimate translation speed %%%



\chapter*{Introduction}

% to have header for non-numbered introduction
\markboth{Introduction}{Introduction}

%\headercit{We need to find Banos' tenth modeling law}{Ren{\'e} Doursat}{}
\headercit{C'est quand on donne un coup de pied dans la fourmilière qu'on se rend compte de toute sa complexité.}{Arnaud Banos}{}

\bigskip

``En conséquence d'un problème technique, le trafic est interrompu sur la ligne B du RER pour une durée indéterminée. Plus d'information seront fournies dès que possible''. Il y a des fortes chances pour que quiconque ayant vécu ou passé un peu de temps en région parisienne ait déjà entendu cette annonce glaçante et en ait subi les conséquences pour le reste de la journée. Mais il ne se doute sûrement pas des ramifications des cascades causales induites par cet évènement presque banal. Les systèmes territoriaux, quelles que soient les aspects considérés pour leur définition, seront toujours extrêmement complexes, les interrelations à de nombreuses échelles spatiales et temporelles participant à la production des comportements émergents observés à tout niveau du système. Martin est un étudiant qui fait l'aller-retour journalier entre Paris et Palaiseau and manquera un examen crucial, ce qui aura un impact profond sur sa vie professionnelle : implications à une longue échelle de temps, une petite échelle spatiale et à la granularité de l'agent. Yuangsi était en train de relier les aéroports d'Orly et Roissy dans son voyage de Londres à Pékin et va manquer son avion ainsi que le mariage de sa soeur : grande échelle spatiale, petite échelle de temps, granularité de l'agent. Une pétition collective émerge des voyageurs, conduisant à la création d'une organisation qui mettra la pression sur les autorités pour qu'elles augmentent le niveau de service : échelle temporelle et spatiales mesoscopique, granularité de l'aggregation d'agents. La recherche de cause possible à l'incident conduira à des processus intriqués à diverses échelles, parmi lesquels aucun ne semble être une meilleure explication ; le développement historique du réseau ferroviaire en région parisienne a conditionné les écolutions futures et le RER B a suivi l'ancienne Ligne de Sceaux, le plan de \noun{Delouvrier} pour le développement régional et son execution partielle, sont également des éléments d'explication des faiblesses structurelles du réseau parisien de transports en commun~\cite{gleyze2005vulnerabilite} ; le motifs pendulaires dus à l'organisation territoriale induisent une surcharge de certaines ligne et ainsi nécessairement une augmentation des incidents d'exploitation. La liste pourrait être ainsi continuée un certain temps, chaque approche apportant sa vision mature correspondant à un corpus de connaissances scientifiques dans des disciplines diverses comme la géographie, l'économie urbains, les transports. Cette anecdote amusante est suffisante pour faire ressentir la complexité des systèmes territoriaux. Notre but ici est de se plonger dans cette complexité, et en particulier donner un point de vue original sur l'étude des relations entre réseaux et territoires. Le choix de cette position sera largement discuté dans une partie thématique, nous nous concentrons à présent sur l'originalité du point de vue que nous allons prendre.



%-------------------------------------------------

\section*{Contexte Scientifique : Paradigmes de la Complexité}

Pour une meilleure introduction du sujet, il est nécessaire d'insister sur le cadre scientifique dans lequel nous nous positionnons. Ce contexte est crucial à la fois pour comprendre les concepts épistémologiques implicites dans nos questions de recherche, et aussi pour être conscient de la variété de méthodes et outils utilisés. La science contemporaine prend progressivement le tournant de la complexité dans de nombreux champs, ce qui implique une mutation épistémologique pour abandonner le réductionnisme strict qui a échoué dans la majorité de ses tentatives de synthèse~\cite{anderson1972more}. Arthur a rappelé récemment~\cite{arthur2015complexity} qu'une mutation des méthodes et paradigmes en était également un enjeu, de par la place grandissante prise par les approches computationnelles qui remplacent les résolutions purement analytiques généralement limité en possibilités de modélisation et de résolution. La capture des \emph{propriétés émergentes} par des modèles de systèmes complexes est une des façons d'interpréter la philosophie de ces approaches.

Ces considérations sont bien connus des Sciences Humaines (qualitatives et quantitatives) pour lesquelles la complexité des agents et systèmes étudiés est une des justifications de leur existence : si les humains étaient des particules, la majorité des disciplines les prenant comme objet d'étude n'auraient jamais émergé puisque la thermodynamique aurait alors résolu la majorité des problèmes sociaux\footnote{bien que cette affirmation soit elle-même discutable, les sciences physiques classiques ayant également échoué à prendre en compte l'irréversibilité et l'évolution de Systèmes Complexes Adaptatifs comme le souligne \noun{Prigogine} dans \cite{prigogine1997end}.}. Elles sont au contraire moins connues et acceptées en sciences ``dures'' comme la physique : \noun{Laughlin} développe dans~\cite{laughlin2006different} une vision de la discipline à la même position de ``frontière des connaissances'' que d'autre champs pouvant paraître moins matures. La plupart des connaissances actuelles concerne des structures classiques simples, alors qu'un grand nombre de système présentent des propriétés \emph{d'auto-organisation}, au sense ou les lois macroscopiques ne sont pas suffisantes pour inférer les propriétés macroscopiques du systèmes à moins que son évolution soit entièrement simulée (plus précisément cette vision peut être prise comme une définition de l'émergence sur laquelle nous reviendrons par la suite, or des propriétés auto-organisées sont par nature émergentes). Cela correspond au premier cauchemar du Démon de Laplace développé dans~\cite{deffuant2015visions}. 
 


As an informal mix of epistemological positions, methods, and fields of applications, \emph{Complexity Science} relies on typical paradigms such as the centrality of emergence and self-organization in most of phenomena of the real world, which make it lie on a frontier of knowledge closer of us than we can think (Laughlin, op.cit. ). Such concepts are indeed not new, as they were already enlighten by Anderson~\cite{anderson1972more}. Even cybernetics can be related to complexity by seing it as a bridge between technology and cognitive science~\cite{wiener1948cybernetics}. Later, synergetics~\cite{haken1980synergetics} paved the way for a theoretical approach of collective phenomena in physics. Reasons for the recent growth of works claiming a CS approach may be various. The explosion of computing power is surely one because of the central role of numerical simulations~\cite{varenne2010simulations}. They could also be the related epistemological progresses : apparition of the notion of perspectivism~\cite{giere2010scientific}, finer reflexions around the notion of model~\cite{varenne2013modeliser}\footnote{In that frame scientific and epistemological progress can not be dissociated and can be seen as coevolving}. The theoretical and empirical potentialities of such approaches play surely a role in their success\footnote{
Although the adoption of new scientific practices may be strongly biased by imitation and lack of originality~\cite{dirk1999measure}, or more ambivalent, by marketing strategies as the fight for funds is becoming a huge obstacle for research~\cite{bollen2014funding}.}, as confirmed in various domains of application (see~\cite{newman2011complex} for a general survey), as for example Network Science~\cite{barabasi2002linked} ; Neuroscience~\cite{koch1999complexity}; Social Sciences ; Geography~\cite{manson2001simplifying}\cite{pumain1997pour} ; Finance with the rising importance of econophysics~\cite{stanley1999econophysics} ; Ecology~\cite{grimm2005pattern}. The Complex Systems Roadmap~\cite{2009arXiv0907.2221B} proposed a double lecture of studies on Complex Systems : an horizontal approach connecting fields of study with transversal questions on theoretical foundations of complexity and empirical common stylized facts, and a vertical conceptions of disciplines, with the aim to construct integrated disciplines and corresponding multi-scale heterogeneous models. Interdisciplinarity is thus central in our scientific background.













