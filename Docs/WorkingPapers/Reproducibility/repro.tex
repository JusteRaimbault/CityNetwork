%%%%%%%%%%%%%%%%%%%%%%%%%%%%%
% Standard header for working papers
%
% WPHeader.tex
%
%%%%%%%%%%%%%%%%%%%%%%%%%%%%%

\documentclass[11pt]{article}

% packages without options
\usepackage{amsmath,bbm}

% geometry
\usepackage[margin=2cm]{geometry}






\title{Reproducibility in Practice : Loosing your Balance on the Shoulder of Giants\bigskip\\
\textit{Working Paper}
}
\author{\noun{Juste Raimbault}}
\date{Monday 9th March}


\maketitle

\begin{abstract}
As scientific reproducibility is an essential requirement for any study, its practice seems to be increasing~\cite{stodden2010scientific} and technical means to achieve it are always more developed (as e.g. ways to make data openly available, or to be transparent on the research process such as \texttt{git}~\cite{ram2013git}, or to integrate document creation and data analysis such as \texttt{knitr}~\cite{xie2013knitr}), at least in the field of numerical modeling and simulation. However, the devil is indeed in the details and obstacles judged at first sight as minor become rapidly a burden for reproducing and using results obtained in some previous researches. We describe two cases studies where models of simulation are apparently highly reproducible but unveil as puzzles on which research-time balance is significantly under zero.
\end{abstract}




%%%%%%%%%%%%%%%%%%%%%%%%%%%%%%%%%%
\section{On the Need to Explicit the Model}

A current myth is that providing entire source code and data will be a sufficient condition for reproducibility. It will work if the objective is to produce exactly same plots or statistical analysis, assuming that code provided is the one which was indeed used to produce the given results. It is however not the nature of reproducibility. First, results must be as much implementation-independent as possible for clear robustness purposes. Then, in relation with the precedent point, one of the purposes of reproducibility is the reuse of methods or results as basis or modules for further research (what includes implementation in another language or adaptation of the method), in the sense that reproducibility is not replicability as it must be adaptable~\cite{drummond2009replicability}.

Our first case study fits exactly that scheme, as it was undoubtedly aimed to be shared with and used by the community since it is a model of simulation provided with the Agent-Based simulation platform NetLogo~\cite{wilensky1999netlogo}. The model is also available online~\cite{de2007netlogo} and is presented as a tool to simulate socio-economic dynamics of low-income residents in a city based on a synthetic urban environment, generated to be close in stylized facts from the real town of Tijuana, Mexico. Beside providing the source code, the model appears to be poorly documented in the literature or in comments and description of the implementation. Comments made thereafter are based on the study of the urban morphogenesis part of the model (setup for the ``residential dynamics'' component) as it is our global context of study~\cite{raimbault2014vers}. In the frame of that study, source code was modified and commented, which last version is available on the repository of the project\footnote{at \texttt{https://github.com/JusteRaimbault/CityNetwork/tree/master/Models/Reproduction/UrbanSuite}}.




\paragraph{Rigorous Formalization}

An obvious part of model construction is its rigorous formalization in a formal framework distinct from source code. There is of course no universal language to formulate it~\cite{banos2013pour}, and many possibilities are offered by various fields (e.g. UML, DEVS, pure mathematical formulation). No paper nor documentation is provided with the model, apart from the embedded NetLogo documentation since it only thematically describes in natural language the ideas behind each step without developing more and provides information about role of different elements of the interface.

This formulation is a key for it to be understood, reproduced and adapted ; but it also avoids implementation biases such as
\begin{itemize}
\item Architecturally dangerous elements : in the model, world context is a torus and agents may ``jump'' in the euclidian representation, what is not acceptable for a 2D projection of real world. To avoid that, many tricky tests and functions were used, including unadvised practices (e.g. dead of agents based on position to avoid them jumping).
\item Coding errors : in an untyped language such as NetLogo, mixing types may conduct to unexpected runtime errors, what is the case of the patch variable \texttt{transport} in the model (although no error occurs in most of run configurations from the interface, what is more dangerous as the developer thinks implementation is secure). Such problems should be avoided if implementation is done from an exact formal description of the model.
\end{itemize}


\paragraph{Transparent Implementation}


\paragraph{Expected Model Behavior}

Whatever the definition, a model can not be reduced to its formulation and/or implementation, as expected model behavior or model usage can be viewed as being part of the model itself. In the frame of \noun{Giere}'s perspectivism~\cite{giere2010scientific}, the definition of model includes the purpose of use but also the agent who aims to use it. Therefore a minimal explication of model behavior and exploration of parameter roles is highly advised to decrease chances of misuses or misinterpretations of it.







\section{On the Need of Exactitude in Model Implementation}






\section*{Conclusion}











%%%%%%%%%%%%%%%%%%%%
%% Biblio
%%%%%%%%%%%%%%%%%%%%

\bibliographystyle{apalike}
\bibliography{/Users/Juste/Documents/ComplexSystems/CityNetwork/Biblio/Bibtex/CityNetwork}


\end{document}
