%%%%%%%%%%%%%%%%%%%%%%%%%%%%%
% Standard header for working papers
%
% WPHeader.tex
%
%%%%%%%%%%%%%%%%%%%%%%%%%%%%%

\documentclass[11pt]{article}

% packages without options
\usepackage{amsmath,bbm}

% geometry
\usepackage[margin=2cm]{geometry}






\title{Systems : an Interdisciplinary Overview\bigskip\\
\textit{Working Paper}
}
\author{\noun{Juste Raimbault}}
\date{10th December 2015}


\maketitle

\justify


\begin{abstract}

\end{abstract}


% Domains reviewed -> that can help to purpose ?
%  (if not can copy-paste newman survey)




\section{Integrative Biology}




\section{Systems in Geography}

\subsection{Systemic Approaches}


Laurini in~\cite{guermond1983techniques}, proposes a short definition of a system as ``a set of interacting elements'', complemented with a longer definition involving external environment, structure, dynamic character and that can have finality. The complexity of a system would be given by the richness of its interactions, what is a fuzzy qualification compared to common complexity measures. The notion of ``general system'', is a structure through which modelings of a real object are realized : the same system allows different visions to yield a model. Each model would be a certain perspective on the object, and the general model an universal model. Its existence should however only stay conceptual, as one can not anticipate e.g. the number of perspectives and their horizons it will be called to produce\footnote{therefore the proposition in [working Paper theory] to construct reversely an object analog to the general system, that we also called the system, but that would be constructed \emph{from} a set of perspective rather than being assumed as preexisting.}. In the same volume, Allen differentiates physical from biological definitions of a system, emphasizing the presence of structure in interactions between elements, of downward causation on micro constituents 


\subsection{Complex Systems and Evolutionary Theories}








%%%%%%%%%%%%%%%%%%%%
%% Biblio
%%%%%%%%%%%%%%%%%%%%

\bibliographystyle{apalike}
\bibliography{/Users/Juste/Documents/ComplexSystems/CityNetwork/Biblio/Bibtex/CityNetwork,systems}


\end{document}
