%%%%%%%%%%%%%%%%%%%%%%%%%%%%%
% Standard header for working papers
%
% WPHeader.tex
%
%%%%%%%%%%%%%%%%%%%%%%%%%%%%%

\documentclass[11pt]{article}

% packages without options
\usepackage{amsmath,bbm}

% geometry
\usepackage[margin=2cm]{geometry}






\title{
% alternative title : A Model-oriented Perspectivist Theory of [Urban ?] [Complex ?] Systems
A Reflexive Theory for the Study of Socio-Technical Systems
\bigskip\\
\textit{Working Paper}
}
\author{\noun{Juste Raimbault}}
\date{18th November 2015}


\maketitle

\justify


\begin{abstract}

\end{abstract}



\section*{Introduction}

The structural misunderstandings between Social Sciences and Humanities on one side, and so-called Exact Sciences on the other side, far from being a generality, seems to have however a significant impact on the structure of scientific knowledge~\cite{2015arXiv151103981H}. In particular, the place of theory (and indeed the signification of this term itself) in the elaboration of knowledge has a totally different place, partly because of the different \emph{perceived complexities}\footnote{We used the term \emph{perceived} as most of systems studied by physics might be described as simple whereas they are intrinsically complex and indeed not well understood~\cite{laughlin2006different}.} of studied objects : for example, mathematical constructions and by extent theoretical physics are \emph{simple} in the sense that they are mostly entierely analytically solvable, whereas Social Science subjects such as humans or society (to give a \emph{clich{\'e}} exemple) are \emph{complex} in the sense of complex systems\footnote{for which no unified definition exists }.



\section*{Objectives}




\section*{Construction of the theory}




\section*{Application : co-evolution of subsystems}



\section*{Discussion}



\section*{Conclusion}





%%%%%%%%%%%%%%%%%%%%
%% Biblio
%%%%%%%%%%%%%%%%%%%%

\bibliographystyle{apalike}
\bibliography{/Users/Juste/Documents/ComplexSystems/CityNetwork/Biblio/Bibtex/CityNetwork}


\end{document}
