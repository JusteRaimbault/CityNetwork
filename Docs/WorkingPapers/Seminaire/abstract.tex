\documentclass[11pt]{article}

% general packages without options
\usepackage{amsmath,amssymb,bbm}
% graphics
\usepackage{graphicx}
% text formatting
\usepackage[document]{ragged2e}
\usepackage{pagecolor,color}

\newcommand{\noun}[1]{\textsc{#1}}

\usepackage[utf8]{inputenc}
\usepackage[T1]{fontenc}
% geometry
\usepackage[margin=2cm]{geometry}

\usepackage{multicol}
\usepackage{setspace}

\usepackage{natbib}
\setlength{\bibsep}{0.0pt}

%\usepackage[french]{babel}

% layout : use fancyhdr package
%\usepackage{fancyhdr}
%\pagestyle{fancy}

% variable to include comments or not in the compilation ; set to 1 to include
\def \draft {1}


% writing utilities

% comments and responses
%  -> use this comment to ask questions on what other wrote/answer questions with optional arguments (up to 4 answers)
\usepackage{xparse}
\usepackage{ifthen}
\DeclareDocumentCommand{\comment}{m o o o o}
{\ifthenelse{\draft=1}{
    \textcolor{red}{\textbf{C : }#1}
    \IfValueT{#2}{\textcolor{blue}{\textbf{A1 : }#2}}
    \IfValueT{#3}{\textcolor{ForestGreen}{\textbf{A2 : }#3}}
    \IfValueT{#4}{\textcolor{red!50!blue}{\textbf{A3 : }#4}}
    \IfValueT{#5}{\textcolor{Aquamarine}{\textbf{A4 : }#5}}
 }{}
}

% todo
\newcommand{\todo}[1]{
\ifthenelse{\draft=1}{\textcolor{red!50!blue}{\textbf{TODO : \textit{#1}}}}{}
}


\makeatletter


\makeatother


\begin{document}







\title{Modeling Urban Morphogenesis
\bigskip\\
\textit{Séminaire Equipe Paris - 20th October 2017}
}
\author{\noun{Juste Raimbault}$^{1,2}$\medskip\\
$^1$ UMR CNRS 8504 Géographie-cités\\
$^2$ UMR-T IFSTTAR 9403 LVMT
}
\date{}

\maketitle

\justify

\pagenumbering{gobble}


\textbf{Keywords : }\textit{Morphogenesis; Urban Morphogenesis; Aggregation-diffusion; Co-evolution}

\bigskip

This presentation aims at demonstrating the abilities of parsimonious urban morphogenesis models to capture complex processes in urban systems. We first clarify the definition of Morphogenesis, from an interdisciplinary perspective, by differentiating it from self-organization with the requirement of an emergent architecture through strong causal links between form and function.


We study then a stochastic model of urban growth generating spatial distributions of population densities at an intermediate mesoscopic scale. The model is based on the antagonist interplay between the two opposite abstract processes of aggregation (preferential attachment) and diffusion (urban sprawl). Introducing indicators to quantify precisely urban form, the model is first statistically validated and intensively explored to understand its complex behavior across the parameter space. We compute real morphological measures on local areas of size 50km covering all European Union, and show that the model can reproduce most of existing urban morphologies in Europe. It implies that the morphological dimension of urban growth processes at this scale are sufficiently captured by the two abstract processes of aggregation and diffusion.

The last part of the presentation is devoted to extend the previous model, in order to capture co-evolution processes between a transportation network and the population distribution. A growing road network is added, for which various heuristics are tested in a multi-modeling fashion (random growth, deterministic and random potential breakdown, cost-driven growth, and biological network growth). The network influences back population growth through the influence of network measures and accessibility on the aggregation potential. The extended model is calibrated both on morphological and network indicators, for which real values are computed on the same areas. The different heuristics are all complementary to cover the real topological space, witnessing the manifestation of various processes in different geographical cases. We then investigate the proximity to data at the second order, in the sense of correlation matrices between indicators, for which the distance appears to be low in some cases. We finally study regimes of lagged correlations between local variables, to grasp dynamical relations that we call causality regimes. Their diversity across the parameter space suggest that the model effectively integrates some of the complexity of co-evolution processes between the road network and population distribution.







%%%%%%%%%%%%%%%%%%%%
%% Biblio
%%%%%%%%%%%%%%%%%%%%
%\tiny

%\begin{multicols}{2}

%\setstretch{0.3}
%\setlength{\parskip}{-0.4em}


%\bibliographystyle{apalike}
%\bibliography{/Users/juste/ComplexSystems/CityNetwork/Biblio/Bibtex/CityNetwork}%,biblio}
%\end{multicols}



\end{document}

