\documentclass[11pt]{article}

% general packages without options
\usepackage{amsmath,amssymb,bbm}
% graphics
\usepackage{graphicx}
% text formatting
\usepackage[document]{ragged2e}
\usepackage{pagecolor,color}

\newcommand{\noun}[1]{\textsc{#1}}

\usepackage[utf8]{inputenc}
\usepackage[T1]{fontenc}
% geometry
\usepackage[margin=2cm]{geometry}

\usepackage{multicol}
\usepackage{setspace}

\usepackage{natbib}
\setlength{\bibsep}{0.0pt}

%\usepackage[french]{babel}

% layout : use fancyhdr package
%\usepackage{fancyhdr}
%\pagestyle{fancy}

% variable to include comments or not in the compilation ; set to 1 to include
\def \draft {1}


% writing utilities

% comments and responses
%  -> use this comment to ask questions on what other wrote/answer questions with optional arguments (up to 4 answers)
\usepackage{xparse}
\usepackage{ifthen}
\DeclareDocumentCommand{\comment}{m o o o o}
{\ifthenelse{\draft=1}{
    \textcolor{red}{\textbf{C : }#1}
    \IfValueT{#2}{\textcolor{blue}{\textbf{A1 : }#2}}
    \IfValueT{#3}{\textcolor{ForestGreen}{\textbf{A2 : }#3}}
    \IfValueT{#4}{\textcolor{red!50!blue}{\textbf{A3 : }#4}}
    \IfValueT{#5}{\textcolor{Aquamarine}{\textbf{A4 : }#5}}
 }{}
}

% todo
\newcommand{\todo}[1]{
\ifthenelse{\draft=1}{\textcolor{red!50!blue}{\textbf{TODO : \textit{#1}}}}{}
}


\makeatletter


\makeatother


\begin{document}







\title{Modeling Urban Morphogenesis
\bigskip\\
\textit{Séminaire Equipe Paris - 20th October 2017}
}
\author{\noun{Juste Raimbault}$^{1,2}$\medskip\\
$^1$ UMR CNRS 8504 Géographie-cités\\
$^2$ UMR-T IFSTTAR 9403 LVMT
}
\date{}

\maketitle

\justify

\pagenumbering{gobble}

\paragraph{Slide 1 et 2}

si on s'interesse au reseau de rues parisien, temoignages de mulptiples processes, d'une dynamique temporelle, d'une croissance de la forme

mais chamgement d'echelle : autres processus, independance ?


\paragraph{Slide 3}

Hiostoire de la morphogenese : Darcy Thompson embryon, puis Turing.

``grille analytique'' : physique, bio, artificiel vs archi/non archi

\paragraph{Slide 4}

def et insister sur imbrication des notions

\paragraph{Slide 5}

ici def du champ scientifique et de la Q de recherche.


% - pourquoi mettre ton adresse mail @polytechnique? C'est un détail certes
%    -> erreur
% - pourquoi faire ta présentation en anglais? C'est un détail aussi, mais un peu moins
%.   -> tout le monde ne parle pas francais
% - Les deux premières cartes : on ne peut pas bien savoir ce que tu vas dire. Mais justement : il est certainement pertinent de structurer ce propos liminaire par des points précis.
% > on veut savoir ce que tu souhaites modéliser? voire, pour quoi faire? (aider à la décision, aider à la compréhension, etc. etc.)
% - what is morphogenisis? : avant cela, dire la place que cela va avoir dans ta présentation, sinon cela fait un peu "parachuté"
%> c'est sympa d'avoir 6 "snapshots" mais il faut placer cela dans une grille analytique, sinon on en ressort juste avec l'idée que "ca existe, et il y a plusieurs choses" : c'est léger
%- la slide avec self organization > morphogenesis > autopoeisis > life n'est pas du tout compréhensible : supprimer cette ligne (qu'apporte t elle?) et mettre surtout en évidence la définition et la diversité des approches entre disciplines
% c'est central pour la comprehension du concept
> sur la modélisation : c'est curieux je trouve que tu te cites en premier. sans doute pas le plus pertinent : le but de cette présentation n'est pas que tu dises "j'ai déjà publié plein de trucs" mais "voici ce que je fais".
A ce stade de la présentation, on ne sait toujours pas : 
> quel est le champ scientifique que tu te proposes d'augmenter?
> quels verrous tu as identifié
> ce que vont faire les modèles que tu vas présenter, en lien avec ces verrous
- du coup tu parles d'un modèle, mais on n'a aucune idée de pourquoi. c'est cela qu'il faut amener. a la limite, pas très grave si tu ne vas pas plus loin que la présentation des principes du modèle. Mais le lecteur doit absolument comprendre sa place dans ton discours.
- sur la slide avec tes 4 cartes de "territoires synthétiques" (je dis cela comme cela, discutable) : en garder seulement 2 pour avoir la place de discuter
- les sorties de modèles ne sont pas du tout amenées. c'est absolument impossible à comprendre. J'ai peine à croire qu'il soit intéressant de garder les 4 courbes. Choisis en une et décortique la. Par exemple, que sont les phases? en quoi est-ce intéressant que les indicateurs ne se comportent pas de façon monotone? (en plus 95% de tes courbes sont quasi monotones donc l'affirmation est spécieuse)
- la slide "path dependance and frozen accidents" à supprimer car absolument impossible à comprendre
- la partie "real data" : il faut repartir du début pour le lecteur : que cherches tu à faires? quelles données tu utilises? quels indicateurs, etc. c'est 4 5 slides et pas une.
- idem pour la calibration du modèle : la courbe PC1 PC2 est jolie, c'est un bon point d'arrivée... mais il faut que l'auditeur y arrive... et que tu l'aides à comprendre. (par exemple, moi, je ne comprends pas : que sont les points? que veut dire PC1 PC2? que sont les snapshots? ; rien n'est évident)
Je pense que tu dois ensuite t'arrêter là. C'est trop long sinon (enfin je n'ai plus en tête précisément la durée). Là tu parles d'un modèle et de sa calibration dans un champ hyper compliqué. C'est déjà bien. La littérature que tu proposes n'est pas étoffée bref pour faire une présentation carrée il y a déjà du boulot. la partie génération de réseau, mets là en "perspectives" en disant (tant mieux pour toi) que tu as déjà des billes là dessus.
- et penses à mettre des numéros sur tes slides










%%%%%%%%%%%%%%%%%%%%
%% Biblio
%%%%%%%%%%%%%%%%%%%%
%\tiny

%\begin{multicols}{2}

%\setstretch{0.3}
%\setlength{\parskip}{-0.4em}


%\bibliographystyle{apalike}
%\bibliography{/Users/juste/ComplexSystems/CityNetwork/Biblio/Bibtex/CityNetwork}%,biblio}
%\end{multicols}



\end{document}

