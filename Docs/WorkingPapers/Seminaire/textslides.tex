\documentclass[11pt]{article}

% general packages without options
\usepackage{amsmath,amssymb,bbm}

\usepackage[margin=1cm]{geometry}

\begin{document}




\paragraph{Intro}

pres en anglais necessaire ? slides au cas ou.

sujet general de ma these. ``Comment définir et/ou caractériser les processus de co-évolution entre réseaux de transports et territoires ? Comment modéliser ces processus, à quelles échelles et par quelles ontologies ?''

je presente une approche particuliere qui est une porte d'entree.

rassurer vous, simplifie et intelligible au max.



\paragraph{Slides Intro Narrative}

si on s'interesse au reseau de rues parisien, temoignages de mulptiples processes superposés, d'une dynamique temporelle, d'une croissance de la forme.

Règles de croissance de la forme ? lien avec fonction ? exemples : dependance au chemin du reseau viaire. Haussman.

mais chamgement d'echelle : autres processus, independance ? Meme systeme ?





\paragraph{Slide Def Morph}

Initialement from biologym empryology.

definition dico : processus d'emergence de la forme

Histoire de la morphogenese : Darcy Thompson embryon, puis Turing propose reaction diffusion equatiosn.

\paragraph{Slide Examples}

``grille analytique'' : physique, bio, artificiel vs archi/non archi

\paragraph{Slide Def Interdisc}

Point de vue interdisc pour isoler une def consistente. champs: dev bio, territorial science, artificial life, psychology.

proposition d'un cadre de notion ``meta-epistemo'' (dont l'instanciation donne une epistemo).

definir les termes.

Def forme : structure topologique / geometrique, avec propriete d'invariance (rotation, translation).

Def fonction : role dans des chaines de processus, dans perspective teleonomique. exemples : fonction d'un organe, fonction d'un programme, fonction urbaine : par exemple fonction de reseau.

Cela a plusieurs niveau - aparté sur emergence faible/emergence forte - lien avec bifurcation (def) - dependance au chemin

definir le morphogneetic engineering : bottom-up architecture ; bio-mimetism.

formuler la definition.

interet ? coherence de la notion ; transferts possibles de connaissance : orientation des modeles, nouvelle vue des concepts theoriques (ex coevol)

\textbf{revenir sur slide examples : selon contexte, what is morphogen and what is not.}

\paragraph{Slide Models for Urban morphogenesis}

ici def du champ scientifique et de la Q de recherche : insister sur interdisc (but pas urban simulation) ; dev modele a visee de comprehension.

illustrer par le RBD.


\paragraph{Slide Rationale}

def border of chaos

def scaling laws, why are they important. (macro features)

recall Turing reaction-diffusion.

explain principle of DLA.

\paragraph{Slide Model description}

precise each parameter.

expliquer espace morphologique, reduction de l'espace morph.

\paragraph{Slide Example}

importance of ``convergence level'' : $P_m / N_G$

narrate each territorial shape.


\paragraph{Slide Model behavior}

expliquer pourquoi calcul intensif

expliquer convergence indicateurs (pas montre)

detailler behavior : phase transition en fonction de alpha - deplacement point critique en fonction $P_m/N_G$ - inversion role de beta - presence d'un minimum : contre-intuitif ! signification : la hierarchie maximale ne donne le systeme le plus hierarchique, pour les faibles diffusions.




\paragraph{Slide Frozen Accidents}

Definir et illustrer la dependance au chemin. ici simplification 1D du modele (explication concrete du temps/espace) : visu claire. patterns echelle macro sont invariables, tandis que micro est fortement path dep, ``chaotique''



\paragraph{Slide Empirical data}

expliquer procedure de calcul. puis classification (pas detailler)


\paragraph{Slide Model calibration}

PC1 = moran + distance ; PC2 = slope + entropy

most real situation fall in the region with intermediate $\alpha$ but quite varying $\beta$

some config difficult : distance high : many highly aggregated centers.



\paragraph{Slide PSE}

mentionner methodes explo, openmole : idee de PSE - potentiel ici : lower bound.



\paragraph{Slide transition}

pourquoi les reseaux sont importants : Dupuy potential/realized networks.

interessant, car fonctionnels mais aussi morphologie propre : typique de morphogenese.


\paragraph{Slide terrain}

le probleme des interactions reseaux territoires ;

detailler exemple HSR : immoobilier suit plus ou moins ; reorganisation territoriale ; attente acteurs gouvernance.

notion de co-evolution : introduire.


\paragraph{Slide modele}

expliquer origines (deux c adres combines) ; rationelle dans la morphogenese et comment cpature la coevol.



\paragraph{Slide specification}

workflow simplifie

\paragraph{Slide Network}

pas de consensus sur network grwoth processes / plus flou : propice au multimodeling

atouts de diverses approches : auto-orga vs planification.


\paragraph{Slide Example Urban Shapes}

Valeur extreme des parameters pour al patch-value : X-driven.


\paragraph{Slide Example Networks}

visual properties of all networks

\paragraph{Slide Feasible space}

Precise fixed density : weak coupling, pas coevol (revient a fixer $N_G = 0$)

Preciser real networks. network indicators : centralités, efficiency, diameter.

selon classes morpho, plus ou moins facile selon heuristiques.


\paragraph{Slide Calibration}

insister sur complemetarite des heuirstiques reseau. peu influence celle ci sur morpho urbaine : hypothese : echelles de temps ?

procedure de calib : iteratif - d'abord indic, puis plus proche matrice de correlation, estimee sur simu proches et repets poru synth, estimee dans l'espace pour reel.

random heuristic pefform bien sur correlations : correlations nulles. biologique trop contraint dans processus ?


\paragraph{Slide Regimes}

reg 1 : network suit pop

reg 2 : idem mais pas dans strucutre locale (pop road $\simeq 0$)

reg3 : nw suit mais que local, pas structure. 

reg 4 : bw anticause access : network and pop avoid congestion ; plus nw suit : somehow circulaire, vraie co-evol ?

\paragraph{Slide Dicsussion}

implication : besoin de la fonction pour reproduire la forme ? (besoin de morphogenese ?) : Q ouverte

developpement : interet plus large, theorie evolutive, model overfitting.

\paragraph{Slide Conclusion}

ouverture : positionnement epistemo.

details science ouverte - remerciements.







% > on veut savoir ce que tu souhaites modéliser? voire, pour quoi faire? (aider à la décision, aider à la compréhension, etc. etc.)
% - what is morphogenisis? : avant cela, dire la place que cela va avoir dans ta présentation, sinon cela fait un peu "parachuté"
%> c'est sympa d'avoir 6 "snapshots" mais il faut placer cela dans une grille analytique, sinon on en ressort juste avec l'idée que "ca existe, et il y a plusieurs choses" : c'est léger
%- la slide avec self organization > morphogenesis > autopoeisis > life n'est pas du tout compréhensible : supprimer cette ligne (qu'apporte t elle?) et mettre surtout en évidence la définition et la diversité des approches entre disciplines
% c'est central pour la comprehension du concept
%> sur la modélisation : c'est curieux je trouve que tu te cites en premier. sans doute pas le plus pertinent : le but de cette présentation n'est pas que tu dises "j'ai déjà publié plein de trucs" mais "voici ce que je fais".
%A ce stade de la présentation, on ne sait toujours pas : 
%> quel est le champ scientifique que tu te proposes d'augmenter?
%> quels verrous tu as identifié
%> ce que vont faire les modèles que tu vas présenter, en lien avec ces verrous
%- du coup tu parles d'un modèle, mais on n'a aucune idée de pourquoi. c'est cela qu'il faut amener. a la limite, pas très grave si tu ne vas pas plus loin que la présentation des principes du modèle. Mais le lecteur doit absolument comprendre sa place dans ton discours.
%- sur la slide avec tes 4 cartes de "territoires synthétiques" (je dis cela comme cela, discutable) : en garder seulement 2 pour avoir la place de discuter
%- les sorties de modèles ne sont pas du tout amenées. c'est absolument impossible à comprendre. J'ai peine à croire qu'il soit intéressant de garder les 4 courbes. Choisis en une et décortique la. Par exemple, que sont les phases? en quoi est-ce intéressant que les indicateurs ne se comportent pas de façon monotone? (en plus 95% de tes courbes sont quasi monotones donc l'affirmation est spécieuse)
%- la partie "real data" : il faut repartir du début pour le lecteur : que cherches tu à faires? quelles données tu utilises? quels indicateurs, etc. c'est 4 5 slides et pas une.
% - idem pour la calibration du modèle : la courbe PC1 PC2 est jolie, c'est un bon point d'arrivée... mais il faut que l'auditeur y arrive... et que tu l'aides à comprendre. (par exemple, moi, je ne comprends pas : que sont les points? que veut dire PC1 PC2? que sont les snapshots? ; rien n'est évident)
% Je pense que tu dois ensuite t'arrêter là. C'est trop long sinon (enfin je n'ai plus en tête précisément la durée). Là tu parles d'un modèle et de sa calibration dans un champ hyper compliqué. C'est déjà bien. La littérature que tu proposes n'est pas étoffée bref pour faire une présentation carrée il y a déjà du boulot. la partie génération de réseau, mets là en "perspectives" en disant (tant mieux pour toi) que tu as déjà des billes là dessus.
% - et penses à mettre des numéros sur tes slides










%%%%%%%%%%%%%%%%%%%%
%% Biblio
%%%%%%%%%%%%%%%%%%%%
%\tiny

%\begin{multicols}{2}

%\setstretch{0.3}
%\setlength{\parskip}{-0.4em}


%\bibliographystyle{apalike}
%\bibliography{/Users/juste/ComplexSystems/CityNetwork/Biblio/Bibtex/CityNetwork}%,biblio}
%\end{multicols}



\end{document}

