%%%%%%%%%%%%%%%%%%%%%%%%%%%%%
% Standard header for working papers
%
% WPHeader.tex
%
%%%%%%%%%%%%%%%%%%%%%%%%%%%%%

\documentclass[11pt]{article}

% packages without options
\usepackage{amsmath,bbm}

% geometry
\usepackage[margin=2cm]{geometry}






\title{Using Synthetic Data to Control on Meta-Parameters for a Model of Simulation\bigskip\\
\textit{Working Paper}
}
\author{\noun{Juste Raimbault}}
\date{Wednesday 25th March}


\maketitle

\begin{abstract}
We formalize briefly a proposition of method that would allow to add controls on meta-parameters, in the sense of parameters driving the represented system at a higher temporal and spatial scale, for a model of simulation. We make the hypothesis that such method is valid under constraints of disjonction for scales and/or ontologies between the model of simulation and the domain of meta-parameters.
\end{abstract}


\paragraph{Context}

Let $M_{m}$ a stochastic model of simulation, which inputs are to simplify initial conditions $D_0$ and parameters $\vec{\alpha}$, and output $M_{m}\left[\vec{\alpha},D_0\right](t)$ at a given time $t$. We assume that it is partially data-driven in the sense that $D_0$ is supposed to represent a real situation at a given time, and model performance is measured by the distance of its output at final time to the real situation at the corresponding time, i.e. error function is of the form $\norm{\Eb{\vec{g}(M_{m}\left[\vec{\alpha},D_0\right](t_f))}-\vec{g}(D_f)}$ where $\vec{g}$ is a deterministic field corresponding to given indicators.



\paragraph{Position of the Problem}

Evaluating the model on real data is rapidly limited in control possibilities, being restricted to the search of datasets allowing natural control groups. Furthermore, statistical behaviors are generally poorly caracterized because of the small number of realizations. Working with synthetic data first allows to solve this issue of robustness of statistics, and then gives possibilities of control on some ``meta-parameters'' in the sense described before.


\paragraph{Proposed method}











%%%%%%%%%%%%%%%%%%%%
%% Biblio
%%%%%%%%%%%%%%%%%%%%

\bibliographystyle{apalike}
\bibliography{/Users/Juste/Documents/ComplexSystems/CityNetwork/Biblio/Bibtex/CityNetwork}


\end{document}
