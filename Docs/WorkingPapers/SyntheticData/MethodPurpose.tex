%%%%%%%%%%%%%%%%%%%%%%%%%%%%%
% Standard header for working papers
%
% WPHeader.tex
%
%%%%%%%%%%%%%%%%%%%%%%%%%%%%%

\documentclass[11pt]{article}

% packages without options
\usepackage{amsmath,bbm}

% geometry
\usepackage[margin=2cm]{geometry}






\title{On the Sensitivity of Computational Models to Meta-parameters\bigskip\\
\textit{Working Paper}
}
\author{\noun{Juste Raimbault}}
\date{Monday June 22th}


\maketitle

\justify


%\begin{abstract}

%\end{abstract}


%%%%%%%%%%%%%%%%%%%%
\section{Introduction}

An advanced knowledge of the behavior of computational models on their parameter space is a necessary condition for deductions of thematic conclusions or their practical application~\cite{banos2013pour}. But the choice of varying parameters is always subjective, as some may be fixed by a real-world parametrization, or other may be interpreted as arbitrarily fixed initial conditions. It raises methodological and epistemological issues for the sensitivity analysis, as the scope of the model may become ill-defined.

Let consider the concrete example of the Schelling Segregation model~\cite{schelling1971dynamic}. One of its crucial features on which the literature has been rather controversial is the influence of the spatial structure of the container on which agents evolve~(\textit{Biblio Marion}). The thematic aim of the project developed in~\cite{cottineau2015revisiting} is to clarify this point through a systematic model exploration. A methodological contribution is the construction of a framework allowing the analysis of the sensitivity of models to \emph{meta-parameters}, i.e. to parameters considered as fixed initial conditions (e.g. the spatial structure for the Schelling model), or to parameters of another model generating an initial configuration, as detailed in Fig.~\ref{} \textit{[insert scheme describing the approach]}, where we have thus a \emph{simple coupling} between models (serial coupling). The benefits of such an approach are various but include for example the knowledge of model behavior in an extended frame, the possibility of statistical control when regressing model outputs, a finer exploration of model derivatives than with a naive approach. Some remarks can be made on the approach : \textit{critiques de Robin}
\begin{itemize}
\item What knowledge are brought by adding the upstream model, rather than for example in the Schelling case exploring a large set of initial geometries ? 

$\rightarrow$ \textit{to obtain a sufficiently large set of initial configuration, one quickly needs a model to generate them ; in that case a quasi-random generation followed by a filtering on morphological constraint will be a morphogenesis model, which parameters are the ones of the generation and the filtering methods. Furthermore, as detailed in next section, the determination of the derivative of the downstream model is made possible by the coupling and knowledge of the upstream model.}
\item Statistical noise is added by coupling models

$\rightarrow$ \textit{Repetitions needed for convergence are indeed larger as the final expectance has to be determined by repeating on the first times the second model ; but it is exactly the same as exploring directly many configuration, to obtain statistical robustness in that case one must repeat on similar configurations.}

\item Complexity is added by coupling models

$\rightarrow$ \textit{In the sense of Varenne~[Ref], coupling is simple and no complexity is thus added. -- develop}

\end{itemize}
 



%%%%%%%%%%%%%%%%%%%%
\section{Formal Analysis}

\subsection{Deterministic Formulation}

One has 

\[
\partial_{\alpha}\left[M_u \circ M_d\right] = \left(\partial_{\alpha} M_u \circ M_d \right)\cdot \partial_{\alpha} M_d
\]

$\rightarrow$ \textit{the sensitivity of the downstream model (Schelling) can be determined by studying the serial coupling and the upstream model ; thematic knowledge : sensitivity to an implicit meta-parameter ; and computational gain : generation of controlled differentiates in the ``initial space'' is quasi impossible.}

\subsection{Stochasticity}

\textit{Dealing with stochasticity in simply coupled models $\rightarrow$ no convergence pb as $\Eb{X}=\Eb{\Eb{X|Y}}$}


%%%%%%%%%%%%%%%%%%%%
\section{Application to Statistical Control}












%%%%%%%%%%%%%%%%%%%%
%% Biblio
%%%%%%%%%%%%%%%%%%%%

\bibliographystyle{apalike}
\bibliography{/Users/Juste/Documents/ComplexSystems/CityNetwork/Biblio/Bibtex/CityNetwork,/Users/Juste/Documents/ComplexSystems/Biblio/Culture/Culture/culture}


\end{document}
