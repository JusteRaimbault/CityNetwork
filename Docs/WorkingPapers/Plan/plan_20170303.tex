%%%%%%%%%%%%%%%%%%%%%%%%%%%%%
% Standard header for working papers
%
% WPHeader.tex
%
%%%%%%%%%%%%%%%%%%%%%%%%%%%%%

\documentclass[11pt]{article}



%%%%%%%%%%%%%%%%%%%%%%%%%%
%% TEMPLATES
%%%%%%%%%%%%%%%%%%%%%%%%%%


% Simple Tabular

%\begin{tabular}{ |c|c|c| } 
% \hline
% cell1 & cell2 & cell3 \\ 
% cell4 & cell5 & cell6 \\ 
% cell7 & cell8 & cell9 \\ 
% \hline
%\end{tabular}





%%%%%%%%%%%%%%%%%%%%%%%%%%
%% Packages
%%%%%%%%%%%%%%%%%%%%%%%%%%



% encoding 
\usepackage[utf8]{inputenc}
\usepackage[T1]{fontenc}


% general packages without options
\usepackage{amsmath,amssymb,amsthm,bbm}

% graphics
\usepackage{graphicx,transparent,eso-pic}

% text formatting
\usepackage[document]{ragged2e}
\usepackage{pagecolor,color}
%\usepackage{ulem}
\usepackage{soul}


% conditions
\usepackage{ifthen}


\usepackage{natbib}


%%%%%%%%%%%%%%%%%%%%%%%%%%
%% Maths environment
%%%%%%%%%%%%%%%%%%%%%%%%%%

%\newtheorem{theorem}{Theorem}[section]
%\newtheorem{lemma}[theorem]{Lemma}
%\newtheorem{proposition}[theorem]{Proposition}
%\newtheorem{corollary}[theorem]{Corollary}

%\newenvironment{proof}[1][Proof]{\begin{trivlist}
%\item[\hskip \labelsep {\bfseries #1}]}{\end{trivlist}}
%\newenvironment{definition}[1][Definition]{\begin{trivlist}
%\item[\hskip \labelsep {\bfseries #1}]}{\end{trivlist}}
%\newenvironment{example}[1][Example]{\begin{trivlist}
%\item[\hskip \labelsep {\bfseries #1}]}{\end{trivlist}}
%\newenvironment{remark}[1][Remark]{\begin{trivlist}
%\item[\hskip \labelsep {\bfseries #1}]}{\end{trivlist}}

%\newcommand{\qed}{\nobreak \ifvmode \relax \else
%      \ifdim\lastskip<1.5em \hskip-\lastskip
%      \hskip1.5em plus0em minus0.5em \fi \nobreak
%      \vrule height0.75em width0.5em depth0.25em\fi}



%% Commands

\newcommand{\noun}[1]{\textsc{#1}}


%% Math

% Operators
\DeclareMathOperator{\Cov}{Cov}
\DeclareMathOperator{\Var}{Var}
\DeclareMathOperator{\E}{\mathbb{E}}
\DeclareMathOperator{\Proba}{\mathbb{P}}

\newcommand{\Covb}[2]{\ensuremath{\Cov\!\left[#1,#2\right]}}
\newcommand{\Eb}[1]{\ensuremath{\E\!\left[#1\right]}}
\newcommand{\Pb}[1]{\ensuremath{\Proba\!\left[#1\right]}}
\newcommand{\Varb}[1]{\ensuremath{\Var\!\left[#1\right]}}

% norm
\newcommand{\norm}[1]{\left\lVert #1 \right\rVert}



% argmin
\DeclareMathOperator*{\argmin}{\arg\!\min}


% amsthm environments
\newtheorem{definition}{Definition}
\newtheorem{proposition}{Proposition}
\newtheorem{assumption}{Assumption}

%% graphics

% renew graphics command for relative path providment only ?
%\renewcommand{\includegraphics[]{}}


\usepackage{url}





% geometry
\usepackage[margin=2cm]{geometry}



% changes

\usepackage{soul}
\soulregister\cite7
\soulregister\citep7
\soulregister\ref7

\usepackage[final]{changes}
%\usepackage{changes}


\setaddedmarkup{\textcolor{black}{\hl{#1}}}
\setdeletedmarkup{\textcolor{red}{\sout{#1}}}



\usepackage{CJKutf8}
%\begin{CJK*}{UTF8}{zhsong}
%文章内容。
%\clearpage\end{CJK*}
\newcommand{\cn}[1]{
  \begin{CJK*}{UTF8}{gbsn}
  #1
  \end{CJK*}
}



% layout : use fancyhdr package
%\usepackage{fancyhdr}
%\pagestyle{fancy}
%
%\makeatletter
%
%\renewcommand{\headrulewidth}{0.4pt}
%\renewcommand{\footrulewidth}{0.4pt}
%\fancyhead[RO,RE]{}
%\fancyhead[LO,LE]{Models for the co-evolution of cities and networks}
%\fancyfoot[RO,RE] {\thepage}
%\fancyfoot[LO,LE] {}
%\fancyfoot[CO,CE] {}
%
%\makeatother
%

%%%%%%%%%%%%%%%%%%%%%
%% Begin doc
%%%%%%%%%%%%%%%%%%%%%

\begin{document}









\title{Vers des Modèles Couplant Développement Urbain et Croissance des Réseaux de Transport\bigskip\\
\textit{Proposition de Plan}
}
\author{\noun{Juste Raimbault}}
%\date{Novembre 2016}


\maketitle

\justify


\begin{abstract}
\end{abstract}


\part*{Introduction}

\textit{Introduction du sujet par exemples concrets ; cadre scientifique ; interdisciplinarité ; sciences des systèmes complexes ; complexité en géographie}

%%%%%%%%%%%%%%%%%%%%%%%%%%
\part{Fondations}
%%%%%%%%%%%%%%%%%%%%%%%%%%


%%%%%%%%%%%%%%%%%%%%%%%%%%
\section{Interactions entre Réseaux et Territoires}

\subsection{Réseaux et Territoires}

\textit{Revue de Littérature thématique ; construction de la question de recherche : introduction progressive de la problématique de co-évolution, précision des objets (réseaux et territoires)}


\subsection{Etude de cas}

\textit{Collection de situations concrètes d'interactions entre réseaux et territoires ; les cas du Grand Paris et du Delta de la Rivière des Perles} \todo{rassembler et synthétiser fiches de lecture}


\subsection{Recherche Qualitative}

\textit{Une expérience en observation flottante : les transports en région parisienne et dans le Delta de la Rivière des Perles} \todo{finir terrains ; écrire compte-rendus/interprétation}

\comment{Recherche quali : peut aller en début de micro ?}


\subsection{Synthèse des connaissances}

\textit{Mise en perspective de la connaissance produite par la thèse (faisant office d'annonce de plan) comme illustration de la co-évolution des connaissances en Géographie Théorique et Quantitative~\cite{raimbault2017theo}}




%%%%%%%%%%%%%%%%%%%%%%%%%%
\section{Modéliser les Interactions entre Réseaux et Territoires}

% chapitre état de l'art

\subsection{Etat de l'art}

\textit{Modéliser en Géographie Théorique et Quantitative ; revue inter-disciplinaire des modèles de croissance urbaine et de réseau}


\comment{Eventuellement : interviews ?}


\subsection{Une approche en épistémologie quantitative}


\comment{Pour les deux études, détails techniques en annexe}

\paragraph{Revue systématique algorithmique}

\textit{Etude algorithmique du paysage scientifique sur les interactions entre réseaux et territoires~\cite{raimbault2015models} : des domaines très cloisonnés}


\paragraph{Bibliométrie indirecte par hyperréseau}


\textit{Raffinement de l'étude précédente par couplage du réseau de citation au réseau sémantique : méthode présentée dans~\cite{raimbault2016indirect} ; application au sujet en cours} \todo{Soumettre papier Cybergeo (Scientometrics) ; appliquer à corpus réseau-territoire ; traduire l'article en remplaçant les résultats. ETA 1w}





%%%%%%%%%%%%%%%%%%%%%%%%%%
\section{Positionnements}


\subsection{Reproductibilité}

\textit{Etudes de cas sur la reproductibilité ; illustration concrète et leçons générales}

\subsection{Données massives et computation}

\textit{Pour un usage précautionneux des données massives et de la computation : rationnelle de~\cite{raimbault2016cautious}}


\subsection{Exploration des modèles}

\textit{Pour une connaissance plus fine et systématique du comportement des modèles : utilisation de données synthétiques pour un contrôle sur les conditions initiales (projet Space Matters \cite{cottineau2015revisiting})}


\subsection{Positionnement épistémologique}

\textit{Pour une science anarchiste (Feyerabend) ; compatibilité avec le Perspectivisme de Giere et pourquoi celui-ci est particulièrement adapté aux paradigmes de la complexité ; multiplicité des lectures de la thèse (voir annexe réflexivité, au delà d'une lecture linéaire)} \todo{à écrire}




%%%%%%%%%%%%%%%%%%%%%%%%%%
%\section{Méthodologie}

% methodo : soit ventilé, soit supprimé


\subsection{A Unified Framework for Models of Urban Growth}

\textit{The various model we will develop could enter a unified framework ; derivation of the link between Gibrat and Simon models}


\subsection{Sensitivity of Urban Scaling to City Definition}

\textit{Analytical validation of the sensitivity of scaling exponents to city definition in a simple model or urban form}

\subsection{Quantifying Robustness through Discrepancy}

\textit{Complex systems are by nature multi-objective : in the particular case of multi-attribute evaluations, we introduce a framework to quantify robustness independently of the model, based on data discrepancy~\cite{raimbault2016discrepancy}}


\subsection{Spatio-temporal correlations and causalities}

\textit{Linking spatial and temporal correlations of geographical indicators in simple cases ; a granger-causality method to identify spatio-temporal causalities}








%%%%%%%%%%%%%%%%%%%%%%%%%%
\part{Matériaux}
%%%%%%%%%%%%%%%%%%%%%%%%%%

\section{Interactions à l'Echelle Microscopique}

\subsection{Equilibre Utilisateur Statique}

\textit{Investigation de l'existence empirique de l'Equilibre Utilisateur Statique~\cite{raimbault2016investigating}}


\subsection{Système de Transport en partage}

\todo{adapter \cite{raimbault2015hybrid} pour montrer l'hétérogénéité et la complexité des interactions à l'échelle microscopique}



\subsection{Transactions immobilières et Grand Paris}

\todo{Recherche de correlations et/ou causalités entre transactions immobilières (base BIEN) et tracé du réseau du métro du Grand Paris}


\section{Morphogenèse Urbaine}

\subsection{Une approche interdisciplinaire de la Morphogenèse}

\textit{Construction épistémologique d'une définition unifiée de la morphogenèse~\cite{antelope2016interdisciplinary}}


\subsection{Morphogenèse Urbaine par Aggregation-Diffusion}

\textit{Modèle de croissance urbaine par processus d'aggregation diffusion, reproduit de manière fine l'ensemble des morphologies urbaines existantes en Europe} \todo{Article PlosOne à finaliser}


\subsection{Génération de systèmes corrélés}

\textit{Couplage faible du modèle précédent à une heuristique de génération de réseau, permet de générer des système couplés à la correlation contrôlée~\cite{raimbault2016generation}}


\section{Théorie Evolutive Urbaine}

\subsection{Correlations entre Forme Urbaine et Forme de Réseau}

\textit{Les correlations spatiales entre indicateurs de forme urbaine et de forme de réseau révèlent la non-stationnarité des interactions, qui peut être reliée à la non-ergodicité sous certaines hypothèses~\cite{raimbault2016cautious}}


\subsection{Causalités spatio-temporelles}


\textit{Exploration synthétique des régimes de causalité du modèle RBD~\cite{raimbault2014hybrid} par la méthode de granger étendue} \todo{Appliquer la méthode aux données sud-africaines - objectif : papier Sageo deadline 31 avril}


\subsection{Effets de Réseaux révélés par un modèle de croissance macroscopique}

\textit{Modèle de Gibrat étendu par interactions gravitaires au premier ordre, par retroaction des flux physiques au second ordre~\cite{raimbault2016models}, révèle effets de réseaux par validation du modèle étendu via critère d'Akaike empirique} \todo{finish papier ASAP, à soumettre à EPB}




%%%%%%%%%%%%%%%%%%%%%%%%%%
\part{Co-évolution}
%%%%%%%%%%%%%%%%%%%%%%%%%%

%%%%%%%%%%%%%%%%%%%%%%%%%%
\section{Co-evolution à l'échelle macroscopique}

\subsection{Extension du modèle d'interaction}

\todo{Rendre le gibrat-intercation co-évolutif ; valider et tester sur réseau de train (base Thévenin) et réseau d'autoroutes (base à créer)}

\subsection{Modèle SimpopSino}

\todo{Adaptation du modèle pour le système de ville Chinois ; première deadline : conférence Medium début juin}


\subsection{Autres applications}

\todo{Application à l'Afrique du Sud}




%%%%%%%%%%%%%%%%%%%%%%%%%%
\section{Co-evolution à l'échelle mesoscopique}

\subsection{Co-evolution des formes}

\paragraph{Modèle de Morphogenèse Urbaine}

\comment{Formulation très générique du modèle ?}

\todo{Rendre le couplage faible de~\cite{raimbault2016generation} fort, induisant un modèle de morphogenèse incluant la co-évolution ; calibration du modèle}

\paragraph{Comparaison des heuristiques de réseau}

\todo{Comparaison de différentes heuristiques de réseau couplées au modèle précédent (par exemple génération de réseau biologique \cite{raimbault2015labex}}



\subsection{Lutetia : un modèle de co-évolution incluant la gouvernance des systèmes de transports}

\paragraph{Modèle}

\textit{Modèle de co-évolution sur le temps long, couplant un LUTI à un module de gouvernance des transports basé sur la théorie des jeux, pour le développement du réseau~\cite{le2015modeling}}

\paragraph{Application au Delta de la Rivière des Perles}

\todo{Calibration et validation du modèle sur le Delta de la Rivière des Perles : objectif Article Transport Geography début mai}




%%%%%%%%%%%%%%%%%%%%%%%%%%
\section{Ouverture}



%%%%%%%%%%%%%%%%%%%%%%%%%%
% Constructions Théoriques 

\subsection{Une Théorie des Systèmes Territoriaux Co-évolutifs en Réseau}

\textit{Développement de la théorie géographique co-construite avec les autres domaines de la thèse, qui couple l'entrée morphogenétique avec la théorie évolutive des villes~\cite{raimbault:halshs-01422484}}


\subsection{Une Théorie abstraite pour modéliser les systèmes socio-techniques}

\textit{Méta-théorie pour formaliser des perspectives de modélisations multiples sur les systèmes socio-techniques} \todo{reste à développer action des modèles sur les données, y associer une structure d'action de monoïde}



\subsection{Perspectives}

\paragraph{Développements Spécifiques}

\textit{Projets de recherche détaillés issus de divers développements (par exemple communication scientifique 
\cite{serra2016game} ; épistémologie quantitative~\cite{raimbault2016techno} ; science ouverte
\cite{cybergeo20})}

\paragraph{Vers un Programme de Recherche}

\textit{Synthèse des axes de recherche révélés tout au long de la thèse, proposition d'un programme de longue durée pour l'étude des systèmes territoriaux complexes}



%%%%%%%%%%%%%%%%%%%%%%%%%%
\part*{Conclusion}
%%%%%%%%%%%%%%%%%%%%%%%%%%

\textit{Conclusion générale}



\part*{Annexes}



% not necessary
%\section{Une approche interdisciplinaire de la morphogenèse}

%\textit{texte complet de~\cite{antelope2016interdisciplinary}}




\section{Développements Techniques}


\textit{Dérivations Analytiques pour diverses parties de la thèse}


\section{Données synthétiques}

\textit{Développement de~\cite{raimbault2016generation} dans le champ de la Finance Quantitative}


\section{Exploration des Modèles}

\textit{Explorations raffinées pour certains modèles ; applications compagnon d'exploration interactive}


\section{Reflexivité}


\textit{Application des outils d'épistémologie quantitative à la thèse elle-même ; statistiques détaillées des différents projets ; graphe des concepts et parties de la thèse (application compagnon ?) et proposition de pistes alternatives de lecture}


\section{Bases de Données}


\textit{Description des bases crées dans le cadre de la thèse : réseau routier simplifié prou l'Europe ; traffic routier en Ile de France ; Données VLib sur 3ans ; Autoroutes dynamiques} \todo{pour la base topologique OSM, data paper (Scientific Data) ; pour la base VLib, data paper Cybergeo Data Papers ?}


\section{Logiciels et Packages}

\textit{Packages réutilisables développés dans le cadre de la thèse : largeNetworkR ; Scientific Corpus Mining}


\section{Architecture et Source}

\textit{Architecture et Source des modèles de simulation et d'analyse de données}


\section{Productivité}

\textit{Outils ouverts pour une productivité scientifique améliorée}




%%%%
% n'a pas sa place dans une thèse -> ?

%\section{Science et Art}

%\textit{Oeuvres d'art sérendipiteuses (composition graphique et poésie) produites dans le cadre de la thèse}


%\newpage

%%%%%%%%%%%%%%%%%%%%
%% Biblio
%%%%%%%%%%%%%%%%%%%%

\bibliographystyle{apalike}
\bibliography{biblio}


\end{document}
