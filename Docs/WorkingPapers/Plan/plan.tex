%% Commands

\newcommand{\noun}[1]{\textsc{#1}}

% command fort head of chapter citation
\newcommand{\headercit}[3]{
\begin{multicols}{2}
\phantom{}
\columnbreak
\textit{#1}

 - \noun{#2}~#3
\end{multicols}
}



%% Math

% Operators
\DeclareMathOperator{\Cov}{Cov}
\DeclareMathOperator{\Var}{Var}
\DeclareMathOperator{\E}{\mathbb{E}}
\DeclareMathOperator{\Proba}{\mathbb{P}}

\newcommand{\Covb}[2]{\ensuremath{\Cov\!\left[#1,#2\right]}}
\newcommand{\Eb}[1]{\ensuremath{\E\!\left[#1\right]}}
\newcommand{\Pb}[1]{\ensuremath{\Proba\!\left[#1\right]}}
\newcommand{\Varb}[1]{\ensuremath{\Var\!\left[#1\right]}}

% norm
\newcommand{\norm}[1]{\| #1 \|}

% independent
\newcommand{\indep}{\rotatebox[origin=c]{90}{$\models$}}


% amsthm environments
\newtheorem{definition}{Definition}
\newtheorem{proposition}{Proposition}
\newtheorem{assumption}{Assumption}
\newtheorem{lemma}{Lemma}

\newenvironment{proof}[1][Proof]{\begin{trivlist}
\item[\hskip \labelsep {\bfseries #1}]}{\end{trivlist}}




\newcommand{\qed}{\nobreak \ifvmode \relax \else
      \ifdim\lastskip<1.5em \hskip-\lastskip
      \hskip1.5em plus0em minus0.5em \fi \nobreak
      \vrule height0.75em width0.5em depth0.25em\fi}



%%%%%%%%%%%%%%%%%%%
%%  Additional packages
%%%%%%%%%%%%%%%%%%%

%\usepackage{subcaption}

\usepackage{amssymb}

\usepackage{multicol}

\usepackage{bbm}


%%%

\renewcommand{\PrelimText}{%
  \footnotesize[\,\today\ at \thistime\ -- \texttt{Thesis}~\myVersion\,]}


%%%%%%%%
% bilingual version
\usepackage{ifthen}

\newcommand{\bpar}[2]{
\ifthenelse{\thelanguage=0}{#1}{}
\ifthenelse{\thelanguage=1}{#2}{}
}

% note : using these commands make section disappear from ide outline, not really practical -> better use classical commands wrapped around \bpar

%\newcommand{\bchapter}[2]{\chapter{\bpar{#1}{#2}}}
%\newcommand{\bsection}[2]{\section{\bpar{#1}{#2}}}
%\newcommand{\bsubsection}[2]{\subsection{\bpar{#1}{#2}}}
%\newcommand{\bsubsubsection}[2]{\subsubsection{\bpar{#1}{#2}}}
%\newcommand{\bchapters}[2]{\chapter*{\bpar{#1}{#2}}}
%\newcommand{\bsections}[2]{\section*{\bpar{#1}{#2}}}
%\newcommand{\bsubsections}[2]{\subsection*{\bpar{#1}{#2}}}
%\newcommand{\bsubsubsections}[2]{\subsubsection*{\bpar{#1}{#2}}}

%\newcommand{\bcaption}[2]{\caption{\bpar{#1}{#2}}}


% only one optional arg with renewcomand : trick using bpar in the optionnal arg (should check other packages)
%\renewcommand{\section}[3][]{
%\ifthenelse{\equal{#1}{}}{
%\section{\bpar{#1}{#2}}
%}{
%\section[#1]{\bpar{#2}{#3}}
%}
%}
% -> RECURSIVE PB : WHY ?

%\renewcommand{\section}[2]{\section{\bpar{#1}{#2}}}
% http://tex.stackexchange.com/questions/22576/redefining-sectioning-commands



%%%%%%%%%%
%  Drafting

% writing utilities

% comments	 and responses
%  -> use this comment to ask questions on what other wrote/answer questions with optional arguments (up to 4 answers)
\usepackage{xparse}
\usepackage{ifthen}
\DeclareDocumentCommand{\comment}{m o o o o}
{\ifthenelse{\draft=1}{
    \textcolor{red}{\textbf{C : }#1}
    \IfValueT{#2}{\textcolor{blue}{\textbf{A1 : }#2}}
    \IfValueT{#3}{\textcolor{ForestGreen}{\textbf{A2 : }#3}}
    \IfValueT{#4}{\textcolor{red!50!blue}{\textbf{A3 : }#4}}
    \IfValueT{#5}{\textcolor{Aquamarine}{\textbf{A4 : }#5}}
 }{}
}


% todo
\newcommand{\todo}[1]{
\ifthenelse{\draft=1}{\textcolor{red!50!blue}{\textbf{TODO : \textit{#1}}}}{}
}



% provisory part, removed if not draft

\newcommand{\provisory}[1]{
\ifthenelse{\draft=1}{#1}{}
}















\title{Vers des Modèles Couplant Développement Urbain et Croissance des Réseaux de Transport\bigskip\\
\textit{Proposition de Plan}
}
\author{\noun{Juste Raimbault}}
%\date{Novembre 2016}


\maketitle

\justify


\begin{abstract}
\end{abstract}


\part*{Introduction}

\textit{Introduction du sujet par exemples concrets ; cadre scientifique ; interdisciplinarité ; sciences des systèmes complexes ; complexité en géographie}

%%%%%%%%%%%%%%%%%%%%%%%%%%
\part{Fondations}
%%%%%%%%%%%%%%%%%%%%%%%%%%


%%%%%%%%%%%%%%%%%%%%%%%%%%
\section{Interactions entre Réseaux et Territoires}

\subsection{Réseaux et Territoires}

\textit{Revue de Littérature thématique ; construction de la question de recherche : introduction progressive de la problématique de co-évolution, précision des objets (réseaux et territoires)}

\subsection{Modélisation}

\textit{Modéliser en Géographie Théorique et Quantitative ; revue inter-disciplinaire des modèles de croissance urbaine et de réseau}


\subsection{Cas d'étude}

\textit{Collection de situations concrètes d'interactions entre réseaux et territoires ; les cas du Grand Paris et du Delta de la Rivière des Perles} \todo{rassembler fiches de lecture} \todo{rassembler et synthétiser fiches de lecture}


\subsection{Recherche Qualitative}

\textit{Une expérience en observation flottante : les transports en région parisienne et dans le Delta de la Rivière des Perles} \todo{finir terrains ; écrire compte-rendus/interprétation}

\subsection{Synthèse des connaissances}

\textit{Mise en perspective de la connaissance produite par la thèse (faisant office d'annonce de plan) comme illustration de la co-évolution des connaissances en Géographie Théorique et Quantitative~\cite{raimbault2017theo}}




%%%%%%%%%%%%%%%%%%%%%%%%%%
\section{Positionnements}


\subsection{Reproductibilité}

\textit{Etude de cas sur la reproductibilité}

\subsection{Données massives et computation}

\textit{Pour un usage précautionneux des données massives et de la computation : rationnelle de~\cite{raimbault2016cautious}}


\subsection{Positionnement épistémologique}






%%%%%%%%%%%%%%%%%%%%%%%%%%
\section{Méthodologie}









%%%%%%%%%%%%%%%%%%%%%%%%%%
\section{Epistémologie Quantitative}

\subsection{Revue systématique algorithmique}

\textit{Etude algorithmique du paysage scientifique sur les interactions entre réseaux et territoires~\cite{raimbault2015models} : des domaines très cloisonnés}


\subsection{Bibliométrie indirecte par hyperréseau}


\textit{Raffinement de l'étude précédente par couplage du réseau de citation au réseau sémantique : méthode présentée dans~\cite{raimbault2016indirect} ; application au sujet en cours} \todo{Soumettre papier Cybergeo (Scientometrics) ; appliquer à corpus réseau-territoire ; traduire l'article en remplaçant les résultats. ETA 1w}

%%%%%%%%%%%%%%%%%%%%%%%%%%
\part{Matériaux}
%%%%%%%%%%%%%%%%%%%%%%%%%%





%%%%%%%%%%%%%%%%%%%%%%%%%%
\part{Synthèse}
%%%%%%%%%%%%%%%%%%%%%%%%%%




%%%%%%%%%%%%%%%%%%%%%%%%%%
\part{Ouverture}
%%%%%%%%%%%%%%%%%%%%%%%%%%






\part*{Conclusion}

\textit{Conclusion générale}



\part*{Annexes}








%\newpage

%%%%%%%%%%%%%%%%%%%%
%% Biblio
%%%%%%%%%%%%%%%%%%%%

\bibliographystyle{apalike}
\bibliography{biblio}


\end{document}
