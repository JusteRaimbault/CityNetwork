%% Commands

\newcommand{\noun}[1]{\textsc{#1}}

% command fort head of chapter citation
\newcommand{\headercit}[3]{
\begin{multicols}{2}
\phantom{}
\columnbreak
\textit{#1}

 - \noun{#2}~#3
\end{multicols}
}



%% Math

% Operators
\DeclareMathOperator{\Cov}{Cov}
\DeclareMathOperator{\Var}{Var}
\DeclareMathOperator{\E}{\mathbb{E}}
\DeclareMathOperator{\Proba}{\mathbb{P}}

\newcommand{\Covb}[2]{\ensuremath{\Cov\!\left[#1,#2\right]}}
\newcommand{\Eb}[1]{\ensuremath{\E\!\left[#1\right]}}
\newcommand{\Pb}[1]{\ensuremath{\Proba\!\left[#1\right]}}
\newcommand{\Varb}[1]{\ensuremath{\Var\!\left[#1\right]}}

% norm
\newcommand{\norm}[1]{\| #1 \|}

% independent
\newcommand{\indep}{\rotatebox[origin=c]{90}{$\models$}}


% amsthm environments
\newtheorem{definition}{Definition}
\newtheorem{proposition}{Proposition}
\newtheorem{assumption}{Assumption}
\newtheorem{lemma}{Lemma}

\newenvironment{proof}[1][Proof]{\begin{trivlist}
\item[\hskip \labelsep {\bfseries #1}]}{\end{trivlist}}




\newcommand{\qed}{\nobreak \ifvmode \relax \else
      \ifdim\lastskip<1.5em \hskip-\lastskip
      \hskip1.5em plus0em minus0.5em \fi \nobreak
      \vrule height0.75em width0.5em depth0.25em\fi}



%%%%%%%%%%%%%%%%%%%
%%  Additional packages
%%%%%%%%%%%%%%%%%%%

%\usepackage{subcaption}

\usepackage{amssymb}

\usepackage{multicol}

\usepackage{bbm}


%%%

\renewcommand{\PrelimText}{%
  \footnotesize[\,\today\ at \thistime\ -- \texttt{Thesis}~\myVersion\,]}


%%%%%%%%
% bilingual version
\usepackage{ifthen}

\newcommand{\bpar}[2]{
\ifthenelse{\thelanguage=0}{#1}{}
\ifthenelse{\thelanguage=1}{#2}{}
}

% note : using these commands make section disappear from ide outline, not really practical -> better use classical commands wrapped around \bpar

%\newcommand{\bchapter}[2]{\chapter{\bpar{#1}{#2}}}
%\newcommand{\bsection}[2]{\section{\bpar{#1}{#2}}}
%\newcommand{\bsubsection}[2]{\subsection{\bpar{#1}{#2}}}
%\newcommand{\bsubsubsection}[2]{\subsubsection{\bpar{#1}{#2}}}
%\newcommand{\bchapters}[2]{\chapter*{\bpar{#1}{#2}}}
%\newcommand{\bsections}[2]{\section*{\bpar{#1}{#2}}}
%\newcommand{\bsubsections}[2]{\subsection*{\bpar{#1}{#2}}}
%\newcommand{\bsubsubsections}[2]{\subsubsection*{\bpar{#1}{#2}}}

%\newcommand{\bcaption}[2]{\caption{\bpar{#1}{#2}}}


% only one optional arg with renewcomand : trick using bpar in the optionnal arg (should check other packages)
%\renewcommand{\section}[3][]{
%\ifthenelse{\equal{#1}{}}{
%\section{\bpar{#1}{#2}}
%}{
%\section[#1]{\bpar{#2}{#3}}
%}
%}
% -> RECURSIVE PB : WHY ?

%\renewcommand{\section}[2]{\section{\bpar{#1}{#2}}}
% http://tex.stackexchange.com/questions/22576/redefining-sectioning-commands



%%%%%%%%%%
%  Drafting

% writing utilities

% comments	 and responses
%  -> use this comment to ask questions on what other wrote/answer questions with optional arguments (up to 4 answers)
\usepackage{xparse}
\usepackage{ifthen}
\DeclareDocumentCommand{\comment}{m o o o o}
{\ifthenelse{\draft=1}{
    \textcolor{red}{\textbf{C : }#1}
    \IfValueT{#2}{\textcolor{blue}{\textbf{A1 : }#2}}
    \IfValueT{#3}{\textcolor{ForestGreen}{\textbf{A2 : }#3}}
    \IfValueT{#4}{\textcolor{red!50!blue}{\textbf{A3 : }#4}}
    \IfValueT{#5}{\textcolor{Aquamarine}{\textbf{A4 : }#5}}
 }{}
}


% todo
\newcommand{\todo}[1]{
\ifthenelse{\draft=1}{\textcolor{red!50!blue}{\textbf{TODO : \textit{#1}}}}{}
}



% provisory part, removed if not draft

\newcommand{\provisory}[1]{
\ifthenelse{\draft=1}{#1}{}
}















\title{Vers des Modèles Couplant Développement Urbain et Croissance des Réseaux de Transport\bigskip\\
\textit{Plan de Thèse}
}
\author{\noun{Juste Raimbault}}
%\date{Novembre 2016}


\maketitle

\justify


\begin{abstract}
\end{abstract}


%\section*{Echéancier}
%
%
%\subsection*{19 octobre}
%
%\paragraph{Todo : Disp 70h}
%
%Integration retours détaillés et du 5/10 [ETA 21h] ; Nettoyage annexes [ETA 10h] ; Complétion commentaires résultats 7 et 8 [ETA 29h]
%
%\paragraph{Contenu}
%
%Thématique et terrain ; revues de littérature.
%
%
%
%
%\subsection*{26 octobre}
%
%\paragraph{Todo : Disp 32h}
%
%Intégration retours 5/10 [ETA 15h] ; Finalisation 7 et 8 [ETA 17h]
%
%\paragraph{Contenu}
%
%Correlations statiques et causality regimes ; modeles macro (statique et coevol)
%
%
%\subsection*{2 novembre}
%
%
%\paragraph{Todo : Disp 32h}
%
%Intégration retours 19/10 [ETA 16h] ; Intro [ETA 16h] 
%
%
%\paragraph{Contenu}
%
%morphogenese et couplage faible
%
%
%\subsection*{9 novembre}
%
%
%\paragraph{Todo : Disp 38h}
%
%Intégration retours 26/10 [ETA 22h] ; Conclusion [ETA 16h] 
%
%
%
%\paragraph{Contenu}
%
%morphogenese co-evolution ; gouvernance (lutetia)
%
%
%
%
%\subsection*{16 novembre}
%
%\paragraph{Todo : Disp 42h}
%
%Intégration retours 02/11 [ETA 22h]
%
%
%\paragraph{Contenu}
%
%positioning (chapitre 3 courant) et theorie (chapitre 9 courant)
%
%\subsection*{23 novembre}
%
%
%\paragraph{Todo : Disp 38h - ETA }
%
%Intégration retours 09/11 et 16/11 [ETA 38h]
%
%
%\paragraph{Contenu}
%
%Wrap-up
%

%\newpage

\vspace{2cm}

\hrule

\vspace{0.5cm}

{\hfill
\Huge \textbf{Plan Détaillé}\hfill
}

\vspace{1.5cm}

\part*{Introduction}

\begin{enumerate}
	\item Introduction du sujet par une anecdote
	\item Le contexte scientifique des Sciences de la Complexité
	\item Sur l'interdisciplinarité : de sa nécessité et de sa difficulté
	\item Illustrations des approches par la Complexité en Géographie
	\item Définition des Villes, Systèmes de Villes et Territoires
	\item Introduction des réseaux, réseaux de transport, définition préliminaire de la co-évolution entre réseaux de transport et territoires. Formulation de la problématique.
	\item Plan détaillé.
\end{enumerate}



%%%%%%%%%%%%%%%%%%%%%%%%%%
\part{Fondations}
%%%%%%%%%%%%%%%%%%%%%%%%%%


%%%%%%%%%%%%%%%%%%%%%%%%%%
\section{Interactions entre Réseaux et Territoires}

\subsection{Territoires et Réseaux}

Une construction théorique et empirique des objets et processus étudiés.

\subsubsection{Territoires et Réseaux : \emph{There and Back Again}}

\begin{enumerate}
	\item On approfondit le concept de Territoire, par des visions complémentaires : les territoires humains (Raffestin) et les territoires des Systèmes de Villes (Pumain).
	\item On précise le concept de réseau, et montre comment celui-ci est intimement lié au Territoire (Dupuy)
	\item Les réseaux façonnent-ils les territoires ? Le débat des \emph{Effets Structurants}.
	\item On définit un objet d'étude global, incorporant territoires et réseaux, que l'on nomme \emph{Systèmes territoriaux}.
\end{enumerate}

\subsubsection{Réseaux de Transport}

% concepts specifiques : accessibilite, mobilite. deja exemples de processus

% exemples d'échelles et processus de la co-évolution

\begin{enumerate}
	\item Développement des spécificités des réseaux de transport et retour sur les effets structurants. On précise l'ambiguïté du débat avec l'exemple du réseau LGV français.
	\item Lien entre mobilité-localisation des actifs : également des phénomènes de co-évolution ? Cela permet de préciser progressivement les échelles temporelles et spatiales.
	\item Le concept d'accessibilité et son rôle potentiel dans la co-évolution.
	\item Dynamiques systémiques sur le temps long, une autre approche des effets structurants.
\end{enumerate}

A ce stade, synthèse préliminaire des processus d'interactions. 


\subsubsection{Des interactions à la co-évolution}


\begin{enumerate}
	\item Le rôle du contexte géographique, illustration du cas particulier des territoires de montagne.
	\item Planification des projets : processus multi-échelles, objectifs de planification rapidement en divergence de la trajectoire effective.
	\item La place de la gouvernance comme processus propre d'interactions ; l'exemple du TOD.
	\item Retour au concept de co-évolution, qui au vu de cet aperçu apparait légitime pour capturer les interdépendances fortes et complexes entre réseaux et territoires, dans une perspective dynamique.
\end{enumerate}



\subsection{Etudes de cas}

Les cas d'étude géographique comme outil pour préciser, spécifier, raffiner les situations de co-évolution.


\subsubsection{Grand Paris}

\begin{enumerate}
	\item Histoire du réseau de transport Parisien : planification successives, décalages temporels entre nécessités territoriales et réalisation des projets.
	\item Emergence d'une gouvernance metropolitaine : illustration des jeux d'acteurs.
	\item Projet du Grand Paris Express : gains d'accessibilité ; introduction à une caractérisation quantitative de la co-évolution.
\end{enumerate}


\subsubsection{Delta de la Rivière des Perles}


\begin{enumerate}
	\item Le contexte de la MCR et la gouvernance des transports
	\item Projet du HKZMB : impact sur l'accessibilité, enjeux de gouvernance.
\end{enumerate}

\subsubsection{Comparabilité}

Peut-on espérer tirer des conclusions générales de deux études de cas ?




\subsection{Recherche Qualitative}

\subsubsection{Réseau à Grande Vitesse}

\begin{enumerate}
	\item Observation de la mise en place du réseau à grande vitesse, répondant à différentes logiques selon temporalités et échelles spatiales.
	\item Suggestion de liens avec la structure urbaine (opération de développement) et la structure sociale (processus indirects).
\end{enumerate}


\subsubsection{Implémentations du TOD}

\begin{enumerate}
	\item La volonté de TOD à Zhuhai reste difficile à implémenter, de par les fortes inerties urbaines.
	\item Au contraire, les villes nouvelles de Hong-Kong témoignent d'une bonne coordination entre transport et urbanisme.
\end{enumerate}



\subsubsection{Observation flottante}

Une observation des systèmes de transport du PRD, Chine et de la métropole parisienne, France, par la méthode d'observation flottante, confirme les aspects multi-échelles, de dépendance au chemin et de particularité pour ces systèmes territoriaux.




\subsection*{Synthèse des processus}

On procède à une synthèse des processus d'interaction, par une entrée par les échelles (stylisées en micro, meso et macro pour l'espace et le temps) et la force du couplage. Un point de vue par les acteurs permet de préciser certains éléments internes aux couplages forts (rôle de la gouvernance et de la structure sociale).



%%%%%%%%%%%%%%%%%%%%%%%%%%
\section{Modéliser les Interactions entre Réseaux et Territoires}

% chapitre état de l'art
% \textit{Revue inter-disciplinaire des modèles traitant les interactions entre réseaux de transport et territoires.}



\subsection{Etat de l'art}

Une revue des approches modélisant les interactions entre réseaux et territoires. On se raccroche à la typologie effectuée au premier chapitre

\subsubsection{Modélisation en géographie quantitative}

\subsubsection{Modéliser les Territoires et les Réseaux}


\subsubsection{Modéliser la co-évolution}





\subsection{Une Approche Epistémologique}


\subsubsection{Revue systématique algorithmique}

%\textit{Etude algorithmique du paysage scientifique sur les interactions entre réseaux et territoires~\cite{raimbault2015models} : des domaines très cloisonnés}


\subsubsection{Bibliométrie indirecte}

%\textit{Raffinement de l'étude précédente par couplage du réseau de citation au réseau sémantique : méthode présentée dans~\cite{raimbault2016indirect}} \todo{Soumettre papier Cybergeo (Scientometrics) ; appliquer à corpus réseau-territoire ; traduire l'article en remplaçant les résultats. ETA 1w}
% \textit{Approfondissement de la revue par des approches automatiques, explorant réseaux de citations et réseaux sémantiques}


\subsubsection{Discussion}





%%%%%%
%% ON HOLD

%\paragraph{Interviews} 
%\textit{Analyse d'interviews d'acteurs académiques sur la question des modèles de co-évolution}




\subsection{Modélographie}

\subsubsection{Revue systématique}


\subsubsection{Modélographie}

\textit{Classification et caractérisation des modèles obtenus par la revue systématique.}


\subsubsection{Discussion}

Synthèse des modèles et processus


%%%%%%%%%%%%%%%%%%%%%%%%%%
\section{Positionnements}



\subsection{Données massives, computation et exploration des modèles}

\textit{Pour un usage précautionneux des données massives et de la computation ; Pour une connaissance plus fine et systématique du comportement des modèles.}


\subsubsection{Pourquoi Modéliser ?}


\subsubsection{Données massives et computation}


\subsubsection{Etendre les analyses de sensibilité}



\subsection{Reproductibilité}

\textit{Toute démarche scientifique doit être reproductible, implications de cette nécessité.}

%\textit{Etudes de cas sur la reproductibilité ; illustration concrète et leçons générales}




\subsection{Positionnement épistémologique}

\subsubsection{Perspectivisme}

\begin{enumerate}
	\item Une approche perspectiviste pour capturer la complexité des systèmes sociaux
	\item Pour une science anarchiste (Feyerabend) ; compatibilité avec le Perspectivisme de Giere
\end{enumerate}

\subsubsection{De la Vie à la Culture}

De la complexité biologique à la complexité sociale


\subsubsection{Nature de la complexité et production de connaissances}

\begin{enumerate}
	\item Diversité des approches de la complexité
	\item Lien entre complexité computationnelle et émergence
	\item Lien entre complexité informationnelle et émergence
	\item Implications pour la production d'une connaissance complexe
\end{enumerate}


%\textit{multiplicité des lectures de la thèse (voir annexe réflexivité, au delà d'une lecture linéaire) $\rightarrow$ presentation JIG et papier CSDM 2017}



\subsubsection*{Implications pratiques}

Des implications pratiques dans trois familles de pratiques : modélisation, Science Ouverte et pratique de l'interdisciplinarité.




\section*{Transition parties I-II}


On est en mesure de donner des caractérisations théorique, empirique et du point de vue de la modélisation de la co-évolution. La deuxième partie vise d'une part à renforcer et préciser celles-ci, d'autre part de construire une caractérisation méthodologique.









%%%%%%%%%%%%%%%%%%%%%%%%%%
\part{Briques Elementaires}
%%%%%%%%%%%%%%%%%%%%%%%%%%



\section{Théorie Evolutive Urbaine}

\textit{Premières preuves d'existence des interactions et de leur forme, ainsi que des processus concernés.}

\subsection{Correlations entre Forme Urbaine et Forme de Réseau}

\textit{Les correlations spatiales entre indicateurs de forme urbaine et de forme de réseau révèlent la non-stationnarité des interactions, qui peut être reliée à la non-ergodicité}


\subsection{Causalités spatio-temporelles}


\textit{Une méthode pour classifier les régimes de co-evolution.}


\subsection{Effets de Réseaux révélés par un modèle de croissance macroscopique}

%\textit{Modèle de Gibrat étendu par interactions gravitaires au premier ordre, par retroaction des flux physiques au second ordre~\cite{raimbault2016models}, révèle effets de réseaux par validation du modèle étendu via critère d'Akaike empirique}

\textit{Un modèle d'interaction entre villes révèle des effets de réseaux}





\section{Morphogenèse Urbaine}

{\color{blue}Une entrée de modélisation alternative par la morphogenèse, construction progressive de modèles.}

\subsection{Une approche interdisciplinaire de la Morphogenèse}

\textit{Construction épistémologique d'une définition unifiée de la morphogenèse}


\subsection{Morphogenèse Urbaine par Aggregation-Diffusion}

\textit{Modèle de croissance urbaine par processus d'aggregation diffusion, reproduit de manière fine l'ensemble des morphologies urbaines existantes en Europe}


\subsection{Génération de systèmes corrélés}

\textit{Couplage faible du modèle précédent à une heuristique de génération de réseau, permet de générer des système couplés à la correlation contrôlée}





%%%%%%%%%%%%%%%%%%%%%%%%%%
\part{Co-évolution}
%%%%%%%%%%%%%%%%%%%%%%%%%%

%\comment{partie agencée par degré de complexité des modèles.}

%%%%%%%%%%%%%%%%%%%%%%%%%%
\section{Co-evolution à l'échelle macroscopique}

\subsection{Exploration de SimpopNet}

\textit{Définition d'indicateurs pour les modèles de co-evolution macroscopique ; régimes produits par le modèle simpopnet}

\subsection{Extension du modèle d'interaction}


\textit{Extension co-évolutive du gibrat-interaction ; régimes produits par le modèle sur systèmes synthétique ; calibration sur donnée réelle pour le réseau ferroviaire français.}


\subsection{Vers le Modèle SimpopSino}

\textit{Proposition d'adaptation du modèle pour le système de ville Chinois}





%%%%%%%%%%%%%%%%%%%%%%%%%%
\section{Co-evolution à l'échelle mesoscopique}


\subsection{Comparaison des heuristiques de réseau}

\textit{Comparaison de différentes heuristiques de génération de réseau}


\subsection{Co-evolution des formes}

\textit{Modèle de morphogenèse co-évolutif ; calibration au premier et second ordre (indicateurs et correlations) ; Régimes de causalité}



\subsection{Lutetia : un modèle de co-évolution incluant la gouvernance des systèmes de transports}

%\paragraph{Modèle}

\textit{Modèle de co-évolution sur le temps long, couplant un LUTI à un module de gouvernance des transports basé sur la théorie des jeux pour le développement du réseau ; Application au Delta de la Rivière des Perles}


%\textit{Application de Lutecia au cas réel de la Mega-city Region du PRD.}

%\todo{Calibration et validation du modèle sur le Delta de la Rivière des Perles : objectif Article Transport Geography début mai}




%%%%%%%%%%%%%%%%%%%%%%%%%%
\section{Ouverture}




\section{Ouverture empirique}

\textit{Des difficultés sont rencontrées si les échelles et le système ne sont pas proprement choisis}

\subsection{Equilibre Utilisateur Statique}

\textit{Investigation de l'existence empirique de l'Equilibre Utilisateur Statique : caractère chaotique des flux de transport routier.}


\subsection{Transport Routier et déterminants des coûts}

\textit{Determinants des couts du carburant aux US : relations indirectes entre réseau et territoires ; non-stationnarité et structure modulaire des systèmes territoriaux.}


%\textit{Paper energy price : justify the presence of a hidden network. Unveils again non-stationarity, and modular structure of territorial systems}





%%%%%%%%%%%%%%%%%%%%%%%%%%
% Constructions Théoriques 

\textit{Constructions théoriques successives, avec un niveau meta progressif}

\subsection{Une Théorie des Systèmes Territoriaux Co-évolutifs en Réseau}

\textit{Développement de la théorie géographique co-construite avec les autres domaines de la thèse, qui couple l'entrée morphogenétique avec la théorie évolutive des villes}





\subsection{Un cadre de connaissances pour une géographie intégrée}

\textit{Précision d'un cadre de connaissances ; Mise en perspective de la connaissance produite par la thèse comme illustration de la co-évolution des connaissances.}

%\comment{here integrate as chapter conclusion reflexion on reflexivity, types of complexities etc. ? (presentation discutant colloque Geodivercity)}


%%%%%%%%%%%%%%%%%%%%%%%%%%
\part*{Conclusion}
%%%%%%%%%%%%%%%%%%%%%%%%%%


\section*{Perspectives}

\paragraph{Développements Spécifiques}

%\textit{Projets de recherche détaillés issus de divers développements (par exemple communication scientifique \cite{serra2016game} ; épistémologie quantitative~\cite{raimbault2016techno} ; science ouverte \cite{cybergeo20})}

\textit{Projets de recherche détaillés issus de divers développements : communication scientifique, épistémologie quantitative, science ouverte}

%\comment{remarque : proposition de cours de modélisation, peut être évoqué ici.}

\paragraph{Vers un Programme de Recherche}

\textit{Synthèse des axes de recherche révélés tout au long de la thèse, proposition d'un programme de longue durée pour l'étude des systèmes territoriaux complexes}



\section*{Conclusion générale}



%\part*{Annexes}



% not necessary
%\section{Une approche interdisciplinaire de la morphogenèse}

%\textit{texte complet de~\cite{antelope2016interdisciplinary}}




%\section{Supplementary Information}


%\subsection{Dérivations}

%\textit{Dérivations Analytiques pour diverses parties de la thèse}


%\subsection{Exploration des Modèles}

%\textit{Explorations raffinées pour certains modèles ; applications compagnon d'exploration interactive}


%\section{Développements Méthodologiques}

%\subsection{A Unified Framework for Models of Urban Growth}

%\textit{The various model we will develop could enter a unified framework ; derivation of the link between Gibrat and Simon models}


%\subsection{Sensitivity of Urban Scaling to City Definition}

%\textit{Analytical validation of the sensitivity of scaling exponents to city definition in a simple model or urban form}

%\subsection{Quantifying Robustness through Discrepancy}

%\textit{Complex systems are by nature multi-objective : in the particular case of multi-attribute evaluations, we introduce a framework to quantify robustness independently of the model, based on data discrepancy~\cite{raimbault2016discrepancy}}


%\subsection{Spatio-temporal correlations and causalities}

%\textit{Linking spatial and temporal correlations of geographical indicators in simple cases ; a granger-causality method to identify spatio-temporal causalities}
% --> integrated in paper Sageo



%\section{Développements Thématiques}

%\textit{Laïus introductif : approche unifiée des Systèmes Complexes, positionner chaque développement dans une vision synthétique globale.} \comment{maybe in conclusion / opening ?}

%\subsection{Données synthétiques}

%\textit{Développement de~\cite{raimbault2016generation} dans le champ de la Finance Quantitative}


%\subsection{Epistémologie Quantitative}

%\textit{CybergeoNetworks : détails méthodologiques, résultats sur Cybergeo. Résultats sur les Brevets.}

%\subsection{Système de Transport en partage}

%\textit{\cite{raimbault2015hybrid} montre l'hétérogénéité et la complexité des interactions à l'échelle microscopique}


%\subsection{Un cadre formel pour modéliser les systèmes socio-techniques}

%\textit{Méta-théorie permettant de formaliser des perspectives de modélisations multiples sur les systèmes socio-techniques.}

%\todo{reste à développer action des modèles sur les données, y associer une structure d'action de monoïde}





%\section{Reflexivité}


%\textit{Application des outils d'épistémologie quantitative à la thèse elle-même ; statistiques détaillées des différents projets ; graphe des concepts et parties de la thèse (application compagnon ?) et proposition de pistes alternatives de lecture}


%\section{Bases de Données}


%\textit{Description des bases crées dans le cadre de la thèse : réseau routier simplifié prou l'Europe ; traffic routier en Ile de France ; Données VLib sur 3ans ; Autoroutes dynamiques} \todo{pour la base topologique OSM, data paper (Scientific Data) ; pour la base VLib, data paper Cybergeo Data Papers ?}


%\section{Logiciels et Packages}

%\textit{Packages réutilisables développés dans le cadre de la thèse : largeNetworkR ; Scientific Corpus Mining}


%\section{Architecture et Source}

%\textit{Architecture et Source des modèles de simulation et d'analyse de données}


%\section{Productivité}

%\textit{Outils ouverts pour une productivité scientifique améliorée}




%%%%
% n'a pas sa place dans une thèse -> ?
%  -- pas forcément, distiller pour les transitions etc --


%\section{Science et Art}

%\textit{Oeuvres d'art sérendipiteuses (composition graphique et poésie) produites dans le cadre de la thèse}


%\newpage

%%%%%%%%%%%%%%%%%%%%
%% Biblio
%%%%%%%%%%%%%%%%%%%%

%\bibliographystyle{apalike}
%\bibliography{biblio}


\end{document}
