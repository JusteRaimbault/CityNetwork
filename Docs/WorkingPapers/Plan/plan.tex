%%%%%%%%%%%%%%%%%%%%%%%%%%%%%
% Standard header for working papers
%
% WPHeader.tex
%
%%%%%%%%%%%%%%%%%%%%%%%%%%%%%

\documentclass[11pt]{article}



%%%%%%%%%%%%%%%%%%%%%%%%%%
%% TEMPLATES
%%%%%%%%%%%%%%%%%%%%%%%%%%


% Simple Tabular

%\begin{tabular}{ |c|c|c| } 
% \hline
% cell1 & cell2 & cell3 \\ 
% cell4 & cell5 & cell6 \\ 
% cell7 & cell8 & cell9 \\ 
% \hline
%\end{tabular}





%%%%%%%%%%%%%%%%%%%%%%%%%%
%% Packages
%%%%%%%%%%%%%%%%%%%%%%%%%%



% encoding 
\usepackage[utf8]{inputenc}
\usepackage[T1]{fontenc}


% general packages without options
\usepackage{amsmath,amssymb,amsthm,bbm}

% graphics
\usepackage{graphicx,transparent,eso-pic}

% text formatting
\usepackage[document]{ragged2e}
\usepackage{pagecolor,color}
%\usepackage{ulem}
\usepackage{soul}


% conditions
\usepackage{ifthen}


\usepackage{natbib}


%%%%%%%%%%%%%%%%%%%%%%%%%%
%% Maths environment
%%%%%%%%%%%%%%%%%%%%%%%%%%

%\newtheorem{theorem}{Theorem}[section]
%\newtheorem{lemma}[theorem]{Lemma}
%\newtheorem{proposition}[theorem]{Proposition}
%\newtheorem{corollary}[theorem]{Corollary}

%\newenvironment{proof}[1][Proof]{\begin{trivlist}
%\item[\hskip \labelsep {\bfseries #1}]}{\end{trivlist}}
%\newenvironment{definition}[1][Definition]{\begin{trivlist}
%\item[\hskip \labelsep {\bfseries #1}]}{\end{trivlist}}
%\newenvironment{example}[1][Example]{\begin{trivlist}
%\item[\hskip \labelsep {\bfseries #1}]}{\end{trivlist}}
%\newenvironment{remark}[1][Remark]{\begin{trivlist}
%\item[\hskip \labelsep {\bfseries #1}]}{\end{trivlist}}

%\newcommand{\qed}{\nobreak \ifvmode \relax \else
%      \ifdim\lastskip<1.5em \hskip-\lastskip
%      \hskip1.5em plus0em minus0.5em \fi \nobreak
%      \vrule height0.75em width0.5em depth0.25em\fi}



%% Commands

\newcommand{\noun}[1]{\textsc{#1}}


%% Math

% Operators
\DeclareMathOperator{\Cov}{Cov}
\DeclareMathOperator{\Var}{Var}
\DeclareMathOperator{\E}{\mathbb{E}}
\DeclareMathOperator{\Proba}{\mathbb{P}}

\newcommand{\Covb}[2]{\ensuremath{\Cov\!\left[#1,#2\right]}}
\newcommand{\Eb}[1]{\ensuremath{\E\!\left[#1\right]}}
\newcommand{\Pb}[1]{\ensuremath{\Proba\!\left[#1\right]}}
\newcommand{\Varb}[1]{\ensuremath{\Var\!\left[#1\right]}}

% norm
\newcommand{\norm}[1]{\left\lVert #1 \right\rVert}



% argmin
\DeclareMathOperator*{\argmin}{\arg\!\min}


% amsthm environments
\newtheorem{definition}{Definition}
\newtheorem{proposition}{Proposition}
\newtheorem{assumption}{Assumption}

%% graphics

% renew graphics command for relative path providment only ?
%\renewcommand{\includegraphics[]{}}


\usepackage{url}





% geometry
\usepackage[margin=2cm]{geometry}



% changes

\usepackage{soul}
\soulregister\cite7
\soulregister\citep7
\soulregister\ref7

\usepackage[final]{changes}
%\usepackage{changes}


\setaddedmarkup{\textcolor{black}{\hl{#1}}}
\setdeletedmarkup{\textcolor{red}{\sout{#1}}}



\usepackage{CJKutf8}
%\begin{CJK*}{UTF8}{zhsong}
%文章内容。
%\clearpage\end{CJK*}
\newcommand{\cn}[1]{
  \begin{CJK*}{UTF8}{gbsn}
  #1
  \end{CJK*}
}



% layout : use fancyhdr package
%\usepackage{fancyhdr}
%\pagestyle{fancy}
%
%\makeatletter
%
%\renewcommand{\headrulewidth}{0.4pt}
%\renewcommand{\footrulewidth}{0.4pt}
%\fancyhead[RO,RE]{}
%\fancyhead[LO,LE]{Models for the co-evolution of cities and networks}
%\fancyfoot[RO,RE] {\thepage}
%\fancyfoot[LO,LE] {}
%\fancyfoot[CO,CE] {}
%
%\makeatother
%

%%%%%%%%%%%%%%%%%%%%%
%% Begin doc
%%%%%%%%%%%%%%%%%%%%%

\begin{document}









\title{Vers des Modèles Couplant Développement Urbain et Croissance des Réseaux de Transport\bigskip\\
\textit{Plan de Thèse}
}
\author{\noun{Juste Raimbault}}
%\date{Novembre 2016}


\maketitle

\justify


\begin{abstract}
\end{abstract}


\section*{Echéancier}


\subsection*{19 octobre}

\paragraph{Todo : Disp 70h}

Integration retours détaillés et du 5/10 [ETA 21h] ; Nettoyage annexes [ETA 10h] ; Complétion commentaires résultats 7 et 8 [ETA 29h]

\paragraph{Contenu}

Thématique et terrain ; revues de littérature.




\subsection*{26 octobre}

\paragraph{Todo : Disp 32h}

Intégration retours 5/10 [ETA 15h] ; Finalisation 7 et 8 [ETA 17h]

\paragraph{Contenu}

Correlations statiques et causality regimes ; modeles macro (statique et coevol)


\subsection*{2 novembre}


\paragraph{Todo : Disp 32h}

Intégration retours 19/10 [ETA 16h] ; Intro [ETA 16h] 


\paragraph{Contenu}

morphogenese et couplage faible


\subsection*{9 novembre}


\paragraph{Todo : Disp 38h}

Intégration retours 26/10 [ETA 22h] ; Conclusion [ETA 16h] 



\paragraph{Contenu}

morphogenese co-evolution ; gouvernance (lutetia)




\subsection*{16 novembre}

\paragraph{Todo : Disp 42h}

Intégration retours 02/11 [ETA 22h]


\paragraph{Contenu}

positioning (chapitre 3 courant) et theorie (chapitre 9 courant)

\subsection*{23 novembre}


\paragraph{Todo : Disp 38h - ETA }

Intégration retours 09/11 et 16/11 [ETA 38h]


\paragraph{Contenu}

Wrap-up


%\newpage

\vspace{2cm}

\hrule

\vspace{0.5cm}

{\hfill
\Huge \textbf{Plan}\hfill
}

\vspace{1.5cm}

\part*{Introduction}

\textit{Introduction du sujet par exemples concrets ; cadre scientifique ; interdisciplinarité ; sciences des systèmes complexes ; complexité en géographie}

%%%%%%%%%%%%%%%%%%%%%%%%%%
\part{Fondations}
%%%%%%%%%%%%%%%%%%%%%%%%%%


%%%%%%%%%%%%%%%%%%%%%%%%%%
\section{Interactions entre Réseaux et Territoires}

\subsection{Réseaux et Territoires}

%\textit{Revue de Littérature thématique ; construction de la question de recherche : introduction progressive de la problématique de co-évolution, précision des objets (réseaux et territoires)}

\textit{Les réseaux de transport façonnent les territoires et réciproquement : exemples d'échelles et processus de la co-évolution.}

\subsection{Etude de cas}

\textit{Illustration par les cas du Grand Paris et du Delta de la Rivière des Perles.} 


\subsection{Recherche Qualitative}

\textit{Illustration par des observations de terrain.}



%%%%%%%%%%%%%%%%%%%%%%%%%%
\section{Modéliser les Interactions entre Réseaux et Territoires}

% chapitre état de l'art

\subsection{Etat de l'art}

\textit{Revue inter-disciplinaire des modèles traitant les interactions entre réseaux de transport et territoires.}


\subsection{Une Approche Epistémologique}

%\paragraph{Interviews} 

%\textit{Analyse d'interviews d'acteurs académiques sur la question des modèles de co-évolution}

%\paragraph{Revue systématique algorithmique}

%\textit{Etude algorithmique du paysage scientifique sur les interactions entre réseaux et territoires~\cite{raimbault2015models} : des domaines très cloisonnés}


%\paragraph{Bibliométrie indirecte par hyperréseau}


%\textit{Raffinement de l'étude précédente par couplage du réseau de citation au réseau sémantique : méthode présentée dans~\cite{raimbault2016indirect}} \todo{Soumettre papier Cybergeo (Scientometrics) ; appliquer à corpus réseau-territoire ; traduire l'article en remplaçant les résultats. ETA 1w}

\textit{Approfondissement de la revue par des approches automatiques, explorant réseaux de citations et réseaux sémantiques}



\subsection{Modélographie}

\textit{Classification et caractérisation des modèles obtenus par la revue systématique.}

%\textit{``Classification'' systématique des modèles existants : processus, échelles, cas d'application (restant à un niveau meta pour les types de modèles pour lesquels on se base déjà sur une revue, par exemple LUTI)}



%%%%%%%%%%%%%%%%%%%%%%%%%%
\section{Positionnements}


\subsection{Reproductibilité}

\textit{Toute démarche scientifique doit être reproductible, implications de cette nécessité.}

%\textit{Etudes de cas sur la reproductibilité ; illustration concrète et leçons générales}

\subsection{Données massives, computation et exploration des modèles}

\textit{Pour un usage précautionneux des données massives et de la computation ; Pour une connaissance plus fine et systématique du comportement des modèles.}



\subsection{Positionnement épistémologique}


\textit{Une approche perspectiviste pour capturer la complexité des systèmes sociaux ; relation avec la nature de la complexité.}

%\textit{Pour une science anarchiste (Feyerabend) ; compatibilité avec le Perspectivisme de Giere et pourquoi celui-ci est particulièrement adapté aux paradigmes de la complexité ; multiplicité des lectures de la thèse (voir annexe réflexivité, au delà d'une lecture linéaire) $\rightarrow$ presentation JIG et papier CSDM 2017}














%%%%%%%%%%%%%%%%%%%%%%%%%%
\part{Matériaux}
%%%%%%%%%%%%%%%%%%%%%%%%%%

%\comment[JR]{Cette partie est la plus délicate dans son organisation et articulation ; bien expliquer les liens et le cheminement - éventuellement déjà introduire de la reflexivité avec un diagramme d'interactions des domaines de connaissance.}


\section{Théorie Evolutive Urbaine}

\textit{Premières preuves d'existence des interactions et de leur forme, ainsi que des processus concernés.}

\subsection{Correlations entre Forme Urbaine et Forme de Réseau}

\textit{Les correlations spatiales entre indicateurs de forme urbaine et de forme de réseau révèlent la non-stationnarité des interactions, qui peut être reliée à la non-ergodicité}


\subsection{Causalités spatio-temporelles}


\textit{Une méthode pour classifier les régimes de co-evolution.}


\subsection{Effets de Réseaux révélés par un modèle de croissance macroscopique}

%\textit{Modèle de Gibrat étendu par interactions gravitaires au premier ordre, par retroaction des flux physiques au second ordre~\cite{raimbault2016models}, révèle effets de réseaux par validation du modèle étendu via critère d'Akaike empirique}

\textit{Un modèle d'interaction entre villes révèle des effets de réseaux}



\section{Interactions à l'Echelle Microscopique}

\textit{Des difficultés sont rencontrées si les échelles et le système ne sont pas proprement choisis}

\subsection{Equilibre Utilisateur Statique}

\textit{Investigation de l'existence empirique de l'Equilibre Utilisateur Statique : caractère chaotique des flux de transport routier.}


\subsection{Transport Routier et déterminants des coûts}

\textit{Determinants des couts du carburant aux US : relations indirectes entre réseau et territoires ; non-stationnarité et structure modulaire des systèmes territoriaux.}


%\textit{Paper energy price : justify the presence of a hidden network. Unveils again non-stationarity, and modular structure of territorial systems}



\subsection{Transactions immobilières et Grand Paris}

\textit{Recherche de correlations et/ou causalités entre transactions immobilières (base BIEN) et tracé du réseau du métro du Grand Paris}



\section{Morphogenèse Urbaine}

{\color{blue}Une entrée de modélisation alternative par la morphogenèse, construction progressive de modèles.}

\subsection{Une approche interdisciplinaire de la Morphogenèse}

\textit{Construction épistémologique d'une définition unifiée de la morphogenèse}


\subsection{Morphogenèse Urbaine par Aggregation-Diffusion}

\textit{Modèle de croissance urbaine par processus d'aggregation diffusion, reproduit de manière fine l'ensemble des morphologies urbaines existantes en Europe}


\subsection{Génération de systèmes corrélés}

\textit{Couplage faible du modèle précédent à une heuristique de génération de réseau, permet de générer des système couplés à la correlation contrôlée}





%%%%%%%%%%%%%%%%%%%%%%%%%%
\part{Co-évolution}
%%%%%%%%%%%%%%%%%%%%%%%%%%

%\comment{partie agencée par degré de complexité des modèles.}

%%%%%%%%%%%%%%%%%%%%%%%%%%
\section{Co-evolution à l'échelle macroscopique}

\subsection{Exploration de SimpopNet}

\textit{Définition d'indicateurs pour les modèles de co-evolution macroscopique ; régimes produits par le modèle simpopnet}

\subsection{Extension du modèle d'interaction}


\textit{Extension co-évolutive du gibrat-interaction ; régimes produits par le modèle sur systèmes synthétique ; calibration sur donnée réelle pour le réseau ferroviaire français.}


\subsection{Vers le Modèle SimpopSino}

\textit{Proposition d'adaptation du modèle pour le système de ville Chinois}





%%%%%%%%%%%%%%%%%%%%%%%%%%
\section{Co-evolution à l'échelle mesoscopique}


\subsection{Comparaison des heuristiques de réseau}

\textit{Comparaison de différentes heuristiques de génération de réseau}


\subsection{Co-evolution des formes}

\textit{Modèle de morphogenèse co-évolutif ; calibration au premier et second ordre (indicateurs et correlations) ; Régimes de causalité}



\subsection{Lutetia : un modèle de co-évolution incluant la gouvernance des systèmes de transports}

%\paragraph{Modèle}

\textit{Modèle de co-évolution sur le temps long, couplant un LUTI à un module de gouvernance des transports basé sur la théorie des jeux pour le développement du réseau ; Application au Delta de la Rivière des Perles}


%\textit{Application de Lutecia au cas réel de la Mega-city Region du PRD.}

%\todo{Calibration et validation du modèle sur le Delta de la Rivière des Perles : objectif Article Transport Geography début mai}




%%%%%%%%%%%%%%%%%%%%%%%%%%
\section{Ouverture}



%%%%%%%%%%%%%%%%%%%%%%%%%%
% Constructions Théoriques 

\textit{Constructions théoriques successives, avec un niveau meta progressif}

\subsection{Une Théorie des Systèmes Territoriaux Co-évolutifs en Réseau}

\textit{Développement de la théorie géographique co-construite avec les autres domaines de la thèse, qui couple l'entrée morphogenétique avec la théorie évolutive des villes}


\subsection{Un cadre formel pour modéliser les systèmes socio-techniques}

\textit{Méta-théorie permettant de formaliser des perspectives de modélisations multiples sur les systèmes socio-techniques.}

%\todo{reste à développer action des modèles sur les données, y associer une structure d'action de monoïde}



\subsection{Un cadre de connaissances pour une géographie intégrée}

\textit{Précision d'un cadre de connaissances ; Mise en perspective de la connaissance produite par la thèse comme illustration de la co-évolution des connaissances.}

%\comment{here integrate as chapter conclusion reflexion on reflexivity, types of complexities etc. ? (presentation discutant colloque Geodivercity)}


%%%%%%%%%%%%%%%%%%%%%%%%%%
\part*{Conclusion}
%%%%%%%%%%%%%%%%%%%%%%%%%%


\section*{Perspectives}

\paragraph{Développements Spécifiques}

%\textit{Projets de recherche détaillés issus de divers développements (par exemple communication scientifique \cite{serra2016game} ; épistémologie quantitative~\cite{raimbault2016techno} ; science ouverte \cite{cybergeo20})}

\textit{Projets de recherche détaillés issus de divers développements : communication scientifique, épistémologie quantitative, science ouverte}

%\comment{remarque : proposition de cours de modélisation, peut être évoqué ici.}

\paragraph{Vers un Programme de Recherche}

\textit{Synthèse des axes de recherche révélés tout au long de la thèse, proposition d'un programme de longue durée pour l'étude des systèmes territoriaux complexes}



\section*{Conclusion générale}



%\part*{Annexes}



% not necessary
%\section{Une approche interdisciplinaire de la morphogenèse}

%\textit{texte complet de~\cite{antelope2016interdisciplinary}}




%\section{Supplementary Information}


%\subsection{Dérivations}

%\textit{Dérivations Analytiques pour diverses parties de la thèse}


%\subsection{Exploration des Modèles}

%\textit{Explorations raffinées pour certains modèles ; applications compagnon d'exploration interactive}


%\section{Développements Méthodologiques}

%\subsection{A Unified Framework for Models of Urban Growth}

%\textit{The various model we will develop could enter a unified framework ; derivation of the link between Gibrat and Simon models}


%\subsection{Sensitivity of Urban Scaling to City Definition}

%\textit{Analytical validation of the sensitivity of scaling exponents to city definition in a simple model or urban form}

%\subsection{Quantifying Robustness through Discrepancy}

%\textit{Complex systems are by nature multi-objective : in the particular case of multi-attribute evaluations, we introduce a framework to quantify robustness independently of the model, based on data discrepancy~\cite{raimbault2016discrepancy}}


%\subsection{Spatio-temporal correlations and causalities}

%\textit{Linking spatial and temporal correlations of geographical indicators in simple cases ; a granger-causality method to identify spatio-temporal causalities}
% --> integrated in paper Sageo



%\section{Développements Thématiques}

%\textit{Laïus introductif : approche unifiée des Systèmes Complexes, positionner chaque développement dans une vision synthétique globale.} \comment{maybe in conclusion / opening ?}

%\subsection{Données synthétiques}

%\textit{Développement de~\cite{raimbault2016generation} dans le champ de la Finance Quantitative}


%\subsection{Epistémologie Quantitative}

%\textit{CybergeoNetworks : détails méthodologiques, résultats sur Cybergeo. Résultats sur les Brevets.}

%\subsection{Système de Transport en partage}

%\textit{\cite{raimbault2015hybrid} montre l'hétérogénéité et la complexité des interactions à l'échelle microscopique}







%\section{Reflexivité}


%\textit{Application des outils d'épistémologie quantitative à la thèse elle-même ; statistiques détaillées des différents projets ; graphe des concepts et parties de la thèse (application compagnon ?) et proposition de pistes alternatives de lecture}


%\section{Bases de Données}


%\textit{Description des bases crées dans le cadre de la thèse : réseau routier simplifié prou l'Europe ; traffic routier en Ile de France ; Données VLib sur 3ans ; Autoroutes dynamiques} \todo{pour la base topologique OSM, data paper (Scientific Data) ; pour la base VLib, data paper Cybergeo Data Papers ?}


%\section{Logiciels et Packages}

%\textit{Packages réutilisables développés dans le cadre de la thèse : largeNetworkR ; Scientific Corpus Mining}


%\section{Architecture et Source}

%\textit{Architecture et Source des modèles de simulation et d'analyse de données}


%\section{Productivité}

%\textit{Outils ouverts pour une productivité scientifique améliorée}




%%%%
% n'a pas sa place dans une thèse -> ?
%  -- pas forcément, distiller pour les transitions etc --


%\section{Science et Art}

%\textit{Oeuvres d'art sérendipiteuses (composition graphique et poésie) produites dans le cadre de la thèse}


%\newpage

%%%%%%%%%%%%%%%%%%%%
%% Biblio
%%%%%%%%%%%%%%%%%%%%

%\bibliographystyle{apalike}
%\bibliography{biblio}


\end{document}
