%%%%%%%%%%%%%%%%%%%%%%%%%%%%%
% Standard header for working papers
%
% WPHeader.tex
%
%%%%%%%%%%%%%%%%%%%%%%%%%%%%%

\documentclass[11pt]{article}

% packages without options
\usepackage{amsmath,bbm}

% geometry
\usepackage[margin=2cm]{geometry}






\title{Reflexions on context and scale of models\bigskip\\
\textit{Working Paper}
}
\author{\noun{Juste Raimbault}}
\date{Thursday 7th May}


\maketitle

\justify


\begin{abstract}
Far from precise geographical case studies that impose a highly precise definitions of context and objects, our subject is relatively open on that point. It should however be situated as most models will find a substance through a particular context or type of application. We present considerations on what would be relevant contexts and scales of application.
\end{abstract}



\section*{Ontological Context}


\section*{Spatial Scale}

We first develop the issue of the spatial scale, as temporal scale should depend on it. We assume that ontological context is fixed, i.e. that we study system of cities ...


Spatial Situation should not be constrained $\rightarrow$ more different case studies all over the world. Temporal situation will fix the system of settlement and will be on the contrary contrained.


\section*{Temporal Scale and Situation}


\section*{Examples of Potential Case Studies}








%%%%%%%%%%%%%%%%%%%%
%% Biblio
%%%%%%%%%%%%%%%%%%%%

\bibliographystyle{apalike}
\bibliography{/Users/Juste/Documents/ComplexSystems/CityNetwork/Biblio/Bibtex/CityNetwork}


\end{document}
