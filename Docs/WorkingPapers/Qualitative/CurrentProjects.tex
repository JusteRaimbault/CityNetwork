%%%%%%%%%%%%%%%%%%%%%%%%%%%%%
% Standard header for working papers
%
% WPHeader.tex
%
%%%%%%%%%%%%%%%%%%%%%%%%%%%%%

\documentclass[11pt]{article}

% packages without options
\usepackage{amsmath,bbm}

% geometry
\usepackage[margin=2cm]{geometry}






\title{French High Speed Network : Pragmatized State Top-down Transportation Planning\bigskip\\
\textit{Working Paper}
}
\author{\noun{Juste Raimbault}}
\date{Friday 22th May}


\maketitle

\justify


\begin{abstract}
We review ongoing transportation infrastructure projects in France, particularly High Speed Lines being built or in diverse stages of the project. This qualitative analysis aims to provide insights into possible mechanisms of transportation planning processes at different scales.
\end{abstract}


%%%%%%%%%%%%%%%%%%%%
\section{Introduction}
%%%%%%%%%%%%%%%%%%%%

Context LGV network : read and synthesize Zembri~\cite{zembri1997fondements,zembri2008contribution}.


%%%%%%%%%%%%%%%%%%%%
\section{Projects}
%%%%%%%%%%%%%%%%%%%%

List and characteristics of reviewed projects.

\begin{itemize}
\item LGV Est
\item LGV Rhin Rhône
\item LGV Bretagne/PdL
\item LGV Ouest Atlantique
\item Lyon Turin ?
\end{itemize}

\texttt{http://www.ligne-montpellier-perpignan.com/}
\texttt{http://www.ocvia.fr/}
etc.


%%%%%%%%%%%%%%%%%%%%
\section{Top-Down Planning versus Local Compromises}
%%%%%%%%%%%%%%%%%%%%

\subsection{Paradigm switching}

From High Speed Lines to mixed ``Lignes Nouvelles'' : pragmatization of the Top-Down project ?

Local Stations/Compromizes ?

\subsection{Communication}

Analyze how project are presented/analyzed.

Particular case of Lyon/Turin.

Contestation of ``big unuseful projects''




%%%%%%%%%%%%%%%%%%%%
%% Biblio
%%%%%%%%%%%%%%%%%%%%

\bibliographystyle{apalike}
\bibliography{/Users/Juste/Documents/ComplexSystems/CityNetwork/Biblio/Bibtex/CityNetwork}


\end{document}


