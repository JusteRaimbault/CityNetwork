\documentclass[english,11pt]{beamer}
%% Commands

\newcommand{\noun}[1]{\textsc{#1}}

% command fort head of chapter citation
\newcommand{\headercit}[3]{
\begin{multicols}{2}
\phantom{}
\columnbreak
\textit{#1}

 - \noun{#2}~#3
\end{multicols}
}



%% Math

% Operators
\DeclareMathOperator{\Cov}{Cov}
\DeclareMathOperator{\Var}{Var}
\DeclareMathOperator{\E}{\mathbb{E}}
\DeclareMathOperator{\Proba}{\mathbb{P}}

\newcommand{\Covb}[2]{\ensuremath{\Cov\!\left[#1,#2\right]}}
\newcommand{\Eb}[1]{\ensuremath{\E\!\left[#1\right]}}
\newcommand{\Pb}[1]{\ensuremath{\Proba\!\left[#1\right]}}
\newcommand{\Varb}[1]{\ensuremath{\Var\!\left[#1\right]}}

% norm
\newcommand{\norm}[1]{\| #1 \|}

% independent
\newcommand{\indep}{\rotatebox[origin=c]{90}{$\models$}}


% amsthm environments
\newtheorem{definition}{Definition}
\newtheorem{proposition}{Proposition}
\newtheorem{assumption}{Assumption}
\newtheorem{lemma}{Lemma}

\newenvironment{proof}[1][Proof]{\begin{trivlist}
\item[\hskip \labelsep {\bfseries #1}]}{\end{trivlist}}




\newcommand{\qed}{\nobreak \ifvmode \relax \else
      \ifdim\lastskip<1.5em \hskip-\lastskip
      \hskip1.5em plus0em minus0.5em \fi \nobreak
      \vrule height0.75em width0.5em depth0.25em\fi}



%%%%%%%%%%%%%%%%%%%
%%  Additional packages
%%%%%%%%%%%%%%%%%%%

%\usepackage{subcaption}

\usepackage{amssymb}

\usepackage{multicol}

\usepackage{bbm}


%%%

\renewcommand{\PrelimText}{%
  \footnotesize[\,\today\ at \thistime\ -- \texttt{Thesis}~\myVersion\,]}


%%%%%%%%
% bilingual version
\usepackage{ifthen}

\newcommand{\bpar}[2]{
\ifthenelse{\thelanguage=0}{#1}{}
\ifthenelse{\thelanguage=1}{#2}{}
}

% note : using these commands make section disappear from ide outline, not really practical -> better use classical commands wrapped around \bpar

%\newcommand{\bchapter}[2]{\chapter{\bpar{#1}{#2}}}
%\newcommand{\bsection}[2]{\section{\bpar{#1}{#2}}}
%\newcommand{\bsubsection}[2]{\subsection{\bpar{#1}{#2}}}
%\newcommand{\bsubsubsection}[2]{\subsubsection{\bpar{#1}{#2}}}
%\newcommand{\bchapters}[2]{\chapter*{\bpar{#1}{#2}}}
%\newcommand{\bsections}[2]{\section*{\bpar{#1}{#2}}}
%\newcommand{\bsubsections}[2]{\subsection*{\bpar{#1}{#2}}}
%\newcommand{\bsubsubsections}[2]{\subsubsection*{\bpar{#1}{#2}}}

%\newcommand{\bcaption}[2]{\caption{\bpar{#1}{#2}}}


% only one optional arg with renewcomand : trick using bpar in the optionnal arg (should check other packages)
%\renewcommand{\section}[3][]{
%\ifthenelse{\equal{#1}{}}{
%\section{\bpar{#1}{#2}}
%}{
%\section[#1]{\bpar{#2}{#3}}
%}
%}
% -> RECURSIVE PB : WHY ?

%\renewcommand{\section}[2]{\section{\bpar{#1}{#2}}}
% http://tex.stackexchange.com/questions/22576/redefining-sectioning-commands



%%%%%%%%%%
%  Drafting

% writing utilities

% comments	 and responses
%  -> use this comment to ask questions on what other wrote/answer questions with optional arguments (up to 4 answers)
\usepackage{xparse}
\usepackage{ifthen}
\DeclareDocumentCommand{\comment}{m o o o o}
{\ifthenelse{\draft=1}{
    \textcolor{red}{\textbf{C : }#1}
    \IfValueT{#2}{\textcolor{blue}{\textbf{A1 : }#2}}
    \IfValueT{#3}{\textcolor{ForestGreen}{\textbf{A2 : }#3}}
    \IfValueT{#4}{\textcolor{red!50!blue}{\textbf{A3 : }#4}}
    \IfValueT{#5}{\textcolor{Aquamarine}{\textbf{A4 : }#5}}
 }{}
}


% todo
\newcommand{\todo}[1]{
\ifthenelse{\draft=1}{\textcolor{red!50!blue}{\textbf{TODO : \textit{#1}}}}{}
}



% provisory part, removed if not draft

\newcommand{\provisory}[1]{
\ifthenelse{\draft=1}{#1}{}
}















\title{Caractérisation et modélisation de la co-évolution\\
des réseaux de transport et des territoires}

\author{J.~Raimbault$^{1,2,3,\ast}$\\
\texttt{juste.raimbault@iscpif.fr}
}


\institute{$^{1}$UPS CNRS 3611 ISCPIF\\
$^{3}$CASA,UCL\\
$^{2}$UMR CNRS 8504 G{\'e}ographie-cit{\'e}s
}


\date{Prix de Thèse Systèmes Complexes\\\smallskip
Lundi 17 juin 2019\\\smallskip
Institut des Systèmes Complexes
}

\frame{\maketitle}


\sframe{Context scientifique}{

}




\sframe{Interactions entre réseaux et territoires}{


\begin{center}

\includegraphics[width=\linewidth]{figures/example-tangjia.jpg}

\end{center}

\medskip

%\begin{justify}
\textit{Observation d'interactions entre transport et ville dans le Delta\\
de la Rivière des Perles : promotion de la grande vitesse,\\
développement urbain ciblé autour des gares.}
%\end{justify}

}



\sframe{Problématique de la thèse}{


Des dynamiques \textit{co-évolutives} entre réseaux de transport et territoires suggérées par de nombreux travaux (Théorie Evolutive des Villes).

\bigskip

\textbf{Axe 1 : } \textit{Comment définir et caractériser empiriquement ces dynamiques co-évolutives ?}

\bigskip

$\rightarrow$ Connaissance par les seules études empiriques qui reste limitée.
%(données pauvres, cas d'étude, temps long, couplage fort).

\bigskip

\textbf{Axe 2 : } \textit{Comment modéliser la co-évolution des réseaux de transport et des territoires ?}

\bigskip

$\rightarrow$ Utilisation de la modélisation comme outil de connaissance.


}


\sframe{Vers une modélisation ? Cartographie des disciplines}{
	
	%\vspace{-0.5cm}
	\includegraphics[width=\linewidth]{figures/quantep-graph.png}
	
	\vspace{-1cm}
	\textit{Multiples points de vue sur les mêmes objets,\\
	 autant de façons complémentaires de les modéliser.}
}



\begin{frame}
	\frametitle{Lecture par les domaines de connaissance}
	\begin{center}
	\vspace{-0.5cm}
	\begin{tikzpicture}
		\node (a) at (-1,1.5) {Empirique};
		\node (b) at (-1,0) {\includegraphics[width=2.5cm]{figures/domconn-empirical.png}};
			\node (c) at (3,5) {\includegraphics[width=4.5cm]{figures/domconn-schema.pdf}};
			\node (d) at (3,3.5) {Théorique};
			\node (e) at (7,1.5) {Modélisation};
			\node (f) at (7,0) {\includegraphics[width=3.5cm]{figures/domconn-modeling.png}};
			\draw[<->] (a) -- (d);
			\draw[<->] (d) -- (e);
			\draw[<->] (a) -- (e);
		\only<2->{
		\hspace{-0.5cm}
			\node (g) at (-1.5,5) {\textit{Méthodes}};
			\node (h) at (-1.5,4) {\includegraphics[width=3.5cm]{figures/domconn-methods.pdf}};
			\node (i) at (7.5,5) {\textit{Outils}};
			\node (j) at (7.2,3.8) {\includegraphics[width=3.2cm]{figures/domconn-tools.pdf}};
			\node (k) at (3,1) {\textit{Données}};
			\node (l) at (3,-0.5) {\includegraphics[width=3.5cm]{figures/domconn-data.pdf}};
		}
	\end{tikzpicture}
	
	\end{center}
	
	
	
\end{frame}


\sframe{Entrée théorique : définitions}{
 
  \textbf{Objets : } 
  
  \begin{itemize}
  	\item Villes et territoires lus au prisme de la \textit{Théorie Evolutive des Villes}
  	\item Réseaux de transport comme matérialisation de ``projets transactionnels'', suivant la \textit{Théorie Territoriale des Réseaux}
  \end{itemize}
 
 \bigskip

\uncover<2->{

\textbf{Processus : }

\textit{Une définition de la co-évolution à trois niveaux : } 
\begin{enumerate}
	\item \textcolor{blue}{niveau des agents}
	\item \textcolor{green}{niveau des populations d'agents (niches)}
	\item \textcolor{red}{niveau global du système}
\end{enumerate}  

}

\bigskip

\uncover<3->{

\textbf{Entrées : }

\begin{enumerate}
	\item \textcolor{blue}{Entrée empirique (niveau microscopique)}
	\item \textcolor{green}{Entrée par la morphogenèse (niveau de la niche)}
	\item \textcolor{red}{Entrée par la théorie évolutive (niveau global)}
\end{enumerate}

}
 
 
}


\sframe{Elaboration d'une méthode de caractérisation}{

\centering

\includegraphics[width=0.9\linewidth]{figures/causality-method.pdf}

}



\sframe{Des observations empiriques contrastées}{

%Application de la méthode de caractérisation à des cas d'études à différentes échelles
%\bigskip

%\begin{columns}
%\begin{column}{0.5\textwidth}


	\includegraphics[width=\textwidth,height=0.8\textheight]{figures/empirical-southafrica.png}
	
	\smallskip
	
	
	\footnotesize
	
	%\vspace{-0.5cm}
	
	
	\begin{justify}
	\textit{Inversion du sens de la causalité entre croissance des populations et de l'accessibilité ferroviaire en Afrique du Sud au cours du 20ème siècle}
	\end{justify}


%\end{column}
%\vrule\hspace{0.2cm}
%	\begin{column}{0.45\textwidth}

}



\sframe{Des observations empiriques contrastées}{


\begin{center}
	\includegraphics[width=0.8\textwidth]{figures/empirical-grdparis.jpg}
\end{center}	

	
	%\smallskip
	
	\footnotesize
	
	\vspace{-0.5cm}
	\begin{justify}
	\textit{Relations plus complexes dans le cas du gain d'accessibilité permis par le Grand Paris Express et les dynamiques socio-économiques des territoires}
	\end{justify}
	
	
%\end{column}
%\end{columns}

% LEGENDES

%\hspace{0.3cm}



}



\sframe{Aperçu des contributions en modélisation}{
	
	
	\textbf{Echelle macroscopique :}
	\begin{itemize}
		\item Modèles d'interaction entre villes incluant le réseau		
		$\rightarrow$ \textit{Démonstration d'effets de réseau ; exploration\\
		 des régimes d'interaction}
	\end{itemize}

	\bigskip
	
	\uncover<2->{
	\textbf{Echelle mesoscopique :}
	\begin{itemize}
		\item Modèle de morphogenèse couplant forme urbaine et réseau
		
		 $\rightarrow$ \textit{Complémentarité de multiples processus ; calibration au premier et second ordre}
		\item Extension et exploration du modèle Lutecia, incluant la gouvernance du système de transport
	\end{itemize}
}
	
}



\sframe{Modèle macroscopique d'interaction}{

\centering

\includegraphics[width=\textwidth]{figures/macrocoevol_model.png}

}



\sframe{Modèles macroscopiques : régimes de co-évolution}{

%\begin{column}{0.3\textwidth}

\centering

	\includegraphics[width=\linewidth]{figures/macro-regimes.png}
	
	\medskip
	
	%\footnotesize
	
	%\begin{justify}

\raggedleft

\vspace{-0.4cm}

	\textit{Multiples régimes mis en évidence dans des configurations synthétiques}


	%\end{justify}
%\end{column}


}








%%%%%
%% Slide meso
%%%%%

\sframe{Modèles mésoscopiques}{


% - complementarité de multiples heuristiques de croissance de reseau
% - calibration au premier et second ordre
% - Lutecia : vers des modèles plus complexes


\footnotesize

Relation entre forme et fonction (morphogenèse) comme paradigme pour modéliser la co-évolution à l'échelle mésoscopique.

\smallskip

%\begin{columns}
%\begin{column}{0.55\textwidth}

%\footnotesize

\justify
\textit{Un modèle par réaction-diffusion et multi-modélisation de la croissance du réseau : complémentarité des heuristiques, calibration sur les formes et leurs corrélations}

\medskip

{\centering
\includegraphics[width=0.3\linewidth,height=0.7\textheight]{figures/meso-nwgrowth.png}
\includegraphics[width=0.7\linewidth,height=0.7\textheight]{figures/meso-calib.jpg}
}

%\medskip

%$\rightarrow$ \textit{Complémentarité des heuristiques de réseau}
%  pour reproduire des formes existantes

%$\rightarrow$ \textit{Calibration sur les formes et leurs corrélations}

%\end{column}
%\vrule
%\hspace{0.2cm}
%\begin{column}{0.35\textwidth}

%\footnotesize


}


%
\sframe{Modèles mésoscopiques}{


\textit{Le modèle Lutecia : vers une prise en compte de la gouvernance pour la croissance des réseaux de transport}

\bigskip

\includegraphics[width=\linewidth]{figures/meso-lutecia.jpg}




%\end{column}
%\end{columns}


}




\sframe{Mise en perspective}{
\centering

\includegraphics[width=\linewidth]{figures/opening-meta.pdf}

}

\sframe{Ouvertures}{

% Vers des théories intégratives des systèmes territoriaux

% donner des pistes
% theorique, epistemo quanti
% ici parler des developpements Lutecia : modeles.
% multi-modeling
%. -> bien parler de tous les domaines.
%. certains elements plus tangibles : va dans le sens de Morency : compris que quelque chose ici. : strqtgeie explorqtion ; redeveloppement langages scalable. : fait comprendre que chapitre creuser pas possible tel quel, passage a l'echelle necessaire, etc.
% 
% slide de fin : riche, permet ouverture.
%
% 

\textbf{Contributions}

\begin{itemize}
	\item \textit{Définition :} relecture possible de la théorie évolutive, ouvre \\
	 des ponts vers l'économie géographique
	\item \textit{Caractérisation : } nombreuses perspectives d'applications\\
	 en géographie, en sciences territoriales
	\item \textit{Modélisation : } des modèles interdisciplinaires ayant vocation\\
	à être couplés et réutilisés
\end{itemize}

\bigskip

\uncover<2->{

\textbf{Perspectives}

\begin{itemize}
	\item Adaptation de Lutecia pour le développement de méthodes d'exploration de modèles spatiaux (développement d'OpenMole)
	\item \textit{Vers des théories intégrées des systèmes territoriaux : } modèles multi-échelles et couplage de la théorie évolutive avec la théorie \\
	 du \textit{Scaling}.
\end{itemize}
}

}




\sframe{Problématique et plan dans les domaines de connaissance}{

\centering
\includegraphics[width=\linewidth]{figures/domconn-organisation.pdf}

}







\sframe{Mise en perspective}{
\centering

\includegraphics[width=\linewidth]{figures/opening-meta.pdf}

}


\sframe{Analyse réflexive}{

}





%%%%%%%%%%%%%%%%%%%%%
%\begin{frame}[allowframebreaks]
%\frametitle{References}
%\bibliographystyle{apalike}
%\bibliography{/Users/juste/ComplexSystems/CityNetwork/Biblio/Bibtex/CityNetwork,biblio}
%\end{frame}
%%%%%%%%%%%%%%%%%%%%%%%%%%%%



\end{document}







