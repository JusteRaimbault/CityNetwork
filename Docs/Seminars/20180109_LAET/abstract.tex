\documentclass[11pt]{article}

% general packages without options
\usepackage{amsmath,amssymb,bbm}
% graphics
\usepackage{graphicx}
% text formatting
\usepackage[document]{ragged2e}
\usepackage{pagecolor,color}

\newcommand{\noun}[1]{\textsc{#1}}

\usepackage[utf8]{inputenc}
\usepackage[T1]{fontenc}
% geometry
\usepackage[margin=2cm]{geometry}

\usepackage{multicol}
\usepackage{setspace}

\usepackage{natbib}
\setlength{\bibsep}{0.0pt}

\usepackage[french]{babel}

% layout : use fancyhdr package
%\usepackage{fancyhdr}
%\pagestyle{fancy}

% variable to include comments or not in the compilation ; set to 1 to include
\def \draft {1}


% writing utilities

% comments and responses
%  -> use this comment to ask questions on what other wrote/answer questions with optional arguments (up to 4 answers)
\usepackage{xparse}
\usepackage{ifthen}
\DeclareDocumentCommand{\comment}{m o o o o}
{\ifthenelse{\draft=1}{
    \textcolor{red}{\textbf{C : }#1}
    \IfValueT{#2}{\textcolor{blue}{\textbf{A1 : }#2}}
    \IfValueT{#3}{\textcolor{ForestGreen}{\textbf{A2 : }#3}}
    \IfValueT{#4}{\textcolor{red!50!blue}{\textbf{A3 : }#4}}
    \IfValueT{#5}{\textcolor{Aquamarine}{\textbf{A4 : }#5}}
 }{}
}

% todo
\newcommand{\todo}[1]{
\ifthenelse{\draft=1}{\textcolor{red!50!blue}{\textbf{TODO : \textit{#1}}}}{}
}


\makeatletter


\makeatother


\begin{document}







\title{Identification de causalités dans des données spatio-temporelles
\bigskip\\
\textit{Séminaire LAET - 9 Janvier 2018}
}
\author{\noun{Juste Raimbault}$^{1,2}$\medskip\\
$^1$ UMR CNRS 8504 Géographie-cités\\
$^2$ UMR-T IFSTTAR 9403 LVMT
}
\date{}

\maketitle

\justify

\pagenumbering{gobble}


\textbf{Mots-clés : }\textit{Causalité Spatio-temporelle ; Interactions Réseaux-Territoires ; Morphogenèse Urbaine ; Grand Paris ; Afrique du Sud ; Réseaux Ferré Français ; Co-évolution}

\bigskip

Cette présentation contribue à la compréhension des processus spatio-temporels fortement couplés, en proposant une méthode générique basée sur la causalité de Granger. Plus précisément, la fouille de donnée et l'analyse spatiale sont utilisées pour revisiter l'étude des motifs de corrélations retardées, ce qui permet ainsi de comprendre les relations causales faibles entre différentes composantes des systèmes territoriaux.

La méthode est validée dans un premier temps sur des données synthétiques, par l'identification de régimes de causalité et de leur diagramme de phase pour un modèle de morphogenèse urbaine couplant croissance du réseau et de la densité. Les graphes de relations causales entre variables qui sont obtenus ne sont pas reliés de manière simple aux paramètres régissant les interactions microscopiques, ce qui témoigne de la capture d'une causalité complexe par la méthode.

L'application au cas des projets de transport du Grand Paris suggère différents liens entre les dynamiques territoriales, plus particulièrement socio-économiques et foncières, et la croissance anticipée du réseau, comme par exemple un processus de spéculation immobilière dans les quartiers des gares planifiées. Nous illustrons d'autres applications sur des temporalités plus longues : dans le cas des dynamiques d'accessibilité ferroviaire et de population urbaines en Afrique du Sud de 1911 à 1990, nous démontrons une inversion du sens de la causalité coïncidant avec la mise en place des politiques d'apartheid, suggérant un effet de celles-ci au second ordre, c'est à dire sur les relations dynamiques entre réseaux et territoires. D'autre part, l'application à la France entre 1830 et 1999 avec le réseau ferroviaire ne fournit quant à elle aucune corrélation significative, confirmant qu'il n'est dans certains cas pas possible d'établir des relations claires.

Nous discutons finalement ces résultats en les plaçant dans la perspective plus large de la co-évolution des réseaux de transport et des territoires, dont nous donnons une définition ainsi que des pistes de modélisation. L'ensemble de ce travail propose ainsi un regard original sur la question des effets structurants des infrastructures de transport.


%%%%%%%%%%%%%%%%%%%%
%% Biblio
%%%%%%%%%%%%%%%%%%%%
%\tiny

%\begin{multicols}{2}

%\setstretch{0.3}
%\setlength{\parskip}{-0.4em}


%\bibliographystyle{apalike}
%\bibliography{/Users/juste/ComplexSystems/CityNetwork/Biblio/Bibtex/CityNetwork}%,biblio}
%\end{multicols}



\end{document}

