\documentclass[11pt]{article}

% general packages without options
\usepackage{amsmath,amssymb,bbm}
% graphics
\usepackage{graphicx}
% text formatting
\usepackage[document]{ragged2e}
\usepackage{pagecolor,color}

\newcommand{\noun}[1]{\textsc{#1}}

\usepackage[utf8]{inputenc}
\usepackage[T1]{fontenc}
% geometry
\usepackage[margin=1.5cm]{geometry}

\usepackage{multicol}
\usepackage{setspace}

\usepackage{natbib}
\setlength{\bibsep}{0.0pt}

\usepackage[french]{babel}

% layout : use fancyhdr package
%\usepackage{fancyhdr}
%\pagestyle{fancy}

% variable to include comments or not in the compilation ; set to 1 to include
\def \draft {1}


% writing utilities

% comments and responses
%  -> use this comment to ask questions on what other wrote/answer questions with optional arguments (up to 4 answers)
\usepackage{xparse}
\usepackage{ifthen}
\DeclareDocumentCommand{\comment}{m o o o o}
{\ifthenelse{\draft=1}{
    \textcolor{red}{\textbf{C : }#1}
    \IfValueT{#2}{\textcolor{blue}{\textbf{A1 : }#2}}
    \IfValueT{#3}{\textcolor{ForestGreen}{\textbf{A2 : }#3}}
    \IfValueT{#4}{\textcolor{red!50!blue}{\textbf{A3 : }#4}}
    \IfValueT{#5}{\textcolor{Aquamarine}{\textbf{A4 : }#5}}
 }{}
}

% todo
\newcommand{\todo}[1]{
\ifthenelse{\draft=1}{\textcolor{red!50!blue}{\textbf{TODO : \textit{#1}}}}{}
}


\makeatletter


\makeatother


\begin{document}







\title{Application de la Morphogen{\`e}se de R{\'e}seaux Biologiques {\`a} la Conception Optimale d’Infrastructures de Transport
\bigskip\bigskip\\
\textit{Poster pr{\'e}sent{\'e} aux Rencontres Dynamite 2015}
\medskip\\5 mai 2015
}
\author{\noun{Juste Raimbault}$^{1,2}$ et \noun{Jorge Gonzalez-Suitt}$^{2}$\medskip\\
$^1$ UMR CNRS 8504 Géographie-cités\\
$^2$ Ecole Polytechnique\\
}
\date{}

\maketitle

\justify

\pagenumbering{gobble}


% Keywords: Transportation planning; Multi-objective optimization; Network self-generation; Biologically inspired network; Emergence in transportation systems.
% The question of multi-objective optimization of corridor path for a new project of transportation infrastructure remains a key issue in transportation planning and territorial intelligence. Many top-down approaches have already been tackled in the literature and are operationally mature. Also mixed top-down and bottom-up approaches such as Land-Use Transport Interaction models show a strong potential in sustainable design of new infrastructures [4, 2]. To compare corridor alternatives, we use a pure bottom-up approach for network self-generation [1]. In particular, this work is based on the model developed in [3], where slime-mold evolution is simulated in order to build a biologically inspired network with emergent robustness and efficiency properties. We constrain this model to take into account the preexistent road system as an exogenous parameter, and we assess the response of the system to the presence of the new infrastructure. After having internally validated the model and performed sensitivity analysis, we apply it first on an abstract situation, then on a real case with GIS data, proposing in both cases a bi-objective Pareto set of optimal solutions regarding robustness and efficiency of the generated network. The proximity of the obtained solution with the one actually proposed by planners validates externally the application of this bottom-up approach to the planning problem.


\textbf{Mots-clés : }\textit{Planification des transports ; Optimisation multi-objectif ; Morphogen{\`e}se des réseaux ; Réseaux bio-inspiré}

\medskip

Le problème de l'optimisation multi-objectif des corridors de transport pour un nouveau projet d'infrastructure de transport reste crucial en planification des transports et en intelligence territoriale. Un certains nombre d'approches \emph{top-down} existent dans la littérature et sont matures sur le plan opérationnel. Des approches mêlant \emph{top-down} et \emph{bottom-up} comme les modèles d'interaction entre usage du sol et transport (LUTI) ont également un fort potentiel pour la conception d'infrastructures soutenables \citep{chang2006models,wegener2004land}. Pour comparer les alternatives entre corridors, nous utilisons une approche entièrement \emph{bottom-up} pour la morphogenèse des réseaux \citep{bebber2007biological}. Plus particulièrement, celle-ci se base sur le modèle développé par \cite{tero2010rules}, au sein duquel l'évolution d'un organisme \emph{slime-mold} est simulée afin de construire un réseau bio-inspiré avec des propriétés émergentes de robustesse et d'efficience. Nous contraignons ce modèle pour prendre en compte un réseau routier préexistant comme paramètre exogène, et nous évaluons la réponse du système à la présence d'une nouvelle infrastructure. Après une validation interne du modèle et une analyse de sensibilité, celui-ci est appliqué sur une configuration stylisée correspondant à une configuration réelle. Nous produisons ainsi un ensemble d'infrastructures optimales au sens de Pareto pour l'optimisation bi-objectif pour la robustesse et l'efficience du réseau généré. Le modèle est également appliqué à un problème de desserte optimale, confirmant les potentialités de cette approche par morphogen{\`e}se pour résoudre des problèmes de planification.





%%%%%%%%%%%%%%%%%%%%
%% Biblio
%%%%%%%%%%%%%%%%%%%%
%\tiny

%\begin{multicols}{2}

%\setstretch{0.3}
%\setlength{\parskip}{-0.4em}


\bibliographystyle{apalike}
\bibliography{biblio}
%\end{multicols}



\end{document}
