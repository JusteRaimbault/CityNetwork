%%%%%%%%%%%%%%%%%%%%%%%%%%%%%
% Standard header for working papers
%
% WPHeader.tex
%
%%%%%%%%%%%%%%%%%%%%%%%%%%%%%

\documentclass[11pt]{article}

% packages without options
\usepackage{amsmath,bbm}

% geometry
\usepackage[margin=2cm]{geometry}






\title{Definitions
\bigskip\\
\textit{Technical Note}
}
\author{\noun{Juste Raimbault}}
\date{January 2016}


\maketitle

\justify



%%%%%%%%%%%%%%%%%%%%
\subsection*{Process}

\paragraph{Geography}

Pumain describes in~\cite{hypergeo} a process as ``a chain of actions and events''. As for dynamics, processes are typical of diachronic temporal approaches, but imply a causality between events, revealing their operational logic. Processes in geography are crucial to understand the production of spatial structures and systems, because of their interpretability.






%%%%%%%%%%%%%%%%%%%%
\subsection*{Accessibility}

Technical vs theoretical definitions of accessibility ?







%%%%%%%%%%%%%%%%%%%%
%% Biblio
%%%%%%%%%%%%%%%%%%%%

\bibliographystyle{apalike}
\bibliography{/Users/Juste/Documents/ComplexSystems/CityNetwork/Biblio/Bibtex/CityNetwork}


\end{document}
