%%%%%%%%%%%%%%%%%%%%%%%%%%%%%
% Standard header for working papers
%
% WPHeader.tex
%
%%%%%%%%%%%%%%%%%%%%%%%%%%%%%

\documentclass[11pt]{article}

% packages without options
\usepackage{amsmath,bbm}

% geometry
\usepackage[margin=2cm]{geometry}






\title{Statistical Investigations on Real Data\bigskip\\
\textit{Technical Note}
}
\author{\noun{Juste Raimbault}}
\date{}


\maketitle

\justify


\begin{abstract}
We describe empirical investigations to be done on real data, through statistical analysis. We formalize therein various hypothesis to be tested.
\end{abstract}


\section{Introduction}


\section{Context Formalization}

\subsection{Variables}

\paragraph{Description}

We assume a dynamic transportation network $n(\vec{x},t)$ within a dynamic territorial landscape $\vec{T}(\vec{x},t)$, which components are to simplify population $p(\vec{x},t)$ and employments $e(\vec{x},t)$. Data is structured the following way :
\begin{itemize}
\item Observation of territorial variables are discretized in space and in time, i.e. the spatial field $\vec{T}$ is summarized by $\mathbf{T} = \left(\vec{T}(\vec{x}_i,t_j^{(T)})\right)_{i,j}$ with $1\leq i \leq N$ and $1\leq j \leq T$. They concretely correspond to census on administrative units (\emph{communes} in our case) at different dates.
\item Network has a continuous spatial position but
\end{itemize}



\paragraph{Definitions}



\subsection{Accessibility}

% accessibility : need to introduce it ?
%  -> read Weibull

The notion of accessibility has been central to regional science since its introduction and systematization in planning around 1970. 

\paragraph{Existence of accessibility}

An elegant axiomatic definition is derived in~\cite{weibull1976axiomatic}. Starting from expected properties of an accessibility function $A$ that associate a value to \emph{attraction} $a$ and distance $d$, defined on the set of discrete spatial configurations $\mathcal{C} = \cup_{n\in \mathbb{N}}{(d_i,a_i)_{1\leq i \leq n}}$. These properties include (among technical others with no thematic meaning) :
\begin{enumerate}
\item $A$ is invariant regarding the order of the configuration
\item $A$ decrease with distance at fixed attraction and increase with attraction at fixed distance
\item $A$ is invariant when adding null attractions and constant configurations
\end{enumerate}

A canonical decomposition of any accessibility function 


\paragraph{Continuous approach and accessibility potential}

% Paul : Helmoltz-Hodge theorem to infer potential field from speed spatial field ?
%  Q : what are trajectories ? dirac field has no rotational -> continuous approach does not work ?






\section{Statistical Tests}


\subsection{Bivariate linear models}

\subsection{Autocorrelated univariate models}

\subsection{Autocorrelated multivariate models}

\subsection{Granger causality tests}

\cite{xie2009streetcars} use Granger causality to link transit with land-use changes.


\subsection{Autoregressive multivariate models}



\subsection{Autoregressive autocorrelated multivariate models}



%%%%%%%%%%%%%%%%%%%%
%% Biblio
%%%%%%%%%%%%%%%%%%%%

\bibliographystyle{apalike}
\bibliography{/Users/Juste/Documents/ComplexSystems/CityNetwork/Biblio/Bibtex/CityNetwork}


\end{document}
