
\documentclass[11pt]{article}

% general packages without options
\usepackage{amsmath,amssymb,bbm}
% graphics
\usepackage{graphicx}
% text formatting
\usepackage[document]{ragged2e}
\usepackage{pagecolor,color}

\newcommand{\noun}[1]{\textsc{#1}}

\usepackage[utf8]{inputenc}
\usepackage[T1]{fontenc}
% geometry
\usepackage[margin=2cm]{geometry}

\usepackage{multicol}
\usepackage{setspace}

\usepackage{natbib}
\setlength{\bibsep}{0.0pt}


% layout : use fancyhdr package
%\usepackage{fancyhdr}
%\pagestyle{fancy}

\makeatletter


\makeatother


\begin{document}







\title{Manifeste pour une Géographie Intégrée\\
\textit{Proposition de Communication, JIG 2017}
}
\author{\noun{Juste Raimbault}$^{1,2}$, \noun{Julien Migozzi}$^{1,3}$ et \noun{Thibault Le Corre}$^1$\\
$^1$ UMR CNRS 8504 Géographie-cités\\
$^2$ UMR-T IFSTTAR 9403 LVMT\\
$^3$ Institut de Géographie Alpine
}
\date{}

\maketitle

\justify

\pagenumbering{gobble}


\textbf{Mots-clés : }\textit{}

\medskip


\paragraph{Contexte}

Les bouleversements techniques et méthodologiques qu'une discipline peut subir sont souvent accompagnés de profondes mutations épistémologiques, voire de la nature même de la discipline. Il est indéniable que les différentes géographies sont actuellement dans cette situation, au regard des nouvelles opportunités en terme de données et de puissance de calcul. Les visions pour des directions futures sont diverses, et peuvent donner l'emphase sur les données massives~(\cite{batty2012smart}) ou sur la simulation de modèles par calcul intensif~(\cite{pumain2017urban}). Il est crucial de rester conscient des pièges que tend l'usage inconsidéré de ces nouvelles techniques~(\cite{raimbault2016cautious}), et une intégration saine des théories, connaissances, outils, méthodes est une réponse possible.


\paragraph{Pour une Géographie Intégrée}

Cette communication se positionne de manière originale en proposant un cadre de connaissances alternatif pour les études ayant une composante quantitative, ou plus précisément se posant dans la lignée de la Géographie Théorique et Quantitative (TQG). Ce cadre tente de répondre aux contraintes suivantes : (i) transcender les frontières artificielles entre quantitatif et qualitatif ; (ii) ne pas favoriser de composante particulière parmi les moyens de production de connaissance (aussi divers que l'ensemble des méthodes qualitatives et quantitatives classiques, les méthodes de modélisation, les approches théoriques, les données, les outils), mais bien le développement conjoint de chaque composante. Nous étendons le cadre de connaissances de~\cite{livet2010ontology}, qui consacre le triptyque des domaines empiriques, conceptuels et de la modélisation, en y ajoutant les domaines à part entière que sont les méthodes, les outils (qu'on peut voir comme des proto-méthodes) et les données. Les interactions entre chaque domaine sont détaillées, comme par exemple le passage des méthodes vers les outils qui consiste en l'implémentation, ou le passage de l'empirique aux méthodes comme prospection méthodologique. Toute démarche de production de connaissance, vue comme une \emph{perspective} au sens de~\cite{giere2010scientific}, est une combinaison complexe des six domaines, les fronts de connaissance dans chacun étant en co-évolution. Nous nommons notre cadre de connaissance \emph{Géographie Intégrée}, pour souligner à la fois l'intégration des différents domaines mais aussi des connaissances qualitatives et quantitatives, puisque les deux se fondent dans chacun des domaines.


\paragraph{Cas d'étude}

L'aventure de l'ERC Geodivercity~\cite{pumain2017book} est l'allégorie du cadre proposé. L'intégration de la théorie, de l'empirique, de la modélisation, mais aussi de la technique et de la méthode, n'a jamais été aussi creusée et renforcée que dans les divers développements du projet. Sans l'accès à la grille de calcul et aux nouveaux algorithmes d'exploration permis par OpenMole, les connaissances tirées du modèle SimpopLocal auraient été moindres, mais les développements techniques ont aussi été conduits par la demande thématique. Nous illustrons par d'autres exemples concrets le rôle de chacun des domaines, et le besoin d'intégration entre chaque.

\textit{Deux cas marchés immobilier : quanti-quali, collection des données et intégration, choix des méthodes et besoins d'outils, etc ?}

Notre troisième exemple étudie les interactions entre réseaux et territoires, du point de vue de la morphologie urbaine et des réseaux. Les corrélations entre indicateurs de forme urbaine et mesures de réseau sont calculées sur l'ensemble de l'Europe. Des outils spécifiques de simplification du réseau routier sont développés pour rendre les traitement faisable en terme de temps de calcul et de mémoire utilisée, et des méthodes liant correlations temporelles et spatiales sont proposées. Les conclusions tirées ont nécessité une intégration des différents domaines.


\paragraph{Application}

Nous proposons finalement d'appliquer notre cadre à une proposition de projet de recherche à grande échelle, portant sur une connaissance comparative à grande échelle des marchés immobiliers. 

\textit{description du projet : contexte, objectifs}

Celui-ci implique des démarches qualitatives poussées (terrains laborieux), des outils d'analyse réellement appropriés encore à developper, des appareils théoriques avancés et des développements techniques potentiellement avancés (pour exemple parmi d'autres : collecte automatique de données, traitement des données massives, plateforme de crowdsourcing). Ces multiples aspects, qui seront nécessairement intégrés, illustrent une application directe de notre cadre de connaissances.








%%%%%%%%%%%%%%%%%%%%
%% Biblio
%%%%%%%%%%%%%%%%%%%%
%\tiny

%\begin{multicols}{2}

%\setstretch{0.3}
%\setlength{\parskip}{-0.4em}


\bibliographystyle{apalike}
\bibliography{biblio}
%\end{multicols}



\end{document}
