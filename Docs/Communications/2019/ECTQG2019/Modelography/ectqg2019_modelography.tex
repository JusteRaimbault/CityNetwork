\documentclass[11pt]{article}
%%%%%%%%%%%%%%%%%%%%%%%%%%%%%
% Standard header for working papers
%
% WPHeader.tex
%
%%%%%%%%%%%%%%%%%%%%%%%%%%%%%

\documentclass[11pt]{article}



%%%%%%%%%%%%%%%%%%%%%%%%%%
%% TEMPLATES
%%%%%%%%%%%%%%%%%%%%%%%%%%


% Simple Tabular

%\begin{tabular}{ |c|c|c| } 
% \hline
% cell1 & cell2 & cell3 \\ 
% cell4 & cell5 & cell6 \\ 
% cell7 & cell8 & cell9 \\ 
% \hline
%\end{tabular}





%%%%%%%%%%%%%%%%%%%%%%%%%%
%% Packages
%%%%%%%%%%%%%%%%%%%%%%%%%%



% encoding 
\usepackage[utf8]{inputenc}
\usepackage[T1]{fontenc}


% general packages without options
\usepackage{amsmath,amssymb,amsthm,bbm}

% graphics
\usepackage{graphicx,transparent,eso-pic}

% text formatting
\usepackage[document]{ragged2e}
\usepackage{pagecolor,color}
%\usepackage{ulem}
\usepackage{soul}


% conditions
\usepackage{ifthen}


\usepackage{natbib}


%%%%%%%%%%%%%%%%%%%%%%%%%%
%% Maths environment
%%%%%%%%%%%%%%%%%%%%%%%%%%

%\newtheorem{theorem}{Theorem}[section]
%\newtheorem{lemma}[theorem]{Lemma}
%\newtheorem{proposition}[theorem]{Proposition}
%\newtheorem{corollary}[theorem]{Corollary}

%\newenvironment{proof}[1][Proof]{\begin{trivlist}
%\item[\hskip \labelsep {\bfseries #1}]}{\end{trivlist}}
%\newenvironment{definition}[1][Definition]{\begin{trivlist}
%\item[\hskip \labelsep {\bfseries #1}]}{\end{trivlist}}
%\newenvironment{example}[1][Example]{\begin{trivlist}
%\item[\hskip \labelsep {\bfseries #1}]}{\end{trivlist}}
%\newenvironment{remark}[1][Remark]{\begin{trivlist}
%\item[\hskip \labelsep {\bfseries #1}]}{\end{trivlist}}

%\newcommand{\qed}{\nobreak \ifvmode \relax \else
%      \ifdim\lastskip<1.5em \hskip-\lastskip
%      \hskip1.5em plus0em minus0.5em \fi \nobreak
%      \vrule height0.75em width0.5em depth0.25em\fi}



%% Commands

\newcommand{\noun}[1]{\textsc{#1}}


%% Math

% Operators
\DeclareMathOperator{\Cov}{Cov}
\DeclareMathOperator{\Var}{Var}
\DeclareMathOperator{\E}{\mathbb{E}}
\DeclareMathOperator{\Proba}{\mathbb{P}}

\newcommand{\Covb}[2]{\ensuremath{\Cov\!\left[#1,#2\right]}}
\newcommand{\Eb}[1]{\ensuremath{\E\!\left[#1\right]}}
\newcommand{\Pb}[1]{\ensuremath{\Proba\!\left[#1\right]}}
\newcommand{\Varb}[1]{\ensuremath{\Var\!\left[#1\right]}}

% norm
\newcommand{\norm}[1]{\left\lVert #1 \right\rVert}



% argmin
\DeclareMathOperator*{\argmin}{\arg\!\min}


% amsthm environments
\newtheorem{definition}{Definition}
\newtheorem{proposition}{Proposition}
\newtheorem{assumption}{Assumption}

%% graphics

% renew graphics command for relative path providment only ?
%\renewcommand{\includegraphics[]{}}


\usepackage{url}





% geometry
\usepackage[margin=2cm]{geometry}



% changes

\usepackage{soul}
\soulregister\cite7
\soulregister\citep7
\soulregister\ref7

\usepackage[final]{changes}
%\usepackage{changes}


\setaddedmarkup{\textcolor{black}{\hl{#1}}}
\setdeletedmarkup{\textcolor{red}{\sout{#1}}}



\usepackage{CJKutf8}
%\begin{CJK*}{UTF8}{zhsong}
%文章内容。
%\clearpage\end{CJK*}
\newcommand{\cn}[1]{
  \begin{CJK*}{UTF8}{gbsn}
  #1
  \end{CJK*}
}



% layout : use fancyhdr package
%\usepackage{fancyhdr}
%\pagestyle{fancy}
%
%\makeatletter
%
%\renewcommand{\headrulewidth}{0.4pt}
%\renewcommand{\footrulewidth}{0.4pt}
%\fancyhead[RO,RE]{}
%\fancyhead[LO,LE]{Models for the co-evolution of cities and networks}
%\fancyfoot[RO,RE] {\thepage}
%\fancyfoot[LO,LE] {}
%\fancyfoot[CO,CE] {}
%
%\makeatother
%

%%%%%%%%%%%%%%%%%%%%%
%% Begin doc
%%%%%%%%%%%%%%%%%%%%%

\begin{document}





\begin{document}

\title{A meta-analysis of models for interactions between transportation networks and territories
\\\medskip
\textit{ECTQG 2019 - Abstract proposal}
}
\author{\noun{J. Raimbault}$^{1,2,3,\ast}$\medskip\\
$^1$ UPS CNRS 3611 ISC-PIF\\
$^2$ CASA, UCL\\
$^3$ UMR CNRS 8504 G{\'e}ographie-cit{\'e}s\\
\medskip\\
$^{\ast}$\texttt{juste.raimbault@polytechnique.edu}
}
\date{}

\maketitle

\justify

\pagenumbering{gobble}


\textbf{Keywords: }\textit{Network-territories interaction models; Systematic review; Meta-analysis}

\medskip


The dynamics of territorial systems have extensively been studied in their interaction with transportation networks which are assumed to play a significant role in their trajectories. Therefore, several modeling approaches focusing on the interactions between transportation networks and territories have been introduced by various disciplines, including for example land-use transport interaction models from planning and transportation science, spatial interaction models or co-evolution models from geography, network growth models from physics. This contribution introduces a systematic review and meta-analysis to understand the nature and properties of these models in relation to their disciplinary context. We construct a corpus of models through a systematic review. The raw corpus after initial keyword requests is composed by around 3800 papers, which were screened for inclusion first on their titles (297 papers kept), then on their full-text content resulting in a study corpus of 145 papers. For each model, properties are extracted including the type of coupling (weak or strong and the direction) between network and territory, spatial and temporal scales, the methodology used, and the discipline, in order to proceed to a meta-analysis of these. Exploratory analysis confirms the diversity of approaches existing, whereas statistical analysis links type of models and disciplines with properties, showing for example the strong influence of the type of coupling with time scale, or of the discipline on the spatial scale. We finally use random forest regression to compare the relative importance of variables to explain model type, and show that among different way to define disciplinary belonging, position in the citation network has the largest influence. This work thus provide a systematic and broad overview on the diversity of approaches to model interaction between networks and territories, and foster the possibilities of a reflexive positioning in the context of building new models.



%%%%%%%%%%%%%%%%%%%%
%% Biblio
%%%%%%%%%%%%%%%%%%%%
%\tiny

%\begin{multicols}{2}

%\setstretch{0.3}
%\setlength{\parskip}{-0.4em}


%\footnotesize

%\bibliographystyle{apalike}
%\bibliography{biblio}
%\end{multicols}



\end{document}
