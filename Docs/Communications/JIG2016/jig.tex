%%%%%%%%%%%%%%%%%%%%%%%%%%%%%
% Standard header for working papers
%
% WPHeader.tex
%
%%%%%%%%%%%%%%%%%%%%%%%%%%%%%

\documentclass[11pt]{article}

% packages without options
\usepackage{amsmath,bbm}

% geometry
\usepackage[margin=2cm]{geometry}






\title{\vspace{-2.5cm}title\\\medskip
\textit{Communication Proposal, JIG 2016}
}
\author{\small\noun{Juste Raimbault}$^{1,2}$\\
\small(1) UMR CNRS 8504 Géographie-cités and (2) UMR-T IFSTTAR 9403 LVMT
}
%\date{15 octobre 2015}
\date{}

\maketitle

\justify

\pagenumbering{gobble}

\textbf{Keywords : }\textit{}

\medskip

The relation between transportation networks and territorial development cannot be ignored by most approaches in territorial science, but remains poorly understood in a quantitative way : the example of \emph{the myth of structural effects of transportation infrastructures}~\cite{offner1993effets} shows that simple causal assumptions do not hold when trying to explain the ``co-evolution'' between transportation network and static components of territorial systems. Dynamical modeling including both evolutions on long time scales has been emphasized as a cornerstone for tackling such research questions in the case of cities systems on long time scale (\cite{bretagnolle:tel-00459720}, p. 152-163). However, a broad interdisciplinary state-of-the-art, based on algorithmic systematic review, done in~\cite{raimbault2015models} shows the quasi-absence of simulations models of that type. At different spatial and temporal scales, various models were proposed, such as e.g. LUTI models at a middle scale where networks are considered static~\cite{iacono2008models}, or network growth models on a longer time scale~\cite{xie2009modeling}, none of which including both network and territory in a dynamic way. The purpose of this communication is in a first part to develop results obtained through a first family of models of simulation that are agent-based toy-models, and then to propose a theoretical framework

A first family of model explores the weak coupling between an population density generation model that is a generalization of the diffusion-limited aggregation model~\cite{batty2006hierarchy}, and network generation heuristics for which biological network generation~\cite{tero2006physarum} and generalized gravity potential rupture are tested.


The conclusions of these first modeling experiments unveil or confirm requirements for a theoretical framework aimed to understand territorial systems. They include in particular a framing of the notion of coupling between subsystems, a precise definition of scale and an emphasis on emergence to take into account multi-scale aspects of systems, the superposition of heterogeneous views and components of a system. Starting from a perspectivist point of view~\cite{giere2010scientific}, we consider a system as a set of perspectives consisting in ontological sets~\cite{livet2010} associated with dataflow machines~\cite{golden2012modeling}. Formal pre-orders between subsets of ontologies, constructed from emergence relations~\cite{bedau2002downward}, yield after a canonical reduction an unique forest representing the structure of the system. Strong coupled components reside within nodes, whereas a temporal scale and ``thematic'' scale (scale for a state variable) can be associated to each level of the forest by construction from dataflow machines timescales. This framework is formally self-consistent and meets our requirements.





%%%%%%%%%%%%%%%%%%%%
%% Biblio
%%%%%%%%%%%%%%%%%%%%
\tiny

\begin{multicols}{2}

\setstretch{0.3}
%\setlength{\parskip}{-0.4em}


\bibliographystyle{apalike}
\bibliography{/Users/Juste/Documents/ComplexSystems/CityNetwork/Biblio/Bibtex/CityNetwork}

\end{multicols}

\end{document}
