\documentclass[11pt]{article}

% general packages without options
\usepackage{amsmath,amssymb,bbm}
% graphics
\usepackage{graphicx}
% text formatting
\usepackage[document]{ragged2e}
\usepackage{pagecolor,color}

\newcommand{\noun}[1]{\textsc{#1}}

\usepackage[utf8]{inputenc}
\usepackage[T1]{fontenc}
% geometry
\usepackage[margin=2cm]{geometry}

\usepackage{multicol}
\usepackage{setspace}

\usepackage{natbib}
\setlength{\bibsep}{0.0pt}

%\usepackage[french]{babel}

% layout : use fancyhdr package
%\usepackage{fancyhdr}
%\pagestyle{fancy}

% variable to include comments or not in the compilation ; set to 1 to include
\def \draft {1}


% writing utilities

% comments and responses
%  -> use this comment to ask questions on what other wrote/answer questions with optional arguments (up to 4 answers)
\usepackage{xparse}
\usepackage{ifthen}
\DeclareDocumentCommand{\comment}{m o o o o}
{\ifthenelse{\draft=1}{
    \textcolor{red}{\textbf{C : }#1}
    \IfValueT{#2}{\textcolor{blue}{\textbf{A1 : }#2}}
    \IfValueT{#3}{\textcolor{ForestGreen}{\textbf{A2 : }#3}}
    \IfValueT{#4}{\textcolor{red!50!blue}{\textbf{A3 : }#4}}
    \IfValueT{#5}{\textcolor{Aquamarine}{\textbf{A4 : }#5}}
 }{}
}

% todo
\newcommand{\todo}[1]{
\ifthenelse{\draft=1}{\textcolor{red!50!blue}{\textbf{TODO : \textit{#1}}}}{}
}


\makeatletter


\makeatother


\begin{document}







\title{Complexity, Complexities and Complex Knowledges
\bigskip\\
\textit{Geodivercity International Workshop, 12-13th October 2017}\\
\textit{Discussion of Pr. Batty: ``Complexity in Urban Systems''}
}
\author{\noun{Juste Raimbault}$^{1,2}$\medskip\\
$^1$ UMR CNRS 8504 Géographie-cités\\
$^2$ UMR-T IFSTTAR 9403 LVMT
}
\date{}

\maketitle

\justify

\pagenumbering{gobble}


%\textbf{Keywords : }\textit{}

\medskip


This discussion aims at exploring the consequences of the existence of different types of complexities in the study of socio-technical systems. We illustrate links between three different type of complexities, namely weak emergence in the sense of~\cite{bedau2002downward}, computational complexity and informational complexity. Emergence and computational complexity are closely linked, as suggested by the computational capabilities of many complex systems.
%The Turing complete properties of many complex systems suggest some cases where emergence implies computational complexity, whereas the recently shown NP-completeness of approximating elementary physical equations~[\cite{2014arXiv1403.7686B}] gives the opposite implication.
Informational complexity can also been shown to play a crucial role in self-organisation, through spatial patterns of information flows for example.
We postulate that complex knowledge of socio-technical systems must capture conjointly different types of complexities and their interactions, and that this property is an other expression of a necessary reflexive nature of complex knowledge.









%%%%%%%%%%%%%%%%%%%%
%% Biblio
%%%%%%%%%%%%%%%%%%%%
%\tiny

%\begin{multicols}{2}

%\setstretch{0.3}
%\setlength{\parskip}{-0.4em}


\bibliographystyle{apalike}
\bibliography{/Users/juste/ComplexSystems/CityNetwork/Biblio/Bibtex/CityNetwork}%,biblio}
%\end{multicols}



\end{document}

