%%%%%%%%%%%%%%%%%%%%%%%%%%%%%
% Standard header for working papers
%
% WPHeader.tex
%
%%%%%%%%%%%%%%%%%%%%%%%%%%%%%

\documentclass[11pt]{article}



%%%%%%%%%%%%%%%%%%%%%%%%%%
%% TEMPLATES
%%%%%%%%%%%%%%%%%%%%%%%%%%


% Simple Tabular

%\begin{tabular}{ |c|c|c| } 
% \hline
% cell1 & cell2 & cell3 \\ 
% cell4 & cell5 & cell6 \\ 
% cell7 & cell8 & cell9 \\ 
% \hline
%\end{tabular}





%%%%%%%%%%%%%%%%%%%%%%%%%%
%% Packages
%%%%%%%%%%%%%%%%%%%%%%%%%%



% encoding 
\usepackage[utf8]{inputenc}
\usepackage[T1]{fontenc}


% general packages without options
\usepackage{amsmath,amssymb,amsthm,bbm}

% graphics
\usepackage{graphicx,transparent,eso-pic}

% text formatting
\usepackage[document]{ragged2e}
\usepackage{pagecolor,color}
%\usepackage{ulem}
\usepackage{soul}


% conditions
\usepackage{ifthen}


\usepackage{natbib}


%%%%%%%%%%%%%%%%%%%%%%%%%%
%% Maths environment
%%%%%%%%%%%%%%%%%%%%%%%%%%

%\newtheorem{theorem}{Theorem}[section]
%\newtheorem{lemma}[theorem]{Lemma}
%\newtheorem{proposition}[theorem]{Proposition}
%\newtheorem{corollary}[theorem]{Corollary}

%\newenvironment{proof}[1][Proof]{\begin{trivlist}
%\item[\hskip \labelsep {\bfseries #1}]}{\end{trivlist}}
%\newenvironment{definition}[1][Definition]{\begin{trivlist}
%\item[\hskip \labelsep {\bfseries #1}]}{\end{trivlist}}
%\newenvironment{example}[1][Example]{\begin{trivlist}
%\item[\hskip \labelsep {\bfseries #1}]}{\end{trivlist}}
%\newenvironment{remark}[1][Remark]{\begin{trivlist}
%\item[\hskip \labelsep {\bfseries #1}]}{\end{trivlist}}

%\newcommand{\qed}{\nobreak \ifvmode \relax \else
%      \ifdim\lastskip<1.5em \hskip-\lastskip
%      \hskip1.5em plus0em minus0.5em \fi \nobreak
%      \vrule height0.75em width0.5em depth0.25em\fi}



%% Commands

\newcommand{\noun}[1]{\textsc{#1}}


%% Math

% Operators
\DeclareMathOperator{\Cov}{Cov}
\DeclareMathOperator{\Var}{Var}
\DeclareMathOperator{\E}{\mathbb{E}}
\DeclareMathOperator{\Proba}{\mathbb{P}}

\newcommand{\Covb}[2]{\ensuremath{\Cov\!\left[#1,#2\right]}}
\newcommand{\Eb}[1]{\ensuremath{\E\!\left[#1\right]}}
\newcommand{\Pb}[1]{\ensuremath{\Proba\!\left[#1\right]}}
\newcommand{\Varb}[1]{\ensuremath{\Var\!\left[#1\right]}}

% norm
\newcommand{\norm}[1]{\left\lVert #1 \right\rVert}



% argmin
\DeclareMathOperator*{\argmin}{\arg\!\min}


% amsthm environments
\newtheorem{definition}{Definition}
\newtheorem{proposition}{Proposition}
\newtheorem{assumption}{Assumption}

%% graphics

% renew graphics command for relative path providment only ?
%\renewcommand{\includegraphics[]{}}


\usepackage{url}





% geometry
\usepackage[margin=2cm]{geometry}



% changes

\usepackage{soul}
\soulregister\cite7
\soulregister\citep7
\soulregister\ref7

\usepackage[final]{changes}
%\usepackage{changes}


\setaddedmarkup{\textcolor{black}{\hl{#1}}}
\setdeletedmarkup{\textcolor{red}{\sout{#1}}}



\usepackage{CJKutf8}
%\begin{CJK*}{UTF8}{zhsong}
%文章内容。
%\clearpage\end{CJK*}
\newcommand{\cn}[1]{
  \begin{CJK*}{UTF8}{gbsn}
  #1
  \end{CJK*}
}



% layout : use fancyhdr package
%\usepackage{fancyhdr}
%\pagestyle{fancy}
%
%\makeatletter
%
%\renewcommand{\headrulewidth}{0.4pt}
%\renewcommand{\footrulewidth}{0.4pt}
%\fancyhead[RO,RE]{}
%\fancyhead[LO,LE]{Models for the co-evolution of cities and networks}
%\fancyfoot[RO,RE] {\thepage}
%\fancyfoot[LO,LE] {}
%\fancyfoot[CO,CE] {}
%
%\makeatother
%

%%%%%%%%%%%%%%%%%%%%%
%% Begin doc
%%%%%%%%%%%%%%%%%%%%%

\begin{document}






% A macro-scale model of co-evolution for cities and transportation networks

%MEDIUM 2017 Conference Guangzhou, 17-18 June 2017
%Session Spatio-temporal Behavior in Complex Urban Systems


\title{城市和交通网络共同演化的宏观模型
\\\bigskip
\bigskip
\bigskip
\textit{MEDIUM 国际会议,广州, 2017年06月17-18日}\\
\textit{}
}\bigskip
\bigskip
\author{\noun{Juste Raimbault}$^{1,2}$\\
\small(1) UMR CNRS 8504 Géographie-cités 和 (2) UMR-T IFSTTAR 9403 LVMT
}
%\date{15 octobre 2015}
\date{}

\maketitle

\justify

\pagenumbering{gobble}



\vspace{0.2cm}

%\textbf{关键词 : }\textit{耦合演进网络化领土系统 ; 基于代理的建模 ; 运输治理 ; 珠江三角洲}

%\vspace{0.5cm}

%The complexity of Urban Systems is closely linked to the co-evolutive character of their different components or agents (Pumain, 1997). 城市系统的复杂性与其不同组成部分或代理人的共同演化特征密切相关 \cite{pumain1997pour}

%In the case of cities and transportation networks, this co-evolution has been shown empirically (Bretagnolle, 2007) but remains poorly understood in terms of its dynamical processes. 在城市和交通网络的情况下,这种共同演化已经凭经验展示出来 \cite{bretagnolle2009villes},但在动力学过程方面仍然知之甚少。

%We introduce a model of spatial interactions between cities at the macro-scale, in the spirit of stochastic urban growth models inheriting from the Gibrat model (Favaro and Pumain, 2011).  我们在宏观尺度上引入了城市间空间相互作用的模型,并以Gibrat模型为继承的随机城市增长模型的精神  \cite{favaro2011gibrat}
 % We include evolving transportation networks, in order to explore stylized hypothesis on the interactions and drivers of the growth of both network and cities. 我们包括不断发展的交通网络,以探索关于网络和城市发展的相互作用和驱动因素的程式化假设。
 %In a multi-modeling fashion, the model can take into account various processes such as between cities direct interactions, network-mediated interactions, feedback of network flows, and for the network demand-induced growth. 在多建模方式中,该模型可以考虑各种过程,例如城市之间的直接交互,网络介导的交互,网络流的反馈以及网络需求诱导的增长。

 % The latter is tested at different abstraction levels that are the time-distance matrix between cities, and physical network growth trying to satisfy greedy time-gain optimization criteria. 后者在不同的抽象级别进行测试,这些抽象级别是城市之间的时间 - 距离矩阵,以及物理网络增长试图满足贪婪的时间增益优化标准。
 %We use as a benchmark network the geographical shortest paths that have been shown in a previous work to already capture network effects (Raimbault, 2016). 我们使用先前工作中已经显示的地理最短路径作为基准网络来捕获网络效应。

  %The model is tested and explored on synthetic city systems, generated following a simple heuristic to follow the rank-size law and Central Place Theory. 该模型在合成城市系统上进行测试和探索,按照简单的启发式生成,遵循等级大小定律和中心位置理论。
   
  %The systematic exploration through intensive computation unveils different interaction regimes across the parameter space. 通过密集计算的系统探索揭示了跨参数空间的不同交互机制。

  %In some, the introduction of the network can drastically change the fate of some cities, whereas the top-distribution hierarchy is reinforced, what is consistent with empirical observations in the literature. 在某些情况下,网络的引入可以极大地改变一些城市的命运,而顶层分布层次结构得到加强,这与文献中的经验观察是一致的。

     %Some regimes actually exhibit circular causalities between network and city growth, corresponding to the intricate co-evolution. 有些政权实际上表现出网络和城市增长之间的循环因果关系,这与复杂的共同进化相对应。
     %The model will be applied to the French Urban System on long time dynamical data with the Pumain-INED database for populations spanning between 1831 and 1999, with the evolving railway network from 1850 to 2000, and a specifically-designed database of the highway networks containing its full genesis from 1950 to 2015. It will also be applied to the Chinese Urban System after 2000 with the High Speed Rail (HSR) network, both realized and planned. 该模型将应用于法国城市系统的长期动态数据与Pumain-INED数据库,人口跨越1831年至1999年,随着1850年至2000年不断发展的铁路网络,以及一个专门设计的高速公路网络数据库 从1950年到2015年它的全部起源。它将在2000年之后应用于中国城市系统,其中包括高速铁路(HSR)网络,既实现又有计划。

      %  Expected results concern both accurate city population growth reproduction, and network patterns, i.e. how does taking into account dynamical networks can introduce further exploratory power in such models, and reciprocally how can such coupled models produce realistic networks compared to more classical autonomous models of network growth. 预期结果涉及准确的城市人口增长再现和网络模式,即如何考虑动态网络可以在这些模型中引入进一步的探索能力,相比之下,与更经典的网络增长自主模型相比,这种耦合模型如何能够产生真实的网络。
       %The role of medium-sized cities on the trajectories of the system can also be examined with the model. 中型城市在系统轨迹上的作用也可以用模型来检验。

       %Finally, a comparison between the urban systems in different geographical and political contexts and at different scales should unveil implications of planning on the interactions between networks and cities, for example by comparing the rather bottom-up growth of the French railway network to the top-down state-planned French highway and Chinese HSR networks. 最后,不同地理和政治背景下以及不同规模的城市系统之间的比较应该揭示规划对网络和城市之间相互作用的影响,例如通过比较法国铁路网络的自下而上的增长与 国家规划的法国高速公路和中国的高铁网络。



城市系统的复杂性与其不同组成部分或代理人的共同演化特征密切相关\cite{pumain1997pour}。在城市和交通网络的情况下,这种共同演化已经凭经验展示出来 \cite{bretagnolle2009villes},但在动力学过程方面仍然知之甚少。我们在宏观尺度上引入了城市间空间相互作用的模型,并以Gibrat模型为继承的随机城市增长模型的精神 \cite{favaro2011gibrat}。我们包括不断发展的交通网络,以探索关于网络和城市发展的相互作用和驱动因素的程式化假设。在多建模方式中,该模型可以考虑各种过程,例如城市之间的直接交互,网络介导的交互,网络流的反馈以及网络需求诱导的增长。后者在不同的抽象级别进行测试,这些抽象级别是城市之间的时间 - 距离矩阵,以及物理网络增长试图满足贪婪的时间增益优化标准。我们使用先前工作中已经显示的地理最短路径作为基准网络来捕获网络效应 \cite{raimbault:halshs-01370274}。该模型在合成城市系统上进行测试和探索,按照简单的启发式生成,遵循等级大小定律和中心位置理论。通过密集计算的系统探索揭示了跨参数空间的不同交互机制。在某些情况下,网络的引入可以极大地改变一些城市的命运,而顶层分布层次结构得到加强,这与文献中的经验观察是一致的。有些政权实际上表现出网络和城市增长之间的循环因果关系,这与复杂的共同进化相对应。该模型将应用于法国城市系统的长期动态数据与Pumain-INED数据库,人口跨越1831年至1999年,随着1850年至2000年不断发展的铁路网络,以及一个专门设计的高速公路网络数据库从1950年到2015年它的全部起源。它将在2000年之后应用于中国城市系统,其中包括高速铁路(HSR)网络,既实现又有计划。预期结果涉及准确的城市人口增长再现和网络模式,即如何考虑动态网络可以在这些模型中引入进一步的探索能力,相比之下,与更经典的网络增长自主模型相比,这种耦合模型如何能够产生真实的网络。中型城市在系统轨迹上的作用也可以用模型来检验。最后,不同地理和政治背景下以及不同规模的城市系统之间的比较应该揭示规划对网络和城市之间相互作用的影响,例如通过比较法国铁路网络的自下而上的增长与国家规划的法国高速公路和中国的高铁网络。









\bigskip
\bigskip

%%%%%%%%%%%%%%%%%%%%
%% Biblio
%%%%%%%%%%%%%%%%%%%%


%\begin{multicols}{2}

%\setstretch{0.3}
%\setlength{\parskip}{-0.4em}


\bibliographystyle{apalike}
\bibliography{biblio}

%\end{multicols}

\end{document}
