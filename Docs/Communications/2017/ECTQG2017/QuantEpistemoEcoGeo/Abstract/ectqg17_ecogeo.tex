\documentclass[11pt]{article}

% general packages without options
\usepackage{amsmath,amssymb,bbm}
% graphics
\usepackage{graphicx}
% text formatting
\usepackage[document]{ragged2e}
\usepackage{pagecolor,color}

\newcommand{\noun}[1]{\textsc{#1}}

\usepackage[utf8]{inputenc}
\usepackage[T1]{fontenc}
% geometry
\usepackage[margin=2cm]{geometry}

\usepackage{multicol}
\usepackage{setspace}

\usepackage{natbib}
\setlength{\bibsep}{0.0pt}

\usepackage[french]{babel}

% layout : use fancyhdr package
%\usepackage{fancyhdr}
%\pagestyle{fancy}

% variable to include comments or not in the compilation ; set to 1 to include
\def \draft {1}


% writing utilities

% comments and responses
%  -> use this comment to ask questions on what other wrote/answer questions with optional arguments (up to 4 answers)
\usepackage{xparse}
\usepackage{ifthen}
\DeclareDocumentCommand{\comment}{m o o o o}
{\ifthenelse{\draft=1}{
    \textcolor{red}{\textbf{C : }#1}
    \IfValueT{#2}{\textcolor{blue}{\textbf{A1 : }#2}}
    \IfValueT{#3}{\textcolor{ForestGreen}{\textbf{A2 : }#3}}
    \IfValueT{#4}{\textcolor{red!50!blue}{\textbf{A3 : }#4}}
    \IfValueT{#5}{\textcolor{Aquamarine}{\textbf{A4 : }#5}}
 }{}
}

% todo
\newcommand{\todo}[1]{
\ifthenelse{\draft=1}{\textcolor{red!50!blue}{\textbf{TODO : \textit{#1}}}}{}
}


\makeatletter


\makeatother


\begin{document}







\title{Forever Torn Apart ? Scientific landscapes around similar objects studied from Economics and Geography perspectives
\bigskip\bigskip\\
\textit{Communication Proposal - ECTQG 2017}
}
\author{\noun{Juste Raimbault}$^{1,2}$\medskip\\
$^1$ UMR CNRS 8504 Géographie-cités\\
$^2$ UMR-T IFSTTAR 9403 LVMT
}
\date{}

\maketitle

\justify

\pagenumbering{gobble}


\textbf{Keywords : }\textit{Quantitative Epistemology ; Citation Network ; Text Mining ; Bridges between Economics and Geography}

\medskip


Understanding Science in a perspectivist approach~\cite{giere2010scientific}, it is natural and necessary that disciplines or fields propose very different \emph{perspectives} on real world objects. The yet-to-explore border regions at the interface, in which interdisciplinarity draws most of its strengths, are however not well understood in terms of processes of knowledge production such as domains cross-fertilisation, but also not necessarily the object of consensuses for research policies. We propose to explore these issues on the particular case of Economics and Geography, between which bridges seem difficult to build in the current state of disciplines. We take a Quantitative Epistemology approach, more precisely by combining citation network analysis with text-mining and semantic network analysis, using methods and tools developed in~\cite{raimbault2016indirect} to reconstruct what can be seen as a \emph{scientific landscape}. As these are quite sensitive to context and choices to be made in exogenous delimitation of studied corpus, and as generality is not the best way to tackle epistemological questions, we choose to work on two case studies of objects that have been extensively studied from both perspectives: Relations between Network and Territories, and Urban Growth. We constitute for each an initial corpus of key references in both disciplines, from which the backward citation network at depth two is reconstructed. We then collect abstracts for a significant proportion of nodes, extract relevant keywords, and couple the citation network with a semantic network which communities for example define endogenous research domains.












%%%%%%%%%%%%%%%%%%%%
%% Biblio
%%%%%%%%%%%%%%%%%%%%
%\tiny

%\begin{multicols}{2}

%\setstretch{0.3}
%\setlength{\parskip}{-0.4em}


\bibliographystyle{apalike}
\bibliography{/Users/Juste/Documents/ComplexSystems/CityNetwork/Biblio/Bibtex/CityNetwork}%,biblio}
%\end{multicols}



\end{document}

