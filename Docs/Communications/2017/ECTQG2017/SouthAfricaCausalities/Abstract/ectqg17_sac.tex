\documentclass[11pt]{article}

% general packages without options
\usepackage{amsmath,amssymb,bbm}
% graphics
\usepackage{graphicx}
% text formatting
\usepackage[document]{ragged2e}
\usepackage{pagecolor,color}

\newcommand{\noun}[1]{\textsc{#1}}

\usepackage[utf8]{inputenc}
\usepackage[T1]{fontenc}
% geometry
\usepackage[margin=2cm]{geometry}

\usepackage{multicol}
\usepackage{setspace}

\usepackage{natbib}
\setlength{\bibsep}{0.0pt}

%\usepackage[french]{babel}

% layout : use fancyhdr package
%\usepackage{fancyhdr}
%\pagestyle{fancy}

% variable to include comments or not in the compilation ; set to 1 to include
\def \draft {1}


% writing utilities

% comments and responses
%  -> use this comment to ask questions on what other wrote/answer questions with optional arguments (up to 4 answers)
\usepackage{xparse}
\usepackage{ifthen}
\DeclareDocumentCommand{\comment}{m o o o o}
{\ifthenelse{\draft=1}{
    \textcolor{red}{\textbf{C : }#1}
    \IfValueT{#2}{\textcolor{blue}{\textbf{A1 : }#2}}
    \IfValueT{#3}{\textcolor{ForestGreen}{\textbf{A2 : }#3}}
    \IfValueT{#4}{\textcolor{red!50!blue}{\textbf{A3 : }#4}}
    \IfValueT{#5}{\textcolor{Aquamarine}{\textbf{A4 : }#5}}
 }{}
}

% todo
\newcommand{\todo}[1]{
\ifthenelse{\draft=1}{\textcolor{red!50!blue}{\textbf{TODO : \textit{#1}}}}{}
}


\makeatletter


\makeatother


\begin{document}







\title{\vspace{-2.5cm}Structural Segregation: Assessing the impact of South African Apartheid on Underlying Dynamics of Interactions between Networks and Territories
\\
\textit{Communication Proposal - ECTQG 2017}
}
\author{\noun{Juste Raimbault}$^{1,2}$% and \noun{Sol{\`e}ne Baffi}$^{1,3}$
\\
$^1$ UMR CNRS 8504 Géographie-cités\\
$^2$ UMR-T IFSTTAR 9403 LVMT%\\
%$^3$ Solene's lab
}
\date{}

\maketitle

\justify

\pagenumbering{gobble}


\textbf{Keywords : }\textit{South Africa ; Network-Territories Interactions ; Railway Network ; Spatio-temporal Causality}

\medskip

Transportation Networks can be leveraged as a powerful socio-economic control tool, with even more significant outcomes when it percolates to their interaction with territories. The case of South Africa is an accurate illustration, as \cite{baffi:tel-01389347} shows that during apartheid railway network planning was used as a racial segregation tool by shaping strongly constrained mobility and accessibility patterns. We propose to investigate the potential \emph{structural} properties of this historical process, by focusing on dynamical patterns of interactions between the railway network and city growth. More precisely, we try to establish if the segregative planning policies did actually modify the trajectory of the coupled system, what would correspond to deeper and wider impacts than the direct ones. We use a comprehensive database that includes the full South African railway network from 1880 to 2000 with opening and closing dates for each station and link, together with a city database spanning from 1911 to 1991 for which consistent ontologies for urban areas have been ensured.%~(Citation database)\cite{}. %TODO cit. database ?
 First, a dynamical study of network measures seem to confirm the hypothesis: a trend rupture in closeness centrality at a roughly constant network size evolution, at a date corresponding to the beginning of official segregative policies, suggests that the planning process after this date had in the best case no global effect on network performance, and in the worst case had intended negative effects on accessibility with the aim to physically segregate more. We then turn to dynamical interactions between the railway network and city growth. For that, we study Granger causalities estimated between cities growth rates and accessibility differentials due to network growth, for all cities and urban areas having a connection to the network. We test both travel-time and population weighted accessibilities, for varying values of distance decay parameter. Lagged correlations are fitted on varying length time windows, to test for potentially varying stationarity scales. We find that results are significant with travel-time accessibility only, autocorrelation dominating with weighted accessibility. A time-window of 30 years appears to be a good compromise between the number of significant correlations ($p<0.1$ for a Fisher test) and the absolute correlation level across all lags and distance decays, what should correspond roughly to the time-stationarity scale of the system.% We observe furthermore a phase transition when distance decay increases, revealing the shift between the spatial scale of urban areas and the scale of the country, what gives local spatial stationarity scale.
  With these settings, we obtain clear causality patterns, namely an inversion of the Granger causality (lagged correlation up to 0.5 for several values of distance decay), from accessibility causing population growth with a lag of 10-20 years before the apartheid (1948), to population growth causing accessibility patterns after the apartheid (lag 20 years). We interpret these as \emph{Structural segregation}, i.e. a significant impact of planning policies on dynamics of interactions between networks and territories. Indeed, the first regime corresponds to direct effect of transportation on migrations in a free context in opposition to the second one. Further work should consist in similar study with more precise socio-economic variables, for example quantifying directly segregation patterns.






%%%%%%%%%%%%%%%%%%%%
%% Biblio
%%%%%%%%%%%%%%%%%%%%
%\tiny

%\begin{multicols}{2}

%\setstretch{0.3}
%\setlength{\parskip}{-0.4em}


\bibliographystyle{apalike}
\bibliography{/Users/Juste/Documents/ComplexSystems/CityNetwork/Biblio/Bibtex/CityNetwork}%,biblio}
%\end{multicols}



\end{document}
