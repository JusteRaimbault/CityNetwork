\documentclass[english,11pt]{beamer}

\DeclareMathOperator{\Cov}{Cov}
\DeclareMathOperator{\Var}{Var}
\DeclareMathOperator{\E}{\mathbb{E}}
\DeclareMathOperator{\Proba}{\mathbb{P}}

\newcommand{\Covb}[2]{\ensuremath{\Cov\!\left[#1,#2\right]}}
\newcommand{\Eb}[1]{\ensuremath{\E\!\left[#1\right]}}
\newcommand{\Pb}[1]{\ensuremath{\Proba\!\left[#1\right]}}
\newcommand{\Varb}[1]{\ensuremath{\Var\!\left[#1\right]}}

% norm
\newcommand{\norm}[1]{\| #1 \|}

\newcommand{\indep}{\rotatebox[origin=c]{90}{$\models$}}





\usepackage{mathptmx,amsmath,amssymb,graphicx,bibentry,bbm,babel,ragged2e}

\makeatletter

\newcommand{\noun}[1]{\textsc{#1}}
\newcommand{\jitem}[1]{\item \begin{justify} #1 \end{justify} \vfill{}}
\newcommand{\sframe}[2]{\frame{\frametitle{#1} #2}}

\newenvironment{centercolumns}{\begin{columns}[c]}{\end{columns}}
%\newenvironment{jitem}{\begin{justify}\begin{itemize}}{\end{itemize}\end{justify}}

\usetheme{Warsaw}
\setbeamertemplate{footline}[text line]{}
\setbeamercolor{structure}{fg=purple!50!blue, bg=purple!50!blue}

\setbeamersize{text margin left=15pt,text margin right=15pt}

\setbeamercovered{transparent}


\@ifundefined{showcaptionsetup}{}{%
 \PassOptionsToPackage{caption=false}{subfig}}
\usepackage{subfig}

\usepackage[utf8]{inputenc}
\usepackage[T1]{fontenc}



\makeatother

\begin{document}


\title{Modeling the Co-evolution of Urban Form and Transportation Networks}

\author{J.~Raimbault$^{1,2,\ast}$
\texttt{juste.raimbault@polytechnique.edu}
}


\institute{$^{1}$UMR CNRS 8504 G{\'e}ographie-cit{\'e}s\\
$^{2}$UMR-T IFSTTAR 9403 LVMT
}


\date{CCS 2017 - Cancun\\\smallskip
Session 3D: Infrastructure, Planning and Environment\\\smallskip
September 18th 2017
}

\frame{\maketitle}





%%%%%%%%%%%%%%%%%%%
%% ABSTRACT
%\textbf{Keywords : }\textit{Urban and Network Morphogenesis Modeling ; Interactions between Networks and Territories ; Co-evolution}





%%%%%%%%%%%%%%%%%
\section{Introduction}
%%%%%%%%%%%%%%%%%



\sframe{Urban Morphogenesis}{

\centering

% striking image

}


\sframe{Modeling Urban Morphogenesis}{

% Urban settlements and transportation networks are widely admitted to be co-evolving in the thematic and empirical studies of territorial systems. However, modeling approaches of such dynamical interactions between networks and territories are less developed. We propose to study this issue at an intermediate scale, focusing on morphological and functional properties of the territorial system in a stylized way.

\justify


\bigskip

\textbf{Research Objective : } 


}



%%%%%%%%%%%%%%%%%
\section{Methods and Results}
%%%%%%%%%%%%%%%%%


\sframe{Model : Rationale}{

% % We introduce a stochastic dynamical model of urban morphogenesis that couples the evolution of population density within grid cells with a growing road network. 


}



\sframe{Model : Specification}{

% With an overall fixed growth rate, new population aggregate preferentially to a potential for which parameters control the dependance to various explicative variables, namely local density, distance to the network, centrality measures within the network and generalized accessibility. A continuous diffusion of population completes the aggregation to translate repulsion processes generally due to congestion. Because of the different time scales of evolution for urban scape and networks, the network grows at fixed time steps, with rules that can be switched in a multi-modeling fashion. A fixed rule ensure connectivity of newly populated patches to the existing network. Two different heuristics are then compared: one based on gravity potential breakdown for which links are created if a generalized interaction potential through a new candidate link exceeds a certain times the potential within the existing network; a second one implementing biological network growth, more precisely a slime mould model.

% Both are complementary since the gravity model would be more typical of planned top-down network evolution, whereas the biological model will translate bottom-up processes of network growth.


}

\sframe{Network Generation}{

% detail of bio heuristic ?
%example-bio-process-0.png

}

\sframe{Generated Urban Shapes}{

% example of coupled configs for different heuristics / parameters

% %\caption{(Left) Example of configuration generated with the mandatory network rule only; (Middle) Intermediate step of the slime mould link generation process, in which self-reinforcement process appears crucial to select among potential new links; (Right) Examples of four different causality regimes, shown here through values of lagged correlation between couples among three variables (density, distance to network, centrality) as a function of time lag.}


}


\sframe{Calibration Method}{


% The model is calibrated at the first order (indicators of urban form and network measures) and at the second order (correlations) with Eurostat population grid coupled with street network from OpenStreetMap through the following workflow: indicators (Moran index, mean distance, hierarchy, entropy for morphology, mean path length, centralities, performance for network) are computed on real areas of width 50km for all Europe (what corresponds to the typical scale of processes the model includes); parameter space of the model is explored using grid computing (with OpenMole model exploration software), from simple synthetic initial configurations (few connected punctual settlements), computing indicators on final simulated configurations;  among candidate parameters for given contiguous (in space and indicator space) real areas on which correlations can be computed, the one with the closest correlation matrix computed on repetitions is chosen.

% -> indicators in appendix

}


\sframe{Results : Network Heuristics}{

% results with static density

% quote density-generation to explain morpho classes
% -> find composition ?

%distance_real_bymorph.png
%feasible_space_withreal_pca_bymorph.png


}


\sframe{Results : Calibration}{

% We obtain a full coverage of real configurations with simulation results in a principal component plan for indicators, for which most of them a close correlation structure is found. Both network heuristics are necessary for the full coverage. The model is thus able to reproduce existing urban form and networks, but also their \emph{interaction} in the sense of correlations.

% nw coverage, density coverage and correlations coverage.

%pca_allobjs.png
%pca_morpho_byheuristic.png
%pca_network_byheuristic.png


%%%
% correlation distances histogram 

% -> TODO add a null model ? (random indics/corrs)

}


\sframe{Results : Causality Regimes}{

% We furthermore study dynamical lagged correlations between normalized returns of population and network patch explicatives variables, exhibiting a large diversity of spatio-temporal causality regimes, where network can drive urban growth, the contrary, or more complex circular causalities, suggesting that the model effectively grasps the dynamical richness of interactions.




}



%%%%%%%%%%%%%%%%%
\section{Discussion}
%%%%%%%%%%%%%%%%%





\sframe{Discussion}{

\justify

\textbf{Implications}

$\rightarrow$
\bigskip

\textbf{Developments}

% NOTE : not fair between heuristics, not same number of params - but open question.

$\rightarrow$ 

}




\sframe{Conclusion}{

\justify

$\rightarrow$ 

\bigskip
\bigskip
\bigskip

\footnotesize{ - Code et data available at\\ \texttt{https://github.com/JusteRaimbault/CityNetwork}

}

}






\sframe{Reserve slides}{

\centering

\Large

\textbf{Reserve Slides}

}

\sframe{Defining co-evolution}{

% geographical theory

\justify

%No clear definition of co-evolution in the literature : \cite{bretagnolle:tel-00459720} distinguishes ``reciprocal adaptation'' where a sense of causality can clearly be identified, from co-evolutive regimes 


%\cite{raimbault2017identification} identifies multiple causality regimes in a simple strongly coupled growth model $\rightarrow$ to be put in perspective with a theoretical definition of co-evolution based on the conjunction of Morphogenesis and the Evolutive Urban Theory, summarised by~\cite{raimbault2017co}

}


\sframe{Data Processing}{


}


\sframe{Morphological Indicators}{

}


\sframe{Network Indicators}{

}


\sframe{Model specification}{

% additional equations for submodels

}

\sframe{Model parameters}{

% summary of parameters ; range


}


\sframe{Calibration : optimal points}{

dists_pareto_i1
dists_pareto_i10

}



%%%%%%%%%%%%%%%%%%%%%
\begin{frame}[allowframebreaks]
\frametitle{References}
\bibliographystyle{apalike}
\bibliography{/Users/juste/ComplexSystems/CityNetwork/Biblio/Bibtex/CityNetwork,biblio}
\end{frame}
%%%%%%%%%%%%%%%%%%%%%%%%%%%%









\end{document}







