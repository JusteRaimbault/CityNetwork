%%%%%%%%%%%%%%%%%%%%%%%%%%%%%
% Standard header for working papers
%
% WPHeader.tex
%
%%%%%%%%%%%%%%%%%%%%%%%%%%%%%

\documentclass[11pt]{article}



%%%%%%%%%%%%%%%%%%%%%%%%%%
%% TEMPLATES
%%%%%%%%%%%%%%%%%%%%%%%%%%


% Simple Tabular

%\begin{tabular}{ |c|c|c| } 
% \hline
% cell1 & cell2 & cell3 \\ 
% cell4 & cell5 & cell6 \\ 
% cell7 & cell8 & cell9 \\ 
% \hline
%\end{tabular}





%%%%%%%%%%%%%%%%%%%%%%%%%%
%% Packages
%%%%%%%%%%%%%%%%%%%%%%%%%%



% encoding 
\usepackage[utf8]{inputenc}
\usepackage[T1]{fontenc}


% general packages without options
\usepackage{amsmath,amssymb,amsthm,bbm}

% graphics
\usepackage{graphicx,transparent,eso-pic}

% text formatting
\usepackage[document]{ragged2e}
\usepackage{pagecolor,color}
%\usepackage{ulem}
\usepackage{soul}


% conditions
\usepackage{ifthen}


\usepackage{natbib}


%%%%%%%%%%%%%%%%%%%%%%%%%%
%% Maths environment
%%%%%%%%%%%%%%%%%%%%%%%%%%

%\newtheorem{theorem}{Theorem}[section]
%\newtheorem{lemma}[theorem]{Lemma}
%\newtheorem{proposition}[theorem]{Proposition}
%\newtheorem{corollary}[theorem]{Corollary}

%\newenvironment{proof}[1][Proof]{\begin{trivlist}
%\item[\hskip \labelsep {\bfseries #1}]}{\end{trivlist}}
%\newenvironment{definition}[1][Definition]{\begin{trivlist}
%\item[\hskip \labelsep {\bfseries #1}]}{\end{trivlist}}
%\newenvironment{example}[1][Example]{\begin{trivlist}
%\item[\hskip \labelsep {\bfseries #1}]}{\end{trivlist}}
%\newenvironment{remark}[1][Remark]{\begin{trivlist}
%\item[\hskip \labelsep {\bfseries #1}]}{\end{trivlist}}

%\newcommand{\qed}{\nobreak \ifvmode \relax \else
%      \ifdim\lastskip<1.5em \hskip-\lastskip
%      \hskip1.5em plus0em minus0.5em \fi \nobreak
%      \vrule height0.75em width0.5em depth0.25em\fi}



%% Commands

\newcommand{\noun}[1]{\textsc{#1}}


%% Math

% Operators
\DeclareMathOperator{\Cov}{Cov}
\DeclareMathOperator{\Var}{Var}
\DeclareMathOperator{\E}{\mathbb{E}}
\DeclareMathOperator{\Proba}{\mathbb{P}}

\newcommand{\Covb}[2]{\ensuremath{\Cov\!\left[#1,#2\right]}}
\newcommand{\Eb}[1]{\ensuremath{\E\!\left[#1\right]}}
\newcommand{\Pb}[1]{\ensuremath{\Proba\!\left[#1\right]}}
\newcommand{\Varb}[1]{\ensuremath{\Var\!\left[#1\right]}}

% norm
\newcommand{\norm}[1]{\left\lVert #1 \right\rVert}



% argmin
\DeclareMathOperator*{\argmin}{\arg\!\min}


% amsthm environments
\newtheorem{definition}{Definition}
\newtheorem{proposition}{Proposition}
\newtheorem{assumption}{Assumption}

%% graphics

% renew graphics command for relative path providment only ?
%\renewcommand{\includegraphics[]{}}


\usepackage{url}





% geometry
\usepackage[margin=2cm]{geometry}



% changes

\usepackage{soul}
\soulregister\cite7
\soulregister\citep7
\soulregister\ref7

\usepackage[final]{changes}
%\usepackage{changes}


\setaddedmarkup{\textcolor{black}{\hl{#1}}}
\setdeletedmarkup{\textcolor{red}{\sout{#1}}}



\usepackage{CJKutf8}
%\begin{CJK*}{UTF8}{zhsong}
%文章内容。
%\clearpage\end{CJK*}
\newcommand{\cn}[1]{
  \begin{CJK*}{UTF8}{gbsn}
  #1
  \end{CJK*}
}



% layout : use fancyhdr package
%\usepackage{fancyhdr}
%\pagestyle{fancy}
%
%\makeatletter
%
%\renewcommand{\headrulewidth}{0.4pt}
%\renewcommand{\footrulewidth}{0.4pt}
%\fancyhead[RO,RE]{}
%\fancyhead[LO,LE]{Models for the co-evolution of cities and networks}
%\fancyfoot[RO,RE] {\thepage}
%\fancyfoot[LO,LE] {}
%\fancyfoot[CO,CE] {}
%
%\makeatother
%

%%%%%%%%%%%%%%%%%%%%%
%% Begin doc
%%%%%%%%%%%%%%%%%%%%%

\begin{document}







\title{Towards a Theory of Co-evolutive Networked Territorial Systems: Insights from Transportation Governance Modeling in Pearl River Delta, China\\\bigskip
\bigskip
\bigskip
\textit{Medium Project Seminar, December 2016}
}\bigskip
\bigskip
\author{\noun{Juste Raimbault}$^{1,2}$\\
\small(1) UMR CNRS 8504 Géographie-cités and (2) UMR-T IFSTTAR 9403 LVMT
}
%\date{15 octobre 2015}
\date{}

\maketitle

\justify

\pagenumbering{gobble}



\vspace{0.2cm}

\textbf{Keywords : }\textit{Co-evolutive Networked Territorial Systems ; Agent-based Modeling ; Transportation Governance ; Pearl River Delta Mega-city Region}

\vspace{0.5cm}

Recent advances in Theoretical and Quantitative Geography have witnessed the co-construction of geographical theories, complex models of simulation and Empirical analysis. We build on this legacy to shed light on relations between territories and networks. Our contribution consists in two distincts parts. First we develop a new theory of territorial systems, that emphasizes on the role of networks in co-evolutive processes. More precisely, we speculate, building on several modeling and empirical previous contributions, that within Pumain's Evolutive Urban Theory~\cite{pumain1997pour}, the existence of co-evolutive subsystems is equivalent to the existence of morphogenetic processes in which networks are significant drivers. Implications include the necessity of networks in explaining territorial systems dynamics, but also a modular decomposition of these systems into local stationary processes in space or time.

In a second part, we discuss practical implications of the theory through the adaptation of a agent-based model introduced in~\cite{le2015modeling}, which simulates the co-evolution of land-use and transportation infrastructure within a Mega-city Region. In particular, it includes game theoretical modeling of decision making processes for transportation governance. The model is partially validated on synthetic data by retrieving expected stylized facts on emergent infrastructure. We propose then to apply it to the Mega-city Region of Pearl River Delta, China, which is very typical of included characteristics such as the presence of a strong competition between inner cities and the recent realization, construction or planning of numerous large-scale transportation infrastructures. Calibration of the model will allow first to infer information on potentially hidden governance processes when applied on real or planned infrastructure, and secondly to compare among different possible governance frameworks when applied on hypothetical optimized infrastructure.



\bigskip
\bigskip

%%%%%%%%%%%%%%%%%%%%
%% Biblio
%%%%%%%%%%%%%%%%%%%%


%\begin{multicols}{2}

%\setstretch{0.3}
%\setlength{\parskip}{-0.4em}


\bibliographystyle{apalike}
\bibliography{biblio}

%\end{multicols}

\end{document}
