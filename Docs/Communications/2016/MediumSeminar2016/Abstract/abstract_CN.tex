%%%%%%%%%%%%%%%%%%%%%%%%%%%%%
% Standard header for working papers
%
% WPHeader.tex
%
%%%%%%%%%%%%%%%%%%%%%%%%%%%%%

\documentclass[11pt]{article}



%%%%%%%%%%%%%%%%%%%%%%%%%%
%% TEMPLATES
%%%%%%%%%%%%%%%%%%%%%%%%%%


% Simple Tabular

%\begin{tabular}{ |c|c|c| } 
% \hline
% cell1 & cell2 & cell3 \\ 
% cell4 & cell5 & cell6 \\ 
% cell7 & cell8 & cell9 \\ 
% \hline
%\end{tabular}





%%%%%%%%%%%%%%%%%%%%%%%%%%
%% Packages
%%%%%%%%%%%%%%%%%%%%%%%%%%



% encoding 
\usepackage[utf8]{inputenc}
\usepackage[T1]{fontenc}


% general packages without options
\usepackage{amsmath,amssymb,amsthm,bbm}

% graphics
\usepackage{graphicx,transparent,eso-pic}

% text formatting
\usepackage[document]{ragged2e}
\usepackage{pagecolor,color}
%\usepackage{ulem}
\usepackage{soul}


% conditions
\usepackage{ifthen}


\usepackage{natbib}


%%%%%%%%%%%%%%%%%%%%%%%%%%
%% Maths environment
%%%%%%%%%%%%%%%%%%%%%%%%%%

%\newtheorem{theorem}{Theorem}[section]
%\newtheorem{lemma}[theorem]{Lemma}
%\newtheorem{proposition}[theorem]{Proposition}
%\newtheorem{corollary}[theorem]{Corollary}

%\newenvironment{proof}[1][Proof]{\begin{trivlist}
%\item[\hskip \labelsep {\bfseries #1}]}{\end{trivlist}}
%\newenvironment{definition}[1][Definition]{\begin{trivlist}
%\item[\hskip \labelsep {\bfseries #1}]}{\end{trivlist}}
%\newenvironment{example}[1][Example]{\begin{trivlist}
%\item[\hskip \labelsep {\bfseries #1}]}{\end{trivlist}}
%\newenvironment{remark}[1][Remark]{\begin{trivlist}
%\item[\hskip \labelsep {\bfseries #1}]}{\end{trivlist}}

%\newcommand{\qed}{\nobreak \ifvmode \relax \else
%      \ifdim\lastskip<1.5em \hskip-\lastskip
%      \hskip1.5em plus0em minus0.5em \fi \nobreak
%      \vrule height0.75em width0.5em depth0.25em\fi}



%% Commands

\newcommand{\noun}[1]{\textsc{#1}}


%% Math

% Operators
\DeclareMathOperator{\Cov}{Cov}
\DeclareMathOperator{\Var}{Var}
\DeclareMathOperator{\E}{\mathbb{E}}
\DeclareMathOperator{\Proba}{\mathbb{P}}

\newcommand{\Covb}[2]{\ensuremath{\Cov\!\left[#1,#2\right]}}
\newcommand{\Eb}[1]{\ensuremath{\E\!\left[#1\right]}}
\newcommand{\Pb}[1]{\ensuremath{\Proba\!\left[#1\right]}}
\newcommand{\Varb}[1]{\ensuremath{\Var\!\left[#1\right]}}

% norm
\newcommand{\norm}[1]{\left\lVert #1 \right\rVert}



% argmin
\DeclareMathOperator*{\argmin}{\arg\!\min}


% amsthm environments
\newtheorem{definition}{Definition}
\newtheorem{proposition}{Proposition}
\newtheorem{assumption}{Assumption}

%% graphics

% renew graphics command for relative path providment only ?
%\renewcommand{\includegraphics[]{}}


\usepackage{url}





% geometry
\usepackage[margin=2cm]{geometry}



% changes

\usepackage{soul}
\soulregister\cite7
\soulregister\citep7
\soulregister\ref7

\usepackage[final]{changes}
%\usepackage{changes}


\setaddedmarkup{\textcolor{black}{\hl{#1}}}
\setdeletedmarkup{\textcolor{red}{\sout{#1}}}



\usepackage{CJKutf8}
%\begin{CJK*}{UTF8}{zhsong}
%文章内容。
%\clearpage\end{CJK*}
\newcommand{\cn}[1]{
  \begin{CJK*}{UTF8}{gbsn}
  #1
  \end{CJK*}
}



% layout : use fancyhdr package
%\usepackage{fancyhdr}
%\pagestyle{fancy}
%
%\makeatletter
%
%\renewcommand{\headrulewidth}{0.4pt}
%\renewcommand{\footrulewidth}{0.4pt}
%\fancyhead[RO,RE]{}
%\fancyhead[LO,LE]{Models for the co-evolution of cities and networks}
%\fancyfoot[RO,RE] {\thepage}
%\fancyfoot[LO,LE] {}
%\fancyfoot[CO,CE] {}
%
%\makeatother
%

%%%%%%%%%%%%%%%%%%%%%
%% Begin doc
%%%%%%%%%%%%%%%%%%%%%

\begin{document}







\title{面向耦合进化的网络化领土系统理论:对中国珠江三角洲地区交通建模的见解
\\\bigskip
\bigskip
\bigskip
\textit{MEDIUM 项目研讨会,2016年12月}
}\bigskip
\bigskip
\author{\noun{Juste Raimbault}$^{1,2}$\\
\small(1) UMR CNRS 8504 Géographie-cités 和 (2) UMR-T IFSTTAR 9403 LVMT
}
%\date{15 octobre 2015}
\date{}

\maketitle

\justify

\pagenumbering{gobble}



\vspace{0.2cm}

\textbf{关键词 : }\textit{耦合演进网络化领土系统 ; 基于代理的建模 ; 运输治理 ; 珠江三角洲}

\vspace{0.5cm}

%
% Theoretical and Quantitative Geography has recently made progress : 理论和定量地理学最近取得了进展
% It includes simultaneously geographical theories, complex models of simulation and empirical analysis : 它同时包括地理理论,模拟的复杂模型和经验分析
% We build on this legacy to study relations between territories and networks 我们在这一遗产基础上研究领土和网络之间的关系
% Our contribution consists in two distinct parts 我们的贡献包括两个不同的部分
% First we develop a new theory of territorial systems 首先,我们开发一个新的领土系统理论
% It emphasizes on the role of networks in coupled evolution processes 它强调网络在耦合演化过程中的作用
% we build on several modeling and empirical previous contributions, in order to speculate the following hypothesis. 我们建立在几个建模和经验以前的贡献,以便推测以下假设。
%  within Pumain's Evolutive Urban Theory 在Pumain的演化城市理论
% the existence of subsystems evolving together is equivalent to the existence of morphogenetic processes in which networks are significant drivers  一起演化的子系统的存在等同于形态发生过程的存在,其中网络是重要的驱动因素
% Implications include the necessity of networks in explaining territorial systems dynamics but also a modular decomposition of these systems into local stationary processes in space or time. 影响包括网络在解释区域系统动力学方面的必要性,但也包括这些系统在空间或时间上的局部固定过程的模块化分解。
% In a second part, we discuss practical implications of the theory on a specific example. 在第二部分中,我们讨论该理论对具体示例的实际含义
% We simulate the coupled evolution of land-use and transportation infrastructure within a Mega-city Region 我们模拟了一个大城市地区土地利用和交通基础设施的耦合演化
% The model was introduced by \cite{le2015modeling}. 该模型由Le Nechet介绍。
% It includes in particular game theoretical modeling of decision making processes for transportation governance. 它特别包括运输治理决策过程的游戏理论模型。
% The model is partially validated on synthetic data by retrieving expected stylized facts on emergent infrastructure. 该模型通过在紧急基础设施上检索预期的风格化事实,对合成数据进行部分验证。
% It is applied to the Mega-city Region of Pearl River Delta. 适用于珠三角大城市地区。
% The case is representative of features included in the model, such as the presence of a strong competition between inner cities and the recent realization, construction and planning of numerous large-scale transportation infrastructures. 该案例代表了模型中包含的特征,例如内城市之间存在强烈的竞争,以及最近实现,建设和规划众多大型交通基础设施。
%  Calibration of the model will in a first time allow to infer information on potentially hidden governance processes when applied to real or planned infrastructures. 模型的校准将在第一时间允许推断关于当应用于实际或计划的基础设施时潜在隐藏的治理过程的信息。
%  It will allow in a second time to compare among different possible governance frameworks when applied on hypothetical optimized infrastructure.


近年来,理论地理与计量地理的最新进展见证了地理学理论、复杂模型以及经验分析的共同建构。
本文基于已有的研究,阐明了领土和网络之间的关系。
我们的贡献包括两个不同的部分。

首先,我们开发了一个新的领土系统理论,它强调网络在耦合演化过程中的作用。
更确切地说,在一些模型和以往实证研究的基础上,根据Pumain的演化城市理论(Pumain, 1997),我们推测共同演化子系统的存在等同于形态发生过程的存在,其中网络是重要的驱动因素。
该理论所产生的影响一方面在于解释了区域系统动力学方面的必要性,另一方面也包括这些系统在时间和空间上局部固定过程的模块化分解。

在第二部分中,本文讨论了该理论的实践意义。文章通过运用基于代理的模型 (Le Néchet and Raimbault, 2015) ,对珠三角城市群地区的土地利用和交通基础设施之间的耦合演进进行模拟。
需要特别说明的是,该模型包括了运输治理决策过程的游戏理论模型。
该模型通过在紧急基础设施上检索预期的典型化事实,对合成数据进行部分验证。
由此我们将其运用于珠江三角洲城市群地区,该地区具有模型中的典型特征,如城市内部存在激烈竞争以及近期完工、建设和规划的众多大型交通基础设施。
在模型的校准方面,首先,允许对当该模型应用于实际或规划的基础设施时所产生的潜在治理过程信息进行推断;其次,基于对假设优化基础设施的应用,在不同的治理框架下进行比较。




\bigskip
\bigskip

%%%%%%%%%%%%%%%%%%%%
%% Biblio
%%%%%%%%%%%%%%%%%%%%


%\begin{multicols}{2}

%\setstretch{0.3}
%\setlength{\parskip}{-0.4em}


\bibliographystyle{apalike}
\bibliography{biblio}

%\end{multicols}

\end{document}
