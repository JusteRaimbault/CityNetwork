\documentclass[11pt]{article}

\usepackage{amsmath,amssymb,bbm}

% graphics
\usepackage{graphicx}

% text formatting
\usepackage[document]{ragged2e}
\usepackage{pagecolor,color}



\usepackage[utf8]{inputenc}
\usepackage[T1]{fontenc}





% geometry
\usepackage[margin=1.5cm]{geometry}

\usepackage{multicol}
\usepackage{setspace}

\usepackage{natbib}
\setlength{\bibsep}{0.0pt}


%%%%%%%%%%%%%%%%%%%%%
%% Begin doc
%%%%%%%%%%%%%%%%%%%%%

\begin{document}




\title{Building simulation models coupling territorial and network dynamics at the interface of disciplines and scales\\
\textit{Spatial Data Science 2021}
}
\author{Juste Raimbault$^{1,2,3}$\\
$^1$ CASA, UCL\\
$^3$ UPS CNRS 3611 ISC-PIF\\
$^3$ UMR CNRS 8504 Géographie-cités
}
\date{}

\maketitle

\justify

\pagenumbering{gobble}

\vspace{-0.5cm}
\textbf{Keywords : }\textit{Transportation Networks; Territorial Systems; Co-evolution; Simulation Models}
% Transportation Networks Territories Co-evolution Simulation

\medskip

Interactions between transportation networks and the dynamics of land-use are crucial to take into account when planning and managing sustainable urban environments. A quantitative understanding of such processes using models has been proposed by several disciplines, including Land-use Transport Interaction models or economic models of transport network growth. A co-evolution approach (in terms of circular causal relations) to modelling interactions between transportation networks and territories can be proposed, to simulate the dynamics of territories on long time scales. Such a viewpoint has however not been extensively explored, as the question is by nature interdisciplinary and at the interface of spatial and temporal scales.

%is central to most approaches in territorial sciences, but remains poorly understood in a quantitative way: the example of \emph{the myth of structural effects of transportation infrastructures}~\citep{offner1993effets} shows that simple causal assumptions do not hold when trying to explain the \emph{co-evolution} between transportation networks and localized components of territorial systems.
%Dynamical modeling including both evolutions has been emphasized as a cornerstone for a better understanding of involved processes in the case of cities systems on long time scales (\citep{bretagnolle:tel-00459720}, p. 162-163).
%However, a broad interdisciplinary state-of-the-art, based on algorithmic systematic review, done in~\citep{raimbault2015models} shows the quasi-absence of models of that type (e.g. LUTI models at a middle scale where networks are considered static~\citep{iacono2008models}, whereas network growth models on a longer time scale~\citep{xie2009modeling}).

The purpose of this communication is to synthesise results obtained with simple agent-based and simulation models at different scales and integrating paradigms from different disciplines, from planning to transport geography, urban geography, physics and economics. These models have the common feature of strongly coupling territorial and transportation network dynamics.

A first model at the scale of the urban area explores the coupling between a population density generation model based on aggregation-diffusion processes with multiple network growth heuristics. This co-evolution model capture in practice the growth of urban form, and multiple dynamical regimes between population and road networks. It is calibrated on empirical morphological and network measures computed on spatial windows covering Europe.

We then develop a family of models at the macroscopic scale to simulate systems of cities, and more particularly the co-evolution of cities and interurban transportation networks. Building on dynamical models developed in the frame of Pumain's evolutionary urban theory, this allows investigating self-reinforcement processes between urban and network hierarchies. We show in particular that these models are effectively capturing a diversity of co-evolution regimes (in the sense of a circular causation) between the properties of network and cities.

%~\citep{batty2006hierarchy}, and network generation heuristics for which biological network generation~\citep{TeroAl10} and generalized gravity potential rupture are tested. Density model is explored and calibrated alone for morphological objectives through intensive computation, against real data from European density grid, which patterns are well reproduced by the calibrated model. Generated density grids are then used as a basis for network generation, which provide a broad spectrum of values for network measures and correlations between morphological and network measures. It shows that the inclusion of transportation network \emph{is not necessary} to reproduce typical patterns of urban growth, but that explanation of processes and interdependence mechanisms can only be done through more complex models.

We finally describe an agent-based model at the metropolitan scale including more complex mechanisms for coupling, in particular a governance process for the growth of the transport network based on game theory, coupled to a Lowry model for land-use dynamics. This model is applied and calibrated on the case study of Pearl River Delta, China.

This synthesis emphasises the role of simulation models in the production of knowledge. Indeed, most results are obtained through the systematic exploration of models and the application of new model validation methods provided by the OpenMOLE platform. Such systematic explorations enable the testing of hypothesis, and clarify theory building when linked to empirical data and stylised facts. This also facilitates a modular approach to modeling, and thus the coupling of concepts and processes from different disciplines.

%capturing the feedback of the territory on the network. The partial validation on stylized facts at different scales (e.g. land-use evolution patterns, network shape) and the exploration of parameter space suggest targeted experiments such as the comparison of governance systems, and pave the way to more operational similar dynamic models.

% out of purpose
%The conclusions of these first modeling experiments unveil or confirm requirements of a theoretical framework for territorial systems modeling. They include in particular a framing of the notion of coupling between subsystems, a precise definition of scale and an emphasis on emergence to take into account multi-scale aspects of systems, the superposition of heterogeneous views and components of a system. Starting from a perspectivist point of view~\citep{giere2010scientific}, we consider a system as a set of perspectives consisting in ontological sets~\citep{livet2010} associated with dataflow machines~\citep{golden2012modeling}. Formal pre-orders between subsets of ontologies, constructed from emergence relations~\citep{bedau2002downward}, yield after a canonical reduction an unique forest representing the structure of the system. Strong coupled components reside within nodes, whereas a temporal scale and ``thematic'' scale (scale for a state variable) can be associated to each node of the forest by construction from dataflow machines timescales. This framework is formally self-consistent and meets our requirements. Its application should in future work guide the construction of operational models of co-evolution.





%%%%%%%%%%%%%%%%%%%%
%% Biblio
%%%%%%%%%%%%%%%%%%%%
%\tiny

%\begin{multicols}{2}

%\setstretch{0.3}
%\setlength{\parskip}{-0.4em}


%\bibliographystyle{apalike}
%\bibliography{/Users/Juste/Documents/ComplexSystems/CityNetwork/Biblio/Bibtex/CityNetwork}

%\end{multicols}



\end{document}
