%%%%%%%%%%%%%%%%%%%%%%%%%%%%%
% Standard header for working papers
%
% WPHeader.tex
%
%%%%%%%%%%%%%%%%%%%%%%%%%%%%%

\documentclass[11pt]{article}

% packages without options
\usepackage{amsmath,bbm}

% geometry
\usepackage[margin=2cm]{geometry}






\title{\vspace{-2.5cm}Generation of Correlated Synthetic Data\\\medskip
\textit{Actes des Journées de Rochebrune 2016}
}
\author{\noun{Juste Raimbault}$^{1,2}$\\
$^{1}$ UMR CNRS 8504 Géographie-cités\\
$^{2}$ UMR-T IFSTTAR 9403 LVMT
}
\date{}%15 janvier 2016 - v1.01}


\maketitle

\justify

\vspace{-0.5cm}
\begin{abstract}
Generation of hybrid synthetic data resembling real data to some criteria is an important methodological and thematic issue in most disciplines which study complex systems. Interdependencies between constituting elements, materialized within respective relations, lead to the emergence of macroscopic patterns. Being able to control the dependance structure and level within a synthetic dataset is thus a source of knowledge on system mechanisms. We propose a methodology consisting in the generation of synthetic datasets on which correlation structure is controlled. The method is applied in a first example on financial time-series and allows to understand the role of interferences between components at different scales on performances of a predictive model. A second application on a geographical system is then proposed, in which the weak coupling between a population density model and a network morphogenesis model allows to simulate territorial configurations. The calibration on morphological objective on european data and intensive model exploration unveils a large spectrum of feasible correlations between morphological and network measures. We demonstrate therein the flexibility of our method and the variety of possible applications.
%La génération de données synthétiques hybrides similaires à des données réelles présente des enjeux méthodologiques et thématiques pour la plupart des disciplines dont l'objet est l'étude de systèmes complexes. Comme l'interdépendance entre les éléments constitutifs d'un système, matérialisée par leur relations, conduit à l'émergence de ses propriétés macroscopiques, une possibilité de contrôle de l'intensité des dépendances dans un jeu de données synthétiques est un instrument de connaissance du comportement du système. Nous proposons une méthodologie de génération de données synthétiques hybrides sur lequel la structure de correlation est contrôlée. La méthode est illustrée sur des séries temporelles financières et permet l'étude de l'interférence entre composantes à différentes fréquences sur la performance d'un modèle prédictif, en fonction des correlations entre composantes à différentes échelles. On présente ensuite une application à un système géographique, dans laquelle le couplage faible d'un modèle de distribution de densité de population avec un modèle de génération de réseau permet la simulation de configurations territoriales, qui sont calibrées selon des objectifs morphologiques sur l'ensemble de l'Europe. L'exploration intensive du modèle permet l'obtention d'un large spectre de valeurs pour la matrice de correlation entre mesures morphologiques et mesures du réseau. On démontre ainsi les possibilités d'applications variées et les potentialités de la méthode.
\end{abstract}

\medskip

\textbf{Keywords : } \textit{Synthetic Data ; Statistical Control ; Correlations ; Financial Time-series ; Land-use Transportation Interactions
}


%%%%%%%%%%%%%%%%%%%%%%
\section{Introduction}
%%%%%%%%%%%%%%%%%%%%%%

The use of synthetic data, in the sense of statistical populations generated randomly under constraints of patterns proximity to the studied system, is a widely used methodology, and more particularly in disciplines related to complex systems such as therapeutic evaluation~\cite{abadie2010synthetic}, territorial science~\cite{moeckel2003creating,pritchard2009advances}, machine learning~\cite{bolon2013review} or bio-informatics~\cite{van2006syntren}. It can consist in data desegregation by creation of a microscopic population with fixed macroscopic properties, or in the creation of new populations at the same scale than a given sample, with criteria of proximity to the real sample. These criteria will depend on expected applications and can for example vary from a restrictive statistical fit on given indicators, to weaker assumptions of similarity in aggregated patterns. In the case of chaotic systems, or systems where emergence plays a strong role, a microscopic property does not directly imply given macroscopic patterns, which reproduction is indeed one aim of modeling and simulation practices in complexity science. With the rise of new computational paradigms~\cite{arthur2015complexity}, data (simulated, measured or hybrid) shape our understanding of complex systems. Methodological tools for data-mining and modeling and simulation (including the generation of synthetic data) are therefore crucial to be developed.
%L'utilisation de données synthétiques, au sens de populations statistiques d'individus générées aléatoirement sous la contrainte de reproduire certaines caractéristiques du système étudié, est une pratique méthodologique largement répandue dans de nombreuses disciplines, et particulièrement pour des problématiques liées aux systèmes complexes, telles que par exemple l'évaluation thérapeutique~\cite{abadie2010synthetic}, l'étude des systèmes territoriaux~\cite{moeckel2003creating,pritchard2009advances}, l'apprentissage statistique~\cite{bolon2013review} ou la bio-informatique~\cite{van2006syntren}. Il peut s'agir d'une désagrégation par création d'une population au niveau microscopique présentant des caractéristiques macroscopiques données, ou bien de la création de nouvelles populations au même niveau d'agrégation qu'un échantillon donné avec un critère de ressemblance aux données réelles.  Le niveau de ce critère peut dépendra des applications attendues et peut par exemple aller de la fidélité des distributions statistiques pour un certain nombre d'indicateurs à des contraintes plus faibles de valeurs pour des indicateurs agrégés, c'est à dire l'existence de motifs macroscopiques similaires. Dans le cas de systèmes chaotiques ou présentant de fortes caractéristiques d'émergence, une contrainte microscopique n'implique pas nécessairement le respect des motifs macroscopiques, et arriver à les reproduire est justement un des enjeux des pratiques de modélisation et simulation en sciences de la complexité. La donnée, qu'elle soit simulée, mesurée ou hybride est au coeur de l'étude des systèmes complexes de par la maturation de nouvelles approches computationelles~\cite{arthur2015complexity}, il est donc essentiel d'étudier des procédures d'extraction d'information des données (fouille de données) et de simulation d'une information similaire (génération de données synthétiques).

Whereas first order (in the sense of distribution moments) is generally well used, it is not systematic nor simple to control generated data structure at second order, i.e. covariance structure between generated variables. Some specific examples can be found, such as in~\cite{ye2011investigation} where the sensitivity of discrete choices models to the distributions of inputs and to their dependance structure is examined. It is also possible to interpret complex networks generative models~\cite{newman2003structure} as the production of an interdependence structure for a system, contained within link topology. We introduce here a generic method taking into account dependance structure for the generation of synthetic datasets, more precisely with the mean of controlled correlation matrices.

%Si le premier ordre est de manière générale bien maîtrisé, il n'est pas systématique ni aisé de contrôler le second ordre, c'est à dire les structures de covariance entre les variables générées, même si des exemples spécifiques existent, comme dans~\cite{ye2011investigation} où la sensibilité des sorties de modèles de choix discrets à la forme des distributions des variables aléatoires ainsi qu'à leur structures de dépendance. Il est également possible d'interpréter les modèles de génération de réseaux complexes~\cite{newman2003structure} comme la création d'une structure d'interdépendance au sein d'un système, représentée par la topologie des liens. Nous proposons ici une méthode générique prenant en compte l'interdépendance lors de la génération de données synthétiques, sous la forme de correlations.

The rest of the paper is organized as follows. The generic method is formally described, to be then applied on very different examples both entering application frame. Each example can be read independently and illustrates potentialities of the method and possible technical limitations. We discuss then possible further developments and applications, in particular for a geographical system.
%La suite de l'article est organisée de la façon suivante. La méthode générique est décrite formellement, pour être appliquée sur deux exemples très différents mais rentrant dans ce même cadre. Chaque exemple peut être lu de manière indépendante et illustre les possibilités offertes par la méthode ainsi que des possibles obstacles à sa mise en oeuvre. Les implications et applications possibles sont ensuite discutées, notamment dans le cas de l'exemple d'un système géographique.


%%%%%%%%%%%%%%%%%%%%%%
\section{Method Formalization}
%%%%%%%%%%%%%%%%%%%%%%

Domain-specific methods aforementioned are too broad to be summarized into a same formalism. We propose a framework as generic as possible, centered on the control of correlations structure in synthetic data.
%L'ensemble des méthodologies mentionnées en introduction sont trop variées pour être résumées par un même formalisme. Nous proposons ici une formulation générique ne dépendant pas du domaine d'application, ciblée sur le contrôle de la structure de correlation des données synthétiques.

Let $\vec{X}_I$ a multidimensional stochastic process (that can be indexed e.g. with time in the case of time-series, but also space, or discrete set abstract indexation). We assume given a real dataset $\mathbf{X}=(X_{i,j})$, interpreted as a set of realizations of the stochastic process. We propose to generate a statistical population $\mathbf{\tilde{X}}=\tilde{X}_{i,j}$ such that
\begin{enumerate}
\item a given criteria of proximity to data is verified, i.e. given a precision $\varepsilon$ and an indicator $f$, we have $\norm{f(\mathbf{X})-f(\mathbf{\tilde{X}})} < \varepsilon$
\item level of correlation is controlled, i.e. given a matrix $R$ fixing correlation structure, $\hat{\Var{}}\left[(\tilde{X}_i)\right] = \Sigma R$, where the covariance matrix is estimated on the synthetic population.
\end{enumerate}

%Soit un processus stochastique multidimensionnel $\vec{X}_I$ (l'ensemble d'indexation pouvant être par exemple le temps dans le cas de séries temporelles, l'espace, ou une indexation quelconque). On se propose, à partir d'un jeu de réalisations $\mathbf{X}=(X_{i,j})$, de générer une population statistique $\mathbf{\tilde{X}}=\tilde{X}_{i,j}$ telle que
%\begin{enumerate}
%\item d'une part un certain critère de proximité aux données est vérifié, i.e. étant donné une précision $\varepsilon$ et un indicateur $f$, $\norm{f(\mathbf{X})-f(\mathbf{\tilde{X}})} < \varepsilon$
%\item d'autre part le niveau de correlation est controlé, i.e. étant donné une matrice fixant une structure de covariance $R$, $\Varb{(\tilde{X}_i)} = R$, où la matrice de variance/covariance est estimée sur la population synthétique.
%\end{enumerate}

The second requirement will generally be conditional to parameter values determining generation procedure, either generation models being simple or complex ($R$ itself is a parameter). Formally, synthetic processes are parametric families $\tilde{X}_i[\vec{\alpha}]$. % TODO : explicit the fact that real data may come out of different parameter values ?
We propose to apply the methodology on very different examples, both typical of complex systems : financial high-frequency time-series and territorial systems. We illustrate the flexibility of the method, and claim to help building interdisciplinary bridges by methodology transposition and reasoning analogy. In the first case, proximity to data is the equality of signals at a fundamental frequency, to which higher frequency synthetic components with controlled correlations are superposed. It follows a logic of hybrid data for which hypothesis or model testing is done on a more realistic context than on purely synthetic data. In the second case, morphological calibration of a population density distribution model allows to respect real data proximity. Correlations of urban form with transportation network measures are empirically obtained by exploration of coupling with a network morphogenesis model. The control is in this case indirect as feasible space is empirically determined.
%La satisfaction du deuxième point sera généralement conditionnée par la valeur de paramètres, dont dépendra la procédure de génération, qu'il s'agisse de modèles simples ou complexes. Formellement, les processus synthétiques sont des familles paramétriques $\tilde{X}_i[\vec{\alpha}]$. Nous proposons de décliner cette méthode sur deux exemples très différents mais tous deux typiques des systèmes complexes : des séries temporelles financières à haute fréquence, et les systèmes territoriaux. On illustre ainsi la flexibilité de la logique, ouvrant des portes interdisciplinaires par l'exportation de méthodes ou raisonnements par exemple. Dans le premier cas, la proximité aux données est l'égalité des signaux à une fréquence fondamentale, auxquels on superpose des composantes synthétiques dont il est facile de contrôler le niveau de correlation. On se place dans une logique de données hybrides, pour tester des hypothèses ou modèles dans un contexte plus proche de la réalité que sur des données purement synthétiques. Dans le deuxième cas, la calibration morphologique d'un modèle de distribution de densité de peuplement permet de respecter le critère de proximité aux données. Les correlations de la forme urbaine avec celle d'un réseau de transport sont ensuite obtenues empiriquement par exploration du couplage avec un modèle de génération de réseau. Leur contrôle est dans ce cas indirect puisque constaté empiriquement.


%%%%%%%%%%%%%%%%%%%%%%
\section{Applications}
%%%%%%%%%%%%%%%%%%%%%%



%%%%%%%%%%%%%%%%%%%%%%
\subsection{Application : financial time-series}


%%%%%%%%%%%%%%%%%%%%%%
\subsubsection{Context}

Our first field of application is that of financial complex systems, of which captured signals, financial time-series, are heterogeneous, multi-scalar and highly non-stationary~\cite{mantegna2000introduction}. Correlations have already been the object of a broad bunch of related literature. For example, Random Matrix Theory allows to undress signal of noise, or at least to estimate the proportion of information undistinguishable from noise, for a correlation matrix computed for a large number of asset with low-frequency signals (daily returns mostly)~\cite{2009arXiv0910.1205B}. Similarly, Complex Network Analysis on networks constructed from correlations, by methods such as Minimal Spanning Tree~\cite{2001PhyA..299...16B} or more refined extensions developed for this purpose~\cite{tumminello2005tool}, yielded promising results such as the reconstruction of economic sectors structure. At high frequency, the precise estimation of of interdependence parameters in the framed of fixed assumptions on asset dynamics, has been extensively studied from a theoretical point of view aimed at refinement of models and estimators~\cite{barndorff2011multivariate}. Theoretical results must be tested on synthetic datasets as they ensure a control of most parameters in order to check that a predicted effect is indeed observable \emph{all things equal otherwise}. For example, \cite{potiron2015estimation} obtains a bias correction for the \emph{Hayashi-Yoshida} estimator (used to estimate integrated covariation between two brownian at high frequency in the case of asynchronous observation times) by deriving a central limit theorem for a general model that endogeneize observation times. Empirical confirmation of estimator improvement is obtained on a synthetic dataset at a fixed correlation level.

%Un premier domaine d'application proposé pour notre méthode est celui des séries temporelles financières, signaux typiques de systèmes complexes hétérogènes et multiscalaires~\cite{mantegna2000introduction} et pour lesquels les corrélations ont fait l'objet d'abondants travaux. Ainsi, l'application de la théorie des matrices aléatoires peut permettre de débruiter, ou du moins d'estimer la part de signal noyée dans le bruit, une matrice de correlations pour un grand nombre d'actifs échantillonnés à faible fréquence (retours journaliers par exemple)~\cite{2009arXiv0910.1205B}. De même, l'analyse de réseaux complexes construits à partir des corrélations, selon des méthodes type arbre couvrant minimal~\cite{2001PhyA..299...16B} ou des extensions raffinées pour cette application précise~\cite{tumminello2005tool}, ont permis d'obtenir des résultats prometteurs, tels la reconstruction de la structure économique des secteurs d'activités. A haute fréquence, l'estimation précise de paramètres d'interdépendance dans le cadre d'hypothèses fixées sur la dynamique, fait l'objet d'importants travaux théoriques dans un but de raffinement des modèles et des estimateurs~\cite{barndorff2011multivariate}. Les résultats théoriques doivent alors être testés sur des jeux de données synthétiques, qui permettent de contrôler un certain nombre de paramètres et de s'assurer qu'un effet prédit par la théorie est bien observable \emph{toutes choses égales par ailleurs}. Par exemple, \cite{potiron2015estimation} dérive une correction du biais de l'estimateur de \emph{Hayashi-Yoshida} qui est un estimateur de la covariance de deux browniens corrélés à haute fréquence dans le cas de temps d'observation asynchrones, par démonstration d'un théorème de la limite centrale pour un modèle généralisé endogénéisant les temps d'observations. La confirmation empirique de l'amélioration de l'estimateur est alors obtenue sur un jeu de données synthétiques à un niveau de corrélation fixé.


%%%%%%%%%%%%%%%%%%%%%%
\subsubsection{Formalization}

\paragraph{Framework}

We consider a network of assets $(X_i(t))_{1\leq i \leq N}$ sampled at high-frequency (typically 1s).

Considérons un réseau d'actifs $(X_i(t))_{1\leq i \leq N}$ échantillonnés à haute fréquence (typiquement 1s). On se place dans un cadre multi-scalaire (utilisé par exemple dans les approches par ondelettes~\cite{ramsey2002wavelets} ou analyses multifractales du signal~\cite{bouchaud2000apparent}) pour interpréter les signaux observés comme la superposition de composantes à des multiples échelles temporelles : $X_i=\sum_{\omega}{X_i^{\omega}}$. On notera $T_i^{\omega} = \sum_{\omega' \leq \omega} X_i^{\omega}$ le signal filtré à une fréquence $\omega$ donnée. Prédire l'évolution d'une composante à une échelle donnée est alors un problème caractéristique de l'étude des systèmes complexes, pour lequel l'enjeu est l'identification de régularités et leur distinction des composantes considérées comme stochastiques en comparaison\footnote{voir~\cite{gell1995quark} pour une discussion étendue sur la construction de \emph{schema} pour l'étude de systèmes complexes adaptatifs (par des systèmes complexes adaptatifs).}. Dans un souci de simplicité, on représente un tel processus par un modèle de prédiction de tendance à une échelle temporelle $\omega_1$ donnée, formellement un estimateur $M_{\omega_1} : (T_i^{\omega_1}(t'))_{t'<t} \mapsto \hat{T_i}^{\omega_1}(t)$ dont l'objectif est la minimisation de l'erreur sur la tendance réelle $\norm{T_i^{\omega_1} - \hat{T}_i^{\omega_1}}$. Dans le cas d'estimateurs auto-regressifs multivariés, la performance dépendra entre autre des correlations respectives entre actifs et il est alors intéressant d'utiliser la méthode pour évaluer celle-ci en fonction de niveaux de correlation à plusieurs échelles. On assume une dynamique de Black-Scholes~\cite{jarrow1999honor} pour les actifs, i.e. $dX = \sigma\cdot dW$ avec $W$ processus de Wiener, ce qui permettra d'obtenir facilement des niveaux de correlation voulus.

\paragraph{Génération des données}

Il est alors aisé de générer $\tilde{X}_i$ tel que $\Varb{\tilde{X}_i^{\omega_1}}=\Sigma R$ ($\Sigma$ variance estimée et $R$ matrice de corrélation fixée), par la simulation de processus de Wiener au niveau de corrélation fixé et tel que $X_i^{\omega \leq \omega_0} = \tilde{X}_i^{\omega \leq \omega_0}$ (critère de proximité au données : les composantes à plus basse fréquence qu'une fréquence fondamentale $\omega_0 < \omega_1$ sont identiques). En effet, si $dW_1 \indep dW_1^{\indep}$ (et $\sigma_1 < \sigma_2$ pour fixer les idées, quitte à échanger les actifs), alors $W_2 = \rho_{12}W_1 + \sqrt{1-\frac{\sigma_1^2}{\sigma_2^2}\cdot\rho_{12}^2}W_1^{\indep}$ est tel que $\rho(dW_1,dW_2)=\rho_{12}$. Les signaux suivants sont construits de la même manière par orthonormalisation de Gram. On isole alors la composante à la fréquence $\omega_1$ voulue par filtrage, c'est à dire $\tilde{X}_i^{\omega_1} = W_i - \mathcal{F}_{\omega_0}[W_i]$ (avec $\mathcal{F}_{\omega_0}$ filtre passe-bas à fréquence de coupage $\omega_0$), puis on reconstruit les signaux synthétiques par $\tilde{X}_i = T_i^{\omega_0} + \tilde{X}_i^{\omega_1}$.



\subsubsection{Implémentation et Résultats}

\paragraph{Méthodologie}

La méthode est testée sur un exemple de deux actifs du marché des devises (EUR/USD et EUR/GBP), sur une période de 6 mois de juin 2015 à novembre 2015. Le nettoyage des données\footnote{obtenues depuis \texttt{http://www.histdata.com/}, sans licence spécifiée, les données nettoyées et filtrées à $\omega_m$ uniquement sont mises en accessibilité pour respect du copyright.}, originellement échantillonnées à l'ordre de la seconde, consiste dans un premier temps à la détermination du support temporel commun maximal (les séquences manquantes étant alors ignorées, par translation verticale des séries, i.e. $S(t):=S(t)\cdot \frac{S(t_{n})}{S(t_{n-1})}$ lorsque $t_{n-1},t_n$ sont les extrémités du ``trou'' et $S(t)$ la valeur de l'actif, ce qui revient à garder la contrainte d'avoir des retours à pas de temps similaires entre actifs). On étudie alors les \emph{log-prix} et \emph{log-retours}, définis par $X(t):=\log{\frac{S(t)}{S_0}}$ et $\Delta X (t) = X(t) - X(t-1)$. Les données brutes sont filtrées à une fréquence $\omega_m = 10\textrm{min}$ (qui sera la fréquence maximale d'étude) pour un souci de performance computationnelle. On utilise un filtre gaussien non causal de largeur totale $\omega$. On fixe $\omega_0=24\textrm{h}$ et on se propose de construire des données synthétiques aux fréquences $\omega_1 = 30\textrm{min},1\textrm{h},2\textrm{h}$. Voir la figure~\ref{fig:example_signal} pour un exemple de la structure du signal à ce différentes échelles.


%%%%%%%%%%%%%%%%%%%
\begin{figure}%[h!]
\centering
\includegraphics[width=0.6\textwidth,height=0.25\textheight]{figures/asset/ex_filtering}
\caption{\textbf{Exemple de la structure multi-scalaire du signal qui sert de base à la construction des signaux synthétiques | } Les \emph{log-prix} sont représentés sur environ 3h pour la journée du 1er novembre 2015 pour l'actif EUR/USD, ainsi que les tendances à 10min (violet) et à 30min.}
\label{fig:example_signal}
\end{figure}
%%%%%%%%%%%%%%%%%%%



Il est crucial de noter l'interférence entre les fréquences $\omega_0$ et $\omega_1$ dans le signal construit : la correlation effectivement estimée est
\[
\rho_{e} = \rho \left[ \Delta \tilde{X}_1 , \Delta \tilde{X}_2 \right] = \rho \left[ \Delta T_1^{\omega_0} + \Delta \tilde{X}_1^{\omega} , \Delta T_2^{\omega_0} + \Delta \tilde{X}_2^{\omega}\right]
\]
ce qui conduit à dériver dans la limite raisonnable $\sigma_1 \gg \sigma_0$ (fréquence fondamentale suffisamment basse), lorsque $\Covb{\Delta \tilde{X}_i^{\omega_1}}{\Delta X_j^{\omega}}=0$ pour tous $i,j,\omega_1 > \omega$, et les retours d'espérance nulle à toutes échelles, en notant $\rho_0 = \rho \left[ \Delta T_1^{\omega_0} , \Delta T_2^{\omega_0} \right]$, $\rho = \rho \left[  \tilde{X}_1^{\omega_1} , \tilde{X}_2^{\omega_1} \right]$, et $\varepsilon_i = \frac{\sigma (\Delta T_i^{\omega_0})}{\sigma \left( \Delta \tilde{X}_i^{\omega_1}\right)}$, la correction sur la correlation effective due aux interférences : la correlation effective est alors au premier ordre

\begin{equation}
\label{eq:eff_corr}
\rho_e = \left[ \varepsilon_1 \varepsilon_2 \rho_0 + \rho \right] \cdot \left[ 1 - \frac{1}{2}\left(\varepsilon_1^2 + \varepsilon_2^2 \right) \right]
\end{equation}

{\noindent}ce qui donne l'expression de la correlation que l'on pourra effectivement simuler dans les données synthétiques.

La correlation est estimée par méthode de Pearson, avec l'estimateur de la covariance au biais corrigé, c'est à dire $\hat{\rho}[X1,X2] = \frac{\hat{C}[X1,X2]}{\sqrt{\hat{\Var{}}[X1]\hat{\Var{}}[X2]}}$, où $\hat{C}[X1,X2] = \frac{1}{(T-1)}\sum_{t} X_1(t)X_2(t) - \frac{1}{T\cdot (T-1)} \sum_t X_1(t) \sum_t X_2(t)$ et $\hat{\Var{}}[X] = \frac{1}{T}\sum_t{X^2(t)}-\left(\frac{1}{T}\sum_tX(t)\right)^2$.

Le modèle de prédiction $M_{\omega_1}$ testé est simplement un modèle \emph{ARMA} pour lequel on fixe les paramètres $p=2,q=0$ (on ne créée pas de correlation retardée, on ne s'attend donc pas à de grand ordre d'auto-regression, les signaux originaux étant à mémoire relativement courte ; de plus le lissage n'est pas nécessaire puisqu'on travaille sur des données filtrées), appliqué de manière adaptative\footnote{il s'agit d'un niveau d'adaptation relativement faible, les paramètres $T_W,p,q$ et même le type de modèle restant fixés. On se place ainsi dans le cadre de~\cite{potiron2016estimating} qui suppose une dynamique localement paramétrique, mais pour lequel on fixe les méta-paramètres de la dynamique. On pourrait imaginer estimer un $T_W$ variable qui s'adapterait pour une meilleure estimation locale, à l'image de l'estimation de paramètres en traitement du signal bayesien effectuée via augmentation de l'état par les paramètres.}. Plus précisément, étant donné une fenêtre temporelle $T_W$, on estime pour tout $t$ le modèle sur $[t-T_W+1,t]$ afin de prédire les signaux à $t+1$.


\paragraph{Implémentation}

L'implémentation est faite en language R, utilisant en particulier la bibliothèque \texttt{MTS}~\cite{Tsay:2015xy} pour les modèles de séries temporelles. Les données nettoyées et le code source sont disponibles de manière ouverte sur le dépôt \texttt{git} du projet\footnote{at \texttt{https://github.com/JusteRaimbault/SynthAsset}}.

\paragraph{Résultats}









La figure~\ref{fig:effective_corrs} donne les correlations effectives calculées sur les données synthétiques. Pour des valeurs standard des paramètres (par exemple pour $\omega_0=24\textrm{h}$, $\omega_1=2\textrm{h}$ et $\rho=-0.5$), on a $\rho_0\simeq 0.71$ et $\varepsilon_i \simeq 0.3$ et ainsi $\left| \rho_e - \rho \right|\simeq 0.05$. On constate dans l'intervalle $\rho \in [-0.5,0.5]$ un bon accord entre la valeur $\rho_e$ prédite par~\ref{eq:eff_corr} et les valeurs observées, et une déviation pour de plus grandes valeurs absolues, d'autant plus grande que $\omega_1$ est petit : cela confirme l'intuition que lorsque la fréquence descend et se rapproche de $\omega_0$, les interférences entre les deux composantes vont devenir non négligeables et invalider les hypothèses d'indépendance par exemple.

On applique ensuite le modèle prédictif décrit ci-dessus aux données synthétiques, afin d'étudier sa performance moyenne en fonction du niveau de correlation des données. Les résultats pour $\omega_1 = 1\textrm{h},1\textrm{h}30,2\textrm{h}$ sont présentés en figure~\ref{fig:model_perf}. Le résultat a priori contre-intuitif d'une performance maximale à correlation nulle pour l'un des actifs confirme l'intérêt d'une génération de données hybrides : l'étude des correlations décalées (\emph{lagged correlations}) montre une dissymétrie présente dans les données réelles, interprété à l'échelle journalière comme une influence augmentée de EURGBP sur EURUSD à 2h de décalage environ. L'existence de ce \emph{lag} permet une ``bonne'' prédiction de EURUSD due à la fréquence fondamentale, perturbée par le bruit ajouté, de façon proportionnelle à sa correlation : plus les bruits sont corrélés, plus le modèle les prendra en compte et se trompera plus à cause du caractère markovien des browniens simulés\footnote{en théorie le modèle utilisé n'a aucun pouvoir prédictif sur des browniens purs}.

L'exemple présenté ici est un \emph{modèle jouet} et n'a pas d'application pratique, mais démontre l'intérêt de l'utilisation des données synthétiques simulées. On peut imaginer simuler des données plus proches de la réalité (existence de motifs réalistes de \emph{lagged correlation} par exemple, modèles plus réalistes que le Black-Scholes) et appliquer la méthode sur des modèles plus opérationnels.




%%%%%%%%%%%%%%%%%%%
\begin{figure}[h!]
\centering
% figure : effective correlations, with confidence intervals (bootstrap), for the 3 filtering scales. ; AND expected corrected corrs from derivation above.
\vspace{-0.7cm}
\includegraphics[width=0.8\textwidth,height=0.3\textheight]{figures/effectiveCorrs_withGoodTh_A4}
\caption{\small\textbf{Correlations effectives obtenues sur les données synthétiques | } Les points représentent les correlations estimées sur une génération d'un jeu de données synthétiques correspondant aux 6 mois de juin à novembre 2015 (barres d'erreurs obtenue par méthode de Fisher standard) ; l'échelle de couleur donne la fréquence de filtrage $\omega_1=10\textrm{min},30\textrm{min},1\textrm{h},2\textrm{h},4\textrm{h}$ ; les courbes sont les valeurs théoriques de $\rho_e$ obtenues par~\ref{eq:eff_corr} avec les volatilités estimées (diagonale en pointillés pour référence) ; l'abscisse de la ligne rouge est la valeur théorique telle que $\rho = \rho_e$ avec les valeurs moyennes de $\varepsilon_i$ sur l'ensemble des points. On observe dans les fortes correlations une déviation des valeurs corrigées, qui peut être dû aux hypothèses d'indépendance ou d'espérance nulle non vérifiées. La dissymétrie de la courbe est causée par la forte valeur positive de $\rho_0 \simeq 0.71$.}
\label{fig:effective_corrs}
\end{figure}
%%%%%%%%%%%%%%%%%%%


%%%%%%%%%%%%%%%%%%%
\begin{figure}[h!]
% figure : performance of arma2 as a function of expected correlation.
%\vspace{-0.7cm}
\centering
\includegraphics[width=0.48\textwidth,height=0.16\textheight]{figures/asset/pred_filt6}
\includegraphics[width=0.48\textwidth,height=0.16\textheight]{figures/asset/pred_filt9}\\
\includegraphics[width=0.48\textwidth,height=0.16\textheight]{figures/asset/pred_filt12}
\includegraphics[width=0.48\textwidth,height=0.16\textheight]{figures/asset/lagged_corrs}
\caption{\small \textbf{Performance d'un modèle prédictif en fonction des correlations simulées.} De gauche à droite et de haut en bas, les trois premiers graphes montrent pour chaque actif la performance normalisée d'un modèle ARMA ($p=2,q=0$), définie par $\pi = \left(\frac{1}{T}\sum_t\left(\tilde{X}_i(t) - M_{\omega_1}\left[\tilde{X}_i\right](t)\right)^2 \right) / \sigma \left[ \tilde{X}_i \right]^2$ (IC à 95\% calculés par $\pi = \bar{\pi} \pm (1.96\cdot \sigma [\pi])/\sqrt{T}$, lissage polynomial local pour aider à la lecture). Il est intéressant de note la courbe en U, pour EURUSD, due à l'interférence entre les composantes par laquelle la correlation des bruits simulés détériore le pouvoir prédictif : l'étude des \emph{lagged correlations} (ici $\rho [\Delta X_{\textrm{EURUSD}}(t),\Delta X_{\textrm{EURGBP}}(t-\tau)]$) sur les données réelles dans le quatrième graphique montre une dissymétrie dans les courbes par rapport au décalage nul $(\tau = 0)$ ce qui devrait permettre à la composante fondamentale d'améliorer le pouvoir de prédiction alors perturbé par les correlations entre les composantes simulées. Les lignes pointillées montrent les pas de temps (en équivalent décalage temporel) utilisés par l'ARMA aux différentes échelles, ce qui permet de lire la correlation correspondante sur la composante fondamentale.}
\label{fig:model_perf}
\end{figure}
%%%%%%%%%%%%%%%%%%%




\newpage




%%%%%%%%%%%%%%%%%%%%%%
\subsection{Application : données géographiques de densité et de réseaux}


%%%%%%%%%%%%%%%%%%%%%%
\subsubsection{Contexte}


En géographie, l'utilisation de données synthétiques est plus généralement axée vers l'utilisation de population synthétiques au sein de modèles basés agents (mobilité, modèles \emph{LUTI})~\cite{pritchard2009advances}. On peut également citer des méthodes d'analyse spatiales qui s'en rapprochent : par exemple, l'extrapolation d'un champ spatial continu à partir d'un échantillon discret, par une estimation par noyaux par exemple, peut être compris comme la génération d'un jeu de données synthétiques (même si ce n'est pas le point de vue initial, comme pour la Regression Géographique Pondérée~\cite{brunsdon1998geographically}, dans laquelle les noyaux de taille variables n'interpolent pas des données au sens propre mais extrapolent des variables abstraites représentant l'interaction entre variables explicites). Dans le domaine de la modélisation en géographie quantitative, dans le cas de \emph{modèles jouets} ou de modèles hybrides, une configuration initiale cohérente est souvent essentielle : un ensemble de configurations initiales possibles est alors un jeu de données synthétiques sur lesquelles le modèle est testé : le premier modèle Simpop~\cite{sanders1997simpop}, pionnier d'une famille de modèles par la suite paramétrisés par des données réelles, pourrait rentrer dans ce cadre mais était lancé sur une spatialisation synthétique unique. De même, il a été souligné la difficulté de générer une configuration initiale pour une infrastructure de transport dans le cas du modèle SimpopNet~\cite{schmitt2014modelisation}, alors qu'il s'agit un point essentiel dans la connaissance du comportement du modèle. Il a récemment été proposé de contrôler systématiquement les effets de la configuration spatiale sur le comportement de modèles de simulation spatialisés~\cite{cottineau2015revisiting}, méthodologie pouvant être interprétée comme un contrôle par données statistiques spatiales. L'enjeu est de pouvoir alors distinguer effets propres dus à la dynamique intrinsèque du modèle, d'effet particuliers dus à la structure géographique du cas d'application. Celui-ci est crucial pour la validation des conclusions issues des pratiques de modélisation et simulation en géographie quantitative.


%%%%%%%%%%%%%%%%%%%%%%
\subsubsection{Formalisation}

Dans notre cas, nous proposons de générer des systèmes de villes représentés par une densité spatiale de population $d(\vec{x})$ et la donnée d'un réseau de transport $n(\vec{x})$, représenté de façon simplifiée, pour lesquels on serait capable de contrôler les correlations entre mesures morphologiques de la densité urbaine et caractéristiques du réseau. La question de l'interaction entre territoire et réseaux de transport est un sujet d'étude classique~\cite{offner1996reseaux} mais extrêmement complexe et difficile à quantifier~\cite{offner1993effets}. Une modélisation dynamique des processus impliqués devrait apporter des connaissances sur ces interactions (\cite{bretagnolle:tel-00459720}, p. 162-163). Dans ce cadre, nous développons un couplage \emph{simple} (c'est à dire sans boucle de rétroaction) entre un modèle de morphogenèse urbaine et un modèle de génération de réseau.


\paragraph{Modèle de densité}

L'utilisation d'un modèle $D$ type agrégation-diffusion~\cite{batty2006hierarchy} permet de générer une distribution discrete de densité. Dans \cite{raimbault2016calibration}, une généralisation de ce modèle est calibré pour des objectifs morphologiques (entropie, hiérarchie, auto-corrélation spatiale, distance moyenne) contre les valeurs réelles calculées sur l'ensemble des grilles de taille 50km extraites de la grille européenne de densité~\cite{eurostat}. Plus précisément, le modèle fonctionne de manière itérative de la façon suivante. Une grille initialement vide de côté $N$, est représentée par la données des populations $(P_i(t))_{1\leq i\leq N^2}$. A chaque pas de temps, jusqu'à ce que la population atteigne une valeur fixée $P_m$,
\begin{itemize}
\item la population totale $P(t)$ est augmentée d'un nombre fixé $N_G$ (taux de croissance), suivant un attachement préférentiel tel que $\Pb{P_i(t+1)=P_i(t)+1|P(t+1)=P(t)+1}=\frac{(P_i(t)/P(t))^{\alpha}}{\sum(P_i(t)/P(t))^{\alpha}}$
\item une diffusion d'une fraction $\beta$ de la population aux 4 plus proches voisins est effectuée $n_d$ fois
\end{itemize}

Les deux processus antagonistes de concentration et d'étalement urbain sont capturés par le modèle, ce qui permet de reproduire assez fidèlement un grand nombre de morphologies existantes.


\paragraph{Modèle de réseau}

D'autre part, on est capable de générer par un modèle $N$ un réseau de transport planaire à une échelle équivalente, étant donné une distribution de densité. La génération du réseau étant conditionnée à la donnée de la densité, les estimateurs des indicateurs de réseau seront conditionnels d'une part, et d'autre part les formes urbaines et du réseau devraient nécessairement être corrélées, les processus n'étant pas indépendants. La nature et la modularité de ces correlations selon la variation des paramètres des modèles restent à déterminer par l'exploration du modèle couplé.

La procédure de génération heuristique de réseau est la suivante :
\begin{enumerate}
\item Un nombre fixé $N_c$ de centres qui seront les premiers noeuds du réseau est distribué selon la distribution de densité, suivant une loi similaire à celle d'agrégation, i.e. la probabilité d'être distribué sur une case est $\frac{(P_i/P)^{\alpha}}{\sum (P_i/P)^{\alpha}}$. La population est ensuite répartie selon les zones de Voronoi des centres, un centre cumulant la population des cases dans son emprise.
\item Les centres sont connectés de façon déterministe par percolation entre plus proches clusters : tant que le réseau n'est pas connexe, les deux composantes connexes les plus proches au sens de la distance minimale entre chacun de leurs sommets sont connectées par le lien réalisant cette distance. On obtient alors un réseau arborescent.
\item Le réseau est alors modulé par ruptures de potentiels afin de se rapprocher de formes réelles. Plus précisément, un potentiel d'interaction gravitaire généralisé entre deux centres $i$ et $j$ est défini par
\[
V_{ij}(d) = \left[ (1 - k_h) + k_h \cdot \left( \frac{P_i P_j}{P^2} \right)^{\gamma} \right]\cdot \exp{\left( -\frac{d}{r_g (1 + d/d_0)} \right)}
\]

où $d$ peut être la distance euclidienne $d_{ij}=d(i,j)$ ou la distance par le réseau $d_N(i,j)$, $k_h \in [0,1]$ un poids permettant de changer le rôle des population dans le potentiel, $\gamma$ régissant la forme de la hiérarchie selon les valeurs des populations, $r_g$ distance caractéristique de décroissance et $d_0$ paramètre de forme.
\item Un nombre $K\cdot N_L$ de nouveaux liens potentiels est pris comme les couples ayant le plus grand potentiel pour la distance euclidienne ($K=5$ est fixé).
\item Parmi les liens potentiels, $N_L$ sont effectivement réalisés, qui sont ceux ayant le plus faible rapport $V_{ij}(d_N)/V_{ij}(d_{ij})$ : à cette étape seul l'écart entre distance euclidienne et distance par le réseau compte, ce rapport ne dépendant plus des populations et étant croissant en $d_N$ à $d_{ij}$ fixé.
\item Le réseau est planarisé par création de noeuds aux intersections éventuelles créées par les nouveaux liens.
\end{enumerate}


Notons que la construction du modèle de génération est heuristique, et que d'autres types de modèles comme un réseau biologique auto-généré~\cite{TeroAl10}, une génération par optimisation locale de contraintes géométriques \cite{barthelemy2008modeling} ou un modèle de percolation plus complexe que celui utilisé, peuvent le remplacer. Ainsi, dans le cadre d'une architecture modulaire où le choix entre différentes implémentations d'une brique fonctionnelle peut être vue comme méta-paramètre~\cite{cottineau2015incremental}, on pourrait choisir la fonction de génération adaptée à un besoin donné (par exemple proximité à des données réelles, contraintes sur les relations entre indicateurs de sortie, variété de formes générées, etc.).


\paragraph{Espace des paramètres}

L'espace des paramètres du modèle couplé\footnote{Le couplage faible permet de limiter le nombre total de paramètres puisqu'un couplage fort incluant des boucles de retroaction comprendrait nécessairement des paramètres supplémentaires pour régler la forme et l'intensité de celles-ci. Pour espérer le diminuer, il faudrait concevoir un modèle intégré, ce qui est différent d'un couplage fort dans le sens où il n'est pas possible de figer l'un des sous-systèmes pour obtenir un modèle de l'autre correspondant au modèle non-couplé.} est constitué des paramètres de génération de densité $\vec{\alpha}_D = (P_m/N_G , \alpha,\beta , n_d)$ (on s'intéresse pour simplifier au rapport entre population et taux de croissance, i.e. le nombre d'étapes nécessaires pour générer) et des paramètres de génération de réseau $\vec{\alpha}_N=(N_C,k_h,\gamma , r_g , d_0)$. On notera $\vec{\alpha} = (\vec{\alpha}_D,\vec{\alpha}_N)$.

\paragraph{Indicateurs}

On quantifie la forme urbaine et la forme du réseau, dans le but de moduler la corrélation entre ces indicateurs. La forme est définie par un vecteur $\vec{M}=(r,\bar{d},\varepsilon,a)$ donnant auto-corrélation spatiale (indice de Moran), distance moyenne, entropie, hiérarchie (voir~\cite{le2015forme} pour une définition précise de ces indicateurs). Les mesures de la forme du réseau $\vec{G} = (\bar{c},\bar{l},\bar{s},\delta)$ sont, avec le réseau noté $(V,E)$,
\begin{itemize}
\item Centralité moyenne $\bar{c}$, définie comme la moyenne de la \emph{betweeness-centrality} (normalisée dans $[0,1]$) sur l'ensemble des liens.
\item Longueur moyenne des chemins $\bar{l}$ définie par $\frac{1}{d_m}\frac{2}{|V|\cdot (|V|-1)}\sum_{i<j}d_N(i,j)$ avec $d_m$ distance de normalisation prise ici comme la diagonale du monde $d_m=\sqrt{2}N$.
\item Vitesse moyenne~\cite{banos2012towards}, qui correspond à la performance du réseau par rapport au trajet à vol d'oiseau, définie par $\bar{s} = \frac{2}{|V|\cdot (|V|-1)}\sum_{i<j}{\frac{d_{ij}}{d_N(i,j)}}$.
\item Diamètre du réseau $\delta = \max_{ij}d_N(i,j)$
\end{itemize}

\paragraph{Covariance et correlation}

On s'intéressera à la matrice de covariance croisée $\Covb{\vec{M}}{\vec{G}}$ entre densité et réseau, estimée sur un jeu de $n$ réalisations à paramètres fixés $(\vec{M}\left[D(\vec{\alpha})\right],\vec{G}\left[N(\vec{\alpha})\right])_{1\leq i\leq n}$ par l'estimateur standard non-biaisé. On prend comme correlation associée la correlation de Pearson estimée de la même façon.



%%%%%%%%%%%%%%%%%%%%%%
\subsubsection{Implémentation}

Le couplage des modèles génératifs est effectué à la fois au niveau formel et au niveau opérationnel, c'est à dire qu'on fait interagir des implémentations indépendantes. Pour cela, le logiciel OpenMole~\cite{reuillon2013openmole} utilisé pour l'exploration intensive, offre le cadre idéal de par son langage modulaire permettant de construire des \emph{workflows} par composition de tâches à loisir et de les brancher sur divers plans d'expérience et sorties. Pour des raisons opérationnelles, le modèle de densité est implémenté en langage \texttt{scala} comme un \texttt{plugin} d'OpenMole, tandis que la génération de réseau est implémentée en langage basé-agent \texttt{NetLogo}~\cite{wilensky1999netlogo}, ce qui facilite l'exploration interactive et construction heuristique interactive. Le code source est disponible pour reproductibilité sur le dépôt du projet\footnote{à l'adresse \texttt{https://github.com/JusteRaimbault/CityNetwork/tree/master/Models/Synthetic}}.






%%%%%%%%%%%%%%
\begin{figure}

\begin{subfigure}[t]{0.35\linewidth}
\includegraphics[width=\textwidth]{figures/hist_crossCorMat_breaks30}
\caption{}
\end{subfigure}
\begin{subfigure}[t]{0.23\linewidth}
\vspace{-6.5cm}
\includegraphics[width=\textwidth]{figures/heatmap_maxAbsCor}\\
\includegraphics[width=\textwidth]{figures/heatmap_amplCor}
\caption{}
\end{subfigure}
\begin{subfigure}[t]{0.4\linewidth}
\includegraphics[width=\textwidth]{figures/pca_meanAbsCor_errorBars}
\caption{}
\end{subfigure}\\
\begin{subfigure}[t]{0.54\linewidth}
\includegraphics[width=\textwidth]{figures/pca_realDistCol_meanAbsCorSize_withSpecificPoints}
\caption{}
\end{subfigure}
\begin{subfigure}[t]{0.45\linewidth}
\vspace{-8.3cm}
   \includegraphics[width=0.49\textwidth]{figures/configs/1_param71861_seed0}
   \includegraphics[width=0.49\textwidth]{figures/configs/2_param71913_seed10}\\
   \includegraphics[width=0.49\textwidth]{figures/configs/3_param71918_seed0}
   \includegraphics[width=0.49\textwidth]{figures/configs/4_param71945_seed0}
   \caption{}
\end{subfigure}

\caption{\small\textbf{Exploration de l'espace des corrélations entre forme urbaine et réseau | } \textbf{(a)} Distribution des correlations croisées entre les vecteurs $\vec{M}$ des indicateurs morphologiques (dans l'ordre index de moran, distance moyenne, entropie, hiérarchie) et $\vec{N}$ des mesures de réseau (centralité, longueur moyenne, vitesse, diamètre). \textbf{(b)} Amplitude des corrélations, définie comme $a_{ij}=\max_k{\rho_{ij}^{(k)}}-\min_k{\rho_{ij}^{(k)}}$, et correlation absolue maximale, $c_{ij}=\max_k\left| \rho_{ij}^{k} \right|$. \textbf{(c)} Représentation des matrices dans un plan principal obtenu par Analyse en Composantes Principales sur la population des matrices (variances cumulées: PC1=38\%, PC2=68\%). Les barres d'erreur sont calculées initialement par des intervalles de confiance à 95\% sur chaque élément de la matrice (par une méthode de Fisher asymptotique standard), puis les bornes supérieures sont prises dans le plan principal. L'échelle de couleur donne la correlation absolue moyenne sur l'ensemble des variables. \textbf{(d)} Représentation dans le plan principal, l'échelle de couleur donnant la proximité aux données réelles définie comme $1 - \min_r \norm{\vec{M}-\vec{M}_r}$ où $\vec{M}_r$ est l'ensemble des mesures morphologiques réelles, la taille des points donne la correlation absolue moyenne. \textbf{(e)} Configurations obtenues sur les 4 points particuliers surlignés en (d), dans l'ordre de gauche à droite et de haut en bas. On reconnait des configurations de villes polycentriques (2 et 4), d'habitat rural diffus (3) et de zone à faible densité (1). Voir en appendice pour l'ensemble des valeurs des paramètres, indicateurs et correlations correspondants. Par exemple, $\bar{d}$ est très corrélé à $\bar{l},\bar{s}$ ($\simeq$0.8) pour (1) mais pas pour (3) alors qu'il s'agit pour les deux d'un environnement rural ; dans un cas urbain on a également beaucoup de variabilité : $\rho[\bar{d},\bar{c}]\simeq 0.34$ pour (4) mais $\simeq-0.41$ pour (2), ce qui s'explique par une plus forte prégnance de la hiérarchie gravitaire dans (2) $\gamma=3.9,k_h=0.7$ (pour (4), $\gamma=1.07,k_h=0.25$), alors que la configuration de densité est la même.}
\label{fig:densnwcor}
\end{figure}
%%%%%%%%%%%%%%






%%%%%%%%%%%%%%%%%%%%%%
\subsubsection{Résultats}

L'étude du modèle de densité seul est développée dans~\cite{raimbault2016calibration}. Il est notamment calibré sur les données de la grille européenne de densité, sur des zones de 50km de côté et de résolution 500m pour lesquelles les valeurs réelles des indicateurs ont été calculées pour l'ensemble de l'Europe. D'autre part, une exploration brutale du modèle permet d'estimer l'ensemble des sorties possibles dans des bornes raisonnables pour les paramètres (grossièrement $\alpha \in [0.5,2],N_G\in [500,3000], P_m \in [10^4,10^5],\beta\in [0,0.2], n_d \in \{ 1, \ldots , 4\}$). La réduction à un plan de l'espace des objectif par une Analyse en Composantes Principales (variance expliquée à deux composantes $\simeq 80\%$) permet d'isoler un nuage de points de sorties recouvrant assez fidèlement le nuage des points réels, ce qui veut dire que le modèle est capable de reproduire morphologiquement l'ensemble des configurations existantes.


% NOT NEEDED - TOO MUCH INFORMATION - ?
%%%%%%%%%%%%%%%%
%\begin{figure}
% figure : density example, exploration and calibration ?
%\end{figure}
%%%%%%%%%%%%%%%%

A densité donnée, l'exploration de l'espace des paramètres du modèle de réseau suggèrent une assez bonne flexibilité sur des indicateurs globaux $\vec{G}$, ainsi que de bonnes propriétés de convergence. Pour une étude du comportement précis, voir l'appendice donnant les regressions traduisant le comportement du modèle couplé. Dans le but d'illustrer la méthode de génération de données synthétiques, l'exploration a été orientée vers l'étude des correlations.

Etant donné la grande dimension relative de l'espace des paramètres, une exploration par grille exhaustive est impossible. On utilise un plan d'expérience par criblage (hypercube latin), avec les bornes indiquées ci-dessus pour $\vec{\alpha}_D$ et pour $\vec{\alpha}_N$, on a $N_C \in [50,120], r_g \in [1,100] , d_0 \in [0.1,10] , k_h \in [0,1] , \gamma \in [0.1,4],N_L\in [4,20]$. Concernant le nombre de réplications du modèle pour chaque valeur des paramètres, moins de 50 sont nécessaires pour obtenir sur les indicateurs des intervalles de confiance à 95\% de taille inférieure aux déviations standard. Pour les correlations, une centaine donne des IC (obtenus par méthode de Fisher) de taille moyenne 0.4, on fixe donc $n=80$ pour l'expérience. La figure~\ref{fig:densnwcor} donne le détail des résultats de l'exploration. On retiendra les résultats marquants suivants au regard de la génération de données synthétiques corrélées :
\begin{itemize}
\item les distributions empiriques des coefficients de correlations entre indicateurs de forme et indicateurs de réseaux ne sont pas simples, pouvant être bimodales (par exemple $\rho_{46}=\rho[r,\bar{l}]$ entre l'index de Moran et le chemin moyen).
\item On arrive à générer un assez haut niveau de correlation pour l'ensemble des indicateurs, la correlation absolue maximale variant entre 0.6 et 0.9 ; l'amplitude varie quant à elle entre 0.9 et 1.6, ce qui permet un large spectre de valeurs. L'espace couvert dans un plan principal a une étendue certaine mais n'est pas uniforme : on ne peut pas moduler à loisir n'importe quel coefficients, ceux-ci étant liés par les processus de génération sous-jacent. Une étude plus fine aux ordres suivants (correlation des correlations) serait nécessaire pour cerner exactement la latitude dans la génération.
\item les points les plus corrélés en moyenne sont également ceux les plus proches des données réelles, ce qui confirme l'intuition d'une forte interdépendance en réalité.
\item Des exemples concrets pris sur des points particuliers distants dans le plan principal montre que des configurations de densité proches peuvent présenter des profils de correlations très différents.
\end{itemize}



%%%%%%%%%%%%%%
%\begin{table}
%regression analysis of param influence on correlations
%  -> Appendice.
%\end{table}
%%%%%%%%%%%%%%





\subsubsection{Extensions possibles}

Il est possible de raffiner cette étude en étendant la méthode de contrôle des correlations. La connaissance très fine du comportement de $N$ (distribution statistiques sur une grille fine de l'espace des paramètres) conditionnée à $D$ devrait permettre de déterminer exhaustivement $N^{<-1>} | D$ et avoir plus de latitude dans la génération des correlations. On pourra également appliquer des algorithmes spécifiques d'exploration pour essayer atteindre des configurations exceptionnelles réalisant un niveau de corrélation voulu, ou au moins pour découvrir l'espace des correlations atteignables par la méthode de génération~\cite{10.1371/journal.pone.0138212}.





%%%%%%%%%%%%%%%%%%%%%%
\section{Discussion}
%%%%%%%%%%%%%%%%%%%%%%



%%%%%%%%%%%%%%%%%%%%%%
\subsection*{Positionnement}

% données hybrides au centre de la démarche d'exploration de modèle, analyse de sensitivité etc.

Notre démarche s'inscrit dans un cadre épistémologique particulier. En effet, d'une part la volonté de multi-disciplinarité et d'autre part l'importance de la composante empirique couplée aux méthodes d'exploration computationelles, en font une approche typique des sciences de la complexité, comme le rappelle la structure de la feuille de route pour les systèmes complexes~\cite{2009arXiv0907.2221B} qui croise des grandes questions transversales aux disciplines à une intégration verticale de celles-ci, qui implique la construction de modèles multi-échelles hétérogènes présentant souvent les aspects précédent. Le croisement de connaissances empiriques issues de la fouille de données avec celles issues de la simulation est souvent central dans leur conception ou leur exploration, et les résultats présentés ici en sont un exemple typique pour le cas de l'exploration.



%%%%%%%%%%%%%%%%%%%%%%
\subsection*{Applications directes}


En partant du deuxième exemple, qui s'est arrêté à la génération des données synthétiques, on peut proposer des pistes d'application directe qui donneront un aperçu de l'éventail des possibilités.

\begin{itemize}
\item La calibration de la composante de génération de réseau, à densité donnée, sur des données réelle de réseau de transport (typiquement routier vu les formes heuristiques obtenues, il devrait par exemple être aisé d'utiliser les données ouvertes d'OpenStreetMap\footnote{\texttt{https://www.openstreetmap.org}} qui sont de qualité raisonnable pour l'Europe, du moins pour la France~\cite{girres2010quality}, avec toutefois des ajustements à faire sur le modèle pour supprimer les effets de bord du à sa structure, par exemple en le faisant générer sur une surface étendue pour ne garder qu'une zone centrale sur laquelle la calibration aurait lieu) permettrait en théorie d'isoler un jeu de paramètres représentant fidèlement des situations existantes à la fois pour la forme urbaine et la forme du réseau. Il serait alors possible de dériver une ``correlation théorique'' pour celles-ci, étant donné qu'une correlation empirique n'est en théorie pas calculable puisqu'une seule instance des processus stochastiques est observée. Vu la non-ergodicité des système urbains~\cite{pumain2012urban}, il y a de fortes chances pour que ces processus soient différents d'une zone géographique à l'autre (ou selon un autre point de vue qu'ils soient dans un autre état des meta-paramètres, dans un autre régime) et que leur interprétation en tant que réalisations d'un même processus stochastique n'ait aucun sens, entrainant l'impossibilité du calcul des covariations. En attribuant un jeu de données synthétiques similaire à une situation donnée, on serait capable de calculer une sorte de \emph{correlation intrinsèque} propre à la situation, qui émerge en fait en réalité des interdépendances temporelles des composantes. Connaitre celle-ci renseigne alors sur ces interdépendances, et donc sur les relations entre réseaux et territoires.
\item Comme déjà évoqué, la plupart des modèles de simulation nécessitent un état initial, généré artificiellement à partir du moment où la paramétrisation n'est pas effectuée totalement à partir de données réelles. Une analyse de sensibilité avancée du modèle implique alors un contrôle sur les paramètres de génération du jeu de données synthétique, vu comme méta-paramètre du modèle~\cite{cottineau2015revisiting}. Dans le cas d'une analyse statistique des sorties du modèle, on est alors capable d'effectuer un contrôle statistique au second ordre.
\item On a étudié des processus stochastiques dans le premier exemple, au sens de séries temporelles aléatoires, alors que le temps ne jouait pas de rôle dans le second. On peut suggérer un couplage fort entre les deux composantes du modèle (ou la construction d'un modèle intégré) et observer les indicateurs et correlations à différents pas de temps de la génération. Dans le cas d'une dynamique, de par les rétroactions, on a nécessairement des effets de propagation et donc l'existence d'interdépendances décalées dans l'espace et le temps~\cite{pigozzi1980interurban}, étendant le domaine d'étude vers une meilleure compréhensions des correlations dynamiques.
\end{itemize}



%%%%%%%%%%%%%%%%%%%%%%
\subsection*{Généralisation}

On s'est limité au contrôle des premiers et second moments des données générées, mais il est possible d'imaginer une généralisation théorique permettant le contrôle des moments à un ordre arbitraire. Toutefois, la difficulté de génération dans un cas concret complexe, comme le montre l'exemple géographique, questionne la possibilité de contrôle aux ordres supérieurs tout en gardant un modèle à la structure cohérente au au nombre de paramètres relativement faibles. Par contre, l'étude de structures de dépendances non-linéaires comme celles utilisées dans~\cite{chicheportiche2013nested} est une piste de développement intéressante.

%%%%%%%%%%%%%%%%%%%%%%
%\subsection*{Autres domaines potentiels d'application}
% ideas of other fields where the generation can happen.
% -> not necessary, suggested in intro ?



%%%%%%%%%%%%%%%%%%%%%%
\section{Conclusion}
%%%%%%%%%%%%%%%%%%%%%%


On a ainsi proposé une méthode abstraite de génération de données synthétiques corrélées à un niveau contrôlé. Son implémentation partielle dans deux domaines très différents montre sa flexibilité et l'éventail des applications potentielles. De manière générale, il est essentiel de généraliser de telles pratiques de validation systématique de modèles par étude statistique, en particulier pour les modèles agents pour lesquels la question de la validation reste encore relativement ouverte.




%%%%%%%%%%%%%%%%%%%%
%% Biblio
%%%%%%%%%%%%%%%%%%%%
\footnotesize

\bibliographystyle{apalike}
\bibliography{/Users/Juste/Documents/ComplexSystems/CityNetwork/Biblio/Bibtex/CityNetwork,biblio/biblio}





\newpage

\normalsize




%%%%%%%%%%%%%%%%%%%%
\section*{Appendice : Comportement du modèle couplé, analyse statistique}

Les tableaux suivants donnent les résultats quantitatifs de l'analyse du modèle territorial couplé. Il s'agit de régression linéaires simples pour les différents indicateurs ainsi que pour les correlations. On donne à chaque fois les déviations standard ainsi que la significativité des coefficients, codée par la \emph{p-valeur} : (***) $p\sim 0$, (**) $p<0.001$, (*) $p<0.01$, (*) $p<0.05$.

\subsection*{Indicateurs}

\begin{center}


\begin{tabular}{|c|p{3.7cm}|p{3.7cm}|p{3.7cm}|p{3.7cm}|}
 \hline
&BW&Pathlength&relspeed&diameter\\\hline
R2&0.6098&0.638 &0.7049 &0.5855\\\hline
(Intercept)&2.106e-01$\pm$ 4.728e-03 (***)&1.0939160$\pm$ 0.0514692 (***)&6.211e-01$\pm$ 9.722e-03 (***)&2.4956084$\pm$ 0.1172324 (***)\\
$\alpha$&-1.127e-02$\pm$ 2.012e-03 (***)&-0.530583$\pm$ 0.021907 (***)&8.469e-02$\pm$ 4.138e-03 (***)&-1.0366776$\pm$ 0.04989 (***)\\
$\beta$&9.430e-03$\pm$ 6.803e-03 ()&0.1738349$\pm$ 0.0740571 (*)&-6.516e-02$\pm$ 1.399e-02 (***)&0.2937746$\pm$ 0.1686813 (.)\\
$n_d$&6.786e-04$\pm$ 6.534e-04 ()&0.0048631$\pm$ 0.0071129 ()&-4.046e-03$\pm$ 1.344e-03 (**)&0.0003039$\pm$ 0.0162012 ()\\
$N_C$&-3.005e-04$\pm$ 2.887e-05 (***)&0.0012026$\pm$ 0.0003143 (***)&-1.214e-03$\pm$ 5.937e-05 (***)&0.0040004$\pm$ 0.0007159 (***)\\
$N_G$&4.800e-02$\pm$ 1.348e-02 (***)&1.4969583$\pm$ 0.1467356 (***)&-1.400e-01$\pm$ 2.772e-02 (***)&3.3021779$\pm$ 0.3342227 (***)\\
$\gamma$&4.615e-03$\pm$ 4.997e-04 (***)&0.0132129$\pm$ 0.0054394 (*)&-7.990e-03$\pm$ 1.027e-03 (***)&0.0389784$\pm$ 0.0123895 (**)\\
$d_0$&2.743e-04$\pm$ 1.971e-04 ()&-0.0029289$\pm$ 0.0021453 ()&-5.688e-04$\pm$ 4.052e-04 ()&-0.0075661$\pm$ 0.0048863 ()\\
$r_g$&-2.726e-05$\pm$ 2.038e-05 ()&0.0001532$\pm$ 0.0002219 ()&5.329e-05$\pm$ 4.191e-05 ()&0.0002358$\pm$ 0.0005054 ()\\
$k_h$&-1.035e-02$\pm$ 1.952e-03 (***)&-0.095491$\pm$ 0.021250 (***)&1.992e-02$\pm$ 4.014e-03 (***)&-0.227141$\pm$ 0.048402 (***)\\
$N_L$&-2.390e-03$\pm$ 1.243e-04 (***)&-0.0021779$\pm$ 0.0013533 ()&1.287e-03$\pm$ 2.556e-04 (***)&-0.0044759$\pm$ 0.0030824 ()\\

 \hline
\end{tabular}

\bigskip

\hspace{-0.4cm}

\begin{tabular}{|c|p{3.7cm}|p{3.7cm}|p{3.7cm}|p{3.7cm}|}
 \hline
&moran&distance&entropy&slope\\\hline
R2&0.7762&0.4753&0.5516&0.6587\\\hline
(Intercept)&1.106e-02$\pm$ 1.348e-02 ()&1.158e+00$\pm$ 3.862e-02 (***)&1.135e+00$\pm$ 3.024e-02 (***)&0.0839654$\pm$ 0.0706085 ()\\
$\alpha$&-1.631e-02$\pm$ 5.739e-03 (**)&-2.549e-01$\pm$ 1.644e-02 (***)&-2.396e-01$\pm$ 1.287e-02 (***)&-0.7543377$\pm$ 0.0300528 (***)\\
$\beta$&5.009e-01$\pm$ 1.940e-02 (***)&-2.497e-02$\pm$ 5.557e-02 ()&4.580e-01$\pm$ 4.351e-02 (***)&1.0974897$\pm$ 0.1015959 (***)\\
$n_d$&2.850e-02$\pm$ 1.863e-03 (***)&-1.069e-02$\pm$ 5.337e-03 (*)&1.630e-02$\pm$ 4.179e-03 (***)&0.0322148$\pm$ 0.0097579 (**)\\
$N_C$&-8.494e-05$\pm$ 8.234e-05 ()&-5.408e-05$\pm$ 2.358e-04 ()&-5.802e-05$\pm$ 1.847e-04 ()&0.0001797$\pm$ 0.0004312 ()\\
$N_G$&-7.710e-01$\pm$ 3.844e-02 (***)&1.222e+00$\pm$ 1.101e-01 (***)&4.276e-01$\pm$ 8.622e-02 (***)&1.0692012$\pm$ 0.2013007 (***)\\
$\gamma$&6.214e-04$\pm$ 1.425e-03 ()&2.219e-03$\pm$ 4.081e-03 ()&2.026e-03$\pm$ 3.196e-03 ()&-0.0015095$\pm$ 0.0074621 ()\\
$d_0$&2.457e-04$\pm$ 5.620e-04 ()&-3.896e-03$\pm$ 1.610e-03 (*)&-2.459e-03$\pm$ 1.261e-03 (.)&-0.0046572$\pm$ 0.0029430 ()\\
$r_g$&4.413e-05$\pm$ 5.813e-05 ()&1.048e-04$\pm$ 1.665e-04 ()&1.069e-04$\pm$ 1.304e-04 ()&0.0003902$\pm$ 0.0003044 ()\\
$k_h$&6.379e-03$\pm$ 5.567e-03 ()&-5.308e-02$\pm$ 1.595e-02 (***)&-3.215e-02$\pm$ 1.249e-02 (*)&-0.0407186$\pm$ 0.0291524 ()\\
$N_L$&5.183e-04$\pm$ 3.545e-04 ()&-4.340e-04$\pm$ 1.015e-03 ()&8.870e-06$\pm$ 7.952e-04 ()&0.0003567$\pm$ 0.0018565 ()\\
 \hline
\end{tabular}


\end{center}


\vspace{0.5cm}


\subsection*{Correlations}

\subsubsection*{Auto-correlations}

\begin{center}

\begin{tabular}{|c|p{3.7cm}|p{3.7cm}|p{3.7cm}|}
 \hline
&$\rho[r,\bar{d}]$&$\rho[r,e]$&$\rho[r,s]$\\\hline
R2&0.5774&0.5093&0.5483\\\hline
(Intercept)&1.226986$\pm$0.034715 (***)&-1.104021$\pm$0.039630 (***)&1.21541$\pm$0.04161 (***)\\
$\alpha$&-0.497699$\pm$0.021858 (***)&0.472135$\pm$0.024953 (***)&-0.55056$\pm$0.02620 (***)\\
$\beta$&0.334188$\pm$0.073967 (***)&-0.606526$\pm$0.084439 (***)&0.46942$\pm$0.08866 (***)\\
$n_d$&0.008372$\pm$0.007066 ()&-0.023801$\pm$0.008066 (**)&0.01310$\pm$0.00847 ()\\
$N_G$&0.665346$\pm$0.146472 (***)&-0.483421$\pm$0.167209 (**)&0.99527$\pm$0.17557 (***)\\
\hline
\end{tabular}

\bigskip

\begin{tabular}{|c|p{3.7cm}|p{3.7cm}|p{3.7cm}|}
 \hline
&$\rho[\bar{d},e]$&$\rho[\bar{d},s]$&$\rho[e,s]$\\\hline
R2&0.5141&0.5383&0.5449\\\hline
(Intercept)&-1.67032$\pm$0.08598 (***)&1.043888$\pm$0.016697 (***)&-1.50417$\pm$0.07318 (***)\\
$\alpha$&1.04667$\pm$0.05414 (***)&-0.218800$\pm$0.010513 (***)&0.95046$\pm$0.04608 (***)\\
$\beta$&-0.91864$\pm$0.18320 (***)&0.153942$\pm$0.035577 (***)&-0.80453$\pm$0.15592 (***)\\
$n_d$&-0.04144$\pm$0.01750 (*)&0.007833$\pm$0.003399 (*)&-0.03624$\pm$0.01489 (*)\\
$N_G$&-2.23787$\pm$0.36278 (***)&0.355478$\pm$0.070450 (***)&-2.00208$\pm$0.30875 (***)\\
 \hline
\end{tabular}



\bigskip

\begin{tabular}{|c|p{3.7cm}|p{3.7cm}|p{3.7cm}|}
 \hline
&$\rho [\bar{c},\bar{l}]$&$\rho [\bar{c},\bar{s}]$&$\rho [\bar{c},\delta]$\\\hline
R2&0.5892&0.5084&0.581\\\hline
(Intercept)&1.1594972$\pm$0.0508300 (***)&-1.099e+00$\pm$5.891e-02 (***)&1.2163505$\pm$0.0595172 (***)\\
$\alpha$&-0.4989055$\pm$0.0216346 (***)&4.722e-01$\pm$2.507e-02 (***)&-0.5547464$\pm$0.0253321 (***)\\
$\beta$&0.3285271$\pm$0.0731373 (***)&-6.064e-01$\pm$8.476e-02 (***)&0.4737039$\pm$0.0856371 (***)\\
$n_d$&0.0079990$\pm$0.0070246 ()&-2.399e-02$\pm$8.141e-03 (**)&0.0123499$\pm$0.0082251 ()\\
$N_C$&0.6966627$\pm$0.1449132 (***)&-4.869e-01$\pm$1.679e-01 (**)&1.0369074$\pm$0.1696801 (***)\\
$N_G$&0.0011097$\pm$0.0003104 (***)&-4.617e-04$\pm$3.597e-04 ()&0.0011038$\pm$0.0003634 (**)\\
$\gamma$&0.0026346$\pm$0.0053719 ()&3.586e-03$\pm$6.225e-03 ()&0.0074892$\pm$0.0062900 ()\\
$d_0$&-0.0009406$\pm$0.0021186 ()&9.630e-05$\pm$2.455e-03 ()&-0.0009453$\pm$0.0024807 ()\\
$r_g$&0.0001819$\pm$0.0002191 ()&9.951e-05$\pm$2.539e-04 ()&0.0002772$\pm$0.0002566 ()\\
$k_h$&-0.0298275$\pm$0.0209863 ()&4.081e-02$\pm$2.432e-02 (.)&-0.0833935$\pm$0.0245731 (***)\\
$N_L$&-0.0015975$\pm$0.0013365 ()&1.384e-04$\pm$1.549e-03 ()&-0.0060473$\pm$0.0015649 (***)\\
 \hline
\end{tabular}

\bigskip

\begin{tabular}{|c|p{3.7cm}|p{3.7cm}|p{3.7cm}|}
 \hline
&$\rho [\bar{l},\bar{s}]$&$\rho [\bar{l},\delta]$&$\rho [\bar{s},\delta]$\\\hline
R2&0.522&0.5494&0.5546\\\hline
(Intercept)&-1.768e+00$\pm$1.266e-01 (***)&1.073e+00$\pm$2.450e-02 (***)&-1.5699072$\pm$0.1075144 (***)\\
$\alpha$&1.043e+00$\pm$5.390e-02 (***)&-2.210e-01$\pm$1.043e-02 (***)&0.9481902$\pm$0.0457610 (***)\\
$\beta$&-9.441e-01$\pm$1.822e-01 (***)&1.550e-01$\pm$3.525e-02 (***)&-0.8269920$\pm$0.1546984 (***)\\
$n_d$&-3.719e-02$\pm$1.750e-02 (*)&7.778e-03$\pm$3.385e-03 (*)&-0.0324681$\pm$0.0148582 (*)\\
$N_C$&-2.229e+00$\pm$3.610e-01 (***)&3.579e-01$\pm$6.984e-02 (***)&-2.0137775$\pm$0.3065172 (***)\\
$N_G$&8.575e-05$\pm$7.734e-04 ()&2.392e-05$\pm$1.496e-04 ()&-0.0002524$\pm$0.0006565 ()\\
$\gamma$&-4.901e-03$\pm$1.338e-02 ()&4.571e-03$\pm$2.589e-03 (.)&-0.0032340$\pm$0.0113624 ()\\
$d_0$&4.492e-03$\pm$5.279e-03 ()&-6.360e-04$\pm$1.021e-03 ()&0.0036384$\pm$0.0044813 ()\\
$r_g$&-3.880e-04$\pm$5.459e-04 ()&-3.725e-05$\pm$1.056e-04 ()&-0.0005394$\pm$0.0004635 ()\\
$k_h$&1.728e-01$\pm$5.229e-02 (**)&-1.419e-02$\pm$1.011e-02 ()&0.1492597$\pm$0.0443898 (***)\\
$N_L$&9.444e-04$\pm$3.330e-03 ()&-2.201e-03$\pm$6.441e-04 (***)&0.0022415$\pm$0.0028269 ()\\
 \hline
\end{tabular}


\end{center}



\subsubsection*{Correlations croisées}


\begin{center}

\begin{tabular}{|c|p{3.7cm}|p{3.7cm}|p{3.7cm}|p{3.7cm}|}
 \hline
&$\rho[r,\bar{c}]$&$\rho[r,\bar{l}]$&$\rho[r,\bar{s}]$&$\rho[r,\delta]$\\\hline
R2&0.08052&0.1734&0.174&0.1187\\\hline
(Intercept)&-0.0262848$\pm$0.0582459 ()&-1.830e-01$\pm$5.801e-02 (**)&-2.800e-01$\pm$6.381e-02 (***)&-0.0396938$\pm$0.0609953 ()\\
$\alpha$&0.0425576$\pm$0.0247910 (.)&2.239e-01$\pm$2.469e-02 (***)&2.369e-01$\pm$2.716e-02 (***)&-0.0546067$\pm$0.0259612 (*)\\
$\beta$&-0.3076801$\pm$0.0838078 (***)&2.970e-02$\pm$8.347e-02 ()&2.487e-02$\pm$9.182e-02 ()&0.4311665$\pm$0.0877639 (***)\\
$n_d$&-0.0225112$\pm$0.0080494 (**)&-2.231e-03$\pm$8.017e-03 ()&8.874e-03$\pm$8.819e-03 ()&0.0418159$\pm$0.0084294 (***)\\
$N_C$&0.0001178$\pm$0.0003557 ()&2.270e-04$\pm$3.542e-04 ()&3.001e-04$\pm$3.897e-04 ()&-0.0004806$\pm$0.0003725 ()\\
$N_G$&0.4455534$\pm$0.1660556 (**)&-5.238e-01$\pm$1.654e-01 (**)&-6.272e-01$\pm$1.819e-01 (***)&-0.2879462$\pm$0.1738941 (.)\\
$\gamma$&-0.0153085$\pm$0.0061556 (*)&4.880e-03$\pm$6.131e-03 ()&-3.259e-03$\pm$6.744e-03 ()&0.0076671$\pm$0.0064462 ()\\
$d_0$&0.0019859$\pm$0.0024277 ()&2.813e-03$\pm$2.418e-03 ()&-3.898e-05$\pm$2.660e-03 ()&0.0009132$\pm$0.0025423 ()\\
$r_g$&0.0001959$\pm$0.0002511 ()&5.983e-05$\pm$2.501e-04 ()&9.584e-05$\pm$2.751e-04 ()&-0.0001899$\pm$0.0002629 ()\\
$k_h$&0.0412129$\pm$0.0240482 (.)&2.120e-02$\pm$2.395e-02 ()&5.927e-02$\pm$2.635e-02 (*)&-0.0178174$\pm$0.0251834 ()\\
$N_L$&-0.0011826$\pm$0.0015315 ()&5.317e-05$\pm$1.525e-03 ()&7.111e-04$\pm$1.678e-03 ()&0.0006752$\pm$0.0016037 ()\\
\hline
\end{tabular}

\bigskip

\begin{tabular}{|c|p{3.7cm}|p{3.7cm}|p{3.7cm}|p{3.7cm}|}
 \hline
&$\rho[\bar{d},\bar{c}]$&$\rho[\bar{d},\bar{l}]$&$\rho[\bar{d},\bar{s}]$&$\rho[\bar{d},\delta]$\\\hline
R2&0.09957&0.726&0.6635&0.2736\\\hline
(Intercept)&-7.875e-03$\pm$1.000e-01 ()&-5.920e-01$\pm$5.909e-02 (***)&-0.8352368$\pm$0.0769938 (***)&-0.2136136$\pm$0.1134114 (.)\\
$\alpha$&1.143e-02$\pm$4.257e-02 ()&7.915e-01$\pm$2.515e-02 (***)&0.8797555$\pm$0.0327706 (***)&-0.0164318$\pm$0.0482709 ()\\
$\beta$&-6.376e-01$\pm$1.439e-01 (***)&-9.049e-02$\pm$8.502e-02 ()&0.0652982$\pm$0.1107835 ()&1.6099651$\pm$0.1631834 (***)\\
$n_d$&-4.068e-02$\pm$1.382e-02 (**)&-2.081e-03$\pm$8.166e-03 ()&0.0208547$\pm$0.0106403 (.)&0.0904758$\pm$0.0156732 (***)\\
$N_C$&3.524e-04$\pm$6.108e-04 ()&5.379e-05$\pm$3.608e-04 ()&0.0001255$\pm$0.0004702 ()&-0.0006020$\pm$0.0006926 ()\\
$N_G$&1.228e+00$\pm$2.851e-01 (***)&-1.668e+00$\pm$1.685e-01 (***)&-1.9878616$\pm$0.2195048 (***)&-1.1801054$\pm$0.3233292 (***)\\
$\gamma$&6.252e-03$\pm$1.057e-02 ()&-3.111e-03$\pm$6.244e-03 ()&-0.0036648$\pm$0.0081369 ()&0.0051939$\pm$0.0119857 ()\\
$d_0$&8.292e-04$\pm$4.169e-03 ()&2.533e-03$\pm$2.463e-03 ()&0.0003107$\pm$0.0032092 ()&-0.0029674$\pm$0.0047271 ()\\
$r_g$&4.879e-05$\pm$4.311e-04 ()&8.729e-05$\pm$2.547e-04 ()&-0.0001203$\pm$0.0003319 ()&0.0002683$\pm$0.0004889 ()\\
$k_h$&3.077e-02$\pm$4.129e-02 ()&2.040e-02$\pm$2.440e-02 ()&0.0403760$\pm$0.0317887 ()&-0.0958810$\pm$0.0468245 (*)\\
$N_L$&-5.190e-03$\pm$2.630e-03 (*)&1.228e-03$\pm$1.554e-03 ()&0.0017721$\pm$0.0020244 ()&0.0011045$\pm$0.0029819 ()\\
\hline
\end{tabular}

\bigskip

\begin{tabular}{|c|p{3.7cm}|p{3.7cm}|p{3.7cm}|p{3.7cm}|}
 \hline
&$\rho[e,\bar{c}]$&$\rho[e,\bar{l}]$&$\rho[e,\bar{s}]$&$\rho[e,\delta]$\\\hline
R2&0.08321&0.1137&0.04931&0.2212\\\hline
(Intercept)&1.229e-01$\pm$6.394e-02 (.)&-0.2235482$\pm$0.0796783 (**)&-1.069e-01$\pm$9.653e-02 ()&0.2851561$\pm$0.0647508 (***)\\
$\alpha$&-1.406e-01$\pm$2.721e-02 (***)&0.2003760$\pm$0.0339132 (***)&1.098e-01$\pm$4.109e-02 (**)&-0.2898858$\pm$0.0275597 (***)\\
$\beta$&3.474e-01$\pm$9.200e-02 (***)&-0.3847257$\pm$0.1146461 (***)&-4.382e-01$\pm$1.389e-01 (**)&0.1606966$\pm$0.0931675 (.)\\
$n_d$&1.386e-02$\pm$8.836e-03 ()&-0.0089831$\pm$0.0110113 ()&-2.277e-02$\pm$1.334e-02 (.)&-0.0058289$\pm$0.0089484 ()\\
$N_C$&-2.203e-04$\pm$3.904e-04 ()&0.0004494$\pm$0.0004866 ()&4.578e-04$\pm$5.895e-04 ()&0.0001080$\pm$0.0003954 ()\\
$N_G$&9.209e-02$\pm$1.823e-01 ()&-0.5779639$\pm$0.2271581 (*)&-4.211e-01$\pm$2.752e-01 ()&0.6379269$\pm$0.1846008 (***)\\
$\gamma$&6.198e-03$\pm$6.757e-03 ()&0.0024670$\pm$0.0084206 ()&3.531e-03$\pm$1.020e-02 ()&-0.0024161$\pm$0.0068431 ()\\
$d_0$&-1.465e-03$\pm$2.665e-03 ()&-0.0024864$\pm$0.0033211 ()&1.782e-03$\pm$4.023e-03 ()&-0.0017413$\pm$0.0026989 ()\\
$r_g$&-7.161e-05$\pm$2.756e-04 ()&-0.0002354$\pm$0.0003435 ()&-4.484e-04$\pm$4.161e-04 ()&-0.0001984$\pm$0.0002791 ()\\
$k_h$&-1.346e-02$\pm$2.640e-02 ()&0.0841865$\pm$0.0328970 (*)&9.604e-02$\pm$3.985e-02 (*)&-0.0061427$\pm$0.0267339 ()\\
$N_L$&9.124e-04$\pm$1.681e-03 ()&-0.0011866$\pm$0.0020950 ()&-8.891e-05$\pm$2.538e-03 ()&-0.0024391$\pm$0.0017025 ()\\
\hline
\end{tabular}

\bigskip

\begin{tabular}{|c|p{3.7cm}|p{3.7cm}|p{3.7cm}|p{3.7cm}|}
 \hline
&$\rho[s,\bar{c}]$&$\rho[s,\bar{l}]$&$\rho[s,\bar{s}]$&$\rho[s,\delta]$\\\hline
R2&0.05977&0.6849&0.5995&0.2071\\\hline
(Intercept)&6.096e-02$\pm$8.136e-02 ()&-0.5291554$\pm$0.0585027 (***)&-6.684e-01$\pm$7.185e-02 (***)&-0.1327534$\pm$0.1006121 ()\\
$\alpha$&-7.487e-02$\pm$3.463e-02 (*)&0.7027361$\pm$0.0249003 (***)&7.155e-01$\pm$3.058e-02 (***)&-0.0529688$\pm$0.0428232 ()\\
$\beta$&-2.238e-01$\pm$1.171e-01 (.)&-0.1743332$\pm$0.0841774 (*)&-1.466e-01$\pm$1.034e-01 ()&1.1937758$\pm$0.1447669 (***)\\
$n_d$&-1.854e-02$\pm$1.124e-02 (.)&-0.0047797$\pm$0.0080849 ()&1.271e-02$\pm$9.929e-03 ()&0.0711684$\pm$0.0139043 (***)\\
$N_C$&2.594e-04$\pm$4.968e-04 ()&-0.0004555$\pm$0.0003573 ()&-5.483e-05$\pm$4.388e-04 ()&-0.0004681$\pm$0.0006144 ()\\
$N_G$&9.746e-01$\pm$2.319e-01 (***)&-1.5677015$\pm$0.1667878 (***)&-1.554e+00$\pm$2.048e-01 (***)&-0.7514017$\pm$0.2868391 (**)\\
$\gamma$&2.456e-03$\pm$8.598e-03 ()&0.0023678$\pm$0.0061827 ()&3.711e-03$\pm$7.593e-03 ()&0.0089657$\pm$0.0106330 ()\\
$d_0$&-1.572e-03$\pm$3.391e-03 ()&0.0047638$\pm$0.0024384 (.)&1.066e-03$\pm$2.995e-03 ()&-0.0017588$\pm$0.0041936 ()\\
$r_g$&5.791e-05$\pm$3.507e-04 ()&-0.0003370$\pm$0.0002522 ()&-4.783e-04$\pm$3.097e-04 ()&0.0000946$\pm$0.0004337 ()\\
$k_h$&3.367e-02$\pm$3.359e-02 ()&0.0492108$\pm$0.0241542 (*)&6.732e-02$\pm$2.966e-02 (*)&-0.0728540$\pm$0.0415401 (.)\\
$N_L$&-5.539e-03$\pm$2.139e-03 (**)&0.0017712$\pm$0.0015382 ()&9.329e-04$\pm$1.889e-03 ()&-0.0004517$\pm$0.0026454 ()\\
\hline
\end{tabular}

\bigskip

\begin{tabular}{|c|p{3.7cm}|p{3.7cm}|p{3.7cm}|p{3.7cm}|}
 \hline
&$\rho[e,\bar{c}]$&$\rho[e,\bar{l}]$&$\rho[e,\bar{s}]$&$\rho[e,\delta]$\\\hline
R2&0.08321&0.1137&0.04931&0.2212\\\hline
(Intercept)&1.229e-01$\pm$6.394e-02 (.)&-0.2235482$\pm$0.0796783 (**)&-1.069e-01$\pm$9.653e-02 ()&0.2851561$\pm$0.0647508 (***)\\
$\alpha$&-1.406e-01$\pm$2.721e-02 (***)&0.2003760$\pm$0.0339132 (***)&1.098e-01$\pm$4.109e-02 (**)&-0.2898858$\pm$0.02756 (***)\\
$\beta$&3.474e-01$\pm$9.200e-02 (***)&-0.3847257$\pm$0.1146461 (***)&-4.382e-01$\pm$1.389e-01 (**)&0.1606966$\pm$0.0931675 (.)\\
$n_d$&1.386e-02$\pm$8.836e-03 ()&-0.0089831$\pm$0.0110113 ()&-2.277e-02$\pm$1.334e-02 (.)&-0.0058289$\pm$0.0089484 ()\\
$N_C$&-2.203e-04$\pm$3.904e-04 ()&0.0004494$\pm$0.0004866 ()&4.578e-04$\pm$5.895e-04 ()&0.0001080$\pm$0.0003954 ()\\
$N_G$&9.209e-02$\pm$1.823e-01 ()&-0.5779639$\pm$0.2271581 (*)&-4.211e-01$\pm$2.752e-01 ()&0.6379269$\pm$0.1846008 (***)\\
$\gamma$&6.198e-03$\pm$6.757e-03 ()&0.0024670$\pm$0.0084206 ()&3.531e-03$\pm$1.020e-02 ()&-0.0024161$\pm$0.0068431 ()\\
$d_0$&-1.465e-03$\pm$2.665e-03 ()&-0.0024864$\pm$0.0033211 ()&1.782e-03$\pm$4.023e-03 ()&-0.0017413$\pm$0.0026989 ()\\
$r_g$&-7.161e-05$\pm$2.756e-04 ()&-0.0002354$\pm$0.0003435 ()&-4.484e-04$\pm$4.161e-04 ()&-0.0001984$\pm$0.0002791 ()\\
$k_h$&-1.346e-02$\pm$2.640e-02 ()&0.0841865$\pm$0.0328970 (*)&9.604e-02$\pm$3.985e-02 (*)&-0.0061427$\pm$0.0267339 ()\\
$N_L$&9.124e-04$\pm$1.681e-03 ()&-0.0011866$\pm$0.0020950 ()&-8.891e-05$\pm$2.538e-03 ()&-0.0024391$\pm$0.0017025 ()\\
\hline
\end{tabular}

\end{center}




\newpage

\section*{Appendice : Configurations particulières}

Les tableaux suivants résument les valeurs des paramètres, indicateurs et correlations pour les 4 configurations présentées en figure~\ref{fig:densnwcor}. On donne les déviations standard pour les indicateurs et les intervalles de confiance à 95\% pour les correlations.

\begin{center}

\begin{tabular}{|c|p{3.7cm}|p{3.7cm}|p{3.7cm}|p{3.7cm}|}
\hline
Configuration&(1)&(2)&(3)&(4)\\\hline
$\alpha$&1.516501e+00&1.555684e+00&1.174289e+00&1.323203e+00\\
$\beta$&4.522689e-02&2.369011e-01&2.878834e-02&2.457331e-02\\
$n_d$&1.523121e+00&2.447648e+00&1.005678e+00&1.682232e+00\\
$N_C$&9.841719e+01&7.605296e+01&1.176334e+02&9.128398e+01\\
$N_G$&2.712064e-02&2.388306e-02&5.093294e-02&1.854230e-02\\
$\gamma$&1.618555e+00&3.917224e+00&2.136885e+00&1.070508e+00\\
$d_0$&1.300443e+00&9.939404e+00&2.947665e+00&1.324972e+00\\
$r_g$&8.576246e+01&8.394425e+01&8.232540e+01&7.438451e+01\\
$k_h$&2.200645e-01&7.177931e-01&8.064035e-01&2.510558e-02\\
$N_L$&1.846435e+01&1.803941e+01&9.706505e+00&1.190123e+01\\
$P_m$&3.657749e+04&9.526370e+04&4.923410e+04&6.140057e+04\\\hline
$\bar{c}$&1.433701e-01$\pm$1.410993e-02&0.1450925$\pm$0.01732669&0.15007478$\pm$0.016914043&0.15753385$\pm$0.016885300\\
$\bar{l}$&1.902390e-01$\pm$1.490413e-01&0.5264323$\pm$0.10235672&0.78084633$\pm$0.091161648&0.61213918$\pm$0.144466259\\
$\bar{s}$&6.567478e-01$\pm$6.161807e-02&0.6266877$\pm$0.03632790&0.55060490$\pm$0.030011454&0.63238422$\pm$0.049560104\\
$\delta$&1.026908e+00$\pm$4.323170e-01&1.5387645$\pm$0.28635511&2.09662379$\pm$0.333273034&1.82876419$\pm$0.287441536\\
$r$&9.582111e-03$\pm$1.303745e-03&0.1870607$\pm$0.02129171&0.00658906$\pm$0.001676165&0.01108170$\pm$0.002666821\\
$\bar{d}$&7.225749e-01$\pm$1.475284e-01&0.7894275$\pm$0.03793392&0.91603864$\pm$0.008777011&0.89253127$\pm$0.044771296\\
$e$&6.529959e-01$\pm$7.448434e-02&0.9656237$\pm$0.01121815&0.94523444$\pm$0.001576657&0.85650444$\pm$0.033640178\\
$s$&-1.009041e+00$\pm$5.755951e-02&-0.6915159$\pm$0.07571627&-0.86066650$\pm$0.011647444&-1.04113626$\pm$0.022954670\\
&1.927581e-02&0.1949022&0.08736587&0.01814215\\\hline
\end{tabular}


\begin{tabular}{|c|p{3.7cm}|p{3.7cm}|p{3.7cm}|p{3.7cm}|}
\hline
Configuration&(1)&(2)&(3)&(4)\\\hline
$\rho[r,\bar{d}]$&-0.4679513 [-6.235562e-01 , -0.2766820]&-0.6021740 [-0.7258547 , -4.407751e-01]&-1.836459e-01 [-0.3877183 , 3.758763e-02]&1.535686e-01 [-0.0684582 , 3.611013e-01]\\
$\rho[r,e]$&-0.4969249 [-6.460989e-01 , -0.3111840]&-0.1625078 [-0.3690476 , 5.932741e-02]&-1.848848e-01 [-0.3888074 , 3.630699e-02]&5.693427e-01 [0.3996153 , 7.013276e-01]\\
$\rho[r,s]$&0.3081761 [9.488394e-02 , 0.4944154]&-0.3421209 [-0.5225573 , -1.323529e-01]&-3.313264e-01 [-0.5136505 , -1.203723e-01]&-4.066184e-01 [-0.5749672 , -2.052376e-01]\\
$\rho[\bar{d},e]$&0.8508536 [7.762552e-01 , 0.9019531]&0.5705894 [0.4011659 , 7.022647e-01]&-5.995528e-05 [-0.2197740 , 2.196598e-01]&4.344970e-01 [0.2374484 , 5.972012e-01]\\
$\rho[\bar{d},s]$&-0.8023042 [-8.688589e-01 , -0.7072636]&0.6692694 [0.5270560 , 7.750198e-01]&6.791720e-02 [-0.1540993 , 2.834050e-01]&-9.559466e-02 [-0.3088251 , 1.267852e-01]\\
$\rho[e,s]$&-0.9009695 [-9.354970e-01 , -0.8493979]&0.8781442 [0.8158375 , 9.202963e-01]&8.468024e-01 [0.7704290 , 8.992143e-01]&-1.789705e-01 [-0.3836031 , 4.241422e-02]\\\hline
$\rho[\bar{c},\bar{l}]$&0.4234633 [2.246481e-01 , 5.884313e-01]&0.7235166 [0.5990261 , 0.8138558]&0.8431956 [0.7652527 , 0.8967725]&0.4006936 [0.1984478 , 0.5702097]\\
$\rho[\bar{c},\bar{s}]$&-0.1023031 [-3.149408e-01 , 1.201137e-01]&-0.7950977 [-0.8638952 , -0.6971747]&-0.7872919 [-0.8585037 , -0.6862905]&-0.4496771 [-0.6092037 , -0.2551716]\\
$\rho[\bar{c},\delta]$&0.2568220 [3.932404e-02 , 4.510850e-01]&0.7049855 [0.5742125 , 0.8006794]&0.8299035 [0.7462630 , 0.8877455]&0.6332603 [0.4803829 , 0.7487918]\\
$\rho[\bar{l},\bar{s}]$&0.4219260 [2.228701e-01 , 5.872064e-01]&-0.7921187 [-0.8618395 , -0.6930156]&-0.8942859 [-0.9310592 , -0.8395275]&-0.8546133 [-0.9044912 , -0.7816736]\\
$\rho[\bar{l},\delta]$&0.6948818 [5.607839e-01 , 7.934560e-01]&0.8158636 [0.7263519 , 0.8781621]&0.8882018 [0.8305739 , 0.9270100]&0.6689605 [0.5266519 , 0.7747964]\\
$\rho[\bar{s},\delta]$&0.3805439 [1.755009e-01 , 5.539444e-01]&-0.6657526 [-0.7724742 , -0.5224595]&-0.7760599 [-0.8507179 , -0.6707077]&-0.6680160 [-0.7741129 , -0.5254169]\\\hline
\end{tabular}


\begin{tabular}{|c|p{3.7cm}|p{3.7cm}|p{3.7cm}|p{3.7cm}|}
\hline
Configuration&(1)&(2)&(3)&(4)\\\hline
$\rho[r,\bar{c}]$&-0.0958671 [-0.3090738 , 0.12651465]&-0.2846750 [-0.47470041 , -0.06929222]&0.106860362 [-0.11557001 , 0.3190854424]&0.166141774 [-0.05560493 , 0.372269294]\\
$\rho[r,\bar{l}]$&0.3184592 [0.1061712 , 0.50298200]&0.3611645 [0.15363946 , 0.53817505]&0.120291668 [-0.10212439 , 0.3312534969]&0.112524033 [-0.10991022 , 0.324224975]\\
$\rho[r,\bar{s}]$&0.3510671 [0.1423288 , 0.52990926]&0.3864280 [0.18217892 , 0.55870790]&-0.207507373 [-0.40859524 , 0.0127927828]&-0.006745464 [-0.22612722 , 0.213287544]\\
$\rho[r,\delta]$&-0.3419770 [-0.5224388 , -0.13219278]&0.4010223 [0.19882412 , 0.57047399]&-0.199327994 [-0.40146255 , 0.0213227474]&-0.101075026 [-0.31382258 , 0.121336498]\\
$\rho[\bar{d},\bar{c}]$&-0.1833004 [-0.3874145 , 0.03794465]&-0.4143381 [-0.58114893 , -0.21411352]&0.006860986 [-0.21317727 , 0.2262368358]&0.348614959 [0.13959020 , 0.527896760]\\
$\rho[\bar{d},\bar{l}]$&0.8121174 [0.7210643 , 0.87559646]&0.7363382 [0.61633582 , 0.82291849]&0.191569016 [-0.02938471 , 0.3946737203]&0.607176014 [0.44710613 , 0.729564080]\\
$\rho[\bar{d},\bar{s}]$&0.8175828 [0.7287819 , 0.87933822]&0.7718832 [0.66493682 , 0.84781438]&-0.219254065 [-0.41879596 , 0.0004862563]&0.731832543 [0.61023974 , 0.819738748]\\
$\rho[\bar{d},\delta]$&-0.8199313 [-0.8809437 , -0.73210512]&0.7272964 [0.60411696 , 0.81653203]&-0.205542803 [-0.40688430 , 0.0148444885]&-0.146644197 [-0.35492533 , 0.075505502]\\
$\rho[e,\bar{c}]$&-0.2355562 [-0.4328696 , -0.01670384]&0.1482222 [-0.07390147 , 0.35633436]&-0.028013851 [-0.24621526 , 0.1928903073]&-0.213237762 [-0.41357774 , 0.006797615]\\
$\rho[e,\bar{l}]$&0.4856866 [0.2977430 , 0.63738583]&-0.3440356 [-0.52413307 , -0.13448447]&0.005838058 [-0.21415354 , 0.2252660012]&-0.259589628 [-0.45344380 , -0.042284479]\\
$\rho[e,\bar{s}]$&0.5700927 [0.4005481 , 0.70189139]&-0.5411854 [-0.68004036 , -0.36485222]&0.206851807 [-0.01347763 , 0.4080244658]&-0.520820100 [-0.66449667 , -0.340011761]\\
$\rho[e,\delta]$&-0.5105626 [-0.6566203 , -0.32759502]&-0.4972353 [-0.64633902 , -0.31155624]&0.180368653 [-0.04097196 , 0.3848345410]&0.007300570 [-0.21275760 , 0.226653900]\\
$\rho[s,\bar{c}]$&-0.5398119 [-0.6789960 , -0.36316900]&-0.3711488 [-0.54631495 , -0.16487721]&0.066233469 [-0.15575000 , 0.2818487302]&0.191933347 [-0.02900679 , 0.394992994]\\
$\rho[s,\bar{l}]$&0.8100685 [0.7181768 , 0.87419174]&0.5960875 [0.43309300 , 0.72133128]&0.103616019 [-0.11880563 , 0.3161357092]&0.293165083 [0.07850495 , 0.481844758]\\
$\rho[s,\bar{s}]$&0.8166515 [0.7274653 , 0.87870120]&0.5501248 [0.37583582 , 0.68682409]&-0.214962957 [-0.41507543 , 0.0049896034]&0.326893808 [0.11547047 , 0.509981729]\\
$\rho[s,\delta]$&-0.7869970 [-0.8582997 , -0.68588011]&0.5902345 [0.42572786 , 0.71697130]&-0.234261611 [-0.43175552 , -0.0153339732]&-0.053173625 [-0.26973912 , 0.168512022]\\\hline
\end{tabular}


\end{center}









\end{document}
