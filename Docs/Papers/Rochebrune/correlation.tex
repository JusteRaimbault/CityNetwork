%%%%%%%%%%%%%%%%%%%%%%%%%%%%%
% Standard header for working papers
%
% WPHeader.tex
%
%%%%%%%%%%%%%%%%%%%%%%%%%%%%%

\documentclass[11pt]{article}

% packages without options
\usepackage{amsmath,bbm}

% geometry
\usepackage[margin=2cm]{geometry}






\title{\vspace{-1.5cm}Génération de Données Synthétiques Correlées\\\medskip
\textit{Intention de communication, Journées de Rochebrune 2016}
}
\author{\noun{Juste Raimbault}$^{1,2}$\\
$^{1}$ UMR CNRS 8504 Géographie-cités\\
$^{2}$ UMR-T IFSTTAR 9403 LVMT
}
\date{15 octobre 2015}


\maketitle

\justify

\vspace{-0.5cm}

L'utilisation de données synthétiques, au sens de populations statistiques d'individus générées aléatoirement sous la contrainte de reproduire certaines caractéristiques du système étudié, est une pratique méthodologique largement répandue dans de nombreuses disciplines, et particulièrement pour des problématiques liées aux systèmes complexes, telles que l'évaluation thérapeutique~\cite{abadie2010synthetic}, la géographie~\cite{moeckel2003creating}, l'apprentissage statistique~\cite{bolon2013review}. Si le premier ordre est bien maitrisé, il n'a à notre connaissance pas été proposé de méthode systématique permettant un contrôle au second ordre, c'est à dire où le niveau de correlation estimé sur les données généré est maitrisé.

\vspace{-0.4cm}
\paragraph{Description générique de la méthode}

Soit un ensemble de processus stochastiques $X_I$ (l'index pouvant être le temps ou l'espace par exemple). On se propose, à partir d'un jeu de réalisations $\mathbf{X}=(X_{i,j})$, de générer une population statistique $\mathbf{\tilde{X}}=\tilde{X}_{i,j}$ telle que : 1. Un certain critère de proximité aux données est vérifié, i.e. $\norm{f(\mathbf{X})-f(\mathbf{\tilde{X}})} < \varepsilon$ ; et 2. Le niveau de correlation est controlé, i.e. pour tout $R$, $\Varb{(\tilde{X}_i)} = R$, où la matrice de variance/covariance est estimée sur la population synthétique.


\vspace{-0.4cm}
\paragraph{Application : séries temporelles financières}

Un premier domaine d'application proposé pour notre méthode est celui des séries temporelles financières, signaux typiques de systèmes complexes hétérogènes et multiscalaires~\cite{mantegna2000introduction} et pour lesquels les corrélations ont fait l'objet d'abondants travaux (voir matrices aléatoires~\cite{2009arXiv0910.1205B}, analyse de réseaux~\cite{tumminello2005tool}).

Considérons un réseau d'actifs $(X_i(t))_{1\leq i \leq N}$ échantillonés à haute fréquence (typiquement 1s), vus comme la superposition de signaux à des multiples échelles temporelles : $X_i=\sum_{\omega}{X_i^{\omega}}$ sur lesquels est appliqué un modèle de prédiction de tendance à une échelle temporelle $\omega_0$ donnée, représenté formellement comme un estimateur $M_{\omega_0} : (X_i) \mapsto \hat{X_i}$ dont l'objectif est la minimisation de $\norm{}$. Dans le cas d'estimateurs auto-regressifs multivariés, la performance dépendra des correlations respectives entre actifs



\vspace{-0.4cm}
\paragraph{Application : données géographiques de densité et de réseaux}

En géographie, l'utilisation de données synthétiques est plutôt axée vers l'utilisation de population synthétiques au sein de modèles agents (mobilité, modèles \emph{LUTI})~\cite{pritchard2009advances}. Il a récemment été proposé de contrôler systématiquement les effets de la configuration spatiale sur le comportement de modèles de simulation spatialisés~\cite{cottineau2015revisiting}.

Dans notre cas, nous proposons de générer des systèmes de villes représentés par une densité spatiale de population $d(\vec{x})$ et la donnée d'un réseau de transport $n(\vec{x})$. L'utilisation d'un modèle type aggrégation-diffusion~\cite{batty2006hierarchy} permet de générer une distribution discrete de densité. Le modèle est calibré pour des objectifs morphologiques (entropie, hiérarchie, autocorrélation, densité) contre les valeurs calculées sur l'ensemble des grilles de taille 50km extraites de la grille européenne de densité~\cite{}. %Cit grille.
D'autre part, 








%%%%%%%%%%%%%%%%%%%%
%% Biblio
%%%%%%%%%%%%%%%%%%%%
\footnotesize

\bibliographystyle{apalike}
\bibliography{/Users/Juste/Documents/ComplexSystems/CityNetwork/Biblio/Bibtex/CityNetwork,biblio}


\end{document}
