\documentclass[letterpaper]{article}

\usepackage{natbib,alifeconf}  %% The order is important


% *****************
%  Requirements:
% *****************
%
% - All pages sized consistently at 8.5 x 11 inches (US letter size).
% - PDF length <= 8 pages for full papers, <=2 pages for extended
%    abstracts.
% - Abstract length <= 250 words.
% - No visible crop marks.
% - Images at no greater than 300 dpi, scaled at 100%.
% - Embedded open type fonts only.
% - All layers flattened.
% - No attachments.
% - All desired links active in the files.

% Note that the PDF file must not exceed 5 MB if it is to be indexed
% by Google Scholar. Additional information about Google Scholar
% can be found here:
% http://www.google.com/intl/en/scholar/inclusion.html.


% If your system does not generate letter format documents by default,
% you can use the following workflow:
% latex example
% bibtex example
% latex example ; latex example
% dvips -o example.ps -t letterSize example.dvi
% ps2pdf example.ps example.pdf


% For pdflatex users:
% The alifeconf style file loads the "graphicx" package, and
% this may lead some users of pdflatex to experience problems.
% These can be fixed by editing the alifeconf.sty file to specify:
% \usepackage[pdftex]{graphicx}
%   instead of
% \usepackage{graphicx}.
% The PDF output generated by pdflatex should match the required
% specifications and obviously the dvips and ps2pdf steps become
% unnecessary.


% Note:  Some laser printers have a serious problem printing TeX
% output. The use of ps type I fonts should avoid this problem.


\title{From co-evolution to morphogenesis: a conceptual approach}
\author{Juste Raimbault$^{1,2}$\\
\mbox{}\\
$^1$UPS CNRS 3611 ISC-PIF, Paris, France \\
$^2$UMR CNRS 8504 Géographie-cités, Paris, France \\
juste.raimbault@iscpif.fr} % email of corresponding author

% For several authors from the same institution use the same number to
% refer to one address.
%
% If the names do not fit well on one line use
%         Author 1, Author 2 ... \\ {\Large\bf Author n} ...\\ ...
%
% If the title and author information do not fit in the area
% allocated, place \setlength\titlebox{<new height>} after the
% \documentclass line where <new height> is 2.25in



\begin{document}
\maketitle

\begin{abstract}
% Abstract length should not exceed 250 words
 
\end{abstract}

\section{Introduction}


\subsubsection{Biological systems and social systems}


The parallel between social and biological systems is not rare, sometimes more from an analogy perspective as for example in West's \emph{Scaling} theory which applies similar growth equations starting from scaling laws, with however inverse conclusions concerning the relation between size and pace of life~\cite{bettencourt2007growth}. Scaling relations do not hold when we try to apply them to a single ant, and they must be applied to the whole ant colony which is then the organism studied. When adding the property of cognition, we confirm that it is the relevant level, since the colony shows advanced cognitive properties, such as the resolution of spatial optimization problems, or the quick answer to an external perturbation. Human social organizations, cities, could be seen as organisms ? \cite{banos2013pour} extends the metaphor of the \emph{urban anthill} but recalls that the parallel stops quickly. We will however see to what extent some concepts from the epistemology of biology can be useful to understand social systems that we propose to study.

We start from the fundamental contribution of Monod in~\cite{monod1970hasard}, which aims at developing crucial epistemological principles for the study of life. Thus, living organisms answer to three essential properties that differentiate them from other systems: (i) the teleonomy , i.e. the property that these are ``objects with a project'', project that is reflected in their structure and the structure of artifacts they produce\footnote{That must not be mistaken with teleology, typical of animist thoughts, that consists in giving a project or a meaning to the universe.}; (ii) the importance of morphogenetic processes in their constitution (see~\ref{sec:interdiscmorphogenesis}); (iii) the property of the invariant reproduction of information defining their structure. Monod furthermore sketches in conclusion some paths towards a theory of cultural evolution. Teleonomy is crucial in social structures, since any organization aims at satisfying a set of objectives, even if in general it will not succeed and the objectives will co-evolve with the organization. This notion of multi-objective optimization is typical of complex socio-technical systems, and will be more crucial than for biological systems.

Moreover, we postulate that the concept of morphogenesis is an essential tool to understand these systems, with a definition very similar to the one used in biology. A more thorough work to build this definition is done in~\ref{sec:interdiscmorphogenesis}, that we will sum up as the existence of relatively autonomous processes guiding the growth of the system and implying causal circular relations between form and function, that witness an emergent architecture. For social systems, isolating the system is more difficult and the notion of boundary will be less struct than for a biological system, but we will indeed find this link between form and function, such as for example the structure of an organization that impacts its functionalities.

Finally, the reproduction of information is at the core of cultural evolution, through the transmission of culture and \emph{memetics}, the difference being that the ratio of scales between the frequency of transmission and mutation and cross-over processes or other non-memetic processes of cultural production is relatively low, whereas is many orders of magnitude in biology.

An example shows that the parallel is not always absurd : \cite{2017arXiv170305917G} proposes an auto-catalytic network model for cognition, that would explain the apparition of cultural evolution through processes that are analogous to the ones that occurred at the apparition of life, i.e. a transition allowing the molecules to be self-sustained and to self-reproduce, mental representations being the analogous of molecules.

But even if processes are at the origin analogous, the nature of evolution is then quite different, as show \cite{vanderLeeuw2009}, darwinian criteria for evolution being not sufficient to explain the evolution of our organized societies. This is a complexity of a different nature in which the role of information flows is crucial (see the role of informational complexity in the next subsection).

One point that also must retain our attention is the greater difficulty to define levels of emergence for social systems: \cite{roth2009reconstruction} underlines the risk to fall into ontological dead-ends if levels were badly defined. He argues that more generally we must go past the single dichotomy micro-macro that is used as a caricature of the concepts of weak emergence, and that ontologies must often be multi-level and imply multiple intermediate levels.

This last question must also be put into perspective with the problem of the existence of strong emergence in social structures, that in sociological terms corresponds to the idea of the existence of ``collective beings''~\cite{angeletti2015etres}. Morin indeed distinguishes living systems of the second type (multi-cellular) and of the third type (social structures), but precises that the \emph{subjects} of the latest are necessarily unachieved\cite{morin1980methode} (p.~852). Thus, emergences from the biological to the social are analogous by stay fundamentally different.






\section{Co-evolution}

\subsubsection{Biology}

The concept of co-evolution in biology is an extension of the well-known concept of \emph{evolution}, that can be tracked back to Darwin. \cite{durham1991coevolution} (p.~22) recalls the components and systemic structures that are necessary to have evolution\footnote{And in that general context, evolution is not restricted to the biology of life and the presence of genes, but also to physical systems verifying these conditions. We will come back to that later.}.

\begin{enumerate}
\item Process of \emph{transmission}, implying transmission units and transmission mechanisms.
\item Process of \emph{transformation}, that necessitates sources of variation.
\item Isolation of sub-systems such that the effects of previous processes are observable in differentiations.
\end{enumerate}

This way, a population submitted to constraints (often conceptually synthesized as a \emph{fitness}) that condition the transmission of the genetic heritage of individuals (transmission), and to random genetic mutations (transformation), will indeed be in evolution in the spatial territories it populates (isolation), and by extension the species to which it can be associated. 

Co-evolution is then defined as an evolutionary change in a characteristic of individuals of a population, in response to a change in a second population, which in turn responds by evolution to the change in the first, as synthesized by~\cite{janzen1980coevolution}. This author furthermore highlights the subtlety of the concept and warns against its unjustified uses: the presence of a congruence between two characteristics that seem adapted one to the other does not necessarily imply a co-evolution, since one species could have adapted alone to one characteristic already present in the other.

This rough presentation partly hides the real complexity of ecosystems: populations are embedded in trophic networks and environments, and co-evolutionary interactions would imply communities of populations from diverse species, as presented by \cite{strauss2005toward} under the appellation of diffuse co-evolution. Similarly, spatio-temporal dynamics are crucial in the realization of these processes: \cite{dybdahl1996geography} study for example the influence of the spatial distribution on patterns of co-evolution for a snail and its parasite, and show that a higher speed of genetic diffusion in space for the parasite drive the co-evolutionary dynamics.


The essential concepts to retain from the biological point of view are thus: (i) existence of evolution processes, in particular transmission and transformation; (ii) in circular schemas between populations in the case of co-evolution; and (iii) in a complex territorial frame (spatio-temporal and environmental in the sense of the rest of the ecosystem).



\subsubsection{Cultural evolution}

This development on co-evolution was brought by the parallel between biological and social systems. The evolution of culture is theorized within a proper field, and witnesses many co-evolutive dynamics. \cite{Mesoudi25072017} recalls the state of knowledge on the subject and future issues, such as the relation with the cumulative nature of culture, the influence of demography in evolution processes, or the construction of phylogenetic methods allowing to reconstruct branches of past evolutionary trees.

To give an example, \cite{carrignon2015modelling} introduces a conceptual frame for the co-evolution of culture and commerce in the case of ancient societies for which there are archeological data, and proposes its implementation with a multi-agent model which dynamics are partly validated by the study of stylized facts produced by the model. The co-evolution is here indeed taken in the sense of a mutual adaptation of socio-spatial structures, at comparable time scales, in this more general frame of cultural evolution.


Cultural evolution would even be indissociable from genetic evolution, since \cite{durham1991coevolution} postulates and illustrates a strong link between the two, that would themselves be in co-evolution. \cite{bull2000meme} explores a stylized model including two types of replicant populations (genes and memes) and shows the existence of phase transitions for the results of the genetic evolution process when the interaction with the cultural replicant is strong.


\subsubsection{Sociology}

The concept was used in sociology and related disciplines such as organisation studies, following the parallel done before the same way as cultural evolution. In the field of the study of organisations, \cite{volberda2003co} develop a conceptual frame of inter-organisational co-evolution in relation with internal management processes, but deplore the absence of empirical studies aiming at quantifying this co-evolution. In the context of production systems management, \cite{tolio2010species} conceptualize an intelligent production chain where product, process and the production system must be in co-evolution.


\subsubsection{Economic geography}

In economic geography, the concept of co-evolution has also largely been used. The idea of evolutionary entities in economy comes in opposition to the neo-classical current which remains a majority, but finds a more and more relevant echo~\cite{nelson2009evolutionary}. \cite{schamp201020} proceeds to an epistemological analysis of the use of co-evolution, and opposes the view of a neo-schumpeterian approach to economy which considers the emergence of populations that evolve from micro-economic rules (what would correspond to a direct and relatively isolationist reading of biological evolution) to a systemic approach that would consider the economy as an evolutive system in a global perspective (what would correspond to diffuse co-evolution that we previously developed), to propose a precise characterization that would correspond to the first case, assuming co-evolving \emph{institutions}. The most important for our purpose is that he underlines the crucial aspect of the choice of populations and of considered entities, of the geographical area, and highlights the importance of the existence of causal circular relations.


Diverse examples of application can be given. \cite{doi:10.1080/00343400802662658} introduce a conceptual frame to allow to conciliate the evolutionary nature of companies, the theory of clusters and knowledge networks, in which the co-evolution between networks and companies  is central, and which is defined as a circular causality between different characteristics of these subsystems. \cite{colletis2010co} introduces a framework for the co-evolution of territories and technology (questioning for example the role of proximity on innovations), that reveals again the importance of the institutional aspect. The framework proposed by \cite{ter2011co} couples the evolutionary approach to companies, the literature on industries and innovation in clusters, and the approach through complex networks of connexions between the latest in the territorial system.

In environmental economics, \cite{kallis2007coevolution} show that ``broad'' approaches (that can consider most of co-dynamics as co-evolutive) are opposed to stricter approaches (in the spirit of the definition given by \cite{schamp201020}), and that in any case a precise definition, not necessarily coming from biology, must be given, in particular for the search of an empirical characterization.

\subsubsection{Geography}

For geography, as we already presented in introduction, the works that are the closest to notions of co-evolution empirically and theoretically are closely linked to the evolutive urban theory. It is not easy to track back in the literature at what time the notion was clearly formalized, but it is clear that it was present since the foundations of the theory as recalls Denise Pumain (see~\ref{app:sec:interviews}): the complex adaptive system is composed of subsystems that are interdependent in a complex way, often with circular causalities. The first models indeed include this vision in an implicit way, but co-evolution is not explicitly highlighted of precisely defined, in terms that would be quantifiable or structurally identifiable. \cite{paulus2004coevolution} brings empirical proofs of mechanisms of co-evolution through the study of the evolution of economic profiles of French cities. The interpretation used by~\cite{schmitt2014modelisation} is based on an entry by the evolutive urban theory, and fundamentally consists in a reading of systems of cities as highly interdependent entities.


% \subsubsection{Physical geography}
% In the study of landscapes, \cite{sheeren2015coevolution} evoke the co-evolution of landscape and agricultural activities, but in fact do not consider any circular effect of one on the other. Their result show a priori that the evolution of agricultural practices yield an evolution of the landscape, and it is not clear to what extent the conceptual frame of co-evolution, evoked without any more details, is used.


\subsubsection{Physics}

Finally, we can mention in an anecdotical way that the term of co-evolution has also been used by physics. Its use for physical systems may induce some debates, depending if we suppose or not that the transmission assumes a transmission of \emph{information}\footnote{Information is defined within the shanonian theory as an occurence probability for a chain of characters. \cite{morin1976methode} shows that the concept of information is indeed far more complex, and that it must be thought conjointly to a given context of the generation of a self-organizing negentropic system, i.e. realizing local decreases in entropy in particular thanks to this information. This type of system is necessarily alive. We will follow here this complex approach to information.}. In the case of a purely physical ontological transmission (\emph{physical beings}), then a large part of physical systems are evolutive. \cite{hopkins2008cosmological} develop a cosmological frame for the co-evolution of cosmic heterogenous objects which presence and dynamics are difficultly explained by more classical theories (some types of galaxies, quasars, supermassive black holes). \cite{antonioni2017coevolution} study the co-evolution between synchronisation and cooperation properties within a Kuramoto oscillators network\footnote{The Kuramoto model studies synchronization within complex systems, by studying the evolution of phases $\theta_i$ coupled by interaction equations $\dot{\vec{\theta}} = \vec{\omega} + \vec{W}\left[\vec{\theta}\right] + \mathbf{B}$ where $\vec{\omega}$ are proper forcing phases and the coupling strength between $i$ and $j$ is given by $\vec{W}_{i} = \sum_j w_{ij} \sin\left(\theta_i - \theta_j\right)$ and $\vec{B}$ is noise.}, showing on the one hand that the concept can be applied to abstract objects, and on the other hand that a complex network of relations between variables can be at the origin of dynamics witnessing circular causalities, i.e. a co-evolution in that sense.


\subsubsection{Synthesis}

Most of these approaches fit in the theory of complex adaptive systems developed by Holland, in particular in~\cite{holland2012signals}: it takes any system as an imbrication of systems of boundaries, that filter signals or objects. Within a given limit, the corresponding subsystem is relatively autonomous from the outside, and is called an \emph{ecological niche}, in a direct correspondence with highly connected communities within trophic or ecological networks. This way, interdependent entities within a niche are said to be co-evolving. We will come back on that approach in our theoretical construction in~\ref{sec:theory} when we will have developed other concepts that are necessary for it.


We retain from this multidisciplinary view of co-evolution the fundamental following points, that are precursors of a proper definition of co-evolution that will be given further, concluding the first part.

\begin{enumerate}
	\item The presence of \emph{evolution processes} is primary, and their definition is almost always based on the existence of transmission and transformation processes.
	\item Co-evolution assumes entities or systems, belonging to distinct classes, which evolutive dynamics are coupled in a circular causal way. Approaches can differ depending on the assumptions of populations of these entities, singular objects, or components of a global system then in mutual interdependency without a direct circularity.
	\item The delineation of systems and subsystems, both in the ontological space (definition of studied objects), but also in space and time, and their distribution in these spaces, is fundamental for the existence of co-evolutionary dynamics, and it seems in a large number of cases, of their empirical characterization.
\end{enumerate}




\subsection{A synthetic definition of co-evolution}


We propose the following entry for the specific case of transportation networks and territories, which echoes to the three main points (existence of evolutive processes, definition of entities or populations, isolation of subsystems in space and time) that we gave in~\ref{sec:epistemology}. It verifies the three following specifications.


First of all, evolutive processes correspond to transformations of components of the territorial system at the different scales: transformation of cities on the long time, of their networks, transmission between cities of socio-economic characteristics carries by microscopic agents but also cultural transmission, reproduction and transformation of agents themselves (firms, households, operators)\footnote{This list is based on assumptions of the evolutive urban theory that we already briefly introduced and that we will develop in itself in Chapter~\ref{ch:evolutiveurban}. It can not be exhaustive, since what would be the ``ADN of a city'' remains an open question as recalls Denise Pumain in a dedicated interview~\ref{app:sec:interviews}.}.

These evolutive processes may imply a co-evolution. Within a territorial system, can simultaneously co-evolve: (i) given entities (a given infrastructure and given characteristics of a given territory for example, i.e. individuals), when their mutual influence will be circularly causal (at the corresponding scale); (ii) populations of entities, what will be translated for example as such type of infrastructure and given territorial components co-evolve at a statistical level in a given geographical region; (iii) all the components of a system at a small geographical scale when there exists strong global interdependencies. Our approach is thus fundamentally \emph{multi-scale} and articulates different significations at different scales.

Finally, the constraint of an isolation implies, in relation with the previous point, that co-evolution and the articulation of significations will have a meaning if there exists spatio-temporal isolations of subsystems in which differente co-evolutions operate, what is directly in accordance with a vision in \emph{Multi-scalar systems of systems}.












\section{Morphogenesis}



%The notion of morphogenesis has been deeply explored and with an interdisciplinary point of view in~\ref{sec:interdiscmorphogenesis}. We recall here important axis and to what extent these contribute to the construction of our theory. Morphogenesis has been formalized especially by~\cite{turing1952chemical} which proposes to isolate elementary chemical rules that could lead to the emergence of the embryo and its form.



The morphogenesis of a system consists in evolution rules that produce the emergence of its successives states, i.e. the precise definition of self-organization, with the additional property that an emergent architecture exists, in the sense of causal circular relations between the form and the function. Progresses towards the understanding of embryo morphogenesis (in particular the isolation of particular processes producing the differentiation of cells from an unique cell) have been made only recently with the use of complexity approaches in integrative biology~\cite{delile2016chapitre}.


In the case of urban systems, the idea of urban morphogenesis, i.e. of self-consistent mechanisms that would produce the urban form, is more used in the field of architecture and urban design (as for example the generative grammar of ``Pattern Language'' of \cite{alexander1977pattern}), in relation with theories of urban form~\cite{moudon1997urban}. This idea can be pushed into very large scales such as the one of the building~\cite{whitehand1999urban} but we will use it more at a mesoscopic scale, in terms of land-use changes within an intermediate scale of territorial systems, with similar ontologies as the urban morphogenesis modeling literature (for example \cite{bonin2012modele} describes a model of urban morphogenesis with qualitative differentiation, whereas \cite{makse1998modeling} give a model of urban growth based on a mono-centric population distribution perturbed with correlated noises).


The concept of morphogenesis is important in our theory in link with modularity and scale. Modularity of a complex system consists in its decomposition into relatively independent sub-modules, and the modular decomposition of a system can be seen as a way to disentangle non-intrinsic correlations~\cite{2015arXiv150904386K} (to have an idea, think of a block diagonalisation of a first order dynamical system). In the context of large-scale cyber-physical systems design and control, similar issues naturally raise and specific techniques are needed to scale up simple control methods~\cite{2017arXiv170105880W}. The isolation of a subsystem yields a corresponding characteristic scale. Isolating possible morphogenesis processes implies a controlled extraction (controlled boundary conditions e.g.) of the considered system, corresponding to a modularity level and thus a scale.


When local processes are not enough to explain the evolution of a system (with reasonable variations of initial conditions), a change of scale is necessary, caused by an underlying phase transition in modularity. The example of metropolitan growth is a good example: complexity of interactions within the metropolitan region will grow with size and the diversity of functions, leading to a change in the scale necessary to understand processes. The characteristic scales and the nature of processes for which these change occur can be precise questions investigated through modeling.

Finally, it is important to remark as we did in~\ref{sec:interdiscmorphogenesis} that a territorial subsystem for which morphogenesis makes sense, which boundaries are well defined and which processes allow it to maintain itself as a network of processes, is close to an \emph{auto-poietic system} in the extended sense of Bourgine in~\cite{bourgine2004autopoiesis}\footnote{Which are however not cognitive, making these morphogenetic systems not alive in the sense of auto-poietic and cognitive. Given the difficulty to define the delineation of cities for example, we will leave open the issue of the existence of auto-poietic territorial systems, and will consider in the following a less restrictive point of view on boundaries.}. These systems regulate then their boundary conditions, what underlines the importance of boundaries that we will finally develop.


\subsection{Linking morphogenesis and co-evolution}


We finally propose a link between morphogenesis and co-evolution. It is brought by Holland which sheds a relevant light through an approach of complex adaptive systems (CAS) by a theory of CAS as agents which fundamental property is to process signals thanks to their boundaries~\cite{holland2012signals}.


In this theory, complex adaptive systems form aggregates at diverse hierarchical levels, which correspond to different level of self-organization, and boundaries are vertically and horizontally intricated in a complex way. That approach introduces the notion of \emph{niche} as a relatively independent subsystem in which ressources circulate (the same way as communities in a network as used in chapter~\ref{ch:modelinginteractions}): numerous illustrations such as economical niches or ecological niches can be given. Agents within a niche are then said to be \emph{co-evolving}.


%Empirically, results obtained witness a co-evolution at the mesoscopic scale such as in~\ref{sec:causalityregimes}, confirming the existence of niches for some aspects of territorial systems. The co-evolution in that sense implies then strong interdependencies with circular causal processes (rejoining the definition we took) and a certain independence regarding the exterior of the niche.



The notion is naturally flexible as it will depend on ontologies, on the resolution, on thresholds, etc. that we consider to define the system. We postulate given the clues of existence obtained in empirical results, but also models reproducing processes in a credible manner under a reasonable independence assumption, that this concept can easily be transmitted to the evolutive urban theory and corresponds to the notion of co-evolution we took (and in particular at the level of a population of entities): co-evolving agents in a system of cities consist in a niche with their own flows, signals and boundaries and thus co-evolving entities in the sense of Holland.




\section{Discussion}





\section{Conclusion}






%\section{Acknowledgements}
%This work was supported by 

\footnotesize
\bibliographystyle{apalike}
\bibliography{biblio,/Users/juste/ComplexSystems/CityNetwork/Biblio/Bibtex/CityNetwork} % replace by the name of your .bib file


\end{document}
