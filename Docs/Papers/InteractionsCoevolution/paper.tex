\documentclass[11pt]{article}

\usepackage{times}

% general packages without options
\usepackage{amsmath,amssymb,bbm}
% graphics
\usepackage{graphicx}
% text formatting
\usepackage[document]{ragged2e}
\usepackage{pagecolor,color}

\newcommand{\noun}[1]{\textsc{#1}}

\usepackage[utf8]{inputenc}
\usepackage[T1]{fontenc}
% geometry
\usepackage[margin=1.5cm]{geometry}

\usepackage{multicol}
\usepackage{setspace}

\usepackage{natbib}
\setlength{\bibsep}{0.0pt}

\usepackage[french]{babel}

% layout : use fancyhdr package
%\usepackage{fancyhdr}
%\pagestyle{fancy}

% variable to include comments or not in the compilation ; set to 1 to include
\def \draft {1}


% writing utilities

% comments and responses
%  -> use this comment to ask questions on what other wrote/answer questions with optional arguments (up to 4 answers)
\usepackage{xparse}
\usepackage{ifthen}
\DeclareDocumentCommand{\comment}{m o o o o}
{\ifthenelse{\draft=1}{
    \textcolor{red}{\textbf{C : }#1}
    \IfValueT{#2}{\textcolor{blue}{\textbf{A1 : }#2}}
    \IfValueT{#3}{\textcolor{ForestGreen}{\textbf{A2 : }#3}}
    \IfValueT{#4}{\textcolor{red!50!blue}{\textbf{A3 : }#4}}
    \IfValueT{#5}{\textcolor{Aquamarine}{\textbf{A4 : }#5}}
 }{}
}

% todo
\newcommand{\todo}[1]{
\ifthenelse{\draft=1}{\textcolor{red!50!blue}{\textbf{TODO : \textit{#1}}}}{}
}


\makeatletter


\makeatother


\linespread{1.25}

\begin{document}







\title{\vspace{-2cm}
Modélisation des interactions entre réseaux de transport et territoires : une approche par la co-évolution
\bigskip\\
\textit{Journée d'étude Pacte-Citeres 2018\\
}
}
\author{\noun{Juste Raimbault}\medskip\\
(UPS CNRS 3611 ISC-PIF et UMR CNRS 8504 Géographie-cités)\\
}
\date{}

\maketitle

\justify

\pagenumbering{gobble}


\textbf{Mots-clés : }\textit{Réseaux de transport ; territoire ; co-évolution ; modélisation}

\medskip


\begin{abstract}
	
\end{abstract}




%%%%%%%%%%%%%%%%%%%%%%
\section{Introduction}


Les potentiels effets des réseaux techniques sur les territoires, et plus particulièrement des réseaux de transport, ont alimenté des débats scientifiques qui restent aujourd'hui relativement ouverts, comme la question de l'identification d'effets structurants des infrastructures \citep{offner1993effets,espacegeo2014effets}. Une entrée pertinente est de comprendre les territoires et les réseaux de transport comme étant en co-évolution, c'est-à-dire exhibant des dynamiques couplées fortement qu'il est difficile d'isoler \citep{bretagnolle:tel-00459720}. Le travail de recherche présenté propose d'explorer cette perspective de co-évolution en l'explorant par l'intermédiaire de la modélisation et de la simulation, considérant le modèle comme un instrument de connaissance à part entière \citep{banos2013pour} complémentaire aux aspects théoriques et empiriques \citep{raimbault2017applied}, et dont l'impact est amplifié par l'utilisation des méthodes et outils d'exploration des modèles et de calcul intensif \citep{pumain2017urban}.

% pour modelographie etc ; plus review cit these

Deux directions complémentaires de modélisation, étendant les travaux antérieurs de modélisation de cette co-évolution à l'échelle macroscopique \citep{baptistemodeling,schmitt2014modelisation} et mesoscopique \citep{raimbault2014hybrid}, sont explorées.



%%%%%%%%%%%%%%%%%%%%%%
\section{Définition de la co-évolution}




%%%%%%%%%%%%%%%%%%%%%%
\section{Echelle macroscopique}

Le premier axe de modélisation se situe à l'échelle macroscopique et se base sur les principes de la théorie évolutive des villes \citep{pumain1997pour}. La famille des modèles Simpop se place majoritairement dans les ontologies et échelles correspondantes, c'est-à-dire des entités élémentaires constituées par les villes elles-mêmes, à l'échelle spatial du système de ville (régionale à continentale) et sur des échelles temporelles relativement longues


\subsection{Effets de réseau}

Un premier modèle de contrôle au sein duquel le réseau est statique mais ayant une retroaction sur les villes, suggère indirectement des effets de réseau. Ce travail préliminaire est détaillé par \cite{raimbault2018indirect} qui détaille le modèle et l'applique au système de ville français sur le temps long (1830-1999). Le modèle travaille sur des populations attendues et capture la complexité par les interactions non-linéaires entre villes et portées par le réseau. Trois processus se superposent pour déterminer le taux de croissance des villes: (i) une croissance endogène fixée par un paramètre, correspondant au modèle de Gibrat; (ii) des processus d'interaction directe exprimés sous la forme d'un potentiel gravitaire influençant le taux de croissance; (iii) une retroaction des flux circulant dans le réseau sur les villes traversées. Le modèle est initialisé avec les populations réelles au début d'une période, puis évalué par comparaison avec les populations simulées sur l'ensemble de la période.


\subsection{Modèle de co-évolution}


Ce modèle est ensuite étendu à un modèle co-évolutif, au sein duquel villes et liens du réseau de transport sont tous les deux dynamiques et en dépendance réciproque. 



L'exploration systématique de ce modèle par l'intermédiaire du logiciel OpenMOLE \citep{reuillon2013openmole} et l'application d'une méthode empirique de caractérisation de la co-évolution \cite{raimbault2017identification} permettent de montrer qu'il capture une grande variété de dynamiques couplées, incluant effectivement des dynamiques co-évolutives.

La calibration sur le système de ville français sur la même durée que le modèle statique, avec données de population et réseaux ferroviaire dynamique, permet de quantifier l'évolution de processus d'interaction comme l'effet tunnel.


%%%%%%%%%%%%%%%%%%%%%%
\section{Echelle mesoscopique}

Le deuxième axe, à l'échelle mesoscopique, considère l'entrée par la morphogenèse urbaine, comprise comme l'émergence simultanée de la forme et de la fonction d'un système \citep{doursat2012morphogenetic}. Celle-ci permet de considérer une description plus fine des territoires, à l'échelle de grilles fines de population (résolution 500m) et de représentation vectorielle du réseau à la même échelle. 

\subsection{Morphogenèse par aggrégation-diffusion}

Les systèmes territoriaux produits sont quantifiés par indicateurs morphologiques pour la population \citep{le2015forme} et indicateurs structurels du réseau. Ces indicateurs et leur corrélations spatiales sont par ailleurs calculés sur des fenêtres de taille équivalente couvrant l'ensemble de l'Europe.

Nous introduisons alors un modèle de morphogenèse capturant la co-évolution de la distribution spatiale de la population et du réseau routier. Le calcul d'indicateurs topologiques pour le réseau routier
 permet de calibrer le modèle et de montrer que les différents processus de croissance de réseau qui ont été inclus suivant un processus de multi-modélisation sont complémentaires pour se rapprocher du maximum de configurations réelles. Enfin, un dernier modèle métropolitain, étendant celui proposé par \citep{lenechet:halshs-01272236}, explore le rôle des processus de gouvernance, en particulier de collaboration entre acteurs locaux par l'intermédiaire de théorie des jeux, dans l'émergence du réseau de transport et son interaction avec la forme urbaine quantifiée par les motifs spatiaux d'accessibilité. Ce modèle permet par exemple de montrer que les dynamiques co-évolutives peuvent être amenées à inverser le comportement des gains d'accessibilité en comparaison à une configuration sans évolution de l'usage du sol, c'est-à-dire changer qualitativement le régime du système métropolitain.


%%%%%%%%%%%%%%%%%%%%%%
\section{Perspectives}

Cette recherche développe des approches complémentaires à différentes échelles des interactions entre réseaux de transport et territoires en modélisant leur co-évolution.

\subsection{Vers des modèles multi-échelle}


Elle ouvre ainsi des perspectives d'approches intégrées, vers des modèles multi-échelles de ces interactions, qui s'avèrent de plus en plus nécessaires pour l'élaboration de modèles opérationnels pouvant être appliqués à l'élaboration de politiques de planification soutenables \citep{rozenblat2018conclusion}.




%%%%%%%%%%%%%%%%%%%%
%% Biblio
%%%%%%%%%%%%%%%%%%%%
%\tiny

%\begin{multicols}{2}

%\setstretch{0.3}
%\setlength{\parskip}{-0.4em}


\bibliographystyle{apalike}
\bibliography{biblio}
%\end{multicols}



\end{document}
