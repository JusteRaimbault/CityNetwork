\documentclass[11pt]{article}

\usepackage{times}

% general packages without options
\usepackage{amsmath,amssymb,bbm}
% graphics
\usepackage{graphicx}
% text formatting
\usepackage[document]{ragged2e}
\usepackage{pagecolor,color}

\newcommand{\noun}[1]{\textsc{#1}}

\usepackage[utf8]{inputenc}
\usepackage[T1]{fontenc}
% geometry
\usepackage[margin=1.5cm]{geometry}

\usepackage{multicol}
\usepackage{setspace}

\usepackage{natbib}
\setlength{\bibsep}{0.0pt}

\usepackage[french]{babel}

% layout : use fancyhdr package
%\usepackage{fancyhdr}
%\pagestyle{fancy}

% variable to include comments or not in the compilation ; set to 1 to include
\def \draft {1}


% writing utilities

% comments and responses
%  -> use this comment to ask questions on what other wrote/answer questions with optional arguments (up to 4 answers)
\usepackage{xparse}
\usepackage{ifthen}
\DeclareDocumentCommand{\comment}{m o o o o}
{\ifthenelse{\draft=1}{
    \textcolor{red}{\textbf{C : }#1}
    \IfValueT{#2}{\textcolor{blue}{\textbf{A1 : }#2}}
    \IfValueT{#3}{\textcolor{ForestGreen}{\textbf{A2 : }#3}}
    \IfValueT{#4}{\textcolor{red!50!blue}{\textbf{A3 : }#4}}
    \IfValueT{#5}{\textcolor{Aquamarine}{\textbf{A4 : }#5}}
 }{}
}

% todo
\newcommand{\todo}[1]{
\ifthenelse{\draft=1}{\textcolor{red!50!blue}{\textbf{TODO : \textit{#1}}}}{}
}


\makeatletter


\makeatother


\linespread{1.25}

\begin{document}







\title{
Modeling interactions between transportation networks and territories: a co-evolution approach
\bigskip\\
\textit{Journée d'étude Pacte-Citeres 2018\\
}
}
\author{\noun{Juste Raimbault}$^{1,2,3,*}$\medskip\\
$^1$UPS CNRS 3611 ISC-PIF\\
$^2$CASA, UCL\\
$^3$UMR CNRS 8504 Géographie-cités\medskip\\
* \texttt{juste.raimbault@polytechnique.edu}
}
\date{}

\maketitle

\justify

%\pagenumbering{gobble}




\medskip


\renewcommand{\abstractname}{}
\begin{abstract}
	\begin{center}
	\textbf{Abstract}
	\end{center}
	
	\medskip
	
	Interactions between transportation networks and territories are the subject of open scientific debates, in particular regarding the possible existence of structuring effects of networks, and linked to crucial practical issues of territorial development. We propose an entry on these through co-evolution, and more particularly by the modeling of co-evolution processes between transportation networks and territories. We construct a multi-disciplinary definition of co-evolution which is proper to territorial systems and which can be tested empirically. We then develop the lessons learnt from the development of two types of models, macroscopic interaction models in systems of cities and mesoscopic morphogenesis models through co-evolution. This research opens the perspective of multi-scale models that could be applied to territorial prospective.
	
	\medskip
	
	\textbf{Keywords: }\textit{Transportation networks ; territory ; co-evolution ; modeling}
	
\end{abstract}



%%%%%%%%%%%%%%%%%%%%%%
\section{Introduction}


%Les potentiels effets des réseaux techniques sur les territoires, et plus particulièrement des réseaux de transport, ont alimenté des débats scientifiques qui restent aujourd'hui relativement ouverts, comme la question de l'identification d'effets structurants des infrastructures \citep{offner1993effets}. Ceux-ci peuvent être observés sur du temps long, mais de nombreux travaux rappellent qu'il faut rester prudent quant à la contingence de chaque situation et aux dangers d'une instrumentalisation politique du concept~\citep{espacegeo2014effets}.
The potential effect of technical networks on territories, and more particularly of transportation networks, have fed scientific debates which remain relatively open in the current state-of-the-art, such as the issue of identifying of structuring effects of infrastructures~\citep{offner1993effets}. These can be observed on long time periods, but several works recall that caution is key regarding the contingency of each situation and the dangers of a political application of the concept~\citep{espacegeo2014effets}.


%Une entrée pertinente pour cette question, explorée par \cite{raimbault2018caracterisation} dont nous faisons ici une synthèse partielle, est de comprendre les territoires et les réseaux de transport comme étant en co-évolution, c'est-à-dire exhibant des dynamiques couplées fortement qu'il est difficile d'isoler \citep{bretagnolle:tel-00459720}. La construction du concept de territoire par \cite{raimbault2018caracterisation} à la croisée d'une lecture Raffestinienne et d'une lecture Pumanienne, c'est-à-dire en croisant le concept de territoire humain \citep{raffestin1988reperes} avec celui de territoire comme espace d'insertion des systèmes de villes \citep{pumain2018evolutionary}, montre que celui-ci implique des relations potentielles entre objets géographiques et par induction des réseaux techniques par concrétisation de projets transactionnels~\citep{dupuy1987vers}. Ainsi, l'interaction réseau-territoire peut intrinsèquement être lue comme endogène au concept de territoire lui-même. Nous allons plus loin en postulant l'existence potentielle d'une \emph{co-évolution} entre réseaux de transports et territoires, que nous tâcherons de définir de manière interdisciplinaire dans une section dédiée ci-dessous. Nous étudions dans notre travail plus particulièrement les \emph{réseaux de transport}, notamment car ceux-ci sont particulièrement représentatifs des potentialités de façonner les territoires qu'on prête aux réseaux techniques par le transport effectif des flux en particulier \citep{bavoux2005geographie}.
A relevant approach to this question, explored by \cite{raimbault2018caracterisation} of which we do a short synthesis here, is to understand territories and transportation networks as co-evolving, i.e. exhibiting strongly coupled dynamics which are difficult to isolate~\citep{bretagnolle:tel-00459720}. The construction of the concept of territory by \cite{raimbault2018caracterisation} at the intersection of Raffestin's approach and Pumain's approach, i.e. by combining the concept of human territory~\citep{raffestin1988reperes} with the one of territory as the space in which systems of cities are embedded~\citep{pumain2018evolutionary}, shows that it implies potential relations between geographical objects, and by induction the emergence of technical networks through the realization of transactional projects~\citep{dupuy1987vers}. Therefore, the interaction between networks and territories can be intrinsically understood as endogenous to the concept of territory itself. We go further by postulating the potential existence of a \emph{co-evolution} between transportation networks and territories, that we will define below from an interdisciplinary point of view in a dedicated section. We study more particularly in our work the \emph{transportation networks}, in particular because these are typically representative of potentialities to shape territories that are attributed to technical networks, more specifically through the effective transportation of flows~\citep{bavoux2005geographie}.


%Le travail de recherche présenté propose d'explorer cette perspective de co-évolution en l'explorant par l'intermédiaire de la modélisation et de la simulation, considérant le modèle comme un instrument de connaissance à part entière \citep{banos2013pour} complémentaire aux aspects théoriques et empiriques \citep{raimbault2017applied}, et dont l'impact est amplifié par l'utilisation des méthodes et outils d'exploration des modèles et de calcul intensif \citep{pumain2017urban}. L'utilisation de modèles de simulation génératifs pour les systèmes considérés permet de construire une connaissance indirecte sur les processus impliqués et par exemple de directement tester et comparer des hypothèses dans ce laboratoire virtuel \citep{epstein1996growing}.
The research work we summarize proposes to explore this perspective of a co-evolution by studying it through the lens of modeling and simulation, considering the model as an instrument of knowledge in itself~\citep{banos2013pour} complementary to the theoretical and empirical aspects~\citep{raimbault2017applied}, and which impact is amplified by the use of methods and tools for model exploration and intensive computation~\citep{pumain2017urban}. The use of generative simulation models for the considered systems allows to construct an indirect knowledge on implied processes and for example to directly test and compare hypothesis in this virtual laboratory~\citep{epstein1996growing}.


%Suivant \cite{raimbault2017invisible} qui établit une cartographie scientifique, les domaines s'étant intéressé à la modélisation des interactions entre réseaux de transport et territoires sont très variés, de la géographie à la planification ou l'économie urbaine et plus récemment la physique, mais ceux-ci semblent être très cloisonnés \citep{raimbault2017models}. Suivant une revue systématique de la littérature et une modélographie menée par \cite{raimbault2018caracterisation}, ainsi qu'une typologie des processus d'interaction, deux échelles apparaissent pertinentes pour saisir l'existence possible de processus de co-évolution: l'échelle mesoscopique, qui s'apparentera typiquement à une échelle spatiale métropolitaine, et l'échelle macroscopique, qui correspond à celle du système de villes.
Following \cite{raimbault2017invisible} which establishes a map of scientific approaches to the question, domains which studied the modeling of interactions between transportation networks and territories are much varied, from geography to planning of urban economics, and more recently physics, but these seem to be highly isolated~\citep{raimbault2017models}. According to a systematic litterature review and a modelography done by~\cite{raimbault2018caracterisation}, and a typology of interaction processes, two scales appear as relevant to unveil the possible existence of co-evolution processes : the mesoscopic scale, which will typically be a metropolitan spatial scale, and the macroscopic scale, which corresponds to the scale of the system of cities.


%Deux directions complémentaires de modélisation correspondant à ces deux échelles ont alors été développées, étendant les travaux antérieurs de modélisation de cette co-évolution à l'échelle macroscopique \citep{baptistemodeling,schmitt2014modelisation} et mesoscopique \citep{raimbault2014hybrid}. Ces deux axes correspondent à des entrées thématiques distinctes, celle de la théorie évolutive urbaine pour le macroscopique \citep{pumain2018evolutionary} et celle de la morphogenèse urbaine pour le mesoscopique, comprise comme la relation émergente entre forme et fonction \citep{doursat2012morphogenetic}.
Two complementary modeling directions corresponding to these two scales have thus been developed, extending the previous works modeling this co-evolution at the macroscopic~\citep{baptistemodeling,schmitt2014modelisation} and mesoscopic~\citep{raimbault2014hybrid} scale. These two axes correspond to different theoretical frameworks, namely the evolutive urban theory for the macroscopic \citep{pumain2018evolutionary} and urban morphogenesis for the mesoscopic, which we understand as the emergent relation between form and function~\citep{doursat2012morphogenetic}.


%Cette contribution se veut comme une synthèse de ces travaux de recherche, et s'organise de la façon suivante: nous définissons dans un premier temps le concept de co-évolution à partir d'une vue multi-disciplinaire. Nous détaillons ensuite les résultats obtenus pour la modélisation à l'échelle macroscopique, puis ceux pour l'échelle mesoscopique. Nous discutons finalement les perspectives ouvertes et les développements futurs dans le cadre de la modélisation de la co-évolution entre réseaux de transports et territoires.
This contribution aims at giving a synthesis of these research works, and is organized the following way : we firstly define the concept of co-evolution from a multi-disciplinary viewpoint. We then detail the results obtained for the modeling at the macroscopic scale, and then the ones for the mesoscopic scale. We finally discuss the perspectives opened and the future developments in the context of modeling co-evolution between transportation networks and territories.


%%%%%%%%%%%%%%%%%%%%%%
\section{Defining co-evolution}


% Ce positionnement sur les systèmes biologiques et sociaux trouve un écho immédiat pour le concept de co-évolution. Il provient en effet de la biologie, où il a été développé à la suite de celui d'évolution, pour être utilisé plus récemment en sciences humaines et sociales. Dans quelle mesure le concept a-t-il été transféré ? Retrouve-t-on un parallèle similaire à celui entre évolution biologique et évolution culturelle ? Nous proposons pour répondre à ces questions d'apporter un bref point de vue multidisciplinaire sur la co-évolution\footnote{La démarche ici est légèrement différente de celle que nous mènerons en~\ref{sec:interdiscmorphogenesis} dans le cas de la morphogenèse, qui sera \emph{interdisciplinaire} au sens où elle cherchera à intégrer les approches, tandis que nous restons ici dans un aperçu des concepts et donc plutôt dans du \emph{multidisciplinaire}. Le concept de \emph{co-évolution} étant clé pour notre travail empirique par la suite, nous en donnerons alors une caractérisation originale et prenons le parti de ne pas tomber dans le syncrétisme intégrateur pour ce concept, mais bien de l'approcher d'un \emph{point de vue géographique}, et même plus précisément dans le cadre des systèmes territoriaux. On pourrait postuler une congruence entre la spécialisation empirique/de modélisation et celle théorique, plaçant notre processus de production de connaissance dans un profil particulier de dynamiques de domaines de connaissance (voir~\ref{sec:knowledgeframework}).}. Nous passons par la suite en revue un large spectre de disciplines, partant de la biologie où le concept a initialement trouvé son origine pour arriver progressivement à des disciplines en relation avec les sciences du territoire.
%Le transfert de concepts entre disciplines étant toujours délicat, nous prenons ici le parti de construire une définition de la co-évolution inspirée par les nombreuses disciplines dans lesquelles il a été décliné.
Transfer of concepts between disciplines is always difficult, and as co-evolution has been used by several disciplines, we propose here a definition inspired by the various ones in which it was developed. A multi-disciplinary review and a more general definition not specific to territorial systems is developed in \cite{raimbault2018co}. The detailed description of the different approaches is given in this paper, of which we give here only a broad summary.


%\subsection{Biologie}
%Le concept de co-évolution en biologie est une extension de celui bien connu d'\emph{évolution}, qui remonte à \noun{Darwin}. \cite{durham1991coevolution} (p.~22) rappelle les composantes et structure systémiques nécessaires pour qu'il y ait évolution
%\footnote{Et dans ce contexte général l'évolution n'est pas réservée à la biologie du vivant et la présence de gènes, mais aussi à des systèmes physiques vérifiant ces conditions. Nous y reviendrons plus loin.}.
%\begin{enumerate}
%	\item Processus de \emph{transmission}, impliquant des unités de transmission et des mécanismes de transmission.
%	\item Processus de \emph{transformation}, nécessitant des sources de variation.
%	\item Isolation de sous-systèmes pour que les effets des processus précédents soient observables dans des différentiations.
%\end{enumerate}
%Ainsi, une population soumise à des contraintes (souvent synthétisées conceptuellement comme une \emph{fitness}) qui conditionnent la transmission du patrimoine génétique des individus (transmission), et à des mutation génétiques aléatoires (transformation), sera bien en évolution dans les territoires spatiaux qu'elle occupe (isolation), et par extension l'espèce à laquelle on peut l'associer. La co-évolution est alors définie comme un changement évolutionnaire dans une caractéristique des individus d'une population, en réponse à un changement dans une deuxième population qui à son tour répond évolutionnairement au changement de la première, comme synthétisé par~\cite{janzen1980coevolution}, qui appuie par ailleurs la subtilité du concept et alerte contre ses utilisations injustifiées : la présence d'une congruence de deux caractéristiques qui semblent adaptées l'une à l'autre n'implique pas l'existence d'une co-évolution, l'une des deux espèces ayant pu s'adapter seule à une caractéristique déjà présente de l'autre. Il faut dans ce contexte garder en tête toute la complexité des systèmes écologiques: les populations s'insèrent dans des réseaux trophiques et des environnements, et les interactions co-évolutionnaires impliqueraient des communautés de populations d'espèces diverses, comme présenté par \cite{strauss2005toward} sous l'appellation de co-évolution diffuse.
%%De même, les dynamiques spatio-temporelles sont cruciales dans la réalisation de ces processus : \cite{dybdahl1996geography} étudient par exemple l'influence de la distribution spatiale sur les motifs de co-évolution pour un escargot et son parasite, et montrent qu'une vitesse de diffusion génétique dans l'espace plus grande pour le parasite conduit les dynamiques de co-évolution.
%%Les concepts essentiels à retenir du point de vue biologique sont ainsi : (i) existence de processus d'évolution, en particulier transmission et transformation ; (ii) dans des schémas circulaires entre populations dans le cas de la co-évolution ; et (iii) dans un cadre territorial (spatio-temporel et environnemental au sens du reste de l'éco-système) complexe.
Originating in biology, the concept of evolution requires typical features for a system to exhibit it~\citep{durham1991coevolution}, namely (i) transmission processes between agents; (ii) transformation processes; and (iii) isolation of sub-system such that differentiations emerge from the previous processes. Co-evolution then corresponds to entangled evolutionary changes in two species~\citep{janzen1980coevolution}. It was generalized to diffuse co-evolution by taking into account the broader context of numerous species in the ecological interaction network and of the environment~\citep{strauss2005toward}.

%\subsection{Evolution culturelle}
%Un parallèle entre systèmes biologiques et systèmes sociaux peut dans certains cas être pertinent. L'évolution de la culture est théorisée est explorée par un champ propre, et n'est pas en reste de dynamiques co-évolutives. \cite{Mesoudi25072017} rappelle l'état des connaissances sur le sujet et les défis à venir, comme la relation avec la nature cumulative de la culture, l'influence de la démographie dans les processus d'évolution, ou la construction de méthodes phylogénétiques permettant de reconstruire des arbres des branchements passés. Pour donner un exemple, \cite{carrignon2015modelling} introduit un cadre conceptuel pour la co-évolution de la culture et du commerce dans le cas de sociétés anciennes sur lesquelles on dispose de données archéologiques, et propose son implémentation par un modèle multi-agents dont les dynamiques sont partiellement validées par l'étude des faits stylisés produits par le modèle. La co-évolution est bien prise ici au sens d'adaptation mutuelle de structures socio-spatiales, à des échelles de temps comparables, dans ce cadre plus général d'évolution culturelle. L'évolution culturelle serait même indissociable de l'évolution génétique, puisque \cite{durham1991coevolution} postule et illustre un lien fort entre les deux, qui seraient eux-mêmes en co-évolution. \cite{bull2000meme} explorent un modèle stylisé impliquant deux populations de répliquants (les gènes et les memes) et montre l'existence de transitions de phase pour les résultats du processus d'évolution génétique lorsque l'interaction avec le répliquant culturel est forte.
%\subsection{Sociologie}
%Le concept a été utilisé en sociologie et disciplines apparentées comme les études de l'organisation, de façon analogue à celle de l'évolution culturelle. Dans le domaine de l'étude des organisations, \cite{volberda2003co} développent un cadre conceptuel de la co-évolution inter-organisationnelle en relations avec les processus de management internes, mais déplore l'absence d'études empiriques cherchant à quantifier cette co-évolution. Dans le cadre de la gestion des systèmes de production, \cite{tolio2010species} conceptualisent un chaine de production intelligente où produit, processus et système de production doivent être en co-évolution.
%\subsection{Economie géographique}
%En économie géographique, le concept de co-évolution a également largement été mobilisée. L'idée d'entités évolutionnaires en économie vient à contre-courant du courant néoclassique qui reste majoritaire, mais trouve un écho de plus en plus pertinent~\citep{nelson2009evolutionary}. \cite{schamp201020} procède à une analyse épistémologique de l'utilisation de la co-évolution, et oppose une approche néo-schumpeterienne de l'économie qui considère l'émergence de populations qui évoluent à partir de règles micro-économiques (qui correspondrait à une lecture directe et relativement isolationniste de l'évolution biologique) à une approche systémique qui considérerait l'économie comme un système évolutif de manière globale (qui correspondrait à l'évolution diffuse que nous avons développé précédemment), pour proposer une caractérisation précise tombant dans le premier cas, qui suppose des \emph{institutions} qui co-évoluent. Le plus important pour notre propos est qu'il souligne l'aspect crucial du choix des population et des entités considérées, de la zone géographique, et appuie l'importance de l'existence de relations causales circulaires.
%Il est possible de donner divers exemples d'application. \cite{doi:10.1080/00343400802662658} introduisent un cadre conceptuel pour permettre de concilier nature évolutionnaire des entreprises, théorie des clusters et réseaux de connaissance, dans lequel la co-évolution entre réseaux et entreprises est centrale, et qui est définie comme une causalité circulaire entre différentes caractéristiques de ces sous-systèmes. \cite{colletis2010co} introduit un cadre de co-évolution des territoires et de la technologie (questionnant par exemple le rôle de la proximité pour les innovations), qui révèle l'importance à nouveau de l'aspect institutionnel. 
%En économie environnementale, \cite{kallis2007coevolution} montre que des approches ``larges'' (pouvant considérer la majorité des co-dynamiques comme co-évolutives) s'opposent à des approches plus strictes (dans l'esprit de la définition donnée par \cite{schamp201020}), et que dans tous les cas une définition précise, ne venant pas forcément de la biologie, doit être donnée, en particulier pour la recherche d'une caractérisation empirique.
%\subsection{Géographie}
%Pour la géographie, comme nous l'avons déjà présenté en introduction, les travaux les plus proches empiriquement et théoriquement des notions de co-évolution sont étroitement liés à la théorie évolutive des villes. Il n'est pas évident de tracer dans la littérature à quel moment la notion a été clairement formalisée, mais il est évident qu'elle était présente dès les fondements de la théorie \citep{pumain1997pour}: le système complexe adaptatif est composé de sous-systèmes en interdépendances complexes, souvent circulairement causales. Les premiers modèles incluent bien cette vision de manière implicite, mais la co-évolution n'est pas appuyée explicitement ou définie précisément, en termes qui seraient quantifiables ou identifiables structurellement. \cite{paulus2004coevolution} amène des indices empiriques de mécanismes de co-évolution par l'étude de l'évolution des profils économiques des villes françaises. L'interprétation utilisée par~\cite{schmitt2014modelisation} repose sur une entrée par la théorie évolutive des villes, et consiste fondamentalement en une lecture des systèmes de villes comme entités fortement interdépendantes.
The concept was transferred to disciplines closer to social and human sciences, including the emerging field of cultural evolution \cite{Mesoudi25072017} for which building bricks of culture are transmitted and mutated, possibly with an interplay with biological evolution itself~\citep{bull2000meme}, but also sociology to interpret for example the interactions between social organizations as entities themselves~\citep{volberda2003co}. In the frame of evolutionary economics~\citep{nelson2009evolutionary}, the concept was also largely applied in economic geography~\citep{schamp201020}, to investigate for example the link between economic clusters and knowledge spillovers~\citep{doi:10.1080/00343400802662658}, the link between territories and technological innovation~\citep{colletis2010co}, or environmental economics issues~\citep{kallis2007coevolution}. The concept was particularly developed in geography in the frame of the evolutive urban theory \citep{pumain1997pour}. Its operational definition in that context taken for example by \cite{paulus2004coevolution} or \cite{schmitt2014modelisation} relies on systems of cities constituted by strongly coupled subsystems and entangled interactions.



% Geographie physique
%En étude des paysages, \cite{sheeren2015coevolution} parlent de co-évolution du paysage et des activités agricoles, mais ne considèrent en fait pas d'effet circulaires de l'un sur l'autre. A priori, leurs résultats montrent que l'évolution des pratiques agricoles entraine une évolution du paysage, et il n'est ainsi pas clair dans quelle mesure le cadre conceptuel de la co-évolution, mentionné sans plus de détails, est mobilisé.

% Physics
% Enfin, on peut noter de manière anecdotique que le terme de co-évolution a également été utilisé par la physique. L'utilisation pour des systèmes physiques peut porter à débat, selon que l'on suppose ou non que la transmission suppose une transmission d'\emph{information}\footnote{L'information est définie dans la théorie shanonienne comme une probabilité d'occurrence d'une chaîne de caractère. \cite{morin1976methode} montre que le concept d'information est en fait bien plus complexe, et qu'il doit être pensé conjointement à un contexte donné de génération d'un système auto-organisateur néguentropique, i.e. réalisant des diminutions locales d'entropie notamment grâce à cette information. Ce type de système est nécessairement vivant. Nous prendrons ici cette vision complexe de l'information.}. Dans le cas d'une transmission ontologique uniquement physique (\emph{êtres physiques}), alors une grande partie des systèmes physiques sont évolutifs. \cite{hopkins2008cosmological} développent un cadre cosmologique pour la co-évolution d'objets cosmiques hétérogènes dont la présence et les dynamiques sont difficilement expliquées par des théories plus classiques (certains types de galaxies, quasars, trous noirs supermassifs). \cite{antonioni2017coevolution} étudient la co-évolution entre des propriétés de synchronisation et de coopération au sein d'un réseau d'oscillateurs de Kuramoto\footnote{Le modèle de Kuramoto s'intéresse à la synchronisation au sein de systèmes complexes, en étudiant l'évolution de phases $\theta_i$ couplée par les équations d'interaction $\dot{\vec{\theta}} = \vec{\omega} + \vec{W}\left[\vec{\theta}\right] + \mathbf{B}$ où $\vec{\omega}$ sont les phases propres de forçage et la force de couplage entre $i$ et $j$ est donnée par $\vec{W}_{i} = \sum_j w_{ij} \sin\left(\theta_i - \theta_j\right)$ et $\vec{B}$ du bruit.}, montrant d'une part que le concept peut être appliqué à des objets abstraits, et d'autre part qu'un réseau de relations complexes entre variables peut être à l'origine de dynamiques présentant des causalités circulaires, c'est-à-dire d'une co-évolution en ce sens.

%\subsection{Synthesis}
%La plupart de ces approches rentrent dans la théorie des systèmes complexes adaptatifs développée par~\cite{holland2012signals} : il voit tout système comme une imbrication de systèmes de limites, filtrant des signaux ou des objets. Au sein d'une limite donnée, le sous-système correspondant est relativement autonome de l'extérieur, est est appelé \emph{niche écologique}, en correspondance directe avec les communautés fortement connectées au sein des réseaux trophiques ou écologiques. Ainsi, des entités interdépendantes au sein d'une niche sont dites en co-évolution. 
We can observe that these various definitions of co-evolution mostly correspond to the theoretical framework introduced by \cite{holland2012signals}, in which hierarchically imbricated subsystems correspond to ecological niches and are therefore containing co-evolutive dynamics.

%Nous retenons de cet aperçu multidisciplinaire de la co-évolution les points fondamentaux suivants précurseurs à une définition propre de la co-évolution.
%\begin{enumerate}
%	\item La présence de \emph{processus d'évolution} est primaire, et leur définition se ramène presque toujours à l'existence de processus de transmission et de transformation.
%	\item La co-évolution suppose des entités ou systèmes, appartenant à des classes distinctes, dont les dynamiques évolutives sont couplées de manière circulaire causale. Les approches peuvent différer selon l'hypothèse de populations de ces entités, d'objets singuliers, ou de composantes d'un système global alors en interdépendance mutuelle sans qu'il y ait circularité directe. % rq : dit-on qu'il y a coevol dans les cas spurieux
%	\item La délimitation des systèmes ou des sous-systèmes, à la fois dans l'espace ontologique (définition des objets étudiés), mais aussi dans l'espace et le temps, ainsi que leur distribution dans ces espaces, est fondamental pour l'existence de dynamiques co-évolutives, et a priori dans un grand nombre de cas, pour leur caractérisation empirique.
%\end{enumerate}


%Nous proposons ainsi la définition suivante pour le cas spécifique des réseaux de transport et des territoires, qui fait écho au trois points essentiels (existence de processus évolutifs, définition des entités ou des populations, isolation de sous-systèmes dans le temps et l'espace) ci-dessous. Celle-ci vérifie les trois spécifications suivantes.
The definition of co-evolution we construct for the particular case of transportation networks and territories is the following, similar to the definition of \cite{raimbault2018co}:

\begin{enumerate}
    \item the evolutive processes are carried by the transformation of territorial components at different scales;
    %Dans un premier temps, les processus évolutifs correspondent aux transformations des composantes du système territorial aux différentes échelles : transformation sur le temps long des villes, de leurs réseaux, transmission entre villes des caractéristiques socio-économiques portées par les agents microscopiques mais aussi transmission culturelle, reproduction et transformation des agents eux-mêmes (firmes, ménages, opérateurs).%\footnote{Cette liste s'appuie sur les hypothèses de la théorie évolutive des villes que nous avons déjà introduite brièvement et que nous développerons à part entière en Chapitre~\ref{ch:evolutiveurban}. Elle ne peut être exhaustive, puisque ce qui ferait ``l'ADN d'une ville'' reste une question ouverte comme nous le rappelle \noun{Denise Pumain} dans un entretien dédié~\ref{app:sec:interviews}.}.
	\item a co-evolution may occur in the strong coupling of such evolutive processes, with different levels of strength, namely as circular causal relation (i) between individual entities; (ii) within populations of individual, i.e. with a certain statistical sense within a given territory; (iii) between most elements of the system with a difficulty to disentangle these relations.
	%Ces processus évolutifs peuvent impliquer une co-évolution. Au sein d'un système territorial, pourront être en co-évolution à la fois : (i) des entités données (telle infrastructure et telles caractéristiques de tel territoire par exemple, c'est-à-dire des individus), lorsque leur influence mutuelle sera circulairement causale (à l'échelle leur correspondant) ; (ii) des populations d'entités, ce qui se traduira par exemple par tel type d'infrastructure et telle composante territoriale co-évoluent au niveau statistique dans une région géographique donnée ; (iii) l'ensemble des composantes d'un système à petite échelle géographique lorsqu'il existe de fortes interdépendances globales. Notre vision est donc fondamentalement \emph{multi-échelles} et articule différentes significations à différentes échelles.
	\item these different level and the spatial isolation typically enhancing evolutionary drift imply the existence of \emph{territorial niches}, i.e. niches of co-evolution, imbricated at different scales.
	%Enfin, la contrainte d'une isolation implique, en lien avec le point précédent, que la co-évolution et l'articulation des significations auront un sens s'il existe des isolations spatio-temporelles de sous-systèmes où s'effectuent les différentes co-évolutions, ce qui est en accord direct avec un vision en \emph{Systèmes de systèmes multi-échelles}.
\end{enumerate}


Our definition is strongly anchored within a dynamical vision of processes, within the initial spirit of the introduction of the concept in biology. The opening on the different levels to which the co-evolution can occur yields a generality but also a precision and the construction of empirical characterization methods, such as the one introduced for the second level (population co-evolution) by \cite{raimbault2017identification}. Furthermore, the implicit integration of the concept of niche implies a suitability with territoriality and its declination at different scales within territorial subsystems both independent and interdependent. Our approach is more general than the notion of congruence proposed by \cite{offner1993effets}, which remains fuzzy in the interdependency relations between the entities concerned, and could be similar to the third level of systemic interdependencies.







%%%%%%%%%%%%%%%%%%%%%%
\section{Macroscopic scale}


%Le premier axe de modélisation se situe à l'échelle macroscopique et se base sur les principes de la théorie évolutive des villes \citep{pumain1997pour}. La famille des modèles Simpop se place majoritairement dans les ontologies et échelles correspondantes, c'est-à-dire des entités élémentaires constituées par les villes elles-mêmes, à l'échelle spatiale du système de ville (régionale à continentale) et sur des échelles temporelles relativement longues (au dessus de la cinquantaine d'années) \citep{pumain2012multi}.
The first modeling axis relates to the macroscopic scale and is based on the principles of the evolutionary urban theory~\citep{pumain1997pour}. The family of Simpop models is mainly situated in corresponding ontologies and scales, i.e. elementary entities constituted by cities themselves, at the spatial scale of the system of cities (regional to continental) and on relatively long time scales (longer than half a century)~\citep{pumain2012multi}.


\subsection{Network effects}

%Un premier modèle de contrôle au sein duquel le réseau est statique mais ayant une retroaction sur les villes, suggère indirectement des effets de réseau. Ce travail préliminaire est détaillé par \cite{raimbault2018indirect} qui détaille le modèle et l'applique au système de ville français sur le temps long (1830-1999). Le modèle travaille sur des populations attendues et capture la complexité par les interactions non-linéaires entre villes et portées par le réseau. Trois processus se superposent pour déterminer le taux de croissance des villes: (i) une croissance endogène fixée par un paramètre, correspondant au modèle de Gibrat; (ii) des processus d'interaction directe exprimés sous la forme d'un potentiel gravitaire influençant le taux de croissance; (iii) une retroaction des flux circulant dans le réseau sur les villes traversées. Le modèle est initialisé avec les populations réelles au début d'une période, puis évalué par comparaison avec les populations simulées sur l'ensemble de la période, sur deux objectifs permettant de prendre en compte l'ajustement total des populations ou l'ajustement de leur logarithme. L'obtention de fronts de Pareto montre qu'il n'est pas possible d'ajuster le modèle uniformément pour l'ensemble du spectre de tailles de villes. On montre que l'ajout de la composante réseau permet une amélioration effective de l'ajustement, ce qui suggère l'existence d'effets de réseaux.
A first model that can be interpreted as a control, in which the network is static but has a retroaction on cities, indirectly suggests network effects. This preliminary work is detailed by~\cite{raimbault2018indirect} which details the model and applies it to the French system of cities on a long time scale (1830-1999). The model is based on expected populations and captures complexity through non-linear interactions between cities, which are carried by the network. Three processes are added to determine the growth rate of cities: (i) an endogenous growth fixed by a parameter, corresponding to the Gibrat model ; (ii) direct interaction processes described as a gravity potential which influences the growth rate ; (iii) a retroaction of flows circulating in the network on the cities traversed. The model is initialized with real populations at the start of a period, and then evaluated by comparison to simulated populations on the full period, on two objectives which allow to take into account the adjustment of the total population or of their logarithm. The production of Pareto fronts shows that it is not possible to uniformly adjust the model for the full spectrum of city sizes. We show that the addition of the network component provides an effective increase in the adjustment, what suggests network effects.


\subsection{Co-evolution model}


%Ce modèle est ensuite étendu à un modèle co-évolutif par \cite{raimbault2018modeling}, au sein duquel villes et liens du réseau de transport sont tous les deux dynamiques et en dépendance réciproque. Ce modèle est proche de~\cite{schmitt2014modelisation} pour les règles d'évolution des populations, et de~\cite{baptistemodeling} pour l'évolution du réseau. Plus précisément, il opère de manière itérative suivant les étapes suivantes: (i) les populations des villes évoluent suivant la spécification du modèle statique décrit ci-dessus; (ii) le réseau évolue, suivant une implémentation abstraite telle que les distances entre villes sont mises à jour par une fonction d'auto-renforcement en fonction des flux entre chaque ville, avec paramètre de seuil. Cette version du modèle est strictement macroscopique et n'inclut pas la forme spatiale du réseau puisqu'elle agit sur la matrice des distances uniquement.
This model is then extended to a co-evolutive model by~\cite{raimbault2018modeling}, in which cities and links of the transportation network are both dynamic and within a reciprocal dependency. This model is close to~\cite{schmitt2014modelisation} for the rules of population evolution, and to~\cite{baptistemodeling} for the evolution of the network. More precisely, it operates iteratively with the following steps : (i) population of cities evolve according the the specification of the static model described above ; (ii) the network evolves, following an abstract implementation such that distances between cities are updated with a self-reinforcement function depending on flows between each city, with a threshold parameter. This version of the model is strictly macroscopic and does not include the spatial form of the network since it updates the distance matrix only.


%L'exploration systématique de ce modèle par l'intermédiaire du logiciel OpenMOLE \citep{reuillon2013openmole} et l'application d'une méthode empirique de caractérisation de la co-évolution \citep{raimbault2017identification} permettent de montrer qu'il capture une grande variété de dynamiques couplées, incluant effectivement des dynamiques co-évolutives: parmi la grande richesse de régimes d'interaction entre variables de réseau et variables de population (au sens de \cite{raimbault2017identification}, par classification des motifs de corrélations retardées), plus de la moitié des régimes sont effectivement co-évolutifs, c'est-à-dire présentant des causalités circulaires. Cet aspect pourrait paraître évident pour un modèle conçu pour intégrer une co-évolution, mais il faut réaliser que les processus intégrés dans les modèles le sont à l'échelle microscopique tandis que la caractérisation est quantifiée à l'échelle macroscopique: il s'agit bien d'une capacité du modèle à faire émerger la co-évolution. Par exemple pour le modèle SimpopNet, \cite{raimbault2018unveiling} montre que les régimes de co-évolution sont beaucoup plus rares et que cet autre modèle produit plus souvent des configurations de type effet structurant ou sans aucune relation. Par ailleurs, l'exploration révèle l'existence d'une valeur optimale pour un paramètre de portée des interactions, à laquelle le système présente une complexité maximale des trajectoires des villes. Cette portée correspond à l'apparition de niches territoriales, au sein desquelles les population sont en co-évolution, correspondant au troisième point de notre définition.
The systematic exploration of this model using the OpenMOLE software~\citep{reuillon2013openmole} and the application of an empirical method to characterize co-evolution~\citep{raimbault2017identification} allow us to show that it captures a large variety of coupled dynamics, including indeed co-evolutive dynamics: among the broad range of interaction regimes between network variables and population variables (in the sense of \cite{raimbault2017identification}, by classifying lagged correlation patterns), more than half of the regimes are effectively co-evolutive, i.e. exhibit circular causalities. This aspect could appear as trivial for a model conceived to integrate a co-evolution, but one has to realize that processes integrated in models are at the microscopic scale whereas the co-evolution is quantified at the macroscopic scale: it is indeed a property of the model to make co-evolution emerge. For example in the case of the SimpopNet model~\citep{schmitt2014modelisation}, \cite{raimbault2018unveiling} shows that co-evolution regimes are much more rare and that this other model produces more often configurations of the type structuring effects or without any relation. Furthermore, the exploration unveils the existence of an optimal value for an interaction range parameter, at which the system exhibits a maximal complexity of city trajectories. This range correspond to the appearance of territorial niches, within which populations are co-evolving, corresponding to the third point of our definition.


%La calibration sur le système de ville français sur la même durée que le modèle statique, avec données de population et réseaux ferroviaire dynamique pris en compte par matrices de distance dynamiques construite à partir des données de \cite{thevenin2013mapping}, révèle des fronts de Pareto entre objectif pour la distance entre villes et objectif de population, suggérant une impossibilité de calibrer ce type de modèle simultanément pour la composante réseau et pour celle territoriale. Par ailleurs, l'ajustement pour la population est amélioré par ce modèle pour un certain nombre de périodes, par rapport au modèle avec réseau statique, suggérant la pertinence de la prise en compte des dynamiques co-évolutives. L'évolution du paramètre de seuil calibré suggère la capture d'un ``effet TGV'' par le modèle, c'est-à-dire l'amélioration de la distance effective pour les métropoles concernées mais une perte de vitesse pour les territoires laissés pour compte.
The calibration on the French system of cities on the same time period than the static model, with population data and dynamical railway network data taken into account with dynamical distance matrices constructed from the database of \cite{thevenin2013mapping}, unveils Pareto fronts between the objective on the distance between cities and population objective, suggesting an impossibility to simultaneously calibrate this type of models for the network component and for the territorial component. Moreover, the adjustment on populations is improved by this model for a certain number of periods, compared to the model with a static network, suggesting the relevance of taking into account co-evolutive dynamics. The evolution of the calibrated threshold parameter suggests that the model captures a ``High Speed Rail (TGV) effect'', i.e. an increase in the effective accessibility for the metropolitan areas concerned but a loss of speed for the territories left behind.


%%%%%%%%%%%%%%%%%%%%%%
\section{Mesoscopic scale}


%Le deuxième axe, à l'échelle mesoscopique, considère l'entrée par la morphogenèse urbaine, comprise comme l'émergence simultanée de la forme et de la fonction d'un système \citep{doursat2012morphogenetic}. Celle-ci permet de considérer une description plus fine des territoires, à l'échelle de grilles fines de population (résolution 500m) et de représentation vectorielle du réseau à la même échelle. 
The second axis, at the mesoscopic scale, considers the approach through urban morphogenesis, understood as the simultaneous emergence of the form and the function of a system \citep{doursat2012morphogenetic}. It allows to consider a more precise description of territories, at the scale of fine population grids (500m resolution) and of a vectorial representation of the network at the same scale.


\subsection{Morphogenesis by aggregation-diffusion}

%Les systèmes territoriaux produits sont quantifiés par indicateurs morphologiques pour la population \citep{le2015forme}. Un premier modèle préliminaire, intégrant la population uniquement, montre que les processus d'agrégation et diffusion sont suffisant pour expliquer la grande majorité des formes urbaines existantes en Europe \citep{raimbault2018calibration}. Ce résultat suggère que la prise en compte de la forme seule peut être effectuée de manière autonome, mais que les processus fonctionnels ne seront alors pas pris en compte au coeur des dynamiques. Afin d'appréhender les processus de morphogenèse, c'est-à-dire le lien fort entre forme et fonction lors de l'émergence de celles-ci, nous prenons le parti d'utiliser le réseau de transport comme proxy des propriétés fonctionnelles des territoires, notamment par les propriétés de centralités. Cela nous amène à introduire un modèle de morphogenèse par co-évolution à l'échelle mesoscopique.
The territorial systems produced are quantified with morphological indicators for the population \citep{le2015forme}. A first preliminary model, integrating only the population, shows that aggregation and diffusion processes are sufficient to explain a large majority of urban forms existing in Europe \citep{raimbault2018calibration}. This result suggests that taking into account the form only can be achieved in an autonomous way, but that functional processes will then not be taken into account in the core of dynamics. In order to grasp morphogenesis processes, i.e. the strong link between form and function during the emergence of these, we follow the idea of using the transportation network as a proxy of functional properties of territories, in particular through centrality properties. This leads us to consider a morphogenesis model by co-evolution at the mesoscopic scale.


\subsection{Morphogenesis by co-evolution}


%Les indicateurs de forme urbaine calculés sur fenêtres glissante de taille 50km pour l'ensemble de l'Europe sont complétés par des indicateurs structurels du réseau routiers, calculés à partir des données OpenStreetMap après un algorithme spécifique de simplification conservant les propriétés topologiques \citep{raimbault2018urban}. Ces indicateurs et leur corrélations spatiales statiques sont ainsi calculés sur des fenêtres de taille équivalente couvrant l'ensemble de l'Europe. Nous montrons par l'étude de la distribution spatiale de ces corrélations la non-stationnarité des processus d'interaction au second ordre, confirmant la pertinence de la notion de niche en tant que sous-système territorial cohérent.
The urban form indicators computed on moving windows of size 50km for the whole Europe are completed with structural network indicators for the road network, computed from OpenStreetMap data after application of a specific simplification algorithm which conserves topological properties \citep{raimbault2018urban}. These indicators and their static spatial correlations are thus computed on windows of a similar size covering the whole Europe. We show through the study of the spatial study of these correlations the non-stationarity of interaction processes at the second order, confirming the relevance of the notion of niche as a consistent territorial sub-system.

%Nous introduisons alors dans \cite{raimbault2018urban} un modèle de morphogenèse capturant la co-évolution de la distribution spatiale de la population et du réseau routier. Ce modèle combine la logique de \cite{raimbault2018calibration} pour la complémentarité des processus d'agrégation et de diffusion à celle de \cite{raimbault2014hybrid} pour la structure hybride en grille et réseau vectoriel ainsi l'influence de variables explicatives locales sur l'évolution territoriale. Ce modèle fonctionne de la façon suivante: (i) les propriétés morphologiques et fonctionnelles locales, intégrées comme variables explicatives locales normalisées (incluant population, distance au réseau, centralité de chemin, centralité de proximité, accessibilité), déterminent la valeur d'une fonction d'utilité à partir de laquelle est ajoutée la nouvelle population par attachement préférentiel, puis une diffusion des populations est effectuée; (ii) le réseau routier évolue suivant des règles dépendant de différentes heuristiques (approche par multi-modélisation), qui incluent, après l'ajout de noeuds préférentiellement à la nouvelle population et leur connection directe au réseau existant, un ajout de liens par réseau aléatoire, rupture de potentiel aléatoire \citep{schmitt2014modelisation}, rupture de potentiel déterministe \citep{raimbault2016generation}, réseau biologique auto-organisé \citep{tero2010rules}, compromis coût-bénéfices \citep{louf2013emergence}.
We then introduce in \cite{raimbault2018urban} a morphogenesis model capturing the co-evolution of the spatial distribution of population and of the road network. This model combines the logic of \cite{raimbault2018calibration} for the complementarity of aggregation and diffusion processes to the one of \cite{raimbault2014hybrid} for the hybrid grid and vectorial network structure and also the influence of local explicative variables on the territorial evolution. This model works the following way: (i) morphological and functional local properties, integrated as local normalized explicative variables (including population, distance to the network, betweenness centrality, closeness centrality, accessibility), determine the value of a utility function from which new population is added through preferential attachment, and a diffusion of population is achieved; (ii) the road network evolves following rules depending on different heuristics (multi-modeling approach), which include, after the addition of nodes preferentially to the new population and their direct connection to the existing network, an addition of links with an heuristic among random links, random potential breakdown \citep{schmitt2014modelisation}, deterministic potential breakdown \citep{raimbault2016generation}, biological self-organized network \citep{tero2010rules}, cost-benefit compromises \citep{louf2013emergence}.


%Le calcul d'indicateurs topologiques pour le réseau routier permet de calibrer le modèle, et \cite{raimbault2018multi} montre que les différents processus de croissance de réseau qui ont été inclus suivant un processus de multi-modélisation sont complémentaires pour se rapprocher du maximum de configurations réelles en termes de réseau. Par ailleurs, le modèle est calibré simultanément sur indicators morphologiques, indicateurs topologiques, et leur matrices de corrélation, et les relativement faibles distances aux données pour un nombre non négligeable de points suggèrent que le modèle est capable de reproduire les résultats des processus au premier ordre (indicateurs) mais aussi au second ordre (interactions entre indicateurs). Concernant les régimes de causalité produits par le modèle, nous obtenons des configurations correspondant à une co-évolution, mais une diversité beaucoup plus faible que pour le modèle plus simple de \cite{raimbault2014hybrid} qui a servi de modèle jouet pour la démonstration de la pertinence de la méthode des régimes de causalité dans \cite{raimbault2017identification}, suggérant une tension entre performance statique (reproduction des résultats des processus) et performance dynamique (reproduction des processus eux-mêmes) pour ce type de modèles.
The computation of topological indicators for the road network allows to calibrate the model, and \cite{raimbault2018multi} shows that the different network growth processes which have been included following the multi-modeling procedure are complementary to reach a maximum a real network configurations. Furthermore, the model is simultaneously calibrated on morphological indicators, topological indicators, and their correlation matrices, and the relatively low distances to data for a non negligible number of points suggest that the model is able to reproduce outcomes of processes at the first order (indicators) but also at the second order (interactions between indicators). Regarding the causality regimes produced by the model, we obtain configurations corresponding to a co-evolution, but a much lower diversity than for the more simple model of \cite{raimbault2014hybrid} which was used as a toy model to show the relevance of the method of causality regimes in \cite{raimbault2017identification}, suggesting a tension between static performance (reproduction of outcomes of processes) and dynamical performance (reproduction of processes themselves) for this kind of models.
 
 

\subsection{Transportation governance}
 
%Enfin, un dernier modèle métropolitain (modèle Lutecia) est décrit dans \cite{raimbault2018caracterisation}, étendant celui proposé par \citep{lenechet:halshs-01272236}, explore le rôle des processus de gouvernance dans la croissance du réseau de transport, au sein d'un modèle de co-évolution. Ici, l'échelle métropolitaine correspond bien à celle mesoscopique. La collaboration entre acteurs locaux pour la construction des infrastructures de transport est prise en compte par l'intermédiaire de théorie des jeux. Cela permet de simuler l'émergence du réseau de transport et son interaction avec la forme urbaine quantifiée par les motifs spatiaux d'accessibilité. Ce modèle permet par exemple de montrer que les dynamiques co-évolutives peuvent être amenées à inverser le comportement des gains d'accessibilité en comparaison à une configuration sans évolution de l'usage du sol, c'est-à-dire changer qualitativement le régime du système métropolitain. La calibration de ce modèle sur le cas stylisé de la méga-région urbaine du Delta de la Rivière des Perles en Chine permet d'extrapoler sur les processus de gouvernance, et suggère que la forme du réseau actuel est plus probablement due soit à des décisions totalement régionales soit totalement locales, mais pas de configuration intermédiaire, en contradiction avec la vision du contexte politique Chinois comme système de décision multi-niveau \citep{liao2017ouverture}.
Finally, a last metropolitan model (Lutecia model) is described in \cite{raimbault2018caracterisation}, extending the one proposed by \citep{lenechet:halshs-01272236}. It explores the role of governance processes in the growth of the transportation network, within a co-evolution model. Here, the metropolitan scale indeed corresponds to the mesoscopic scale. The collaboration between local actors for the construction of transportation infrastructures is included using game theory paradigms. This allows to simulate the emergence of the transportation network and its interaction with the urban form quantified by spatial accessibility patterns. This model allows for example to show that co-evolutive dynamics can in some case lead to the inversion of the behavior of accessibility gains in comparison to a situation without evolution of land-use, i.e. to qualitatively change the regime of the metropolitan system. The calibration of this model on the stylized case of the mega-urban region of Pearl River Delta in China allows to extrapolate on governance processes, and suggests that the shape of the current network is more likely to be the consequence either of fully regional decisions or of fully local decisions, but never an intermediate configuration, contradicting the view of the Chinese political context as a multi-level decision system~\citep{liao2017ouverture}.



%%%%%%%%%%%%%%%%%%%%%%
\section{Perspectives}

%Cette recherche a ainsi développé des approches complémentaires à différentes échelles des interactions entre réseaux de transport et territoires en modélisant leur co-évolution. Nous détaillons finalement des perspectives de développement immédiats ouverts par ce travail.
This research therefore developed complementary approaches at different scales of interactions between transportation networks and territories by modeling their co-evolution. We finally detail some immediate development perspectives opened by this work.


\subsection{Developments at the macroscopic scale}


Our macroscopic models have not been tested yet on other urban systems and other time spans, and future developments will have to study which conclusions obtained here are specific to the French urban system on these periods, and which are more general are could be more generic in systems of cities. The application of the model to other systems of cities also recalls the difficulty to define urban systems. In our case, a strong bias must be induced by the fact of considering France only, since the insertion of its urban system within an European system is a reality we had to neglect. The span and scale of such models is always a difficult subject. We rely here on the administrative consistence and on the consistence of the database \citep{pumain1986fichier}, but the sensitivity to the definition of the system and to its spatial extent must still be tested.


Furthermore, the calibration used only the railway network for the distances between cities. Considering a single transportation mode is naturally a reduction, and an immediate direction for developments is testing the model with real distance matrices for other types of networks, such as the freeway network which has undergone a significant growth in France in the second half of the 20th century. This application would necessitate the construction of a dynamical database for the freeway network spanning 1950-2015, since classical bases (IGN or OpenStreetMap) do not integrate the opening date of segments. A natural extension of the model would then consist in the implementation of a multilayer network, what is a typical approach to represent multi-modal transportation networks~\citep{gallotti2014anatomy}. Each layer of the transportation network should have a co-evolutive dynamic with populations, possibly with the existence of inter-layer dynamics.



\subsection{Developments at the mesoscopic scale}


The issue of the generic character of the co-evolution morphogenesis model is also opened, i.e. if it would work similarly to reproduce urban forms on very different systems such as the United States or China. A first interesting development would be to test it on these systems and at slightly different scales (1km cell size for example).


Furthermore, the Lutecia model is also a fundamental contribution towards the inclusion of more complex processes implied in co-evolution, such as the governance of the transportation system. This model paves the way to a new generation of models, that could potentially become operational in the case of regional systems with a very high evolution speed such as in the Chinese case.


\subsection{Towards multi-scale models}


Our work finally opens perspectives for integrated approaches, towards multi-scale models of these interactions, which appear to be more and more necessary for the construction of operational models that can be applied to the design of sustainable planning policies \citep{rozenblat2018conclusion}.


A first contribution towards multi-scalar models would be to take into account in a finer way the physical network in macroscopic models, what is for example the object of~\cite{mimeur:tel-01451164}, which produces interesting results regarding the influence of the centralization of network investment decision-making on final forms, but keeps static populations and does not produce a co-evolution model. Similarly, the choice of indicators to quantify the distance of the simulated network to a real network is a difficult question in that context: indicators such as the number of intersections taken by~\cite{mimeur:tel-01451164} is associated to procedural modeling and does not reflect structural indicators. It is probably for the same reason that~\cite{schmitt2014modelisation} is only interested in population trajectories and not in network indicators: the superposition and adjustment of population and network dynamics at different scales remains an open problem.


A second entry consists in the integration of the mesoscopic morphogenesis model into population of models in interaction. This approach would allow to take into account the non-stationarity of territorial systems that we moreover showed empirically. Here at the mesoscopic scale, total population and growth rate are fixed by exogenous conditions due to processes at the macroscopic scale. It is in particular the aim of spatial growth models such as the macroscopic model we already introduced to determine such parameters through the relations between cities as agents. It would then be possible to condition the morphological development of each area to the values of parameters determined at the upper level. In that context, one must remain careful on the role of emergence: should the emergent urban form influence the macroscopic behavior in its turn ? Such complex multi-scalar models are promising but must be considered with caution for the level of complexity required and the way to couple scales.


\section{Conclusion}

%Nous avons ici fait une synthèse des principaux résultats de \cite{raimbault2018caracterisation}, confirmant la pertinence de l'approche par co-évolution dans l'appréhension des interactions entre réseaux de transport et territoires, en particulier pour la modélisation de ces interactions. Nous avons développé une définition précise de la co-évolution, par ailleurs associée à une méthode de caractérisation empirique. Les modèles à différentes échelles, et la perspective de modèles multi-échelles, pourront devenir des outils précieux de prospective territoriale dans le cadre des transitions durables sur le temps long, permettant de quantifier les systèmes territoriaux possibles et ceux souhaitables au regard de la soutenabilité, en prenant en compte des processus et des échelles relativement délaissés dans la littérature.
We did here a synthesis of main results of \cite{raimbault2018caracterisation}, confirming the relevance of the co-evolution approach to focus on interactions between transportation networks and territories, in particular to model these interactions. We developed a precise definition of co-evolution, moreover associated to an empirical characterization method. The models at different scales, and the perspective of multi-scale models, can become precious tools for territorial prospective in the context of sustainable transitions on long time, allowing to quantify the possible territorial systems and the one desirables regarding sustainability, taking into account scales and processes still relatively poorly tackled in the literature.


%%%%%%%%%%%%%%%%%%%%
%% Biblio
%%%%%%%%%%%%%%%%%%%%
%\tiny

%\begin{multicols}{2}

%\setstretch{0.3}
%\setlength{\parskip}{-0.4em}


\bibliographystyle{apalike}
\bibliography{biblio}
%\end{multicols}



\end{document}
