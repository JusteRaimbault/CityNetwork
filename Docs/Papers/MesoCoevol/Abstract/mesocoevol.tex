\documentclass[10pt]{article}

% general packages without options
\usepackage{amsmath,amssymb,bbm}
% graphics
\usepackage{graphicx}
% text formatting
\usepackage[document]{ragged2e}
\usepackage{pagecolor,color}

\newcommand{\noun}[1]{\textsc{#1}}

\usepackage[utf8]{inputenc}
\usepackage[T1]{fontenc}
% geometry
\usepackage[margin=2cm]{geometry}

\usepackage{multicol}
\usepackage{setspace}

\usepackage{natbib}
\setlength{\bibsep}{0.0pt}

%\usepackage[french]{babel}

% layout : use fancyhdr package
%\usepackage{fancyhdr}
%\pagestyle{fancy}

% variable to include comments or not in the compilation ; set to 1 to include
\def \draft {1}


% writing utilities

% comments and responses
%  -> use this comment to ask questions on what other wrote/answer questions with optional arguments (up to 4 answers)
\usepackage{xparse}
\usepackage{ifthen}
\DeclareDocumentCommand{\comment}{m o o o o}
{\ifthenelse{\draft=1}{
    \textcolor{red}{\textbf{C : }#1}
    \IfValueT{#2}{\textcolor{blue}{\textbf{A1 : }#2}}
    \IfValueT{#3}{\textcolor{ForestGreen}{\textbf{A2 : }#3}}
    \IfValueT{#4}{\textcolor{red!50!blue}{\textbf{A3 : }#4}}
    \IfValueT{#5}{\textcolor{Aquamarine}{\textbf{A4 : }#5}}
 }{}
}

% todo
\newcommand{\todo}[1]{
\ifthenelse{\draft=1}{\textcolor{red!50!blue}{\textbf{TODO : \textit{#1}}}}{}
}


\makeatletter


\makeatother


\begin{document}







\title{An Urban Morphogenesis Model Capturing Interactions between Networks and Territories\\\medskip
\textit{Abstract proposal for Mathematics of Urban Morphology}
}

\author{\noun{Juste Raimbault}$^{1,2}$
\\
$^1$ UMR CNRS 8504 Géographie-cités\\
$^2$ UMR-T IFSTTAR 9403 LVMT
}
\date{}

\maketitle

\justify

\pagenumbering{gobble}


\textbf{Keywords : }\textit{Urban Morphogenesis; Reaction-diffusion; Population distribution; Transportation Network; Co-evolution}

\bigskip

The strong coupling between urban development and transportation networks has been largely documented in the empirical literature, yet few models to simulate urban growth include it. We postulate that these two dimensions are complementary aspects of urban morphogenesis, and introduce accordingly a generative model of urban growth. The model is at a mesoscopic scale, in the sense of a fine resolution (under 1km) and intermediate spatial extent (50$\sim$100km). Given a fixed exogenous growth rate, population are dispatched following a preferential attachment that depends on a potential controlled by local urban form (density, distance to network) and network measures (centralities and generalized accessibilities), and then diffused in space to capture urban sprawl. Network growth is included through a multi-modeling paradigm: implemented heuristics include biological network generation and gravity potential breakdown.


Besides, we introduce a typology of territories, based on coupled urban form and network topology measures. It is established on windows spanning full European Union, given population density data and OpenStreetMap road network. This typology is used to calibrate the model: we proceed to a static calibration, in the sense that final configurations only are fitted, but both at the first (measures) and second (correlations) order. The later captures indirectly relations between the network and the urban frame. The model is able to reproduce most of existing configurations. The study of lagged correlations within synthetic model runs furthermore unveils a large variety of Granger causality regimes, confirming the ability of the model to grasp the complexity of situations existing empirically.


%%%%%%%%%%%%%%%%%%%%
%% Biblio
%%%%%%%%%%%%%%%%%%%%
%\tiny

%\begin{multicols}{2}

%\setstretch{0.3}
%\setlength{\parskip}{-0.4em}


%\bibliographystyle{apalike}
%\bibliography{/Users/Juste/Documents/ComplexSystems/CityNetwork/Biblio/Bibtex/CityNetwork}%,biblio}
%\end{multicols}



\end{document}
