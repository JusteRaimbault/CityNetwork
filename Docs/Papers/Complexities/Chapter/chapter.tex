%%%%%%%%%%%%%%%%%%%%%%%%%%%%%
% Standard header for working papers
%
% WPHeader.tex
%
%%%%%%%%%%%%%%%%%%%%%%%%%%%%%

\documentclass[11pt]{article}



%%%%%%%%%%%%%%%%%%%%%%%%%%
%% TEMPLATES
%%%%%%%%%%%%%%%%%%%%%%%%%%


% Simple Tabular

%\begin{tabular}{ |c|c|c| } 
% \hline
% cell1 & cell2 & cell3 \\ 
% cell4 & cell5 & cell6 \\ 
% cell7 & cell8 & cell9 \\ 
% \hline
%\end{tabular}





%%%%%%%%%%%%%%%%%%%%%%%%%%
%% Packages
%%%%%%%%%%%%%%%%%%%%%%%%%%



% encoding 
\usepackage[utf8]{inputenc}
\usepackage[T1]{fontenc}


% general packages without options
\usepackage{amsmath,amssymb,amsthm,bbm}

% graphics
\usepackage{graphicx,transparent,eso-pic}

% text formatting
\usepackage[document]{ragged2e}
\usepackage{pagecolor,color}
%\usepackage{ulem}
\usepackage{soul}


% conditions
\usepackage{ifthen}


\usepackage{natbib}


%%%%%%%%%%%%%%%%%%%%%%%%%%
%% Maths environment
%%%%%%%%%%%%%%%%%%%%%%%%%%

%\newtheorem{theorem}{Theorem}[section]
%\newtheorem{lemma}[theorem]{Lemma}
%\newtheorem{proposition}[theorem]{Proposition}
%\newtheorem{corollary}[theorem]{Corollary}

%\newenvironment{proof}[1][Proof]{\begin{trivlist}
%\item[\hskip \labelsep {\bfseries #1}]}{\end{trivlist}}
%\newenvironment{definition}[1][Definition]{\begin{trivlist}
%\item[\hskip \labelsep {\bfseries #1}]}{\end{trivlist}}
%\newenvironment{example}[1][Example]{\begin{trivlist}
%\item[\hskip \labelsep {\bfseries #1}]}{\end{trivlist}}
%\newenvironment{remark}[1][Remark]{\begin{trivlist}
%\item[\hskip \labelsep {\bfseries #1}]}{\end{trivlist}}

%\newcommand{\qed}{\nobreak \ifvmode \relax \else
%      \ifdim\lastskip<1.5em \hskip-\lastskip
%      \hskip1.5em plus0em minus0.5em \fi \nobreak
%      \vrule height0.75em width0.5em depth0.25em\fi}



%% Commands

\newcommand{\noun}[1]{\textsc{#1}}


%% Math

% Operators
\DeclareMathOperator{\Cov}{Cov}
\DeclareMathOperator{\Var}{Var}
\DeclareMathOperator{\E}{\mathbb{E}}
\DeclareMathOperator{\Proba}{\mathbb{P}}

\newcommand{\Covb}[2]{\ensuremath{\Cov\!\left[#1,#2\right]}}
\newcommand{\Eb}[1]{\ensuremath{\E\!\left[#1\right]}}
\newcommand{\Pb}[1]{\ensuremath{\Proba\!\left[#1\right]}}
\newcommand{\Varb}[1]{\ensuremath{\Var\!\left[#1\right]}}

% norm
\newcommand{\norm}[1]{\left\lVert #1 \right\rVert}



% argmin
\DeclareMathOperator*{\argmin}{\arg\!\min}


% amsthm environments
\newtheorem{definition}{Definition}
\newtheorem{proposition}{Proposition}
\newtheorem{assumption}{Assumption}

%% graphics

% renew graphics command for relative path providment only ?
%\renewcommand{\includegraphics[]{}}


\usepackage{url}





% geometry
\usepackage[margin=2cm]{geometry}



% changes

\usepackage{soul}
\soulregister\cite7
\soulregister\citep7
\soulregister\ref7

\usepackage[final]{changes}
%\usepackage{changes}


\setaddedmarkup{\textcolor{black}{\hl{#1}}}
\setdeletedmarkup{\textcolor{red}{\sout{#1}}}



\usepackage{CJKutf8}
%\begin{CJK*}{UTF8}{zhsong}
%文章内容。
%\clearpage\end{CJK*}
\newcommand{\cn}[1]{
  \begin{CJK*}{UTF8}{gbsn}
  #1
  \end{CJK*}
}



% layout : use fancyhdr package
%\usepackage{fancyhdr}
%\pagestyle{fancy}
%
%\makeatletter
%
%\renewcommand{\headrulewidth}{0.4pt}
%\renewcommand{\footrulewidth}{0.4pt}
%\fancyhead[RO,RE]{}
%\fancyhead[LO,LE]{Models for the co-evolution of cities and networks}
%\fancyfoot[RO,RE] {\thepage}
%\fancyfoot[LO,LE] {}
%\fancyfoot[CO,CE] {}
%
%\makeatother
%

%%%%%%%%%%%%%%%%%%%%%
%% Begin doc
%%%%%%%%%%%%%%%%%%%%%

\begin{document}






\title*{Relating complexities for the reflexive study of complex systems}
\titlerunning{Relating complexities} 
\author{Juste Raimbault}
% Use \authorrunning{Short Title} for an abbreviated version of
% your contribution title if the original one is too long
\institute{Juste Raimbault \at UPS CNRS 3611 ISC-PIF and UMR CNRS 8504 G{\'e}ographie-cit{\'e}s, \email{juste.raimbault@polytechnique.edu}
}
%
% Use the package "url.sty" to avoid
% problems with special characters
% used in your e-mail or web address
%
\maketitle


\abstract{Several approaches and corresponding definitions of complexity have been developed in different fields. Urban systems are the archetype of complex socio-technical systems concerned with these different viewpoints. We suggest in this chapter that links between three types of complexity, namely emergence, computational complexity and informational complexity, can establish. We discuss the implication of these links on the necessity of reflexivity to produce a knowledge of the complex, and how this connects to the interdisciplinary of approaches in particular for socio-technical systems. We finally synthesize this positioning under a new epistemological framework called \emph{applied perspectivism}.
\medskip\\
\textbf{Keywords : }\textit{Complexity; Socio-technical Systems; Reflexivity; Knowledge of Knowledge}
}





%%%%%%%%%%%%%%%%%%%%
\section{Introduction}


The many disciplines concerned with the study of urban systems have the common point of having postulated their \emph{complexity}: Batty's ``new science of cities'' is rooted within complexity science paradigms \cite{batty2007cities,batty2013new}; Pumain's evolutive urban theory makes the fundamental assumption of urban systems as co-evolving complex systems \cite{pumain2017geography,pumain1997pour}; Haken's synergetics have application in urban space cognition for example \cite{e18060197}; recent works by physicists have applied tools from statistical physics to urban systems \cite{west2017scale}; architecture has a long tradition in integrating complexity in the urban fabric \cite{alexander1977pattern}; the new economic geography is aware of the complexity of its objects of study despite its reductionism in methods used \cite{krugman1994complex} and has developed an evolutionary branch \cite{cooke2018evolutionary}; artificial life  to give a few examples.

What is meant by complexity remains however not well defined, and as \cite{chu2008criteria} recalls several definitions coexist and there is a priori no reason to think that these could converge. Interestingly, a reductionist view on this matter would hope for some unified definition, whereas most complexity approaches will take this diversity as an asset to be developed (take for example the complex thinking advocated by \cite{morin1991methode} which relies on the progressive integration of multiple disciplines and thus viewpoints on the world, towards a dissolution of these disciplines). The previous chapter \cite{batty2018which} has investigated this issue by providing a broad historical overview of complexity approaches to urban systems. \cite{manson2001simplifying} develops different notions of complexity and their potential in the study of geographical systems.

We propose here a theoretical entry to a limited but effective integration of some approaches to complexity to produce a framework to study complex socio-technical systems. More precisely, we illustrate links between three types of complexities (emergence, computational complexity, and informational complexity, which choice will be discuss below), and develop the implication of these links on the production of a knowledge of the complex. This work is at a somehow general level since we introduce a development modestly contributing (i.e. in our context of the study of urban systems) to the \emph{knowledge of knowledge} \cite{edgar1986methode}. An implicit aim is to interrogate the links between complexity and processes of knowledge production.


Our argument will (i) establish some links between different types of complexities; (ii) explore the necessity of reflexivity possibly as a consequence of these links; (iii) suggest a practical framework to apply these principles to the construction of integrative approaches to complex systems. Therefore, we will sequentially follow this outline:
\begin{enumerate}
	\item introduce more precisely the types of complexities we consider;
	\item detail the links which are currently not obvious in the literature (i.e. between computational complexity and informational complexity with emergence);
	\item recall reasons in favor of the necessity of reflexivity;
	\item sketch how the production of knowledge on complex systems falls at the intersection of different complexities and how this implies reflexivity;
	\item develop implications for interdisciplinarity and introduce the corresponding applied perspectivism framework.
\end{enumerate}







%%%%%%%%%%%%%%%%%%%%%%%%
\section{Complexity and complexities}
%%%%%%%%%%%%%%%%%%%%%%%%


What is meant by complexity of a system often leads to misunderstandings since it can be qualified according to different dimensions and visions. We distinguish first the complexity in the sense of weak emergence and autonomy between the different levels of a system, and on which different positions can be developed as in \cite{deffuant2015visions}. We will not enter a finer granularity, the vision of social complexity giving even more nightmares to the Laplace daemon, and since it can be understood as a stronger emergence (in the sense of weak and strong emergence as viewed by \cite{bedau2002downward}).

We thus simplify and assume that the nature of systems plays a secondary role in our reflexion, and therefore consider complexity in the sense of an emergence. This choice answers the following rationale: % TODO link with sociophysics manifest ?


Moreover, we distinguish two other ``types'' of complexity, namely computational complexity and informational complexity, that can be seen as measures of complexity, but that are not directly equivalent to emergence, since there exists no systematic link between the three. We can for example consider the use of a simulation model, for which interactions between elementary agents translate as a coded message at the upper level: it is then possible by exploiting the degrees of freedom to minimize the quantity of information contained in the message. The different languages require different cognitive efforts and compress the information in a different way, having different levels of measurable complexity~\cite{febres2013complexity}. In a similar way, architectural artefacts are the result of a process of natural and cultural evolution, and witness more or less this trajectory.


Numerous other conceptual or operational characterizations of complexity exist, and it is clear that the scientific community has not converged on a unique definition~\cite{chu2008criteria}, which proposes to continue exploring the different existing approaches, as proxies of complexity in the case of an essentialism, or as concepts in themselves. This approach that is in a certain way reflexive, since the complexity should emerge naturally from the interaction between these different approaches studying complexity, hence the reflexivity. We propose to focus on these three concepts in particular, for which the relations are already not evident.



Indeed, links between these three types of complexity are not systematic, and depend on the type of system. Epistemological links can however be introduced. We will develop the links between emergence and the two other complexities, since the link between computational complexity and informational complexity is relatively well explored, and corresponds to issues in the compression of information and signal processing, or moreover in cryptography.


\cite{thurner2017three} three approaches to entropy do not coincide for complex enough systems.



\section{Computational complexity and emergence}

Different clues suggest a certain necessity of computational complexity to have emergence in complex systems, whereas reciprocally a certain number of adaptive complex systems have high computational capabilities.



A first link where computational complexity implies emergence is suggested by an algorithmic study of fundamental problems in quantum physics. Indeed, \cite{2014arXiv1403.7686B} shows that the resolution of the Schrödinger equation with any Hamiltonian is a NP-hard and NP-complete problem, and thus that the acceptation of $\mathbf{P}\neq\mathbf{NP}$ implies a qualitative separation between the microscopic quantum level and the macroscopic level of the observation. Therefore, it is indeed the complexity (here in the sense of their computation) of interactions in a system and its environment that implies the apparent collapse of the wave function, what rejoins the approach of \noun{Gell-Mann} by quantum decoherence~\cite{gell1996quantum}, which explains that probabilities can only be associated to decoherent histories (in which correlations have led the system to follow a trajectory at the macroscopic scale). The \emph{Quantum Measurement Problem} arises when we consider a microscopic wave function giving the state of a system that can be the superposition of several states, and consists in a theoretical paradox, on the one hand the measures being always deterministic whereas the system has probabilities for states, and on the other hand the issue of the non-existence of superposed macroscopic states (collapse of the wave function). As reviewed by~\cite{schlosshauer2005decoherence}, different epistemological interpretations of quantum physics are linked to different explanations of this paradox, including the ``classical'' Copenhagen one which attributes to the act of observation the role of collapsing the wave function. \noun{Gell-Mann} recalls that this interpretation is not absurd since it is indeed the correlations between the quantum object and the world that product the decoherent history, but that it is far more specific, and that the collapse happens in the emergence itself: the cat is either dead or living, but not both, before we open the box. The paradox of the Schrödinger cat appears then as a fundamentally reductionist perspective, since it assumes that the superposition of states can propagate through the successive levels and that there would be no emergence, in the sense of the constitution of an autonomous upper level. In other terms, the work of \cite{2014arXiv1403.7686B} suggests that computational complexity is sufficient for the presence of emergence. This effective separation of scales does not a priori imply that the lower level does not play a crucial role, since \cite{vattay2015quantum} proves that the properties of quantum criticality are typical of molecules of the living, without a priori any specificity for life in this complex determination by lower scales: \cite{2016arXiv161102269V} has recently introduced a new approach linking quantum theories and general relativity in which it is shown that gravity is an emergent phenomenon and that path-dependency in the deformation of the original space introduces a supplementary term at the macroscopic level, that allows to explain deviations attributed up to now to \emph{dark matter}.

\cite{WolfE8678} physical and bio complexity (reductionist ?)


%  bizarre : epistemo de la QM pas bien developpée ? (que physiciens cloisonnés ? du coup philo de comptoir ? et conflits avec informaticiens ? check si Moore en parle)


Reciprocally, the link between computational complexity and emergence is revealed by questions linked to the nature of computation~\cite{moore2011nature}. Cellular automatons, that are moreover crucial for the understanding of several complex systems, have been shown as Turing-complete (a system is said to be Turing-complete if it is able to compute the same functions than a Turing machine, commonly accepted as all what is ``computable'' (\noun{Church}'s thesis). We recall that a Turing machine is a finite automaton with an infinite writing band~\cite{moore2011nature}), such as the Game of Life~\cite{beer2004autopoiesis}. There even exists a programming language allowing to code in the \emph{Game of Life}, available at \url{https://github.com/QuestForTetris}. Its genesis finds its origin in a challenge posted on \emph{codegolf} aiming at the conception of a Tetris, and ended in an extremely advanced collaborative project. Some organisms without a central nervous system are capable of solving difficult decisional problems~\cite{reid2016decision}. An ant-based algorithm is shown by~\cite{Pintea2017} as solving a Generalized Travelling Salesman Problem (GTSP), problem which is NP-difficult. This fundamental link had already been conceived by \noun{Turing}, since beyond his fundamental contributions to contemporary computer science, he studied morphogenesis and tried to produce chemical models to explain it~\cite{turing1952chemical} (that were far from actually explaining it 
%- it is still not well understood today, see~\ref{sec:interdiscmorphogenesis} -
 but which conceptual contributions were fundamental, in particular for the notion of reaction-diffusion). We moreover know that a minimum of complexity in terms of constituting interactions in a particular case of agent-based system (models of boolean networks), and thus in terms of possible emergences, implies a lower bound on computational complexity, which becomes significant as soon as interactions with the environment are added~\cite{tovsic2017boolean}.



\cite{2017arXiv170404231E} quantum computation reduces drastically memory needed


\section{Informational complexity and emergence}


\cite{e18060197} information and self-organization

idea : link with Gershenson approach through information and the juice : explicitly develop applied perspectivism ?

idem multi-scale info Bar-Yam \cite{allen2017multiscale}


Informational complexity (see \cite{dedeo2016information} for a smooth introduction to information theory), or the quantity of information contained in a system and the way it is stored, also bears some fundamental links with emergence. Information is equivalent to the entropy of a system and thus to its degree of organisation - this what allows to solve the apparent paradox of the Maxwell Daemon that would be able to diminish the entropy of an isolated system and thus contradict the second law of thermodynamics: it indeed uses the information on positions and velocities of molecules of the system, and its action balances to loss of entropy through its captation of information (the Maxwell Daemon is more than an intellectual construction: \cite{cottet2017observing} implements experimentally a daemon at the quantic level).

This notion of local increase in entropy has been largely studied by \noun{Chua} under the form of the \emph{Local Activity Principle}, which is introduced as a third principle of thermodynamics, allowing to explain with mathematical arguments the self-organization for a certain class of complex systems that typically involve reaction-diffusion equations~\cite{mainzer2013local}.


The way information is stored and compressed is essential for life, since the ADN is indeed an information storage system, which role at different levels is far from being fully understood. Cultural complexity also witnesses of an information storage at different levels, for example within individuals but also within artefacts and institutions, and information flows that necessarily deal with the two other types of complexities. Information flows are essential for self-organization in a multi-agent system. Collective behaviors of fishes or birds are typical examples used to illustrate emergence and belong to the canonic examples of complex systems. We only begin to understand how these flows structure the system, and what are the spatial patterns of information transfer within a \emph{flock} for example: \cite{crosato2017informative} introduce first empirical results with transfer entropy for fishes and lay the methodological basis of this kind of studies. 

% idem work on flocks prix de these SC



%%%%%%%%%%%%%%%%%%%%%
\section{Reflexivity in the study of complex systems}
%%%%%%%%%%%%%%%%%%%%%



Furthermore, one aspect of knowledge production on complex systems which seems to be recurrent and even inevitable, is a certain level of reflexivity (and that would be inherent to complex system in comparison to simple systems, as we will develop further). We mean by this term both a practical reflexivity, i.e. a necessity to increase the level of abstraction, such as the need to reconstruct in an endogenous way the disciplines in which a reflexion aims at positioning as proposed by \cite{2017arXiv171200805R}, or to reflect on the epistemological nature of modeling when constructing a model, but also a theoretical reflexivity in the sense that theoretical apparels or produced concepts can recursively apply to themselves. The practical inevitability of reflexivity is well-known in social sciences and humanities, but \cite{bourdieu2004science} postulated this would be more generally linked to the nature of scientific knowledge which is inherently social \cite{maton2003reflexivity}. % TODO a clarifier
This does not apply

This practical observation can be related to old epistemological debates questioning the possibility of an objective knowledge of the universe that would be independent of our cognitive structure, somehow opposed to the necessity of an ``evolutive rationality'' implying that our cognitive system, product of the evolution, mirrors the complex processes that led to its emergence, and that any knowledge structure will be consequently reflexive
%\footnote{We thank here \noun{D. Pumain} to have formulated this alternative view on the problem that we will develop in the following.}
. We naturally do not pretend here to bring a response to such a broad and vague question as such, but we propose a potential link between this reflexivity and the nature of complexity. 







%%%%%%%%%%%%%%%%%%%
\section{Production of knowledge}
%%%%%%%%%%%%%%%%%%%

\subsection{From complexity to reflexivity}

We know have enough material to come to reflexivity. It is possible to position knowledge production at the intersection of interactions between types of complexity developed above. First of all, knowledge as we consider it can not be dissociated from a collective construction, and implies thus an encoding and a transmission of information: it is at an other level all problematics linked to scientific communication. The production of knowledge thus necessitates this first interaction between computational complexity and informational complexity. The link between informational complexity and emergence is introduced if we consider the establishment of knowledge as a morphogenetic process. It is shown by \cite{antelope2016interdisciplinary} that the link between form and function is fundamental in psychology: we can interpret it as a link between information and meaning, since semantics of a cognitive object can not be considered without a function. \cite{hofstadter1980godel} recalls the importance of symbols at different levels for the emergence of a thought, that consist in signals at an intermediate level. Finally, the last relation between computational complexity and emergence is the one allowing us a positioning in particular on knowledge production on complex systems, the previous links being applicable to any type of knowledge.

Therefore, any \emph{knowledge of the complex} embraces not only all complexities and their relations in its content, but also in its nature as we just showed. The structure of knowledge in terms of complexity is analog to the structure of systems its studies. We postulate that this structural correspondence implies a certain recursivity, and thus a certain level of \emph{reflexivity} (in the sens of knowledge of itself and its own conditions).


\subsection{The complexity of interdisciplinarity}


We can try to extend to reflexivity in terms of a reflexion on the disciplinary positioning: following \cite{pumain2005cumulativite}, the complexity of an approach is also linked to the diversity of viewpoints that are necessary to construct it. To reach this new type of complexity, for which links with the previous types naturally appear (for example, \cite{gell1995quark} considers the effective complexity as an \emph{Algorithmic Information Content} (close to Kolmogorov complexity) of a Complex Adaptive System \emph{which is observing an other} Complex Adaptive System, what gives their importance to informational and computational complexities and suggests the importance of the observational viewpoint, and by extension of their combination - what furthermore must be related to the perspectivist approach of complex sciences presented above), that would be a supplementary dimension linked to the knowledge of complex systems, reflexivity must be at the core of the approach. \cite{read2009innovation} recall that innovation has been made possible when societies reached the ability to produce and diffuse innovation on their own structure, i.e when they were able to reach a certain level of reflexivity. The \emph{knowledge of the complex} would thus be the product and the support of its own evolution thanks to reflexivity which played a fundamental role in the evolution of the cognitive system: we could thus suggest to gather these considerations, as proposed by \noun{Pumain}, as a new epistemological notion of \emph{evolutive rationality}.

% develop a bit more applied perspectivism here ?

This view of complexity can be formulated as a research program, for the development of an \emph{applied perspectivism}, which has already been sketch for example by \cite{2018arXiv180807282B}.


These approaches could not be tackled in a simple way. Indeed, we can remark that given the law of \emph{requisite complexity}, proposed by \cite{gershenson2015requisite} as an extension of \emph{requisite variety}~\cite{ashby1991requisite}. One of the crucial principles of cybernetics, the \emph{requisite variety}, postulates that to control a system having a certain number of states, the controller must have at least as much states. \noun{Gershenson} proposes a conceptual extension of complexity, which can be justified for example by \cite{allen2017multiscale} which introduce the multi-scale \emph{requisite variety}, showing the compatibility with a theory of complexity based on information theory. Therefore the \emph{knowledge of the complex} will necessarily have to be a \emph{complex knowledge}. This other point of view reinforces the necessity of reflexivity, since following \noun{Morin} (see for example \cite{morin1991methode} on the production of knowledge), the \emph{knowledge of knowledge} is central in the construction of a complex thinking.



%%%%%%%%%%%%%%%%%%%%
\section{Discussion}
%%%%%%%%%%%%%%%%%%%%


\subsection{Towards an applied epistemological framework}

\subsubsection{Applied perspectivism}

\cite{muelder2018} different outcomes with different formalizations of the same theory


\subsubsection{Link with knowledge domains}





\subsection{Implications for urban theories}

% implications pour les SHS et theories urbaines

% lien avec White : artificial intelligence etc ; evoquer projet postdoc ?

\cite{white2017necessity}

%\comment[AB]{personnellement je ne suis pas très fan de cette idée de « connaissance complexe » ou même de « pensée complexe » (ok je suis un mauvais morinien !). }[(JR) mal formule peu etre, ``connaissance du complexe'' $\rightarrow$ cf dernière phrase : serait equivalent selon le point de vue du controle ; rejoint Morin]

% Remarques : 
% portugali semantic information : chaud à introduire.
% check paper Valentina ``Practical Reflexivity''

\cite{anzoise2017perception}

% RElire Morin sur la pensée complexe



\cite{shah2006building} quanti-quali bridged by framework ?

% To conclude this epistemological section, we propose to synthesize all the ideas introduced as concrete manifestations that directly yield from them, and that strongly condition all the forms and semantics of knowledge introduced in the following. These directions (that we will not go up to name principles since they are only at the state of sketch) can be grouped into three large families: modeling practices, Open Science practices, and epistemology. On the domain of modeling practices, in each section emerge different axis that are more or less complementary:

%	\item Modeling, which will be in most cases equivalent to simulation, must be understood as an indirect tool of knowledge on processes within a complex system or on its structure (according to the section on ``why modeling''), and models will necessarily have to be complex (following the reflexion on the different types of complexity) in the sense that they capture a phenomenon of weak emergence, but still respecting constraints of parsimony.
%	\item The exploration of models is fully contained in the modeling enterprise (see reproducibility), and intensive computation is a cornerstone to efficiently explore simulation models (see intensive computation). Sensitivity analysis methods must be questioned and extended if needed (as illustrates the example of the sensitivity to space).
%	\item As suggested by the perspectivist positioning, the coupling of models will have to play a crucial role in the capture of complexity.


Finally, from the epistemological point of view, we can also find ``practical'' implications that will naturally be more implicit in our approach, but not less structuring:

\begin{enumerate}
	\item Our inspiration will essentially be interdisciplinary and will aim at combining different points of view.
	\item Different knowledge domains (concepts that are developed in~\cite{raimbault2017applied}, and that we can understand to simplify in the sense of theoretical, modeling and empirical domains introduced by~\cite{livet2010}) can not be dissociated for any approach of scientific production, and we will use them in a strongly dependent way.
	\item Our approach will have to imply a certain level of reflexivity.
	\item The construction of a complex knowledge (\cite{morin1991methode}) is neither inductive nor deductive, but constructive in the idea of a morphogenesis of knowledge: it can be for example difficult to clearly identify precise ``scientific deadlocks'' since this metaphor assumes that an already constructed problem has to be unlocked, and even to constrain notions, concepts, objects or models in strict analytical frameworks, by categorizing them following a fixed classification, whereas the issue is to understand if the construction of categories is relevant. Doing it a posteriori is similar to a negation of the circularity and recursivity of knowledge production. The elaboration of ways to report that translate the diachronic character and the evolutive properties of it is an open problem.
\end{enumerate}




%%%%%%%%%%%%%%%%%%%%
%% Biblio
%%%%%%%%%%%%%%%%%%%%

%\footnotesize

%\begin{multicols}{2}
\bibliographystyle{apalike}
\bibliography{biblio}
%\end{multicols}





\end{document}
