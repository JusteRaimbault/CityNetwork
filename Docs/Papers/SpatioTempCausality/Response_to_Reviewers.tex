%% start of file `template.tex'.
%% Copyright 2006-2013 Xavier Danaux (xdanaux@gmail.com).
%
% This work may be distributed and/or modified under the
% conditions of the LaTeX Project Public License version 1.3c,
% available at http://www.latex-project.org/lppl/.


\documentclass[11pt,a4paper,sans]{moderncv}        % possible options include font size ('10pt', '11pt' and '12pt'), paper size ('a4paper', 'letterpaper', 'a5paper', 'legalpaper', 'executivepaper' and 'landscape') and font family ('sans' and 'roman')

\usepackage[document]{ragged2e}
% pour justifier


% moderncv themes
\moderncvstyle{banking}                            % style options are 'casual' (default), 'classic', 'oldstyle' and 'banking'
\moderncvcolor{red}                                % color options 'blue' (default), 'orange', 'green', 'red', 'purple', 'grey' and 'black'
\renewcommand{\familydefault}{\rmdefault}         % to set the default font; use '\sfdefault' for the default sans serif font, '\rmdefault' for the default roman one, or any tex font name
%\nopagenumbers{}                                  % uncomment to suppress automatic page numbering for CVs longer than one page


% character encoding
\usepackage[utf8]{inputenc}                       % if you are not using xelatex ou lualatex, replace by the encoding you are using
%\usepackage{CJKutf8}                              % if you need to use CJK to typeset your resume in Chinese, Japanese or Korean

% adjust the page margins
\usepackage[scale=0.75]{geometry}
%\setlength{\hintscolumnwidth}{3cm}                % if you want to change the width of the column with the dates
%\setlength{\makecvtitlenamewidth}{10cm}           % for the 'classic' style, if you want to force the width allocated to your name and avoid line breaks. be careful though, the length is normally calculated to avoid any overlap with your personal info; use this at your own typographical risks...

\usepackage{xparse}
\DeclareDocumentCommand{\comment}{m o o o o}
{%
    \textcolor{red}{#1}
    \IfValueT{#2}{\textcolor{blue}{#2}}
    \IfValueT{#3}{\textcolor{ForestGreen}{#3}}
    \IfValueT{#4}{\textcolor{red!50!blue}{#4}}
    \IfValueT{#5}{\textcolor{Aquamarine}{#5}}
}



%----------------------------------------------------------------------------------
%            content
%----------------------------------------------------------------------------------
%-----       letter       ---------------------------------------------------------


% must contain :
%Summarize the study’s contribution to the scientific literature
%Relate the study to previously published work
%Specify the type of article (for example, research article, systematic review, meta-analysis, clinical trial)
%Describe any prior interactions with PLOS regarding the submitted manuscript
%Suggest appropriate Academic Editors to handle your manuscript (see the full list of Academic Editors)
%List any opposed reviewers
\firstname{}
\lastname{}
\begin{document}


% recipient data
\recipient{SAGEO 2017 Conference, Editeurs}{}
\date{\today}
\opening{Cher Editeurs,}
\closing{Cordialement,\\
Juste Raimbault\\
Université Paris 7 - UMR CNRS 8504 Géographie-cités
}
         % use an optional argument to use a string other than "Enclosure", or redefine \enclname
\makelettertitle

\justify
\justify


Je vous remercie pour la consideration de cette article pour presentation a la conference Sageo. Je tiens toutefois a attirer votre attention sur la legerete des revues et leur manque de precision, ces retours etant de tres faible valeur pour enrichir la substance scientifique de l'article sur ce sujet ouvert difficile. Celles-ci font plus office de filtre de selection pour la conference, qui est naturellement l'une des fonctions de la revue mais ne devrait pas etre le but premier. J'espere que vous considerez ces remarques et serez plus severe avec les relecteurs a l'avenir.

Voici ci-dessous les reponses point par point aux maigres remarques, qui ont ete prise en compte pour la revision de l'article lorsque celles-ci etaient assez precises.


\textbf{Relecteur 1}

\begin{itemize}
	\item \textit{Forme et lisibilité :  1 - Acceptable - rédaction peu pédagogique} : la longueur requise pour l'article empeche de developper des explications qui sont par ailleurs acquises dans les domaines concernes (par exemple pour la classification non supervisee). Le but d'un article n'est pas de faire un cours ou un tutorial.
	\item \textit{Qualité de la démarche scientifique :  2 - Quelques faiblesses} : remarque generale, impossible a prendre en compte.
	\item \textit{Qualité de l'état de l'art et des références :  0 - Incomplet - Des raccourcis principalement} : La longueur de l'article ne permet pas de s'etendre sur les relations villes-transport ou la notion de causalite en geographie. Les references fondamentales et les plus proches de la demarche etaient deja incluses, un paragraphe de revue plus proche de l'econometrie a ete ajoute.
	\item \textit{Adéquation aux domaines d'expertise du relecteur :  1 - De bonnes notions du domaine - lacune sur l'approche méthodologique} : l'article introduit une methode innovante que le relecteur ne semble saisir.
	\item \textit{Dans le paragraphe 3.2.3   et sur la figure 4 la composante temporelle n'est pas visible} Tous les graphes sont en fonction du lag $\tau$, je ne comprends pas cette remarque, qui temoigne que le relecteur n'a pas lu en detail.
	\item \textit{On observe bien des différences sur la figure 4, mais l'interprétation rédigée à la fin du paragraphe 3.2.3 n'y fait pas référence (si ce n’est à croire l’auteur sur parole)} Idem. le maximum de $\rho(\tau)$ donne l'existence potentielle d'une causalite, qui est bien referee dans le texte. la sensibilite au parametre de decay est secondaire.
	\item \textit{Ce papier nécessite une rédaction beaucoup plus pédagogique pour pouvoir être jugé correctement. Il en reste cependant le sentiment d'une approche originale et rigoureuse qui mérite d'être exposée.} Le relecteur n'ayant soit pas pris le temps de lire en detail, soit place ses limitations sur le compte de la pedagogie.
	\item \textit{figures 1 et 2  : légendes trop petites} OK mais pas le temps de refaire les figures.
	\item \textit{Préciser la définition de patch ; Préciser le définition de "corrélation retardée"} : pas l'espace pour redefinir le RBD ; les lagged correlations sont definies au paragraophe precedent. fucking didnt read the paper !
	\item \textit{la figure 2 est vraiment difficile à lire (par exemple je n'arrive pas à distinguer de transition pour théta=2 et k=5).} ce nest pas mon probleme.
	\item \textit{De même, la lecture page 8 est tout aussi difficile " la distance au réseau cause .....une valeur intermediaire ". A la fin de cette lecture cette application de démontre rien.} elle demontre que le relecteur n'a pas lu ni compris le papier. merci de me faire perdre mon temps.
\end{itemize}




\textbf{Relecteur 2}

\begin{itemize}
	\item \textit{Faire une passe sur l'orthographe : des définitionS, causalité baséE, applications ... concrèteS, cetTE article, causalités ... évidentEs, données synthétiqueS, accessibilité généraliséeS, variables elleS, applications potentielLEs 862360 -> 862 360 vectorialisées -> vectorisées}
	\item \textit{ figure 2 : 1ère ligne au centre : redondante avec celle de gauche à supprimer => permet d'agrandir celle de droite 2ème ligne qui est plus intéressante} Je ne comprends pas cette remarque ; la derivee du clustering coef est differente de celui-ci, et pas directement lisible dans la figure.
\end{itemize}











\makeletterclosing





\end{document}


%% end of file `template.tex'.