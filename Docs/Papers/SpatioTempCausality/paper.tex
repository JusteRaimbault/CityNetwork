% adapté de la RIG version française

\documentclass[french]{./sageo}

\confShortName{SAGEO'2017}
\confLongName{SAGEO'2017 - Rouen, 6-9 novembre 2017}

\usepackage[utf8]{inputenc} 
\usepackage[T1]{fontenc}
\usepackage{lmodern}
\usepackage{textcomp}
\usepackage{amsmath}
\usepackage{graphicx}
\usepackage{multirow}
\usepackage[noend]{algorithmic}
\usepackage[linesnumbered,ruled,vlined,boxed,commentsnumbered]{algorithm2e}

\firstpagenumber{1}

% \title[short header title]{title}
\title{Une méthode générique d'identification de causalités dans des données spatio-temporelles}

\author[1,2]{Juste}{Raimbault}


% addresses are automatically numbered
\address{UMR CNRS 8504 Géographie-cités}
        {}
\address{UMR-T 9403 IFSTTAR LVMT}
        {juste.raimbault@polytechnique.edu}       

\resume{Résumé.}

\motscles{Quelques mots clés}

\keywords{En anglais}

\abstract{Abstract in English}

\begin{document}

\maketitle

\newpage




\section{Introduction}









\section{Conclusion}




%%%%%%%%%%%%%%%
%% Biblio
%%%%%%%%%%%%%%%

%\bibliography{biblio}







%%%%%%%%%%%%%%%
%% TEMPLATES
%%%%%%%%%%%%%%%

%
%
%
%\section{Section 1}
%
%\subsection{Sous-section 1}
%
%\subsubsection{Sous-sous-section 1}
%
%\begin{figure*}[h]
%   \centering  \includegraphics[width=8cm]{grenouille.jpg}
%  \caption{\label{fig:1} Une grenouille bien verte.}
%\end{figure*}
%
%\subsection{Sous-section 2}
%
%\section{Section 2}
%
%Listes :
%\begin{itemize}
%\item ligne 1 (cf. équation \ref{eq:form1})
%\item ligne 2 (cf. équation \ref{eq:form2})
%\end{itemize}
%
%Formules :
%
%\begin{equation}
%	R = \frac{d_1}{d_2}
%	\label{eq:form1}
%\end{equation}
%
%\begin{equation}
%	\sin(\alpha) = \frac{h}{l}
%    \label{eq:form2}
%\end{equation}
%
%\begin{table*}[h!]
%\begin{center}
%\caption{\label{tab:1} Exemple de tableau}
% \scriptsize
%      \begin{tabular}{|c|c|c|c|c|}
%   \hline
%   \multirow{2}*{ Clients} & \multicolumn{2}{| c |}{Départ }  & \multicolumn{2}{| c |}{ Arrivée}\\
%   \cline{2-5}
%      & Station  & Période de Temps & Station  & Période de Temps \\\hline
%   client 1 (c1) & 3  & 2 & 1 & 4 \\\hline
%   client 2 (c2)   &   2  & 2 & 3 & 3\\\hline
%   client 3 (c3)   &   2 & 2 & 3 & 4  \\\hline
%   client 4 (c4)   &   3 & 2 & 2 & 3  \\\hline
%   client 5 (c5)   &   3 & 2 & 2 & 4  \\\hline
%  client 6 (c6)   &   2  & 4 & 3 & 5\\\hline
%   client 7 (c7)   &   3  & 3 & 2 & 6  \\\hline
%   client 8 (c8)   &   1 & 5 & 3 & 6 \\\hline
%   client 9 (c9) & 2  & 6 & 3 & 7 \\\hline
%    client 10 (c10)  &   3 & 7 & 1 & 9 \\\hline
%   client 11 (c11) & 1  & 6 & 2 & 7 \\\hline
%%\hline
%\end{tabular}
%\end{center}
%\end{table*}
%
%Exemple d'algorithme :
%
%\begin{algorithm}[h!]
%\label {algo}
% \KwData{
% $G(V,A,C,R,U)$ \; \tcc{commentaire}}
% \KwResult{
% $Paths_{Cars}, \; Relocation,\; SatisfiedDemands, Paths_{Agents}  $
% }
% initialization\;
% $Paths_{Agents} \gets \emptyset $ /* l'ensemble de chemins... */ \;
% $j \gets 1$ \;
% $costPath_j  \gets 0$ \;
% \While { $ (j \le  nb_{Veh}) \wedge (costPath_j \leq 0)  $ }
% {
% $ path_j \gets Dijkstra(G(V,A, C,R)) $ \;
%$ costPath_j \gets  Cost (path_j)$ \;
%  \ForAll { $(v^{k}_{t'},v^{i}_{t}) \in path   $}
%{
% \ForAll {$ U_{r^{i}_{t'',t''+1}} $  }
% {....
% }
%$ Paths_{Cars} \gets Paths_{Cars} \cup path_j $  \;
% }
%$j \gets j+1$ \;
% }
%$Paths_{Agents} \gets  routeAgents (Relocation ); $
% \caption{un algorithme très glouton}
%\end{algorithm}
%









\end{document}