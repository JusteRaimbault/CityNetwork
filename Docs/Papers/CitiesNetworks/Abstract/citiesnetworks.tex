\documentclass[10pt]{article}

% general packages without options
\usepackage{amsmath,amssymb,bbm}
% graphics
\usepackage{graphicx}
% text formatting
\usepackage[document]{ragged2e}
\usepackage{pagecolor,color}

\newcommand{\noun}[1]{\textsc{#1}}

\usepackage[utf8]{inputenc}
\usepackage[T1]{fontenc}
% geometry
\usepackage[margin=2cm]{geometry}

\usepackage{multicol}
\usepackage{setspace}

\usepackage{natbib}
\setlength{\bibsep}{0.0pt}

%\usepackage[french]{babel}

% layout : use fancyhdr package
%\usepackage{fancyhdr}
%\pagestyle{fancy}

% variable to include comments or not in the compilation ; set to 1 to include
\def \draft {1}


% writing utilities

% comments and responses
%  -> use this comment to ask questions on what other wrote/answer questions with optional arguments (up to 4 answers)
\usepackage{xparse}
\usepackage{ifthen}
\DeclareDocumentCommand{\comment}{m o o o o}
{\ifthenelse{\draft=1}{
    \textcolor{red}{\textbf{C : }#1}
    \IfValueT{#2}{\textcolor{blue}{\textbf{A1 : }#2}}
    \IfValueT{#3}{\textcolor{ForestGreen}{\textbf{A2 : }#3}}
    \IfValueT{#4}{\textcolor{red!50!blue}{\textbf{A3 : }#4}}
    \IfValueT{#5}{\textcolor{Aquamarine}{\textbf{A4 : }#5}}
 }{}
}

% todo
\newcommand{\todo}[1]{
\ifthenelse{\draft=1}{\textcolor{red!50!blue}{\textbf{TODO : \textit{#1}}}}{}
}


\makeatletter


\makeatother


\begin{document}







\title{\vspace{-0.5cm}Models for the Co-evolution of Cities and Networks
\\
\textit{Abstract Proposal for Handbook on Cities and Networks}
}
\author{\noun{Juste Raimbault}$^{1,2}$
\\
$^1$ UMR CNRS 8504 Géographie-cités\\
$^2$ UMR-T IFSTTAR 9403 LVMT
}
\date{}

\maketitle

\justify

\pagenumbering{gobble}


\textbf{Keywords : }\textit{Urban System; Co-evolution; Transportation Network}

\medskip


The complexity of interactions between Networks and Territories has been widely acknowledged empirically, in particular through the existence of circular causal relations in their co-development. A relevant framework to understand this phenomenon is to view it as a \emph{co-evolution}, that we use for territorial systems within Pumain's Evolutive Urban Theory that understands urban systems as complex adaptive systems. Considering modeling as a primary source of indirect knowledge on processes and dynamics, we investigate models endogeneizing this co-evolution, in the particular case of cities and transportation networks. We first proceed to an extended systematic review, based on scientific disciplines mapping through citation network and semantic analysis, in order to shed a light on the existing modeling approaches that span in very diverse disciplines, from planning and urban geography to economics and physics, each having different research questions, ontologies and scales. This systematic review, when consolidated with models purpose, application case, scales and ontologies for objects and processes, yields a modelography including for example land-use Transport Interaction Models, Network growth models or Urban Systems models. This first stage lead us to identify two typical scales at which co-evolution models are relevant, with associated paradigms: the mesoscopic scale in an approach of Urban Morphogenesis, and the macroscopic scale in the context of Evolutive Urban Theory.


The mesoscopic model we develop has a fine resolution (under 1km) and intermediate spatial extent (50$\sim$100km). Given a fixed exogenous growth rate, population are dispatched following a preferential attachment that depends on a potential controlled by local urban form (density, distance to network) and network measures (centralities and generalized accessibilities), and then diffused in space to capture urban sprawl. Network growth is included through a multi-modeling paradigm: implemented heuristics include biological network generation and gravity potential breakdown. The model is calibrated on measures for Urban Form and network topology, computed on windows spanning full European Union and China, given population density data and OpenStreetMap road network. The calibration is static in the sense that fit is done on final configurations only, but is done at the first (measures) and second (correlations) order, the later capturing indirectly relations between the network and the urban frame. The model is able to reproduce most of existing configurations. The study of lagged correlations within synthetic model runs furthermore unveils a large variety of causality regimes, confirming the ability of the model to grasp the complexity of situations existing empirically.


At the macroscopic scale, growth rates are endogenous and interactions between cities are their main driver. As a deterministic extension of the Gibrat model, our model considers gravity-based flows within the network to determine growth rates. These have an effect at the first order (direct interaction) but also at the second order (effect of cumulated flows on traversed cities) what allows to capture centrality effects. Network growth follows a demand-induced thresholded growth scheme, that can occur at the global level or locally. The exploration of the model on synthetic city systems unveils patterns of complexity compared to a basic interaction model without co-evolution. We apply the model on the French city system, with population data spanning 1831-1999. We compare behaviors with different network data, namely dynamical railway network (1850-2000) and dynamical highway network (1950-2000). Proceeding to locally stationary in time calibrations, we find that cities population prediction are always more accurate than a static model, even when controlling on additional number of parameters with a specifically designed empirical AIC criterion. Concerning accuracy of generated network, the model is more performant with the highway network, probably because of the multiple regimes that existed in railway history.


These complementary modeling efforts pave the way for future research towards operational models, with applications in local and territorial planning by helping to design policies integrating transportation and urban development in a dynamical and strongly coupled way.



%%%%%%%%%%%%%%%%%%%%
%% Biblio
%%%%%%%%%%%%%%%%%%%%
%\tiny

%\begin{multicols}{2}

%\setstretch{0.3}
%\setlength{\parskip}{-0.4em}


\bibliographystyle{apalike}
\bibliography{/Users/Juste/Documents/ComplexSystems/CityNetwork/Biblio/Bibtex/CityNetwork}%,biblio}
%\end{multicols}



\end{document}
