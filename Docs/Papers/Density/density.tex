% Template for PLoS
% Version 3.1 February 2015
%
% To compile to pdf, run:
% latex plos.template
% bibtex plos.template
% latex plos.template
% latex plos.template
% dvipdf plos.template
%
% % % % % % % % % % % % % % % % % % % % % %
%
% -- IMPORTANT NOTE
%
% This template contains comments intended 
% to minimize problems and delays during our production 
% process. Please follow the template instructions
% whenever possible.
%
% % % % % % % % % % % % % % % % % % % % % % % 
%
% Once your paper is accepted for publication, 
% PLEASE REMOVE ALL TRACKED CHANGES in this file and leave only
% the final text of your manuscript.
%
% There are no restrictions on package use within the LaTeX files except that 
% no packages listed in the template may be deleted.
%
% Please do not include colors or graphics in the text.
%
% Please do not create a heading level below \subsection. For 3rd level headings, use \paragraph{}.
%
% % % % % % % % % % % % % % % % % % % % % % %
%
% -- FIGURES AND TABLES
%
% Please include tables/figure captions directly after the paragraph where they are first cited in the text.
%
% DO NOT INCLUDE GRAPHICS IN YOUR MANUSCRIPT
% - Figures should be uploaded separately from your manuscript file. 
% - Figures generated using LaTeX should be extracted and removed from the PDF before submission. 
% - Figures containing multiple panels/subfigures must be combined into one image file before submission.
% For figure citations, please use "Fig." instead of "Figure".
% See http://www.plosone.org/static/figureGuidelines for PLOS figure guidelines.
%
% Tables should be cell-based and may not contain:
% - tabs/spacing/line breaks within cells to alter layout or alignment
% - vertically-merged cells (no tabular environments within tabular environments, do not use \multirow)
% - colors, shading, or graphic objects
% See http://www.plosone.org/static/figureGuidelines#tables for table guidelines.
%
% For tables that exceed the width of the text column, use the adjustwidth environment as illustrated in the example table in text below.
%
% % % % % % % % % % % % % % % % % % % % % % % %
%
% -- EQUATIONS, MATH SYMBOLS, SUBSCRIPTS, AND SUPERSCRIPTS
%
% IMPORTANT
% Below are a few tips to help format your equations and other special characters according to our specifications. For more tips to help reduce the possibility of formatting errors during conversion, please see our LaTeX guidelines at http://www.plosone.org/static/latexGuidelines
%
% Please be sure to include all portions of an equation in the math environment.
%
% Do not include text that is not math in the math environment. For example, CO2 will be CO\textsubscript{2}.
%
% Please add line breaks to long display equations when possible in order to fit size of the column. 
%
% For inline equations, please do not include punctuation (commas, etc) within the math environment unless this is part of the equation.
%
% % % % % % % % % % % % % % % % % % % % % % % % 
%
% Please contact latex@plos.org with any questions.
%
% % % % % % % % % % % % % % % % % % % % % % % %

\documentclass[10pt,letterpaper]{article}
\usepackage[top=0.85in,left=2.75in,footskip=0.75in]{geometry}

% Use adjustwidth environment to exceed column width (see example table in text)
\usepackage{changepage}

% Use Unicode characters when possible
\usepackage[utf8]{inputenc}

% textcomp package and marvosym package for additional characters
\usepackage{textcomp,marvosym}

% fixltx2e package for \textsubscript
\usepackage{fixltx2e}

% amsmath and amssymb packages, useful for mathematical formulas and symbols
\usepackage{amsmath,amssymb}

% cite package, to clean up citations in the main text. Do not remove.
\usepackage{cite}

% Use nameref to cite supporting information files (see Supporting Information section for more info)
\usepackage{nameref,hyperref}

% line numbers
\usepackage[right]{lineno}

% ligatures disabled
\usepackage{microtype}
\DisableLigatures[f]{encoding = *, family = * }

% rotating package for sideways tables
\usepackage{rotating}

% Remove comment for double spacing
%\usepackage{setspace} 
%\doublespacing

% Text layout
\raggedright
\setlength{\parindent}{0.5cm}
\textwidth 5.25in 
\textheight 8.75in

% Bold the 'Figure #' in the caption and separate it from the title/caption with a period
% Captions will be left justified
\usepackage[aboveskip=1pt,labelfont=bf,labelsep=period,justification=raggedright,singlelinecheck=off]{caption}

% Use the PLoS provided BiBTeX style
\bibliographystyle{plos2015}

% Remove brackets from numbering in List of References
\makeatletter
\renewcommand{\@biblabel}[1]{\quad#1.}
\makeatother

% Leave date blank
\date{}

% Header and Footer with logo
\usepackage{lastpage,fancyhdr,graphicx}
\usepackage{epstopdf}
\pagestyle{myheadings}
\pagestyle{fancy}
\fancyhf{}
\lhead{\includegraphics[width=2.0in]{PLOS-submission.eps}}
\rfoot{\thepage/\pageref{LastPage}}
\renewcommand{\footrule}{\hrule height 2pt \vspace{2mm}}
\fancyheadoffset[L]{2.25in}
\fancyfootoffset[L]{2.25in}
\lfoot{\sf PLOS}

%% Include all macros below

\newcommand{\lorem}{{\bf LOREM}}
\newcommand{\ipsum}{{\bf IPSUM}}

%% END MACROS SECTION


\begin{document}
\vspace*{0.35in}

% Title must be 250 characters or less.
% Please capitalize all terms in the title except conjunctions, prepositions, and articles.
\begin{flushleft}
{\Large
\textbf\newline{Calibration of a Spatialized Urban Growth Model - Submission to PLOS Journals}
}
\newline
% Insert author names, affiliations and corresponding author email (do not include titles, positions, or degrees).
\\
Raimbault Juste\textsuperscript{1}
\\
\bigskip
\bf{1} UMR CNRS 8504 G{\'e}ographie-cit{\'e}s, Universit{\'e} Paris 7, Paris, France
\\
\bigskip

% Insert additional author notes using the symbols described below. Insert symbol callouts after author names as necessary.
% 
% Remove or comment out the author notes below if they aren't used.
%
% Primary Equal Contribution Note
%\Yinyang These authors contributed equally to this work.

% Additional Equal Contribution Note
% Also use this double-dagger symbol for special authorship notes, such as senior authorship.
%\ddag These authors also contributed equally to this work.

% Current address notes
%\textcurrency a Insert current address of first author with an address update
% \textcurrency b Insert current address of second author with an address update
% \textcurrency c Insert current address of third author with an address update

% Deceased author note
%\dag Deceased

% Group/Consortium Author Note
%\textpilcrow Membership list can be found in the Acknowledgments section.

% Use the asterisk to denote corresponding authorship and provide email address in note below.
* juste.raimbault@etu.univ-paris-diderot.fr

\end{flushleft}
% Please keep the abstract below 300 words
\section*{Abstract}




% Please keep the Author Summary between 150 and 200 words
% Use first person. PLOS ONE authors please skip this step. 
% Author Summary not valid for PLOS ONE submissions.   
%\section*{Author Summary}


\linenumbers





%%%%%%%%%%%%%%%%%%%%%%%%
\section*{Introduction}

Urban Systems are complex socio-technical objects






%%%%%%%%%%%%%%%%%%%%%%%%
% You may title this section "Methods" or "Models". 
% "Models" is not a valid title for PLoS ONE authors. However, PLoS ONE
% authors may use "Analysis" 
\section*{Materials and Methods}

\subsection*{The urban growth model}



\subsection*{Indicators}

As our model is only density-based, we propose to quantify its outputs through spatial morphology, i.e. characteristics of density spatial distribution. We need therefore quantities having a certain level of robustness and invariance. For example, two polycentric cities should be classified as morphologically close whereas a direct comparison of distributions (Earth Mover Distance e.g.) could give a very high distance between configurations depending on center positions. To tackle this issue, we refer to the Urban Morphology Analysis litterature which proposes an extensive set of indicators to describe urban form~\cite{tsai2005quantifying}. The number of dimensions can be reduced to obtain a robust description with relatively independant indicators~\cite{Schwarz201029}. For the choice of indicators, we follow the analysis done in~\cite{le2015forme} where a typology of large european cities is obtained in consistence with qualitative knowledge. Let denote $(P_i)_{1\leq i \leq N}$ the population of cells, sorted in decreasing order, and $d_{ij}$ the distance between cells $i,j$. The indicators are the following :

\begin{enumerate}
\item Rank-size slope, expressing the degree of hierarchy in the distribution
\item Distribution Entropy
\[
\mathcal{E} = \sum_{i=1}^{N}P_i
\]

\end{enumerate}



%%%%%%%%%%%%%%%%%%%%%%%%
% Results and Discussion can be combined.
\section*{Results}

The model was implemented in NetLogo~\cite{wilensky1999netlogo} for profiling, computation of indicators being delegated for performance reasons to a R script using the dedicated \texttt{raster} package~\cite{}.



%%%%%%%%%%%%%%%%%%%%%%%%
\subsection*{Generation of urban patterns.}




%%%%%%%%%%%%%%%%%%%%%%%%
\subsection*{Model Behavior}

In the study of such a computational model of simulation, the lack of analytical tractability must be balanced by an extensive knowledge 

\paragraph{Convergence}


\paragraph{Exploration of parameter space}


\paragraph{Statistical analysis.}


%%%%%%%%%%%%%%%%%%%%%%%%
\subsection*{Model Calibration}

\paragraph{Real Data} We use the population density grid provided openly by 


\paragraph{Calibration Process}



%%%%%%%%%%%%%%%%%%%%%%%%
\section*{Discussion}


%%%%%%%%%%%%%%%%%%%%%%%%
\subsection*{Integration into a multi-scale growth model}





%%%%%%%%%%%%%%%%%%%%%%%%
\section*{Conclusion}




%%%%%%%%%%%%%%%%%%%%%%%%
\section*{Supporting Information}

% Include only the SI item label in the subsection heading. Use the \nameref{label} command to cite SI items in the text.
\subsection*{S1 Figure}
\label{S1_Figure}
{\bf Bold the first sentence.}



%%%%%%%%%%%%%%%%%%%%%%%%
\section*{Acknowledgments}



\nolinenumbers

%\section*{References}
% Either type in your references using
% \begin{thebibliography}{}
% \bibitem{}
% Text
% \end{thebibliography}
%
% OR
%
% Compile your BiBTeX database using our plos2015.bst
% style file and paste the contents of your .bbl file
% here.
% 


\bibliographystyle{plos2015}
\bibliography{/Users/Juste/Documents/ComplexSystems/CityNetwork/Biblio/Bibtex/CityNetwork}

%
%\begin{thebibliography}{1}
%
%\bibitem{tsai2005quantifying}
%Tsai YH.
%\newblock Quantifying urban form: compactness versus' sprawl'.
%\newblock Urban studies. 2005;42(1):141--161.
%
%\bibitem{Schwarz201029}
%Schwarz N.
%\newblock Urban form revisited---Selecting indicators for characterising
%  European cities.
%\newblock Landscape and Urban Planning. 2010;96(1):29 -- 47.
%\newblock Available from:
%  \url{http://www.sciencedirect.com/science/article/pii/S0169204610000320}.
%
%\bibitem{le2015forme}
%Le~N{\'e}chet F.
%\newblock De la forme urbaine {\`a} la structure m{\'e}tropolitaine: une
%  typologie de la configuration interne des densit{\'e}s pour les principales
%  m{\'e}tropoles europ{\'e}ennes de l'Audit Urbain.
%\newblock Cybergeo: European Journal of Geography. 2015;.
%
%\bibitem{wilensky1999netlogo}
%Wilensky U.
%\newblock NetLogo. 1999;.
%
%\end{thebibliography}






%%%%%%%%%%%%%% Tab template %%%%%%%%%%%%%%%%

%\begin{table}[!ht]
%\begin{adjustwidth}{-2.25in}{0in} % Comment out/remove adjustwidth environment if table fits in text column.
%\caption{
%{\bf Table caption Nulla mi mi, venenatis sed ipsum varius, volutpat euismod diam.}}
%\begin{tabular}{|l|l|l|l|l|l|l|l|}
%\hline
%\multicolumn{4}{|l|}{\bf Heading1} & \multicolumn{4}{|l|}{\bf Heading2}\\ \hline
%$cell1 row1$ & cell2 row 1 & cell3 row 1 & cell4 row 1 & cell5 row 1 & cell6 row 1 & cell7 row 1 & cell8 row 1\\ \hline
%$cell1 row2$ & cell2 row 2 & cell3 row 2 & cell4 row 2 & cell5 row 2 & cell6 row 2 & cell7 row 2 & cell8 row 2\\ \hline
%$cell1 row3$ & cell2 row 3 & cell3 row 3 & cell4 row 3 & cell5 row 3 & cell6 row 3 & cell7 row 3 & cell8 row 3\\ \hline
%\end{tabular}
%\begin{flushleft} Table notes Phasellus venenatis, tortor nec vestibulum mattis, massa tortor interdum felis, nec pellentesque metus tortor nec nisl. Ut ornare mauris tellus, vel dapibus arcu suscipit sed.
%\end{flushleft}
%\label{table1}
%\end{adjustwidth}
%\end{table}


%%%%%%%%%%%%% Fig. Template %%%%%%%%%%%%%%%%%

%\begin{figure}[h]
%\caption{{\bf Figure Title first bold sentence Nulla mi mi, venenatis sed ipsum varius, volutpat euismod diam.}
%Figure Caption Proin rutrum vel massa non gravida. Quisque tempor sem et dignissim rutrum. A: Lorem ipsum dolor sit amet. B: Consectetur adipiscing elit.}
%\label{fig1}
%\end{figure}




%%%%%%%%%%%%%% Eq. Template %%%%%%%%%%%%%%%%%

%\begin{equation}\label{eq:schemeP} 
%D_{coll} = \frac{D_f+\frac{[S]^2}{K_D S_T} D_S} {1+\frac{[S]^2}{K_D S_T}}, 
%D_{sm} = \frac{D_f+ \frac{[S]}{K_D} D_S}{1+\frac{[S]}{K_D}},
%\end{equation}





\end{document}

