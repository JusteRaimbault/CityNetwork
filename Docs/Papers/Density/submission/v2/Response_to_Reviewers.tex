%% start of file `template.tex'.
%% Copyright 2006-2013 Xavier Danaux (xdanaux@gmail.com).
%
% This work may be distributed and/or modified under the
% conditions of the LaTeX Project Public License version 1.3c,
% available at http://www.latex-project.org/lppl/.


\documentclass[11pt,a4paper,sans]{moderncv}        % possible options include font size ('10pt', '11pt' and '12pt'), paper size ('a4paper', 'letterpaper', 'a5paper', 'legalpaper', 'executivepaper' and 'landscape') and font family ('sans' and 'roman')

\usepackage[document]{ragged2e}
% pour justifier


% moderncv themes
\moderncvstyle{banking}                            % style options are 'casual' (default), 'classic', 'oldstyle' and 'banking'
\moderncvcolor{red}                                % color options 'blue' (default), 'orange', 'green', 'red', 'purple', 'grey' and 'black'
\renewcommand{\familydefault}{\rmdefault}         % to set the default font; use '\sfdefault' for the default sans serif font, '\rmdefault' for the default roman one, or any tex font name
%\nopagenumbers{}                                  % uncomment to suppress automatic page numbering for CVs longer than one page


% character encoding
\usepackage[utf8]{inputenc}                       % if you are not using xelatex ou lualatex, replace by the encoding you are using
%\usepackage{CJKutf8}                              % if you need to use CJK to typeset your resume in Chinese, Japanese or Korean

% adjust the page margins
\usepackage[scale=0.9]{geometry}
%\setlength{\hintscolumnwidth}{3cm}                % if you want to change the width of the column with the dates
%\setlength{\makecvtitlenamewidth}{10cm}           % for the 'classic' style, if you want to force the width allocated to your name and avoid line breaks. be careful though, the length is normally calculated to avoid any overlap with your personal info; use this at your own typographical risks...

\usepackage{xparse}
\DeclareDocumentCommand{\comment}{m o o o o}
{%
    \textcolor{red}{#1}
    \IfValueT{#2}{\textcolor{blue}{#2}}
    \IfValueT{#3}{\textcolor{ForestGreen}{#3}}
    \IfValueT{#4}{\textcolor{red!50!blue}{#4}}
    \IfValueT{#5}{\textcolor{Aquamarine}{#5}}
}


% to show numerical labels in the bibliography (default is to show no labels); only useful if you make citations in your resume
%\makeatletter
%\renewcommand*{\bibliographyitemlabel}{\@biblabel{\arabic{enumiv}}}
%\makeatother
%\renewcommand*{\bibliographyitemlabel}{[\arabic{enumiv}]}% CONSIDER REPLACING THE ABOVE BY THIS

% bibliography with mutiple entries
%\usepackage{multibib}
%\newcites{book,misc}{{Books},{Others}}
%----------------------------------------------------------------------------------
%            content
%----------------------------------------------------------------------------------
%-----       letter       ---------------------------------------------------------


% must contain :
%Summarize the study’s contribution to the scientific literature
%Relate the study to previously published work
%Specify the type of article (for example, research article, systematic review, meta-analysis, clinical trial)
%Describe any prior interactions with PLOS regarding the submitted manuscript
%Suggest appropriate Academic Editors to handle your manuscript (see the full list of Academic Editors)
%List any opposed reviewers
\firstname{}
\lastname{}
\begin{document}


% recipient data
\recipient{Editor PLOS ONE}{}
\date{\today}
\opening{Dear Editor,}
\closing{Yours faithfully,\\
Juste Raimbault\\
Université Paris 7 - UMR CNRS 8504 Géographie-cités
}
         % use an optional argument to use a string other than "Enclosure", or redefine \enclname
\makelettertitle

\justify


Thank you for considering the manuscript ``Calibration of a Density-based Model of Urban Morphogenesis'' for possible publication in PLOS ONE. The suggestions and comments will undoubtedly be of great value to the paper. The paper was updated accordingly.

First, the following general changes were done:

\begin{itemize}
	\item A supplementary analysis was developed as a supplementary material, namely the sensitivity of indicators computed on real data to windows size. This gives stronger confidence in the results obtained.
	\item Several details or more precise explanations were added.
	\item The protocol of the study was archived on protocols.io as advised by the editor and the corresponding doi was added in the paper.
	\item A screening for spelling errors and English mistakes was done.
	\item The typo on the value of the entropy was corrected.
	\item Long paragraphs were split to ease readability.
\end{itemize}


I provide you now the point-by-point response to referees' reports.

Concerning the comments specific to the first referee:

\begin{enumerate}
	% 1. The interaction between aggregation and diffusion determines the morphological dimension of urban growth processes. Then, how to classify the interaction into the degrees and types?
	\item Concerning the understanding of the interaction between aggregation and diffusion and its correspondance to emerging forms, I recalled that only a reference to the full behavior which is given in Supplementary material can unveil properties of a specific regime. The configuration detailed in main text is one example.
	% 2. Page 9, line 92: I assume that the author means “50-100km” for the mesoscopic scale rather than “50 100km.” Then, what are the ranges for microscopic and macroscopic scales corresponding to this study?
	\item There was indeed a typo in the spatial range corresponding to the scale of the model, which is 50-100km. In comparison, the microscopic scale would be under 10km and the macroscopic scale at 500-1000km. This useful remark was added.
	% 3. The choice of upper limits for model parameters in Table 1 is without explanation.
	\item An explanation for the choice of parameters boundaries was added.
	% 4. Page 11, line 143: I assume that the author means “ε = 1 means that the population is uniformly distributed” rather than “ε = 0.”
	%\item
	% 5. Page 14, Caption in Figure 2: The values for total population and growth rate in rural area are larger than those in intermediate settlements.
	\item Concerning the caption in Figure 2 with high values of population for rural areas, a paragraph was added to recall that the interpretations are done on renormalized densities and that the total populations has less influence than the ratio between population and growth rate.
	% 6. Some equations are not numbered.
	\item All equations were numbered.
	% 7. Some of the citation and the list of references do not meet the standards.
	\item The bibliography was screened and updated to meet the standards.
\end{enumerate}


The comments of the second referee were addressed before in the common general section above.

%This paper proposes a parcimonious model for describing at meso-scale the spatial distribution of urban growth including two antagonistic abstract processes of aggregation and sprawl. The importance of this contribution is that besides the abstract model simulating urban morphogenesis there is an attempt to validate it on a significant set of empirical data, encompassing 500 urban areas in Europe and using a dedicated measurement method for comparing observed and simulated urban forms. It represents also a considerable computational work and a remarkable example of a complete and open exploration of a model, the data and experiments are made available. To my opinion the paper is ready for publication (after correcting a few spelling errors) and it should be promoted on the PLOS site.
%The paper relies first on an updated state of the art in quantitative geography including cellular automata and fractals. The central part of the model that includes five parameters only is a process of population growth over a grid including a mechanism of preferential attachment and a local diffusion effect. The model written in Netlogo and Scala is used first for generating a variety of theoretical shapes (whose ability of morphological indicators to differentiate the configurations is tested) and the OpenMole simulation platform is used for fully exploring the parameter space before it is calibrated on the European spatial distribution of population densities.
%Spelling errors: On line 143 there is a spelling mistake to be corrected:
% E = 0 means that all the population is in one cell whereas E = 0 means that…and
% Line 238 write gives (instead of giver), line 247 values (instead of valued)


Concerning the third referee's comments, the following precise points were treated:


%The above-referred paper develops an interesting model for generating different urban morphologies based on two mechanisms of diffusion and aggregation. The model behavior is controlled over European countries and calibration results and model outputs are provided in a brief and succinct manner. I think the paper is well-organized and well-written (despite some minor language errors) and it adds to the literature. I recommend publication of the paper subject to a minor revision.

\begin{enumerate}
	% 1. Introduction: Majority of the cited works belong to more than a decade ago. Please update the manuscript in terms of the recent studies in the literature. For example, the following papers use the concepts of diffusion and aggregation (i.e. diffusion-coalescence or centralization-decentralization) to explain the process of urban growth at finer scales compared to your study.
	% Sakieh, Y., & Salmanmahiny, A. (2016). Treating a cancerous landscape: Implications from medical sciences for urban and landscape planning in a developing region. Habitat International, 55, 180–191.
	% Dietzel, C., Oguz, H., Hemphill, J. J., Clarke, K. C., & Gazulis, N. (2005). Diffusion and coalescence of the Houston Metropolitan Area: evidence supporting a new urban theory. Environment and Planning B: Planning and Design, 32(2), 231–246.
	\item Some literature was added, including the relevant suggestions that were made.
	% 2. Materials and Methods: this section is very well explained. Please provide a flowchart and describe the major steps of the study.
	\item A flowchart describing the main step of the study was added.
	% 3. Discussion: Please discuss the main findings of your study and functionality of your model compared to recent studies and models in the literature.
	% 4. Please highlight the practical applications of the diffusion-aggregation model at a continental scale.
	\item A part was added in the discussion to discuss main findings, the position regarding existing literature, and possible practical applications of using this model at a continental scale.
	% 5. Please clearly explain the sensitivity analysis of the model in a separate section.
	\item The sensitivity analysis of the model is indirectly done both in the assessment of sensitivity to stochasticity, and in the exploration of model behavior which covers the full parameter space. The corresponding section was renamed accordingly.
	% 6. Figures and tables are of very good quality. Well done!
	%\item -> nothing to do
	% 7. The paper is overall well-written; however, there are some minor language errors. English proof-reading in terms of the following items is recommended:
	% - absence of articles where they are required
	% - errors of common English usage
	% - failure to use the possessive case when it is needed
	% - clash of singular and plural in the same sentence
	% - wrong choice of words
	% - proper of capitalization.
	%\item  -> ok general section.
\end{enumerate}














\justify




\makeletterclosing





\end{document}


%% end of file `template.tex'.