\input{si.tex}



\section*{S2 Text : Semi-analytical analysis of the simplified model}


\subsection*{Partial Differential Equation}

We propose to derive the PDE in a simplified setting. To recall the configuration given in main text, the system has one dimension, such that $x\in \mathbb{R}$ with $1/\delta x$ cells of size $\delta x$, and we use the expected values of cell population $p(x,t) = \Eb{P(x,t)}$. We furthermore take $n_d=1$. Larger values would imply derivatives at an order higher than 2 but the following results on the existence of a stationary solution should still hold. 

Denoting $\tilde{p}(x,t)$ the intermediate populations obtained after the aggregation stage, we have

\[
\tilde{p}(x,t) = p(x,t) + N_g\cdot \frac{p(x,t)^{\alpha}}{\sum_x p(x,t)^{\alpha}}
\]

since all populations units are added independently. If $\delta x \ll 1$ then $\sum_x p^{\alpha} \simeq \int_x p(x,t)^{\alpha}dx$ and we write this quantity $P_{\alpha}(t)$. We furthermore write $p=p(x,t)$ and $\tilde{p} = \tilde{p}(x,t)$ in the following for readability.

The diffusion step is then deterministic, and for any cell not on the border ($0<x<1$), if $\delta t$ is the interval between two time steps, we have

\[
\begin{split}
p(x,t+\delta t) & = (1 - \beta) \cdot \tilde{p} + \frac{\beta}{2} \left[\tilde{p}(x-\delta x,t) + \tilde{p}(x+\delta x,t)\right]\\
& = \tilde{p} + \frac{\beta}{2} \left[\left(\tilde{p}(x+\delta x,t) - \tilde{p}\right) - \left(\tilde{p} - \tilde{p}(x-\delta x,t)\right)\right]
\end{split}
\]

Assuming the partial derivatives exist, and as $\delta x \ll 1$, we make the approximation $\tilde{p}(x+\delta x,t) - \tilde{p} \simeq \delta x\cdot \frac{\partial \tilde{p}}{\partial{x}}(x,t)$, what gives 

\[
\left(\tilde{p}(x+\delta x,t) - \tilde{p}\right) - \left(\tilde{p} - \tilde{p}(x-\delta x,t)\right) = \delta x \cdot \left(\frac{\partial \tilde{p}}{\partial{x}}(x,t) - \frac{\partial \tilde{p}}{\partial{x}}(x - \delta x,t)\right)
\]

and therefore at the second order

\[
p(x,t+\delta t) = \tilde{p} + \frac{\beta \delta x^2}{2} \cdot \frac{\partial^2 \tilde{p}}{\partial x^2}
\]

Substituting $\tilde{p}$ gives

\[
\begin{split}
\frac{\partial^2 \tilde{p}}{\partial x^2} & = \frac{\partial^2 p}{\partial x^2} + \frac{N_G}{P_\alpha}\cdot \frac{\partial}{\partial x}\left[\alpha \frac{\partial p}{\partial x} p^{\alpha - 1}\right]\\
& = \frac{\partial^2 p}{\partial x^2} + \alpha \frac{N_G}{P_\alpha} \left[\frac{\partial^2 p}{\partial x^2} p^{\alpha - 1} + (\alpha - 1) \left( \frac{\partial p}{\partial x}\right)^2 p^{\alpha - 2}\right]
\end{split}
\]

By supposing that $\frac{\partial p}{\partial t}$ exists and that $\delta t$ is small, we have $p(x,t+\delta t) - p(x,t) \simeq \delta t\frac{\partial p}{\partial t}$, what finally yields , by combining the results above, the partial differential equation


\begin{equation}\label{eq:pde}
\delta t \cdot \frac{\partial p}{\partial t} = \frac{N_G \cdot p^{\alpha}}{P_{\alpha}(t)} + \frac{\alpha \beta (\alpha - 1) \delta x^2}{2}\cdot \frac{N_G \cdot p^{\alpha-2}}{P_{\alpha}(t)} \cdot \left(\frac{\partial p}{\partial x}\right)^2 + \frac{\beta \delta x^2}{2} \cdot \frac{\partial^2 p}{\partial x^2} \cdot\left[ 1 + \alpha \frac{N_G p^{\alpha - 1}}{P_{\alpha(t)}} \right]
\end{equation}



Initial conditions should be specified as $p_0(x) = p(x,t_0)$. To have a well-posed problem similar to more classical PDE problems, we need to assume a domain and boundary conditions. A finite support is expressed by $p(x,t)=0$ for all $t$ and $x$ such that $\left|x\right|>x_m$.

%An infinite domain implies that density, in the sense of population proportion $d(x,t) = \frac{p(x,t)}{P_1(t)}$, goes to zero anywhere when time goes to infinity. Indeed, $P_1(t)=N_G\cdot t$. If $d(x,t)$ does not vanish, there exist $t_1$ such that  -> not that simple indeed


\subsection*{Stationary solution for density}

The non-linearity and the integral terms making the equation above out of the scope for analytical resolution, we study its behavior numerically in some case. We show that on a finite domain




%%%%%%%%%%%%%%%%%%%%
\begin{figure}
\centering
\includegraphics[width=\textwidth]{figuresraw/stationary}
\caption{}
\label{}
\end{figure}
%%%%%%%%%%%%%%%%%%%%






\subsection*{Randomness and bifurcations}

The previous analyses were done on a deterministic version of the system. How can randomness influence the final configuration ?




% \subsection*{Stationary radius}
% indeed at intermediate time - scaling behavior interesting, but not mature.




\end{document}
