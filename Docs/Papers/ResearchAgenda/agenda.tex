%%%%%%%%%%%%%%%%%%%%%%%%%%%%%
% Standard header for working papers
%
% WPHeader.tex
%
%%%%%%%%%%%%%%%%%%%%%%%%%%%%%

\documentclass[11pt]{article}

% packages without options
\usepackage{amsmath,bbm}

% geometry
\usepackage[margin=2cm]{geometry}






\title{Towards a dynamic modeling of coevolution processes between transportation and land-use : Construction of a research agenda\bigskip\\
\textit{Working Paper}
}
\author{\noun{Juste Raimbault}}
\date{Friday October 9th}


\maketitle

\justify


\begin{abstract}
\textit{TBW}
\end{abstract}


%%%%%%%%%%%%%%%%%%%%
\section{Introduction}
%%%%%%%%%%%%%%%%%%%%


\subsection{Context}

% small paragraph on issues related to interaction land-use/transportation.




% now take a detour within sci context

Within an evolutive urban theory~\cite{pumain2006evolutionary}, a considerable body of knowledge on urban systems self-organisation has recently been built through the construction, the exploration and the calibration of thematic-based models of simulation, of which the serie of Simpop models is emblematic~\cite{pumain2012multi}. The elaboration of an integrated platform for the construction and the evaluation of geographical models, including the development of the user-friendly, yet powerful by the transparent access to grid computation ressources, Model Experiment software OpenMole~\cite{reuillon2013openmole}, but also an epistemological framework and associated meta-heuristics for model validation~\cite{rey2015plateforme}, was central for the establishment of evidence-based thematic conclusions, which differentiation with the consequent previous amount of geographical research lead by similar methods of agent-based modeling and simulation was indeed the introduction of novel methods and tools going a step further for the validation stage. A illustrative example is the application of the Calibration Profile algorithm (which reveals a single parameter influence on model performance within the whole parameter space) to the sufficient and necessary parameters to reproduce existing urban systems patterns on a long time scale by the SimpopLocal model~\cite{schmitt2014half}, and other methods such as PSE algorithm aimed to detect rare outputs of a model, were successively applied, to the Marius model in that case~\cite{10.1371/journal.pone.0138212}.

At first sight this methodological and scientific context seems rather disconnected from our geographical objects of study which are the processes of coevolution between transportation network and urban growth, in a generic form (i.e. at any scale temporal and spatial scales, and in any geographical context) in a first approach and of course geographically contextualized once the entreprise of this paper will have been completed. These works are indeed the giants on which shoulder we intend to stand on. We rely on Bretagnolle concluding considerations in~\cite{bretagnolle:tel-00459720}, insisting on the need to pursue the various empirical findings on long-time network and cities interactions, by modeling approaches which should shed light on underlying coevolution processes. We propose to explore that paradigm which has been poorly tackled and has many obstacles associated with. Theoretically, Bretagnolle's work is positioning precisely within the evolutionnary urban framework, which assets include the compliance with complexity approaches which allow to take into account the particularities of urban systems such as their non-ergodicity~\cite{pumain2012urban}. Methodologically, it seems intuitively suitable to our purpose, what will be confirmed further.


\subsection{Modeling the coevolution : overview of the scientific landscape}

% brief overview, as in ecqtg paper.


\subsection{Proposing a research agenda}

% justification of vrious axes ? ! not exhaustive at all.

% rely on approaches having tried ? -> what was considered, what lacked ?


% announce plan

The rest of the paper is organised as follows




%%%%%%%%%%%%%%%%%%%%
\section{Giere's Deamon, or when disciplinary compartimentation narrows perspectives}
%%%%%%%%%%%%%%%%%%%%

% First track : Applied epistemology to understand the complementarity yet diversification of sci approaches. // Caruso and Krugman reconcile urban economics with ABM ? 


% results of ecqtg, completed with citation network [scholar api etc -> cit repo ? doi ? ]


% perspectives on datamining approach



%%%%%%%%%%%%%%%%%%%%
\section{Empirical analysis : ``Lost in the Smog''}
%%%%%%%%%%%%%%%%%%%%

% Parallel with scaling : different conclusions, different objects, different scales etc.

% -> need of precise empirical constatations within model context - NO GENERALITIES
% here evoque a general form linking Barthelemy and Raimbault ? -> machine learning based approaches to estimate transition matrix ?

% -> Essential question of scale AND ontology
% question of scale : systematic large scale correlations study





%%%%%%%%%%%%%%%%%%%%
\section{Methodological Foundations, need for concrete}
%%%%%%%%%%%%%%%%%%%%

% -> behavior of coupled models : example of difficult ?
% -> framework for model coupling (ebimm Clem Paul etc) : extending to other modeling tools ?
% -> Statistical control, synthetic data and -- Space Matters --






%%%%%%%%%%%%%%%%%%%%
\section{Modeling the Governance : the Grand Pari(s)}
%%%%%%%%%%%%%%%%%%%%

% bring in by examples (Offner book ?) the question of gvernance.
% why necessary, why diificult
% first prospects with lutecia ? -- Quote Metropolsim and transmodyn transition.




%%%%%%%%%%%%%%%%%%%%
\section{Proposition of a research agenda, towards calibrated dynamic models of coevolution}
%%%%%%%%%%%%%%%%%%%%


% assemble various tracks

% proposition of projects, concrete tracks.





%%%%%%%%%%%%%%%%%%%%
\section{Conclusion}
%%%%%%%%%%%%%%%%%%%%










%%%%%%%%%%%%%%%%%%%%
%% Biblio
%%%%%%%%%%%%%%%%%%%%

\bibliographystyle{apalike}
\bibliography{/Users/Juste/Documents/ComplexSystems/CityNetwork/Biblio/Bibtex/CityNetwork,biblio}


\end{document}
