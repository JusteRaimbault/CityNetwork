%%%%%%%%%%%%%%%%%%%%%%%%%%%%%
% Standard header for working papers
%
% WPHeader.tex
%
%%%%%%%%%%%%%%%%%%%%%%%%%%%%%

\documentclass[11pt]{article}

% packages without options
\usepackage{amsmath,bbm}

% geometry
\usepackage[margin=2cm]{geometry}






\title{A Theory of co-evolutive networked territorial systems : Exemplification of Network Necessity\bigskip\\
\textit{Working Paper}
}
\author{\noun{Juste Raimbault}}
\date{Date}


\maketitle

\justify


\begin{abstract}
This paper is the application of a theoretical paper developing a theory of co-evolutive networked territorial systems. We apply simple models of urban growth for systems of cities, which include in particular the role of physical networks.
\end{abstract}



\section{Context and Objective}


\subsection{Exemplifying Network Necessity}

% what plan to do
% how does goes in the direction of theory confirmation ?

We propose to support our hypothesis that \textit{physical transportation networks are necessary to explain the morphogenesis of territorial systems} (aka \textit{Network Necessity}) by showing on a relatively simple case that the integration of physical networks into some models effectively increase their explanative power (being careful on the precise definition of model improvement to avoid overfitting). We work on simple territorial systems that are country-wide city systems, and more particularly French cities, on a time scale corresponding to that spatial scale, i.e. two last centuries. Taking into account physical networks can improve the understanding of city growth within that system in two ways : a qualitative one, for which the extended model would exhibit qualitative features corresponding to stylized facts empirically observed but that more basic models do not manage to reproduce, and a quantitative way, in the sense that model extension improves explained variance further than the mechanic improvement due to the introduction of supplementary degrees of freedom. If at least one of these is unveiled in our particular case, the evidence will support the theory at these scale and in this context.


\subsection{Model context}

\cite{bretagnolle2000long} already propose a spatial extension of the Gibrat model (\textit{detail})

\cite{favaro2011gibrat} is a more refined extension with economic cycles

% marius as an extension of Gibrat ? in the deterministic case.


\section{Model Description}

\subsection{From Gibrat to Marius : the dilemma of formulation}

% stochastic vs deterministic formulation ? precise possible formulations and what choice means.

%  - Gabaix \cite{gabaix1999zipf} details stationary distribution of Gibrat model - link with Zipf law. why we do not accept this "explanation" : NOT stationary. depends on time scales to reach stationarity ?
%  - bayesian iterative formulation ? (mcmc)
%  - how does formulation influence ? equivalence in certain cases between stoch-cov and interdependent expectancies ?




\subsection{Model description}






\section{Results}

\subsection{Data}



\subsection{Implementation}


\subsection{Model Exploration}

% qualitative behavior : hierarchy inversions/ trajectories of normalized populations (cf papier Denise Anne etc)
% need of synthetic data ?


\subsection{Model Calibration}

% here put empirical AIC ?



\section{Discussion}

% how this example illustrates well the theory






\section{Supplementary Materials}


\subsection{Integrating Gibrat}

Analytical resolution is possible for some aspects of the Gibrat model. We detail here the computation for some.

\paragraph{Expectancies} If working with expectancies, it makes no sense to proceed to Monte Carlo simulation as a direct resolution gives a deterministic recurrence relation on expectancies. Let $\mu_t = \Eb{P(t)}$

\paragraph{Covariance}


\paragraph{Distribution}



\subsection{A Bayesian iterative approach}

Readers familiar with Bayesian Signal Processing techniques will have remarked that similarities exist between the described models and iterative filters such as the Kalman filter, more precisely particle filters.



\subsection{Model explanative power : propositions for an empirical AIC}




%%%%%%%%%%%%%%%%%%%%
%% Biblio
%%%%%%%%%%%%%%%%%%%%

\bibliographystyle{apalike}
\bibliography{/Users/Juste/Documents/ComplexSystems/CityNetwork/Biblio/Bibtex/CityNetwork}


\end{document}
