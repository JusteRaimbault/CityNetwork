%%%%%%%%%%%%%%%%%%%%%%%%%%%%%
% Standard header for working papers
%
% WPHeader.tex
%
%%%%%%%%%%%%%%%%%%%%%%%%%%%%%

\documentclass[11pt]{article}

% packages without options
\usepackage{amsmath,bbm}

% geometry
\usepackage[margin=2cm]{geometry}






\title{Reading Record\bigskip\\
\cite{offner1996reseaux}
}
\author{\noun{Juste Raimbault}}
\date{Date}


\maketitle

\textbf{\textit{Reading Record for \cite{offner1996reseaux}}}


\section{Introduction}

\textit{Preface by Raffestin : }

Network and Territories : two notions that can represent both geographical materiality and theoretical domain.

Crucial notion : articulation space-network-territory. Need for a theory, some ``laws of progressive composition''.

Sketch of a theory (from Nicolas Curien) : territoriality = projection of a system of intentions into space ; systems of intention are expressed within material and social networks. Successive temporal imbrications [dynamical causalities ? ]. Importance of power in the process (network operators) : role of governance. 

Role of information : modification of mobility but no decrease. Role of social networks and company networks. Network of cities : also conceptual indetermination.


\section{Linear Reading}

\subsection{Introduction}

Various questions from crossing of different disciplines : do networks contribute to territorial fabric ? correspondance between network and territory types ? Deconstruction of territories by increasing networking ?

$\rightarrow$ different significations and notions across disciplines.

First technical networks : materialisation ; II : network operators ; III : communication networks and IV social networks ; V and VI : companies and city networks.  \textit{Overview of semantic variations linking networks and territories}

\subsection{Technical Networks}












%%%%%%%%%%%%%%%%%%%%
%% Biblio
%%%%%%%%%%%%%%%%%%%%

\bibliographystyle{apalike}
\bibliography{/Users/Juste/Documents/ComplexSystems/CityNetwork/Biblio/Bibtex/CityNetwork}


\end{document}
