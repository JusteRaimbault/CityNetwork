%%%%%%%%%%%%%%%%%%%%%%%%%%%%%
% Standard header for working papers
%
% WPHeader.tex
%
%%%%%%%%%%%%%%%%%%%%%%%%%%%%%

\documentclass[11pt]{article}

% packages without options
\usepackage{amsmath,bbm}

% geometry
\usepackage[margin=2cm]{geometry}






\title{Reading Record\bigskip\\
\cite{}
}
\author{\noun{Juste Raimbault}}
\date{Date}


\maketitle

\textbf{\textit{Reading Record for \cite{}}}



\section{Introduction}

Synergetics introduced for the study of physical complex systems by Haken around 1984.

\textit{TODO : detail Haken approach}

$\rightarrow$ particular application to systems of cities ; part of a larger corpus of theoretical and quantitative geography developed since ? at Geocites.




\section{Linear Reading}

\subsection*{Introduction}

Particularity of systems of cities : exchange networks and strong interdependencies. ``Breakdowns and continuity''. Since beginning of 19th, diffusion of social and technical innovations ; stable and auto-reproducing system.

Regularities and fluctuations in such systems. Studied period : 1954-1982.

Concepts of complex systems : non-linearity, centrality of interactions, self-organization far from equilibrium (SOC ?), determinism and stochasticity. Not only conceptual parallelism, but demonstration of application.

\subsection{The system of Cities}

The city as a spatial entity, coarse-graining at this level. Delimitation of the system ? technical constraints more than theoretical.

\subsubsection{City size dynamics}

\paragraph{Theories of city growth}

Classical models ; growth by steps (specialization then diversification) ; agglomeration economies (in both sense, with negative feedbacks)

\paragraph{Evolution of the system of cities}

More qualitative (geographical ?) approaches : cf. Berry~\cite{berry1964cities} ; Pumain. For French Urban System : before 19th, quite independent evolutions (Gibrat ?), later ``temporal autocorrelation of growth rates''. Thesis of Guerin-Pace : coupling macro (stable, trend) with micro fluctuations.

\subsubsection{Cities and cycles}

Urban life cycles : more for intra-urban characteristics. At a greater scale : cycles of innovation ; link with role of technical innovations in Schumpeterian theory. Economic cycles have deeply shaped French urban system. rq: product cycles more localized and precise than economic cycles, more complex to analyze.

\textbf{Combination of two diffusion processes :}
\begin{itemize}
\item spatial diffusion (core-periphery)
\item hierarchical diffusion (schematically, more refined diffusion at different level occurs in reality)
\end{itemize}


\subsubsection{Innovation cycles and evolution of the system of cities}

Understanding link between relative growth and diffusion of innovation is complex. Interferences between cycles, spatial and temporal. ``Multi-dimensional diffusion analysis''.

Various scales can be taken :
\begin{itemize}
\item scale of the process
\item scale of the firm
\item scale of city evolution, innovation as a driver of urban growth
\end{itemize}

\paragraph{Long-time approach}

Marchetti technological substitution model : $f\in [0,1]$ fraction of city in urban system, then $\log{\frac{f}{1-f}} = a\cdot t + b \implies f = \frac{e^{at+b}}{1+e^{at+b}} = \frac{e^{\frac{t-t_0}{\tau}}}{1+e^{\frac{t-t_0}{\tau}}}$. $\rightarrow$ different growth factors for different spatial entities ? fraction is then normalized fraction. PB : assumes stationarity of processes ? or assumed $\tau (t)$ with $\tau \sim_{- \infty} \tau_{-} > 0 $ and $\tau \sim_{\infty} \tau_{+} < 0 $. [quite strange here]. Idea : Growth/decline cycles.


\paragraph{Short and Middle temporal scales}

more precise data at these scales. tertiary/secondary substitution, position of agglomeration on a logistic curve. (cf Marchetti model ?). 

ex tertiarization french agglos : variety of situations. approach however too general ? (sector, etc particularities). typology by sectors : constant repartition between cities ; homogeneisation ; growth and reinforcment of inequalities.


\subsubsection{Dynamical models applied to urban systems evolution [dynamics ?]}

Bruxelles, Prigogine and Synergetics, Haken.

particularity of models :
\begin{itemize}
\item designed for complex systems ; spatio-temporal
\item differential equations : continuity of urban change
\item non-linear equations, taking feedbacks into account
\item classical assumptions can easily be integrated
\item out-of equilibrium, non-unicity of eq. ; ``evolve in time'' : non-stationarity ?
\item qualitative bifurcations possible because of equilibrium multiplicity
\end{itemize}


With $\mathbf{P}$ populations, $\mathbf{X}$ state variables, $\mathbf{\mu}$ parameters, most general equation is

\[
\frac{\partial \mathbf{P}}{\partial t} = \mathbf{F}(\mathbf{\mu},\mathbf{X},\mathbf{P})
\]

\textit{here cross terms between state variables vanish, we have with less generality,} $\frac{\partial P_i}{\partial t} = F_i(\mu, \mathbf{X}_i,\mathbf{P})$.

various formulations have been proposed ; models not tractable analytically if take multiplicity of interactions into account.

Noise term $\varepsilon$ can be added. example of ``deterministic'' trend plus noise curve.

Other formulation : Master equation approach, from synergetics.


\subsection{Multiscalarity : from individuals to city systems}

Emergence of city dynamics from its micro components.

Two visions :
\begin{itemize}
\item micro description of behavior
\item statistical distribution
\end{itemize}

what about coupling both ? example of migration process : discriminating character is crucial.

\subsubsection{Difficulty of migration models typology}

Variety of applications (explanation of behavior, flows distribution, planning, embedding into larger model) ; choice of explanatory variables depends on application ; variety of formalizations (from Markov chains to econometric, log-linear, micro-utility, gravitation, etc) ; different role of time : all these contribute to difficulty of typology.

\subsubsection{Individual behavior : micro-geographical models}

log-linear models : explain the indicator of move for one household.

Logit and Nested logit models. [// discrete choices]

examples and biblio.

\subsubsection{Meso and Macro scales}

Meso : flows between entities ; Macro : global organisation of system.

\paragraph{Econometric models}

explanatory vars ; utility etc : works at an interregional level.

\paragraph{Spatial interaction models : from gravitation to entropy maximization}

importance of distance for interactions. 

Most general gravitation model : $M_{ij} = k P_i P_j \exp{(-\frac{d_{ij}}{d_0})}(1+\alpha c_{ij})$ (or power law instead of exponential), where $c_ij$ captures ratio surfaces / common frontier (to include travel possibility). Rq : thematic based ok, but could fit anything ? pb of equifinality again.

Wilson model, based on entropy maximisation, generalizes gravitation model.

Tobler : not explain migrations but include them accurately.

ex application for France.


\paragraph{More general models}

Lowry : gravitation plus incomes and unemployment.

Batty 1983 dynamic model of simulation.

\subsubsection{Attempts for micro-macro integrations}

Different variables ; temporal scales. According to Weidlich, reductionism vs holistic approach - close to autonomy of weak emergence.

Probabilistic interpretation of gravitation model, extension with conditional probas.

Leeds : synthetic population from census data, simulation : too heavy computationnaly [rq : different today ? cf all city systems simulations ?] : top-down approach.

vs bottom-up : use micro-geo data. example of difference between expectations and reality for individuals. Other study : importance of macro factors.

$\rightarrow$ various scales, sometimes contradictory results ; logic for each scale and difficulty of explicit link between scales.


\subsection{Answers from Synergetics}

Interdependance and cooperation between subsystems : auto-organisation.

Typical characteristics of complex systems (recall from dyn. models ?)

\begin{itemize}
\item ``Hierarchy between scales, each level of aggregation is well defined, perceptible as an entity at a given scale''. \textbf{RQ : very close to ontological decomposition, furthermore here detailed as ``perceptible''} $\rightarrow$ \textbf{TO BE DETAILED}
\item complex interactions, non-linearity between elements at a given level, can influence upper level (RQ : links \emph{between} levels ?)
\item Combination of deterministic trends and stochastic fluctuations.
\item Possibility of bifurcation because of non-linearity (oscillating or chaotic behaviors) - ex. phase transition.
\end{itemize}


For city systems : behavior of macro depending on micro constituents ? Weidlich and Haag have first used synergetics to study the evolution of an urban system.

\subsubsection{Master Equations system}

$\mathbf{n}$ cities populations, $\sum_i n_i (t) = N(t)$, evolution of $\Pb{\mathbf{n},t}$ ? Conservation of probas. Master equation : describes dynamics $\frac{\partial \Pb{\mathbf{n},t}}{\partial t}$.

Integration of migration process : exchanges with external world ; exchanges between cities of the system.

Derivation : proba of jump $\mathbf{n}(t) \rightarrow \mathbf{n}(t + dt)$ at micro level, developed at the first order, so-called ``individual transition rate'' is derivative ; at level of city ``configurational transition rate'' - both are linked simply.

$\rightarrow$ Master eq. with transition matrix, classic definition.

\[
\frac{\partial \Pb{\mathbf{n},t}}{\partial t} = \sum_{ij}{w_{ij}(n'_{ij}) \Pb{n'_{ij},t}} - \sum_{ij}{w_{ij}(\mathbf{n}\Pb{\mathbf{n},t}}
\]


\subsubsection{From master equation to mean values}


expectancy taken in master eq to have deterministic eqs., because single realization observed. \textbf{RQ : loose interest of master eq, take only mean on all possible configurations. ISSUE as indeed linear aggregation : ignore possible emergence from stochastic fluctuations and chaotic }









%%%%%%%%%%%%%%%%%%%%
%% Biblio
%%%%%%%%%%%%%%%%%%%%

\bibliographystyle{apalike}
\bibliography{/Users/Juste/Documents/ComplexSystems/CityNetwork/Biblio/Bibtex/CityNetwork}


\end{document}

