%%%%%%%%%%%%%%%%%%%%%%%%%%%%%
% Standard header for working papers
%
% WPHeader.tex
%
%%%%%%%%%%%%%%%%%%%%%%%%%%%%%

\documentclass[9pt]{article}

%%%%%%%%%%%%%%%%%%%%
%% Include general header where common packages are defined
%%%%%%%%%%%%%%%%%%%%



%%%%%%%%%%%%%%%%%%%%%%%%%%
%% TEMPLATES
%%%%%%%%%%%%%%%%%%%%%%%%%%


% Simple Tabular

%\begin{tabular}{ |c|c|c| } 
% \hline
% cell1 & cell2 & cell3 \\ 
% cell4 & cell5 & cell6 \\ 
% cell7 & cell8 & cell9 \\ 
% \hline
%\end{tabular}





%%%%%%%%%%%%%%%%%%%%%%%%%%
%% Packages
%%%%%%%%%%%%%%%%%%%%%%%%%%


% general packages without options
\usepackage{amsmath,amssymb,bbm}

% graphics
\usepackage{graphicx}

% text formatting
\usepackage[document]{ragged2e}
\usepackage{pagecolor,color}








%%%%%%%%%%%%%%%%%%%%%%%%%%
%% Maths environment
%%%%%%%%%%%%%%%%%%%%%%%%%%

%\newtheorem{theorem}{Theorem}[section]
%\newtheorem{lemma}[theorem]{Lemma}
%\newtheorem{proposition}[theorem]{Proposition}
%\newtheorem{corollary}[theorem]{Corollary}

%\newenvironment{proof}[1][Proof]{\begin{trivlist}
%\item[\hskip \labelsep {\bfseries #1}]}{\end{trivlist}}
%\newenvironment{definition}[1][Definition]{\begin{trivlist}
%\item[\hskip \labelsep {\bfseries #1}]}{\end{trivlist}}
%\newenvironment{example}[1][Example]{\begin{trivlist}
%\item[\hskip \labelsep {\bfseries #1}]}{\end{trivlist}}
%\newenvironment{remark}[1][Remark]{\begin{trivlist}
%\item[\hskip \labelsep {\bfseries #1}]}{\end{trivlist}}

%\newcommand{\qed}{\nobreak \ifvmode \relax \else
%      \ifdim\lastskip<1.5em \hskip-\lastskip
%      \hskip1.5em plus0em minus0.5em \fi \nobreak
%      \vrule height0.75em width0.5em depth0.25em\fi}



%%%%%%%%%%%%%%%%%%%%
%% Idem general commands
%%%%%%%%%%%%%%%%%%%%
%% Commands

\newcommand{\noun}[1]{\textsc{#1}}


%% Math

% Operators
\DeclareMathOperator{\Cov}{Cov}
\DeclareMathOperator{\Var}{Var}
\DeclareMathOperator{\E}{\mathbb{E}}
\DeclareMathOperator{\Proba}{\mathbb{P}}

\newcommand{\Covb}[2]{\ensuremath{\Cov\!\left[#1,#2\right]}}
\newcommand{\Eb}[1]{\ensuremath{\E\!\left[#1\right]}}
\newcommand{\Pb}[1]{\ensuremath{\Proba\!\left[#1\right]}}
\newcommand{\Varb}[1]{\ensuremath{\Var\!\left[#1\right]}}

% norm
\newcommand{\norm}[1]{\left\lVert #1 \right\rVert}



% argmin
\DeclareMathOperator*{\argmin}{\arg\!\min}


% amsthm environments
\newtheorem{definition}{Definition}
\newtheorem{proposition}{Proposition}
\newtheorem{assumption}{Assumption}

%% graphics

% renew graphics command for relative path providment only ?
%\renewcommand{\includegraphics[]{}}









\usepackage[utf8]{inputenc}
\usepackage[T1]{fontenc}





% geometry
\usepackage[margin=1.5cm]{geometry}

\usepackage{multicol}
\usepackage{setspace}

\usepackage{natbib}
\setlength{\bibsep}{0.0pt}


% layout : use fancyhdr package
\usepackage{fancyhdr}
\pagestyle{fancy}

\makeatletter

\renewcommand{\headrulewidth}{0.4pt}
\renewcommand{\footrulewidth}{0.4pt}
\fancyhead[RO,RE]{\textit{Working Paper}}
\fancyhead[LO,LE]{G{\'e}ographie-Cit{\'e}s/LVMT}
\fancyfoot[RO,RE] {\thepage}
\fancyfoot[LO,LE] {\noun{J. Raimbault}}
\fancyfoot[CO,CE] {}

\makeatother


%%%%%%%%%%%%%%%%%%%%%
%% Begin doc
%%%%%%%%%%%%%%%%%%%%%

\begin{document}







\title{Reading Record\bigskip\\
\cite{racine1971quantitative}
}
%\author{\noun{Juste Raimbault}}
%\date{Date}


\maketitle

\textbf{\textit{Reading Record for \cite{racine1971quantitative}}}


Transformation of geo in 50-70s. US -> UK, en parallele Suede et URSS.

Paul Claval précurseur en france.

Depasse debat quanti-quali, mais plutot vraie question est : geo peut elle devenir science experimentale ? $\rightarrow$ \textit{rejoint l'evidence based}.

introduction of maths as one cause of opposition ?

(... orography and rainfall)

Jean Labasse : importance de la contingence en geo, methodes quanti ``specialement urgentes et specialement delicate''. planification autoroutes : pas que ordi, jugement a la fin.

(... amenagement de l'espace)


Méthodes actuelles : neguentropie ; defi multidisciplinaire.

(... map making) : package programs $\rightarrow$ \textit{open sci}

(... rural geo)

(... non parametric stats)

JB Racine : distinction quantification et analyse quanti. finalité de la recherche geo. problemes classiques connus en quanti. \textbf{Lien necessaire entre quanti et theorique}. Quantifier pour mieux qualifier. Reduction dimension. 

(... system analysis : field theory, berry geo matrix)

Necessite d'un dialogue. Position de la geo ? quantification n'est pas rationalisation. attention a pas prendre moyen pour une fin, courbes pour elles memes `` de nombreux auteurs auraient honte de leurs ecrits si les idees confuses qu'ils masquent sous des courbes complexes etaient exprimees en un francais precis'' $\rightarrow$ \textit{fine arete, dififculte et danger de la computation : shift de la pb vers bug data etc ? ou juste le nouveau ?}

conclusion : pour une phenomenologie de la connaissance geo. specificite de l'objet geo. (etapes d'une connaissance geo). probabilistic explanation. 

cit de conclusion :
``en vous initiant aux nouvelles methodes pour renouveller la geo et accroitre sa portee et son efficacite, n'oubliez jamais que vous vous donnez des moyens puissants, mais qu'il faut les dominer par une vision philosophique de la geographie''












%%%%%%%%%%%%%%%%%%%%
%% Biblio
%%%%%%%%%%%%%%%%%%%%

\bibliographystyle{apalike}
\bibliography{/Users/Juste/Documents/ComplexSystems/CityNetwork/Biblio/Bibtex/CityNetwork}


\end{document}
