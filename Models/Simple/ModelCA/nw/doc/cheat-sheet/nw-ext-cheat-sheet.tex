% compile with xelatex, e.g.: `/usr/bin/xelatex nw-ext-cheat-sheet.tex`
\documentclass[10pt]{article}
\usepackage{fontspec}
\setmainfont{Linux Libertine O}
\setmonofont[Scale=MatchLowercase]{DejaVu Sans Mono}
\usepackage[margin=0.75in]{geometry}
\usepackage[parfill]{parskip}

\newlength{\docindent}
\setlength{\docindent}{0.5cm}
\usepackage{fancyvrb}
\fvset{xleftmargin=\docindent}
\usepackage[english]{babel} %
\usepackage[babel]{csquotes} %
\MakeOuterQuote{"}
\usepackage{url}
\usepackage{color}

\newenvironment{prim}{
  \vskip 6 pt
  \ttfamily
  \bfseries
}{
  \vskip 2 pt
  \hrule
  \vskip 6 pt
}

\newenvironment{doc}{
  \leftskip 0.5 cm
}
{}

\newcommand{\param}[1]{\texttt{\textit{\textmd{#1}}}}

\definecolor{gray}{rgb}{0.9,0.9,0.9}

\newcommand{\cat}[1]{
  \vspace{1em}
  \colorbox{gray}{\makebox[\textwidth][l]{
    \vspace{1em}
    \scshape \bfseries \large #1
  }
}}

\begin{document}
{
  \begin{center}
  \Huge NetLogo NW Extension \textsuperscript{\small(1.0.0-RC4)} — Cheat Sheet
  \\
  \normalsize For download and complete documentation, see: \\
  \url{https://github.com/NetLogo/NW-Extension}
  \end{center}
  \vspace{1 em}
}

\cat{General Primitives}

\begin{prim}
nw:set-context \param{turtleset} \param{linkset}
\end{prim}

\begin{doc}
Specifies the set of turtles and the set of links that the extension will consider
to be the current graph. All the turtles from \param{turtleset} and all the
links from \param{linkset} that connect two turtles from turtleset will be
included.
\end{doc}

\begin{prim}
nw:get-context
\end{prim}

Reports the content of the current graph context as a list containing two
agentsets: the agentset of turtles that are part of the context and the agentset
of links that are part of the context.

\begin{prim}
nw:with-context \param{turtleset} \param{linkset} \param{command-block}
\end{prim}

Executes the \param{command-block} with the context temporarily set to \param{turtleset} and \param{linkset}.
After \param{command-block} finishes running, the previous context will be restored.

\cat{Centrality Primitives}

\begin{prim}
nw:betweenness-centrality,
nw:eigenvector-centrality,
nw:page-rank
nw:closeness-centrality
nw:weighted-closeness-centrality
\end{prim}
\begin{doc}
These primitives calculate different centrality measures for a turtle. Example:
\begin{Verbatim}
ask turtles [ set size nw:betweenness-centrality ]
\end{Verbatim}
\end{doc}

\cat{Clustering Measures}

\begin{prim}
nw:clustering-coefficient
\end{prim}
\begin{doc}
This primitive measures how densely connected a turtle's neighbors are. It can be used to measure how much nodes tend to cluster together in a network. Example:
\begin{Verbatim}
ask turtles [ set label nw:clustering-coefficient ]
\end{Verbatim}
\end{doc}

\cat{Distance and Path-Finding Primitives}

\begin{prim}
nw:distance-to \param{target-turtle}\\
nw:weighted-distance-to \param{target-turtle weight-variable-name}
\end{prim}

\begin{doc}
Finds the shortest path to the target turtle and reports the total distance for
this path, or false if no path exists in the current context. The
\texttt{nw:distance-to} version of the primitive assumes that each link counts
for a distance of one. The \texttt{nw:weighted-distance-to} version accepts a
\param{weight-variable-name} parameter, which must be a string naming the link
variable to use as the weight of each link in distance calculations. The weights cannot
be negative numbers.\\Example:

\begin{Verbatim}
ask turtle 0 [ show nw:distance-to turtle 2 ]
ask turtle 0 [ show nw:weighted-distance-to turtle 2 "weight" ]
\end{Verbatim}
\end{doc}

\begin{prim}
nw:path-to \param{target-turtle}\\
nw:turtles-on-path-to \param{target-turtle}\\
nw:weighted-path-to \param{target-turtle weight-variable-name}\\
nw:turtles-on-weighted-path-to \param{target-turtle weight-variable-name}
\end{prim}

\begin{doc}
Finds the shortest path to the target turtle and reports the actual path between
the source and the target turtle. The \texttt{nw:path-to} and
\texttt{nw:weighted-path-to} variants will report the list of links that
constitute the path, while the \texttt{nw:turtles-on-path-to} and
\texttt{nw:turtles-on-weighted-path-to} variants will report the list of turtles
along the path, including the source and destination turtles.
As with the link distance primitives, the \texttt{nw:weighted-path-to} and
\texttt{nw:turtles-on-weighted-path-to} accept a \param{weight-variable-name}
parameter, which must be a string naming the link variable to use as the weight
of each link in distance calculations. The weights cannot be negative numbers.
If no path exist between the source and the target turtles, all primitives will
report an empty list. Examples:

\begin{Verbatim}
ask turtle 0 [ show nw:path-to turtle 2 ]
ask turtle 0 [ show nw:turtles-on-path-to turtle 2 ]
ask turtle 0 [ show nw:weighted-path-to turtle 2 "weight" ]
ask turtle 0 [ show nw:turtles-on-weighted-path-to turtle 2 "weight" ]
\end{Verbatim}
\end{doc}

\begin{prim}
nw:turtles-in-radius \param{radius}\\
nw:turtles-in-reverse-radius \param{radius}
\end{prim}

\begin{doc}
Returns the set of turtles within the given distance (number of links followed)
of the calling turtle in the current context. Both forms include the calling
turtle, whom you can exclude with \texttt{other} if need be.
The \texttt{turtles-in-radius} form will follow both undirected links and
directed out links. The \texttt{turtles-in-reverse-radius} form will follow both
undirected links and directed in links. Example:

\begin{Verbatim}
ask turtle 0 [ show sort nw:turtles-in-radius 1 ]
\end{Verbatim}
\end{doc}

\begin{prim}
nw:mean-path-length\\
nw:mean-weighted-path-length \param{weight-variable-name}
\end{prim}

\begin{doc}
Reports the average shortest-path length between all distinct pairs of nodes in
the current snapshot. If the \texttt{nw:mean-weighted-path-length} is used, the
distances will be calculated using \param{weight-variable-name}. The weights
cannot be negative numbers. Reports false unless paths exist between all pairs.
Examples:

\begin{Verbatim}
show nw:mean-path-length
show nw:mean-weighted-path-length "weight"
\end{Verbatim}
\end{doc}

\cat{Clusterers and Clique Finder Primitives}

\begin{prim}
nw:bicomponent-clusters
\end{prim}

\begin{doc}
Reports the list of bicomponent clusters in the current network context. The
result is reported as a list of agentsets of turtles. One turtle can be a member
of more than one bicomponent at once. Example:
\begin{Verbatim}
let clusters nw:bicomponent-clusters
\end{Verbatim}
\end{doc}

\begin{prim}
nw:weak-component-clusters
\end{prim}

\begin{doc}
Reports the list of "weakly" connected components in the current network
context. The result is reported as a list of agentsets of turtles. One turtle
cannot be a member of more than one weakly connected component at once.
Example:
\begin{Verbatim}
let clusters nw:weak-component-clusters
\end{Verbatim}
\end{doc}

\begin{prim}
nw:maximal-cliques
\end{prim}

\begin{doc}
A clique is a subset of a network in which every node has a direct link to every
other node. A maximal clique is a clique that is not, itself, contained in a
bigger clique. The result is reported as a list of agentsets of turtles. One
turtle can be a member of more than one maximal clique at once. The primitive
uses the Bron–Kerbosch algorithm and only works with undirected links.
Example:
\begin{Verbatim}
let cliques nw:maximal-cliques
\end{Verbatim}
\end{doc}

\pagebreak

\begin{prim}
nw:biggest-maximal-cliques
\end{prim}

\begin{doc}
The biggest maximal cliques are, as the name implies, the biggest cliques in the
current network. Often, more than one clique are tied for the title of biggest
clique, so the result if reported as a list of agentsets. Example:
\begin{Verbatim}
let biggest-clique one-of nw:biggest-maximal-cliques
\end{Verbatim}
\end{doc}

\cat{Generator Primitives}

\begin{prim}
nw:generate-preferential-attachment
\param{turtle-breed link-breed nb-nodes [ commands ]}\\
nw:generate-random
\param{turtle-breed link-breed nb-nodes connection-prb [ commands ]}\\
nw:generate-small-world
\param{turtle-breed link-breed rows cols exp toroidal? [ commands ]}\\
nw:generate-lattice-2d
\param{turtle-breed link-breed rows cols exp toroidal? [ commands ]}\\
nw:generate-ring
\param{turtle-breed link-breed nb-nodes [ commands ]}\\
nw:generate-star
\param{turtle-breed link-breed nb-nodes [ commands ]}\\
nw:generate-wheel
\param{turtle-breed link-breed nb-nodes [ commands ]}\\
nw:generate-wheel-inward
\param{turtle-breed link-breed nb-nodes [ commands ]}\\
nw:generate-wheel-outward
\param{turtle-breed link-breed nb-nodes [ commands ]}
\end{prim}

\begin{doc}
The generators are amongst the only primitives that do not operate on the
current network context. Instead, all of them take a \param{turtle-breed} and a
\param{link-breed} as inputs and generate a new network using the given breeds.
Examples:

\begin{Verbatim}
nw:generate-preferential-attachment turtles links 100 [ set color red ]
nw:generate-random turtles links 100 0.5 [ set color green ]
nw:generate-small-world turtles links 10 10 2.0 false [ set color blue ]
nw:generate-wheel turtles links 100 [ set color yellow ]
\end{Verbatim}
\end{doc}

\cat{Import/Export Primitives}

\begin{prim}
nw:save-matrix \param{file-name}
\end{prim}

\begin{doc}
Saves the current network to \param{file-name}, as a text file, in the
form of a simple connection matrix. At the moment, \texttt{nw:save-matrix} does not
support link weights. Every link is represented as a \texttt{1.00} in the connection
matrix. Example:

\begin{Verbatim}
nw:save-matrix "matrix.txt"
\end{Verbatim}
\end{doc}

\begin{prim}
nw:load-matrix \param{file-name turtle-breed link-breed [ commands ]}
\end{prim}

\begin{doc}
Generates a new network according to the connection matrix saved in
\param{file-name}, using \param{turtle-breed} and \param{link-breed} to create
the new turtles and links. Please be aware that the breeds that you use to load
the matrix may be different from those that you used when you saved it. Example:

\begin{Verbatim}
nw:load-matrix "matrix.txt" turtles links
\end{Verbatim}
\end{doc}

\begin{prim}
nw:save-graphml \param{file-name}
\end{prim}

\begin{doc}
Saves the current network, as defined by \texttt{nw:set-context} in the GraphML format,
including every attribute of the turtles and links. Example:
\begin{Verbatim}
nw:save-graphml "example.graphml"
\end{Verbatim}
\end{doc}

\begin{prim}
nw:load-graphml \param{file-name [ commands ]}
\end{prim}
\begin{doc}

Loads a GraphML file into NetLogo. Tries to assign the attribute values
defined in the GraphML file to NetLogo agent variables of the same names (this
is not case sensitive). The first one it tries to set is \texttt{breed} if it is
there, so the turtle or link will get the right breed and, hence, the right
breed variables. Undefined variables or breeds are ignored. An optional command
block can be executed for each newly created turtle.
\begin{Verbatim}
nw:load-graphml "example.graphml" [ set color red ]
\end{Verbatim}
\end{doc}

\end{document}
